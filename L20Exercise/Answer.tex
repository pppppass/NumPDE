
%! TeX encoding = UTF-8
%! TeX program = LuaLaTeX

\documentclass[english, nochinese]{pnote}
\usepackage[paper]{pdef}
\usepackage{tikz-cd}

\title{Answers to Exercises (Lecture 20)}
\author{Zhihan Li, 1600010653}
\date{December 25, 2018}

\begin{document}

\maketitle

\textbf{Problem 1. (Page 283 Exercise 1).} \textit{Proof.} The variational problem yields
\begin{equation}
R \rbr{u} = 0.
\end{equation}
This means $ R \rbr{u} = R \rbr{ u - u_h } $
and hence by the Cauchy--Schwarz inequality
\begin{equation}
\abs{ R \rbr{ u - u_h } \rbr{w} } \le \abs{ u - u_h }_1 \abs{w}_1 + \norm{ u - u_h }_0 \norm{w}_0 + \abs{\nvbr{\rbr{ u - u_h }}_1} \abs{\nvbr{w}_1}
\end{equation}
and we have proved in previous exercises in $V$ (we note that $ u - u_h \in V $ also)
\begin{equation}
\abs{\nvbr{v}_1} \le \norm{v}_0 \le \abs{v}_1
\end{equation}
and this yields
\begin{equation}
\abs{ R \rbr{ u - u_h } \rbr{w} } \le 2 \norm{ u - u_h }_1 \norm{ u - u_h }_1
\end{equation}
and
\begin{equation}
\norm{ R \rbr{ u - u_h } } \le 2 \norm{ u - u_h }_1.
\end{equation}
Again we have
\begin{equation}
\norm{ R \rbr{ u - u_h } \rbr{ u - u_h } } = \abs{ u - u_h }_1^2 + \norm{ u - u_h }_0^2 + \nvbr{\rbr{ u - u_h }_1}^2 \ge \norm{ u - u_h }_1^2,
\end{equation}
we deduce
\begin{equation}
\norm{ R \rbr{ u - u_h } } \ge \norm{ u - u_h }_1.
\end{equation}
\hfill$\Box$

\textbf{Problem 2. (Page 284 Exercise 2).} \textit{Proof.} It suffices to prove for $ v_h \in V_h $,
\begin{equation}
R \rbr{u_h} \rbr{v_h} = 0.
\end{equation}
This directly follows from the variational problem. By setting $ \Pi_h w = v_h $ we can prove the original inequality.

Consider $ R \rbr{u_h} \rbr{v} $ where $ v = w - \Pi_h w $. The terms $ g w \rbr{1} $, $ u_h \rbr{1} w \rbr{1} $ drop. The term
\begin{equation}
\int_0^1 u_h' w'
\end{equation}
also drops because on each interval $ \sbr{ x_i, x_{ i + 1 } } $ we have
\begin{equation}
\int_{x_i}^{x_{ i + 1 }} u_h' w' = \nvbr{u_h'}_{\sbr{ x_i, x_{ i + 1 }}} \nvbr{w}_{x_i}^{x_{ i + 1 }} = 0.
\end{equation}
As a result,
\begin{equation}
R \rbr{u_h} \rbr{w} = \int_0^1 \rbr{ f - u_h } w
\end{equation}
and
\begin{equation}
\abs{ R \rbr{u_h} \rbr{w} } \le \norm{ f - u_h }_0 \norm{w}_0
\end{equation}
as desired.
\hfill$\Box$

\textbf{Problem 3. (Page 284 Exercise 3).} \textit{Proof.} Consider the standard interpolation operator $\hat{\Pi}$ on $ \sbr{ 0, 1 } $. Define $ F_i : \sbr{ 0, 1 } \rightarrow I_i = \sbr{ x_{ i - 1 }, x_i }, t \mapsto x_{ i - 1 } + t \rbr{ x_i - x_{ i - 1 } } $, and we observe that
\begin{equation}
\begin{tikzcd}
H^1 \sbr{ x_{ i - 1 }, x_i } \arrow[r, "\nvbr{\Pi_h}_{I_i}"] \arrow[d, "F_i^{\text{t}}"'] & P_1 \sbr{ x_{ i - 1 }, x_i } \arrow[d, "F_i^{\text{t}}"] \\
H^1 \sbr{ 0, 1 } \arrow[r, "\hat{\Pi}"'] & P_1 \sbr{ 0, 1 }
\end{tikzcd}
\end{equation}
commutes. Since $ I - \hat{\Pi} $ is a $P_0$-invariant operator, we have for $ \hat{w} \in H \rbr{ 0, 1 } $
\begin{equation}
\norm{ \hat{w} - \hat{\Pi} \hat{w} }_0 \le C \abs{\hat{w}}_1.
\end{equation}
This implies
\begin{equation}
\norm{ w - \Pi_h w }_{ 0, I_i } \le C h_i \abs{w}_{ 1, I_i }
\end{equation}
by scaling.
\hfill$\Box$

\end{document}
