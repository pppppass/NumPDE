%! TeX encoding = UTF-8
%! TeX program = LuaLaTeX

\documentclass[english, nochinese]{pnote}
\usepackage[paper, safebm]{pdef}
\usepackage{esvect}

\title{Answers to Exercises (Week 01)}
\author{Zhihan Li, 1600010653}
\date{September 26, 2018}

\begin{document}

\maketitle

\textbf{Problem 1. (Page 29 Exercise 1)} \textit{Proof.} The quasi-uniformness means the existence of uniform constant $ C_1 > 0 $, such that
\begin{equation}
h_{\text{max}} \le C_1 h_{\text{min}}.
\end{equation}
Note that we consider the grid step $h$ to be $ h_{\text{max}} $ here. Denote the ``control rectangle'' to be
\begin{equation}
R_{\mathbf{j}} = \cbr{ \mathbf{x} = \rbr{ x_1, x_2, \cdots, x_n } : \frac{1}{2} \rbr{ l_{ \mathbf{j}, i } + x_{ \mathbf{j}, i } } \le x_i < \frac{1}{2} \rbr{ u_{\mathbf{j}, i } + x_{ \mathbf{j}, i } } },
\end{equation}
where $ l_{ \mathbf{j}, i } $ and $ u_{ \mathbf{j}, i } $ are the $i$-th component of adjacent nodes to $\mathbf{x}_{\mathbf{j}}$ on the $i$-th axis.
Therefore, the control volume $ \omega_{\mathbf{j}} = R_{\mathbf{j}} \cap \Omega $ with $ V_{\mathbf{j}} = \mathop{\text{meas}} \rbr{\omega_{\mathbf{j}}} $,
and we have directly
\begin{equation}
V_{\mathbf{j}} \le h^n
\end{equation}
since $ \frac{1}{2} \rbr{ u_{ \mathbf{j}, i } - l_{ \mathbf{j}, i } } \le h_{\text{max}} = h $.
Because $\Omega$ is itself a Lipchitz domain and $\mathbf{x}_{\mathbf{j}}$ is located inside both $\omega_{\mathbf{j}}$ and $\Omega$, there exists another uniform constant $C_2$ such that
\begin{equation}
V_{\mathbf{j}} \ge C_2 h_{\text{min}}^n \ge C_1^{-n} C_2 h^n.
\end{equation}
Combining these, we obtain
\begin{equation}
C_1^{ -n / p } C_2^{ 1 / p } h^{ n / p } \rbr{\sum_{ \mathbf{j} \in J }{\abs{U_{\mathbf{j}}^p}}}^{ 1 / p } \le \rbr{\sum_{ \mathbf{j} \in J }{ V_{\mathbf{j}} \abs{U_{\mathbf{j}}^p} }}^{ 1 / p } \le h^{ n / p } \rbr{\sum_{ \mathbf{j} \in J }{\abs{U_{\mathbf{j}}^p}}}^{ 1 / p }.
\end{equation}

\hfill$\Box$

\textit{Proof of the existence of $C_2$.} The existence of $C_2$ heavily relies on $\Omega$ being a Lipchitz domain. Because $\Omega$ is bounded, $ \pd \Omega $ is compact and we may extract a finite opening covering of $ \pd \Omega $ such that $\Omega$ is the epigraph of a Lipchitz function locally. We may then assume the Lipchitz constant is uniform as $L$. Note that $L$ is indepenedent of $h$. For $\mathbf{j}$ and $ B_{\mathbf{j}} := B \rbr{ \mathbf{x}_{\mathbf{j}}, \frac{1}{2} h_{\text{min}} } $, we have $ B_{\mathbf{j}} \subseteq R_{\mathbf{j}} $ and therefore $ B_{\mathbf{j}} \cap \Omega \subseteq \omega_{\mathbf{j}} $. Change to the coordinate system $ \rbr{ \mathbf{u}, v } \in \mathbb{R}^{ n - 1 } \times \mathbb{R} $ where $\Omega$ can be expressed as the epigraph of a $L$-Lipchitz function $g$ locally and fix $\mathbf{x}_{\mathbf{j}}$ to the origin. In this coordinate system, we have for $ \mathbf{u} \in \mathbb{R}^{ n - 1 } $,
\begin{equation}
g \rbr{\mathbf{u}} \le L \norm{u},
\end{equation}
which means the cone $\Gamma_{\mathbf{j}}$ defined by epigraph $ v > L \norm{\mathbf{u}} $ satisfies $ B_{\mathbf{j}} \cap \Gamma_{\mathbf{j}} \subseteq B_{\mathbf{j}} \cap \Omega \subseteq \omega_{\mathbf{j}} $. Note that the measure $ \mathop{\text{meas}} \rbr{ B_{\mathbf{j}} \cap \Gamma_{\mathbf{j}} } $ is exactly $ G \rbr{L} h_{\text{min}}^d $ according to the scaling invariance of the cone $\Gamma_{\mathbf{j}}$, and therefore we have
\begin{equation}
V_{\mathbf{j}} \ge G \rbr{L} h_{\text{min}}^d.
\end{equation}
and we may set $ C_2 = G \rbr{L} $.

\textbf{Problem 2. (Page 30 Exercise 3)} \textit{Answer.} Consider the scheme (1.3.9) first. Plugging the Taylor's theorem with Peano form of the remainder into $ T_{ i, j } = \rbr{ L_{ \Delta x, \Delta y } - L } u $, we obtain the local truncation error 
\begin{equation}
\begin{split}
T_{ i, j } &= - \Delta x^2 \rbr{ \frac{1}{24} a_{ x x x } u_x + \frac{1}{8} a_{ x x } u_{ x x } + \frac{1}{6} a_x u_{ x x x } + \frac{1}{12} a u_{ x x x x } } \\
&- \Delta y^2 \rbr{ \frac{1}{24} a_{ y y y } u_y + \frac{1}{8} a_{ y y } u_{ y y } + \frac{1}{6} a_y u_{ y y y } + \frac{1}{12} a u_{ y y y y } } \\
&+ \Delta x^2 \rbr{ \frac{1}{6} v^1_{ x x x } u + \frac{1}{2} v^1_{ x x } u_x + \frac{1}{2} v^1_x u_{ x x } + \frac{1}{6} v^1 u_{ x x x } } \\
&+ \Delta y^2 \rbr{ \frac{1}{6} v^2_{ y y y } u + \frac{1}{2} v^2_{ y y } u_y + \frac{1}{2} v^2_y u_{ y y } + \frac{1}{6} v^2 u_{ y y y } } \\
&+ o \rbr{ \Delta x^2 + \Delta y^2 }.
\end{split}
\end{equation}

For the scheme (1.3.11), the local truncation error is
\begin{equation}
\begin{split}
T_{ i, j } &= - \Delta x^2 \rbr{ \frac{1}{24} a_{ x x x } u_x + \frac{1}{8} a_{ x x } u_{ x x } + \frac{1}{6} a_x u_{ x x x } + \frac{1}{12} a u_{ x x x x } } \\
&- \Delta y^2 \rbr{ \frac{1}{24} a_{ y y y } u_y + \frac{1}{8} a_{ y y } u_{ y y } + \frac{1}{6} a_y u_{ y y y } + \frac{1}{12} a u_{ y y y y } } \\
&+ \Delta x^2 \rbr{ \frac{1}{24} v^1_{ x x x } u + \frac{1}{8} v^1_{ x x } u_x + \frac{1}{4} v^1_x u_{ x x } + \frac{1}{6} v^1 u_{ x x x } } \\
&+ \Delta y^2 \rbr{ \frac{1}{24} v^2_{ y y y } u + \frac{1}{8} v^2_{ y y } u_y + \frac{1}{4} v^2_y u_{ y y } + \frac{1}{6} v^2 u_{ y y y } } \\
&+ o \rbr{ \Delta x^2 + \Delta y^2 }.
\end{split}
\end{equation}

\end{document}
