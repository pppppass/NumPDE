%! TeX encoding = UTF-8
%! TeX program = LuaLaTeX

\documentclass[english, nochinese]{pnote}
\usepackage[paper, cgu]{pdef}

\title{Answers to Exercises (Week 04)}
\author{Zhihan Li, 1600010653}
\date{October 16, 2018}

\begin{document}

\maketitle

\textbf{Problem 1. (Page 83 Exercise 5)} \textit{Proof.} Since
\begin{gather}
\frac{\Delta_{ + t }}{ \Delta t } t = 1, \\
\frac{\delta_x^2}{\rbr{ \Delta x }^2} \rbr{ \frac{1}{2} x \rbr{ 1 - x } } = 1,
\end{gather}
we deduce
\begin{gather}
L_{\rbr{ h, \tau }} \rbr{ e - \varPhi }_j^m = L_{\rbr{ h, \tau }} U_j^m - L_{\rbr{ h, \tau }} u_j^m - L_{\rbr{ h, \tau }} \varPhi_j^m = - T_j^m + \norm{T}_{\infty} \ge 0, \\
L_{\rbr{ h, \tau }} \rbr{ e - \varPsi }_j^m = L_{\rbr{ h, \tau }} U_j^m - L_{\rbr{ h, \tau }} u_j^m - L_{\rbr{ h, \tau }} \varPsi_j^m = - T_j^m + \norm{T}_{\infty} \ge 0.
\end{gather}
For a fixed $m$, we define nodes lying on $ \sbr{ 0, 1 } \times \cbr{0} \cup \cbr{ 0, 1 } \times \rbr{ 0, t_{\text{max}} } $. Similarly $J_{\Omega}$ is defined to be nodes in $ \rbr{ 0, 1 } \times \rsbr{ 0, t_{\text{max}} } $. When $ \mu \le 1 / 2 $, the difference operator yields equation
\begin{equation}
U_j^{ m + 1 } = \rbr{ 1 - 2 \mu } U_j^m + \mu U_{ j - 1 }^m + \mu U_{ j + 1 }^m,
\end{equation}
which satisfies the condition of maximal principle, i.e., this operator is $J_D$ connected, diagonally dominant with all diagonal entries negative and all off-diagonal entries non-negative (since $ 1 - 2 \mu \ge 0 $). Hence, maximal principle applies and $ e - \varPhi $ can never attain positive strict maximum in the interior of $J_{\Omega}$. This means
\begin{equation}
\rbr{ e - \varPhi }_j^m \le \norm{\rbr{ e - \varPhi }_+}_{ \infty, J_D } \le \max_{ 0 \le i \le N } \abs{e_i^0} + \max_{ 0 < l < m } \max \cbr{ \abs{e_0^l}, \abs{e_N^l} }
\end{equation}
and therefore
\begin{equation}
e_j^m \le \max_{ 0 \le i \le N } \abs{e_i^0} + \max_{ 0 < l < m } \max \cbr{ \abs{e_0^l}, \abs{e_N^l} } + \varPhi_j^m.
\end{equation}
Substitute $e_j^m$ by $-e_j^m$ we deduce
\begin{equation}
\abs{e_j^m} \le \max_{ 0 \le i \le N } \abs{e_i^0} + \max_{ 0 < l < m } \max \cbr{ \abs{e_0^l}, \abs{e_N^l} } + \varPhi_j^m.
\end{equation}
Similar argument applies for $\varPsi$ and we reach
\begin{equation}
\begin{split}
\abs{e_j^m} &\le \max_{ 0 \le i \le N } \abs{e_i^0} + \max_{ 0 < l < m } \max \cbr{ \abs{e_0^l}, \abs{e_N^l} } + \min \cbr{ \varPhi_j^m, \varPsi_j^m } \\
&\le \max_{ 0 \le i \le N } \abs{e_i^0} + \max_{ 0 < l < m } \max \cbr{ \abs{e_0^l}, \abs{e_N^l} } + \min \cbr{ t_m, \frac{1}{2} x_j \rbr{ 1 - x_j } } \norm{T}_{ \infty, \Omega_{t_{\text{max}}} }
\end{split}
\end{equation}
as desired.
\hfill$\Box$

\textbf{Problem 2.} \textit{Proof.} For a fixed $m$ we define identical $J_D$ and $J_{\Omega}$ as in Problem 1. The equation is
\begin{equation}
U_j^{ m + 1 } = \frac{1}{ 1 + 2 \mu } \rbr{ U_j^m + \mu U_{ j - 1 }^{ m + 1 } + \mu U_{ j + 1 }^{ m + 1 } }.
\end{equation}
Since $ 1 + 2 \mu > 0 $, the operator is $J_D$ connected, diagonally dominant with all diagonal entries negative and all off-diagonal entries non-negative.

Denote $ e_j^{ m + 1 } - \norm{e^m}_{\infty} - \tau T^{ m + 1 } = \tilde{e}_j^{ m + 1 } $. Since
\begin{equation}
\begin{split}
&\ptrel{=} \rbr{ 1 + 2 \mu } \tilde{e}_j^{ m + 1 } - \mu \tilde{e}_{ j - 1 }^{ m + 1 } - \mu \tilde{e}_{ j + 1 }^{ m + 1 } \\
&= e_j^m - \norm{e^m}_{\infty} - \tau T_j^{ m + 1 } - \tau T^{ m + 1 } \le 0,
\end{split}
\end{equation}
we deduce
\begin{equation}
\tilde{e}_j^{ m + 1 } \le \max \cbr{ \rbr{ \tilde{e}_0^{ m + 1 } }_+, \rbr{ \tilde{e}_N^{ m + 1 } }_+ } \le \max \cbr{ \abs{e_0^{ m + 1 }} - \norm{e^m}_{\infty}, \abs{e_N^{ m + 1 }} - \norm{e^m}_{\infty}, 0 }
\end{equation}
and further 
\begin{equation}
e_j^{ m + 1 } \le \max \cbr{ \abs{e_0^{ m + 1 }}, \abs{e_N^{ m + 1 }}, \norm{e^m}_{\infty} } + \tau T^{ m + 1 }.
\end{equation}
Alternating the sign of $ e_j^{ m + 1 } $ we obtain
\begin{equation}
\norm{e^{ m + 1 }}_{\infty} \le \max \cbr{ \abs{e_0^{ m + 1 }}, \abs{e_N^{ m + 1 }}, \norm{e^m}_{\infty} } + \tau T^{ m + 1 }.
\end{equation}
By recursive expanding, it follows
\begin{equation}
\norm{e^{ m + 1 }}_{\infty} \le \max \cbr{ \norm{e^0}_{\infty}, \max_{ 1 \le l \le m + 1 } \max \cbr{ \abs{e_0^l}, \abs{e_N^l} } } + \tau \sum_{ l = 1 }^{ m + 1 } T^l
\end{equation}
as desired.
\hfill$\Box$

\textbf{Problem 3. (Page 84 Exercise 12)} \textit{Answer.} The conclusions are respectively:
\begin{partlist}
\item When $ \mu \le 1 $, the scheme is $L^{\infty}$ stable and
\begin{equation}
\norm{U^{ m + 1 }}_{\infty} \le \max \cbr{ \norm{U^0}_{\infty}, \max_{ 1 \le l \le m + 1 } \max \cbr{ \abs{U_0^l}, \abs{U_N^l} } }
\end{equation}
and
\begin{equation}
\norm{e}_{ \infty, \Omega_{t_{\text{max}}} } \le \norm{e^0}_{\infty} + \max_{ 0 < m \tau \le t_{\text{max}}} \cbr{ \abs{e_0^m}, \abs{e_N^m} } + t_{\text{max}} \norm{T}_{\infty};
\end{equation}
\item Suppose $u$ is sufficiently smooth, and then if $ \mu \le 1 $,
\begin{equation}
\norm{e}_{ \infty, \Omega_{t_{\text{max}}} } \le \frac{1}{12} \tau \mu M_{ x x x x } + \frac{1}{12} \tau^2 M_{ x x x x x x }
\end{equation}
and approximation error of the scheme is $ O \rbr{\tau} $;
\item If $ M_{ x x x x x x } $ is bounded on $ \rbr{ 0, 1 } \times \rbr{ 0, t_{\text{max}} } $, then for the refinement path $ \rbr{ h_i, \tau_i } $ with $ \mu_i = \tau_i / h_i^2 \le 1 / 2 $, solution $U^{\rbr{i}}$ converges to $u$ in order $ O \rbr{h_i^2} $;
\item When $ \mu < +\infty $, the scheme is $L^2$ stable;
\item If $ \int_0^{t_{\text{max}}} \norm{ u_{ x x x x x x } \rbr{ \cdot, t } }_2 \sd t $ is bounded on $ \rbr{ 0, 1 } \times \rbr{ 0, t_{\text{max}} } $, then for the refinement path $ \rbr{ h_i, \tau_i } $ with $ \mu_i = \tau_i / h_i^2 < +\infty $, solution $U^{\rbr{i}}$ converges to $u$ with order $ O \rbr{ h_i^2 + \tau_i^2 } $;
\item If $ u^0 = \sum_{ k = -\infty }^{\infty} a_k \se^{ \si k \spi x } $ satisfies $ \sum_{ k = -\infty }^{\infty} \abs{a_k} < \infty $, then for $ \mu < +\infty $ we have
\begin{equation}
\lim_{ \tau \rightarrow 0 } \norm{e}_{ \infty, \Omega_{t_{\text{max}}} } = 0.
\end{equation}
\end{partlist}

\end{document}
