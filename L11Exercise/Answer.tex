
%! TeX encoding = UTF-8
%! TeX program = LuaLaTeX

\documentclass[english, nochinese]{pnote}
\usepackage[paper]{pdef}

\title{Answers to Exercises (Lecture 11)}
\author{Zhihan Li, 1600010653}
\date{November 14, 2018}

\begin{document}

\maketitle

\textbf{Problem 1. (Page 178 Exercise 1)} \textit{Proof.} The amplification factor of the Fourier mode $ \se^{ \si j k h } $, $\lambda_k$ satisfies
\begin{equation}
\lambda_k^2 = 1 - 2 \si \nu \sin k h \lambda_k + 2 \mu \rbr{ 2 \lambda_k \cos k h - \lambda_k^2 - 1 }
\end{equation}
or
\begin{equation} \label{Eq:Quad}
- 4 \mu \cos k h + 2 \si \nu \sin k h = -\rbr{ 1 + 2 \mu } \lambda_k - \rbr{ -1 + 2 \mu } \frac{1}{\lambda_k}.
\end{equation}
If $ \lambda_k = r \cos \theta + \si r \sin \theta $ for $ r > 1 $, we have
\begin{equation}
\begin{split}
&\ptrel{=} - \rbr{ 1 + 2 \mu } \lambda_k - \rbr{ -1 + 2 \mu } \frac{1}{\lambda_k} \\
&= \rbr{ -\rbr{ 1 + 2 \mu } r - \rbr{ -1 + 2 \mu } \frac{1}{r} } \cos \theta \\
&+ \si \rbr{ -\rbr{ 1 + 2 \mu } r + \rbr{ -1 + 2 \mu } \frac{1}{r} } \sin \theta.
\end{split}
\end{equation}
Note that \eqref{Eq:Quad} requires
\begin{equation}
\rbr{\frac{ \rbr{ 1 + 2 \mu } r + \rbr{ -1 + 2 \mu } / r }{ 4 \mu }}^2 \cos^2 \theta + \rbr{\frac{ \rbr{ 1 + 2 \mu } r + \rbr{ -1 + 2 \mu } / r }{ 2 \nu }}^2 \sin^2 \theta = 1.
\end{equation}
However, this can never be satisfied since
\begin{gather}
\rbr{\frac{ \rbr{ 1 + 2 \mu } r + \rbr{ -1 + 2 \mu } / r }{ 4 \mu }}^2 = \rbr{ \frac{1}{2} \rbr{ r + \frac{1}{r} } + \frac{1}{ 2 \mu } \rbr{ r - \frac{1}{r} } }^2 > 1, \\
\rbr{\frac{ \rbr{ 1 + 2 \mu } r - \rbr{ -1 + 2 \mu } / r }{ 2 \nu }}^2 = \rbr{ \frac{1}{ 2 \nu } \rbr{ r + \frac{1}{r} } + \frac{\mu}{\nu}{ r - \frac{1}{r} } }^2 > 1
\end{gather}
as long as $ \nu^2 \le 1 $. This means $ \abs{\lambda_k} \le 1 $ and therefore the scheme is $L^2$ stable.

Local truncation error expanded at the node $ \rbr{ j, n } $ is
\begin{equation}
\begin{split}
T_j^n &= \frac{a}{3} h^2 u_{ x x x } - \frac{c}{6} h^2 u_{ x x x x } + 2 c \frac{\tau^2}{h^2} u_{ t t } + O \rbr{ \tau^2 + h^4 + \frac{\tau^4}{h^2} } \\
&= O \rbr{ \tau^2 + h^2 + \frac{\tau^2}{h^2} }.
\end{split}
\end{equation}
As a result, the consistency requires
\begin{equation}
\tau = o \rbr{h}
\end{equation}
or $ \nu \rightarrow 0 $.
\hfill$\Box$

\textbf{Problem 2. (Page 178 Exercise 2)} \textit{Proof.} Assume
\begin{equation}
\norm{ \rbr{ B_1^{-1} B_0 }^m U^0 } \le K_1 \norm{U^0}
\end{equation}
for $ m \tau \le t_{\text{max}} $ and
\begin{gather}
\norm{B_1^{-1}} \le K_2 \tau, \\
\abs{ \alpha \rbr{x} } \le M.
\end{gather}
As a result we have
\begin{equation}
\norm{ B_1^{-1} A U^0 } \le K_2 M \tau \norm{U^0}.
\end{equation}
Finally, for the solution of $ B_1 U^{ m + 1 } = B_0 U^m + A U^m $, 
\begin{equation}
\begin{split}
\norm{U^m} &= \norm{ \rbr{ B_1^{-1} B_0 + B_1^{-1} A }^m U^0 } \\
&\le \sum_{ i = 0 }^m \binom{m}{i} K_1^{ i + 1 } \rbr{ K_2 M \tau }^i \norm{U^0} \\
&= K_1 \rbr{ 1 + K_1 K_2 M \tau }^m \norm{U^0} \\
&\le K_1 \exp \rbr{ K_1 K_2 M t_{\text{max}} } \norm{U^0}
\end{split}
\end{equation}
whenever $ m \tau \le t_{\text{max}} $. This means the second scheme is stable with respect to $\norm{\cdot}$.

We first note that the stability of these two schemes are equivalent. This is because
\begin{equation}
B_1 U^{ m + 1 } = B_0 U^m = \rbr{ B_0 + A } U^m - A U^m.
\end{equation}

When $ \alpha \equiv 0 $ the condition of $L^2$ stability is $ \mu = c \tau / h^2 \le 1 / 2 $. According to the conclusion above, the scheme is stable for bounded $\alpha$ whenever $ \mu \le 1 / 2 $.
\hfill$\Box$

\textbf{Problem 3. (Page 178 Exercise 4)} \textit{Answer.} The amplification factor is
\begin{equation}
\lambda_k = 1 - 2 \sin^2 \frac{ k h }{2} \rbr{ \nu^2 + 2 \mu - 2 \mu \nu \sin^2 \frac{ k h }{2} } - \si \nu \sin k h \rbr{ 1 - 2 \mu \sin^2 \frac{ k h }{2} }.
\end{equation}

Putting in $ k h = \spi / 2 $, from $ \Im \lambda_k \le 1 + K \tau $ we know
\begin{equation}
\nu \rbr{ 1 - \mu } \ge -1 - K \tau,
\end{equation}
which is bounded below. If $ \varlimsup \nu = p > 0 $, we definitely have $ \varlimsup \mu = +\infty $ and this leads to contradiction. Consequently, $ \nu < 1 $ in the limiting sense. Plugging in $ k h = \spi $, from $ \Re \lambda_k \ge -1 - K \tau $ we obtain
\begin{equation}
K \tau + \rbr{ 2 \mu - 1 - \nu } \rbr{ \nu - 1 } \ge 0.
\end{equation}
Since we knows $ \varlimsup \nu = \lim \nu = 0 $, we have
\begin{equation}
2 \varlimsup \mu \le 1
\end{equation}
or $ \varlimsup \mu \le 1 / 2 $. In this case, $ \nu = \sqrt{ a^2 \mu \tau / c } = O \rbr{\sqrt{\tau}} $. As a result,
\begin{equation}
\abs{\lambda_k} = \abs{ 1 - 4 \mu \sin^2 \frac{ k h }{2} \rbr{ 1 - \nu \sin^2 \frac{ k h }{2} } }+ O \rbr{\nu^2} \le 1 + O \rbr{\tau},
\end{equation}
i.e., von Neumann stability is satisfied.

Since the modes are decaying due to the diffusion term, the practical stability requires $ \abs{\lambda_k} \le 1 $. Again plugging in $ k h = \spi$ yields
\begin{equation}
\rbr{ 2 \mu - 1 - \nu } \rbr{ \nu - 1 } \ge 0
\end{equation}
and therefore
\begin{equation}
2 \mu - 1 \le \nu \le 1
\end{equation}
in the limiting sense by noticing that $ \lim \nu = 0 $. When $ \mu = 1 / 2 $, denote $ t = \sin^2 k h / 2 $, then
\begin{equation}
\abs{\lambda_k}^2 = 1 - \rbr{ 4 t - 4 t^2 - 4 t^2 \nu + 4 t^3 \nu } - \rbr{ 4 t^3 \nu - 12 t^3 \nu^2 + 8 t^3 \nu^3 } - \rbr{ 4 t^2 \nu^2 - 4 t^2 \nu^4 }.
\end{equation}
Since
\begin{gather}
t - t^2 - t^2 \nu + t^3 \nu = t \rbr{ 1 - t } \rbr{ 1 - t \nu } \ge 0, \\
t^3 \nu - 3 t^3 \nu + 2 t^3 \nu^3 = t^3 \nu \rbr{ 1 - \nu } \rbr{ 1 - 2 \nu } \ge 0, \\
t^2 \nu^2 - t^2 \nu^4 = t^2 \nu^2 \rbr{ 1 - \nu^2 } \ge 0,
\end{gather}
where the last two inequalities holds in the limiting sense, we deduce
\begin{equation}
\abs{\lambda_k} \le 1
\end{equation}
for small enough $ \rbr{ h, \tau } $ and therefore the practical stability is attained.

\end{document}
