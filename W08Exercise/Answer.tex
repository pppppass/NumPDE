
%! TeX encoding = UTF-8
%! TeX program = LuaLaTeX

\documentclass[english, nochinese]{pnote}
\usepackage[paper]{pdef}

\title{Answers to Exercises (Week 08)}
\author{Zhihan Li, 1600010653}
\date{November 5, 2018}

\begin{document}

\maketitle

% \textbf{Problem 1. (Page 154 Exercise 9)} \textit{Answer.} When $u$ is Lipchitz continuous,
% \begin{gather}
% \overline{u}_j^m - u_j^m = O \rbr{h}, \\
% \frac{1}{2} \rbr{ \overline{u}_j^m + \overline{u}_{ j + 1 }^m } - \overline{u}_{ j + 1 / 2 }^m = O \rbr{h}, \\
% f \rbr{\overline{u}_j^m} - f \rbr{u_j^m} = O \rbr{h}, \\
% \frac{\tau}{2} f \rbr{u_j^m} - F \rbr{ x_j, \sbr{ t_m, t_{ m + 1 / 2 } } } = O \rbr{\tau}.
% \end{gather}
% The conservation law on $ \sbr{ x_j, x_{ j + 1 } } \times \sbr{ t_m, t_{ m + 1 / 2 } } $ reads
% \begin{equation}
% \overline{u}_{ j + 1 / 2 }^{ m + 1 / 2 } = \overline{u}_{ j + 1 / 2 }^m - \frac{1}{h} \rbr{ F \rbr{ x_{ j + 1 }, \sbr{ t_m, t_{ m + 1 / 2 } } } - F \rbr{ x_j, \sbr{ t_m, t_{ m + 1 / 2 } } } }.
% \end{equation}
% Compared with 
% \begin{equation}
% U_{ j + 1 / 2 }^{ m + 1 / 2 } = \frac{1}{2} \rbr{ \overline{u}_j^m + \overline{u}_{ j + 1 }^m } - \frac{\tau}{ 2 h } \rbr{ f \rbr{\overline{u_{ j + 1 }^m}} - f \rbr{\overline{u_j^m}} },
% \end{equation}
% we have
% \begin{equation}
% U_{ j + 1 / 2 }^{ m + 1 / 2 } = u_{ j + 1 / 2 }^{ m + 1 / 2 } + O \rbr{h} + O \rbr{h} + \frac{1}{h} \rbr{ \tau O \rbr{h} + O \rbr{\tau} } = u_{ j + 1 / 2 }^{ m + 1 / 2 } + O \rbr{ h + \frac{\tau}{h} }.
% \end{equation}
% Similarly,
% \begin{equation}
% U_j^{ m + 1 } = \overline{u}_j^m - \frac{\tau}{h} \rbr{ f \rbr{U_{ j + 1 / 2 }^{ m + 1 / 2 }}  - f \rbr{U_{ j - 1 / 2 }^{ m + 1 / 2 }} }
% \end{equation}
% yields
% \begin{equation}
% U_j^{ m + 1 } = \overline{u}_j^{ m + 1 } + \frac{1}{h} \rbr{ \tau O \rbr{ h + \frac{\tau}{h} } + O \rbr{\tau} } = \overline{u}_j^{ m + 1 } + O \rbr{ \frac{\tau}{h} + \frac{\tau^2}{h^2} }.
% \end{equation}

\textbf{Problem 1. (Page 154 Exercise 9)} \textit{Proof.} Here the numerical flux on $ x_{ j + 1 / 2 } $ and $ \sbr{ t_m, t_{ m + 1 } } $, say $ F_{ j + 1 / 2 }^{ m + 1 / 2 } $, is given by
\begin{equation}
F_{ j + 1 / 2 }^{ m + 1 / 2 } = \tau f \rbr{U_{ j + 1 / 2 }^{ m + 1 / 2 }}.
\end{equation}
According to (3.3.12), $ F_{ j + 1 / 2 }^{ m + 1 / 2 } $ is continuous as a function of $U_j^m$ and $ U_{ j + 1 }^m $. Moreover, when $f$ is Lipchitz, $F$ is then Lipchitz. When $U_j^m$ are identically equal to a given quantity $U$, we knows that
\begin{gather}
U_{ j + 1 / 2 }^{ m + 1 / 2 } = U, \\
F_{ j + 1 / 2 }^{ m + 1 / 2 } = \tau f \rbr{U},
\end{gather}
which coincides with the real flux. This verifies that the scheme is consistent as a scheme of conservation laws.

For the local truncation error, put in $ U_{ j + 1 }^m = \overline{u}_{ j + 1 }^m $. We have
\begin{equation}
U_{ j + 1 }^m  = u_{ j + 1 }^m + O \rbr{h^2}.
\end{equation}
This implies
\begin{equation}
f \rbr{U_{ j + 1 }^m} = \rbr{ f \rbr{u} }_{ j + 1 }^m + O \rbr{h^2}
\end{equation}
and
\begin{equation}
\frac{\tau}{2} f \rbr{U_{ j + 1 }^m} = f \rbr{ x_{ j + 1 }; \sbr{ t_m, t_{ m + 1 / 2 } } } + O \rbr{ \tau^2 + h^2 \tau }
\end{equation}
and
\begin{equation}
\begin{split}
\frac{\tau}{ 2 h } \rbr{ f \rbr{U_{ j + 1 }^m} - f \rbr{U_j^m} } &= \frac{1}{h} \rbr{ f \rbr{ x_{ j + 1 }; \sbr{ t_m, t_{ m + 1 / 2 } } } - f \rbr{ x_j; \sbr{ t_m, t_{ m + 1 / 2 } } } } \\
&+ O \rbr{ \tau^2 + h^2 \tau }.
\end{split}
\end{equation}
Here the order will not decrease because for any leading term $ A_j^m h^{\alpha} \tau^{\beta} $, we have
\begin{equation}
\frac{1}{h} \rbr{ A_{ j + 1 }^m h^{\alpha} \tau^{\beta} - A_j^m h^{\alpha} \tau^{\beta} } = \rbr{A_x}_{ j + 1 / 2 }^m h^{\alpha} \tau^{\beta} = O \rbr{ h^{\alpha} \tau^{\beta} }.
\end{equation}
Together with
\begin{equation}
\frac{1}{2} \rbr{ U_j^m + U_{ j + 1 }^m } = \overline{u}_{ j + 1 / 2 }^m + O \rbr{h^2}
\end{equation}
and the conversation law
\begin{equation}
\overline{u}_{ j + 1 / 2 }^{ m + 1 / 2 } = \overline{u}_{ j + 1 / 2 }^m - \frac{\tau}{ 2 h } \rbr{ f \rbr{U_{ j + 1 }^m} - f \rbr{U_j^m} }
\end{equation}
we have
\begin{equation}
U_{ j + 1 / 2 }^{ m + 1 / 2 } = u_{ j + 1 / 2 }^{ m + 1 / 2 } + O \rbr{ h^2 + \tau^2 }.
\end{equation}
Similarly, we have
\begin{equation}
U_j^m = \overline{u}_j^m + \tau O \rbr{ h^2 + \tau^2 }.
\end{equation}
As a result, the local truncation error is
\begin{equation}
\frac{1}{\tau} \tau \rbr{ O \rbr{ h^2 \tau + \tau^3 } } = O \rbr{ h^2 + \tau^2 }.
\end{equation}
(Note that we should multiply $ 1 / \tau $ here because we want to consider the approximation of differential operators.)

When $ f \rbr{u} = a u $, we have
\begin{equation}
U_{ j + 1 / 2 }^{ m + 1 / 2 } = \rbr{ \frac{1}{2} + \frac{ a \tau }{ 2 h } } U_j^m + \rbr{ \frac{1}{2} - \frac{ a \tau }{ 2 h } } U_{ j + 1 }^m
\end{equation}
and
\begin{equation}
\begin{split}
U_j^{ m + 1 } &= \rbr{ 1 - \frac{ a \tau }{h} \rbr{ \frac{1}{2} + \frac{ a \tau }{ 2 h } } + \frac{ a \tau }{h} \rbr{ \frac{1}{2} - \frac{ a \tau }{ 2 h } } } U_j^m \\
&+ -\frac{ a \tau }{h} \rbr{ \frac{1}{2} - \frac{ a \tau }{ 2 h } } U_{ j + 1 }^m \\
&+ \frac{ a \tau }{h} \rbr{ \frac{1}{2} + \frac{ a \tau }{ 2 h } } U_{ j - 1 }^m.
\end{split}
\end{equation}
Expansion yields
\begin{equation}
U_j^{ m + 1 } = U_j^m - \frac{ a \tau }{ 2 h } \rbr{ U_{ j + 1 }^m - U_{ j - 1 }^m } + \frac{ a^2 \tau^2 }{ 2 h^2 } \rbr{ U_{ j + 1 }^m - 2 U_j^m + U_{ j - 1 }^m }
\end{equation}
which coincides the Lax--Wendroff scheme.
\hfill$\Box$

\textbf{Problem 2. (Page 154 Exercise 11)} \textit{Proof.} The Mac Cormack can be cast as the conservative scheme for numerical flux
\begin{equation}
F_{ j + 1 / 2 }^{ m + 1 / 2 } = \frac{\tau}{2} \rbr{ f \rbr{U_j^{ m + \ast }} + f \rbr{U_{ j + 1 }^m} }.
\end{equation}
Identical argument as in Problem 1 applies and this scheme is consistent when $f$ is Lipchitz.

For the local truncation error, we put in $ U_j^m = \overline{u}_j^m $. Again using the technique in Problem 1,
\begin{equation}
f \rbr{U_j^m} = \rbr{ f \rbr{u} }_j^m + O \rbr{h^2}
\end{equation}
and
\begin{equation}
\frac{1}{h} \rbr{ f \rbr{U_{ j + 1 }^m} - f \rbr{U_j^m} } = \rbr{ f \rbr{u}_x }_{ j + 1 / 2 }^m + O \rbr{h^2}.
\end{equation}
This means
\begin{equation}
U_j^{ m + \ast } = u_j^m - \rbr{ f \rbr{u}_x }_{ j + 1 / 2 }^m \tau + O \rbr{ h^2 \tau }.
\end{equation}
Note that
\begin{equation}
U_j^m = u_j^m + O \rbr{h^2}
\end{equation}
As a result,
\begin{equation}
\begin{split}
&\ptrel{=} \frac{\tau}{2} \rbr{ f \rbr{U_j^{ m + \ast }} + f \rbr{U_{ j + 1 }^m} } \\
&= \frac{1}{2} \rbr{ \rbr{ f \rbr{u} }_j^m + \rbr{ f \rbr{u} }_{ j + 1 }^m } \tau - \frac{1}{2} \rbr{ f' \rbr{u} f \rbr{u}_x }_{ j + 1 / 2 }^m \tau^2 + O \rbr{h^2} \\
&= \rbr{ f \rbr{u} }_{ j + 1 / 2 }^m \tau + \frac{1}{2} \rbr{\rbr{ f \rbr{u} }_t} \tau^2 + O \rbr{h^2} \\
&= f \rbr{ x_{ j + 1 / 2 }, \sbr{ t_m, t_{ m + 1 } } } + O \rbr{ h^2 + \tau^2 }.
\end{split}
\end{equation}
This implies
\begin{equation}
U_j^m = \overline{u}_j^m + O \rbr{ h^2 \tau + \tau^3 }
\end{equation}
and local truncation error has the order $ O \rbr{ h^2 + \tau^2 } $.

When $ f \rbr{u} = a u $,
\begin{equation}
U_j^{ m + \ast } = U_j^m - \frac{ a \tau }{h} \rbr{ U_{ j + 1 }^m - U_j^m }
\end{equation}
and
\begin{equation}
\begin{split}
U_j^{ m + 1 } &= U_j^m - \frac{ a \tau }{ 2 h } \rbr{ U_{ j + 1 }^m - U_j^m } \\
&- \frac{ a \tau }{ 2 h } \rbr{ U_j^m - U_{ j - 1 }^m } + \frac{ a^2 \tau^2 }{ 2 h^2 } \rbr{ U_{ j + 1 }^m - 2 U_j^m + U_{ j - 1 }^m } \\
&= U_j^m - \frac{ a \tau }{ 2 h } \rbr{ U_{ j + 1 }^m - U_{ j - 1 }^m } \\
&+ \frac{ a^2 \tau^2 }{ 2 h^2 } \rbr{ U_{ j + 1 }^m - 2 U_j^m + U_{ j - 1 }^m },
\end{split}
\end{equation}
which is exactly the Lax--Wendroff scheme.
\hfill$\Box$

\end{document}
