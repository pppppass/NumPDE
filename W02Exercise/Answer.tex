%! TeX encoding = UTF-8
%! TeX program = LuaLaTeX

\documentclass[english, nochinese]{pnote}
\usepackage{siunitx}
\usepackage{bm}
\usepackage{mathrsfs}
\usepackage{esint}
\usepackage{esvect}
\usepackage[paper, cgu]{pdef}

\title{Answers to Exercises (Week 02)}
\author{Zhihan Li, 1600010653}
\date{October 10, 2018}

\begin{document}

\maketitle

\textbf{Problem 1. (Page 30 Exercise 7)} \textit{Proof.} Because we are to approximate $ g \rbr{P^{\ast}} = \pdl{\bm{\nu}_{P^{\ast}}} u \rbr{P^{\ast}} $, therefore the local truncation error is
\begin{equation}
\begin{split}
T_P &= \frac{ u_P - u_W }{h_x} \cos \alpha + \frac{ u_P - u_S }{h_y} \sin \alpha - \pdl{\bm{\nu}_{P^{\ast}}} u \rbr{P^{\ast}} \\
&= \nabla u \rbr{P} \cdot \bm{\nu}_{P^{\ast}} - \frac{1}{2} h_x u_{ x x } \rbr{P} - \frac{1}{2} h_y u_{ y y } \rbr{P} - \nabla u \rbr{P^{\ast}} \cdot \bm{\nu}_{P^{\ast}} + o \rbr{ h_x + h_y } \\
&= -\frac{1}{2} h_x u_{ x x } \rbr{P} - \frac{1}{2} h_y u_{ y y } \rbr{P} - \vv{ P P^{\ast} }^{\text{T}} \nabla^2 u \rbr{P} \bm{\nu}_{P^{\ast}} + o \rbr{ h_x + h_y } \\
&= O \rbr{ h_x + h_y } = O \rbr{h}
\end{split}
\end{equation}
\hfill$\Box$

\textbf{Problem 2. (Page 30 Exercise 10)} \textit{Proof.} Apply maximal principle to $ U_{\mathbf{j}} - V_{\mathbf{j}} $. Because
\begin{equation}
L_h \rbr{ U_{\mathbf{j}} - V_{\mathbf{j}} } \ge -\abs{ L_h U_{\mathbf{j}} } - L_h V_{\mathbf{j}} \ge 0
\end{equation}
for $ \mathbf{j} \in J_{\Omega} $ and
\begin{equation}
U_{\mathbf{j}} - V_{\mathbf{j}} \le \abs{U_{\mathbf{j}}} - V_{\mathbf{j}} \le 0
\end{equation}
for $ \mathbf{j} \in J_D $,
therefore for $ \mathbf{j} \in J_{\Omega} $
\begin{equation}
U_{\mathbf{j}} - V_{\mathbf{j}} \le 0.
\end{equation}
Then apply maximal principle to $ -U_{\mathbf{j}} - V_{\mathbf{j}} $. Because
\begin{equation}
L_h \rbr{ -U_{\mathbf{j}} - V_{\mathbf{j}} } \ge -\abs{ L_h U_{\mathbf{j}} } - L_h V_{\mathbf{j}} \ge 0
\end{equation}
for $ \mathbf{j} \in J_{\Omega} $ and
\begin{equation}
-U_{\mathbf{j}} - V_{\mathbf{j}} \le \abs{U_{\mathbf{j}}} - V_{\mathbf{j}} \le 0
\end{equation}
for $ \mathbf{j} \in J_D $,
therefore for $ \mathbf{j} \in J_{\Omega} $
\begin{equation}
-U_{\mathbf{j}} - V_{\mathbf{j}} \le 0.
\end{equation}
Combining these two arguments, we have
\begin{equation}
\abs{U_{\mathbf{j}}} \le V_{\mathbf{j}}
\end{equation}
for $ \mathbf{j} \in J_{\Omega} $.
\hfill$\Box$

\end{document}
