%! TeX encoding = UTF-8
%! TeX program = LuaLaTeX

\documentclass[english, nochinese]{pnote}
\usepackage[paper]{pdef}

\title{Answers to Exercises (Week 03)}
\author{Zhihan Li, 1600010653}
\date{October 4, 2018}

\begin{document}

\maketitle

\textbf{Problem 1. (Page 30 Exercise 11)} \textit{Proof.} Choose non-negative $\varPhi$ such that 
\begin{equation}
L_h \varPhi_{\mathbf{j}} \ge 1
\end{equation}
for all $ \mathbf{j} \in J_{\Omega} $. Denote
\begin{equation}
d_\mathbf{j} = U_{\mathbf{j}} - u_{\mathbf{j}} - h^2 \psi_{\mathbf{j}}.
\end{equation}
We have $ \abs{ L_h d_{\mathbf{j}} } \le C h^4 $ with a uniform constant $C$ for all $ \mathbf{j} \in J_{\Omega} $ and $ d_{\mathbf{j}} = 0 $ for $ \mathbf{j} \in J_D $. The comparison theorem yields
\begin{equation}
\max_{ \mathbf{j} \in J_{\Omega} } \abs{d_\mathbf{j}} \le \max_{ \mathbf{j} \in J_D } \abs{d_\mathbf{j}} + \rbr{ \max_{ \mathbf{j} \in J_D } \abs{\varPhi_{\mathbf{j}}} } \rbr{ \max_{ \mathbf{j} \in J_{\Omega} } \abs{ L_h d_{\mathbf{j}} } } \le C h^4 \rbr{ \max_{ \mathbf{j} \in J_D } \abs{\varPhi_{\mathbf{j}}} }
\end{equation}
and therefore $ d_{\mathbf{j}} = O \rbr{h^4} $ for $ \mathbf{j} \in J_{\Omega} $.
Combing estimation of $d_{\mathbf{j}}$ on both $J_{\Omega}$ and $J_D$, we have
\begin{equation}
U_{\mathbf{j}} = u_{\mathbf{j}} + h^2 \psi_{\mathbf{j}} + d_{\mathbf{j}} = u_{\mathbf{j}} + h^2 \psi_{\mathbf{j}} + O \rbr{h^4}
\end{equation}
for $ \mathbf{j} \in J_h $ as desired.
\hfill$\Box$

\textbf{Problem 2. (Page 30 Exercise 12)} \textit{Proof.} Since $ J_{ h / 2 } \subseteq J_h $, we have for $ \mathbf{j} \in J_h $,
\begin{gather}
\label{Eq:EstH}
U_{ h, \mathbf{j} } = u_{\mathbf{j}} + h^2 \psi_{\mathbf{j}} + O \rbr{h^4}, \\
\label{Eq:EstH2}
U_{ h / 2, \mathbf{j} } = u_{\mathbf{j}} + \rbr{ h / 2 }^2 \psi_{\mathbf{j}} + O \rbr{h^4}.
\end{gather}
Subtract \eqref{Eq:EstH} from four times \eqref{Eq:EstH2} to cancel out $\psi_{\mathbf{j}}$, we deduce
\begin{equation}
4 U_{ h / 2, \mathbf{j} } - U_{ h, \mathbf{j} } = 3 u_{\mathbf{j}} + O \rbr{h^4}
\end{equation}
and
\begin{equation}
U_{ h, \mathbf{j} } - u_{\mathbf{j}} = \frac{4}{3} \rbr{ U_{ h, \mathbf{j} } - U_{ h / 2, \mathbf{j} } } + O \rbr{h^4}
\end{equation}
as desired.
\hfill$\Box$

\textbf{Problem 3. (Page 83 Exercise 1)} \textit{Answer.} The heat equation in a one dimensional homogenous media with no heat source and sink can be described as
\begin{equation}
\begin{cases}
u_t = \kappa^{-1} a u_{ x x }, & x \in \rbr{ 0, L }, t > 0; \\
u \rbr{ x, 0 } = u^0 \rbr{x}, & x \in \sbr{ 0, L }; \\
u \rbr{ 0, t } = u \rbr{ L, t } = 0, & t \ge 0,
\end{cases}
\end{equation}
as shown in (2.1.4) (2.1.5) and (2.1.6). Choose any $T$, which is of dimension $\mathsf{\Theta}$, to be the characteristic absolute temperature. We can than perform the nondimensionalization
\begin{gather}
\tilde{u} = \frac{u}{T}, \\
\tilde{x} = \frac{x}{L}, \\
\tilde{t} = \frac{ a t }{ L^2 \kappa }.
\end{gather}
Since the dimension of $a$ is $ \mathsf{L} \mathsf{M} \mathsf{T}^{-3} \mathsf{\Theta}^{-1} $ and that of $\kappa$ is $ \mathsf{L}^{-1} \mathsf{M} \mathsf{T}^{-2} \mathsf{\Theta}^{-1} $, we deduce that $\tilde{u}$, $\tilde{x}$ and $\tilde{t}$ are of null dimension. Since
\begin{gather}
\tilde{u}_{\tilde{t}} = \frac{ L^2 \kappa }{ a T } u_t, \\
\tilde{u}_{ \tilde{x} \tilde{x} } = \frac{ L^2 }{T} u_{ x x },
\end{gather}
the corresponding equation is
\begin{equation}
\begin{cases}
\tilde{u}_{\tilde{t}} = \tilde{u}_{ \tilde{x} \tilde{x} }, & \tilde{x} \in \rbr{ 0, 1 }, \tilde{t} > 0; \\
\tilde{u} \rbr{ \tilde{x}, 0 } = \tilde{u}^0 \rbr{\tilde{x}}, & \tilde{x} \in \sbr{ 0, 1 }; \\
\tilde{u} \rbr{ 0, \tilde{t} } = \tilde{u} \rbr{ 1, \tilde{t} } = 0, & \tilde{t} \ge 0
\end{cases}
\end{equation}
as desired, with
\begin{equation}
\tilde{u}^0 \rbr{\tilde{x}} = u^0 \rbr{x}.
\end{equation}

\end{document}
