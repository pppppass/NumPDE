
%! TeX encoding = UTF-8
%! TeX program = LuaLaTeX

\documentclass[english, nochinese]{pnote}
\usepackage{siunitx}
\usepackage[paper, cgu]{pdef}
\usepackage{pgf}
\usepackage{esint}

\title{Report of Project 6}
\author{Zhihan Li, 1600010653}
\date{December 21, 2018}

\begin{document}

\maketitle
\textbf{Problem. (Page 233 Coding Exercise 2)} \textit{Answer.} We implement solvers with triangle elements and rectangle elements and test the solver with two equations. The details and numerical results are summarized in the following sections.

\section{Models}

We tackle the Poisson equation with mixed boundary conditions in this project. The equation reads
\begin{equation}
\begin{cases}
-\Delta u = f, & \rbr{ x, y } \in \Omega = \rbr{ 0, 1 } \times \rbr{ 0, 1 }; \\
u = g_1 & \rbr{ x, y } \in \pdl{\text{W}} \Omega = \cbr{0} \times \sbr{ 0, 1 }; \\
-u_y + \beta_1 u = g_2, & \rbr{ x, y } \in \pdl{\text{S}} \Omega = \sbr{ 0, 1 } \times \cbr{0}; \\
u_x + \beta_2 u = g_3, & \rbr{ x, y } \in \pdl{\text{E}} \Omega = \cbr{1} \times \sbr{ 0, 1 }; \\
u_y + \beta_3 u = g_4 & \rbr{ x, y } \in \pdl{\text{N}} \Omega = \sbr{ 0, 1 } \times \cbr{1}.
\end{cases}
\end{equation}
We reuse models from Project 1. The first equation is
\begin{gather}
f \rbr{ x, y } = 2 \spi^2 \sin \spi x \cos \spi y, \\
g_1 \rbr{y} = 0, \\
\beta_1 = 0, \\
g_2 \rbr{x} = 0, \\
\beta_2 = 1, \\
g_3 \rbr{y} = -\spi \cos \spi y, \\
\beta_3 = 2, \\
g_4 \rbr{x} = -2 \sin \spi x.
\end{gather}
The analytical solution is
\begin{equation}
u \rbr{ x, y } = \sin \spi x \cos \spi y.
\end{equation}
The second equation is
\begin{gather}
f \rbr{ x, y } = 0, \\
g_1 \rbr{y} = \frac{1}{2} \log \rbr{ 1 + y^2 } + \arctan y, \\
\beta_1 = 2, \\
g_2 \rbr{x} = 2 \log \rbr{ x + 1 } - \frac{1}{ 1 + x }, \\
\beta_2 = 0, \\
g_3 \rbr{x} = \frac{ 2 - y }{ 4 + y^2 }, \\
\beta_3 = 1, \\
g_4 \rbr{x} = \frac{1}{2} \rbr{ \rbr{ x + 1 }^2 + 1 } + \frac{\spi}{2} - \arctan \rbr{ x + 1 } + \frac{ x + 2 }{ \rbr{ x + 1 }^2 + 1 }.
\end{gather}

\section{Solvers}

\subsection{Variation problem}

From the problem, we can see that the boundary condition on $ \pdl{\text{W}} \Omega $ is coercive boundary condition, while others are free boundary conditions. Take $ V = \cbr{ v \in H^1 \rbr{\Omega}, \nvbr{v}_{ \pdl{\text{W}} \Omega } = 0 } $. For $ v \in V $, we have
\begin{equation}
\begin{split}
\int_{\Omega} f v &= -\int_{\Omega} \Delta u v \\
&= \int_{\Omega} \nabla u \cdot \nabla v - \int_{ \pd \Omega } \pdl{\mathbf{n}} u v \\
&= \int_{\Omega} \nabla u \cdot \nabla v - \int_{ \pdl{\text{S}} \Omega } \rbr{-u_y} v - \int_{ \pdl{\text{E}} \Omega } u_x v - \int_{ \pdl{\text{N}} \Omega } u_y v \\
&= \int_{\Omega} \nabla u \cdot \nabla v + \beta_1 \int_{ \pdl{\text{S}} \Omega } u v + \beta_2 \int_{ \pdl{\text{E}} \Omega } u v + \beta_3 \int_{ \pdl{\text{N}} \Omega } u v \\
&- \int_{ \pdl{\text{S}} \Omega } g_2 v - \int_{ \pdl{\text{E}} \Omega } g_3 v - \int_{ \pdl{\text{N}} \Omega } g_4 v.
\end{split}
\end{equation}
As a result, the variation problem is to find $ u \in U = \cbr{ u \in H^1 \rbr{\Omega}, \nvbr{u}_{ \pdl{\text{W}} \Omega } = g_1 } $, such that for any $ v \in V $,
\begin{equation}
\int_{\Omega} \nabla u \cdot \nabla v + \beta_1 \int_{ \pdl{\text{S}} \Omega } u v + \beta_2 \int_{ \pdl{\text{E}} \Omega } u v + \beta_3 \int_{ \pdl{\text{N}} \Omega } u v = \int_{\Omega} f v + \int_{ \pdl{\text{S}} \Omega } g_2 v + \int_{ \pdl{\text{E}} \Omega } g_3 v + \int_{ \pdl{\text{N}} \Omega } g_4 v.
\end{equation}

\subsection{Triangle element}

\subsubsection{Finite element space}

We first adopt a triangle Lagrange element space to solve these equation. To be exact, we denote
\begin{equation}
h = \frac{1}{N}
\end{equation}
and
\begin{gather}
x_i = i h, \\
y_j = j h,
\end{gather}
and we consider triangles
\begin{equation}
\rbr{ x_i, y_j }, \rbr{ x_{ i + 1 }, y_j }, \rbr{ x_{ i + 1 }, y_{ j + 1 } }
\end{equation}
and
\begin{equation}
\rbr{ x_{ i + 1 }, y_{ j + 1 } }, \rbr{ x_i, y_{ j + 1 } }, \rbr{ x_i, y_j }
\end{equation}
with $ 0 \le i, j \le N - 1 $ as elements in $\mathcal{T}$. The test space is discretized as
\begin{equation}
V_h = \cbr{ v_h \in C^0 \rbr{\Omega} : \nvbr{v_h}_K \in P_1 \rbr{K}, \forall K; v_h \rbr{P} = 0, \forall P \in \pdl{\text{W}} \Omega }
\end{equation}
and trial space
\begin{equation}
U_h = \cbr{ u_h \in C^0 \rbr{\Omega} : \nvbr{u_h}_K \in P_1 \rbr{K}, \forall K; u_h \rbr{P} = g_1 \rbr{P}, \forall P \in \pdl{\text{W}} \Omega }.
\end{equation}
Here $P_K$ is directly $ P_1 \rbr{K} $, and the degrees of freedom $\Sigma_K$ are evaluation functionals at three vertices of $K$. As a results, the degrees of freedom of the whole finite element space are evaluation functionals at nodes $ \rbr{ x_i, y_j } $ for $ 1 \le i \le N $, $ 0 \le j \le N $.

\subsubsection{Linear system}

We first construct the stiffness matrix $A$, which is actually summation of $A^K$ and $A^E$, namely the stiffness matrix on each element $K$, which serves for
\begin{equation}
\int_{\Omega} \nabla u \cdot \nabla v,
\end{equation}
and that on the boundary $E$, which corresponds to
\begin{equation}
\int_{ \pdl{\cdot} \Omega } u v.
\end{equation}
For $K$ with vertices $ P_1, P_2, P_3 $, direct integration of
\begin{equation}
A^K_{ P_i, P_j } = \int_K \nabla \lambda_{P_i}^K \cdot \nabla \lambda_{P_j}^K
\end{equation}
yields (arranged by $ P_1, P_2, P_3 $)
\begin{equation}
A^K = \msbr{ 1 / 2 & 0 & -1 / 2 \\ 0 & 1 / 2 & -1 / 2 \\ -1 / 2 & -1 / 2 & 1 }.
\end{equation}
On edges $E$ on $ \pd \Omega $ with vertices $ P_1, P_2 $, we have
\begin{equation}
A^E_{ P_i, P_j } = \int_E \lambda_{P_i}^E \lambda_{P_j}^E
\end{equation}
and (arranged by $ P_1, P_2 $)
\begin{equation}
A^E = h \msbr{ 1 / 3 & 1 / 6 \\ 1 / 6 & 1 / 3 }.
\end{equation}
Combining these all, we have
\begin{equation}
A = \sum_K A^K + \beta_1 \sum_{ E \subseteq \pdl{\text{S}} \Omega } A^E + \beta_2 \sum_{ E \subseteq \pdl{\text{E}} \Omega } A^E + \beta_3 \sum_{ E \subseteq \pdl{\text{N}} \Omega } A^E.
\end{equation}

Construction of the right hand side is more subtle. The first components of the right hand side is the terms
\begin{equation}
\int_{\Omega} f v.
\end{equation}
We deploy Simpson's quadrature on each element $K$, which consists of $ P_1, P_2, P_3 $, by defining
\begin{equation}
\rbr{\int_K}_h f = \frac{1}{3} \rbr{ f \rbr{\frac{ P_1 + P_2 }{2}} + f \rbr{\frac{ P_2 + P_3 }{2}} + f \rbr{\frac{ P_3 + P_1 }{2}} } m \rbr{K}.
\end{equation}
The corresponding right side vector is (arranged by $ P_1, P_2, P_3 $)
\begin{equation}
R^K = \msbr{ 1 / 6 & 0 & 1 / 6 \\ 1 / 6 & 1 / 6 & 0 \\ 0 & 1 / 6 & 1 / 6 } \msbr{ f \rbr{ \rbr{ P_1 + P_2 } / 2 } \\ f \rbr{ \rbr{ P_2 + P_3 } / 2 } \\ f \rbr{ \rbr{ P_3 + P_1 } / 2 } } h^2.
\end{equation}
One need to notice that the integral operator
\begin{equation}
I^K_h : f \mapsto \rbr{\int_K}_h f
\end{equation}
is a third-order polynomial-invariant operator. Moreover,
\begin{equation}
I^K_h f = I^K_h \Pi_{ P_3 \rbr{K} } f = I^K \Pi_{ P_3 \rbr{K} } f.
\end{equation}
where
\begin{equation}
I^K : f \mapsto \int_K f,
\end{equation}
we have
\begin{equation}
\abs{ I^K_h f - I^K f } = \abs{ I^K \rbr{ \Pi_{ P_3 \rbr{K} } - I } f } \le \norm{ \rbr{ \Pi_{ P_3 \rbr{K} } - I } f }_{ 0, 1 } \le C_1 h^4 \abs{f}_{ 4, 1 }
\end{equation}
for sufficiently small $f$. The second component of the right hand side is the terms
\begin{equation}
\int_{ \pd{\cdot} \Omega } g_{\cdot} v.
\end{equation}
For an edge $E$ on $ \pd \Omega $ with vertices $P_1$ and $P_2$, we have (arranged by $ P_1, P_2 $)
\begin{equation}
R^E = \msbr{ 1 / 6 & 1 / 3 & 0 \\ 0 & 1 / 3 & 1 / 6 } \msbr{ g_{\cdot} \rbr{P_1} \\ g_{\cdot} \rbr{ \rbr{ P_1 + P_2 } / 2 } \\ g_{\cdot} \rbr{P_2} } h.
\end{equation}
The quadrature is again of order $ O \rbr{h^4} $. The third component follows from $ u_h \rbr{P} = g_1 \rbr{P} $ for $ P \in \pdl{\text{W}} \Omega $ and $A$. Since the values at $ \pdl{\text{W}} \Omega $ is fixed, there is no need to solve for them. If we decompose the space to be $ \pdl{\text{W}} \Omega $ and $ \rbr{ \pdl{\text{W}} \Omega }^{\complement} $, we should add
\begin{equation}
-A_{ \rbr{ \pdl{\text{W}} \Omega }^{\complement} \pdl{\text{W}} \Omega } \rbr{u_h}_{ \pdl{\text{W}} \Omega } = -A_{ \rbr{ \pdl{\text{W}} \Omega }^{\complement} \pdl{\text{W}} \Omega } \rbr{g_1}_{ \pdl{\text{W}} \Omega }
\end{equation}
to the right hand side according to block notation. In conclusion, the linear system is
\begin{equation}
R = \sum_K R^K + \sum_{ E \subseteq \pdl{\text{S}} \Omega } R^E + \sum_{ E \subseteq \pdl{\text{E}} \Omega } R^E + \sum_{ E \subseteq \pdl{\text{N}} \Omega } R^E - A_{ \rbr{ \pdl{\text{W}} \Omega }^{\complement} \pdl{\text{W}} \Omega } \rbr{g_1}_{ \pdl{\text{W}} \Omega }.
\end{equation}

We solve the equation of
\begin{equation}
A u_h = R
\end{equation}
by conjugate gradient method since $A$ is already a symmetric positive definite matrix.

\subsection{Rectangle element}

\subsubsection{Finite element space}

We then deploy rectangle Lagrange element space to solve these equation. We consider rectangle
\begin{equation}
\rbr{ x_i, y_j }, \rbr{ x_{ i + 1 }, y_j }, \rbr{ x_{ i + 1 }, y_{ j + 1 } }, \rbr{ x_i, y_{ j + 1 } }
\end{equation}
with $ 0 \le i, j \le N - 1 $ as elements in $\mathcal{T}$. The test space is again discretized as
\begin{equation}
V_h = \cbr{ v_h \in C^0 \rbr{\Omega} : \nvbr{v_h}_K \in Q_1 \rbr{K}, \forall K; v_h \rbr{P} = 0, \forall P \in \pdl{\text{W}} \Omega }
\end{equation}
and trial space
\begin{equation}
U_h = \cbr{ u_h \in C^0 \rbr{\Omega} : \nvbr{u_h}_K \in Q_1 \rbr{K}, \forall K; u_h \rbr{P} = g_1 \rbr{P}, \forall P \in \pdl{\text{W}} \Omega }.
\end{equation}
The degrees of freedom of the whole finite element space are again evaluation functionals at nodes $ \rbr{ x_i, y_j } $ for $ 1 \le i \le N $, $ 0 \le j \le N $.

\subsubsection{Linear system}

We construct the stiffness matrix $A$ here, which is actually summation of $A^K$ and $A^E$, namely the stiffness matrix on each element $K$ and on the boundary $E$. For $K$ with vertices $ P_1, P_2, P_3, P_4 $ (counter-clock-wise), direct integration of
\begin{equation}
A^K_{ P_i, P_j } = \int_K \nabla \lambda_{P_i}^K \cdot \nabla \lambda_{P_j}^K
\end{equation}
yields (arranged by $ P_1, P_2, P_3, P_4 $)
\begin{equation}
A^K = \msbr{ 1 / 3 & -1 / 6 & -1 / 3 & -1 / 6 \\ -1 / 6 & 1 / 3 & -1 / 6 & -1 / 3 \\ 1 / 3 & -1 / 6 & 1 / 3 & -1 / 6 \\ -1 / 6 & -1 / 3 & -1 / 6 & 1 / 3 }.
\end{equation}
The stiffness matrix on the edges are identical to that of triangle elements. Combining these all, we have
\begin{equation}
A = \sum_K A^K + \beta_1 \sum_{ E \subseteq \pdl{\text{S}} \Omega } A^E + \beta_2 \sum_{ E \subseteq \pdl{\text{E}} \Omega } A^E + \beta_3 \sum_{ E \subseteq \pdl{\text{N}} \Omega } A^E.
\end{equation}

Construction of the right hand side is again subtle. We deploy Simpson's quadrature for $R^K$ on $K$, which consists of $ P_1, P_2, P_3, P_4 $ counter-clock-wise, by
\begin{equation}
R^K = \msbr{ 1 / 36 & 1 / 18 & 0 & 1 / 18 & 1 / 9 & 0 & 0 & 0 & 0 \\ 0 & 1 / 18 & 1 / 36 & 0 & 1 / 9 & 1 / 18 & 0 & 0 & 0 \\ 0 & 0 & 0 & 0 & 1 / 9 & 1 / 18 & 0 & 1 / 18 & 1 / 36 \\ 0 & 0 & 0 & 1 / 18 & 1 / 9 & 0 & 1 / 36 & 1 / 18 & 0 } \msbr{ f \rbr{P_1} \\ f \rbr{ \rbr{ P_1 + P_2 } / 2 } \\ f \rbr{P_2} \\ f \rbr{ \rbr{ P_1 + P_4 } / 2 } \\ f \rbr{ \rbr{ P_1 + P_2 + P_3 + P_4 } / 4 } \\ f \rbr{ P_2 + P_3 } / 2 \\ f \rbr{P_4} \\ f \rbr{ \rbr{ P_3 + P_4 } / 2 } \\ f \rbr{P_3} } h^2.
\end{equation}
The quadrature is again of order $ O \rbr{h^4} $. The other two components at the right hand side are identical to the triangle element case. In conclusion, the linear system is
\begin{equation}
R = \sum_K R^K + \sum_{ E \subseteq \pdl{\text{S}} \Omega } R^E + \sum_{ E \subseteq \pdl{\text{E}} \Omega } R^E + \sum_{ E \subseteq \pdl{\text{N}} \Omega } R^E - A_{ \rbr{ \pdl{\text{W}} \Omega }^{\complement} \pdl{\text{W}} \Omega } \rbr{g_1}_{ \pdl{\text{W}} \Omega }.
\end{equation}

\section{Numerical experiment}

\subsection{First equation}

We solve the equation and plot the solution, the errors at the node, and the errors of the whole function in Figure \ref{Fig:Tri1} and \ref{Fig:Rect1}. The $L^{\infty}$ tolerance of conjugate gradient method is 1.0e-11. It should be noticed that the ``checkerboard'' effect reflects the reality, since $u_h$ is piece-wise linear, while $u$ is smooth, and therefore the maximal points of error is expected to locate at interior of elements instead of at vertices. One may also observe that the error heat map of triangle elements is not symmetric: the triangulation we use is not symmetric after all.

\begin{figure}[htbp]
{
\centering
\scalebox{0.7}{%% Creator: Matplotlib, PGF backend
%%
%% To include the figure in your LaTeX document, write
%%   \input{<filename>.pgf}
%%
%% Make sure the required packages are loaded in your preamble
%%   \usepackage{pgf}
%%
%% Figures using additional raster images can only be included by \input if
%% they are in the same directory as the main LaTeX file. For loading figures
%% from other directories you can use the `import` package
%%   \usepackage{import}
%% and then include the figures with
%%   \import{<path to file>}{<filename>.pgf}
%%
%% Matplotlib used the following preamble
%%   \usepackage{fontspec}
%%   \setmainfont{DejaVuSerif.ttf}[Path=/home/lzh/anaconda3/envs/numpde/lib/python3.7/site-packages/matplotlib/mpl-data/fonts/ttf/]
%%   \setsansfont{DejaVuSans.ttf}[Path=/home/lzh/anaconda3/envs/numpde/lib/python3.7/site-packages/matplotlib/mpl-data/fonts/ttf/]
%%   \setmonofont{DejaVuSansMono.ttf}[Path=/home/lzh/anaconda3/envs/numpde/lib/python3.7/site-packages/matplotlib/mpl-data/fonts/ttf/]
%%
\begingroup%
\makeatletter%
\begin{pgfpicture}%
\pgfpathrectangle{\pgfpointorigin}{\pgfqpoint{8.000000in}{12.000000in}}%
\pgfusepath{use as bounding box, clip}%
\begin{pgfscope}%
\pgfsetbuttcap%
\pgfsetmiterjoin%
\definecolor{currentfill}{rgb}{1.000000,1.000000,1.000000}%
\pgfsetfillcolor{currentfill}%
\pgfsetlinewidth{0.000000pt}%
\definecolor{currentstroke}{rgb}{1.000000,1.000000,1.000000}%
\pgfsetstrokecolor{currentstroke}%
\pgfsetdash{}{0pt}%
\pgfpathmoveto{\pgfqpoint{0.000000in}{0.000000in}}%
\pgfpathlineto{\pgfqpoint{8.000000in}{0.000000in}}%
\pgfpathlineto{\pgfqpoint{8.000000in}{12.000000in}}%
\pgfpathlineto{\pgfqpoint{0.000000in}{12.000000in}}%
\pgfpathclose%
\pgfusepath{fill}%
\end{pgfscope}%
\begin{pgfscope}%
\pgfsetbuttcap%
\pgfsetmiterjoin%
\definecolor{currentfill}{rgb}{1.000000,1.000000,1.000000}%
\pgfsetfillcolor{currentfill}%
\pgfsetlinewidth{0.000000pt}%
\definecolor{currentstroke}{rgb}{0.000000,0.000000,0.000000}%
\pgfsetstrokecolor{currentstroke}%
\pgfsetstrokeopacity{0.000000}%
\pgfsetdash{}{0pt}%
\pgfpathmoveto{\pgfqpoint{4.705208in}{8.480556in}}%
\pgfpathlineto{\pgfqpoint{7.801389in}{8.480556in}}%
\pgfpathlineto{\pgfqpoint{7.801389in}{11.627778in}}%
\pgfpathlineto{\pgfqpoint{4.705208in}{11.627778in}}%
\pgfpathclose%
\pgfusepath{fill}%
\end{pgfscope}%
\begin{pgfscope}%
\pgfsetbuttcap%
\pgfsetroundjoin%
\definecolor{currentfill}{rgb}{0.000000,0.000000,0.000000}%
\pgfsetfillcolor{currentfill}%
\pgfsetlinewidth{0.803000pt}%
\definecolor{currentstroke}{rgb}{0.000000,0.000000,0.000000}%
\pgfsetstrokecolor{currentstroke}%
\pgfsetdash{}{0pt}%
\pgfsys@defobject{currentmarker}{\pgfqpoint{0.000000in}{-0.048611in}}{\pgfqpoint{0.000000in}{0.000000in}}{%
\pgfpathmoveto{\pgfqpoint{0.000000in}{0.000000in}}%
\pgfpathlineto{\pgfqpoint{0.000000in}{-0.048611in}}%
\pgfusepath{stroke,fill}%
}%
\begin{pgfscope}%
\pgfsys@transformshift{4.809858in}{8.480556in}%
\pgfsys@useobject{currentmarker}{}%
\end{pgfscope}%
\end{pgfscope}%
\begin{pgfscope}%
\definecolor{textcolor}{rgb}{0.000000,0.000000,0.000000}%
\pgfsetstrokecolor{textcolor}%
\pgfsetfillcolor{textcolor}%
\pgftext[x=4.809858in,y=8.383333in,,top]{\color{textcolor}\sffamily\fontsize{10.000000}{12.000000}\selectfont −2}%
\end{pgfscope}%
\begin{pgfscope}%
\pgfsetbuttcap%
\pgfsetroundjoin%
\definecolor{currentfill}{rgb}{0.000000,0.000000,0.000000}%
\pgfsetfillcolor{currentfill}%
\pgfsetlinewidth{0.803000pt}%
\definecolor{currentstroke}{rgb}{0.000000,0.000000,0.000000}%
\pgfsetstrokecolor{currentstroke}%
\pgfsetdash{}{0pt}%
\pgfsys@defobject{currentmarker}{\pgfqpoint{0.000000in}{-0.048611in}}{\pgfqpoint{0.000000in}{0.000000in}}{%
\pgfpathmoveto{\pgfqpoint{0.000000in}{0.000000in}}%
\pgfpathlineto{\pgfqpoint{0.000000in}{-0.048611in}}%
\pgfusepath{stroke,fill}%
}%
\begin{pgfscope}%
\pgfsys@transformshift{5.531578in}{8.480556in}%
\pgfsys@useobject{currentmarker}{}%
\end{pgfscope}%
\end{pgfscope}%
\begin{pgfscope}%
\definecolor{textcolor}{rgb}{0.000000,0.000000,0.000000}%
\pgfsetstrokecolor{textcolor}%
\pgfsetfillcolor{textcolor}%
\pgftext[x=5.531578in,y=8.383333in,,top]{\color{textcolor}\sffamily\fontsize{10.000000}{12.000000}\selectfont −1}%
\end{pgfscope}%
\begin{pgfscope}%
\pgfsetbuttcap%
\pgfsetroundjoin%
\definecolor{currentfill}{rgb}{0.000000,0.000000,0.000000}%
\pgfsetfillcolor{currentfill}%
\pgfsetlinewidth{0.803000pt}%
\definecolor{currentstroke}{rgb}{0.000000,0.000000,0.000000}%
\pgfsetstrokecolor{currentstroke}%
\pgfsetdash{}{0pt}%
\pgfsys@defobject{currentmarker}{\pgfqpoint{0.000000in}{-0.048611in}}{\pgfqpoint{0.000000in}{0.000000in}}{%
\pgfpathmoveto{\pgfqpoint{0.000000in}{0.000000in}}%
\pgfpathlineto{\pgfqpoint{0.000000in}{-0.048611in}}%
\pgfusepath{stroke,fill}%
}%
\begin{pgfscope}%
\pgfsys@transformshift{6.253299in}{8.480556in}%
\pgfsys@useobject{currentmarker}{}%
\end{pgfscope}%
\end{pgfscope}%
\begin{pgfscope}%
\definecolor{textcolor}{rgb}{0.000000,0.000000,0.000000}%
\pgfsetstrokecolor{textcolor}%
\pgfsetfillcolor{textcolor}%
\pgftext[x=6.253299in,y=8.383333in,,top]{\color{textcolor}\sffamily\fontsize{10.000000}{12.000000}\selectfont 0}%
\end{pgfscope}%
\begin{pgfscope}%
\pgfsetbuttcap%
\pgfsetroundjoin%
\definecolor{currentfill}{rgb}{0.000000,0.000000,0.000000}%
\pgfsetfillcolor{currentfill}%
\pgfsetlinewidth{0.803000pt}%
\definecolor{currentstroke}{rgb}{0.000000,0.000000,0.000000}%
\pgfsetstrokecolor{currentstroke}%
\pgfsetdash{}{0pt}%
\pgfsys@defobject{currentmarker}{\pgfqpoint{0.000000in}{-0.048611in}}{\pgfqpoint{0.000000in}{0.000000in}}{%
\pgfpathmoveto{\pgfqpoint{0.000000in}{0.000000in}}%
\pgfpathlineto{\pgfqpoint{0.000000in}{-0.048611in}}%
\pgfusepath{stroke,fill}%
}%
\begin{pgfscope}%
\pgfsys@transformshift{6.975019in}{8.480556in}%
\pgfsys@useobject{currentmarker}{}%
\end{pgfscope}%
\end{pgfscope}%
\begin{pgfscope}%
\definecolor{textcolor}{rgb}{0.000000,0.000000,0.000000}%
\pgfsetstrokecolor{textcolor}%
\pgfsetfillcolor{textcolor}%
\pgftext[x=6.975019in,y=8.383333in,,top]{\color{textcolor}\sffamily\fontsize{10.000000}{12.000000}\selectfont 1}%
\end{pgfscope}%
\begin{pgfscope}%
\pgfsetbuttcap%
\pgfsetroundjoin%
\definecolor{currentfill}{rgb}{0.000000,0.000000,0.000000}%
\pgfsetfillcolor{currentfill}%
\pgfsetlinewidth{0.803000pt}%
\definecolor{currentstroke}{rgb}{0.000000,0.000000,0.000000}%
\pgfsetstrokecolor{currentstroke}%
\pgfsetdash{}{0pt}%
\pgfsys@defobject{currentmarker}{\pgfqpoint{0.000000in}{-0.048611in}}{\pgfqpoint{0.000000in}{0.000000in}}{%
\pgfpathmoveto{\pgfqpoint{0.000000in}{0.000000in}}%
\pgfpathlineto{\pgfqpoint{0.000000in}{-0.048611in}}%
\pgfusepath{stroke,fill}%
}%
\begin{pgfscope}%
\pgfsys@transformshift{7.696739in}{8.480556in}%
\pgfsys@useobject{currentmarker}{}%
\end{pgfscope}%
\end{pgfscope}%
\begin{pgfscope}%
\definecolor{textcolor}{rgb}{0.000000,0.000000,0.000000}%
\pgfsetstrokecolor{textcolor}%
\pgfsetfillcolor{textcolor}%
\pgftext[x=7.696739in,y=8.383333in,,top]{\color{textcolor}\sffamily\fontsize{10.000000}{12.000000}\selectfont 2}%
\end{pgfscope}%
\begin{pgfscope}%
\definecolor{textcolor}{rgb}{0.000000,0.000000,0.000000}%
\pgfsetstrokecolor{textcolor}%
\pgfsetfillcolor{textcolor}%
\pgftext[x=6.253299in,y=8.193365in,,top]{\color{textcolor}\sffamily\fontsize{10.000000}{12.000000}\selectfont \(\displaystyle x\)}%
\end{pgfscope}%
\begin{pgfscope}%
\pgfsetbuttcap%
\pgfsetroundjoin%
\definecolor{currentfill}{rgb}{0.000000,0.000000,0.000000}%
\pgfsetfillcolor{currentfill}%
\pgfsetlinewidth{0.803000pt}%
\definecolor{currentstroke}{rgb}{0.000000,0.000000,0.000000}%
\pgfsetstrokecolor{currentstroke}%
\pgfsetdash{}{0pt}%
\pgfsys@defobject{currentmarker}{\pgfqpoint{-0.048611in}{0.000000in}}{\pgfqpoint{0.000000in}{0.000000in}}{%
\pgfpathmoveto{\pgfqpoint{0.000000in}{0.000000in}}%
\pgfpathlineto{\pgfqpoint{-0.048611in}{0.000000in}}%
\pgfusepath{stroke,fill}%
}%
\begin{pgfscope}%
\pgfsys@transformshift{4.705208in}{8.623611in}%
\pgfsys@useobject{currentmarker}{}%
\end{pgfscope}%
\end{pgfscope}%
\begin{pgfscope}%
\definecolor{textcolor}{rgb}{0.000000,0.000000,0.000000}%
\pgfsetstrokecolor{textcolor}%
\pgfsetfillcolor{textcolor}%
\pgftext[x=4.387107in,y=8.570850in,left,base]{\color{textcolor}\sffamily\fontsize{10.000000}{12.000000}\selectfont 0.0}%
\end{pgfscope}%
\begin{pgfscope}%
\pgfsetbuttcap%
\pgfsetroundjoin%
\definecolor{currentfill}{rgb}{0.000000,0.000000,0.000000}%
\pgfsetfillcolor{currentfill}%
\pgfsetlinewidth{0.803000pt}%
\definecolor{currentstroke}{rgb}{0.000000,0.000000,0.000000}%
\pgfsetstrokecolor{currentstroke}%
\pgfsetdash{}{0pt}%
\pgfsys@defobject{currentmarker}{\pgfqpoint{-0.048611in}{0.000000in}}{\pgfqpoint{0.000000in}{0.000000in}}{%
\pgfpathmoveto{\pgfqpoint{0.000000in}{0.000000in}}%
\pgfpathlineto{\pgfqpoint{-0.048611in}{0.000000in}}%
\pgfusepath{stroke,fill}%
}%
\begin{pgfscope}%
\pgfsys@transformshift{4.705208in}{9.006386in}%
\pgfsys@useobject{currentmarker}{}%
\end{pgfscope}%
\end{pgfscope}%
\begin{pgfscope}%
\definecolor{textcolor}{rgb}{0.000000,0.000000,0.000000}%
\pgfsetstrokecolor{textcolor}%
\pgfsetfillcolor{textcolor}%
\pgftext[x=4.387107in,y=8.953625in,left,base]{\color{textcolor}\sffamily\fontsize{10.000000}{12.000000}\selectfont 0.2}%
\end{pgfscope}%
\begin{pgfscope}%
\pgfsetbuttcap%
\pgfsetroundjoin%
\definecolor{currentfill}{rgb}{0.000000,0.000000,0.000000}%
\pgfsetfillcolor{currentfill}%
\pgfsetlinewidth{0.803000pt}%
\definecolor{currentstroke}{rgb}{0.000000,0.000000,0.000000}%
\pgfsetstrokecolor{currentstroke}%
\pgfsetdash{}{0pt}%
\pgfsys@defobject{currentmarker}{\pgfqpoint{-0.048611in}{0.000000in}}{\pgfqpoint{0.000000in}{0.000000in}}{%
\pgfpathmoveto{\pgfqpoint{0.000000in}{0.000000in}}%
\pgfpathlineto{\pgfqpoint{-0.048611in}{0.000000in}}%
\pgfusepath{stroke,fill}%
}%
\begin{pgfscope}%
\pgfsys@transformshift{4.705208in}{9.389162in}%
\pgfsys@useobject{currentmarker}{}%
\end{pgfscope}%
\end{pgfscope}%
\begin{pgfscope}%
\definecolor{textcolor}{rgb}{0.000000,0.000000,0.000000}%
\pgfsetstrokecolor{textcolor}%
\pgfsetfillcolor{textcolor}%
\pgftext[x=4.387107in,y=9.336400in,left,base]{\color{textcolor}\sffamily\fontsize{10.000000}{12.000000}\selectfont 0.4}%
\end{pgfscope}%
\begin{pgfscope}%
\pgfsetbuttcap%
\pgfsetroundjoin%
\definecolor{currentfill}{rgb}{0.000000,0.000000,0.000000}%
\pgfsetfillcolor{currentfill}%
\pgfsetlinewidth{0.803000pt}%
\definecolor{currentstroke}{rgb}{0.000000,0.000000,0.000000}%
\pgfsetstrokecolor{currentstroke}%
\pgfsetdash{}{0pt}%
\pgfsys@defobject{currentmarker}{\pgfqpoint{-0.048611in}{0.000000in}}{\pgfqpoint{0.000000in}{0.000000in}}{%
\pgfpathmoveto{\pgfqpoint{0.000000in}{0.000000in}}%
\pgfpathlineto{\pgfqpoint{-0.048611in}{0.000000in}}%
\pgfusepath{stroke,fill}%
}%
\begin{pgfscope}%
\pgfsys@transformshift{4.705208in}{9.771937in}%
\pgfsys@useobject{currentmarker}{}%
\end{pgfscope}%
\end{pgfscope}%
\begin{pgfscope}%
\definecolor{textcolor}{rgb}{0.000000,0.000000,0.000000}%
\pgfsetstrokecolor{textcolor}%
\pgfsetfillcolor{textcolor}%
\pgftext[x=4.387107in,y=9.719175in,left,base]{\color{textcolor}\sffamily\fontsize{10.000000}{12.000000}\selectfont 0.6}%
\end{pgfscope}%
\begin{pgfscope}%
\pgfsetbuttcap%
\pgfsetroundjoin%
\definecolor{currentfill}{rgb}{0.000000,0.000000,0.000000}%
\pgfsetfillcolor{currentfill}%
\pgfsetlinewidth{0.803000pt}%
\definecolor{currentstroke}{rgb}{0.000000,0.000000,0.000000}%
\pgfsetstrokecolor{currentstroke}%
\pgfsetdash{}{0pt}%
\pgfsys@defobject{currentmarker}{\pgfqpoint{-0.048611in}{0.000000in}}{\pgfqpoint{0.000000in}{0.000000in}}{%
\pgfpathmoveto{\pgfqpoint{0.000000in}{0.000000in}}%
\pgfpathlineto{\pgfqpoint{-0.048611in}{0.000000in}}%
\pgfusepath{stroke,fill}%
}%
\begin{pgfscope}%
\pgfsys@transformshift{4.705208in}{10.154712in}%
\pgfsys@useobject{currentmarker}{}%
\end{pgfscope}%
\end{pgfscope}%
\begin{pgfscope}%
\definecolor{textcolor}{rgb}{0.000000,0.000000,0.000000}%
\pgfsetstrokecolor{textcolor}%
\pgfsetfillcolor{textcolor}%
\pgftext[x=4.387107in,y=10.101951in,left,base]{\color{textcolor}\sffamily\fontsize{10.000000}{12.000000}\selectfont 0.8}%
\end{pgfscope}%
\begin{pgfscope}%
\pgfsetbuttcap%
\pgfsetroundjoin%
\definecolor{currentfill}{rgb}{0.000000,0.000000,0.000000}%
\pgfsetfillcolor{currentfill}%
\pgfsetlinewidth{0.803000pt}%
\definecolor{currentstroke}{rgb}{0.000000,0.000000,0.000000}%
\pgfsetstrokecolor{currentstroke}%
\pgfsetdash{}{0pt}%
\pgfsys@defobject{currentmarker}{\pgfqpoint{-0.048611in}{0.000000in}}{\pgfqpoint{0.000000in}{0.000000in}}{%
\pgfpathmoveto{\pgfqpoint{0.000000in}{0.000000in}}%
\pgfpathlineto{\pgfqpoint{-0.048611in}{0.000000in}}%
\pgfusepath{stroke,fill}%
}%
\begin{pgfscope}%
\pgfsys@transformshift{4.705208in}{10.537488in}%
\pgfsys@useobject{currentmarker}{}%
\end{pgfscope}%
\end{pgfscope}%
\begin{pgfscope}%
\definecolor{textcolor}{rgb}{0.000000,0.000000,0.000000}%
\pgfsetstrokecolor{textcolor}%
\pgfsetfillcolor{textcolor}%
\pgftext[x=4.387107in,y=10.484726in,left,base]{\color{textcolor}\sffamily\fontsize{10.000000}{12.000000}\selectfont 1.0}%
\end{pgfscope}%
\begin{pgfscope}%
\pgfsetbuttcap%
\pgfsetroundjoin%
\definecolor{currentfill}{rgb}{0.000000,0.000000,0.000000}%
\pgfsetfillcolor{currentfill}%
\pgfsetlinewidth{0.803000pt}%
\definecolor{currentstroke}{rgb}{0.000000,0.000000,0.000000}%
\pgfsetstrokecolor{currentstroke}%
\pgfsetdash{}{0pt}%
\pgfsys@defobject{currentmarker}{\pgfqpoint{-0.048611in}{0.000000in}}{\pgfqpoint{0.000000in}{0.000000in}}{%
\pgfpathmoveto{\pgfqpoint{0.000000in}{0.000000in}}%
\pgfpathlineto{\pgfqpoint{-0.048611in}{0.000000in}}%
\pgfusepath{stroke,fill}%
}%
\begin{pgfscope}%
\pgfsys@transformshift{4.705208in}{10.920263in}%
\pgfsys@useobject{currentmarker}{}%
\end{pgfscope}%
\end{pgfscope}%
\begin{pgfscope}%
\definecolor{textcolor}{rgb}{0.000000,0.000000,0.000000}%
\pgfsetstrokecolor{textcolor}%
\pgfsetfillcolor{textcolor}%
\pgftext[x=4.387107in,y=10.867501in,left,base]{\color{textcolor}\sffamily\fontsize{10.000000}{12.000000}\selectfont 1.2}%
\end{pgfscope}%
\begin{pgfscope}%
\pgfsetbuttcap%
\pgfsetroundjoin%
\definecolor{currentfill}{rgb}{0.000000,0.000000,0.000000}%
\pgfsetfillcolor{currentfill}%
\pgfsetlinewidth{0.803000pt}%
\definecolor{currentstroke}{rgb}{0.000000,0.000000,0.000000}%
\pgfsetstrokecolor{currentstroke}%
\pgfsetdash{}{0pt}%
\pgfsys@defobject{currentmarker}{\pgfqpoint{-0.048611in}{0.000000in}}{\pgfqpoint{0.000000in}{0.000000in}}{%
\pgfpathmoveto{\pgfqpoint{0.000000in}{0.000000in}}%
\pgfpathlineto{\pgfqpoint{-0.048611in}{0.000000in}}%
\pgfusepath{stroke,fill}%
}%
\begin{pgfscope}%
\pgfsys@transformshift{4.705208in}{11.303038in}%
\pgfsys@useobject{currentmarker}{}%
\end{pgfscope}%
\end{pgfscope}%
\begin{pgfscope}%
\definecolor{textcolor}{rgb}{0.000000,0.000000,0.000000}%
\pgfsetstrokecolor{textcolor}%
\pgfsetfillcolor{textcolor}%
\pgftext[x=4.387107in,y=11.250277in,left,base]{\color{textcolor}\sffamily\fontsize{10.000000}{12.000000}\selectfont 1.4}%
\end{pgfscope}%
\begin{pgfscope}%
\definecolor{textcolor}{rgb}{0.000000,0.000000,0.000000}%
\pgfsetstrokecolor{textcolor}%
\pgfsetfillcolor{textcolor}%
\pgftext[x=4.331551in,y=10.054167in,,bottom,rotate=90.000000]{\color{textcolor}\sffamily\fontsize{10.000000}{12.000000}\selectfont \(\displaystyle U\)}%
\end{pgfscope}%
\begin{pgfscope}%
\pgfpathrectangle{\pgfqpoint{4.705208in}{8.480556in}}{\pgfqpoint{3.096181in}{3.147222in}}%
\pgfusepath{clip}%
\pgfsetrectcap%
\pgfsetroundjoin%
\pgfsetlinewidth{1.505625pt}%
\definecolor{currentstroke}{rgb}{0.121569,0.466667,0.705882}%
\pgfsetstrokecolor{currentstroke}%
\pgfsetdash{}{0pt}%
\pgfpathmoveto{\pgfqpoint{4.845944in}{10.537488in}}%
\pgfpathlineto{\pgfqpoint{4.918116in}{10.537488in}}%
\pgfpathlineto{\pgfqpoint{4.990288in}{10.537488in}}%
\pgfpathlineto{\pgfqpoint{5.062460in}{10.537488in}}%
\pgfpathlineto{\pgfqpoint{5.134632in}{10.537488in}}%
\pgfpathlineto{\pgfqpoint{5.206804in}{10.537488in}}%
\pgfpathlineto{\pgfqpoint{5.278976in}{10.537488in}}%
\pgfpathlineto{\pgfqpoint{5.351148in}{10.537488in}}%
\pgfpathlineto{\pgfqpoint{5.423320in}{10.537488in}}%
\pgfpathlineto{\pgfqpoint{5.495492in}{10.537488in}}%
\pgfpathlineto{\pgfqpoint{5.567664in}{10.537488in}}%
\pgfpathlineto{\pgfqpoint{5.639836in}{10.537488in}}%
\pgfpathlineto{\pgfqpoint{5.712008in}{10.537485in}}%
\pgfpathlineto{\pgfqpoint{5.784180in}{10.537527in}}%
\pgfpathlineto{\pgfqpoint{5.856352in}{10.537055in}}%
\pgfpathlineto{\pgfqpoint{5.928524in}{10.541265in}}%
\pgfpathlineto{\pgfqpoint{6.000696in}{10.511827in}}%
\pgfpathlineto{\pgfqpoint{6.072869in}{10.666763in}}%
\pgfpathlineto{\pgfqpoint{6.145041in}{10.079928in}}%
\pgfpathlineto{\pgfqpoint{6.217213in}{11.484722in}}%
\pgfpathlineto{\pgfqpoint{6.289385in}{9.892843in}}%
\pgfpathlineto{\pgfqpoint{6.361557in}{8.671561in}}%
\pgfpathlineto{\pgfqpoint{6.433729in}{8.623634in}}%
\pgfpathlineto{\pgfqpoint{6.505901in}{8.623611in}}%
\pgfpathlineto{\pgfqpoint{6.578073in}{8.623611in}}%
\pgfpathlineto{\pgfqpoint{6.650245in}{8.623611in}}%
\pgfpathlineto{\pgfqpoint{6.722417in}{8.623611in}}%
\pgfpathlineto{\pgfqpoint{6.794589in}{8.623611in}}%
\pgfpathlineto{\pgfqpoint{6.866761in}{8.623611in}}%
\pgfpathlineto{\pgfqpoint{6.938933in}{8.623611in}}%
\pgfpathlineto{\pgfqpoint{7.011105in}{8.623611in}}%
\pgfpathlineto{\pgfqpoint{7.083277in}{8.623611in}}%
\pgfpathlineto{\pgfqpoint{7.155449in}{8.623611in}}%
\pgfpathlineto{\pgfqpoint{7.227621in}{8.623611in}}%
\pgfpathlineto{\pgfqpoint{7.299793in}{8.623611in}}%
\pgfpathlineto{\pgfqpoint{7.371965in}{8.623611in}}%
\pgfpathlineto{\pgfqpoint{7.444137in}{8.623611in}}%
\pgfpathlineto{\pgfqpoint{7.516309in}{8.623611in}}%
\pgfpathlineto{\pgfqpoint{7.588481in}{8.623611in}}%
\pgfpathlineto{\pgfqpoint{7.660653in}{8.623611in}}%
\pgfusepath{stroke}%
\end{pgfscope}%
\begin{pgfscope}%
\pgfpathrectangle{\pgfqpoint{4.705208in}{8.480556in}}{\pgfqpoint{3.096181in}{3.147222in}}%
\pgfusepath{clip}%
\pgfsetrectcap%
\pgfsetroundjoin%
\pgfsetlinewidth{1.505625pt}%
\definecolor{currentstroke}{rgb}{1.000000,0.498039,0.054902}%
\pgfsetstrokecolor{currentstroke}%
\pgfsetdash{}{0pt}%
\pgfpathmoveto{\pgfqpoint{4.845944in}{10.537488in}}%
\pgfpathlineto{\pgfqpoint{4.918116in}{10.537488in}}%
\pgfpathlineto{\pgfqpoint{4.990288in}{10.537488in}}%
\pgfpathlineto{\pgfqpoint{5.062460in}{10.537488in}}%
\pgfpathlineto{\pgfqpoint{5.134632in}{10.537488in}}%
\pgfpathlineto{\pgfqpoint{5.206804in}{10.537488in}}%
\pgfpathlineto{\pgfqpoint{5.278976in}{10.537488in}}%
\pgfpathlineto{\pgfqpoint{5.351148in}{10.537488in}}%
\pgfpathlineto{\pgfqpoint{5.423320in}{10.537488in}}%
\pgfpathlineto{\pgfqpoint{5.495492in}{10.537488in}}%
\pgfpathlineto{\pgfqpoint{5.567664in}{10.537488in}}%
\pgfpathlineto{\pgfqpoint{5.639836in}{10.537488in}}%
\pgfpathlineto{\pgfqpoint{5.712008in}{10.537486in}}%
\pgfpathlineto{\pgfqpoint{5.784180in}{10.537513in}}%
\pgfpathlineto{\pgfqpoint{5.856352in}{10.537206in}}%
\pgfpathlineto{\pgfqpoint{5.928524in}{10.539939in}}%
\pgfpathlineto{\pgfqpoint{6.000696in}{10.520844in}}%
\pgfpathlineto{\pgfqpoint{6.072869in}{10.620010in}}%
\pgfpathlineto{\pgfqpoint{6.145041in}{10.229883in}}%
\pgfpathlineto{\pgfqpoint{6.217213in}{11.106486in}}%
\pgfpathlineto{\pgfqpoint{6.289385in}{10.059026in}}%
\pgfpathlineto{\pgfqpoint{6.361557in}{8.772030in}}%
\pgfpathlineto{\pgfqpoint{6.433729in}{8.624187in}}%
\pgfpathlineto{\pgfqpoint{6.505901in}{8.623611in}}%
\pgfpathlineto{\pgfqpoint{6.578073in}{8.623611in}}%
\pgfpathlineto{\pgfqpoint{6.650245in}{8.623611in}}%
\pgfpathlineto{\pgfqpoint{6.722417in}{8.623611in}}%
\pgfpathlineto{\pgfqpoint{6.794589in}{8.623611in}}%
\pgfpathlineto{\pgfqpoint{6.866761in}{8.623611in}}%
\pgfpathlineto{\pgfqpoint{6.938933in}{8.623611in}}%
\pgfpathlineto{\pgfqpoint{7.011105in}{8.623611in}}%
\pgfpathlineto{\pgfqpoint{7.083277in}{8.623611in}}%
\pgfpathlineto{\pgfqpoint{7.155449in}{8.623611in}}%
\pgfpathlineto{\pgfqpoint{7.227621in}{8.623611in}}%
\pgfpathlineto{\pgfqpoint{7.299793in}{8.623611in}}%
\pgfpathlineto{\pgfqpoint{7.371965in}{8.623611in}}%
\pgfpathlineto{\pgfqpoint{7.444137in}{8.623611in}}%
\pgfpathlineto{\pgfqpoint{7.516309in}{8.623611in}}%
\pgfpathlineto{\pgfqpoint{7.588481in}{8.623611in}}%
\pgfpathlineto{\pgfqpoint{7.660653in}{8.623611in}}%
\pgfusepath{stroke}%
\end{pgfscope}%
\begin{pgfscope}%
\pgfpathrectangle{\pgfqpoint{4.705208in}{8.480556in}}{\pgfqpoint{3.096181in}{3.147222in}}%
\pgfusepath{clip}%
\pgfsetrectcap%
\pgfsetroundjoin%
\pgfsetlinewidth{1.505625pt}%
\definecolor{currentstroke}{rgb}{0.172549,0.627451,0.172549}%
\pgfsetstrokecolor{currentstroke}%
\pgfsetdash{}{0pt}%
\pgfpathmoveto{\pgfqpoint{4.845944in}{10.537488in}}%
\pgfpathlineto{\pgfqpoint{4.918116in}{10.537488in}}%
\pgfpathlineto{\pgfqpoint{4.990288in}{10.537488in}}%
\pgfpathlineto{\pgfqpoint{5.062460in}{10.537488in}}%
\pgfpathlineto{\pgfqpoint{5.134632in}{10.537488in}}%
\pgfpathlineto{\pgfqpoint{5.206804in}{10.537488in}}%
\pgfpathlineto{\pgfqpoint{5.278976in}{10.537488in}}%
\pgfpathlineto{\pgfqpoint{5.351148in}{10.537488in}}%
\pgfpathlineto{\pgfqpoint{5.423320in}{10.537488in}}%
\pgfpathlineto{\pgfqpoint{5.495492in}{10.537488in}}%
\pgfpathlineto{\pgfqpoint{5.567664in}{10.537488in}}%
\pgfpathlineto{\pgfqpoint{5.639836in}{10.537488in}}%
\pgfpathlineto{\pgfqpoint{5.712008in}{10.537488in}}%
\pgfpathlineto{\pgfqpoint{5.784180in}{10.537488in}}%
\pgfpathlineto{\pgfqpoint{5.856352in}{10.537488in}}%
\pgfpathlineto{\pgfqpoint{5.928524in}{10.537488in}}%
\pgfpathlineto{\pgfqpoint{6.000696in}{10.537488in}}%
\pgfpathlineto{\pgfqpoint{6.072869in}{10.537488in}}%
\pgfpathlineto{\pgfqpoint{6.145041in}{10.537488in}}%
\pgfpathlineto{\pgfqpoint{6.217213in}{10.537488in}}%
\pgfpathlineto{\pgfqpoint{6.289385in}{10.098932in}}%
\pgfpathlineto{\pgfqpoint{6.361557in}{9.049059in}}%
\pgfpathlineto{\pgfqpoint{6.433729in}{8.636715in}}%
\pgfpathlineto{\pgfqpoint{6.505901in}{8.623615in}}%
\pgfpathlineto{\pgfqpoint{6.578073in}{8.623611in}}%
\pgfpathlineto{\pgfqpoint{6.650245in}{8.623611in}}%
\pgfpathlineto{\pgfqpoint{6.722417in}{8.623611in}}%
\pgfpathlineto{\pgfqpoint{6.794589in}{8.623611in}}%
\pgfpathlineto{\pgfqpoint{6.866761in}{8.623611in}}%
\pgfpathlineto{\pgfqpoint{6.938933in}{8.623611in}}%
\pgfpathlineto{\pgfqpoint{7.011105in}{8.623611in}}%
\pgfpathlineto{\pgfqpoint{7.083277in}{8.623611in}}%
\pgfpathlineto{\pgfqpoint{7.155449in}{8.623611in}}%
\pgfpathlineto{\pgfqpoint{7.227621in}{8.623611in}}%
\pgfpathlineto{\pgfqpoint{7.299793in}{8.623611in}}%
\pgfpathlineto{\pgfqpoint{7.371965in}{8.623611in}}%
\pgfpathlineto{\pgfqpoint{7.444137in}{8.623611in}}%
\pgfpathlineto{\pgfqpoint{7.516309in}{8.623611in}}%
\pgfpathlineto{\pgfqpoint{7.588481in}{8.623611in}}%
\pgfpathlineto{\pgfqpoint{7.660653in}{8.623611in}}%
\pgfusepath{stroke}%
\end{pgfscope}%
\begin{pgfscope}%
\pgfpathrectangle{\pgfqpoint{4.705208in}{8.480556in}}{\pgfqpoint{3.096181in}{3.147222in}}%
\pgfusepath{clip}%
\pgfsetrectcap%
\pgfsetroundjoin%
\pgfsetlinewidth{1.505625pt}%
\definecolor{currentstroke}{rgb}{0.839216,0.152941,0.156863}%
\pgfsetstrokecolor{currentstroke}%
\pgfsetdash{}{0pt}%
\pgfpathmoveto{\pgfqpoint{4.845944in}{10.537488in}}%
\pgfpathlineto{\pgfqpoint{4.918116in}{10.537488in}}%
\pgfpathlineto{\pgfqpoint{4.990288in}{10.537488in}}%
\pgfpathlineto{\pgfqpoint{5.062460in}{10.537488in}}%
\pgfpathlineto{\pgfqpoint{5.134632in}{10.537488in}}%
\pgfpathlineto{\pgfqpoint{5.206804in}{10.537488in}}%
\pgfpathlineto{\pgfqpoint{5.278976in}{10.537488in}}%
\pgfpathlineto{\pgfqpoint{5.351148in}{10.537488in}}%
\pgfpathlineto{\pgfqpoint{5.423320in}{10.537488in}}%
\pgfpathlineto{\pgfqpoint{5.495492in}{10.537488in}}%
\pgfpathlineto{\pgfqpoint{5.567664in}{10.537488in}}%
\pgfpathlineto{\pgfqpoint{5.639836in}{10.537488in}}%
\pgfpathlineto{\pgfqpoint{5.712008in}{10.537488in}}%
\pgfpathlineto{\pgfqpoint{5.784180in}{10.537488in}}%
\pgfpathlineto{\pgfqpoint{5.856352in}{10.537488in}}%
\pgfpathlineto{\pgfqpoint{5.928524in}{10.537488in}}%
\pgfpathlineto{\pgfqpoint{6.000696in}{10.537488in}}%
\pgfpathlineto{\pgfqpoint{6.072869in}{10.537488in}}%
\pgfpathlineto{\pgfqpoint{6.145041in}{10.537488in}}%
\pgfpathlineto{\pgfqpoint{6.217213in}{10.537488in}}%
\pgfpathlineto{\pgfqpoint{6.289385in}{8.623611in}}%
\pgfpathlineto{\pgfqpoint{6.361557in}{8.623611in}}%
\pgfpathlineto{\pgfqpoint{6.433729in}{8.623611in}}%
\pgfpathlineto{\pgfqpoint{6.505901in}{8.623611in}}%
\pgfpathlineto{\pgfqpoint{6.578073in}{8.623611in}}%
\pgfpathlineto{\pgfqpoint{6.650245in}{8.623611in}}%
\pgfpathlineto{\pgfqpoint{6.722417in}{8.623611in}}%
\pgfpathlineto{\pgfqpoint{6.794589in}{8.623611in}}%
\pgfpathlineto{\pgfqpoint{6.866761in}{8.623611in}}%
\pgfpathlineto{\pgfqpoint{6.938933in}{8.623611in}}%
\pgfpathlineto{\pgfqpoint{7.011105in}{8.623611in}}%
\pgfpathlineto{\pgfqpoint{7.083277in}{8.623611in}}%
\pgfpathlineto{\pgfqpoint{7.155449in}{8.623611in}}%
\pgfpathlineto{\pgfqpoint{7.227621in}{8.623611in}}%
\pgfpathlineto{\pgfqpoint{7.299793in}{8.623611in}}%
\pgfpathlineto{\pgfqpoint{7.371965in}{8.623611in}}%
\pgfpathlineto{\pgfqpoint{7.444137in}{8.623611in}}%
\pgfpathlineto{\pgfqpoint{7.516309in}{8.623611in}}%
\pgfpathlineto{\pgfqpoint{7.588481in}{8.623611in}}%
\pgfpathlineto{\pgfqpoint{7.660653in}{8.623611in}}%
\pgfusepath{stroke}%
\end{pgfscope}%
\begin{pgfscope}%
\pgfpathrectangle{\pgfqpoint{4.705208in}{8.480556in}}{\pgfqpoint{3.096181in}{3.147222in}}%
\pgfusepath{clip}%
\pgfsetrectcap%
\pgfsetroundjoin%
\pgfsetlinewidth{0.501875pt}%
\definecolor{currentstroke}{rgb}{0.000000,0.000000,0.000000}%
\pgfsetstrokecolor{currentstroke}%
\pgfsetdash{}{0pt}%
\pgfpathmoveto{\pgfqpoint{4.845944in}{10.537488in}}%
\pgfpathlineto{\pgfqpoint{4.918116in}{10.537488in}}%
\pgfpathlineto{\pgfqpoint{4.990288in}{10.537488in}}%
\pgfpathlineto{\pgfqpoint{5.062460in}{10.537488in}}%
\pgfpathlineto{\pgfqpoint{5.134632in}{10.537488in}}%
\pgfpathlineto{\pgfqpoint{5.206804in}{10.537488in}}%
\pgfpathlineto{\pgfqpoint{5.278976in}{10.537488in}}%
\pgfpathlineto{\pgfqpoint{5.351148in}{10.537488in}}%
\pgfpathlineto{\pgfqpoint{5.423320in}{10.537488in}}%
\pgfpathlineto{\pgfqpoint{5.495492in}{10.537488in}}%
\pgfpathlineto{\pgfqpoint{5.567664in}{10.537488in}}%
\pgfpathlineto{\pgfqpoint{5.639836in}{10.537488in}}%
\pgfpathlineto{\pgfqpoint{5.712008in}{10.537488in}}%
\pgfpathlineto{\pgfqpoint{5.784180in}{10.537488in}}%
\pgfpathlineto{\pgfqpoint{5.856352in}{10.537488in}}%
\pgfpathlineto{\pgfqpoint{5.928524in}{10.537488in}}%
\pgfpathlineto{\pgfqpoint{6.000696in}{10.537488in}}%
\pgfpathlineto{\pgfqpoint{6.072869in}{10.537488in}}%
\pgfpathlineto{\pgfqpoint{6.145041in}{10.537488in}}%
\pgfpathlineto{\pgfqpoint{6.217213in}{10.537488in}}%
\pgfpathlineto{\pgfqpoint{6.289385in}{10.537488in}}%
\pgfpathlineto{\pgfqpoint{6.361557in}{8.623611in}}%
\pgfpathlineto{\pgfqpoint{6.433729in}{8.623611in}}%
\pgfpathlineto{\pgfqpoint{6.505901in}{8.623611in}}%
\pgfpathlineto{\pgfqpoint{6.578073in}{8.623611in}}%
\pgfpathlineto{\pgfqpoint{6.650245in}{8.623611in}}%
\pgfpathlineto{\pgfqpoint{6.722417in}{8.623611in}}%
\pgfpathlineto{\pgfqpoint{6.794589in}{8.623611in}}%
\pgfpathlineto{\pgfqpoint{6.866761in}{8.623611in}}%
\pgfpathlineto{\pgfqpoint{6.938933in}{8.623611in}}%
\pgfpathlineto{\pgfqpoint{7.011105in}{8.623611in}}%
\pgfpathlineto{\pgfqpoint{7.083277in}{8.623611in}}%
\pgfpathlineto{\pgfqpoint{7.155449in}{8.623611in}}%
\pgfpathlineto{\pgfqpoint{7.227621in}{8.623611in}}%
\pgfpathlineto{\pgfqpoint{7.299793in}{8.623611in}}%
\pgfpathlineto{\pgfqpoint{7.371965in}{8.623611in}}%
\pgfpathlineto{\pgfqpoint{7.444137in}{8.623611in}}%
\pgfpathlineto{\pgfqpoint{7.516309in}{8.623611in}}%
\pgfpathlineto{\pgfqpoint{7.588481in}{8.623611in}}%
\pgfpathlineto{\pgfqpoint{7.660653in}{8.623611in}}%
\pgfusepath{stroke}%
\end{pgfscope}%
\begin{pgfscope}%
\pgfsetrectcap%
\pgfsetmiterjoin%
\pgfsetlinewidth{0.803000pt}%
\definecolor{currentstroke}{rgb}{0.000000,0.000000,0.000000}%
\pgfsetstrokecolor{currentstroke}%
\pgfsetdash{}{0pt}%
\pgfpathmoveto{\pgfqpoint{4.705208in}{8.480556in}}%
\pgfpathlineto{\pgfqpoint{4.705208in}{11.627778in}}%
\pgfusepath{stroke}%
\end{pgfscope}%
\begin{pgfscope}%
\pgfsetrectcap%
\pgfsetmiterjoin%
\pgfsetlinewidth{0.803000pt}%
\definecolor{currentstroke}{rgb}{0.000000,0.000000,0.000000}%
\pgfsetstrokecolor{currentstroke}%
\pgfsetdash{}{0pt}%
\pgfpathmoveto{\pgfqpoint{7.801389in}{8.480556in}}%
\pgfpathlineto{\pgfqpoint{7.801389in}{11.627778in}}%
\pgfusepath{stroke}%
\end{pgfscope}%
\begin{pgfscope}%
\pgfsetrectcap%
\pgfsetmiterjoin%
\pgfsetlinewidth{0.803000pt}%
\definecolor{currentstroke}{rgb}{0.000000,0.000000,0.000000}%
\pgfsetstrokecolor{currentstroke}%
\pgfsetdash{}{0pt}%
\pgfpathmoveto{\pgfqpoint{4.705208in}{8.480556in}}%
\pgfpathlineto{\pgfqpoint{7.801389in}{8.480556in}}%
\pgfusepath{stroke}%
\end{pgfscope}%
\begin{pgfscope}%
\pgfsetrectcap%
\pgfsetmiterjoin%
\pgfsetlinewidth{0.803000pt}%
\definecolor{currentstroke}{rgb}{0.000000,0.000000,0.000000}%
\pgfsetstrokecolor{currentstroke}%
\pgfsetdash{}{0pt}%
\pgfpathmoveto{\pgfqpoint{4.705208in}{11.627778in}}%
\pgfpathlineto{\pgfqpoint{7.801389in}{11.627778in}}%
\pgfusepath{stroke}%
\end{pgfscope}%
\begin{pgfscope}%
\definecolor{textcolor}{rgb}{0.000000,0.000000,0.000000}%
\pgfsetstrokecolor{textcolor}%
\pgfsetfillcolor{textcolor}%
\pgftext[x=6.253299in,y=11.711111in,,base]{\color{textcolor}\sffamily\fontsize{12.000000}{14.400000}\selectfont \(\displaystyle  t = 0.2 \)}%
\end{pgfscope}%
\begin{pgfscope}%
\pgfsetbuttcap%
\pgfsetmiterjoin%
\definecolor{currentfill}{rgb}{1.000000,1.000000,1.000000}%
\pgfsetfillcolor{currentfill}%
\pgfsetlinewidth{0.000000pt}%
\definecolor{currentstroke}{rgb}{0.000000,0.000000,0.000000}%
\pgfsetstrokecolor{currentstroke}%
\pgfsetstrokeopacity{0.000000}%
\pgfsetdash{}{0pt}%
\pgfpathmoveto{\pgfqpoint{0.780208in}{4.530556in}}%
\pgfpathlineto{\pgfqpoint{3.876389in}{4.530556in}}%
\pgfpathlineto{\pgfqpoint{3.876389in}{7.677778in}}%
\pgfpathlineto{\pgfqpoint{0.780208in}{7.677778in}}%
\pgfpathclose%
\pgfusepath{fill}%
\end{pgfscope}%
\begin{pgfscope}%
\pgfsetbuttcap%
\pgfsetroundjoin%
\definecolor{currentfill}{rgb}{0.000000,0.000000,0.000000}%
\pgfsetfillcolor{currentfill}%
\pgfsetlinewidth{0.803000pt}%
\definecolor{currentstroke}{rgb}{0.000000,0.000000,0.000000}%
\pgfsetstrokecolor{currentstroke}%
\pgfsetdash{}{0pt}%
\pgfsys@defobject{currentmarker}{\pgfqpoint{0.000000in}{-0.048611in}}{\pgfqpoint{0.000000in}{0.000000in}}{%
\pgfpathmoveto{\pgfqpoint{0.000000in}{0.000000in}}%
\pgfpathlineto{\pgfqpoint{0.000000in}{-0.048611in}}%
\pgfusepath{stroke,fill}%
}%
\begin{pgfscope}%
\pgfsys@transformshift{0.884858in}{4.530556in}%
\pgfsys@useobject{currentmarker}{}%
\end{pgfscope}%
\end{pgfscope}%
\begin{pgfscope}%
\definecolor{textcolor}{rgb}{0.000000,0.000000,0.000000}%
\pgfsetstrokecolor{textcolor}%
\pgfsetfillcolor{textcolor}%
\pgftext[x=0.884858in,y=4.433333in,,top]{\color{textcolor}\sffamily\fontsize{10.000000}{12.000000}\selectfont −2}%
\end{pgfscope}%
\begin{pgfscope}%
\pgfsetbuttcap%
\pgfsetroundjoin%
\definecolor{currentfill}{rgb}{0.000000,0.000000,0.000000}%
\pgfsetfillcolor{currentfill}%
\pgfsetlinewidth{0.803000pt}%
\definecolor{currentstroke}{rgb}{0.000000,0.000000,0.000000}%
\pgfsetstrokecolor{currentstroke}%
\pgfsetdash{}{0pt}%
\pgfsys@defobject{currentmarker}{\pgfqpoint{0.000000in}{-0.048611in}}{\pgfqpoint{0.000000in}{0.000000in}}{%
\pgfpathmoveto{\pgfqpoint{0.000000in}{0.000000in}}%
\pgfpathlineto{\pgfqpoint{0.000000in}{-0.048611in}}%
\pgfusepath{stroke,fill}%
}%
\begin{pgfscope}%
\pgfsys@transformshift{1.606578in}{4.530556in}%
\pgfsys@useobject{currentmarker}{}%
\end{pgfscope}%
\end{pgfscope}%
\begin{pgfscope}%
\definecolor{textcolor}{rgb}{0.000000,0.000000,0.000000}%
\pgfsetstrokecolor{textcolor}%
\pgfsetfillcolor{textcolor}%
\pgftext[x=1.606578in,y=4.433333in,,top]{\color{textcolor}\sffamily\fontsize{10.000000}{12.000000}\selectfont −1}%
\end{pgfscope}%
\begin{pgfscope}%
\pgfsetbuttcap%
\pgfsetroundjoin%
\definecolor{currentfill}{rgb}{0.000000,0.000000,0.000000}%
\pgfsetfillcolor{currentfill}%
\pgfsetlinewidth{0.803000pt}%
\definecolor{currentstroke}{rgb}{0.000000,0.000000,0.000000}%
\pgfsetstrokecolor{currentstroke}%
\pgfsetdash{}{0pt}%
\pgfsys@defobject{currentmarker}{\pgfqpoint{0.000000in}{-0.048611in}}{\pgfqpoint{0.000000in}{0.000000in}}{%
\pgfpathmoveto{\pgfqpoint{0.000000in}{0.000000in}}%
\pgfpathlineto{\pgfqpoint{0.000000in}{-0.048611in}}%
\pgfusepath{stroke,fill}%
}%
\begin{pgfscope}%
\pgfsys@transformshift{2.328299in}{4.530556in}%
\pgfsys@useobject{currentmarker}{}%
\end{pgfscope}%
\end{pgfscope}%
\begin{pgfscope}%
\definecolor{textcolor}{rgb}{0.000000,0.000000,0.000000}%
\pgfsetstrokecolor{textcolor}%
\pgfsetfillcolor{textcolor}%
\pgftext[x=2.328299in,y=4.433333in,,top]{\color{textcolor}\sffamily\fontsize{10.000000}{12.000000}\selectfont 0}%
\end{pgfscope}%
\begin{pgfscope}%
\pgfsetbuttcap%
\pgfsetroundjoin%
\definecolor{currentfill}{rgb}{0.000000,0.000000,0.000000}%
\pgfsetfillcolor{currentfill}%
\pgfsetlinewidth{0.803000pt}%
\definecolor{currentstroke}{rgb}{0.000000,0.000000,0.000000}%
\pgfsetstrokecolor{currentstroke}%
\pgfsetdash{}{0pt}%
\pgfsys@defobject{currentmarker}{\pgfqpoint{0.000000in}{-0.048611in}}{\pgfqpoint{0.000000in}{0.000000in}}{%
\pgfpathmoveto{\pgfqpoint{0.000000in}{0.000000in}}%
\pgfpathlineto{\pgfqpoint{0.000000in}{-0.048611in}}%
\pgfusepath{stroke,fill}%
}%
\begin{pgfscope}%
\pgfsys@transformshift{3.050019in}{4.530556in}%
\pgfsys@useobject{currentmarker}{}%
\end{pgfscope}%
\end{pgfscope}%
\begin{pgfscope}%
\definecolor{textcolor}{rgb}{0.000000,0.000000,0.000000}%
\pgfsetstrokecolor{textcolor}%
\pgfsetfillcolor{textcolor}%
\pgftext[x=3.050019in,y=4.433333in,,top]{\color{textcolor}\sffamily\fontsize{10.000000}{12.000000}\selectfont 1}%
\end{pgfscope}%
\begin{pgfscope}%
\pgfsetbuttcap%
\pgfsetroundjoin%
\definecolor{currentfill}{rgb}{0.000000,0.000000,0.000000}%
\pgfsetfillcolor{currentfill}%
\pgfsetlinewidth{0.803000pt}%
\definecolor{currentstroke}{rgb}{0.000000,0.000000,0.000000}%
\pgfsetstrokecolor{currentstroke}%
\pgfsetdash{}{0pt}%
\pgfsys@defobject{currentmarker}{\pgfqpoint{0.000000in}{-0.048611in}}{\pgfqpoint{0.000000in}{0.000000in}}{%
\pgfpathmoveto{\pgfqpoint{0.000000in}{0.000000in}}%
\pgfpathlineto{\pgfqpoint{0.000000in}{-0.048611in}}%
\pgfusepath{stroke,fill}%
}%
\begin{pgfscope}%
\pgfsys@transformshift{3.771739in}{4.530556in}%
\pgfsys@useobject{currentmarker}{}%
\end{pgfscope}%
\end{pgfscope}%
\begin{pgfscope}%
\definecolor{textcolor}{rgb}{0.000000,0.000000,0.000000}%
\pgfsetstrokecolor{textcolor}%
\pgfsetfillcolor{textcolor}%
\pgftext[x=3.771739in,y=4.433333in,,top]{\color{textcolor}\sffamily\fontsize{10.000000}{12.000000}\selectfont 2}%
\end{pgfscope}%
\begin{pgfscope}%
\definecolor{textcolor}{rgb}{0.000000,0.000000,0.000000}%
\pgfsetstrokecolor{textcolor}%
\pgfsetfillcolor{textcolor}%
\pgftext[x=2.328299in,y=4.243365in,,top]{\color{textcolor}\sffamily\fontsize{10.000000}{12.000000}\selectfont \(\displaystyle x\)}%
\end{pgfscope}%
\begin{pgfscope}%
\pgfsetbuttcap%
\pgfsetroundjoin%
\definecolor{currentfill}{rgb}{0.000000,0.000000,0.000000}%
\pgfsetfillcolor{currentfill}%
\pgfsetlinewidth{0.803000pt}%
\definecolor{currentstroke}{rgb}{0.000000,0.000000,0.000000}%
\pgfsetstrokecolor{currentstroke}%
\pgfsetdash{}{0pt}%
\pgfsys@defobject{currentmarker}{\pgfqpoint{-0.048611in}{0.000000in}}{\pgfqpoint{0.000000in}{0.000000in}}{%
\pgfpathmoveto{\pgfqpoint{0.000000in}{0.000000in}}%
\pgfpathlineto{\pgfqpoint{-0.048611in}{0.000000in}}%
\pgfusepath{stroke,fill}%
}%
\begin{pgfscope}%
\pgfsys@transformshift{0.780208in}{4.673611in}%
\pgfsys@useobject{currentmarker}{}%
\end{pgfscope}%
\end{pgfscope}%
\begin{pgfscope}%
\definecolor{textcolor}{rgb}{0.000000,0.000000,0.000000}%
\pgfsetstrokecolor{textcolor}%
\pgfsetfillcolor{textcolor}%
\pgftext[x=0.462107in,y=4.620850in,left,base]{\color{textcolor}\sffamily\fontsize{10.000000}{12.000000}\selectfont 0.0}%
\end{pgfscope}%
\begin{pgfscope}%
\pgfsetbuttcap%
\pgfsetroundjoin%
\definecolor{currentfill}{rgb}{0.000000,0.000000,0.000000}%
\pgfsetfillcolor{currentfill}%
\pgfsetlinewidth{0.803000pt}%
\definecolor{currentstroke}{rgb}{0.000000,0.000000,0.000000}%
\pgfsetstrokecolor{currentstroke}%
\pgfsetdash{}{0pt}%
\pgfsys@defobject{currentmarker}{\pgfqpoint{-0.048611in}{0.000000in}}{\pgfqpoint{0.000000in}{0.000000in}}{%
\pgfpathmoveto{\pgfqpoint{0.000000in}{0.000000in}}%
\pgfpathlineto{\pgfqpoint{-0.048611in}{0.000000in}}%
\pgfusepath{stroke,fill}%
}%
\begin{pgfscope}%
\pgfsys@transformshift{0.780208in}{5.031939in}%
\pgfsys@useobject{currentmarker}{}%
\end{pgfscope}%
\end{pgfscope}%
\begin{pgfscope}%
\definecolor{textcolor}{rgb}{0.000000,0.000000,0.000000}%
\pgfsetstrokecolor{textcolor}%
\pgfsetfillcolor{textcolor}%
\pgftext[x=0.462107in,y=4.979178in,left,base]{\color{textcolor}\sffamily\fontsize{10.000000}{12.000000}\selectfont 0.2}%
\end{pgfscope}%
\begin{pgfscope}%
\pgfsetbuttcap%
\pgfsetroundjoin%
\definecolor{currentfill}{rgb}{0.000000,0.000000,0.000000}%
\pgfsetfillcolor{currentfill}%
\pgfsetlinewidth{0.803000pt}%
\definecolor{currentstroke}{rgb}{0.000000,0.000000,0.000000}%
\pgfsetstrokecolor{currentstroke}%
\pgfsetdash{}{0pt}%
\pgfsys@defobject{currentmarker}{\pgfqpoint{-0.048611in}{0.000000in}}{\pgfqpoint{0.000000in}{0.000000in}}{%
\pgfpathmoveto{\pgfqpoint{0.000000in}{0.000000in}}%
\pgfpathlineto{\pgfqpoint{-0.048611in}{0.000000in}}%
\pgfusepath{stroke,fill}%
}%
\begin{pgfscope}%
\pgfsys@transformshift{0.780208in}{5.390267in}%
\pgfsys@useobject{currentmarker}{}%
\end{pgfscope}%
\end{pgfscope}%
\begin{pgfscope}%
\definecolor{textcolor}{rgb}{0.000000,0.000000,0.000000}%
\pgfsetstrokecolor{textcolor}%
\pgfsetfillcolor{textcolor}%
\pgftext[x=0.462107in,y=5.337505in,left,base]{\color{textcolor}\sffamily\fontsize{10.000000}{12.000000}\selectfont 0.4}%
\end{pgfscope}%
\begin{pgfscope}%
\pgfsetbuttcap%
\pgfsetroundjoin%
\definecolor{currentfill}{rgb}{0.000000,0.000000,0.000000}%
\pgfsetfillcolor{currentfill}%
\pgfsetlinewidth{0.803000pt}%
\definecolor{currentstroke}{rgb}{0.000000,0.000000,0.000000}%
\pgfsetstrokecolor{currentstroke}%
\pgfsetdash{}{0pt}%
\pgfsys@defobject{currentmarker}{\pgfqpoint{-0.048611in}{0.000000in}}{\pgfqpoint{0.000000in}{0.000000in}}{%
\pgfpathmoveto{\pgfqpoint{0.000000in}{0.000000in}}%
\pgfpathlineto{\pgfqpoint{-0.048611in}{0.000000in}}%
\pgfusepath{stroke,fill}%
}%
\begin{pgfscope}%
\pgfsys@transformshift{0.780208in}{5.748595in}%
\pgfsys@useobject{currentmarker}{}%
\end{pgfscope}%
\end{pgfscope}%
\begin{pgfscope}%
\definecolor{textcolor}{rgb}{0.000000,0.000000,0.000000}%
\pgfsetstrokecolor{textcolor}%
\pgfsetfillcolor{textcolor}%
\pgftext[x=0.462107in,y=5.695833in,left,base]{\color{textcolor}\sffamily\fontsize{10.000000}{12.000000}\selectfont 0.6}%
\end{pgfscope}%
\begin{pgfscope}%
\pgfsetbuttcap%
\pgfsetroundjoin%
\definecolor{currentfill}{rgb}{0.000000,0.000000,0.000000}%
\pgfsetfillcolor{currentfill}%
\pgfsetlinewidth{0.803000pt}%
\definecolor{currentstroke}{rgb}{0.000000,0.000000,0.000000}%
\pgfsetstrokecolor{currentstroke}%
\pgfsetdash{}{0pt}%
\pgfsys@defobject{currentmarker}{\pgfqpoint{-0.048611in}{0.000000in}}{\pgfqpoint{0.000000in}{0.000000in}}{%
\pgfpathmoveto{\pgfqpoint{0.000000in}{0.000000in}}%
\pgfpathlineto{\pgfqpoint{-0.048611in}{0.000000in}}%
\pgfusepath{stroke,fill}%
}%
\begin{pgfscope}%
\pgfsys@transformshift{0.780208in}{6.106923in}%
\pgfsys@useobject{currentmarker}{}%
\end{pgfscope}%
\end{pgfscope}%
\begin{pgfscope}%
\definecolor{textcolor}{rgb}{0.000000,0.000000,0.000000}%
\pgfsetstrokecolor{textcolor}%
\pgfsetfillcolor{textcolor}%
\pgftext[x=0.462107in,y=6.054161in,left,base]{\color{textcolor}\sffamily\fontsize{10.000000}{12.000000}\selectfont 0.8}%
\end{pgfscope}%
\begin{pgfscope}%
\pgfsetbuttcap%
\pgfsetroundjoin%
\definecolor{currentfill}{rgb}{0.000000,0.000000,0.000000}%
\pgfsetfillcolor{currentfill}%
\pgfsetlinewidth{0.803000pt}%
\definecolor{currentstroke}{rgb}{0.000000,0.000000,0.000000}%
\pgfsetstrokecolor{currentstroke}%
\pgfsetdash{}{0pt}%
\pgfsys@defobject{currentmarker}{\pgfqpoint{-0.048611in}{0.000000in}}{\pgfqpoint{0.000000in}{0.000000in}}{%
\pgfpathmoveto{\pgfqpoint{0.000000in}{0.000000in}}%
\pgfpathlineto{\pgfqpoint{-0.048611in}{0.000000in}}%
\pgfusepath{stroke,fill}%
}%
\begin{pgfscope}%
\pgfsys@transformshift{0.780208in}{6.465251in}%
\pgfsys@useobject{currentmarker}{}%
\end{pgfscope}%
\end{pgfscope}%
\begin{pgfscope}%
\definecolor{textcolor}{rgb}{0.000000,0.000000,0.000000}%
\pgfsetstrokecolor{textcolor}%
\pgfsetfillcolor{textcolor}%
\pgftext[x=0.462107in,y=6.412489in,left,base]{\color{textcolor}\sffamily\fontsize{10.000000}{12.000000}\selectfont 1.0}%
\end{pgfscope}%
\begin{pgfscope}%
\pgfsetbuttcap%
\pgfsetroundjoin%
\definecolor{currentfill}{rgb}{0.000000,0.000000,0.000000}%
\pgfsetfillcolor{currentfill}%
\pgfsetlinewidth{0.803000pt}%
\definecolor{currentstroke}{rgb}{0.000000,0.000000,0.000000}%
\pgfsetstrokecolor{currentstroke}%
\pgfsetdash{}{0pt}%
\pgfsys@defobject{currentmarker}{\pgfqpoint{-0.048611in}{0.000000in}}{\pgfqpoint{0.000000in}{0.000000in}}{%
\pgfpathmoveto{\pgfqpoint{0.000000in}{0.000000in}}%
\pgfpathlineto{\pgfqpoint{-0.048611in}{0.000000in}}%
\pgfusepath{stroke,fill}%
}%
\begin{pgfscope}%
\pgfsys@transformshift{0.780208in}{6.823579in}%
\pgfsys@useobject{currentmarker}{}%
\end{pgfscope}%
\end{pgfscope}%
\begin{pgfscope}%
\definecolor{textcolor}{rgb}{0.000000,0.000000,0.000000}%
\pgfsetstrokecolor{textcolor}%
\pgfsetfillcolor{textcolor}%
\pgftext[x=0.462107in,y=6.770817in,left,base]{\color{textcolor}\sffamily\fontsize{10.000000}{12.000000}\selectfont 1.2}%
\end{pgfscope}%
\begin{pgfscope}%
\pgfsetbuttcap%
\pgfsetroundjoin%
\definecolor{currentfill}{rgb}{0.000000,0.000000,0.000000}%
\pgfsetfillcolor{currentfill}%
\pgfsetlinewidth{0.803000pt}%
\definecolor{currentstroke}{rgb}{0.000000,0.000000,0.000000}%
\pgfsetstrokecolor{currentstroke}%
\pgfsetdash{}{0pt}%
\pgfsys@defobject{currentmarker}{\pgfqpoint{-0.048611in}{0.000000in}}{\pgfqpoint{0.000000in}{0.000000in}}{%
\pgfpathmoveto{\pgfqpoint{0.000000in}{0.000000in}}%
\pgfpathlineto{\pgfqpoint{-0.048611in}{0.000000in}}%
\pgfusepath{stroke,fill}%
}%
\begin{pgfscope}%
\pgfsys@transformshift{0.780208in}{7.181907in}%
\pgfsys@useobject{currentmarker}{}%
\end{pgfscope}%
\end{pgfscope}%
\begin{pgfscope}%
\definecolor{textcolor}{rgb}{0.000000,0.000000,0.000000}%
\pgfsetstrokecolor{textcolor}%
\pgfsetfillcolor{textcolor}%
\pgftext[x=0.462107in,y=7.129145in,left,base]{\color{textcolor}\sffamily\fontsize{10.000000}{12.000000}\selectfont 1.4}%
\end{pgfscope}%
\begin{pgfscope}%
\pgfsetbuttcap%
\pgfsetroundjoin%
\definecolor{currentfill}{rgb}{0.000000,0.000000,0.000000}%
\pgfsetfillcolor{currentfill}%
\pgfsetlinewidth{0.803000pt}%
\definecolor{currentstroke}{rgb}{0.000000,0.000000,0.000000}%
\pgfsetstrokecolor{currentstroke}%
\pgfsetdash{}{0pt}%
\pgfsys@defobject{currentmarker}{\pgfqpoint{-0.048611in}{0.000000in}}{\pgfqpoint{0.000000in}{0.000000in}}{%
\pgfpathmoveto{\pgfqpoint{0.000000in}{0.000000in}}%
\pgfpathlineto{\pgfqpoint{-0.048611in}{0.000000in}}%
\pgfusepath{stroke,fill}%
}%
\begin{pgfscope}%
\pgfsys@transformshift{0.780208in}{7.540235in}%
\pgfsys@useobject{currentmarker}{}%
\end{pgfscope}%
\end{pgfscope}%
\begin{pgfscope}%
\definecolor{textcolor}{rgb}{0.000000,0.000000,0.000000}%
\pgfsetstrokecolor{textcolor}%
\pgfsetfillcolor{textcolor}%
\pgftext[x=0.462107in,y=7.487473in,left,base]{\color{textcolor}\sffamily\fontsize{10.000000}{12.000000}\selectfont 1.6}%
\end{pgfscope}%
\begin{pgfscope}%
\definecolor{textcolor}{rgb}{0.000000,0.000000,0.000000}%
\pgfsetstrokecolor{textcolor}%
\pgfsetfillcolor{textcolor}%
\pgftext[x=0.406551in,y=6.104167in,,bottom,rotate=90.000000]{\color{textcolor}\sffamily\fontsize{10.000000}{12.000000}\selectfont \(\displaystyle U\)}%
\end{pgfscope}%
\begin{pgfscope}%
\pgfpathrectangle{\pgfqpoint{0.780208in}{4.530556in}}{\pgfqpoint{3.096181in}{3.147222in}}%
\pgfusepath{clip}%
\pgfsetrectcap%
\pgfsetroundjoin%
\pgfsetlinewidth{1.505625pt}%
\definecolor{currentstroke}{rgb}{0.121569,0.466667,0.705882}%
\pgfsetstrokecolor{currentstroke}%
\pgfsetdash{}{0pt}%
\pgfpathmoveto{\pgfqpoint{0.920944in}{6.465251in}}%
\pgfpathlineto{\pgfqpoint{0.993116in}{6.465251in}}%
\pgfpathlineto{\pgfqpoint{1.065288in}{6.465251in}}%
\pgfpathlineto{\pgfqpoint{1.137460in}{6.465251in}}%
\pgfpathlineto{\pgfqpoint{1.209632in}{6.465251in}}%
\pgfpathlineto{\pgfqpoint{1.281804in}{6.465251in}}%
\pgfpathlineto{\pgfqpoint{1.353976in}{6.465251in}}%
\pgfpathlineto{\pgfqpoint{1.426148in}{6.465251in}}%
\pgfpathlineto{\pgfqpoint{1.498320in}{6.465250in}}%
\pgfpathlineto{\pgfqpoint{1.570492in}{6.465254in}}%
\pgfpathlineto{\pgfqpoint{1.642664in}{6.465223in}}%
\pgfpathlineto{\pgfqpoint{1.714836in}{6.465441in}}%
\pgfpathlineto{\pgfqpoint{1.787008in}{6.464110in}}%
\pgfpathlineto{\pgfqpoint{1.859180in}{6.471076in}}%
\pgfpathlineto{\pgfqpoint{1.931352in}{6.440356in}}%
\pgfpathlineto{\pgfqpoint{2.003524in}{6.551433in}}%
\pgfpathlineto{\pgfqpoint{2.075696in}{6.233411in}}%
\pgfpathlineto{\pgfqpoint{2.147869in}{6.905039in}}%
\pgfpathlineto{\pgfqpoint{2.220041in}{6.012608in}}%
\pgfpathlineto{\pgfqpoint{2.292213in}{6.262360in}}%
\pgfpathlineto{\pgfqpoint{2.364385in}{7.534722in}}%
\pgfpathlineto{\pgfqpoint{2.436557in}{5.745546in}}%
\pgfpathlineto{\pgfqpoint{2.508729in}{4.705283in}}%
\pgfpathlineto{\pgfqpoint{2.580901in}{4.673621in}}%
\pgfpathlineto{\pgfqpoint{2.653073in}{4.673611in}}%
\pgfpathlineto{\pgfqpoint{2.725245in}{4.673611in}}%
\pgfpathlineto{\pgfqpoint{2.797417in}{4.673611in}}%
\pgfpathlineto{\pgfqpoint{2.869589in}{4.673611in}}%
\pgfpathlineto{\pgfqpoint{2.941761in}{4.673611in}}%
\pgfpathlineto{\pgfqpoint{3.013933in}{4.673611in}}%
\pgfpathlineto{\pgfqpoint{3.086105in}{4.673611in}}%
\pgfpathlineto{\pgfqpoint{3.158277in}{4.673611in}}%
\pgfpathlineto{\pgfqpoint{3.230449in}{4.673611in}}%
\pgfpathlineto{\pgfqpoint{3.302621in}{4.673611in}}%
\pgfpathlineto{\pgfqpoint{3.374793in}{4.673611in}}%
\pgfpathlineto{\pgfqpoint{3.446965in}{4.673611in}}%
\pgfpathlineto{\pgfqpoint{3.519137in}{4.673611in}}%
\pgfpathlineto{\pgfqpoint{3.591309in}{4.673611in}}%
\pgfpathlineto{\pgfqpoint{3.663481in}{4.673611in}}%
\pgfpathlineto{\pgfqpoint{3.735653in}{4.673611in}}%
\pgfusepath{stroke}%
\end{pgfscope}%
\begin{pgfscope}%
\pgfpathrectangle{\pgfqpoint{0.780208in}{4.530556in}}{\pgfqpoint{3.096181in}{3.147222in}}%
\pgfusepath{clip}%
\pgfsetrectcap%
\pgfsetroundjoin%
\pgfsetlinewidth{1.505625pt}%
\definecolor{currentstroke}{rgb}{1.000000,0.498039,0.054902}%
\pgfsetstrokecolor{currentstroke}%
\pgfsetdash{}{0pt}%
\pgfpathmoveto{\pgfqpoint{0.920944in}{6.465251in}}%
\pgfpathlineto{\pgfqpoint{0.993116in}{6.465251in}}%
\pgfpathlineto{\pgfqpoint{1.065288in}{6.465251in}}%
\pgfpathlineto{\pgfqpoint{1.137460in}{6.465251in}}%
\pgfpathlineto{\pgfqpoint{1.209632in}{6.465251in}}%
\pgfpathlineto{\pgfqpoint{1.281804in}{6.465251in}}%
\pgfpathlineto{\pgfqpoint{1.353976in}{6.465251in}}%
\pgfpathlineto{\pgfqpoint{1.426148in}{6.465251in}}%
\pgfpathlineto{\pgfqpoint{1.498320in}{6.465250in}}%
\pgfpathlineto{\pgfqpoint{1.570492in}{6.465253in}}%
\pgfpathlineto{\pgfqpoint{1.642664in}{6.465233in}}%
\pgfpathlineto{\pgfqpoint{1.714836in}{6.465374in}}%
\pgfpathlineto{\pgfqpoint{1.787008in}{6.464513in}}%
\pgfpathlineto{\pgfqpoint{1.859180in}{6.469009in}}%
\pgfpathlineto{\pgfqpoint{1.931352in}{6.449176in}}%
\pgfpathlineto{\pgfqpoint{2.003524in}{6.520223in}}%
\pgfpathlineto{\pgfqpoint{2.075696in}{6.312656in}}%
\pgfpathlineto{\pgfqpoint{2.147869in}{6.737870in}}%
\pgfpathlineto{\pgfqpoint{2.220041in}{6.141005in}}%
\pgfpathlineto{\pgfqpoint{2.292213in}{6.360959in}}%
\pgfpathlineto{\pgfqpoint{2.364385in}{7.069697in}}%
\pgfpathlineto{\pgfqpoint{2.436557in}{5.992983in}}%
\pgfpathlineto{\pgfqpoint{2.508729in}{4.807398in}}%
\pgfpathlineto{\pgfqpoint{2.580901in}{4.674134in}}%
\pgfpathlineto{\pgfqpoint{2.653073in}{4.673611in}}%
\pgfpathlineto{\pgfqpoint{2.725245in}{4.673611in}}%
\pgfpathlineto{\pgfqpoint{2.797417in}{4.673611in}}%
\pgfpathlineto{\pgfqpoint{2.869589in}{4.673611in}}%
\pgfpathlineto{\pgfqpoint{2.941761in}{4.673611in}}%
\pgfpathlineto{\pgfqpoint{3.013933in}{4.673611in}}%
\pgfpathlineto{\pgfqpoint{3.086105in}{4.673611in}}%
\pgfpathlineto{\pgfqpoint{3.158277in}{4.673611in}}%
\pgfpathlineto{\pgfqpoint{3.230449in}{4.673611in}}%
\pgfpathlineto{\pgfqpoint{3.302621in}{4.673611in}}%
\pgfpathlineto{\pgfqpoint{3.374793in}{4.673611in}}%
\pgfpathlineto{\pgfqpoint{3.446965in}{4.673611in}}%
\pgfpathlineto{\pgfqpoint{3.519137in}{4.673611in}}%
\pgfpathlineto{\pgfqpoint{3.591309in}{4.673611in}}%
\pgfpathlineto{\pgfqpoint{3.663481in}{4.673611in}}%
\pgfpathlineto{\pgfqpoint{3.735653in}{4.673611in}}%
\pgfusepath{stroke}%
\end{pgfscope}%
\begin{pgfscope}%
\pgfpathrectangle{\pgfqpoint{0.780208in}{4.530556in}}{\pgfqpoint{3.096181in}{3.147222in}}%
\pgfusepath{clip}%
\pgfsetrectcap%
\pgfsetroundjoin%
\pgfsetlinewidth{1.505625pt}%
\definecolor{currentstroke}{rgb}{0.172549,0.627451,0.172549}%
\pgfsetstrokecolor{currentstroke}%
\pgfsetdash{}{0pt}%
\pgfpathmoveto{\pgfqpoint{0.920944in}{6.465251in}}%
\pgfpathlineto{\pgfqpoint{0.993116in}{6.465251in}}%
\pgfpathlineto{\pgfqpoint{1.065288in}{6.465251in}}%
\pgfpathlineto{\pgfqpoint{1.137460in}{6.465251in}}%
\pgfpathlineto{\pgfqpoint{1.209632in}{6.465251in}}%
\pgfpathlineto{\pgfqpoint{1.281804in}{6.465251in}}%
\pgfpathlineto{\pgfqpoint{1.353976in}{6.465251in}}%
\pgfpathlineto{\pgfqpoint{1.426148in}{6.465251in}}%
\pgfpathlineto{\pgfqpoint{1.498320in}{6.465251in}}%
\pgfpathlineto{\pgfqpoint{1.570492in}{6.465251in}}%
\pgfpathlineto{\pgfqpoint{1.642664in}{6.465251in}}%
\pgfpathlineto{\pgfqpoint{1.714836in}{6.465251in}}%
\pgfpathlineto{\pgfqpoint{1.787008in}{6.465251in}}%
\pgfpathlineto{\pgfqpoint{1.859180in}{6.465251in}}%
\pgfpathlineto{\pgfqpoint{1.931352in}{6.465251in}}%
\pgfpathlineto{\pgfqpoint{2.003524in}{6.465251in}}%
\pgfpathlineto{\pgfqpoint{2.075696in}{6.465251in}}%
\pgfpathlineto{\pgfqpoint{2.147869in}{6.465251in}}%
\pgfpathlineto{\pgfqpoint{2.220041in}{6.465251in}}%
\pgfpathlineto{\pgfqpoint{2.292213in}{6.465251in}}%
\pgfpathlineto{\pgfqpoint{2.364385in}{6.409108in}}%
\pgfpathlineto{\pgfqpoint{2.436557in}{6.029311in}}%
\pgfpathlineto{\pgfqpoint{2.508729in}{5.142881in}}%
\pgfpathlineto{\pgfqpoint{2.580901in}{4.696403in}}%
\pgfpathlineto{\pgfqpoint{2.653073in}{4.673632in}}%
\pgfpathlineto{\pgfqpoint{2.725245in}{4.673611in}}%
\pgfpathlineto{\pgfqpoint{2.797417in}{4.673611in}}%
\pgfpathlineto{\pgfqpoint{2.869589in}{4.673611in}}%
\pgfpathlineto{\pgfqpoint{2.941761in}{4.673611in}}%
\pgfpathlineto{\pgfqpoint{3.013933in}{4.673611in}}%
\pgfpathlineto{\pgfqpoint{3.086105in}{4.673611in}}%
\pgfpathlineto{\pgfqpoint{3.158277in}{4.673611in}}%
\pgfpathlineto{\pgfqpoint{3.230449in}{4.673611in}}%
\pgfpathlineto{\pgfqpoint{3.302621in}{4.673611in}}%
\pgfpathlineto{\pgfqpoint{3.374793in}{4.673611in}}%
\pgfpathlineto{\pgfqpoint{3.446965in}{4.673611in}}%
\pgfpathlineto{\pgfqpoint{3.519137in}{4.673611in}}%
\pgfpathlineto{\pgfqpoint{3.591309in}{4.673611in}}%
\pgfpathlineto{\pgfqpoint{3.663481in}{4.673611in}}%
\pgfpathlineto{\pgfqpoint{3.735653in}{4.673611in}}%
\pgfusepath{stroke}%
\end{pgfscope}%
\begin{pgfscope}%
\pgfpathrectangle{\pgfqpoint{0.780208in}{4.530556in}}{\pgfqpoint{3.096181in}{3.147222in}}%
\pgfusepath{clip}%
\pgfsetrectcap%
\pgfsetroundjoin%
\pgfsetlinewidth{1.505625pt}%
\definecolor{currentstroke}{rgb}{0.839216,0.152941,0.156863}%
\pgfsetstrokecolor{currentstroke}%
\pgfsetdash{}{0pt}%
\pgfpathmoveto{\pgfqpoint{0.920944in}{6.465251in}}%
\pgfpathlineto{\pgfqpoint{0.993116in}{6.465251in}}%
\pgfpathlineto{\pgfqpoint{1.065288in}{6.465251in}}%
\pgfpathlineto{\pgfqpoint{1.137460in}{6.465251in}}%
\pgfpathlineto{\pgfqpoint{1.209632in}{6.465251in}}%
\pgfpathlineto{\pgfqpoint{1.281804in}{6.465251in}}%
\pgfpathlineto{\pgfqpoint{1.353976in}{6.465251in}}%
\pgfpathlineto{\pgfqpoint{1.426148in}{6.465251in}}%
\pgfpathlineto{\pgfqpoint{1.498320in}{6.465251in}}%
\pgfpathlineto{\pgfqpoint{1.570492in}{6.465251in}}%
\pgfpathlineto{\pgfqpoint{1.642664in}{6.465251in}}%
\pgfpathlineto{\pgfqpoint{1.714836in}{6.465251in}}%
\pgfpathlineto{\pgfqpoint{1.787008in}{6.465251in}}%
\pgfpathlineto{\pgfqpoint{1.859180in}{6.465251in}}%
\pgfpathlineto{\pgfqpoint{1.931352in}{6.465251in}}%
\pgfpathlineto{\pgfqpoint{2.003524in}{6.465251in}}%
\pgfpathlineto{\pgfqpoint{2.075696in}{6.465251in}}%
\pgfpathlineto{\pgfqpoint{2.147869in}{6.465251in}}%
\pgfpathlineto{\pgfqpoint{2.220041in}{6.465251in}}%
\pgfpathlineto{\pgfqpoint{2.292213in}{6.465251in}}%
\pgfpathlineto{\pgfqpoint{2.364385in}{4.673611in}}%
\pgfpathlineto{\pgfqpoint{2.436557in}{4.673611in}}%
\pgfpathlineto{\pgfqpoint{2.508729in}{4.673611in}}%
\pgfpathlineto{\pgfqpoint{2.580901in}{4.673611in}}%
\pgfpathlineto{\pgfqpoint{2.653073in}{4.673611in}}%
\pgfpathlineto{\pgfqpoint{2.725245in}{4.673611in}}%
\pgfpathlineto{\pgfqpoint{2.797417in}{4.673611in}}%
\pgfpathlineto{\pgfqpoint{2.869589in}{4.673611in}}%
\pgfpathlineto{\pgfqpoint{2.941761in}{4.673611in}}%
\pgfpathlineto{\pgfqpoint{3.013933in}{4.673611in}}%
\pgfpathlineto{\pgfqpoint{3.086105in}{4.673611in}}%
\pgfpathlineto{\pgfqpoint{3.158277in}{4.673611in}}%
\pgfpathlineto{\pgfqpoint{3.230449in}{4.673611in}}%
\pgfpathlineto{\pgfqpoint{3.302621in}{4.673611in}}%
\pgfpathlineto{\pgfqpoint{3.374793in}{4.673611in}}%
\pgfpathlineto{\pgfqpoint{3.446965in}{4.673611in}}%
\pgfpathlineto{\pgfqpoint{3.519137in}{4.673611in}}%
\pgfpathlineto{\pgfqpoint{3.591309in}{4.673611in}}%
\pgfpathlineto{\pgfqpoint{3.663481in}{4.673611in}}%
\pgfpathlineto{\pgfqpoint{3.735653in}{4.673611in}}%
\pgfusepath{stroke}%
\end{pgfscope}%
\begin{pgfscope}%
\pgfpathrectangle{\pgfqpoint{0.780208in}{4.530556in}}{\pgfqpoint{3.096181in}{3.147222in}}%
\pgfusepath{clip}%
\pgfsetrectcap%
\pgfsetroundjoin%
\pgfsetlinewidth{0.501875pt}%
\definecolor{currentstroke}{rgb}{0.000000,0.000000,0.000000}%
\pgfsetstrokecolor{currentstroke}%
\pgfsetdash{}{0pt}%
\pgfpathmoveto{\pgfqpoint{0.920944in}{6.465251in}}%
\pgfpathlineto{\pgfqpoint{0.993116in}{6.465251in}}%
\pgfpathlineto{\pgfqpoint{1.065288in}{6.465251in}}%
\pgfpathlineto{\pgfqpoint{1.137460in}{6.465251in}}%
\pgfpathlineto{\pgfqpoint{1.209632in}{6.465251in}}%
\pgfpathlineto{\pgfqpoint{1.281804in}{6.465251in}}%
\pgfpathlineto{\pgfqpoint{1.353976in}{6.465251in}}%
\pgfpathlineto{\pgfqpoint{1.426148in}{6.465251in}}%
\pgfpathlineto{\pgfqpoint{1.498320in}{6.465251in}}%
\pgfpathlineto{\pgfqpoint{1.570492in}{6.465251in}}%
\pgfpathlineto{\pgfqpoint{1.642664in}{6.465251in}}%
\pgfpathlineto{\pgfqpoint{1.714836in}{6.465251in}}%
\pgfpathlineto{\pgfqpoint{1.787008in}{6.465251in}}%
\pgfpathlineto{\pgfqpoint{1.859180in}{6.465251in}}%
\pgfpathlineto{\pgfqpoint{1.931352in}{6.465251in}}%
\pgfpathlineto{\pgfqpoint{2.003524in}{6.465251in}}%
\pgfpathlineto{\pgfqpoint{2.075696in}{6.465251in}}%
\pgfpathlineto{\pgfqpoint{2.147869in}{6.465251in}}%
\pgfpathlineto{\pgfqpoint{2.220041in}{6.465251in}}%
\pgfpathlineto{\pgfqpoint{2.292213in}{6.465251in}}%
\pgfpathlineto{\pgfqpoint{2.364385in}{6.465251in}}%
\pgfpathlineto{\pgfqpoint{2.436557in}{6.465251in}}%
\pgfpathlineto{\pgfqpoint{2.508729in}{4.673611in}}%
\pgfpathlineto{\pgfqpoint{2.580901in}{4.673611in}}%
\pgfpathlineto{\pgfqpoint{2.653073in}{4.673611in}}%
\pgfpathlineto{\pgfqpoint{2.725245in}{4.673611in}}%
\pgfpathlineto{\pgfqpoint{2.797417in}{4.673611in}}%
\pgfpathlineto{\pgfqpoint{2.869589in}{4.673611in}}%
\pgfpathlineto{\pgfqpoint{2.941761in}{4.673611in}}%
\pgfpathlineto{\pgfqpoint{3.013933in}{4.673611in}}%
\pgfpathlineto{\pgfqpoint{3.086105in}{4.673611in}}%
\pgfpathlineto{\pgfqpoint{3.158277in}{4.673611in}}%
\pgfpathlineto{\pgfqpoint{3.230449in}{4.673611in}}%
\pgfpathlineto{\pgfqpoint{3.302621in}{4.673611in}}%
\pgfpathlineto{\pgfqpoint{3.374793in}{4.673611in}}%
\pgfpathlineto{\pgfqpoint{3.446965in}{4.673611in}}%
\pgfpathlineto{\pgfqpoint{3.519137in}{4.673611in}}%
\pgfpathlineto{\pgfqpoint{3.591309in}{4.673611in}}%
\pgfpathlineto{\pgfqpoint{3.663481in}{4.673611in}}%
\pgfpathlineto{\pgfqpoint{3.735653in}{4.673611in}}%
\pgfusepath{stroke}%
\end{pgfscope}%
\begin{pgfscope}%
\pgfsetrectcap%
\pgfsetmiterjoin%
\pgfsetlinewidth{0.803000pt}%
\definecolor{currentstroke}{rgb}{0.000000,0.000000,0.000000}%
\pgfsetstrokecolor{currentstroke}%
\pgfsetdash{}{0pt}%
\pgfpathmoveto{\pgfqpoint{0.780208in}{4.530556in}}%
\pgfpathlineto{\pgfqpoint{0.780208in}{7.677778in}}%
\pgfusepath{stroke}%
\end{pgfscope}%
\begin{pgfscope}%
\pgfsetrectcap%
\pgfsetmiterjoin%
\pgfsetlinewidth{0.803000pt}%
\definecolor{currentstroke}{rgb}{0.000000,0.000000,0.000000}%
\pgfsetstrokecolor{currentstroke}%
\pgfsetdash{}{0pt}%
\pgfpathmoveto{\pgfqpoint{3.876389in}{4.530556in}}%
\pgfpathlineto{\pgfqpoint{3.876389in}{7.677778in}}%
\pgfusepath{stroke}%
\end{pgfscope}%
\begin{pgfscope}%
\pgfsetrectcap%
\pgfsetmiterjoin%
\pgfsetlinewidth{0.803000pt}%
\definecolor{currentstroke}{rgb}{0.000000,0.000000,0.000000}%
\pgfsetstrokecolor{currentstroke}%
\pgfsetdash{}{0pt}%
\pgfpathmoveto{\pgfqpoint{0.780208in}{4.530556in}}%
\pgfpathlineto{\pgfqpoint{3.876389in}{4.530556in}}%
\pgfusepath{stroke}%
\end{pgfscope}%
\begin{pgfscope}%
\pgfsetrectcap%
\pgfsetmiterjoin%
\pgfsetlinewidth{0.803000pt}%
\definecolor{currentstroke}{rgb}{0.000000,0.000000,0.000000}%
\pgfsetstrokecolor{currentstroke}%
\pgfsetdash{}{0pt}%
\pgfpathmoveto{\pgfqpoint{0.780208in}{7.677778in}}%
\pgfpathlineto{\pgfqpoint{3.876389in}{7.677778in}}%
\pgfusepath{stroke}%
\end{pgfscope}%
\begin{pgfscope}%
\definecolor{textcolor}{rgb}{0.000000,0.000000,0.000000}%
\pgfsetstrokecolor{textcolor}%
\pgfsetfillcolor{textcolor}%
\pgftext[x=2.328299in,y=7.761111in,,base]{\color{textcolor}\sffamily\fontsize{12.000000}{14.400000}\selectfont \(\displaystyle  t = 0.4 \)}%
\end{pgfscope}%
\begin{pgfscope}%
\pgfsetbuttcap%
\pgfsetmiterjoin%
\definecolor{currentfill}{rgb}{1.000000,1.000000,1.000000}%
\pgfsetfillcolor{currentfill}%
\pgfsetlinewidth{0.000000pt}%
\definecolor{currentstroke}{rgb}{0.000000,0.000000,0.000000}%
\pgfsetstrokecolor{currentstroke}%
\pgfsetstrokeopacity{0.000000}%
\pgfsetdash{}{0pt}%
\pgfpathmoveto{\pgfqpoint{4.705208in}{4.530556in}}%
\pgfpathlineto{\pgfqpoint{7.801389in}{4.530556in}}%
\pgfpathlineto{\pgfqpoint{7.801389in}{7.677778in}}%
\pgfpathlineto{\pgfqpoint{4.705208in}{7.677778in}}%
\pgfpathclose%
\pgfusepath{fill}%
\end{pgfscope}%
\begin{pgfscope}%
\pgfsetbuttcap%
\pgfsetroundjoin%
\definecolor{currentfill}{rgb}{0.000000,0.000000,0.000000}%
\pgfsetfillcolor{currentfill}%
\pgfsetlinewidth{0.803000pt}%
\definecolor{currentstroke}{rgb}{0.000000,0.000000,0.000000}%
\pgfsetstrokecolor{currentstroke}%
\pgfsetdash{}{0pt}%
\pgfsys@defobject{currentmarker}{\pgfqpoint{0.000000in}{-0.048611in}}{\pgfqpoint{0.000000in}{0.000000in}}{%
\pgfpathmoveto{\pgfqpoint{0.000000in}{0.000000in}}%
\pgfpathlineto{\pgfqpoint{0.000000in}{-0.048611in}}%
\pgfusepath{stroke,fill}%
}%
\begin{pgfscope}%
\pgfsys@transformshift{4.809858in}{4.530556in}%
\pgfsys@useobject{currentmarker}{}%
\end{pgfscope}%
\end{pgfscope}%
\begin{pgfscope}%
\definecolor{textcolor}{rgb}{0.000000,0.000000,0.000000}%
\pgfsetstrokecolor{textcolor}%
\pgfsetfillcolor{textcolor}%
\pgftext[x=4.809858in,y=4.433333in,,top]{\color{textcolor}\sffamily\fontsize{10.000000}{12.000000}\selectfont −2}%
\end{pgfscope}%
\begin{pgfscope}%
\pgfsetbuttcap%
\pgfsetroundjoin%
\definecolor{currentfill}{rgb}{0.000000,0.000000,0.000000}%
\pgfsetfillcolor{currentfill}%
\pgfsetlinewidth{0.803000pt}%
\definecolor{currentstroke}{rgb}{0.000000,0.000000,0.000000}%
\pgfsetstrokecolor{currentstroke}%
\pgfsetdash{}{0pt}%
\pgfsys@defobject{currentmarker}{\pgfqpoint{0.000000in}{-0.048611in}}{\pgfqpoint{0.000000in}{0.000000in}}{%
\pgfpathmoveto{\pgfqpoint{0.000000in}{0.000000in}}%
\pgfpathlineto{\pgfqpoint{0.000000in}{-0.048611in}}%
\pgfusepath{stroke,fill}%
}%
\begin{pgfscope}%
\pgfsys@transformshift{5.531578in}{4.530556in}%
\pgfsys@useobject{currentmarker}{}%
\end{pgfscope}%
\end{pgfscope}%
\begin{pgfscope}%
\definecolor{textcolor}{rgb}{0.000000,0.000000,0.000000}%
\pgfsetstrokecolor{textcolor}%
\pgfsetfillcolor{textcolor}%
\pgftext[x=5.531578in,y=4.433333in,,top]{\color{textcolor}\sffamily\fontsize{10.000000}{12.000000}\selectfont −1}%
\end{pgfscope}%
\begin{pgfscope}%
\pgfsetbuttcap%
\pgfsetroundjoin%
\definecolor{currentfill}{rgb}{0.000000,0.000000,0.000000}%
\pgfsetfillcolor{currentfill}%
\pgfsetlinewidth{0.803000pt}%
\definecolor{currentstroke}{rgb}{0.000000,0.000000,0.000000}%
\pgfsetstrokecolor{currentstroke}%
\pgfsetdash{}{0pt}%
\pgfsys@defobject{currentmarker}{\pgfqpoint{0.000000in}{-0.048611in}}{\pgfqpoint{0.000000in}{0.000000in}}{%
\pgfpathmoveto{\pgfqpoint{0.000000in}{0.000000in}}%
\pgfpathlineto{\pgfqpoint{0.000000in}{-0.048611in}}%
\pgfusepath{stroke,fill}%
}%
\begin{pgfscope}%
\pgfsys@transformshift{6.253299in}{4.530556in}%
\pgfsys@useobject{currentmarker}{}%
\end{pgfscope}%
\end{pgfscope}%
\begin{pgfscope}%
\definecolor{textcolor}{rgb}{0.000000,0.000000,0.000000}%
\pgfsetstrokecolor{textcolor}%
\pgfsetfillcolor{textcolor}%
\pgftext[x=6.253299in,y=4.433333in,,top]{\color{textcolor}\sffamily\fontsize{10.000000}{12.000000}\selectfont 0}%
\end{pgfscope}%
\begin{pgfscope}%
\pgfsetbuttcap%
\pgfsetroundjoin%
\definecolor{currentfill}{rgb}{0.000000,0.000000,0.000000}%
\pgfsetfillcolor{currentfill}%
\pgfsetlinewidth{0.803000pt}%
\definecolor{currentstroke}{rgb}{0.000000,0.000000,0.000000}%
\pgfsetstrokecolor{currentstroke}%
\pgfsetdash{}{0pt}%
\pgfsys@defobject{currentmarker}{\pgfqpoint{0.000000in}{-0.048611in}}{\pgfqpoint{0.000000in}{0.000000in}}{%
\pgfpathmoveto{\pgfqpoint{0.000000in}{0.000000in}}%
\pgfpathlineto{\pgfqpoint{0.000000in}{-0.048611in}}%
\pgfusepath{stroke,fill}%
}%
\begin{pgfscope}%
\pgfsys@transformshift{6.975019in}{4.530556in}%
\pgfsys@useobject{currentmarker}{}%
\end{pgfscope}%
\end{pgfscope}%
\begin{pgfscope}%
\definecolor{textcolor}{rgb}{0.000000,0.000000,0.000000}%
\pgfsetstrokecolor{textcolor}%
\pgfsetfillcolor{textcolor}%
\pgftext[x=6.975019in,y=4.433333in,,top]{\color{textcolor}\sffamily\fontsize{10.000000}{12.000000}\selectfont 1}%
\end{pgfscope}%
\begin{pgfscope}%
\pgfsetbuttcap%
\pgfsetroundjoin%
\definecolor{currentfill}{rgb}{0.000000,0.000000,0.000000}%
\pgfsetfillcolor{currentfill}%
\pgfsetlinewidth{0.803000pt}%
\definecolor{currentstroke}{rgb}{0.000000,0.000000,0.000000}%
\pgfsetstrokecolor{currentstroke}%
\pgfsetdash{}{0pt}%
\pgfsys@defobject{currentmarker}{\pgfqpoint{0.000000in}{-0.048611in}}{\pgfqpoint{0.000000in}{0.000000in}}{%
\pgfpathmoveto{\pgfqpoint{0.000000in}{0.000000in}}%
\pgfpathlineto{\pgfqpoint{0.000000in}{-0.048611in}}%
\pgfusepath{stroke,fill}%
}%
\begin{pgfscope}%
\pgfsys@transformshift{7.696739in}{4.530556in}%
\pgfsys@useobject{currentmarker}{}%
\end{pgfscope}%
\end{pgfscope}%
\begin{pgfscope}%
\definecolor{textcolor}{rgb}{0.000000,0.000000,0.000000}%
\pgfsetstrokecolor{textcolor}%
\pgfsetfillcolor{textcolor}%
\pgftext[x=7.696739in,y=4.433333in,,top]{\color{textcolor}\sffamily\fontsize{10.000000}{12.000000}\selectfont 2}%
\end{pgfscope}%
\begin{pgfscope}%
\definecolor{textcolor}{rgb}{0.000000,0.000000,0.000000}%
\pgfsetstrokecolor{textcolor}%
\pgfsetfillcolor{textcolor}%
\pgftext[x=6.253299in,y=4.243365in,,top]{\color{textcolor}\sffamily\fontsize{10.000000}{12.000000}\selectfont \(\displaystyle x\)}%
\end{pgfscope}%
\begin{pgfscope}%
\pgfsetbuttcap%
\pgfsetroundjoin%
\definecolor{currentfill}{rgb}{0.000000,0.000000,0.000000}%
\pgfsetfillcolor{currentfill}%
\pgfsetlinewidth{0.803000pt}%
\definecolor{currentstroke}{rgb}{0.000000,0.000000,0.000000}%
\pgfsetstrokecolor{currentstroke}%
\pgfsetdash{}{0pt}%
\pgfsys@defobject{currentmarker}{\pgfqpoint{-0.048611in}{0.000000in}}{\pgfqpoint{0.000000in}{0.000000in}}{%
\pgfpathmoveto{\pgfqpoint{0.000000in}{0.000000in}}%
\pgfpathlineto{\pgfqpoint{-0.048611in}{0.000000in}}%
\pgfusepath{stroke,fill}%
}%
\begin{pgfscope}%
\pgfsys@transformshift{4.705208in}{4.673611in}%
\pgfsys@useobject{currentmarker}{}%
\end{pgfscope}%
\end{pgfscope}%
\begin{pgfscope}%
\definecolor{textcolor}{rgb}{0.000000,0.000000,0.000000}%
\pgfsetstrokecolor{textcolor}%
\pgfsetfillcolor{textcolor}%
\pgftext[x=4.298741in,y=4.620850in,left,base]{\color{textcolor}\sffamily\fontsize{10.000000}{12.000000}\selectfont 0.00}%
\end{pgfscope}%
\begin{pgfscope}%
\pgfsetbuttcap%
\pgfsetroundjoin%
\definecolor{currentfill}{rgb}{0.000000,0.000000,0.000000}%
\pgfsetfillcolor{currentfill}%
\pgfsetlinewidth{0.803000pt}%
\definecolor{currentstroke}{rgb}{0.000000,0.000000,0.000000}%
\pgfsetstrokecolor{currentstroke}%
\pgfsetdash{}{0pt}%
\pgfsys@defobject{currentmarker}{\pgfqpoint{-0.048611in}{0.000000in}}{\pgfqpoint{0.000000in}{0.000000in}}{%
\pgfpathmoveto{\pgfqpoint{0.000000in}{0.000000in}}%
\pgfpathlineto{\pgfqpoint{-0.048611in}{0.000000in}}%
\pgfusepath{stroke,fill}%
}%
\begin{pgfscope}%
\pgfsys@transformshift{4.705208in}{5.103134in}%
\pgfsys@useobject{currentmarker}{}%
\end{pgfscope}%
\end{pgfscope}%
\begin{pgfscope}%
\definecolor{textcolor}{rgb}{0.000000,0.000000,0.000000}%
\pgfsetstrokecolor{textcolor}%
\pgfsetfillcolor{textcolor}%
\pgftext[x=4.298741in,y=5.050372in,left,base]{\color{textcolor}\sffamily\fontsize{10.000000}{12.000000}\selectfont 0.25}%
\end{pgfscope}%
\begin{pgfscope}%
\pgfsetbuttcap%
\pgfsetroundjoin%
\definecolor{currentfill}{rgb}{0.000000,0.000000,0.000000}%
\pgfsetfillcolor{currentfill}%
\pgfsetlinewidth{0.803000pt}%
\definecolor{currentstroke}{rgb}{0.000000,0.000000,0.000000}%
\pgfsetstrokecolor{currentstroke}%
\pgfsetdash{}{0pt}%
\pgfsys@defobject{currentmarker}{\pgfqpoint{-0.048611in}{0.000000in}}{\pgfqpoint{0.000000in}{0.000000in}}{%
\pgfpathmoveto{\pgfqpoint{0.000000in}{0.000000in}}%
\pgfpathlineto{\pgfqpoint{-0.048611in}{0.000000in}}%
\pgfusepath{stroke,fill}%
}%
\begin{pgfscope}%
\pgfsys@transformshift{4.705208in}{5.532657in}%
\pgfsys@useobject{currentmarker}{}%
\end{pgfscope}%
\end{pgfscope}%
\begin{pgfscope}%
\definecolor{textcolor}{rgb}{0.000000,0.000000,0.000000}%
\pgfsetstrokecolor{textcolor}%
\pgfsetfillcolor{textcolor}%
\pgftext[x=4.298741in,y=5.479895in,left,base]{\color{textcolor}\sffamily\fontsize{10.000000}{12.000000}\selectfont 0.50}%
\end{pgfscope}%
\begin{pgfscope}%
\pgfsetbuttcap%
\pgfsetroundjoin%
\definecolor{currentfill}{rgb}{0.000000,0.000000,0.000000}%
\pgfsetfillcolor{currentfill}%
\pgfsetlinewidth{0.803000pt}%
\definecolor{currentstroke}{rgb}{0.000000,0.000000,0.000000}%
\pgfsetstrokecolor{currentstroke}%
\pgfsetdash{}{0pt}%
\pgfsys@defobject{currentmarker}{\pgfqpoint{-0.048611in}{0.000000in}}{\pgfqpoint{0.000000in}{0.000000in}}{%
\pgfpathmoveto{\pgfqpoint{0.000000in}{0.000000in}}%
\pgfpathlineto{\pgfqpoint{-0.048611in}{0.000000in}}%
\pgfusepath{stroke,fill}%
}%
\begin{pgfscope}%
\pgfsys@transformshift{4.705208in}{5.962179in}%
\pgfsys@useobject{currentmarker}{}%
\end{pgfscope}%
\end{pgfscope}%
\begin{pgfscope}%
\definecolor{textcolor}{rgb}{0.000000,0.000000,0.000000}%
\pgfsetstrokecolor{textcolor}%
\pgfsetfillcolor{textcolor}%
\pgftext[x=4.298741in,y=5.909418in,left,base]{\color{textcolor}\sffamily\fontsize{10.000000}{12.000000}\selectfont 0.75}%
\end{pgfscope}%
\begin{pgfscope}%
\pgfsetbuttcap%
\pgfsetroundjoin%
\definecolor{currentfill}{rgb}{0.000000,0.000000,0.000000}%
\pgfsetfillcolor{currentfill}%
\pgfsetlinewidth{0.803000pt}%
\definecolor{currentstroke}{rgb}{0.000000,0.000000,0.000000}%
\pgfsetstrokecolor{currentstroke}%
\pgfsetdash{}{0pt}%
\pgfsys@defobject{currentmarker}{\pgfqpoint{-0.048611in}{0.000000in}}{\pgfqpoint{0.000000in}{0.000000in}}{%
\pgfpathmoveto{\pgfqpoint{0.000000in}{0.000000in}}%
\pgfpathlineto{\pgfqpoint{-0.048611in}{0.000000in}}%
\pgfusepath{stroke,fill}%
}%
\begin{pgfscope}%
\pgfsys@transformshift{4.705208in}{6.391702in}%
\pgfsys@useobject{currentmarker}{}%
\end{pgfscope}%
\end{pgfscope}%
\begin{pgfscope}%
\definecolor{textcolor}{rgb}{0.000000,0.000000,0.000000}%
\pgfsetstrokecolor{textcolor}%
\pgfsetfillcolor{textcolor}%
\pgftext[x=4.298741in,y=6.338941in,left,base]{\color{textcolor}\sffamily\fontsize{10.000000}{12.000000}\selectfont 1.00}%
\end{pgfscope}%
\begin{pgfscope}%
\pgfsetbuttcap%
\pgfsetroundjoin%
\definecolor{currentfill}{rgb}{0.000000,0.000000,0.000000}%
\pgfsetfillcolor{currentfill}%
\pgfsetlinewidth{0.803000pt}%
\definecolor{currentstroke}{rgb}{0.000000,0.000000,0.000000}%
\pgfsetstrokecolor{currentstroke}%
\pgfsetdash{}{0pt}%
\pgfsys@defobject{currentmarker}{\pgfqpoint{-0.048611in}{0.000000in}}{\pgfqpoint{0.000000in}{0.000000in}}{%
\pgfpathmoveto{\pgfqpoint{0.000000in}{0.000000in}}%
\pgfpathlineto{\pgfqpoint{-0.048611in}{0.000000in}}%
\pgfusepath{stroke,fill}%
}%
\begin{pgfscope}%
\pgfsys@transformshift{4.705208in}{6.821225in}%
\pgfsys@useobject{currentmarker}{}%
\end{pgfscope}%
\end{pgfscope}%
\begin{pgfscope}%
\definecolor{textcolor}{rgb}{0.000000,0.000000,0.000000}%
\pgfsetstrokecolor{textcolor}%
\pgfsetfillcolor{textcolor}%
\pgftext[x=4.298741in,y=6.768463in,left,base]{\color{textcolor}\sffamily\fontsize{10.000000}{12.000000}\selectfont 1.25}%
\end{pgfscope}%
\begin{pgfscope}%
\pgfsetbuttcap%
\pgfsetroundjoin%
\definecolor{currentfill}{rgb}{0.000000,0.000000,0.000000}%
\pgfsetfillcolor{currentfill}%
\pgfsetlinewidth{0.803000pt}%
\definecolor{currentstroke}{rgb}{0.000000,0.000000,0.000000}%
\pgfsetstrokecolor{currentstroke}%
\pgfsetdash{}{0pt}%
\pgfsys@defobject{currentmarker}{\pgfqpoint{-0.048611in}{0.000000in}}{\pgfqpoint{0.000000in}{0.000000in}}{%
\pgfpathmoveto{\pgfqpoint{0.000000in}{0.000000in}}%
\pgfpathlineto{\pgfqpoint{-0.048611in}{0.000000in}}%
\pgfusepath{stroke,fill}%
}%
\begin{pgfscope}%
\pgfsys@transformshift{4.705208in}{7.250748in}%
\pgfsys@useobject{currentmarker}{}%
\end{pgfscope}%
\end{pgfscope}%
\begin{pgfscope}%
\definecolor{textcolor}{rgb}{0.000000,0.000000,0.000000}%
\pgfsetstrokecolor{textcolor}%
\pgfsetfillcolor{textcolor}%
\pgftext[x=4.298741in,y=7.197986in,left,base]{\color{textcolor}\sffamily\fontsize{10.000000}{12.000000}\selectfont 1.50}%
\end{pgfscope}%
\begin{pgfscope}%
\pgfsetbuttcap%
\pgfsetroundjoin%
\definecolor{currentfill}{rgb}{0.000000,0.000000,0.000000}%
\pgfsetfillcolor{currentfill}%
\pgfsetlinewidth{0.803000pt}%
\definecolor{currentstroke}{rgb}{0.000000,0.000000,0.000000}%
\pgfsetstrokecolor{currentstroke}%
\pgfsetdash{}{0pt}%
\pgfsys@defobject{currentmarker}{\pgfqpoint{-0.048611in}{0.000000in}}{\pgfqpoint{0.000000in}{0.000000in}}{%
\pgfpathmoveto{\pgfqpoint{0.000000in}{0.000000in}}%
\pgfpathlineto{\pgfqpoint{-0.048611in}{0.000000in}}%
\pgfusepath{stroke,fill}%
}%
\begin{pgfscope}%
\pgfsys@transformshift{4.705208in}{7.680271in}%
\pgfsys@useobject{currentmarker}{}%
\end{pgfscope}%
\end{pgfscope}%
\begin{pgfscope}%
\definecolor{textcolor}{rgb}{0.000000,0.000000,0.000000}%
\pgfsetstrokecolor{textcolor}%
\pgfsetfillcolor{textcolor}%
\pgftext[x=4.298741in,y=7.627509in,left,base]{\color{textcolor}\sffamily\fontsize{10.000000}{12.000000}\selectfont 1.75}%
\end{pgfscope}%
\begin{pgfscope}%
\definecolor{textcolor}{rgb}{0.000000,0.000000,0.000000}%
\pgfsetstrokecolor{textcolor}%
\pgfsetfillcolor{textcolor}%
\pgftext[x=4.243186in,y=6.104167in,,bottom,rotate=90.000000]{\color{textcolor}\sffamily\fontsize{10.000000}{12.000000}\selectfont \(\displaystyle U\)}%
\end{pgfscope}%
\begin{pgfscope}%
\pgfpathrectangle{\pgfqpoint{4.705208in}{4.530556in}}{\pgfqpoint{3.096181in}{3.147222in}}%
\pgfusepath{clip}%
\pgfsetrectcap%
\pgfsetroundjoin%
\pgfsetlinewidth{1.505625pt}%
\definecolor{currentstroke}{rgb}{0.121569,0.466667,0.705882}%
\pgfsetstrokecolor{currentstroke}%
\pgfsetdash{}{0pt}%
\pgfpathmoveto{\pgfqpoint{4.845944in}{6.391702in}}%
\pgfpathlineto{\pgfqpoint{4.918116in}{6.391702in}}%
\pgfpathlineto{\pgfqpoint{4.990288in}{6.391702in}}%
\pgfpathlineto{\pgfqpoint{5.062460in}{6.391702in}}%
\pgfpathlineto{\pgfqpoint{5.134632in}{6.391702in}}%
\pgfpathlineto{\pgfqpoint{5.206804in}{6.391702in}}%
\pgfpathlineto{\pgfqpoint{5.278976in}{6.391701in}}%
\pgfpathlineto{\pgfqpoint{5.351148in}{6.391713in}}%
\pgfpathlineto{\pgfqpoint{5.423320in}{6.391643in}}%
\pgfpathlineto{\pgfqpoint{5.495492in}{6.392005in}}%
\pgfpathlineto{\pgfqpoint{5.567664in}{6.390327in}}%
\pgfpathlineto{\pgfqpoint{5.639836in}{6.397191in}}%
\pgfpathlineto{\pgfqpoint{5.712008in}{6.372774in}}%
\pgfpathlineto{\pgfqpoint{5.784180in}{6.446716in}}%
\pgfpathlineto{\pgfqpoint{5.856352in}{6.261472in}}%
\pgfpathlineto{\pgfqpoint{5.928524in}{6.626393in}}%
\pgfpathlineto{\pgfqpoint{6.000696in}{6.119814in}}%
\pgfpathlineto{\pgfqpoint{6.072869in}{6.442428in}}%
\pgfpathlineto{\pgfqpoint{6.145041in}{6.814703in}}%
\pgfpathlineto{\pgfqpoint{6.217213in}{5.727270in}}%
\pgfpathlineto{\pgfqpoint{6.289385in}{6.223086in}}%
\pgfpathlineto{\pgfqpoint{6.361557in}{7.534722in}}%
\pgfpathlineto{\pgfqpoint{6.433729in}{5.706799in}}%
\pgfpathlineto{\pgfqpoint{6.505901in}{4.701786in}}%
\pgfpathlineto{\pgfqpoint{6.578073in}{4.673618in}}%
\pgfpathlineto{\pgfqpoint{6.650245in}{4.673611in}}%
\pgfpathlineto{\pgfqpoint{6.722417in}{4.673611in}}%
\pgfpathlineto{\pgfqpoint{6.794589in}{4.673611in}}%
\pgfpathlineto{\pgfqpoint{6.866761in}{4.673611in}}%
\pgfpathlineto{\pgfqpoint{6.938933in}{4.673611in}}%
\pgfpathlineto{\pgfqpoint{7.011105in}{4.673611in}}%
\pgfpathlineto{\pgfqpoint{7.083277in}{4.673611in}}%
\pgfpathlineto{\pgfqpoint{7.155449in}{4.673611in}}%
\pgfpathlineto{\pgfqpoint{7.227621in}{4.673611in}}%
\pgfpathlineto{\pgfqpoint{7.299793in}{4.673611in}}%
\pgfpathlineto{\pgfqpoint{7.371965in}{4.673611in}}%
\pgfpathlineto{\pgfqpoint{7.444137in}{4.673611in}}%
\pgfpathlineto{\pgfqpoint{7.516309in}{4.673611in}}%
\pgfpathlineto{\pgfqpoint{7.588481in}{4.673611in}}%
\pgfpathlineto{\pgfqpoint{7.660653in}{4.673611in}}%
\pgfusepath{stroke}%
\end{pgfscope}%
\begin{pgfscope}%
\pgfpathrectangle{\pgfqpoint{4.705208in}{4.530556in}}{\pgfqpoint{3.096181in}{3.147222in}}%
\pgfusepath{clip}%
\pgfsetrectcap%
\pgfsetroundjoin%
\pgfsetlinewidth{1.505625pt}%
\definecolor{currentstroke}{rgb}{1.000000,0.498039,0.054902}%
\pgfsetstrokecolor{currentstroke}%
\pgfsetdash{}{0pt}%
\pgfpathmoveto{\pgfqpoint{4.845944in}{6.391702in}}%
\pgfpathlineto{\pgfqpoint{4.918116in}{6.391702in}}%
\pgfpathlineto{\pgfqpoint{4.990288in}{6.391702in}}%
\pgfpathlineto{\pgfqpoint{5.062460in}{6.391702in}}%
\pgfpathlineto{\pgfqpoint{5.134632in}{6.391702in}}%
\pgfpathlineto{\pgfqpoint{5.206804in}{6.391702in}}%
\pgfpathlineto{\pgfqpoint{5.278976in}{6.391701in}}%
\pgfpathlineto{\pgfqpoint{5.351148in}{6.391709in}}%
\pgfpathlineto{\pgfqpoint{5.423320in}{6.391664in}}%
\pgfpathlineto{\pgfqpoint{5.495492in}{6.391897in}}%
\pgfpathlineto{\pgfqpoint{5.567664in}{6.390815in}}%
\pgfpathlineto{\pgfqpoint{5.639836in}{6.395237in}}%
\pgfpathlineto{\pgfqpoint{5.712008in}{6.379496in}}%
\pgfpathlineto{\pgfqpoint{5.784180in}{6.426905in}}%
\pgfpathlineto{\pgfqpoint{5.856352in}{6.306579in}}%
\pgfpathlineto{\pgfqpoint{5.928524in}{6.540301in}}%
\pgfpathlineto{\pgfqpoint{6.000696in}{6.203090in}}%
\pgfpathlineto{\pgfqpoint{6.072869in}{6.440008in}}%
\pgfpathlineto{\pgfqpoint{6.145041in}{6.631387in}}%
\pgfpathlineto{\pgfqpoint{6.217213in}{5.939763in}}%
\pgfpathlineto{\pgfqpoint{6.289385in}{6.297683in}}%
\pgfpathlineto{\pgfqpoint{6.361557in}{7.031557in}}%
\pgfpathlineto{\pgfqpoint{6.433729in}{5.976410in}}%
\pgfpathlineto{\pgfqpoint{6.505901in}{4.805832in}}%
\pgfpathlineto{\pgfqpoint{6.578073in}{4.674126in}}%
\pgfpathlineto{\pgfqpoint{6.650245in}{4.673611in}}%
\pgfpathlineto{\pgfqpoint{6.722417in}{4.673611in}}%
\pgfpathlineto{\pgfqpoint{6.794589in}{4.673611in}}%
\pgfpathlineto{\pgfqpoint{6.866761in}{4.673611in}}%
\pgfpathlineto{\pgfqpoint{6.938933in}{4.673611in}}%
\pgfpathlineto{\pgfqpoint{7.011105in}{4.673611in}}%
\pgfpathlineto{\pgfqpoint{7.083277in}{4.673611in}}%
\pgfpathlineto{\pgfqpoint{7.155449in}{4.673611in}}%
\pgfpathlineto{\pgfqpoint{7.227621in}{4.673611in}}%
\pgfpathlineto{\pgfqpoint{7.299793in}{4.673611in}}%
\pgfpathlineto{\pgfqpoint{7.371965in}{4.673611in}}%
\pgfpathlineto{\pgfqpoint{7.444137in}{4.673611in}}%
\pgfpathlineto{\pgfqpoint{7.516309in}{4.673611in}}%
\pgfpathlineto{\pgfqpoint{7.588481in}{4.673611in}}%
\pgfpathlineto{\pgfqpoint{7.660653in}{4.673611in}}%
\pgfusepath{stroke}%
\end{pgfscope}%
\begin{pgfscope}%
\pgfpathrectangle{\pgfqpoint{4.705208in}{4.530556in}}{\pgfqpoint{3.096181in}{3.147222in}}%
\pgfusepath{clip}%
\pgfsetrectcap%
\pgfsetroundjoin%
\pgfsetlinewidth{1.505625pt}%
\definecolor{currentstroke}{rgb}{0.172549,0.627451,0.172549}%
\pgfsetstrokecolor{currentstroke}%
\pgfsetdash{}{0pt}%
\pgfpathmoveto{\pgfqpoint{4.845944in}{6.391702in}}%
\pgfpathlineto{\pgfqpoint{4.918116in}{6.391702in}}%
\pgfpathlineto{\pgfqpoint{4.990288in}{6.391702in}}%
\pgfpathlineto{\pgfqpoint{5.062460in}{6.391702in}}%
\pgfpathlineto{\pgfqpoint{5.134632in}{6.391702in}}%
\pgfpathlineto{\pgfqpoint{5.206804in}{6.391702in}}%
\pgfpathlineto{\pgfqpoint{5.278976in}{6.391702in}}%
\pgfpathlineto{\pgfqpoint{5.351148in}{6.391702in}}%
\pgfpathlineto{\pgfqpoint{5.423320in}{6.391702in}}%
\pgfpathlineto{\pgfqpoint{5.495492in}{6.391702in}}%
\pgfpathlineto{\pgfqpoint{5.567664in}{6.391702in}}%
\pgfpathlineto{\pgfqpoint{5.639836in}{6.391702in}}%
\pgfpathlineto{\pgfqpoint{5.712008in}{6.391702in}}%
\pgfpathlineto{\pgfqpoint{5.784180in}{6.391702in}}%
\pgfpathlineto{\pgfqpoint{5.856352in}{6.391702in}}%
\pgfpathlineto{\pgfqpoint{5.928524in}{6.391702in}}%
\pgfpathlineto{\pgfqpoint{6.000696in}{6.391702in}}%
\pgfpathlineto{\pgfqpoint{6.072869in}{6.391702in}}%
\pgfpathlineto{\pgfqpoint{6.145041in}{6.391702in}}%
\pgfpathlineto{\pgfqpoint{6.217213in}{6.391702in}}%
\pgfpathlineto{\pgfqpoint{6.289385in}{6.385055in}}%
\pgfpathlineto{\pgfqpoint{6.361557in}{6.318161in}}%
\pgfpathlineto{\pgfqpoint{6.433729in}{5.968095in}}%
\pgfpathlineto{\pgfqpoint{6.505901in}{5.150405in}}%
\pgfpathlineto{\pgfqpoint{6.578073in}{4.700579in}}%
\pgfpathlineto{\pgfqpoint{6.650245in}{4.673645in}}%
\pgfpathlineto{\pgfqpoint{6.722417in}{4.673611in}}%
\pgfpathlineto{\pgfqpoint{6.794589in}{4.673611in}}%
\pgfpathlineto{\pgfqpoint{6.866761in}{4.673611in}}%
\pgfpathlineto{\pgfqpoint{6.938933in}{4.673611in}}%
\pgfpathlineto{\pgfqpoint{7.011105in}{4.673611in}}%
\pgfpathlineto{\pgfqpoint{7.083277in}{4.673611in}}%
\pgfpathlineto{\pgfqpoint{7.155449in}{4.673611in}}%
\pgfpathlineto{\pgfqpoint{7.227621in}{4.673611in}}%
\pgfpathlineto{\pgfqpoint{7.299793in}{4.673611in}}%
\pgfpathlineto{\pgfqpoint{7.371965in}{4.673611in}}%
\pgfpathlineto{\pgfqpoint{7.444137in}{4.673611in}}%
\pgfpathlineto{\pgfqpoint{7.516309in}{4.673611in}}%
\pgfpathlineto{\pgfqpoint{7.588481in}{4.673611in}}%
\pgfpathlineto{\pgfqpoint{7.660653in}{4.673611in}}%
\pgfusepath{stroke}%
\end{pgfscope}%
\begin{pgfscope}%
\pgfpathrectangle{\pgfqpoint{4.705208in}{4.530556in}}{\pgfqpoint{3.096181in}{3.147222in}}%
\pgfusepath{clip}%
\pgfsetrectcap%
\pgfsetroundjoin%
\pgfsetlinewidth{1.505625pt}%
\definecolor{currentstroke}{rgb}{0.839216,0.152941,0.156863}%
\pgfsetstrokecolor{currentstroke}%
\pgfsetdash{}{0pt}%
\pgfpathmoveto{\pgfqpoint{4.845944in}{6.391702in}}%
\pgfpathlineto{\pgfqpoint{4.918116in}{6.391702in}}%
\pgfpathlineto{\pgfqpoint{4.990288in}{6.391702in}}%
\pgfpathlineto{\pgfqpoint{5.062460in}{6.391702in}}%
\pgfpathlineto{\pgfqpoint{5.134632in}{6.391702in}}%
\pgfpathlineto{\pgfqpoint{5.206804in}{6.391702in}}%
\pgfpathlineto{\pgfqpoint{5.278976in}{6.391702in}}%
\pgfpathlineto{\pgfqpoint{5.351148in}{6.391702in}}%
\pgfpathlineto{\pgfqpoint{5.423320in}{6.391702in}}%
\pgfpathlineto{\pgfqpoint{5.495492in}{6.391702in}}%
\pgfpathlineto{\pgfqpoint{5.567664in}{6.391702in}}%
\pgfpathlineto{\pgfqpoint{5.639836in}{6.391702in}}%
\pgfpathlineto{\pgfqpoint{5.712008in}{6.391702in}}%
\pgfpathlineto{\pgfqpoint{5.784180in}{6.391702in}}%
\pgfpathlineto{\pgfqpoint{5.856352in}{6.391702in}}%
\pgfpathlineto{\pgfqpoint{5.928524in}{6.391702in}}%
\pgfpathlineto{\pgfqpoint{6.000696in}{6.391702in}}%
\pgfpathlineto{\pgfqpoint{6.072869in}{6.391702in}}%
\pgfpathlineto{\pgfqpoint{6.145041in}{6.391702in}}%
\pgfpathlineto{\pgfqpoint{6.217213in}{6.391702in}}%
\pgfpathlineto{\pgfqpoint{6.289385in}{4.673611in}}%
\pgfpathlineto{\pgfqpoint{6.361557in}{4.673611in}}%
\pgfpathlineto{\pgfqpoint{6.433729in}{4.673611in}}%
\pgfpathlineto{\pgfqpoint{6.505901in}{4.673611in}}%
\pgfpathlineto{\pgfqpoint{6.578073in}{4.673611in}}%
\pgfpathlineto{\pgfqpoint{6.650245in}{4.673611in}}%
\pgfpathlineto{\pgfqpoint{6.722417in}{4.673611in}}%
\pgfpathlineto{\pgfqpoint{6.794589in}{4.673611in}}%
\pgfpathlineto{\pgfqpoint{6.866761in}{4.673611in}}%
\pgfpathlineto{\pgfqpoint{6.938933in}{4.673611in}}%
\pgfpathlineto{\pgfqpoint{7.011105in}{4.673611in}}%
\pgfpathlineto{\pgfqpoint{7.083277in}{4.673611in}}%
\pgfpathlineto{\pgfqpoint{7.155449in}{4.673611in}}%
\pgfpathlineto{\pgfqpoint{7.227621in}{4.673611in}}%
\pgfpathlineto{\pgfqpoint{7.299793in}{4.673611in}}%
\pgfpathlineto{\pgfqpoint{7.371965in}{4.673611in}}%
\pgfpathlineto{\pgfqpoint{7.444137in}{4.673611in}}%
\pgfpathlineto{\pgfqpoint{7.516309in}{4.673611in}}%
\pgfpathlineto{\pgfqpoint{7.588481in}{4.673611in}}%
\pgfpathlineto{\pgfqpoint{7.660653in}{4.673611in}}%
\pgfusepath{stroke}%
\end{pgfscope}%
\begin{pgfscope}%
\pgfpathrectangle{\pgfqpoint{4.705208in}{4.530556in}}{\pgfqpoint{3.096181in}{3.147222in}}%
\pgfusepath{clip}%
\pgfsetrectcap%
\pgfsetroundjoin%
\pgfsetlinewidth{0.501875pt}%
\definecolor{currentstroke}{rgb}{0.000000,0.000000,0.000000}%
\pgfsetstrokecolor{currentstroke}%
\pgfsetdash{}{0pt}%
\pgfpathmoveto{\pgfqpoint{4.845944in}{6.391702in}}%
\pgfpathlineto{\pgfqpoint{4.918116in}{6.391702in}}%
\pgfpathlineto{\pgfqpoint{4.990288in}{6.391702in}}%
\pgfpathlineto{\pgfqpoint{5.062460in}{6.391702in}}%
\pgfpathlineto{\pgfqpoint{5.134632in}{6.391702in}}%
\pgfpathlineto{\pgfqpoint{5.206804in}{6.391702in}}%
\pgfpathlineto{\pgfqpoint{5.278976in}{6.391702in}}%
\pgfpathlineto{\pgfqpoint{5.351148in}{6.391702in}}%
\pgfpathlineto{\pgfqpoint{5.423320in}{6.391702in}}%
\pgfpathlineto{\pgfqpoint{5.495492in}{6.391702in}}%
\pgfpathlineto{\pgfqpoint{5.567664in}{6.391702in}}%
\pgfpathlineto{\pgfqpoint{5.639836in}{6.391702in}}%
\pgfpathlineto{\pgfqpoint{5.712008in}{6.391702in}}%
\pgfpathlineto{\pgfqpoint{5.784180in}{6.391702in}}%
\pgfpathlineto{\pgfqpoint{5.856352in}{6.391702in}}%
\pgfpathlineto{\pgfqpoint{5.928524in}{6.391702in}}%
\pgfpathlineto{\pgfqpoint{6.000696in}{6.391702in}}%
\pgfpathlineto{\pgfqpoint{6.072869in}{6.391702in}}%
\pgfpathlineto{\pgfqpoint{6.145041in}{6.391702in}}%
\pgfpathlineto{\pgfqpoint{6.217213in}{6.391702in}}%
\pgfpathlineto{\pgfqpoint{6.289385in}{6.391702in}}%
\pgfpathlineto{\pgfqpoint{6.361557in}{6.391702in}}%
\pgfpathlineto{\pgfqpoint{6.433729in}{6.391702in}}%
\pgfpathlineto{\pgfqpoint{6.505901in}{4.673611in}}%
\pgfpathlineto{\pgfqpoint{6.578073in}{4.673611in}}%
\pgfpathlineto{\pgfqpoint{6.650245in}{4.673611in}}%
\pgfpathlineto{\pgfqpoint{6.722417in}{4.673611in}}%
\pgfpathlineto{\pgfqpoint{6.794589in}{4.673611in}}%
\pgfpathlineto{\pgfqpoint{6.866761in}{4.673611in}}%
\pgfpathlineto{\pgfqpoint{6.938933in}{4.673611in}}%
\pgfpathlineto{\pgfqpoint{7.011105in}{4.673611in}}%
\pgfpathlineto{\pgfqpoint{7.083277in}{4.673611in}}%
\pgfpathlineto{\pgfqpoint{7.155449in}{4.673611in}}%
\pgfpathlineto{\pgfqpoint{7.227621in}{4.673611in}}%
\pgfpathlineto{\pgfqpoint{7.299793in}{4.673611in}}%
\pgfpathlineto{\pgfqpoint{7.371965in}{4.673611in}}%
\pgfpathlineto{\pgfqpoint{7.444137in}{4.673611in}}%
\pgfpathlineto{\pgfqpoint{7.516309in}{4.673611in}}%
\pgfpathlineto{\pgfqpoint{7.588481in}{4.673611in}}%
\pgfpathlineto{\pgfqpoint{7.660653in}{4.673611in}}%
\pgfusepath{stroke}%
\end{pgfscope}%
\begin{pgfscope}%
\pgfsetrectcap%
\pgfsetmiterjoin%
\pgfsetlinewidth{0.803000pt}%
\definecolor{currentstroke}{rgb}{0.000000,0.000000,0.000000}%
\pgfsetstrokecolor{currentstroke}%
\pgfsetdash{}{0pt}%
\pgfpathmoveto{\pgfqpoint{4.705208in}{4.530556in}}%
\pgfpathlineto{\pgfqpoint{4.705208in}{7.677778in}}%
\pgfusepath{stroke}%
\end{pgfscope}%
\begin{pgfscope}%
\pgfsetrectcap%
\pgfsetmiterjoin%
\pgfsetlinewidth{0.803000pt}%
\definecolor{currentstroke}{rgb}{0.000000,0.000000,0.000000}%
\pgfsetstrokecolor{currentstroke}%
\pgfsetdash{}{0pt}%
\pgfpathmoveto{\pgfqpoint{7.801389in}{4.530556in}}%
\pgfpathlineto{\pgfqpoint{7.801389in}{7.677778in}}%
\pgfusepath{stroke}%
\end{pgfscope}%
\begin{pgfscope}%
\pgfsetrectcap%
\pgfsetmiterjoin%
\pgfsetlinewidth{0.803000pt}%
\definecolor{currentstroke}{rgb}{0.000000,0.000000,0.000000}%
\pgfsetstrokecolor{currentstroke}%
\pgfsetdash{}{0pt}%
\pgfpathmoveto{\pgfqpoint{4.705208in}{4.530556in}}%
\pgfpathlineto{\pgfqpoint{7.801389in}{4.530556in}}%
\pgfusepath{stroke}%
\end{pgfscope}%
\begin{pgfscope}%
\pgfsetrectcap%
\pgfsetmiterjoin%
\pgfsetlinewidth{0.803000pt}%
\definecolor{currentstroke}{rgb}{0.000000,0.000000,0.000000}%
\pgfsetstrokecolor{currentstroke}%
\pgfsetdash{}{0pt}%
\pgfpathmoveto{\pgfqpoint{4.705208in}{7.677778in}}%
\pgfpathlineto{\pgfqpoint{7.801389in}{7.677778in}}%
\pgfusepath{stroke}%
\end{pgfscope}%
\begin{pgfscope}%
\definecolor{textcolor}{rgb}{0.000000,0.000000,0.000000}%
\pgfsetstrokecolor{textcolor}%
\pgfsetfillcolor{textcolor}%
\pgftext[x=6.253299in,y=7.761111in,,base]{\color{textcolor}\sffamily\fontsize{12.000000}{14.400000}\selectfont \(\displaystyle  t = 0.6 \)}%
\end{pgfscope}%
\begin{pgfscope}%
\pgfsetbuttcap%
\pgfsetmiterjoin%
\definecolor{currentfill}{rgb}{1.000000,1.000000,1.000000}%
\pgfsetfillcolor{currentfill}%
\pgfsetlinewidth{0.000000pt}%
\definecolor{currentstroke}{rgb}{0.000000,0.000000,0.000000}%
\pgfsetstrokecolor{currentstroke}%
\pgfsetstrokeopacity{0.000000}%
\pgfsetdash{}{0pt}%
\pgfpathmoveto{\pgfqpoint{0.780208in}{0.580556in}}%
\pgfpathlineto{\pgfqpoint{3.876389in}{0.580556in}}%
\pgfpathlineto{\pgfqpoint{3.876389in}{3.727778in}}%
\pgfpathlineto{\pgfqpoint{0.780208in}{3.727778in}}%
\pgfpathclose%
\pgfusepath{fill}%
\end{pgfscope}%
\begin{pgfscope}%
\pgfsetbuttcap%
\pgfsetroundjoin%
\definecolor{currentfill}{rgb}{0.000000,0.000000,0.000000}%
\pgfsetfillcolor{currentfill}%
\pgfsetlinewidth{0.803000pt}%
\definecolor{currentstroke}{rgb}{0.000000,0.000000,0.000000}%
\pgfsetstrokecolor{currentstroke}%
\pgfsetdash{}{0pt}%
\pgfsys@defobject{currentmarker}{\pgfqpoint{0.000000in}{-0.048611in}}{\pgfqpoint{0.000000in}{0.000000in}}{%
\pgfpathmoveto{\pgfqpoint{0.000000in}{0.000000in}}%
\pgfpathlineto{\pgfqpoint{0.000000in}{-0.048611in}}%
\pgfusepath{stroke,fill}%
}%
\begin{pgfscope}%
\pgfsys@transformshift{0.884858in}{0.580556in}%
\pgfsys@useobject{currentmarker}{}%
\end{pgfscope}%
\end{pgfscope}%
\begin{pgfscope}%
\definecolor{textcolor}{rgb}{0.000000,0.000000,0.000000}%
\pgfsetstrokecolor{textcolor}%
\pgfsetfillcolor{textcolor}%
\pgftext[x=0.884858in,y=0.483333in,,top]{\color{textcolor}\sffamily\fontsize{10.000000}{12.000000}\selectfont −2}%
\end{pgfscope}%
\begin{pgfscope}%
\pgfsetbuttcap%
\pgfsetroundjoin%
\definecolor{currentfill}{rgb}{0.000000,0.000000,0.000000}%
\pgfsetfillcolor{currentfill}%
\pgfsetlinewidth{0.803000pt}%
\definecolor{currentstroke}{rgb}{0.000000,0.000000,0.000000}%
\pgfsetstrokecolor{currentstroke}%
\pgfsetdash{}{0pt}%
\pgfsys@defobject{currentmarker}{\pgfqpoint{0.000000in}{-0.048611in}}{\pgfqpoint{0.000000in}{0.000000in}}{%
\pgfpathmoveto{\pgfqpoint{0.000000in}{0.000000in}}%
\pgfpathlineto{\pgfqpoint{0.000000in}{-0.048611in}}%
\pgfusepath{stroke,fill}%
}%
\begin{pgfscope}%
\pgfsys@transformshift{1.606578in}{0.580556in}%
\pgfsys@useobject{currentmarker}{}%
\end{pgfscope}%
\end{pgfscope}%
\begin{pgfscope}%
\definecolor{textcolor}{rgb}{0.000000,0.000000,0.000000}%
\pgfsetstrokecolor{textcolor}%
\pgfsetfillcolor{textcolor}%
\pgftext[x=1.606578in,y=0.483333in,,top]{\color{textcolor}\sffamily\fontsize{10.000000}{12.000000}\selectfont −1}%
\end{pgfscope}%
\begin{pgfscope}%
\pgfsetbuttcap%
\pgfsetroundjoin%
\definecolor{currentfill}{rgb}{0.000000,0.000000,0.000000}%
\pgfsetfillcolor{currentfill}%
\pgfsetlinewidth{0.803000pt}%
\definecolor{currentstroke}{rgb}{0.000000,0.000000,0.000000}%
\pgfsetstrokecolor{currentstroke}%
\pgfsetdash{}{0pt}%
\pgfsys@defobject{currentmarker}{\pgfqpoint{0.000000in}{-0.048611in}}{\pgfqpoint{0.000000in}{0.000000in}}{%
\pgfpathmoveto{\pgfqpoint{0.000000in}{0.000000in}}%
\pgfpathlineto{\pgfqpoint{0.000000in}{-0.048611in}}%
\pgfusepath{stroke,fill}%
}%
\begin{pgfscope}%
\pgfsys@transformshift{2.328299in}{0.580556in}%
\pgfsys@useobject{currentmarker}{}%
\end{pgfscope}%
\end{pgfscope}%
\begin{pgfscope}%
\definecolor{textcolor}{rgb}{0.000000,0.000000,0.000000}%
\pgfsetstrokecolor{textcolor}%
\pgfsetfillcolor{textcolor}%
\pgftext[x=2.328299in,y=0.483333in,,top]{\color{textcolor}\sffamily\fontsize{10.000000}{12.000000}\selectfont 0}%
\end{pgfscope}%
\begin{pgfscope}%
\pgfsetbuttcap%
\pgfsetroundjoin%
\definecolor{currentfill}{rgb}{0.000000,0.000000,0.000000}%
\pgfsetfillcolor{currentfill}%
\pgfsetlinewidth{0.803000pt}%
\definecolor{currentstroke}{rgb}{0.000000,0.000000,0.000000}%
\pgfsetstrokecolor{currentstroke}%
\pgfsetdash{}{0pt}%
\pgfsys@defobject{currentmarker}{\pgfqpoint{0.000000in}{-0.048611in}}{\pgfqpoint{0.000000in}{0.000000in}}{%
\pgfpathmoveto{\pgfqpoint{0.000000in}{0.000000in}}%
\pgfpathlineto{\pgfqpoint{0.000000in}{-0.048611in}}%
\pgfusepath{stroke,fill}%
}%
\begin{pgfscope}%
\pgfsys@transformshift{3.050019in}{0.580556in}%
\pgfsys@useobject{currentmarker}{}%
\end{pgfscope}%
\end{pgfscope}%
\begin{pgfscope}%
\definecolor{textcolor}{rgb}{0.000000,0.000000,0.000000}%
\pgfsetstrokecolor{textcolor}%
\pgfsetfillcolor{textcolor}%
\pgftext[x=3.050019in,y=0.483333in,,top]{\color{textcolor}\sffamily\fontsize{10.000000}{12.000000}\selectfont 1}%
\end{pgfscope}%
\begin{pgfscope}%
\pgfsetbuttcap%
\pgfsetroundjoin%
\definecolor{currentfill}{rgb}{0.000000,0.000000,0.000000}%
\pgfsetfillcolor{currentfill}%
\pgfsetlinewidth{0.803000pt}%
\definecolor{currentstroke}{rgb}{0.000000,0.000000,0.000000}%
\pgfsetstrokecolor{currentstroke}%
\pgfsetdash{}{0pt}%
\pgfsys@defobject{currentmarker}{\pgfqpoint{0.000000in}{-0.048611in}}{\pgfqpoint{0.000000in}{0.000000in}}{%
\pgfpathmoveto{\pgfqpoint{0.000000in}{0.000000in}}%
\pgfpathlineto{\pgfqpoint{0.000000in}{-0.048611in}}%
\pgfusepath{stroke,fill}%
}%
\begin{pgfscope}%
\pgfsys@transformshift{3.771739in}{0.580556in}%
\pgfsys@useobject{currentmarker}{}%
\end{pgfscope}%
\end{pgfscope}%
\begin{pgfscope}%
\definecolor{textcolor}{rgb}{0.000000,0.000000,0.000000}%
\pgfsetstrokecolor{textcolor}%
\pgfsetfillcolor{textcolor}%
\pgftext[x=3.771739in,y=0.483333in,,top]{\color{textcolor}\sffamily\fontsize{10.000000}{12.000000}\selectfont 2}%
\end{pgfscope}%
\begin{pgfscope}%
\definecolor{textcolor}{rgb}{0.000000,0.000000,0.000000}%
\pgfsetstrokecolor{textcolor}%
\pgfsetfillcolor{textcolor}%
\pgftext[x=2.328299in,y=0.293365in,,top]{\color{textcolor}\sffamily\fontsize{10.000000}{12.000000}\selectfont \(\displaystyle x\)}%
\end{pgfscope}%
\begin{pgfscope}%
\pgfsetbuttcap%
\pgfsetroundjoin%
\definecolor{currentfill}{rgb}{0.000000,0.000000,0.000000}%
\pgfsetfillcolor{currentfill}%
\pgfsetlinewidth{0.803000pt}%
\definecolor{currentstroke}{rgb}{0.000000,0.000000,0.000000}%
\pgfsetstrokecolor{currentstroke}%
\pgfsetdash{}{0pt}%
\pgfsys@defobject{currentmarker}{\pgfqpoint{-0.048611in}{0.000000in}}{\pgfqpoint{0.000000in}{0.000000in}}{%
\pgfpathmoveto{\pgfqpoint{0.000000in}{0.000000in}}%
\pgfpathlineto{\pgfqpoint{-0.048611in}{0.000000in}}%
\pgfusepath{stroke,fill}%
}%
\begin{pgfscope}%
\pgfsys@transformshift{0.780208in}{0.723611in}%
\pgfsys@useobject{currentmarker}{}%
\end{pgfscope}%
\end{pgfscope}%
\begin{pgfscope}%
\definecolor{textcolor}{rgb}{0.000000,0.000000,0.000000}%
\pgfsetstrokecolor{textcolor}%
\pgfsetfillcolor{textcolor}%
\pgftext[x=0.373741in,y=0.670850in,left,base]{\color{textcolor}\sffamily\fontsize{10.000000}{12.000000}\selectfont 0.00}%
\end{pgfscope}%
\begin{pgfscope}%
\pgfsetbuttcap%
\pgfsetroundjoin%
\definecolor{currentfill}{rgb}{0.000000,0.000000,0.000000}%
\pgfsetfillcolor{currentfill}%
\pgfsetlinewidth{0.803000pt}%
\definecolor{currentstroke}{rgb}{0.000000,0.000000,0.000000}%
\pgfsetstrokecolor{currentstroke}%
\pgfsetdash{}{0pt}%
\pgfsys@defobject{currentmarker}{\pgfqpoint{-0.048611in}{0.000000in}}{\pgfqpoint{0.000000in}{0.000000in}}{%
\pgfpathmoveto{\pgfqpoint{0.000000in}{0.000000in}}%
\pgfpathlineto{\pgfqpoint{-0.048611in}{0.000000in}}%
\pgfusepath{stroke,fill}%
}%
\begin{pgfscope}%
\pgfsys@transformshift{0.780208in}{1.143596in}%
\pgfsys@useobject{currentmarker}{}%
\end{pgfscope}%
\end{pgfscope}%
\begin{pgfscope}%
\definecolor{textcolor}{rgb}{0.000000,0.000000,0.000000}%
\pgfsetstrokecolor{textcolor}%
\pgfsetfillcolor{textcolor}%
\pgftext[x=0.373741in,y=1.090835in,left,base]{\color{textcolor}\sffamily\fontsize{10.000000}{12.000000}\selectfont 0.25}%
\end{pgfscope}%
\begin{pgfscope}%
\pgfsetbuttcap%
\pgfsetroundjoin%
\definecolor{currentfill}{rgb}{0.000000,0.000000,0.000000}%
\pgfsetfillcolor{currentfill}%
\pgfsetlinewidth{0.803000pt}%
\definecolor{currentstroke}{rgb}{0.000000,0.000000,0.000000}%
\pgfsetstrokecolor{currentstroke}%
\pgfsetdash{}{0pt}%
\pgfsys@defobject{currentmarker}{\pgfqpoint{-0.048611in}{0.000000in}}{\pgfqpoint{0.000000in}{0.000000in}}{%
\pgfpathmoveto{\pgfqpoint{0.000000in}{0.000000in}}%
\pgfpathlineto{\pgfqpoint{-0.048611in}{0.000000in}}%
\pgfusepath{stroke,fill}%
}%
\begin{pgfscope}%
\pgfsys@transformshift{0.780208in}{1.563582in}%
\pgfsys@useobject{currentmarker}{}%
\end{pgfscope}%
\end{pgfscope}%
\begin{pgfscope}%
\definecolor{textcolor}{rgb}{0.000000,0.000000,0.000000}%
\pgfsetstrokecolor{textcolor}%
\pgfsetfillcolor{textcolor}%
\pgftext[x=0.373741in,y=1.510820in,left,base]{\color{textcolor}\sffamily\fontsize{10.000000}{12.000000}\selectfont 0.50}%
\end{pgfscope}%
\begin{pgfscope}%
\pgfsetbuttcap%
\pgfsetroundjoin%
\definecolor{currentfill}{rgb}{0.000000,0.000000,0.000000}%
\pgfsetfillcolor{currentfill}%
\pgfsetlinewidth{0.803000pt}%
\definecolor{currentstroke}{rgb}{0.000000,0.000000,0.000000}%
\pgfsetstrokecolor{currentstroke}%
\pgfsetdash{}{0pt}%
\pgfsys@defobject{currentmarker}{\pgfqpoint{-0.048611in}{0.000000in}}{\pgfqpoint{0.000000in}{0.000000in}}{%
\pgfpathmoveto{\pgfqpoint{0.000000in}{0.000000in}}%
\pgfpathlineto{\pgfqpoint{-0.048611in}{0.000000in}}%
\pgfusepath{stroke,fill}%
}%
\begin{pgfscope}%
\pgfsys@transformshift{0.780208in}{1.983567in}%
\pgfsys@useobject{currentmarker}{}%
\end{pgfscope}%
\end{pgfscope}%
\begin{pgfscope}%
\definecolor{textcolor}{rgb}{0.000000,0.000000,0.000000}%
\pgfsetstrokecolor{textcolor}%
\pgfsetfillcolor{textcolor}%
\pgftext[x=0.373741in,y=1.930805in,left,base]{\color{textcolor}\sffamily\fontsize{10.000000}{12.000000}\selectfont 0.75}%
\end{pgfscope}%
\begin{pgfscope}%
\pgfsetbuttcap%
\pgfsetroundjoin%
\definecolor{currentfill}{rgb}{0.000000,0.000000,0.000000}%
\pgfsetfillcolor{currentfill}%
\pgfsetlinewidth{0.803000pt}%
\definecolor{currentstroke}{rgb}{0.000000,0.000000,0.000000}%
\pgfsetstrokecolor{currentstroke}%
\pgfsetdash{}{0pt}%
\pgfsys@defobject{currentmarker}{\pgfqpoint{-0.048611in}{0.000000in}}{\pgfqpoint{0.000000in}{0.000000in}}{%
\pgfpathmoveto{\pgfqpoint{0.000000in}{0.000000in}}%
\pgfpathlineto{\pgfqpoint{-0.048611in}{0.000000in}}%
\pgfusepath{stroke,fill}%
}%
\begin{pgfscope}%
\pgfsys@transformshift{0.780208in}{2.403552in}%
\pgfsys@useobject{currentmarker}{}%
\end{pgfscope}%
\end{pgfscope}%
\begin{pgfscope}%
\definecolor{textcolor}{rgb}{0.000000,0.000000,0.000000}%
\pgfsetstrokecolor{textcolor}%
\pgfsetfillcolor{textcolor}%
\pgftext[x=0.373741in,y=2.350791in,left,base]{\color{textcolor}\sffamily\fontsize{10.000000}{12.000000}\selectfont 1.00}%
\end{pgfscope}%
\begin{pgfscope}%
\pgfsetbuttcap%
\pgfsetroundjoin%
\definecolor{currentfill}{rgb}{0.000000,0.000000,0.000000}%
\pgfsetfillcolor{currentfill}%
\pgfsetlinewidth{0.803000pt}%
\definecolor{currentstroke}{rgb}{0.000000,0.000000,0.000000}%
\pgfsetstrokecolor{currentstroke}%
\pgfsetdash{}{0pt}%
\pgfsys@defobject{currentmarker}{\pgfqpoint{-0.048611in}{0.000000in}}{\pgfqpoint{0.000000in}{0.000000in}}{%
\pgfpathmoveto{\pgfqpoint{0.000000in}{0.000000in}}%
\pgfpathlineto{\pgfqpoint{-0.048611in}{0.000000in}}%
\pgfusepath{stroke,fill}%
}%
\begin{pgfscope}%
\pgfsys@transformshift{0.780208in}{2.823538in}%
\pgfsys@useobject{currentmarker}{}%
\end{pgfscope}%
\end{pgfscope}%
\begin{pgfscope}%
\definecolor{textcolor}{rgb}{0.000000,0.000000,0.000000}%
\pgfsetstrokecolor{textcolor}%
\pgfsetfillcolor{textcolor}%
\pgftext[x=0.373741in,y=2.770776in,left,base]{\color{textcolor}\sffamily\fontsize{10.000000}{12.000000}\selectfont 1.25}%
\end{pgfscope}%
\begin{pgfscope}%
\pgfsetbuttcap%
\pgfsetroundjoin%
\definecolor{currentfill}{rgb}{0.000000,0.000000,0.000000}%
\pgfsetfillcolor{currentfill}%
\pgfsetlinewidth{0.803000pt}%
\definecolor{currentstroke}{rgb}{0.000000,0.000000,0.000000}%
\pgfsetstrokecolor{currentstroke}%
\pgfsetdash{}{0pt}%
\pgfsys@defobject{currentmarker}{\pgfqpoint{-0.048611in}{0.000000in}}{\pgfqpoint{0.000000in}{0.000000in}}{%
\pgfpathmoveto{\pgfqpoint{0.000000in}{0.000000in}}%
\pgfpathlineto{\pgfqpoint{-0.048611in}{0.000000in}}%
\pgfusepath{stroke,fill}%
}%
\begin{pgfscope}%
\pgfsys@transformshift{0.780208in}{3.243523in}%
\pgfsys@useobject{currentmarker}{}%
\end{pgfscope}%
\end{pgfscope}%
\begin{pgfscope}%
\definecolor{textcolor}{rgb}{0.000000,0.000000,0.000000}%
\pgfsetstrokecolor{textcolor}%
\pgfsetfillcolor{textcolor}%
\pgftext[x=0.373741in,y=3.190761in,left,base]{\color{textcolor}\sffamily\fontsize{10.000000}{12.000000}\selectfont 1.50}%
\end{pgfscope}%
\begin{pgfscope}%
\pgfsetbuttcap%
\pgfsetroundjoin%
\definecolor{currentfill}{rgb}{0.000000,0.000000,0.000000}%
\pgfsetfillcolor{currentfill}%
\pgfsetlinewidth{0.803000pt}%
\definecolor{currentstroke}{rgb}{0.000000,0.000000,0.000000}%
\pgfsetstrokecolor{currentstroke}%
\pgfsetdash{}{0pt}%
\pgfsys@defobject{currentmarker}{\pgfqpoint{-0.048611in}{0.000000in}}{\pgfqpoint{0.000000in}{0.000000in}}{%
\pgfpathmoveto{\pgfqpoint{0.000000in}{0.000000in}}%
\pgfpathlineto{\pgfqpoint{-0.048611in}{0.000000in}}%
\pgfusepath{stroke,fill}%
}%
\begin{pgfscope}%
\pgfsys@transformshift{0.780208in}{3.663508in}%
\pgfsys@useobject{currentmarker}{}%
\end{pgfscope}%
\end{pgfscope}%
\begin{pgfscope}%
\definecolor{textcolor}{rgb}{0.000000,0.000000,0.000000}%
\pgfsetstrokecolor{textcolor}%
\pgfsetfillcolor{textcolor}%
\pgftext[x=0.373741in,y=3.610747in,left,base]{\color{textcolor}\sffamily\fontsize{10.000000}{12.000000}\selectfont 1.75}%
\end{pgfscope}%
\begin{pgfscope}%
\definecolor{textcolor}{rgb}{0.000000,0.000000,0.000000}%
\pgfsetstrokecolor{textcolor}%
\pgfsetfillcolor{textcolor}%
\pgftext[x=0.318186in,y=2.154167in,,bottom,rotate=90.000000]{\color{textcolor}\sffamily\fontsize{10.000000}{12.000000}\selectfont \(\displaystyle U\)}%
\end{pgfscope}%
\begin{pgfscope}%
\pgfpathrectangle{\pgfqpoint{0.780208in}{0.580556in}}{\pgfqpoint{3.096181in}{3.147222in}}%
\pgfusepath{clip}%
\pgfsetrectcap%
\pgfsetroundjoin%
\pgfsetlinewidth{1.505625pt}%
\definecolor{currentstroke}{rgb}{0.121569,0.466667,0.705882}%
\pgfsetstrokecolor{currentstroke}%
\pgfsetdash{}{0pt}%
\pgfpathmoveto{\pgfqpoint{0.920944in}{2.403552in}}%
\pgfpathlineto{\pgfqpoint{0.993116in}{2.403552in}}%
\pgfpathlineto{\pgfqpoint{1.065288in}{2.403552in}}%
\pgfpathlineto{\pgfqpoint{1.137460in}{2.403553in}}%
\pgfpathlineto{\pgfqpoint{1.209632in}{2.403549in}}%
\pgfpathlineto{\pgfqpoint{1.281804in}{2.403569in}}%
\pgfpathlineto{\pgfqpoint{1.353976in}{2.403473in}}%
\pgfpathlineto{\pgfqpoint{1.426148in}{2.403890in}}%
\pgfpathlineto{\pgfqpoint{1.498320in}{2.402253in}}%
\pgfpathlineto{\pgfqpoint{1.570492in}{2.408016in}}%
\pgfpathlineto{\pgfqpoint{1.642664in}{2.390049in}}%
\pgfpathlineto{\pgfqpoint{1.714836in}{2.438815in}}%
\pgfpathlineto{\pgfqpoint{1.787008in}{2.326353in}}%
\pgfpathlineto{\pgfqpoint{1.859180in}{2.537993in}}%
\pgfpathlineto{\pgfqpoint{1.931352in}{2.238941in}}%
\pgfpathlineto{\pgfqpoint{2.003524in}{2.483984in}}%
\pgfpathlineto{\pgfqpoint{2.075696in}{2.562188in}}%
\pgfpathlineto{\pgfqpoint{2.147869in}{2.005417in}}%
\pgfpathlineto{\pgfqpoint{2.220041in}{2.621540in}}%
\pgfpathlineto{\pgfqpoint{2.292213in}{2.814897in}}%
\pgfpathlineto{\pgfqpoint{2.364385in}{1.618088in}}%
\pgfpathlineto{\pgfqpoint{2.436557in}{2.182112in}}%
\pgfpathlineto{\pgfqpoint{2.508729in}{3.584722in}}%
\pgfpathlineto{\pgfqpoint{2.580901in}{1.809503in}}%
\pgfpathlineto{\pgfqpoint{2.653073in}{0.755296in}}%
\pgfpathlineto{\pgfqpoint{2.725245in}{0.723620in}}%
\pgfpathlineto{\pgfqpoint{2.797417in}{0.723611in}}%
\pgfpathlineto{\pgfqpoint{2.869589in}{0.723611in}}%
\pgfpathlineto{\pgfqpoint{2.941761in}{0.723611in}}%
\pgfpathlineto{\pgfqpoint{3.013933in}{0.723611in}}%
\pgfpathlineto{\pgfqpoint{3.086105in}{0.723611in}}%
\pgfpathlineto{\pgfqpoint{3.158277in}{0.723611in}}%
\pgfpathlineto{\pgfqpoint{3.230449in}{0.723611in}}%
\pgfpathlineto{\pgfqpoint{3.302621in}{0.723611in}}%
\pgfpathlineto{\pgfqpoint{3.374793in}{0.723611in}}%
\pgfpathlineto{\pgfqpoint{3.446965in}{0.723611in}}%
\pgfpathlineto{\pgfqpoint{3.519137in}{0.723611in}}%
\pgfpathlineto{\pgfqpoint{3.591309in}{0.723611in}}%
\pgfpathlineto{\pgfqpoint{3.663481in}{0.723611in}}%
\pgfpathlineto{\pgfqpoint{3.735653in}{0.723611in}}%
\pgfusepath{stroke}%
\end{pgfscope}%
\begin{pgfscope}%
\pgfpathrectangle{\pgfqpoint{0.780208in}{0.580556in}}{\pgfqpoint{3.096181in}{3.147222in}}%
\pgfusepath{clip}%
\pgfsetrectcap%
\pgfsetroundjoin%
\pgfsetlinewidth{1.505625pt}%
\definecolor{currentstroke}{rgb}{1.000000,0.498039,0.054902}%
\pgfsetstrokecolor{currentstroke}%
\pgfsetdash{}{0pt}%
\pgfpathmoveto{\pgfqpoint{0.920944in}{2.403552in}}%
\pgfpathlineto{\pgfqpoint{0.993116in}{2.403552in}}%
\pgfpathlineto{\pgfqpoint{1.065288in}{2.403552in}}%
\pgfpathlineto{\pgfqpoint{1.137460in}{2.403553in}}%
\pgfpathlineto{\pgfqpoint{1.209632in}{2.403550in}}%
\pgfpathlineto{\pgfqpoint{1.281804in}{2.403563in}}%
\pgfpathlineto{\pgfqpoint{1.353976in}{2.403501in}}%
\pgfpathlineto{\pgfqpoint{1.426148in}{2.403770in}}%
\pgfpathlineto{\pgfqpoint{1.498320in}{2.402715in}}%
\pgfpathlineto{\pgfqpoint{1.570492in}{2.406426in}}%
\pgfpathlineto{\pgfqpoint{1.642664in}{2.394843in}}%
\pgfpathlineto{\pgfqpoint{1.714836in}{2.426198in}}%
\pgfpathlineto{\pgfqpoint{1.787008in}{2.353208in}}%
\pgfpathlineto{\pgfqpoint{1.859180in}{2.489841in}}%
\pgfpathlineto{\pgfqpoint{1.931352in}{2.291205in}}%
\pgfpathlineto{\pgfqpoint{2.003524in}{2.463705in}}%
\pgfpathlineto{\pgfqpoint{2.075696in}{2.489381in}}%
\pgfpathlineto{\pgfqpoint{2.147869in}{2.147400in}}%
\pgfpathlineto{\pgfqpoint{2.220041in}{2.543857in}}%
\pgfpathlineto{\pgfqpoint{2.292213in}{2.639817in}}%
\pgfpathlineto{\pgfqpoint{2.364385in}{1.868912in}}%
\pgfpathlineto{\pgfqpoint{2.436557in}{2.264038in}}%
\pgfpathlineto{\pgfqpoint{2.508729in}{3.062379in}}%
\pgfpathlineto{\pgfqpoint{2.580901in}{2.067051in}}%
\pgfpathlineto{\pgfqpoint{2.653073in}{0.868661in}}%
\pgfpathlineto{\pgfqpoint{2.725245in}{0.724246in}}%
\pgfpathlineto{\pgfqpoint{2.797417in}{0.723611in}}%
\pgfpathlineto{\pgfqpoint{2.869589in}{0.723611in}}%
\pgfpathlineto{\pgfqpoint{2.941761in}{0.723611in}}%
\pgfpathlineto{\pgfqpoint{3.013933in}{0.723611in}}%
\pgfpathlineto{\pgfqpoint{3.086105in}{0.723611in}}%
\pgfpathlineto{\pgfqpoint{3.158277in}{0.723611in}}%
\pgfpathlineto{\pgfqpoint{3.230449in}{0.723611in}}%
\pgfpathlineto{\pgfqpoint{3.302621in}{0.723611in}}%
\pgfpathlineto{\pgfqpoint{3.374793in}{0.723611in}}%
\pgfpathlineto{\pgfqpoint{3.446965in}{0.723611in}}%
\pgfpathlineto{\pgfqpoint{3.519137in}{0.723611in}}%
\pgfpathlineto{\pgfqpoint{3.591309in}{0.723611in}}%
\pgfpathlineto{\pgfqpoint{3.663481in}{0.723611in}}%
\pgfpathlineto{\pgfqpoint{3.735653in}{0.723611in}}%
\pgfusepath{stroke}%
\end{pgfscope}%
\begin{pgfscope}%
\pgfpathrectangle{\pgfqpoint{0.780208in}{0.580556in}}{\pgfqpoint{3.096181in}{3.147222in}}%
\pgfusepath{clip}%
\pgfsetrectcap%
\pgfsetroundjoin%
\pgfsetlinewidth{1.505625pt}%
\definecolor{currentstroke}{rgb}{0.172549,0.627451,0.172549}%
\pgfsetstrokecolor{currentstroke}%
\pgfsetdash{}{0pt}%
\pgfpathmoveto{\pgfqpoint{0.920944in}{2.403552in}}%
\pgfpathlineto{\pgfqpoint{0.993116in}{2.403552in}}%
\pgfpathlineto{\pgfqpoint{1.065288in}{2.403552in}}%
\pgfpathlineto{\pgfqpoint{1.137460in}{2.403552in}}%
\pgfpathlineto{\pgfqpoint{1.209632in}{2.403552in}}%
\pgfpathlineto{\pgfqpoint{1.281804in}{2.403552in}}%
\pgfpathlineto{\pgfqpoint{1.353976in}{2.403552in}}%
\pgfpathlineto{\pgfqpoint{1.426148in}{2.403552in}}%
\pgfpathlineto{\pgfqpoint{1.498320in}{2.403552in}}%
\pgfpathlineto{\pgfqpoint{1.570492in}{2.403552in}}%
\pgfpathlineto{\pgfqpoint{1.642664in}{2.403552in}}%
\pgfpathlineto{\pgfqpoint{1.714836in}{2.403552in}}%
\pgfpathlineto{\pgfqpoint{1.787008in}{2.403552in}}%
\pgfpathlineto{\pgfqpoint{1.859180in}{2.403552in}}%
\pgfpathlineto{\pgfqpoint{1.931352in}{2.403552in}}%
\pgfpathlineto{\pgfqpoint{2.003524in}{2.403552in}}%
\pgfpathlineto{\pgfqpoint{2.075696in}{2.403552in}}%
\pgfpathlineto{\pgfqpoint{2.147869in}{2.403552in}}%
\pgfpathlineto{\pgfqpoint{2.220041in}{2.403552in}}%
\pgfpathlineto{\pgfqpoint{2.292213in}{2.403552in}}%
\pgfpathlineto{\pgfqpoint{2.364385in}{2.402761in}}%
\pgfpathlineto{\pgfqpoint{2.436557in}{2.392842in}}%
\pgfpathlineto{\pgfqpoint{2.508729in}{2.323047in}}%
\pgfpathlineto{\pgfqpoint{2.580901in}{1.987906in}}%
\pgfpathlineto{\pgfqpoint{2.653073in}{1.202287in}}%
\pgfpathlineto{\pgfqpoint{2.725245in}{0.752547in}}%
\pgfpathlineto{\pgfqpoint{2.797417in}{0.723652in}}%
\pgfpathlineto{\pgfqpoint{2.869589in}{0.723611in}}%
\pgfpathlineto{\pgfqpoint{2.941761in}{0.723611in}}%
\pgfpathlineto{\pgfqpoint{3.013933in}{0.723611in}}%
\pgfpathlineto{\pgfqpoint{3.086105in}{0.723611in}}%
\pgfpathlineto{\pgfqpoint{3.158277in}{0.723611in}}%
\pgfpathlineto{\pgfqpoint{3.230449in}{0.723611in}}%
\pgfpathlineto{\pgfqpoint{3.302621in}{0.723611in}}%
\pgfpathlineto{\pgfqpoint{3.374793in}{0.723611in}}%
\pgfpathlineto{\pgfqpoint{3.446965in}{0.723611in}}%
\pgfpathlineto{\pgfqpoint{3.519137in}{0.723611in}}%
\pgfpathlineto{\pgfqpoint{3.591309in}{0.723611in}}%
\pgfpathlineto{\pgfqpoint{3.663481in}{0.723611in}}%
\pgfpathlineto{\pgfqpoint{3.735653in}{0.723611in}}%
\pgfusepath{stroke}%
\end{pgfscope}%
\begin{pgfscope}%
\pgfpathrectangle{\pgfqpoint{0.780208in}{0.580556in}}{\pgfqpoint{3.096181in}{3.147222in}}%
\pgfusepath{clip}%
\pgfsetrectcap%
\pgfsetroundjoin%
\pgfsetlinewidth{1.505625pt}%
\definecolor{currentstroke}{rgb}{0.839216,0.152941,0.156863}%
\pgfsetstrokecolor{currentstroke}%
\pgfsetdash{}{0pt}%
\pgfpathmoveto{\pgfqpoint{0.920944in}{2.403552in}}%
\pgfpathlineto{\pgfqpoint{0.993116in}{2.403552in}}%
\pgfpathlineto{\pgfqpoint{1.065288in}{2.403552in}}%
\pgfpathlineto{\pgfqpoint{1.137460in}{2.403552in}}%
\pgfpathlineto{\pgfqpoint{1.209632in}{2.403552in}}%
\pgfpathlineto{\pgfqpoint{1.281804in}{2.403552in}}%
\pgfpathlineto{\pgfqpoint{1.353976in}{2.403552in}}%
\pgfpathlineto{\pgfqpoint{1.426148in}{2.403552in}}%
\pgfpathlineto{\pgfqpoint{1.498320in}{2.403552in}}%
\pgfpathlineto{\pgfqpoint{1.570492in}{2.403552in}}%
\pgfpathlineto{\pgfqpoint{1.642664in}{2.403552in}}%
\pgfpathlineto{\pgfqpoint{1.714836in}{2.403552in}}%
\pgfpathlineto{\pgfqpoint{1.787008in}{2.403552in}}%
\pgfpathlineto{\pgfqpoint{1.859180in}{2.403552in}}%
\pgfpathlineto{\pgfqpoint{1.931352in}{2.403552in}}%
\pgfpathlineto{\pgfqpoint{2.003524in}{2.403552in}}%
\pgfpathlineto{\pgfqpoint{2.075696in}{2.403552in}}%
\pgfpathlineto{\pgfqpoint{2.147869in}{2.403552in}}%
\pgfpathlineto{\pgfqpoint{2.220041in}{2.403552in}}%
\pgfpathlineto{\pgfqpoint{2.292213in}{2.403552in}}%
\pgfpathlineto{\pgfqpoint{2.364385in}{0.723611in}}%
\pgfpathlineto{\pgfqpoint{2.436557in}{0.723611in}}%
\pgfpathlineto{\pgfqpoint{2.508729in}{0.723611in}}%
\pgfpathlineto{\pgfqpoint{2.580901in}{0.723611in}}%
\pgfpathlineto{\pgfqpoint{2.653073in}{0.723611in}}%
\pgfpathlineto{\pgfqpoint{2.725245in}{0.723611in}}%
\pgfpathlineto{\pgfqpoint{2.797417in}{0.723611in}}%
\pgfpathlineto{\pgfqpoint{2.869589in}{0.723611in}}%
\pgfpathlineto{\pgfqpoint{2.941761in}{0.723611in}}%
\pgfpathlineto{\pgfqpoint{3.013933in}{0.723611in}}%
\pgfpathlineto{\pgfqpoint{3.086105in}{0.723611in}}%
\pgfpathlineto{\pgfqpoint{3.158277in}{0.723611in}}%
\pgfpathlineto{\pgfqpoint{3.230449in}{0.723611in}}%
\pgfpathlineto{\pgfqpoint{3.302621in}{0.723611in}}%
\pgfpathlineto{\pgfqpoint{3.374793in}{0.723611in}}%
\pgfpathlineto{\pgfqpoint{3.446965in}{0.723611in}}%
\pgfpathlineto{\pgfqpoint{3.519137in}{0.723611in}}%
\pgfpathlineto{\pgfqpoint{3.591309in}{0.723611in}}%
\pgfpathlineto{\pgfqpoint{3.663481in}{0.723611in}}%
\pgfpathlineto{\pgfqpoint{3.735653in}{0.723611in}}%
\pgfusepath{stroke}%
\end{pgfscope}%
\begin{pgfscope}%
\pgfpathrectangle{\pgfqpoint{0.780208in}{0.580556in}}{\pgfqpoint{3.096181in}{3.147222in}}%
\pgfusepath{clip}%
\pgfsetrectcap%
\pgfsetroundjoin%
\pgfsetlinewidth{0.501875pt}%
\definecolor{currentstroke}{rgb}{0.000000,0.000000,0.000000}%
\pgfsetstrokecolor{currentstroke}%
\pgfsetdash{}{0pt}%
\pgfpathmoveto{\pgfqpoint{0.920944in}{2.403552in}}%
\pgfpathlineto{\pgfqpoint{0.993116in}{2.403552in}}%
\pgfpathlineto{\pgfqpoint{1.065288in}{2.403552in}}%
\pgfpathlineto{\pgfqpoint{1.137460in}{2.403552in}}%
\pgfpathlineto{\pgfqpoint{1.209632in}{2.403552in}}%
\pgfpathlineto{\pgfqpoint{1.281804in}{2.403552in}}%
\pgfpathlineto{\pgfqpoint{1.353976in}{2.403552in}}%
\pgfpathlineto{\pgfqpoint{1.426148in}{2.403552in}}%
\pgfpathlineto{\pgfqpoint{1.498320in}{2.403552in}}%
\pgfpathlineto{\pgfqpoint{1.570492in}{2.403552in}}%
\pgfpathlineto{\pgfqpoint{1.642664in}{2.403552in}}%
\pgfpathlineto{\pgfqpoint{1.714836in}{2.403552in}}%
\pgfpathlineto{\pgfqpoint{1.787008in}{2.403552in}}%
\pgfpathlineto{\pgfqpoint{1.859180in}{2.403552in}}%
\pgfpathlineto{\pgfqpoint{1.931352in}{2.403552in}}%
\pgfpathlineto{\pgfqpoint{2.003524in}{2.403552in}}%
\pgfpathlineto{\pgfqpoint{2.075696in}{2.403552in}}%
\pgfpathlineto{\pgfqpoint{2.147869in}{2.403552in}}%
\pgfpathlineto{\pgfqpoint{2.220041in}{2.403552in}}%
\pgfpathlineto{\pgfqpoint{2.292213in}{2.403552in}}%
\pgfpathlineto{\pgfqpoint{2.364385in}{2.403552in}}%
\pgfpathlineto{\pgfqpoint{2.436557in}{2.403552in}}%
\pgfpathlineto{\pgfqpoint{2.508729in}{2.403552in}}%
\pgfpathlineto{\pgfqpoint{2.580901in}{2.403552in}}%
\pgfpathlineto{\pgfqpoint{2.653073in}{0.723611in}}%
\pgfpathlineto{\pgfqpoint{2.725245in}{0.723611in}}%
\pgfpathlineto{\pgfqpoint{2.797417in}{0.723611in}}%
\pgfpathlineto{\pgfqpoint{2.869589in}{0.723611in}}%
\pgfpathlineto{\pgfqpoint{2.941761in}{0.723611in}}%
\pgfpathlineto{\pgfqpoint{3.013933in}{0.723611in}}%
\pgfpathlineto{\pgfqpoint{3.086105in}{0.723611in}}%
\pgfpathlineto{\pgfqpoint{3.158277in}{0.723611in}}%
\pgfpathlineto{\pgfqpoint{3.230449in}{0.723611in}}%
\pgfpathlineto{\pgfqpoint{3.302621in}{0.723611in}}%
\pgfpathlineto{\pgfqpoint{3.374793in}{0.723611in}}%
\pgfpathlineto{\pgfqpoint{3.446965in}{0.723611in}}%
\pgfpathlineto{\pgfqpoint{3.519137in}{0.723611in}}%
\pgfpathlineto{\pgfqpoint{3.591309in}{0.723611in}}%
\pgfpathlineto{\pgfqpoint{3.663481in}{0.723611in}}%
\pgfpathlineto{\pgfqpoint{3.735653in}{0.723611in}}%
\pgfusepath{stroke}%
\end{pgfscope}%
\begin{pgfscope}%
\pgfsetrectcap%
\pgfsetmiterjoin%
\pgfsetlinewidth{0.803000pt}%
\definecolor{currentstroke}{rgb}{0.000000,0.000000,0.000000}%
\pgfsetstrokecolor{currentstroke}%
\pgfsetdash{}{0pt}%
\pgfpathmoveto{\pgfqpoint{0.780208in}{0.580556in}}%
\pgfpathlineto{\pgfqpoint{0.780208in}{3.727778in}}%
\pgfusepath{stroke}%
\end{pgfscope}%
\begin{pgfscope}%
\pgfsetrectcap%
\pgfsetmiterjoin%
\pgfsetlinewidth{0.803000pt}%
\definecolor{currentstroke}{rgb}{0.000000,0.000000,0.000000}%
\pgfsetstrokecolor{currentstroke}%
\pgfsetdash{}{0pt}%
\pgfpathmoveto{\pgfqpoint{3.876389in}{0.580556in}}%
\pgfpathlineto{\pgfqpoint{3.876389in}{3.727778in}}%
\pgfusepath{stroke}%
\end{pgfscope}%
\begin{pgfscope}%
\pgfsetrectcap%
\pgfsetmiterjoin%
\pgfsetlinewidth{0.803000pt}%
\definecolor{currentstroke}{rgb}{0.000000,0.000000,0.000000}%
\pgfsetstrokecolor{currentstroke}%
\pgfsetdash{}{0pt}%
\pgfpathmoveto{\pgfqpoint{0.780208in}{0.580556in}}%
\pgfpathlineto{\pgfqpoint{3.876389in}{0.580556in}}%
\pgfusepath{stroke}%
\end{pgfscope}%
\begin{pgfscope}%
\pgfsetrectcap%
\pgfsetmiterjoin%
\pgfsetlinewidth{0.803000pt}%
\definecolor{currentstroke}{rgb}{0.000000,0.000000,0.000000}%
\pgfsetstrokecolor{currentstroke}%
\pgfsetdash{}{0pt}%
\pgfpathmoveto{\pgfqpoint{0.780208in}{3.727778in}}%
\pgfpathlineto{\pgfqpoint{3.876389in}{3.727778in}}%
\pgfusepath{stroke}%
\end{pgfscope}%
\begin{pgfscope}%
\definecolor{textcolor}{rgb}{0.000000,0.000000,0.000000}%
\pgfsetstrokecolor{textcolor}%
\pgfsetfillcolor{textcolor}%
\pgftext[x=2.328299in,y=3.811111in,,base]{\color{textcolor}\sffamily\fontsize{12.000000}{14.400000}\selectfont \(\displaystyle  t = 0.8 \)}%
\end{pgfscope}%
\begin{pgfscope}%
\pgfsetbuttcap%
\pgfsetmiterjoin%
\definecolor{currentfill}{rgb}{1.000000,1.000000,1.000000}%
\pgfsetfillcolor{currentfill}%
\pgfsetlinewidth{0.000000pt}%
\definecolor{currentstroke}{rgb}{0.000000,0.000000,0.000000}%
\pgfsetstrokecolor{currentstroke}%
\pgfsetstrokeopacity{0.000000}%
\pgfsetdash{}{0pt}%
\pgfpathmoveto{\pgfqpoint{4.705208in}{0.580556in}}%
\pgfpathlineto{\pgfqpoint{7.801389in}{0.580556in}}%
\pgfpathlineto{\pgfqpoint{7.801389in}{3.727778in}}%
\pgfpathlineto{\pgfqpoint{4.705208in}{3.727778in}}%
\pgfpathclose%
\pgfusepath{fill}%
\end{pgfscope}%
\begin{pgfscope}%
\pgfsetbuttcap%
\pgfsetroundjoin%
\definecolor{currentfill}{rgb}{0.000000,0.000000,0.000000}%
\pgfsetfillcolor{currentfill}%
\pgfsetlinewidth{0.803000pt}%
\definecolor{currentstroke}{rgb}{0.000000,0.000000,0.000000}%
\pgfsetstrokecolor{currentstroke}%
\pgfsetdash{}{0pt}%
\pgfsys@defobject{currentmarker}{\pgfqpoint{0.000000in}{-0.048611in}}{\pgfqpoint{0.000000in}{0.000000in}}{%
\pgfpathmoveto{\pgfqpoint{0.000000in}{0.000000in}}%
\pgfpathlineto{\pgfqpoint{0.000000in}{-0.048611in}}%
\pgfusepath{stroke,fill}%
}%
\begin{pgfscope}%
\pgfsys@transformshift{4.809858in}{0.580556in}%
\pgfsys@useobject{currentmarker}{}%
\end{pgfscope}%
\end{pgfscope}%
\begin{pgfscope}%
\definecolor{textcolor}{rgb}{0.000000,0.000000,0.000000}%
\pgfsetstrokecolor{textcolor}%
\pgfsetfillcolor{textcolor}%
\pgftext[x=4.809858in,y=0.483333in,,top]{\color{textcolor}\sffamily\fontsize{10.000000}{12.000000}\selectfont −2}%
\end{pgfscope}%
\begin{pgfscope}%
\pgfsetbuttcap%
\pgfsetroundjoin%
\definecolor{currentfill}{rgb}{0.000000,0.000000,0.000000}%
\pgfsetfillcolor{currentfill}%
\pgfsetlinewidth{0.803000pt}%
\definecolor{currentstroke}{rgb}{0.000000,0.000000,0.000000}%
\pgfsetstrokecolor{currentstroke}%
\pgfsetdash{}{0pt}%
\pgfsys@defobject{currentmarker}{\pgfqpoint{0.000000in}{-0.048611in}}{\pgfqpoint{0.000000in}{0.000000in}}{%
\pgfpathmoveto{\pgfqpoint{0.000000in}{0.000000in}}%
\pgfpathlineto{\pgfqpoint{0.000000in}{-0.048611in}}%
\pgfusepath{stroke,fill}%
}%
\begin{pgfscope}%
\pgfsys@transformshift{5.531578in}{0.580556in}%
\pgfsys@useobject{currentmarker}{}%
\end{pgfscope}%
\end{pgfscope}%
\begin{pgfscope}%
\definecolor{textcolor}{rgb}{0.000000,0.000000,0.000000}%
\pgfsetstrokecolor{textcolor}%
\pgfsetfillcolor{textcolor}%
\pgftext[x=5.531578in,y=0.483333in,,top]{\color{textcolor}\sffamily\fontsize{10.000000}{12.000000}\selectfont −1}%
\end{pgfscope}%
\begin{pgfscope}%
\pgfsetbuttcap%
\pgfsetroundjoin%
\definecolor{currentfill}{rgb}{0.000000,0.000000,0.000000}%
\pgfsetfillcolor{currentfill}%
\pgfsetlinewidth{0.803000pt}%
\definecolor{currentstroke}{rgb}{0.000000,0.000000,0.000000}%
\pgfsetstrokecolor{currentstroke}%
\pgfsetdash{}{0pt}%
\pgfsys@defobject{currentmarker}{\pgfqpoint{0.000000in}{-0.048611in}}{\pgfqpoint{0.000000in}{0.000000in}}{%
\pgfpathmoveto{\pgfqpoint{0.000000in}{0.000000in}}%
\pgfpathlineto{\pgfqpoint{0.000000in}{-0.048611in}}%
\pgfusepath{stroke,fill}%
}%
\begin{pgfscope}%
\pgfsys@transformshift{6.253299in}{0.580556in}%
\pgfsys@useobject{currentmarker}{}%
\end{pgfscope}%
\end{pgfscope}%
\begin{pgfscope}%
\definecolor{textcolor}{rgb}{0.000000,0.000000,0.000000}%
\pgfsetstrokecolor{textcolor}%
\pgfsetfillcolor{textcolor}%
\pgftext[x=6.253299in,y=0.483333in,,top]{\color{textcolor}\sffamily\fontsize{10.000000}{12.000000}\selectfont 0}%
\end{pgfscope}%
\begin{pgfscope}%
\pgfsetbuttcap%
\pgfsetroundjoin%
\definecolor{currentfill}{rgb}{0.000000,0.000000,0.000000}%
\pgfsetfillcolor{currentfill}%
\pgfsetlinewidth{0.803000pt}%
\definecolor{currentstroke}{rgb}{0.000000,0.000000,0.000000}%
\pgfsetstrokecolor{currentstroke}%
\pgfsetdash{}{0pt}%
\pgfsys@defobject{currentmarker}{\pgfqpoint{0.000000in}{-0.048611in}}{\pgfqpoint{0.000000in}{0.000000in}}{%
\pgfpathmoveto{\pgfqpoint{0.000000in}{0.000000in}}%
\pgfpathlineto{\pgfqpoint{0.000000in}{-0.048611in}}%
\pgfusepath{stroke,fill}%
}%
\begin{pgfscope}%
\pgfsys@transformshift{6.975019in}{0.580556in}%
\pgfsys@useobject{currentmarker}{}%
\end{pgfscope}%
\end{pgfscope}%
\begin{pgfscope}%
\definecolor{textcolor}{rgb}{0.000000,0.000000,0.000000}%
\pgfsetstrokecolor{textcolor}%
\pgfsetfillcolor{textcolor}%
\pgftext[x=6.975019in,y=0.483333in,,top]{\color{textcolor}\sffamily\fontsize{10.000000}{12.000000}\selectfont 1}%
\end{pgfscope}%
\begin{pgfscope}%
\pgfsetbuttcap%
\pgfsetroundjoin%
\definecolor{currentfill}{rgb}{0.000000,0.000000,0.000000}%
\pgfsetfillcolor{currentfill}%
\pgfsetlinewidth{0.803000pt}%
\definecolor{currentstroke}{rgb}{0.000000,0.000000,0.000000}%
\pgfsetstrokecolor{currentstroke}%
\pgfsetdash{}{0pt}%
\pgfsys@defobject{currentmarker}{\pgfqpoint{0.000000in}{-0.048611in}}{\pgfqpoint{0.000000in}{0.000000in}}{%
\pgfpathmoveto{\pgfqpoint{0.000000in}{0.000000in}}%
\pgfpathlineto{\pgfqpoint{0.000000in}{-0.048611in}}%
\pgfusepath{stroke,fill}%
}%
\begin{pgfscope}%
\pgfsys@transformshift{7.696739in}{0.580556in}%
\pgfsys@useobject{currentmarker}{}%
\end{pgfscope}%
\end{pgfscope}%
\begin{pgfscope}%
\definecolor{textcolor}{rgb}{0.000000,0.000000,0.000000}%
\pgfsetstrokecolor{textcolor}%
\pgfsetfillcolor{textcolor}%
\pgftext[x=7.696739in,y=0.483333in,,top]{\color{textcolor}\sffamily\fontsize{10.000000}{12.000000}\selectfont 2}%
\end{pgfscope}%
\begin{pgfscope}%
\definecolor{textcolor}{rgb}{0.000000,0.000000,0.000000}%
\pgfsetstrokecolor{textcolor}%
\pgfsetfillcolor{textcolor}%
\pgftext[x=6.253299in,y=0.293365in,,top]{\color{textcolor}\sffamily\fontsize{10.000000}{12.000000}\selectfont \(\displaystyle x\)}%
\end{pgfscope}%
\begin{pgfscope}%
\pgfsetbuttcap%
\pgfsetroundjoin%
\definecolor{currentfill}{rgb}{0.000000,0.000000,0.000000}%
\pgfsetfillcolor{currentfill}%
\pgfsetlinewidth{0.803000pt}%
\definecolor{currentstroke}{rgb}{0.000000,0.000000,0.000000}%
\pgfsetstrokecolor{currentstroke}%
\pgfsetdash{}{0pt}%
\pgfsys@defobject{currentmarker}{\pgfqpoint{-0.048611in}{0.000000in}}{\pgfqpoint{0.000000in}{0.000000in}}{%
\pgfpathmoveto{\pgfqpoint{0.000000in}{0.000000in}}%
\pgfpathlineto{\pgfqpoint{-0.048611in}{0.000000in}}%
\pgfusepath{stroke,fill}%
}%
\begin{pgfscope}%
\pgfsys@transformshift{4.705208in}{0.723611in}%
\pgfsys@useobject{currentmarker}{}%
\end{pgfscope}%
\end{pgfscope}%
\begin{pgfscope}%
\definecolor{textcolor}{rgb}{0.000000,0.000000,0.000000}%
\pgfsetstrokecolor{textcolor}%
\pgfsetfillcolor{textcolor}%
\pgftext[x=4.298741in,y=0.670850in,left,base]{\color{textcolor}\sffamily\fontsize{10.000000}{12.000000}\selectfont 0.00}%
\end{pgfscope}%
\begin{pgfscope}%
\pgfsetbuttcap%
\pgfsetroundjoin%
\definecolor{currentfill}{rgb}{0.000000,0.000000,0.000000}%
\pgfsetfillcolor{currentfill}%
\pgfsetlinewidth{0.803000pt}%
\definecolor{currentstroke}{rgb}{0.000000,0.000000,0.000000}%
\pgfsetstrokecolor{currentstroke}%
\pgfsetdash{}{0pt}%
\pgfsys@defobject{currentmarker}{\pgfqpoint{-0.048611in}{0.000000in}}{\pgfqpoint{0.000000in}{0.000000in}}{%
\pgfpathmoveto{\pgfqpoint{0.000000in}{0.000000in}}%
\pgfpathlineto{\pgfqpoint{-0.048611in}{0.000000in}}%
\pgfusepath{stroke,fill}%
}%
\begin{pgfscope}%
\pgfsys@transformshift{4.705208in}{1.140646in}%
\pgfsys@useobject{currentmarker}{}%
\end{pgfscope}%
\end{pgfscope}%
\begin{pgfscope}%
\definecolor{textcolor}{rgb}{0.000000,0.000000,0.000000}%
\pgfsetstrokecolor{textcolor}%
\pgfsetfillcolor{textcolor}%
\pgftext[x=4.298741in,y=1.087885in,left,base]{\color{textcolor}\sffamily\fontsize{10.000000}{12.000000}\selectfont 0.25}%
\end{pgfscope}%
\begin{pgfscope}%
\pgfsetbuttcap%
\pgfsetroundjoin%
\definecolor{currentfill}{rgb}{0.000000,0.000000,0.000000}%
\pgfsetfillcolor{currentfill}%
\pgfsetlinewidth{0.803000pt}%
\definecolor{currentstroke}{rgb}{0.000000,0.000000,0.000000}%
\pgfsetstrokecolor{currentstroke}%
\pgfsetdash{}{0pt}%
\pgfsys@defobject{currentmarker}{\pgfqpoint{-0.048611in}{0.000000in}}{\pgfqpoint{0.000000in}{0.000000in}}{%
\pgfpathmoveto{\pgfqpoint{0.000000in}{0.000000in}}%
\pgfpathlineto{\pgfqpoint{-0.048611in}{0.000000in}}%
\pgfusepath{stroke,fill}%
}%
\begin{pgfscope}%
\pgfsys@transformshift{4.705208in}{1.557681in}%
\pgfsys@useobject{currentmarker}{}%
\end{pgfscope}%
\end{pgfscope}%
\begin{pgfscope}%
\definecolor{textcolor}{rgb}{0.000000,0.000000,0.000000}%
\pgfsetstrokecolor{textcolor}%
\pgfsetfillcolor{textcolor}%
\pgftext[x=4.298741in,y=1.504920in,left,base]{\color{textcolor}\sffamily\fontsize{10.000000}{12.000000}\selectfont 0.50}%
\end{pgfscope}%
\begin{pgfscope}%
\pgfsetbuttcap%
\pgfsetroundjoin%
\definecolor{currentfill}{rgb}{0.000000,0.000000,0.000000}%
\pgfsetfillcolor{currentfill}%
\pgfsetlinewidth{0.803000pt}%
\definecolor{currentstroke}{rgb}{0.000000,0.000000,0.000000}%
\pgfsetstrokecolor{currentstroke}%
\pgfsetdash{}{0pt}%
\pgfsys@defobject{currentmarker}{\pgfqpoint{-0.048611in}{0.000000in}}{\pgfqpoint{0.000000in}{0.000000in}}{%
\pgfpathmoveto{\pgfqpoint{0.000000in}{0.000000in}}%
\pgfpathlineto{\pgfqpoint{-0.048611in}{0.000000in}}%
\pgfusepath{stroke,fill}%
}%
\begin{pgfscope}%
\pgfsys@transformshift{4.705208in}{1.974716in}%
\pgfsys@useobject{currentmarker}{}%
\end{pgfscope}%
\end{pgfscope}%
\begin{pgfscope}%
\definecolor{textcolor}{rgb}{0.000000,0.000000,0.000000}%
\pgfsetstrokecolor{textcolor}%
\pgfsetfillcolor{textcolor}%
\pgftext[x=4.298741in,y=1.921955in,left,base]{\color{textcolor}\sffamily\fontsize{10.000000}{12.000000}\selectfont 0.75}%
\end{pgfscope}%
\begin{pgfscope}%
\pgfsetbuttcap%
\pgfsetroundjoin%
\definecolor{currentfill}{rgb}{0.000000,0.000000,0.000000}%
\pgfsetfillcolor{currentfill}%
\pgfsetlinewidth{0.803000pt}%
\definecolor{currentstroke}{rgb}{0.000000,0.000000,0.000000}%
\pgfsetstrokecolor{currentstroke}%
\pgfsetdash{}{0pt}%
\pgfsys@defobject{currentmarker}{\pgfqpoint{-0.048611in}{0.000000in}}{\pgfqpoint{0.000000in}{0.000000in}}{%
\pgfpathmoveto{\pgfqpoint{0.000000in}{0.000000in}}%
\pgfpathlineto{\pgfqpoint{-0.048611in}{0.000000in}}%
\pgfusepath{stroke,fill}%
}%
\begin{pgfscope}%
\pgfsys@transformshift{4.705208in}{2.391751in}%
\pgfsys@useobject{currentmarker}{}%
\end{pgfscope}%
\end{pgfscope}%
\begin{pgfscope}%
\definecolor{textcolor}{rgb}{0.000000,0.000000,0.000000}%
\pgfsetstrokecolor{textcolor}%
\pgfsetfillcolor{textcolor}%
\pgftext[x=4.298741in,y=2.338990in,left,base]{\color{textcolor}\sffamily\fontsize{10.000000}{12.000000}\selectfont 1.00}%
\end{pgfscope}%
\begin{pgfscope}%
\pgfsetbuttcap%
\pgfsetroundjoin%
\definecolor{currentfill}{rgb}{0.000000,0.000000,0.000000}%
\pgfsetfillcolor{currentfill}%
\pgfsetlinewidth{0.803000pt}%
\definecolor{currentstroke}{rgb}{0.000000,0.000000,0.000000}%
\pgfsetstrokecolor{currentstroke}%
\pgfsetdash{}{0pt}%
\pgfsys@defobject{currentmarker}{\pgfqpoint{-0.048611in}{0.000000in}}{\pgfqpoint{0.000000in}{0.000000in}}{%
\pgfpathmoveto{\pgfqpoint{0.000000in}{0.000000in}}%
\pgfpathlineto{\pgfqpoint{-0.048611in}{0.000000in}}%
\pgfusepath{stroke,fill}%
}%
\begin{pgfscope}%
\pgfsys@transformshift{4.705208in}{2.808786in}%
\pgfsys@useobject{currentmarker}{}%
\end{pgfscope}%
\end{pgfscope}%
\begin{pgfscope}%
\definecolor{textcolor}{rgb}{0.000000,0.000000,0.000000}%
\pgfsetstrokecolor{textcolor}%
\pgfsetfillcolor{textcolor}%
\pgftext[x=4.298741in,y=2.756024in,left,base]{\color{textcolor}\sffamily\fontsize{10.000000}{12.000000}\selectfont 1.25}%
\end{pgfscope}%
\begin{pgfscope}%
\pgfsetbuttcap%
\pgfsetroundjoin%
\definecolor{currentfill}{rgb}{0.000000,0.000000,0.000000}%
\pgfsetfillcolor{currentfill}%
\pgfsetlinewidth{0.803000pt}%
\definecolor{currentstroke}{rgb}{0.000000,0.000000,0.000000}%
\pgfsetstrokecolor{currentstroke}%
\pgfsetdash{}{0pt}%
\pgfsys@defobject{currentmarker}{\pgfqpoint{-0.048611in}{0.000000in}}{\pgfqpoint{0.000000in}{0.000000in}}{%
\pgfpathmoveto{\pgfqpoint{0.000000in}{0.000000in}}%
\pgfpathlineto{\pgfqpoint{-0.048611in}{0.000000in}}%
\pgfusepath{stroke,fill}%
}%
\begin{pgfscope}%
\pgfsys@transformshift{4.705208in}{3.225821in}%
\pgfsys@useobject{currentmarker}{}%
\end{pgfscope}%
\end{pgfscope}%
\begin{pgfscope}%
\definecolor{textcolor}{rgb}{0.000000,0.000000,0.000000}%
\pgfsetstrokecolor{textcolor}%
\pgfsetfillcolor{textcolor}%
\pgftext[x=4.298741in,y=3.173059in,left,base]{\color{textcolor}\sffamily\fontsize{10.000000}{12.000000}\selectfont 1.50}%
\end{pgfscope}%
\begin{pgfscope}%
\pgfsetbuttcap%
\pgfsetroundjoin%
\definecolor{currentfill}{rgb}{0.000000,0.000000,0.000000}%
\pgfsetfillcolor{currentfill}%
\pgfsetlinewidth{0.803000pt}%
\definecolor{currentstroke}{rgb}{0.000000,0.000000,0.000000}%
\pgfsetstrokecolor{currentstroke}%
\pgfsetdash{}{0pt}%
\pgfsys@defobject{currentmarker}{\pgfqpoint{-0.048611in}{0.000000in}}{\pgfqpoint{0.000000in}{0.000000in}}{%
\pgfpathmoveto{\pgfqpoint{0.000000in}{0.000000in}}%
\pgfpathlineto{\pgfqpoint{-0.048611in}{0.000000in}}%
\pgfusepath{stroke,fill}%
}%
\begin{pgfscope}%
\pgfsys@transformshift{4.705208in}{3.642856in}%
\pgfsys@useobject{currentmarker}{}%
\end{pgfscope}%
\end{pgfscope}%
\begin{pgfscope}%
\definecolor{textcolor}{rgb}{0.000000,0.000000,0.000000}%
\pgfsetstrokecolor{textcolor}%
\pgfsetfillcolor{textcolor}%
\pgftext[x=4.298741in,y=3.590094in,left,base]{\color{textcolor}\sffamily\fontsize{10.000000}{12.000000}\selectfont 1.75}%
\end{pgfscope}%
\begin{pgfscope}%
\definecolor{textcolor}{rgb}{0.000000,0.000000,0.000000}%
\pgfsetstrokecolor{textcolor}%
\pgfsetfillcolor{textcolor}%
\pgftext[x=4.243186in,y=2.154167in,,bottom,rotate=90.000000]{\color{textcolor}\sffamily\fontsize{10.000000}{12.000000}\selectfont \(\displaystyle U\)}%
\end{pgfscope}%
\begin{pgfscope}%
\pgfpathrectangle{\pgfqpoint{4.705208in}{0.580556in}}{\pgfqpoint{3.096181in}{3.147222in}}%
\pgfusepath{clip}%
\pgfsetrectcap%
\pgfsetroundjoin%
\pgfsetlinewidth{1.505625pt}%
\definecolor{currentstroke}{rgb}{0.121569,0.466667,0.705882}%
\pgfsetstrokecolor{currentstroke}%
\pgfsetdash{}{0pt}%
\pgfpathmoveto{\pgfqpoint{4.845944in}{2.391751in}}%
\pgfpathlineto{\pgfqpoint{4.918116in}{2.391752in}}%
\pgfpathlineto{\pgfqpoint{4.990288in}{2.391746in}}%
\pgfpathlineto{\pgfqpoint{5.062460in}{2.391772in}}%
\pgfpathlineto{\pgfqpoint{5.134632in}{2.391666in}}%
\pgfpathlineto{\pgfqpoint{5.206804in}{2.392072in}}%
\pgfpathlineto{\pgfqpoint{5.278976in}{2.390649in}}%
\pgfpathlineto{\pgfqpoint{5.351148in}{2.395163in}}%
\pgfpathlineto{\pgfqpoint{5.423320in}{2.382323in}}%
\pgfpathlineto{\pgfqpoint{5.495492in}{2.414609in}}%
\pgfpathlineto{\pgfqpoint{5.567664in}{2.344344in}}%
\pgfpathlineto{\pgfqpoint{5.639836in}{2.472286in}}%
\pgfpathlineto{\pgfqpoint{5.712008in}{2.289961in}}%
\pgfpathlineto{\pgfqpoint{5.784180in}{2.459646in}}%
\pgfpathlineto{\pgfqpoint{5.856352in}{2.446129in}}%
\pgfpathlineto{\pgfqpoint{5.928524in}{2.175013in}}%
\pgfpathlineto{\pgfqpoint{6.000696in}{2.615340in}}%
\pgfpathlineto{\pgfqpoint{6.072869in}{2.463875in}}%
\pgfpathlineto{\pgfqpoint{6.145041in}{1.913949in}}%
\pgfpathlineto{\pgfqpoint{6.217213in}{2.707948in}}%
\pgfpathlineto{\pgfqpoint{6.289385in}{2.820677in}}%
\pgfpathlineto{\pgfqpoint{6.361557in}{1.536324in}}%
\pgfpathlineto{\pgfqpoint{6.433729in}{2.076456in}}%
\pgfpathlineto{\pgfqpoint{6.505901in}{3.584722in}}%
\pgfpathlineto{\pgfqpoint{6.578073in}{1.913599in}}%
\pgfpathlineto{\pgfqpoint{6.650245in}{0.763600in}}%
\pgfpathlineto{\pgfqpoint{6.722417in}{0.723626in}}%
\pgfpathlineto{\pgfqpoint{6.794589in}{0.723611in}}%
\pgfpathlineto{\pgfqpoint{6.866761in}{0.723611in}}%
\pgfpathlineto{\pgfqpoint{6.938933in}{0.723611in}}%
\pgfpathlineto{\pgfqpoint{7.011105in}{0.723611in}}%
\pgfpathlineto{\pgfqpoint{7.083277in}{0.723611in}}%
\pgfpathlineto{\pgfqpoint{7.155449in}{0.723611in}}%
\pgfpathlineto{\pgfqpoint{7.227621in}{0.723611in}}%
\pgfpathlineto{\pgfqpoint{7.299793in}{0.723611in}}%
\pgfpathlineto{\pgfqpoint{7.371965in}{0.723611in}}%
\pgfpathlineto{\pgfqpoint{7.444137in}{0.723611in}}%
\pgfpathlineto{\pgfqpoint{7.516309in}{0.723611in}}%
\pgfpathlineto{\pgfqpoint{7.588481in}{0.723611in}}%
\pgfpathlineto{\pgfqpoint{7.660653in}{0.723611in}}%
\pgfusepath{stroke}%
\end{pgfscope}%
\begin{pgfscope}%
\pgfpathrectangle{\pgfqpoint{4.705208in}{0.580556in}}{\pgfqpoint{3.096181in}{3.147222in}}%
\pgfusepath{clip}%
\pgfsetrectcap%
\pgfsetroundjoin%
\pgfsetlinewidth{1.505625pt}%
\definecolor{currentstroke}{rgb}{1.000000,0.498039,0.054902}%
\pgfsetstrokecolor{currentstroke}%
\pgfsetdash{}{0pt}%
\pgfpathmoveto{\pgfqpoint{4.845944in}{2.391751in}}%
\pgfpathlineto{\pgfqpoint{4.918116in}{2.391752in}}%
\pgfpathlineto{\pgfqpoint{4.990288in}{2.391748in}}%
\pgfpathlineto{\pgfqpoint{5.062460in}{2.391764in}}%
\pgfpathlineto{\pgfqpoint{5.134632in}{2.391696in}}%
\pgfpathlineto{\pgfqpoint{5.206804in}{2.391958in}}%
\pgfpathlineto{\pgfqpoint{5.278976in}{2.391041in}}%
\pgfpathlineto{\pgfqpoint{5.351148in}{2.393949in}}%
\pgfpathlineto{\pgfqpoint{5.423320in}{2.385665in}}%
\pgfpathlineto{\pgfqpoint{5.495492in}{2.406479in}}%
\pgfpathlineto{\pgfqpoint{5.567664in}{2.360847in}}%
\pgfpathlineto{\pgfqpoint{5.639836in}{2.443907in}}%
\pgfpathlineto{\pgfqpoint{5.712008in}{2.322874in}}%
\pgfpathlineto{\pgfqpoint{5.784180in}{2.440327in}}%
\pgfpathlineto{\pgfqpoint{5.856352in}{2.417451in}}%
\pgfpathlineto{\pgfqpoint{5.928524in}{2.257908in}}%
\pgfpathlineto{\pgfqpoint{6.000696in}{2.530509in}}%
\pgfpathlineto{\pgfqpoint{6.072869in}{2.426324in}}%
\pgfpathlineto{\pgfqpoint{6.145041in}{2.086489in}}%
\pgfpathlineto{\pgfqpoint{6.217213in}{2.582224in}}%
\pgfpathlineto{\pgfqpoint{6.289385in}{2.646921in}}%
\pgfpathlineto{\pgfqpoint{6.361557in}{1.803229in}}%
\pgfpathlineto{\pgfqpoint{6.433729in}{2.182435in}}%
\pgfpathlineto{\pgfqpoint{6.505901in}{3.054184in}}%
\pgfpathlineto{\pgfqpoint{6.578073in}{2.140791in}}%
\pgfpathlineto{\pgfqpoint{6.650245in}{0.892273in}}%
\pgfpathlineto{\pgfqpoint{6.722417in}{0.724502in}}%
\pgfpathlineto{\pgfqpoint{6.794589in}{0.723611in}}%
\pgfpathlineto{\pgfqpoint{6.866761in}{0.723611in}}%
\pgfpathlineto{\pgfqpoint{6.938933in}{0.723611in}}%
\pgfpathlineto{\pgfqpoint{7.011105in}{0.723611in}}%
\pgfpathlineto{\pgfqpoint{7.083277in}{0.723611in}}%
\pgfpathlineto{\pgfqpoint{7.155449in}{0.723611in}}%
\pgfpathlineto{\pgfqpoint{7.227621in}{0.723611in}}%
\pgfpathlineto{\pgfqpoint{7.299793in}{0.723611in}}%
\pgfpathlineto{\pgfqpoint{7.371965in}{0.723611in}}%
\pgfpathlineto{\pgfqpoint{7.444137in}{0.723611in}}%
\pgfpathlineto{\pgfqpoint{7.516309in}{0.723611in}}%
\pgfpathlineto{\pgfqpoint{7.588481in}{0.723611in}}%
\pgfpathlineto{\pgfqpoint{7.660653in}{0.723611in}}%
\pgfusepath{stroke}%
\end{pgfscope}%
\begin{pgfscope}%
\pgfpathrectangle{\pgfqpoint{4.705208in}{0.580556in}}{\pgfqpoint{3.096181in}{3.147222in}}%
\pgfusepath{clip}%
\pgfsetrectcap%
\pgfsetroundjoin%
\pgfsetlinewidth{1.505625pt}%
\definecolor{currentstroke}{rgb}{0.172549,0.627451,0.172549}%
\pgfsetstrokecolor{currentstroke}%
\pgfsetdash{}{0pt}%
\pgfpathmoveto{\pgfqpoint{4.845944in}{2.391751in}}%
\pgfpathlineto{\pgfqpoint{4.918116in}{2.391751in}}%
\pgfpathlineto{\pgfqpoint{4.990288in}{2.391751in}}%
\pgfpathlineto{\pgfqpoint{5.062460in}{2.391751in}}%
\pgfpathlineto{\pgfqpoint{5.134632in}{2.391751in}}%
\pgfpathlineto{\pgfqpoint{5.206804in}{2.391751in}}%
\pgfpathlineto{\pgfqpoint{5.278976in}{2.391751in}}%
\pgfpathlineto{\pgfqpoint{5.351148in}{2.391751in}}%
\pgfpathlineto{\pgfqpoint{5.423320in}{2.391751in}}%
\pgfpathlineto{\pgfqpoint{5.495492in}{2.391751in}}%
\pgfpathlineto{\pgfqpoint{5.567664in}{2.391751in}}%
\pgfpathlineto{\pgfqpoint{5.639836in}{2.391751in}}%
\pgfpathlineto{\pgfqpoint{5.712008in}{2.391751in}}%
\pgfpathlineto{\pgfqpoint{5.784180in}{2.391751in}}%
\pgfpathlineto{\pgfqpoint{5.856352in}{2.391751in}}%
\pgfpathlineto{\pgfqpoint{5.928524in}{2.391751in}}%
\pgfpathlineto{\pgfqpoint{6.000696in}{2.391751in}}%
\pgfpathlineto{\pgfqpoint{6.072869in}{2.391751in}}%
\pgfpathlineto{\pgfqpoint{6.145041in}{2.391751in}}%
\pgfpathlineto{\pgfqpoint{6.217213in}{2.391751in}}%
\pgfpathlineto{\pgfqpoint{6.289385in}{2.391655in}}%
\pgfpathlineto{\pgfqpoint{6.361557in}{2.390241in}}%
\pgfpathlineto{\pgfqpoint{6.433729in}{2.378652in}}%
\pgfpathlineto{\pgfqpoint{6.505901in}{2.307596in}}%
\pgfpathlineto{\pgfqpoint{6.578073in}{1.978693in}}%
\pgfpathlineto{\pgfqpoint{6.650245in}{1.205386in}}%
\pgfpathlineto{\pgfqpoint{6.722417in}{0.753707in}}%
\pgfpathlineto{\pgfqpoint{6.794589in}{0.723657in}}%
\pgfpathlineto{\pgfqpoint{6.866761in}{0.723611in}}%
\pgfpathlineto{\pgfqpoint{6.938933in}{0.723611in}}%
\pgfpathlineto{\pgfqpoint{7.011105in}{0.723611in}}%
\pgfpathlineto{\pgfqpoint{7.083277in}{0.723611in}}%
\pgfpathlineto{\pgfqpoint{7.155449in}{0.723611in}}%
\pgfpathlineto{\pgfqpoint{7.227621in}{0.723611in}}%
\pgfpathlineto{\pgfqpoint{7.299793in}{0.723611in}}%
\pgfpathlineto{\pgfqpoint{7.371965in}{0.723611in}}%
\pgfpathlineto{\pgfqpoint{7.444137in}{0.723611in}}%
\pgfpathlineto{\pgfqpoint{7.516309in}{0.723611in}}%
\pgfpathlineto{\pgfqpoint{7.588481in}{0.723611in}}%
\pgfpathlineto{\pgfqpoint{7.660653in}{0.723611in}}%
\pgfusepath{stroke}%
\end{pgfscope}%
\begin{pgfscope}%
\pgfpathrectangle{\pgfqpoint{4.705208in}{0.580556in}}{\pgfqpoint{3.096181in}{3.147222in}}%
\pgfusepath{clip}%
\pgfsetrectcap%
\pgfsetroundjoin%
\pgfsetlinewidth{1.505625pt}%
\definecolor{currentstroke}{rgb}{0.839216,0.152941,0.156863}%
\pgfsetstrokecolor{currentstroke}%
\pgfsetdash{}{0pt}%
\pgfpathmoveto{\pgfqpoint{4.845944in}{2.391751in}}%
\pgfpathlineto{\pgfqpoint{4.918116in}{2.391751in}}%
\pgfpathlineto{\pgfqpoint{4.990288in}{2.391751in}}%
\pgfpathlineto{\pgfqpoint{5.062460in}{2.391751in}}%
\pgfpathlineto{\pgfqpoint{5.134632in}{2.391751in}}%
\pgfpathlineto{\pgfqpoint{5.206804in}{2.391751in}}%
\pgfpathlineto{\pgfqpoint{5.278976in}{2.391751in}}%
\pgfpathlineto{\pgfqpoint{5.351148in}{2.391751in}}%
\pgfpathlineto{\pgfqpoint{5.423320in}{2.391751in}}%
\pgfpathlineto{\pgfqpoint{5.495492in}{2.391751in}}%
\pgfpathlineto{\pgfqpoint{5.567664in}{2.391751in}}%
\pgfpathlineto{\pgfqpoint{5.639836in}{2.391751in}}%
\pgfpathlineto{\pgfqpoint{5.712008in}{2.391751in}}%
\pgfpathlineto{\pgfqpoint{5.784180in}{2.391751in}}%
\pgfpathlineto{\pgfqpoint{5.856352in}{2.391751in}}%
\pgfpathlineto{\pgfqpoint{5.928524in}{2.391751in}}%
\pgfpathlineto{\pgfqpoint{6.000696in}{2.391751in}}%
\pgfpathlineto{\pgfqpoint{6.072869in}{2.391751in}}%
\pgfpathlineto{\pgfqpoint{6.145041in}{2.391751in}}%
\pgfpathlineto{\pgfqpoint{6.217213in}{2.391751in}}%
\pgfpathlineto{\pgfqpoint{6.289385in}{0.723611in}}%
\pgfpathlineto{\pgfqpoint{6.361557in}{0.723611in}}%
\pgfpathlineto{\pgfqpoint{6.433729in}{0.723611in}}%
\pgfpathlineto{\pgfqpoint{6.505901in}{0.723611in}}%
\pgfpathlineto{\pgfqpoint{6.578073in}{0.723611in}}%
\pgfpathlineto{\pgfqpoint{6.650245in}{0.723611in}}%
\pgfpathlineto{\pgfqpoint{6.722417in}{0.723611in}}%
\pgfpathlineto{\pgfqpoint{6.794589in}{0.723611in}}%
\pgfpathlineto{\pgfqpoint{6.866761in}{0.723611in}}%
\pgfpathlineto{\pgfqpoint{6.938933in}{0.723611in}}%
\pgfpathlineto{\pgfqpoint{7.011105in}{0.723611in}}%
\pgfpathlineto{\pgfqpoint{7.083277in}{0.723611in}}%
\pgfpathlineto{\pgfqpoint{7.155449in}{0.723611in}}%
\pgfpathlineto{\pgfqpoint{7.227621in}{0.723611in}}%
\pgfpathlineto{\pgfqpoint{7.299793in}{0.723611in}}%
\pgfpathlineto{\pgfqpoint{7.371965in}{0.723611in}}%
\pgfpathlineto{\pgfqpoint{7.444137in}{0.723611in}}%
\pgfpathlineto{\pgfqpoint{7.516309in}{0.723611in}}%
\pgfpathlineto{\pgfqpoint{7.588481in}{0.723611in}}%
\pgfpathlineto{\pgfqpoint{7.660653in}{0.723611in}}%
\pgfusepath{stroke}%
\end{pgfscope}%
\begin{pgfscope}%
\pgfpathrectangle{\pgfqpoint{4.705208in}{0.580556in}}{\pgfqpoint{3.096181in}{3.147222in}}%
\pgfusepath{clip}%
\pgfsetrectcap%
\pgfsetroundjoin%
\pgfsetlinewidth{0.501875pt}%
\definecolor{currentstroke}{rgb}{0.000000,0.000000,0.000000}%
\pgfsetstrokecolor{currentstroke}%
\pgfsetdash{}{0pt}%
\pgfpathmoveto{\pgfqpoint{4.845944in}{2.391751in}}%
\pgfpathlineto{\pgfqpoint{4.918116in}{2.391751in}}%
\pgfpathlineto{\pgfqpoint{4.990288in}{2.391751in}}%
\pgfpathlineto{\pgfqpoint{5.062460in}{2.391751in}}%
\pgfpathlineto{\pgfqpoint{5.134632in}{2.391751in}}%
\pgfpathlineto{\pgfqpoint{5.206804in}{2.391751in}}%
\pgfpathlineto{\pgfqpoint{5.278976in}{2.391751in}}%
\pgfpathlineto{\pgfqpoint{5.351148in}{2.391751in}}%
\pgfpathlineto{\pgfqpoint{5.423320in}{2.391751in}}%
\pgfpathlineto{\pgfqpoint{5.495492in}{2.391751in}}%
\pgfpathlineto{\pgfqpoint{5.567664in}{2.391751in}}%
\pgfpathlineto{\pgfqpoint{5.639836in}{2.391751in}}%
\pgfpathlineto{\pgfqpoint{5.712008in}{2.391751in}}%
\pgfpathlineto{\pgfqpoint{5.784180in}{2.391751in}}%
\pgfpathlineto{\pgfqpoint{5.856352in}{2.391751in}}%
\pgfpathlineto{\pgfqpoint{5.928524in}{2.391751in}}%
\pgfpathlineto{\pgfqpoint{6.000696in}{2.391751in}}%
\pgfpathlineto{\pgfqpoint{6.072869in}{2.391751in}}%
\pgfpathlineto{\pgfqpoint{6.145041in}{2.391751in}}%
\pgfpathlineto{\pgfqpoint{6.217213in}{2.391751in}}%
\pgfpathlineto{\pgfqpoint{6.289385in}{2.391751in}}%
\pgfpathlineto{\pgfqpoint{6.361557in}{2.391751in}}%
\pgfpathlineto{\pgfqpoint{6.433729in}{2.391751in}}%
\pgfpathlineto{\pgfqpoint{6.505901in}{2.391751in}}%
\pgfpathlineto{\pgfqpoint{6.578073in}{2.391751in}}%
\pgfpathlineto{\pgfqpoint{6.650245in}{0.723611in}}%
\pgfpathlineto{\pgfqpoint{6.722417in}{0.723611in}}%
\pgfpathlineto{\pgfqpoint{6.794589in}{0.723611in}}%
\pgfpathlineto{\pgfqpoint{6.866761in}{0.723611in}}%
\pgfpathlineto{\pgfqpoint{6.938933in}{0.723611in}}%
\pgfpathlineto{\pgfqpoint{7.011105in}{0.723611in}}%
\pgfpathlineto{\pgfqpoint{7.083277in}{0.723611in}}%
\pgfpathlineto{\pgfqpoint{7.155449in}{0.723611in}}%
\pgfpathlineto{\pgfqpoint{7.227621in}{0.723611in}}%
\pgfpathlineto{\pgfqpoint{7.299793in}{0.723611in}}%
\pgfpathlineto{\pgfqpoint{7.371965in}{0.723611in}}%
\pgfpathlineto{\pgfqpoint{7.444137in}{0.723611in}}%
\pgfpathlineto{\pgfqpoint{7.516309in}{0.723611in}}%
\pgfpathlineto{\pgfqpoint{7.588481in}{0.723611in}}%
\pgfpathlineto{\pgfqpoint{7.660653in}{0.723611in}}%
\pgfusepath{stroke}%
\end{pgfscope}%
\begin{pgfscope}%
\pgfsetrectcap%
\pgfsetmiterjoin%
\pgfsetlinewidth{0.803000pt}%
\definecolor{currentstroke}{rgb}{0.000000,0.000000,0.000000}%
\pgfsetstrokecolor{currentstroke}%
\pgfsetdash{}{0pt}%
\pgfpathmoveto{\pgfqpoint{4.705208in}{0.580556in}}%
\pgfpathlineto{\pgfqpoint{4.705208in}{3.727778in}}%
\pgfusepath{stroke}%
\end{pgfscope}%
\begin{pgfscope}%
\pgfsetrectcap%
\pgfsetmiterjoin%
\pgfsetlinewidth{0.803000pt}%
\definecolor{currentstroke}{rgb}{0.000000,0.000000,0.000000}%
\pgfsetstrokecolor{currentstroke}%
\pgfsetdash{}{0pt}%
\pgfpathmoveto{\pgfqpoint{7.801389in}{0.580556in}}%
\pgfpathlineto{\pgfqpoint{7.801389in}{3.727778in}}%
\pgfusepath{stroke}%
\end{pgfscope}%
\begin{pgfscope}%
\pgfsetrectcap%
\pgfsetmiterjoin%
\pgfsetlinewidth{0.803000pt}%
\definecolor{currentstroke}{rgb}{0.000000,0.000000,0.000000}%
\pgfsetstrokecolor{currentstroke}%
\pgfsetdash{}{0pt}%
\pgfpathmoveto{\pgfqpoint{4.705208in}{0.580556in}}%
\pgfpathlineto{\pgfqpoint{7.801389in}{0.580556in}}%
\pgfusepath{stroke}%
\end{pgfscope}%
\begin{pgfscope}%
\pgfsetrectcap%
\pgfsetmiterjoin%
\pgfsetlinewidth{0.803000pt}%
\definecolor{currentstroke}{rgb}{0.000000,0.000000,0.000000}%
\pgfsetstrokecolor{currentstroke}%
\pgfsetdash{}{0pt}%
\pgfpathmoveto{\pgfqpoint{4.705208in}{3.727778in}}%
\pgfpathlineto{\pgfqpoint{7.801389in}{3.727778in}}%
\pgfusepath{stroke}%
\end{pgfscope}%
\begin{pgfscope}%
\definecolor{textcolor}{rgb}{0.000000,0.000000,0.000000}%
\pgfsetstrokecolor{textcolor}%
\pgfsetfillcolor{textcolor}%
\pgftext[x=6.253299in,y=3.811111in,,base]{\color{textcolor}\sffamily\fontsize{12.000000}{14.400000}\selectfont \(\displaystyle  t = 1.0 \)}%
\end{pgfscope}%
\begin{pgfscope}%
\pgfsetbuttcap%
\pgfsetmiterjoin%
\definecolor{currentfill}{rgb}{1.000000,1.000000,1.000000}%
\pgfsetfillcolor{currentfill}%
\pgfsetlinewidth{0.000000pt}%
\definecolor{currentstroke}{rgb}{0.000000,0.000000,0.000000}%
\pgfsetstrokecolor{currentstroke}%
\pgfsetstrokeopacity{0.000000}%
\pgfsetdash{}{0pt}%
\pgfpathmoveto{\pgfqpoint{0.780208in}{8.480556in}}%
\pgfpathlineto{\pgfqpoint{3.876389in}{8.480556in}}%
\pgfpathlineto{\pgfqpoint{3.876389in}{11.627778in}}%
\pgfpathlineto{\pgfqpoint{0.780208in}{11.627778in}}%
\pgfpathclose%
\pgfusepath{fill}%
\end{pgfscope}%
\begin{pgfscope}%
\pgfsetbuttcap%
\pgfsetroundjoin%
\definecolor{currentfill}{rgb}{0.000000,0.000000,0.000000}%
\pgfsetfillcolor{currentfill}%
\pgfsetlinewidth{0.803000pt}%
\definecolor{currentstroke}{rgb}{0.000000,0.000000,0.000000}%
\pgfsetstrokecolor{currentstroke}%
\pgfsetdash{}{0pt}%
\pgfsys@defobject{currentmarker}{\pgfqpoint{0.000000in}{-0.048611in}}{\pgfqpoint{0.000000in}{0.000000in}}{%
\pgfpathmoveto{\pgfqpoint{0.000000in}{0.000000in}}%
\pgfpathlineto{\pgfqpoint{0.000000in}{-0.048611in}}%
\pgfusepath{stroke,fill}%
}%
\begin{pgfscope}%
\pgfsys@transformshift{0.884858in}{8.480556in}%
\pgfsys@useobject{currentmarker}{}%
\end{pgfscope}%
\end{pgfscope}%
\begin{pgfscope}%
\definecolor{textcolor}{rgb}{0.000000,0.000000,0.000000}%
\pgfsetstrokecolor{textcolor}%
\pgfsetfillcolor{textcolor}%
\pgftext[x=0.884858in,y=8.383333in,,top]{\color{textcolor}\sffamily\fontsize{10.000000}{12.000000}\selectfont −2}%
\end{pgfscope}%
\begin{pgfscope}%
\pgfsetbuttcap%
\pgfsetroundjoin%
\definecolor{currentfill}{rgb}{0.000000,0.000000,0.000000}%
\pgfsetfillcolor{currentfill}%
\pgfsetlinewidth{0.803000pt}%
\definecolor{currentstroke}{rgb}{0.000000,0.000000,0.000000}%
\pgfsetstrokecolor{currentstroke}%
\pgfsetdash{}{0pt}%
\pgfsys@defobject{currentmarker}{\pgfqpoint{0.000000in}{-0.048611in}}{\pgfqpoint{0.000000in}{0.000000in}}{%
\pgfpathmoveto{\pgfqpoint{0.000000in}{0.000000in}}%
\pgfpathlineto{\pgfqpoint{0.000000in}{-0.048611in}}%
\pgfusepath{stroke,fill}%
}%
\begin{pgfscope}%
\pgfsys@transformshift{1.606578in}{8.480556in}%
\pgfsys@useobject{currentmarker}{}%
\end{pgfscope}%
\end{pgfscope}%
\begin{pgfscope}%
\definecolor{textcolor}{rgb}{0.000000,0.000000,0.000000}%
\pgfsetstrokecolor{textcolor}%
\pgfsetfillcolor{textcolor}%
\pgftext[x=1.606578in,y=8.383333in,,top]{\color{textcolor}\sffamily\fontsize{10.000000}{12.000000}\selectfont −1}%
\end{pgfscope}%
\begin{pgfscope}%
\pgfsetbuttcap%
\pgfsetroundjoin%
\definecolor{currentfill}{rgb}{0.000000,0.000000,0.000000}%
\pgfsetfillcolor{currentfill}%
\pgfsetlinewidth{0.803000pt}%
\definecolor{currentstroke}{rgb}{0.000000,0.000000,0.000000}%
\pgfsetstrokecolor{currentstroke}%
\pgfsetdash{}{0pt}%
\pgfsys@defobject{currentmarker}{\pgfqpoint{0.000000in}{-0.048611in}}{\pgfqpoint{0.000000in}{0.000000in}}{%
\pgfpathmoveto{\pgfqpoint{0.000000in}{0.000000in}}%
\pgfpathlineto{\pgfqpoint{0.000000in}{-0.048611in}}%
\pgfusepath{stroke,fill}%
}%
\begin{pgfscope}%
\pgfsys@transformshift{2.328299in}{8.480556in}%
\pgfsys@useobject{currentmarker}{}%
\end{pgfscope}%
\end{pgfscope}%
\begin{pgfscope}%
\definecolor{textcolor}{rgb}{0.000000,0.000000,0.000000}%
\pgfsetstrokecolor{textcolor}%
\pgfsetfillcolor{textcolor}%
\pgftext[x=2.328299in,y=8.383333in,,top]{\color{textcolor}\sffamily\fontsize{10.000000}{12.000000}\selectfont 0}%
\end{pgfscope}%
\begin{pgfscope}%
\pgfsetbuttcap%
\pgfsetroundjoin%
\definecolor{currentfill}{rgb}{0.000000,0.000000,0.000000}%
\pgfsetfillcolor{currentfill}%
\pgfsetlinewidth{0.803000pt}%
\definecolor{currentstroke}{rgb}{0.000000,0.000000,0.000000}%
\pgfsetstrokecolor{currentstroke}%
\pgfsetdash{}{0pt}%
\pgfsys@defobject{currentmarker}{\pgfqpoint{0.000000in}{-0.048611in}}{\pgfqpoint{0.000000in}{0.000000in}}{%
\pgfpathmoveto{\pgfqpoint{0.000000in}{0.000000in}}%
\pgfpathlineto{\pgfqpoint{0.000000in}{-0.048611in}}%
\pgfusepath{stroke,fill}%
}%
\begin{pgfscope}%
\pgfsys@transformshift{3.050019in}{8.480556in}%
\pgfsys@useobject{currentmarker}{}%
\end{pgfscope}%
\end{pgfscope}%
\begin{pgfscope}%
\definecolor{textcolor}{rgb}{0.000000,0.000000,0.000000}%
\pgfsetstrokecolor{textcolor}%
\pgfsetfillcolor{textcolor}%
\pgftext[x=3.050019in,y=8.383333in,,top]{\color{textcolor}\sffamily\fontsize{10.000000}{12.000000}\selectfont 1}%
\end{pgfscope}%
\begin{pgfscope}%
\pgfsetbuttcap%
\pgfsetroundjoin%
\definecolor{currentfill}{rgb}{0.000000,0.000000,0.000000}%
\pgfsetfillcolor{currentfill}%
\pgfsetlinewidth{0.803000pt}%
\definecolor{currentstroke}{rgb}{0.000000,0.000000,0.000000}%
\pgfsetstrokecolor{currentstroke}%
\pgfsetdash{}{0pt}%
\pgfsys@defobject{currentmarker}{\pgfqpoint{0.000000in}{-0.048611in}}{\pgfqpoint{0.000000in}{0.000000in}}{%
\pgfpathmoveto{\pgfqpoint{0.000000in}{0.000000in}}%
\pgfpathlineto{\pgfqpoint{0.000000in}{-0.048611in}}%
\pgfusepath{stroke,fill}%
}%
\begin{pgfscope}%
\pgfsys@transformshift{3.771739in}{8.480556in}%
\pgfsys@useobject{currentmarker}{}%
\end{pgfscope}%
\end{pgfscope}%
\begin{pgfscope}%
\definecolor{textcolor}{rgb}{0.000000,0.000000,0.000000}%
\pgfsetstrokecolor{textcolor}%
\pgfsetfillcolor{textcolor}%
\pgftext[x=3.771739in,y=8.383333in,,top]{\color{textcolor}\sffamily\fontsize{10.000000}{12.000000}\selectfont 2}%
\end{pgfscope}%
\begin{pgfscope}%
\definecolor{textcolor}{rgb}{0.000000,0.000000,0.000000}%
\pgfsetstrokecolor{textcolor}%
\pgfsetfillcolor{textcolor}%
\pgftext[x=2.328299in,y=8.193365in,,top]{\color{textcolor}\sffamily\fontsize{10.000000}{12.000000}\selectfont \(\displaystyle x\)}%
\end{pgfscope}%
\begin{pgfscope}%
\pgfsetbuttcap%
\pgfsetroundjoin%
\definecolor{currentfill}{rgb}{0.000000,0.000000,0.000000}%
\pgfsetfillcolor{currentfill}%
\pgfsetlinewidth{0.803000pt}%
\definecolor{currentstroke}{rgb}{0.000000,0.000000,0.000000}%
\pgfsetstrokecolor{currentstroke}%
\pgfsetdash{}{0pt}%
\pgfsys@defobject{currentmarker}{\pgfqpoint{-0.048611in}{0.000000in}}{\pgfqpoint{0.000000in}{0.000000in}}{%
\pgfpathmoveto{\pgfqpoint{0.000000in}{0.000000in}}%
\pgfpathlineto{\pgfqpoint{-0.048611in}{0.000000in}}%
\pgfusepath{stroke,fill}%
}%
\begin{pgfscope}%
\pgfsys@transformshift{0.780208in}{8.623611in}%
\pgfsys@useobject{currentmarker}{}%
\end{pgfscope}%
\end{pgfscope}%
\begin{pgfscope}%
\definecolor{textcolor}{rgb}{0.000000,0.000000,0.000000}%
\pgfsetstrokecolor{textcolor}%
\pgfsetfillcolor{textcolor}%
\pgftext[x=0.462107in,y=8.570850in,left,base]{\color{textcolor}\sffamily\fontsize{10.000000}{12.000000}\selectfont 0.0}%
\end{pgfscope}%
\begin{pgfscope}%
\pgfsetbuttcap%
\pgfsetroundjoin%
\definecolor{currentfill}{rgb}{0.000000,0.000000,0.000000}%
\pgfsetfillcolor{currentfill}%
\pgfsetlinewidth{0.803000pt}%
\definecolor{currentstroke}{rgb}{0.000000,0.000000,0.000000}%
\pgfsetstrokecolor{currentstroke}%
\pgfsetdash{}{0pt}%
\pgfsys@defobject{currentmarker}{\pgfqpoint{-0.048611in}{0.000000in}}{\pgfqpoint{0.000000in}{0.000000in}}{%
\pgfpathmoveto{\pgfqpoint{0.000000in}{0.000000in}}%
\pgfpathlineto{\pgfqpoint{-0.048611in}{0.000000in}}%
\pgfusepath{stroke,fill}%
}%
\begin{pgfscope}%
\pgfsys@transformshift{0.780208in}{9.195833in}%
\pgfsys@useobject{currentmarker}{}%
\end{pgfscope}%
\end{pgfscope}%
\begin{pgfscope}%
\definecolor{textcolor}{rgb}{0.000000,0.000000,0.000000}%
\pgfsetstrokecolor{textcolor}%
\pgfsetfillcolor{textcolor}%
\pgftext[x=0.462107in,y=9.143072in,left,base]{\color{textcolor}\sffamily\fontsize{10.000000}{12.000000}\selectfont 0.2}%
\end{pgfscope}%
\begin{pgfscope}%
\pgfsetbuttcap%
\pgfsetroundjoin%
\definecolor{currentfill}{rgb}{0.000000,0.000000,0.000000}%
\pgfsetfillcolor{currentfill}%
\pgfsetlinewidth{0.803000pt}%
\definecolor{currentstroke}{rgb}{0.000000,0.000000,0.000000}%
\pgfsetstrokecolor{currentstroke}%
\pgfsetdash{}{0pt}%
\pgfsys@defobject{currentmarker}{\pgfqpoint{-0.048611in}{0.000000in}}{\pgfqpoint{0.000000in}{0.000000in}}{%
\pgfpathmoveto{\pgfqpoint{0.000000in}{0.000000in}}%
\pgfpathlineto{\pgfqpoint{-0.048611in}{0.000000in}}%
\pgfusepath{stroke,fill}%
}%
\begin{pgfscope}%
\pgfsys@transformshift{0.780208in}{9.768056in}%
\pgfsys@useobject{currentmarker}{}%
\end{pgfscope}%
\end{pgfscope}%
\begin{pgfscope}%
\definecolor{textcolor}{rgb}{0.000000,0.000000,0.000000}%
\pgfsetstrokecolor{textcolor}%
\pgfsetfillcolor{textcolor}%
\pgftext[x=0.462107in,y=9.715294in,left,base]{\color{textcolor}\sffamily\fontsize{10.000000}{12.000000}\selectfont 0.4}%
\end{pgfscope}%
\begin{pgfscope}%
\pgfsetbuttcap%
\pgfsetroundjoin%
\definecolor{currentfill}{rgb}{0.000000,0.000000,0.000000}%
\pgfsetfillcolor{currentfill}%
\pgfsetlinewidth{0.803000pt}%
\definecolor{currentstroke}{rgb}{0.000000,0.000000,0.000000}%
\pgfsetstrokecolor{currentstroke}%
\pgfsetdash{}{0pt}%
\pgfsys@defobject{currentmarker}{\pgfqpoint{-0.048611in}{0.000000in}}{\pgfqpoint{0.000000in}{0.000000in}}{%
\pgfpathmoveto{\pgfqpoint{0.000000in}{0.000000in}}%
\pgfpathlineto{\pgfqpoint{-0.048611in}{0.000000in}}%
\pgfusepath{stroke,fill}%
}%
\begin{pgfscope}%
\pgfsys@transformshift{0.780208in}{10.340278in}%
\pgfsys@useobject{currentmarker}{}%
\end{pgfscope}%
\end{pgfscope}%
\begin{pgfscope}%
\definecolor{textcolor}{rgb}{0.000000,0.000000,0.000000}%
\pgfsetstrokecolor{textcolor}%
\pgfsetfillcolor{textcolor}%
\pgftext[x=0.462107in,y=10.287516in,left,base]{\color{textcolor}\sffamily\fontsize{10.000000}{12.000000}\selectfont 0.6}%
\end{pgfscope}%
\begin{pgfscope}%
\pgfsetbuttcap%
\pgfsetroundjoin%
\definecolor{currentfill}{rgb}{0.000000,0.000000,0.000000}%
\pgfsetfillcolor{currentfill}%
\pgfsetlinewidth{0.803000pt}%
\definecolor{currentstroke}{rgb}{0.000000,0.000000,0.000000}%
\pgfsetstrokecolor{currentstroke}%
\pgfsetdash{}{0pt}%
\pgfsys@defobject{currentmarker}{\pgfqpoint{-0.048611in}{0.000000in}}{\pgfqpoint{0.000000in}{0.000000in}}{%
\pgfpathmoveto{\pgfqpoint{0.000000in}{0.000000in}}%
\pgfpathlineto{\pgfqpoint{-0.048611in}{0.000000in}}%
\pgfusepath{stroke,fill}%
}%
\begin{pgfscope}%
\pgfsys@transformshift{0.780208in}{10.912500in}%
\pgfsys@useobject{currentmarker}{}%
\end{pgfscope}%
\end{pgfscope}%
\begin{pgfscope}%
\definecolor{textcolor}{rgb}{0.000000,0.000000,0.000000}%
\pgfsetstrokecolor{textcolor}%
\pgfsetfillcolor{textcolor}%
\pgftext[x=0.462107in,y=10.859738in,left,base]{\color{textcolor}\sffamily\fontsize{10.000000}{12.000000}\selectfont 0.8}%
\end{pgfscope}%
\begin{pgfscope}%
\pgfsetbuttcap%
\pgfsetroundjoin%
\definecolor{currentfill}{rgb}{0.000000,0.000000,0.000000}%
\pgfsetfillcolor{currentfill}%
\pgfsetlinewidth{0.803000pt}%
\definecolor{currentstroke}{rgb}{0.000000,0.000000,0.000000}%
\pgfsetstrokecolor{currentstroke}%
\pgfsetdash{}{0pt}%
\pgfsys@defobject{currentmarker}{\pgfqpoint{-0.048611in}{0.000000in}}{\pgfqpoint{0.000000in}{0.000000in}}{%
\pgfpathmoveto{\pgfqpoint{0.000000in}{0.000000in}}%
\pgfpathlineto{\pgfqpoint{-0.048611in}{0.000000in}}%
\pgfusepath{stroke,fill}%
}%
\begin{pgfscope}%
\pgfsys@transformshift{0.780208in}{11.484722in}%
\pgfsys@useobject{currentmarker}{}%
\end{pgfscope}%
\end{pgfscope}%
\begin{pgfscope}%
\definecolor{textcolor}{rgb}{0.000000,0.000000,0.000000}%
\pgfsetstrokecolor{textcolor}%
\pgfsetfillcolor{textcolor}%
\pgftext[x=0.462107in,y=11.431961in,left,base]{\color{textcolor}\sffamily\fontsize{10.000000}{12.000000}\selectfont 1.0}%
\end{pgfscope}%
\begin{pgfscope}%
\definecolor{textcolor}{rgb}{0.000000,0.000000,0.000000}%
\pgfsetstrokecolor{textcolor}%
\pgfsetfillcolor{textcolor}%
\pgftext[x=0.406551in,y=10.054167in,,bottom,rotate=90.000000]{\color{textcolor}\sffamily\fontsize{10.000000}{12.000000}\selectfont \(\displaystyle U\)}%
\end{pgfscope}%
\begin{pgfscope}%
\pgfpathrectangle{\pgfqpoint{0.780208in}{8.480556in}}{\pgfqpoint{3.096181in}{3.147222in}}%
\pgfusepath{clip}%
\pgfsetrectcap%
\pgfsetroundjoin%
\pgfsetlinewidth{0.501875pt}%
\definecolor{currentstroke}{rgb}{0.000000,0.000000,0.000000}%
\pgfsetstrokecolor{currentstroke}%
\pgfsetdash{}{0pt}%
\pgfpathmoveto{\pgfqpoint{0.920944in}{11.484722in}}%
\pgfpathlineto{\pgfqpoint{0.993116in}{11.484722in}}%
\pgfpathlineto{\pgfqpoint{1.065288in}{11.484722in}}%
\pgfpathlineto{\pgfqpoint{1.137460in}{11.484722in}}%
\pgfpathlineto{\pgfqpoint{1.209632in}{11.484722in}}%
\pgfpathlineto{\pgfqpoint{1.281804in}{11.484722in}}%
\pgfpathlineto{\pgfqpoint{1.353976in}{11.484722in}}%
\pgfpathlineto{\pgfqpoint{1.426148in}{11.484722in}}%
\pgfpathlineto{\pgfqpoint{1.498320in}{11.484722in}}%
\pgfpathlineto{\pgfqpoint{1.570492in}{11.484722in}}%
\pgfpathlineto{\pgfqpoint{1.642664in}{11.484722in}}%
\pgfpathlineto{\pgfqpoint{1.714836in}{11.484722in}}%
\pgfpathlineto{\pgfqpoint{1.787008in}{11.484722in}}%
\pgfpathlineto{\pgfqpoint{1.859180in}{11.484722in}}%
\pgfpathlineto{\pgfqpoint{1.931352in}{11.484722in}}%
\pgfpathlineto{\pgfqpoint{2.003524in}{11.484722in}}%
\pgfpathlineto{\pgfqpoint{2.075696in}{11.484722in}}%
\pgfpathlineto{\pgfqpoint{2.147869in}{11.484722in}}%
\pgfpathlineto{\pgfqpoint{2.220041in}{11.484722in}}%
\pgfpathlineto{\pgfqpoint{2.292213in}{11.484722in}}%
\pgfpathlineto{\pgfqpoint{2.364385in}{8.623611in}}%
\pgfpathlineto{\pgfqpoint{2.436557in}{8.623611in}}%
\pgfpathlineto{\pgfqpoint{2.508729in}{8.623611in}}%
\pgfpathlineto{\pgfqpoint{2.580901in}{8.623611in}}%
\pgfpathlineto{\pgfqpoint{2.653073in}{8.623611in}}%
\pgfpathlineto{\pgfqpoint{2.725245in}{8.623611in}}%
\pgfpathlineto{\pgfqpoint{2.797417in}{8.623611in}}%
\pgfpathlineto{\pgfqpoint{2.869589in}{8.623611in}}%
\pgfpathlineto{\pgfqpoint{2.941761in}{8.623611in}}%
\pgfpathlineto{\pgfqpoint{3.013933in}{8.623611in}}%
\pgfpathlineto{\pgfqpoint{3.086105in}{8.623611in}}%
\pgfpathlineto{\pgfqpoint{3.158277in}{8.623611in}}%
\pgfpathlineto{\pgfqpoint{3.230449in}{8.623611in}}%
\pgfpathlineto{\pgfqpoint{3.302621in}{8.623611in}}%
\pgfpathlineto{\pgfqpoint{3.374793in}{8.623611in}}%
\pgfpathlineto{\pgfqpoint{3.446965in}{8.623611in}}%
\pgfpathlineto{\pgfqpoint{3.519137in}{8.623611in}}%
\pgfpathlineto{\pgfqpoint{3.591309in}{8.623611in}}%
\pgfpathlineto{\pgfqpoint{3.663481in}{8.623611in}}%
\pgfpathlineto{\pgfqpoint{3.735653in}{8.623611in}}%
\pgfusepath{stroke}%
\end{pgfscope}%
\begin{pgfscope}%
\pgfsetrectcap%
\pgfsetmiterjoin%
\pgfsetlinewidth{0.803000pt}%
\definecolor{currentstroke}{rgb}{0.000000,0.000000,0.000000}%
\pgfsetstrokecolor{currentstroke}%
\pgfsetdash{}{0pt}%
\pgfpathmoveto{\pgfqpoint{0.780208in}{8.480556in}}%
\pgfpathlineto{\pgfqpoint{0.780208in}{11.627778in}}%
\pgfusepath{stroke}%
\end{pgfscope}%
\begin{pgfscope}%
\pgfsetrectcap%
\pgfsetmiterjoin%
\pgfsetlinewidth{0.803000pt}%
\definecolor{currentstroke}{rgb}{0.000000,0.000000,0.000000}%
\pgfsetstrokecolor{currentstroke}%
\pgfsetdash{}{0pt}%
\pgfpathmoveto{\pgfqpoint{3.876389in}{8.480556in}}%
\pgfpathlineto{\pgfqpoint{3.876389in}{11.627778in}}%
\pgfusepath{stroke}%
\end{pgfscope}%
\begin{pgfscope}%
\pgfsetrectcap%
\pgfsetmiterjoin%
\pgfsetlinewidth{0.803000pt}%
\definecolor{currentstroke}{rgb}{0.000000,0.000000,0.000000}%
\pgfsetstrokecolor{currentstroke}%
\pgfsetdash{}{0pt}%
\pgfpathmoveto{\pgfqpoint{0.780208in}{8.480556in}}%
\pgfpathlineto{\pgfqpoint{3.876389in}{8.480556in}}%
\pgfusepath{stroke}%
\end{pgfscope}%
\begin{pgfscope}%
\pgfsetrectcap%
\pgfsetmiterjoin%
\pgfsetlinewidth{0.803000pt}%
\definecolor{currentstroke}{rgb}{0.000000,0.000000,0.000000}%
\pgfsetstrokecolor{currentstroke}%
\pgfsetdash{}{0pt}%
\pgfpathmoveto{\pgfqpoint{0.780208in}{11.627778in}}%
\pgfpathlineto{\pgfqpoint{3.876389in}{11.627778in}}%
\pgfusepath{stroke}%
\end{pgfscope}%
\begin{pgfscope}%
\definecolor{textcolor}{rgb}{0.000000,0.000000,0.000000}%
\pgfsetstrokecolor{textcolor}%
\pgfsetfillcolor{textcolor}%
\pgftext[x=2.328299in,y=11.711111in,,base]{\color{textcolor}\sffamily\fontsize{12.000000}{14.400000}\selectfont \(\displaystyle  t = 0.0 \)}%
\end{pgfscope}%
\begin{pgfscope}%
\pgfsetbuttcap%
\pgfsetmiterjoin%
\definecolor{currentfill}{rgb}{1.000000,1.000000,1.000000}%
\pgfsetfillcolor{currentfill}%
\pgfsetfillopacity{0.800000}%
\pgfsetlinewidth{1.003750pt}%
\definecolor{currentstroke}{rgb}{0.800000,0.800000,0.800000}%
\pgfsetstrokecolor{currentstroke}%
\pgfsetstrokeopacity{0.800000}%
\pgfsetdash{}{0pt}%
\pgfpathmoveto{\pgfqpoint{1.566737in}{9.523690in}}%
\pgfpathlineto{\pgfqpoint{3.779167in}{9.523690in}}%
\pgfpathquadraticcurveto{\pgfqpoint{3.806944in}{9.523690in}}{\pgfqpoint{3.806944in}{9.551468in}}%
\pgfpathlineto{\pgfqpoint{3.806944in}{10.556865in}}%
\pgfpathquadraticcurveto{\pgfqpoint{3.806944in}{10.584643in}}{\pgfqpoint{3.779167in}{10.584643in}}%
\pgfpathlineto{\pgfqpoint{1.566737in}{10.584643in}}%
\pgfpathquadraticcurveto{\pgfqpoint{1.538959in}{10.584643in}}{\pgfqpoint{1.538959in}{10.556865in}}%
\pgfpathlineto{\pgfqpoint{1.538959in}{9.551468in}}%
\pgfpathquadraticcurveto{\pgfqpoint{1.538959in}{9.523690in}}{\pgfqpoint{1.566737in}{9.523690in}}%
\pgfpathclose%
\pgfusepath{stroke,fill}%
\end{pgfscope}%
\begin{pgfscope}%
\pgfsetrectcap%
\pgfsetroundjoin%
\pgfsetlinewidth{1.505625pt}%
\definecolor{currentstroke}{rgb}{0.121569,0.466667,0.705882}%
\pgfsetstrokecolor{currentstroke}%
\pgfsetdash{}{0pt}%
\pgfpathmoveto{\pgfqpoint{1.594515in}{10.472176in}}%
\pgfpathlineto{\pgfqpoint{1.872293in}{10.472176in}}%
\pgfusepath{stroke}%
\end{pgfscope}%
\begin{pgfscope}%
\definecolor{textcolor}{rgb}{0.000000,0.000000,0.000000}%
\pgfsetstrokecolor{textcolor}%
\pgfsetfillcolor{textcolor}%
\pgftext[x=1.983404in,y=10.423564in,left,base]{\color{textcolor}\sffamily\fontsize{10.000000}{12.000000}\selectfont Richtmyer}%
\end{pgfscope}%
\begin{pgfscope}%
\pgfsetrectcap%
\pgfsetroundjoin%
\pgfsetlinewidth{1.505625pt}%
\definecolor{currentstroke}{rgb}{1.000000,0.498039,0.054902}%
\pgfsetstrokecolor{currentstroke}%
\pgfsetdash{}{0pt}%
\pgfpathmoveto{\pgfqpoint{1.594515in}{10.268318in}}%
\pgfpathlineto{\pgfqpoint{1.872293in}{10.268318in}}%
\pgfusepath{stroke}%
\end{pgfscope}%
\begin{pgfscope}%
\definecolor{textcolor}{rgb}{0.000000,0.000000,0.000000}%
\pgfsetstrokecolor{textcolor}%
\pgfsetfillcolor{textcolor}%
\pgftext[x=1.983404in,y=10.219707in,left,base]{\color{textcolor}\sffamily\fontsize{10.000000}{12.000000}\selectfont Lax--Wendroff}%
\end{pgfscope}%
\begin{pgfscope}%
\pgfsetrectcap%
\pgfsetroundjoin%
\pgfsetlinewidth{1.505625pt}%
\definecolor{currentstroke}{rgb}{0.172549,0.627451,0.172549}%
\pgfsetstrokecolor{currentstroke}%
\pgfsetdash{}{0pt}%
\pgfpathmoveto{\pgfqpoint{1.594515in}{10.064461in}}%
\pgfpathlineto{\pgfqpoint{1.872293in}{10.064461in}}%
\pgfusepath{stroke}%
\end{pgfscope}%
\begin{pgfscope}%
\definecolor{textcolor}{rgb}{0.000000,0.000000,0.000000}%
\pgfsetstrokecolor{textcolor}%
\pgfsetfillcolor{textcolor}%
\pgftext[x=1.983404in,y=10.015850in,left,base]{\color{textcolor}\sffamily\fontsize{10.000000}{12.000000}\selectfont Conservative upwind}%
\end{pgfscope}%
\begin{pgfscope}%
\pgfsetrectcap%
\pgfsetroundjoin%
\pgfsetlinewidth{1.505625pt}%
\definecolor{currentstroke}{rgb}{0.839216,0.152941,0.156863}%
\pgfsetstrokecolor{currentstroke}%
\pgfsetdash{}{0pt}%
\pgfpathmoveto{\pgfqpoint{1.594515in}{9.860604in}}%
\pgfpathlineto{\pgfqpoint{1.872293in}{9.860604in}}%
\pgfusepath{stroke}%
\end{pgfscope}%
\begin{pgfscope}%
\definecolor{textcolor}{rgb}{0.000000,0.000000,0.000000}%
\pgfsetstrokecolor{textcolor}%
\pgfsetfillcolor{textcolor}%
\pgftext[x=1.983404in,y=9.811993in,left,base]{\color{textcolor}\sffamily\fontsize{10.000000}{12.000000}\selectfont Non-conservative upwind}%
\end{pgfscope}%
\begin{pgfscope}%
\pgfsetrectcap%
\pgfsetroundjoin%
\pgfsetlinewidth{0.501875pt}%
\definecolor{currentstroke}{rgb}{0.000000,0.000000,0.000000}%
\pgfsetstrokecolor{currentstroke}%
\pgfsetdash{}{0pt}%
\pgfpathmoveto{\pgfqpoint{1.594515in}{9.656747in}}%
\pgfpathlineto{\pgfqpoint{1.872293in}{9.656747in}}%
\pgfusepath{stroke}%
\end{pgfscope}%
\begin{pgfscope}%
\definecolor{textcolor}{rgb}{0.000000,0.000000,0.000000}%
\pgfsetstrokecolor{textcolor}%
\pgfsetfillcolor{textcolor}%
\pgftext[x=1.983404in,y=9.608136in,left,base]{\color{textcolor}\sffamily\fontsize{10.000000}{12.000000}\selectfont Analytical}%
\end{pgfscope}%
\end{pgfpicture}%
\makeatother%
\endgroup%
}
\caption{Heat map of solutions using triangle element for the first equation}
\label{Fig:Tri1}
}
{
\footnotesize Top left: $u_h$ with $ N = 256 $; top right: $ u_h - u $ at nodes with $ N = 256 $; bottom left: $ u_h - u $ at $\Omega$ with $ N = 256 $; bottom right: $ u_h - u $ at $\Omega$ with $ N = 16 $.
}
\end{figure}

\begin{figure}[htbp]
{
\centering
\scalebox{0.7}{%% Creator: Matplotlib, PGF backend
%%
%% To include the figure in your LaTeX document, write
%%   \input{<filename>.pgf}
%%
%% Make sure the required packages are loaded in your preamble
%%   \usepackage{pgf}
%%
%% Figures using additional raster images can only be included by \input if
%% they are in the same directory as the main LaTeX file. For loading figures
%% from other directories you can use the `import` package
%%   \usepackage{import}
%% and then include the figures with
%%   \import{<path to file>}{<filename>.pgf}
%%
%% Matplotlib used the following preamble
%%   \usepackage{fontspec}
%%   \setmainfont{DejaVuSerif.ttf}[Path=/home/lzh/anaconda3/envs/numpde/lib/python3.7/site-packages/matplotlib/mpl-data/fonts/ttf/]
%%   \setsansfont{DejaVuSans.ttf}[Path=/home/lzh/anaconda3/envs/numpde/lib/python3.7/site-packages/matplotlib/mpl-data/fonts/ttf/]
%%   \setmonofont{DejaVuSansMono.ttf}[Path=/home/lzh/anaconda3/envs/numpde/lib/python3.7/site-packages/matplotlib/mpl-data/fonts/ttf/]
%%
\begingroup%
\makeatletter%
\begin{pgfpicture}%
\pgfpathrectangle{\pgfpointorigin}{\pgfqpoint{8.000000in}{6.000000in}}%
\pgfusepath{use as bounding box, clip}%
\begin{pgfscope}%
\pgfsetbuttcap%
\pgfsetmiterjoin%
\definecolor{currentfill}{rgb}{1.000000,1.000000,1.000000}%
\pgfsetfillcolor{currentfill}%
\pgfsetlinewidth{0.000000pt}%
\definecolor{currentstroke}{rgb}{1.000000,1.000000,1.000000}%
\pgfsetstrokecolor{currentstroke}%
\pgfsetdash{}{0pt}%
\pgfpathmoveto{\pgfqpoint{0.000000in}{0.000000in}}%
\pgfpathlineto{\pgfqpoint{8.000000in}{0.000000in}}%
\pgfpathlineto{\pgfqpoint{8.000000in}{6.000000in}}%
\pgfpathlineto{\pgfqpoint{0.000000in}{6.000000in}}%
\pgfpathclose%
\pgfusepath{fill}%
\end{pgfscope}%
\begin{pgfscope}%
\pgfsetbuttcap%
\pgfsetmiterjoin%
\definecolor{currentfill}{rgb}{1.000000,1.000000,1.000000}%
\pgfsetfillcolor{currentfill}%
\pgfsetlinewidth{0.000000pt}%
\definecolor{currentstroke}{rgb}{0.000000,0.000000,0.000000}%
\pgfsetstrokecolor{currentstroke}%
\pgfsetstrokeopacity{0.000000}%
\pgfsetdash{}{0pt}%
\pgfpathmoveto{\pgfqpoint{0.717778in}{3.518562in}}%
\pgfpathlineto{\pgfqpoint{2.918806in}{3.518562in}}%
\pgfpathlineto{\pgfqpoint{2.918806in}{5.719590in}}%
\pgfpathlineto{\pgfqpoint{0.717778in}{5.719590in}}%
\pgfpathclose%
\pgfusepath{fill}%
\end{pgfscope}%
\begin{pgfscope}%
\pgfpathrectangle{\pgfqpoint{0.717778in}{3.518562in}}{\pgfqpoint{2.201028in}{2.201028in}}%
\pgfusepath{clip}%
\pgfsys@transformshift{0.717778in}{3.518562in}%
\pgftext[left,bottom]{\pgfimage[interpolate=true,width=2.210000in,height=2.210000in]{Figure2-img0.png}}%
\end{pgfscope}%
\begin{pgfscope}%
\pgfsetbuttcap%
\pgfsetroundjoin%
\definecolor{currentfill}{rgb}{0.000000,0.000000,0.000000}%
\pgfsetfillcolor{currentfill}%
\pgfsetlinewidth{0.803000pt}%
\definecolor{currentstroke}{rgb}{0.000000,0.000000,0.000000}%
\pgfsetstrokecolor{currentstroke}%
\pgfsetdash{}{0pt}%
\pgfsys@defobject{currentmarker}{\pgfqpoint{0.000000in}{-0.048611in}}{\pgfqpoint{0.000000in}{0.000000in}}{%
\pgfpathmoveto{\pgfqpoint{0.000000in}{0.000000in}}%
\pgfpathlineto{\pgfqpoint{0.000000in}{-0.048611in}}%
\pgfusepath{stroke,fill}%
}%
\begin{pgfscope}%
\pgfsys@transformshift{0.722060in}{3.518562in}%
\pgfsys@useobject{currentmarker}{}%
\end{pgfscope}%
\end{pgfscope}%
\begin{pgfscope}%
\definecolor{textcolor}{rgb}{0.000000,0.000000,0.000000}%
\pgfsetstrokecolor{textcolor}%
\pgfsetfillcolor{textcolor}%
\pgftext[x=0.722060in,y=3.421340in,,top]{\color{textcolor}\sffamily\fontsize{10.000000}{12.000000}\selectfont 0}%
\end{pgfscope}%
\begin{pgfscope}%
\pgfsetbuttcap%
\pgfsetroundjoin%
\definecolor{currentfill}{rgb}{0.000000,0.000000,0.000000}%
\pgfsetfillcolor{currentfill}%
\pgfsetlinewidth{0.803000pt}%
\definecolor{currentstroke}{rgb}{0.000000,0.000000,0.000000}%
\pgfsetstrokecolor{currentstroke}%
\pgfsetdash{}{0pt}%
\pgfsys@defobject{currentmarker}{\pgfqpoint{0.000000in}{-0.048611in}}{\pgfqpoint{0.000000in}{0.000000in}}{%
\pgfpathmoveto{\pgfqpoint{0.000000in}{0.000000in}}%
\pgfpathlineto{\pgfqpoint{0.000000in}{-0.048611in}}%
\pgfusepath{stroke,fill}%
}%
\begin{pgfscope}%
\pgfsys@transformshift{1.578491in}{3.518562in}%
\pgfsys@useobject{currentmarker}{}%
\end{pgfscope}%
\end{pgfscope}%
\begin{pgfscope}%
\definecolor{textcolor}{rgb}{0.000000,0.000000,0.000000}%
\pgfsetstrokecolor{textcolor}%
\pgfsetfillcolor{textcolor}%
\pgftext[x=1.578491in,y=3.421340in,,top]{\color{textcolor}\sffamily\fontsize{10.000000}{12.000000}\selectfont 100}%
\end{pgfscope}%
\begin{pgfscope}%
\pgfsetbuttcap%
\pgfsetroundjoin%
\definecolor{currentfill}{rgb}{0.000000,0.000000,0.000000}%
\pgfsetfillcolor{currentfill}%
\pgfsetlinewidth{0.803000pt}%
\definecolor{currentstroke}{rgb}{0.000000,0.000000,0.000000}%
\pgfsetstrokecolor{currentstroke}%
\pgfsetdash{}{0pt}%
\pgfsys@defobject{currentmarker}{\pgfqpoint{0.000000in}{-0.048611in}}{\pgfqpoint{0.000000in}{0.000000in}}{%
\pgfpathmoveto{\pgfqpoint{0.000000in}{0.000000in}}%
\pgfpathlineto{\pgfqpoint{0.000000in}{-0.048611in}}%
\pgfusepath{stroke,fill}%
}%
\begin{pgfscope}%
\pgfsys@transformshift{2.434922in}{3.518562in}%
\pgfsys@useobject{currentmarker}{}%
\end{pgfscope}%
\end{pgfscope}%
\begin{pgfscope}%
\definecolor{textcolor}{rgb}{0.000000,0.000000,0.000000}%
\pgfsetstrokecolor{textcolor}%
\pgfsetfillcolor{textcolor}%
\pgftext[x=2.434922in,y=3.421340in,,top]{\color{textcolor}\sffamily\fontsize{10.000000}{12.000000}\selectfont 200}%
\end{pgfscope}%
\begin{pgfscope}%
\definecolor{textcolor}{rgb}{0.000000,0.000000,0.000000}%
\pgfsetstrokecolor{textcolor}%
\pgfsetfillcolor{textcolor}%
\pgftext[x=1.818292in,y=3.231371in,,top]{\color{textcolor}\sffamily\fontsize{10.000000}{12.000000}\selectfont \(\displaystyle x\)}%
\end{pgfscope}%
\begin{pgfscope}%
\pgfsetbuttcap%
\pgfsetroundjoin%
\definecolor{currentfill}{rgb}{0.000000,0.000000,0.000000}%
\pgfsetfillcolor{currentfill}%
\pgfsetlinewidth{0.803000pt}%
\definecolor{currentstroke}{rgb}{0.000000,0.000000,0.000000}%
\pgfsetstrokecolor{currentstroke}%
\pgfsetdash{}{0pt}%
\pgfsys@defobject{currentmarker}{\pgfqpoint{-0.048611in}{0.000000in}}{\pgfqpoint{0.000000in}{0.000000in}}{%
\pgfpathmoveto{\pgfqpoint{0.000000in}{0.000000in}}%
\pgfpathlineto{\pgfqpoint{-0.048611in}{0.000000in}}%
\pgfusepath{stroke,fill}%
}%
\begin{pgfscope}%
\pgfsys@transformshift{0.717778in}{3.522844in}%
\pgfsys@useobject{currentmarker}{}%
\end{pgfscope}%
\end{pgfscope}%
\begin{pgfscope}%
\definecolor{textcolor}{rgb}{0.000000,0.000000,0.000000}%
\pgfsetstrokecolor{textcolor}%
\pgfsetfillcolor{textcolor}%
\pgftext[x=0.532190in,y=3.470083in,left,base]{\color{textcolor}\sffamily\fontsize{10.000000}{12.000000}\selectfont 0}%
\end{pgfscope}%
\begin{pgfscope}%
\pgfsetbuttcap%
\pgfsetroundjoin%
\definecolor{currentfill}{rgb}{0.000000,0.000000,0.000000}%
\pgfsetfillcolor{currentfill}%
\pgfsetlinewidth{0.803000pt}%
\definecolor{currentstroke}{rgb}{0.000000,0.000000,0.000000}%
\pgfsetstrokecolor{currentstroke}%
\pgfsetdash{}{0pt}%
\pgfsys@defobject{currentmarker}{\pgfqpoint{-0.048611in}{0.000000in}}{\pgfqpoint{0.000000in}{0.000000in}}{%
\pgfpathmoveto{\pgfqpoint{0.000000in}{0.000000in}}%
\pgfpathlineto{\pgfqpoint{-0.048611in}{0.000000in}}%
\pgfusepath{stroke,fill}%
}%
\begin{pgfscope}%
\pgfsys@transformshift{0.717778in}{3.951060in}%
\pgfsys@useobject{currentmarker}{}%
\end{pgfscope}%
\end{pgfscope}%
\begin{pgfscope}%
\definecolor{textcolor}{rgb}{0.000000,0.000000,0.000000}%
\pgfsetstrokecolor{textcolor}%
\pgfsetfillcolor{textcolor}%
\pgftext[x=0.443825in,y=3.898298in,left,base]{\color{textcolor}\sffamily\fontsize{10.000000}{12.000000}\selectfont 50}%
\end{pgfscope}%
\begin{pgfscope}%
\pgfsetbuttcap%
\pgfsetroundjoin%
\definecolor{currentfill}{rgb}{0.000000,0.000000,0.000000}%
\pgfsetfillcolor{currentfill}%
\pgfsetlinewidth{0.803000pt}%
\definecolor{currentstroke}{rgb}{0.000000,0.000000,0.000000}%
\pgfsetstrokecolor{currentstroke}%
\pgfsetdash{}{0pt}%
\pgfsys@defobject{currentmarker}{\pgfqpoint{-0.048611in}{0.000000in}}{\pgfqpoint{0.000000in}{0.000000in}}{%
\pgfpathmoveto{\pgfqpoint{0.000000in}{0.000000in}}%
\pgfpathlineto{\pgfqpoint{-0.048611in}{0.000000in}}%
\pgfusepath{stroke,fill}%
}%
\begin{pgfscope}%
\pgfsys@transformshift{0.717778in}{4.379275in}%
\pgfsys@useobject{currentmarker}{}%
\end{pgfscope}%
\end{pgfscope}%
\begin{pgfscope}%
\definecolor{textcolor}{rgb}{0.000000,0.000000,0.000000}%
\pgfsetstrokecolor{textcolor}%
\pgfsetfillcolor{textcolor}%
\pgftext[x=0.355460in,y=4.326514in,left,base]{\color{textcolor}\sffamily\fontsize{10.000000}{12.000000}\selectfont 100}%
\end{pgfscope}%
\begin{pgfscope}%
\pgfsetbuttcap%
\pgfsetroundjoin%
\definecolor{currentfill}{rgb}{0.000000,0.000000,0.000000}%
\pgfsetfillcolor{currentfill}%
\pgfsetlinewidth{0.803000pt}%
\definecolor{currentstroke}{rgb}{0.000000,0.000000,0.000000}%
\pgfsetstrokecolor{currentstroke}%
\pgfsetdash{}{0pt}%
\pgfsys@defobject{currentmarker}{\pgfqpoint{-0.048611in}{0.000000in}}{\pgfqpoint{0.000000in}{0.000000in}}{%
\pgfpathmoveto{\pgfqpoint{0.000000in}{0.000000in}}%
\pgfpathlineto{\pgfqpoint{-0.048611in}{0.000000in}}%
\pgfusepath{stroke,fill}%
}%
\begin{pgfscope}%
\pgfsys@transformshift{0.717778in}{4.807491in}%
\pgfsys@useobject{currentmarker}{}%
\end{pgfscope}%
\end{pgfscope}%
\begin{pgfscope}%
\definecolor{textcolor}{rgb}{0.000000,0.000000,0.000000}%
\pgfsetstrokecolor{textcolor}%
\pgfsetfillcolor{textcolor}%
\pgftext[x=0.355460in,y=4.754729in,left,base]{\color{textcolor}\sffamily\fontsize{10.000000}{12.000000}\selectfont 150}%
\end{pgfscope}%
\begin{pgfscope}%
\pgfsetbuttcap%
\pgfsetroundjoin%
\definecolor{currentfill}{rgb}{0.000000,0.000000,0.000000}%
\pgfsetfillcolor{currentfill}%
\pgfsetlinewidth{0.803000pt}%
\definecolor{currentstroke}{rgb}{0.000000,0.000000,0.000000}%
\pgfsetstrokecolor{currentstroke}%
\pgfsetdash{}{0pt}%
\pgfsys@defobject{currentmarker}{\pgfqpoint{-0.048611in}{0.000000in}}{\pgfqpoint{0.000000in}{0.000000in}}{%
\pgfpathmoveto{\pgfqpoint{0.000000in}{0.000000in}}%
\pgfpathlineto{\pgfqpoint{-0.048611in}{0.000000in}}%
\pgfusepath{stroke,fill}%
}%
\begin{pgfscope}%
\pgfsys@transformshift{0.717778in}{5.235706in}%
\pgfsys@useobject{currentmarker}{}%
\end{pgfscope}%
\end{pgfscope}%
\begin{pgfscope}%
\definecolor{textcolor}{rgb}{0.000000,0.000000,0.000000}%
\pgfsetstrokecolor{textcolor}%
\pgfsetfillcolor{textcolor}%
\pgftext[x=0.355460in,y=5.182945in,left,base]{\color{textcolor}\sffamily\fontsize{10.000000}{12.000000}\selectfont 200}%
\end{pgfscope}%
\begin{pgfscope}%
\pgfsetbuttcap%
\pgfsetroundjoin%
\definecolor{currentfill}{rgb}{0.000000,0.000000,0.000000}%
\pgfsetfillcolor{currentfill}%
\pgfsetlinewidth{0.803000pt}%
\definecolor{currentstroke}{rgb}{0.000000,0.000000,0.000000}%
\pgfsetstrokecolor{currentstroke}%
\pgfsetdash{}{0pt}%
\pgfsys@defobject{currentmarker}{\pgfqpoint{-0.048611in}{0.000000in}}{\pgfqpoint{0.000000in}{0.000000in}}{%
\pgfpathmoveto{\pgfqpoint{0.000000in}{0.000000in}}%
\pgfpathlineto{\pgfqpoint{-0.048611in}{0.000000in}}%
\pgfusepath{stroke,fill}%
}%
\begin{pgfscope}%
\pgfsys@transformshift{0.717778in}{5.663922in}%
\pgfsys@useobject{currentmarker}{}%
\end{pgfscope}%
\end{pgfscope}%
\begin{pgfscope}%
\definecolor{textcolor}{rgb}{0.000000,0.000000,0.000000}%
\pgfsetstrokecolor{textcolor}%
\pgfsetfillcolor{textcolor}%
\pgftext[x=0.355460in,y=5.611160in,left,base]{\color{textcolor}\sffamily\fontsize{10.000000}{12.000000}\selectfont 250}%
\end{pgfscope}%
\begin{pgfscope}%
\definecolor{textcolor}{rgb}{0.000000,0.000000,0.000000}%
\pgfsetstrokecolor{textcolor}%
\pgfsetfillcolor{textcolor}%
\pgftext[x=0.299904in,y=4.619076in,,bottom,rotate=90.000000]{\color{textcolor}\sffamily\fontsize{10.000000}{12.000000}\selectfont \(\displaystyle y\)}%
\end{pgfscope}%
\begin{pgfscope}%
\pgfsetrectcap%
\pgfsetmiterjoin%
\pgfsetlinewidth{0.803000pt}%
\definecolor{currentstroke}{rgb}{0.000000,0.000000,0.000000}%
\pgfsetstrokecolor{currentstroke}%
\pgfsetdash{}{0pt}%
\pgfpathmoveto{\pgfqpoint{0.717778in}{3.518562in}}%
\pgfpathlineto{\pgfqpoint{0.717778in}{5.719590in}}%
\pgfusepath{stroke}%
\end{pgfscope}%
\begin{pgfscope}%
\pgfsetrectcap%
\pgfsetmiterjoin%
\pgfsetlinewidth{0.803000pt}%
\definecolor{currentstroke}{rgb}{0.000000,0.000000,0.000000}%
\pgfsetstrokecolor{currentstroke}%
\pgfsetdash{}{0pt}%
\pgfpathmoveto{\pgfqpoint{2.918806in}{3.518562in}}%
\pgfpathlineto{\pgfqpoint{2.918806in}{5.719590in}}%
\pgfusepath{stroke}%
\end{pgfscope}%
\begin{pgfscope}%
\pgfsetrectcap%
\pgfsetmiterjoin%
\pgfsetlinewidth{0.803000pt}%
\definecolor{currentstroke}{rgb}{0.000000,0.000000,0.000000}%
\pgfsetstrokecolor{currentstroke}%
\pgfsetdash{}{0pt}%
\pgfpathmoveto{\pgfqpoint{0.717778in}{3.518562in}}%
\pgfpathlineto{\pgfqpoint{2.918806in}{3.518562in}}%
\pgfusepath{stroke}%
\end{pgfscope}%
\begin{pgfscope}%
\pgfsetrectcap%
\pgfsetmiterjoin%
\pgfsetlinewidth{0.803000pt}%
\definecolor{currentstroke}{rgb}{0.000000,0.000000,0.000000}%
\pgfsetstrokecolor{currentstroke}%
\pgfsetdash{}{0pt}%
\pgfpathmoveto{\pgfqpoint{0.717778in}{5.719590in}}%
\pgfpathlineto{\pgfqpoint{2.918806in}{5.719590in}}%
\pgfusepath{stroke}%
\end{pgfscope}%
\begin{pgfscope}%
\pgfpathrectangle{\pgfqpoint{3.056370in}{3.443136in}}{\pgfqpoint{0.117594in}{2.351880in}}%
\pgfusepath{clip}%
\pgfsetbuttcap%
\pgfsetmiterjoin%
\definecolor{currentfill}{rgb}{1.000000,1.000000,1.000000}%
\pgfsetfillcolor{currentfill}%
\pgfsetlinewidth{0.010037pt}%
\definecolor{currentstroke}{rgb}{1.000000,1.000000,1.000000}%
\pgfsetstrokecolor{currentstroke}%
\pgfsetdash{}{0pt}%
\pgfpathmoveto{\pgfqpoint{3.056370in}{3.443136in}}%
\pgfpathlineto{\pgfqpoint{3.056370in}{3.452323in}}%
\pgfpathlineto{\pgfqpoint{3.056370in}{5.785829in}}%
\pgfpathlineto{\pgfqpoint{3.056370in}{5.795016in}}%
\pgfpathlineto{\pgfqpoint{3.173964in}{5.795016in}}%
\pgfpathlineto{\pgfqpoint{3.173964in}{5.785829in}}%
\pgfpathlineto{\pgfqpoint{3.173964in}{3.452323in}}%
\pgfpathlineto{\pgfqpoint{3.173964in}{3.443136in}}%
\pgfpathclose%
\pgfusepath{stroke,fill}%
\end{pgfscope}%
\begin{pgfscope}%
\pgfsys@transformshift{3.060000in}{3.440000in}%
\pgftext[left,bottom]{\pgfimage[interpolate=true,width=0.110000in,height=2.360000in]{Figure2-img1.png}}%
\end{pgfscope}%
\begin{pgfscope}%
\pgfsetbuttcap%
\pgfsetroundjoin%
\definecolor{currentfill}{rgb}{0.000000,0.000000,0.000000}%
\pgfsetfillcolor{currentfill}%
\pgfsetlinewidth{0.803000pt}%
\definecolor{currentstroke}{rgb}{0.000000,0.000000,0.000000}%
\pgfsetstrokecolor{currentstroke}%
\pgfsetdash{}{0pt}%
\pgfsys@defobject{currentmarker}{\pgfqpoint{0.000000in}{0.000000in}}{\pgfqpoint{0.048611in}{0.000000in}}{%
\pgfpathmoveto{\pgfqpoint{0.000000in}{0.000000in}}%
\pgfpathlineto{\pgfqpoint{0.048611in}{0.000000in}}%
\pgfusepath{stroke,fill}%
}%
\begin{pgfscope}%
\pgfsys@transformshift{3.173964in}{3.443151in}%
\pgfsys@useobject{currentmarker}{}%
\end{pgfscope}%
\end{pgfscope}%
\begin{pgfscope}%
\definecolor{textcolor}{rgb}{0.000000,0.000000,0.000000}%
\pgfsetstrokecolor{textcolor}%
\pgfsetfillcolor{textcolor}%
\pgftext[x=3.271186in,y=3.390389in,left,base]{\color{textcolor}\sffamily\fontsize{10.000000}{12.000000}\selectfont −1.0}%
\end{pgfscope}%
\begin{pgfscope}%
\pgfsetbuttcap%
\pgfsetroundjoin%
\definecolor{currentfill}{rgb}{0.000000,0.000000,0.000000}%
\pgfsetfillcolor{currentfill}%
\pgfsetlinewidth{0.803000pt}%
\definecolor{currentstroke}{rgb}{0.000000,0.000000,0.000000}%
\pgfsetstrokecolor{currentstroke}%
\pgfsetdash{}{0pt}%
\pgfsys@defobject{currentmarker}{\pgfqpoint{0.000000in}{0.000000in}}{\pgfqpoint{0.048611in}{0.000000in}}{%
\pgfpathmoveto{\pgfqpoint{0.000000in}{0.000000in}}%
\pgfpathlineto{\pgfqpoint{0.048611in}{0.000000in}}%
\pgfusepath{stroke,fill}%
}%
\begin{pgfscope}%
\pgfsys@transformshift{3.173964in}{4.031113in}%
\pgfsys@useobject{currentmarker}{}%
\end{pgfscope}%
\end{pgfscope}%
\begin{pgfscope}%
\definecolor{textcolor}{rgb}{0.000000,0.000000,0.000000}%
\pgfsetstrokecolor{textcolor}%
\pgfsetfillcolor{textcolor}%
\pgftext[x=3.271186in,y=3.978352in,left,base]{\color{textcolor}\sffamily\fontsize{10.000000}{12.000000}\selectfont −0.5}%
\end{pgfscope}%
\begin{pgfscope}%
\pgfsetbuttcap%
\pgfsetroundjoin%
\definecolor{currentfill}{rgb}{0.000000,0.000000,0.000000}%
\pgfsetfillcolor{currentfill}%
\pgfsetlinewidth{0.803000pt}%
\definecolor{currentstroke}{rgb}{0.000000,0.000000,0.000000}%
\pgfsetstrokecolor{currentstroke}%
\pgfsetdash{}{0pt}%
\pgfsys@defobject{currentmarker}{\pgfqpoint{0.000000in}{0.000000in}}{\pgfqpoint{0.048611in}{0.000000in}}{%
\pgfpathmoveto{\pgfqpoint{0.000000in}{0.000000in}}%
\pgfpathlineto{\pgfqpoint{0.048611in}{0.000000in}}%
\pgfusepath{stroke,fill}%
}%
\begin{pgfscope}%
\pgfsys@transformshift{3.173964in}{4.619076in}%
\pgfsys@useobject{currentmarker}{}%
\end{pgfscope}%
\end{pgfscope}%
\begin{pgfscope}%
\definecolor{textcolor}{rgb}{0.000000,0.000000,0.000000}%
\pgfsetstrokecolor{textcolor}%
\pgfsetfillcolor{textcolor}%
\pgftext[x=3.271186in,y=4.566314in,left,base]{\color{textcolor}\sffamily\fontsize{10.000000}{12.000000}\selectfont 0.0}%
\end{pgfscope}%
\begin{pgfscope}%
\pgfsetbuttcap%
\pgfsetroundjoin%
\definecolor{currentfill}{rgb}{0.000000,0.000000,0.000000}%
\pgfsetfillcolor{currentfill}%
\pgfsetlinewidth{0.803000pt}%
\definecolor{currentstroke}{rgb}{0.000000,0.000000,0.000000}%
\pgfsetstrokecolor{currentstroke}%
\pgfsetdash{}{0pt}%
\pgfsys@defobject{currentmarker}{\pgfqpoint{0.000000in}{0.000000in}}{\pgfqpoint{0.048611in}{0.000000in}}{%
\pgfpathmoveto{\pgfqpoint{0.000000in}{0.000000in}}%
\pgfpathlineto{\pgfqpoint{0.048611in}{0.000000in}}%
\pgfusepath{stroke,fill}%
}%
\begin{pgfscope}%
\pgfsys@transformshift{3.173964in}{5.207039in}%
\pgfsys@useobject{currentmarker}{}%
\end{pgfscope}%
\end{pgfscope}%
\begin{pgfscope}%
\definecolor{textcolor}{rgb}{0.000000,0.000000,0.000000}%
\pgfsetstrokecolor{textcolor}%
\pgfsetfillcolor{textcolor}%
\pgftext[x=3.271186in,y=5.154277in,left,base]{\color{textcolor}\sffamily\fontsize{10.000000}{12.000000}\selectfont 0.5}%
\end{pgfscope}%
\begin{pgfscope}%
\pgfsetbuttcap%
\pgfsetroundjoin%
\definecolor{currentfill}{rgb}{0.000000,0.000000,0.000000}%
\pgfsetfillcolor{currentfill}%
\pgfsetlinewidth{0.803000pt}%
\definecolor{currentstroke}{rgb}{0.000000,0.000000,0.000000}%
\pgfsetstrokecolor{currentstroke}%
\pgfsetdash{}{0pt}%
\pgfsys@defobject{currentmarker}{\pgfqpoint{0.000000in}{0.000000in}}{\pgfqpoint{0.048611in}{0.000000in}}{%
\pgfpathmoveto{\pgfqpoint{0.000000in}{0.000000in}}%
\pgfpathlineto{\pgfqpoint{0.048611in}{0.000000in}}%
\pgfusepath{stroke,fill}%
}%
\begin{pgfscope}%
\pgfsys@transformshift{3.173964in}{5.795001in}%
\pgfsys@useobject{currentmarker}{}%
\end{pgfscope}%
\end{pgfscope}%
\begin{pgfscope}%
\definecolor{textcolor}{rgb}{0.000000,0.000000,0.000000}%
\pgfsetstrokecolor{textcolor}%
\pgfsetfillcolor{textcolor}%
\pgftext[x=3.271186in,y=5.742240in,left,base]{\color{textcolor}\sffamily\fontsize{10.000000}{12.000000}\selectfont 1.0}%
\end{pgfscope}%
\begin{pgfscope}%
\pgfsetbuttcap%
\pgfsetmiterjoin%
\pgfsetlinewidth{0.803000pt}%
\definecolor{currentstroke}{rgb}{0.000000,0.000000,0.000000}%
\pgfsetstrokecolor{currentstroke}%
\pgfsetdash{}{0pt}%
\pgfpathmoveto{\pgfqpoint{3.056370in}{3.443136in}}%
\pgfpathlineto{\pgfqpoint{3.056370in}{3.452323in}}%
\pgfpathlineto{\pgfqpoint{3.056370in}{5.785829in}}%
\pgfpathlineto{\pgfqpoint{3.056370in}{5.795016in}}%
\pgfpathlineto{\pgfqpoint{3.173964in}{5.795016in}}%
\pgfpathlineto{\pgfqpoint{3.173964in}{5.785829in}}%
\pgfpathlineto{\pgfqpoint{3.173964in}{3.452323in}}%
\pgfpathlineto{\pgfqpoint{3.173964in}{3.443136in}}%
\pgfpathclose%
\pgfusepath{stroke}%
\end{pgfscope}%
\begin{pgfscope}%
\pgfsetbuttcap%
\pgfsetmiterjoin%
\definecolor{currentfill}{rgb}{1.000000,1.000000,1.000000}%
\pgfsetfillcolor{currentfill}%
\pgfsetlinewidth{0.000000pt}%
\definecolor{currentstroke}{rgb}{0.000000,0.000000,0.000000}%
\pgfsetstrokecolor{currentstroke}%
\pgfsetstrokeopacity{0.000000}%
\pgfsetdash{}{0pt}%
\pgfpathmoveto{\pgfqpoint{4.555278in}{3.518562in}}%
\pgfpathlineto{\pgfqpoint{6.756306in}{3.518562in}}%
\pgfpathlineto{\pgfqpoint{6.756306in}{5.719590in}}%
\pgfpathlineto{\pgfqpoint{4.555278in}{5.719590in}}%
\pgfpathclose%
\pgfusepath{fill}%
\end{pgfscope}%
\begin{pgfscope}%
\pgfpathrectangle{\pgfqpoint{4.555278in}{3.518562in}}{\pgfqpoint{2.201028in}{2.201028in}}%
\pgfusepath{clip}%
\pgfsys@transformshift{4.555278in}{3.518562in}%
\pgftext[left,bottom]{\pgfimage[interpolate=true,width=2.210000in,height=2.210000in]{Figure2-img2.png}}%
\end{pgfscope}%
\begin{pgfscope}%
\pgfsetbuttcap%
\pgfsetroundjoin%
\definecolor{currentfill}{rgb}{0.000000,0.000000,0.000000}%
\pgfsetfillcolor{currentfill}%
\pgfsetlinewidth{0.803000pt}%
\definecolor{currentstroke}{rgb}{0.000000,0.000000,0.000000}%
\pgfsetstrokecolor{currentstroke}%
\pgfsetdash{}{0pt}%
\pgfsys@defobject{currentmarker}{\pgfqpoint{0.000000in}{-0.048611in}}{\pgfqpoint{0.000000in}{0.000000in}}{%
\pgfpathmoveto{\pgfqpoint{0.000000in}{0.000000in}}%
\pgfpathlineto{\pgfqpoint{0.000000in}{-0.048611in}}%
\pgfusepath{stroke,fill}%
}%
\begin{pgfscope}%
\pgfsys@transformshift{4.559560in}{3.518562in}%
\pgfsys@useobject{currentmarker}{}%
\end{pgfscope}%
\end{pgfscope}%
\begin{pgfscope}%
\definecolor{textcolor}{rgb}{0.000000,0.000000,0.000000}%
\pgfsetstrokecolor{textcolor}%
\pgfsetfillcolor{textcolor}%
\pgftext[x=4.559560in,y=3.421340in,,top]{\color{textcolor}\sffamily\fontsize{10.000000}{12.000000}\selectfont 0}%
\end{pgfscope}%
\begin{pgfscope}%
\pgfsetbuttcap%
\pgfsetroundjoin%
\definecolor{currentfill}{rgb}{0.000000,0.000000,0.000000}%
\pgfsetfillcolor{currentfill}%
\pgfsetlinewidth{0.803000pt}%
\definecolor{currentstroke}{rgb}{0.000000,0.000000,0.000000}%
\pgfsetstrokecolor{currentstroke}%
\pgfsetdash{}{0pt}%
\pgfsys@defobject{currentmarker}{\pgfqpoint{0.000000in}{-0.048611in}}{\pgfqpoint{0.000000in}{0.000000in}}{%
\pgfpathmoveto{\pgfqpoint{0.000000in}{0.000000in}}%
\pgfpathlineto{\pgfqpoint{0.000000in}{-0.048611in}}%
\pgfusepath{stroke,fill}%
}%
\begin{pgfscope}%
\pgfsys@transformshift{5.415991in}{3.518562in}%
\pgfsys@useobject{currentmarker}{}%
\end{pgfscope}%
\end{pgfscope}%
\begin{pgfscope}%
\definecolor{textcolor}{rgb}{0.000000,0.000000,0.000000}%
\pgfsetstrokecolor{textcolor}%
\pgfsetfillcolor{textcolor}%
\pgftext[x=5.415991in,y=3.421340in,,top]{\color{textcolor}\sffamily\fontsize{10.000000}{12.000000}\selectfont 100}%
\end{pgfscope}%
\begin{pgfscope}%
\pgfsetbuttcap%
\pgfsetroundjoin%
\definecolor{currentfill}{rgb}{0.000000,0.000000,0.000000}%
\pgfsetfillcolor{currentfill}%
\pgfsetlinewidth{0.803000pt}%
\definecolor{currentstroke}{rgb}{0.000000,0.000000,0.000000}%
\pgfsetstrokecolor{currentstroke}%
\pgfsetdash{}{0pt}%
\pgfsys@defobject{currentmarker}{\pgfqpoint{0.000000in}{-0.048611in}}{\pgfqpoint{0.000000in}{0.000000in}}{%
\pgfpathmoveto{\pgfqpoint{0.000000in}{0.000000in}}%
\pgfpathlineto{\pgfqpoint{0.000000in}{-0.048611in}}%
\pgfusepath{stroke,fill}%
}%
\begin{pgfscope}%
\pgfsys@transformshift{6.272422in}{3.518562in}%
\pgfsys@useobject{currentmarker}{}%
\end{pgfscope}%
\end{pgfscope}%
\begin{pgfscope}%
\definecolor{textcolor}{rgb}{0.000000,0.000000,0.000000}%
\pgfsetstrokecolor{textcolor}%
\pgfsetfillcolor{textcolor}%
\pgftext[x=6.272422in,y=3.421340in,,top]{\color{textcolor}\sffamily\fontsize{10.000000}{12.000000}\selectfont 200}%
\end{pgfscope}%
\begin{pgfscope}%
\definecolor{textcolor}{rgb}{0.000000,0.000000,0.000000}%
\pgfsetstrokecolor{textcolor}%
\pgfsetfillcolor{textcolor}%
\pgftext[x=5.655792in,y=3.231371in,,top]{\color{textcolor}\sffamily\fontsize{10.000000}{12.000000}\selectfont \(\displaystyle x\)}%
\end{pgfscope}%
\begin{pgfscope}%
\pgfsetbuttcap%
\pgfsetroundjoin%
\definecolor{currentfill}{rgb}{0.000000,0.000000,0.000000}%
\pgfsetfillcolor{currentfill}%
\pgfsetlinewidth{0.803000pt}%
\definecolor{currentstroke}{rgb}{0.000000,0.000000,0.000000}%
\pgfsetstrokecolor{currentstroke}%
\pgfsetdash{}{0pt}%
\pgfsys@defobject{currentmarker}{\pgfqpoint{-0.048611in}{0.000000in}}{\pgfqpoint{0.000000in}{0.000000in}}{%
\pgfpathmoveto{\pgfqpoint{0.000000in}{0.000000in}}%
\pgfpathlineto{\pgfqpoint{-0.048611in}{0.000000in}}%
\pgfusepath{stroke,fill}%
}%
\begin{pgfscope}%
\pgfsys@transformshift{4.555278in}{3.522844in}%
\pgfsys@useobject{currentmarker}{}%
\end{pgfscope}%
\end{pgfscope}%
\begin{pgfscope}%
\definecolor{textcolor}{rgb}{0.000000,0.000000,0.000000}%
\pgfsetstrokecolor{textcolor}%
\pgfsetfillcolor{textcolor}%
\pgftext[x=4.369690in,y=3.470083in,left,base]{\color{textcolor}\sffamily\fontsize{10.000000}{12.000000}\selectfont 0}%
\end{pgfscope}%
\begin{pgfscope}%
\pgfsetbuttcap%
\pgfsetroundjoin%
\definecolor{currentfill}{rgb}{0.000000,0.000000,0.000000}%
\pgfsetfillcolor{currentfill}%
\pgfsetlinewidth{0.803000pt}%
\definecolor{currentstroke}{rgb}{0.000000,0.000000,0.000000}%
\pgfsetstrokecolor{currentstroke}%
\pgfsetdash{}{0pt}%
\pgfsys@defobject{currentmarker}{\pgfqpoint{-0.048611in}{0.000000in}}{\pgfqpoint{0.000000in}{0.000000in}}{%
\pgfpathmoveto{\pgfqpoint{0.000000in}{0.000000in}}%
\pgfpathlineto{\pgfqpoint{-0.048611in}{0.000000in}}%
\pgfusepath{stroke,fill}%
}%
\begin{pgfscope}%
\pgfsys@transformshift{4.555278in}{3.951060in}%
\pgfsys@useobject{currentmarker}{}%
\end{pgfscope}%
\end{pgfscope}%
\begin{pgfscope}%
\definecolor{textcolor}{rgb}{0.000000,0.000000,0.000000}%
\pgfsetstrokecolor{textcolor}%
\pgfsetfillcolor{textcolor}%
\pgftext[x=4.281325in,y=3.898298in,left,base]{\color{textcolor}\sffamily\fontsize{10.000000}{12.000000}\selectfont 50}%
\end{pgfscope}%
\begin{pgfscope}%
\pgfsetbuttcap%
\pgfsetroundjoin%
\definecolor{currentfill}{rgb}{0.000000,0.000000,0.000000}%
\pgfsetfillcolor{currentfill}%
\pgfsetlinewidth{0.803000pt}%
\definecolor{currentstroke}{rgb}{0.000000,0.000000,0.000000}%
\pgfsetstrokecolor{currentstroke}%
\pgfsetdash{}{0pt}%
\pgfsys@defobject{currentmarker}{\pgfqpoint{-0.048611in}{0.000000in}}{\pgfqpoint{0.000000in}{0.000000in}}{%
\pgfpathmoveto{\pgfqpoint{0.000000in}{0.000000in}}%
\pgfpathlineto{\pgfqpoint{-0.048611in}{0.000000in}}%
\pgfusepath{stroke,fill}%
}%
\begin{pgfscope}%
\pgfsys@transformshift{4.555278in}{4.379275in}%
\pgfsys@useobject{currentmarker}{}%
\end{pgfscope}%
\end{pgfscope}%
\begin{pgfscope}%
\definecolor{textcolor}{rgb}{0.000000,0.000000,0.000000}%
\pgfsetstrokecolor{textcolor}%
\pgfsetfillcolor{textcolor}%
\pgftext[x=4.192960in,y=4.326514in,left,base]{\color{textcolor}\sffamily\fontsize{10.000000}{12.000000}\selectfont 100}%
\end{pgfscope}%
\begin{pgfscope}%
\pgfsetbuttcap%
\pgfsetroundjoin%
\definecolor{currentfill}{rgb}{0.000000,0.000000,0.000000}%
\pgfsetfillcolor{currentfill}%
\pgfsetlinewidth{0.803000pt}%
\definecolor{currentstroke}{rgb}{0.000000,0.000000,0.000000}%
\pgfsetstrokecolor{currentstroke}%
\pgfsetdash{}{0pt}%
\pgfsys@defobject{currentmarker}{\pgfqpoint{-0.048611in}{0.000000in}}{\pgfqpoint{0.000000in}{0.000000in}}{%
\pgfpathmoveto{\pgfqpoint{0.000000in}{0.000000in}}%
\pgfpathlineto{\pgfqpoint{-0.048611in}{0.000000in}}%
\pgfusepath{stroke,fill}%
}%
\begin{pgfscope}%
\pgfsys@transformshift{4.555278in}{4.807491in}%
\pgfsys@useobject{currentmarker}{}%
\end{pgfscope}%
\end{pgfscope}%
\begin{pgfscope}%
\definecolor{textcolor}{rgb}{0.000000,0.000000,0.000000}%
\pgfsetstrokecolor{textcolor}%
\pgfsetfillcolor{textcolor}%
\pgftext[x=4.192960in,y=4.754729in,left,base]{\color{textcolor}\sffamily\fontsize{10.000000}{12.000000}\selectfont 150}%
\end{pgfscope}%
\begin{pgfscope}%
\pgfsetbuttcap%
\pgfsetroundjoin%
\definecolor{currentfill}{rgb}{0.000000,0.000000,0.000000}%
\pgfsetfillcolor{currentfill}%
\pgfsetlinewidth{0.803000pt}%
\definecolor{currentstroke}{rgb}{0.000000,0.000000,0.000000}%
\pgfsetstrokecolor{currentstroke}%
\pgfsetdash{}{0pt}%
\pgfsys@defobject{currentmarker}{\pgfqpoint{-0.048611in}{0.000000in}}{\pgfqpoint{0.000000in}{0.000000in}}{%
\pgfpathmoveto{\pgfqpoint{0.000000in}{0.000000in}}%
\pgfpathlineto{\pgfqpoint{-0.048611in}{0.000000in}}%
\pgfusepath{stroke,fill}%
}%
\begin{pgfscope}%
\pgfsys@transformshift{4.555278in}{5.235706in}%
\pgfsys@useobject{currentmarker}{}%
\end{pgfscope}%
\end{pgfscope}%
\begin{pgfscope}%
\definecolor{textcolor}{rgb}{0.000000,0.000000,0.000000}%
\pgfsetstrokecolor{textcolor}%
\pgfsetfillcolor{textcolor}%
\pgftext[x=4.192960in,y=5.182945in,left,base]{\color{textcolor}\sffamily\fontsize{10.000000}{12.000000}\selectfont 200}%
\end{pgfscope}%
\begin{pgfscope}%
\pgfsetbuttcap%
\pgfsetroundjoin%
\definecolor{currentfill}{rgb}{0.000000,0.000000,0.000000}%
\pgfsetfillcolor{currentfill}%
\pgfsetlinewidth{0.803000pt}%
\definecolor{currentstroke}{rgb}{0.000000,0.000000,0.000000}%
\pgfsetstrokecolor{currentstroke}%
\pgfsetdash{}{0pt}%
\pgfsys@defobject{currentmarker}{\pgfqpoint{-0.048611in}{0.000000in}}{\pgfqpoint{0.000000in}{0.000000in}}{%
\pgfpathmoveto{\pgfqpoint{0.000000in}{0.000000in}}%
\pgfpathlineto{\pgfqpoint{-0.048611in}{0.000000in}}%
\pgfusepath{stroke,fill}%
}%
\begin{pgfscope}%
\pgfsys@transformshift{4.555278in}{5.663922in}%
\pgfsys@useobject{currentmarker}{}%
\end{pgfscope}%
\end{pgfscope}%
\begin{pgfscope}%
\definecolor{textcolor}{rgb}{0.000000,0.000000,0.000000}%
\pgfsetstrokecolor{textcolor}%
\pgfsetfillcolor{textcolor}%
\pgftext[x=4.192960in,y=5.611160in,left,base]{\color{textcolor}\sffamily\fontsize{10.000000}{12.000000}\selectfont 250}%
\end{pgfscope}%
\begin{pgfscope}%
\definecolor{textcolor}{rgb}{0.000000,0.000000,0.000000}%
\pgfsetstrokecolor{textcolor}%
\pgfsetfillcolor{textcolor}%
\pgftext[x=4.137404in,y=4.619076in,,bottom,rotate=90.000000]{\color{textcolor}\sffamily\fontsize{10.000000}{12.000000}\selectfont \(\displaystyle y\)}%
\end{pgfscope}%
\begin{pgfscope}%
\pgfsetrectcap%
\pgfsetmiterjoin%
\pgfsetlinewidth{0.803000pt}%
\definecolor{currentstroke}{rgb}{0.000000,0.000000,0.000000}%
\pgfsetstrokecolor{currentstroke}%
\pgfsetdash{}{0pt}%
\pgfpathmoveto{\pgfqpoint{4.555278in}{3.518562in}}%
\pgfpathlineto{\pgfqpoint{4.555278in}{5.719590in}}%
\pgfusepath{stroke}%
\end{pgfscope}%
\begin{pgfscope}%
\pgfsetrectcap%
\pgfsetmiterjoin%
\pgfsetlinewidth{0.803000pt}%
\definecolor{currentstroke}{rgb}{0.000000,0.000000,0.000000}%
\pgfsetstrokecolor{currentstroke}%
\pgfsetdash{}{0pt}%
\pgfpathmoveto{\pgfqpoint{6.756306in}{3.518562in}}%
\pgfpathlineto{\pgfqpoint{6.756306in}{5.719590in}}%
\pgfusepath{stroke}%
\end{pgfscope}%
\begin{pgfscope}%
\pgfsetrectcap%
\pgfsetmiterjoin%
\pgfsetlinewidth{0.803000pt}%
\definecolor{currentstroke}{rgb}{0.000000,0.000000,0.000000}%
\pgfsetstrokecolor{currentstroke}%
\pgfsetdash{}{0pt}%
\pgfpathmoveto{\pgfqpoint{4.555278in}{3.518562in}}%
\pgfpathlineto{\pgfqpoint{6.756306in}{3.518562in}}%
\pgfusepath{stroke}%
\end{pgfscope}%
\begin{pgfscope}%
\pgfsetrectcap%
\pgfsetmiterjoin%
\pgfsetlinewidth{0.803000pt}%
\definecolor{currentstroke}{rgb}{0.000000,0.000000,0.000000}%
\pgfsetstrokecolor{currentstroke}%
\pgfsetdash{}{0pt}%
\pgfpathmoveto{\pgfqpoint{4.555278in}{5.719590in}}%
\pgfpathlineto{\pgfqpoint{6.756306in}{5.719590in}}%
\pgfusepath{stroke}%
\end{pgfscope}%
\begin{pgfscope}%
\pgfpathrectangle{\pgfqpoint{6.893870in}{3.443136in}}{\pgfqpoint{0.117594in}{2.351880in}}%
\pgfusepath{clip}%
\pgfsetbuttcap%
\pgfsetmiterjoin%
\definecolor{currentfill}{rgb}{1.000000,1.000000,1.000000}%
\pgfsetfillcolor{currentfill}%
\pgfsetlinewidth{0.010037pt}%
\definecolor{currentstroke}{rgb}{1.000000,1.000000,1.000000}%
\pgfsetstrokecolor{currentstroke}%
\pgfsetdash{}{0pt}%
\pgfpathmoveto{\pgfqpoint{6.893870in}{3.443136in}}%
\pgfpathlineto{\pgfqpoint{6.893870in}{3.452323in}}%
\pgfpathlineto{\pgfqpoint{6.893870in}{5.785829in}}%
\pgfpathlineto{\pgfqpoint{6.893870in}{5.795016in}}%
\pgfpathlineto{\pgfqpoint{7.011464in}{5.795016in}}%
\pgfpathlineto{\pgfqpoint{7.011464in}{5.785829in}}%
\pgfpathlineto{\pgfqpoint{7.011464in}{3.452323in}}%
\pgfpathlineto{\pgfqpoint{7.011464in}{3.443136in}}%
\pgfpathclose%
\pgfusepath{stroke,fill}%
\end{pgfscope}%
\begin{pgfscope}%
\pgfsys@transformshift{6.890000in}{3.440000in}%
\pgftext[left,bottom]{\pgfimage[interpolate=true,width=0.120000in,height=2.360000in]{Figure2-img3.png}}%
\end{pgfscope}%
\begin{pgfscope}%
\pgfsetbuttcap%
\pgfsetroundjoin%
\definecolor{currentfill}{rgb}{0.000000,0.000000,0.000000}%
\pgfsetfillcolor{currentfill}%
\pgfsetlinewidth{0.803000pt}%
\definecolor{currentstroke}{rgb}{0.000000,0.000000,0.000000}%
\pgfsetstrokecolor{currentstroke}%
\pgfsetdash{}{0pt}%
\pgfsys@defobject{currentmarker}{\pgfqpoint{0.000000in}{0.000000in}}{\pgfqpoint{0.048611in}{0.000000in}}{%
\pgfpathmoveto{\pgfqpoint{0.000000in}{0.000000in}}%
\pgfpathlineto{\pgfqpoint{0.048611in}{0.000000in}}%
\pgfusepath{stroke,fill}%
}%
\begin{pgfscope}%
\pgfsys@transformshift{7.011464in}{3.682065in}%
\pgfsys@useobject{currentmarker}{}%
\end{pgfscope}%
\end{pgfscope}%
\begin{pgfscope}%
\definecolor{textcolor}{rgb}{0.000000,0.000000,0.000000}%
\pgfsetstrokecolor{textcolor}%
\pgfsetfillcolor{textcolor}%
\pgftext[x=7.108686in,y=3.629303in,left,base]{\color{textcolor}\sffamily\fontsize{10.000000}{12.000000}\selectfont −0.000010}%
\end{pgfscope}%
\begin{pgfscope}%
\pgfsetbuttcap%
\pgfsetroundjoin%
\definecolor{currentfill}{rgb}{0.000000,0.000000,0.000000}%
\pgfsetfillcolor{currentfill}%
\pgfsetlinewidth{0.803000pt}%
\definecolor{currentstroke}{rgb}{0.000000,0.000000,0.000000}%
\pgfsetstrokecolor{currentstroke}%
\pgfsetdash{}{0pt}%
\pgfsys@defobject{currentmarker}{\pgfqpoint{0.000000in}{0.000000in}}{\pgfqpoint{0.048611in}{0.000000in}}{%
\pgfpathmoveto{\pgfqpoint{0.000000in}{0.000000in}}%
\pgfpathlineto{\pgfqpoint{0.048611in}{0.000000in}}%
\pgfusepath{stroke,fill}%
}%
\begin{pgfscope}%
\pgfsys@transformshift{7.011464in}{4.150573in}%
\pgfsys@useobject{currentmarker}{}%
\end{pgfscope}%
\end{pgfscope}%
\begin{pgfscope}%
\definecolor{textcolor}{rgb}{0.000000,0.000000,0.000000}%
\pgfsetstrokecolor{textcolor}%
\pgfsetfillcolor{textcolor}%
\pgftext[x=7.108686in,y=4.097811in,left,base]{\color{textcolor}\sffamily\fontsize{10.000000}{12.000000}\selectfont −0.000005}%
\end{pgfscope}%
\begin{pgfscope}%
\pgfsetbuttcap%
\pgfsetroundjoin%
\definecolor{currentfill}{rgb}{0.000000,0.000000,0.000000}%
\pgfsetfillcolor{currentfill}%
\pgfsetlinewidth{0.803000pt}%
\definecolor{currentstroke}{rgb}{0.000000,0.000000,0.000000}%
\pgfsetstrokecolor{currentstroke}%
\pgfsetdash{}{0pt}%
\pgfsys@defobject{currentmarker}{\pgfqpoint{0.000000in}{0.000000in}}{\pgfqpoint{0.048611in}{0.000000in}}{%
\pgfpathmoveto{\pgfqpoint{0.000000in}{0.000000in}}%
\pgfpathlineto{\pgfqpoint{0.048611in}{0.000000in}}%
\pgfusepath{stroke,fill}%
}%
\begin{pgfscope}%
\pgfsys@transformshift{7.011464in}{4.619081in}%
\pgfsys@useobject{currentmarker}{}%
\end{pgfscope}%
\end{pgfscope}%
\begin{pgfscope}%
\definecolor{textcolor}{rgb}{0.000000,0.000000,0.000000}%
\pgfsetstrokecolor{textcolor}%
\pgfsetfillcolor{textcolor}%
\pgftext[x=7.108686in,y=4.566319in,left,base]{\color{textcolor}\sffamily\fontsize{10.000000}{12.000000}\selectfont 0.000000}%
\end{pgfscope}%
\begin{pgfscope}%
\pgfsetbuttcap%
\pgfsetroundjoin%
\definecolor{currentfill}{rgb}{0.000000,0.000000,0.000000}%
\pgfsetfillcolor{currentfill}%
\pgfsetlinewidth{0.803000pt}%
\definecolor{currentstroke}{rgb}{0.000000,0.000000,0.000000}%
\pgfsetstrokecolor{currentstroke}%
\pgfsetdash{}{0pt}%
\pgfsys@defobject{currentmarker}{\pgfqpoint{0.000000in}{0.000000in}}{\pgfqpoint{0.048611in}{0.000000in}}{%
\pgfpathmoveto{\pgfqpoint{0.000000in}{0.000000in}}%
\pgfpathlineto{\pgfqpoint{0.048611in}{0.000000in}}%
\pgfusepath{stroke,fill}%
}%
\begin{pgfscope}%
\pgfsys@transformshift{7.011464in}{5.087589in}%
\pgfsys@useobject{currentmarker}{}%
\end{pgfscope}%
\end{pgfscope}%
\begin{pgfscope}%
\definecolor{textcolor}{rgb}{0.000000,0.000000,0.000000}%
\pgfsetstrokecolor{textcolor}%
\pgfsetfillcolor{textcolor}%
\pgftext[x=7.108686in,y=5.034827in,left,base]{\color{textcolor}\sffamily\fontsize{10.000000}{12.000000}\selectfont 0.000005}%
\end{pgfscope}%
\begin{pgfscope}%
\pgfsetbuttcap%
\pgfsetroundjoin%
\definecolor{currentfill}{rgb}{0.000000,0.000000,0.000000}%
\pgfsetfillcolor{currentfill}%
\pgfsetlinewidth{0.803000pt}%
\definecolor{currentstroke}{rgb}{0.000000,0.000000,0.000000}%
\pgfsetstrokecolor{currentstroke}%
\pgfsetdash{}{0pt}%
\pgfsys@defobject{currentmarker}{\pgfqpoint{0.000000in}{0.000000in}}{\pgfqpoint{0.048611in}{0.000000in}}{%
\pgfpathmoveto{\pgfqpoint{0.000000in}{0.000000in}}%
\pgfpathlineto{\pgfqpoint{0.048611in}{0.000000in}}%
\pgfusepath{stroke,fill}%
}%
\begin{pgfscope}%
\pgfsys@transformshift{7.011464in}{5.556096in}%
\pgfsys@useobject{currentmarker}{}%
\end{pgfscope}%
\end{pgfscope}%
\begin{pgfscope}%
\definecolor{textcolor}{rgb}{0.000000,0.000000,0.000000}%
\pgfsetstrokecolor{textcolor}%
\pgfsetfillcolor{textcolor}%
\pgftext[x=7.108686in,y=5.503335in,left,base]{\color{textcolor}\sffamily\fontsize{10.000000}{12.000000}\selectfont 0.000010}%
\end{pgfscope}%
\begin{pgfscope}%
\pgfsetbuttcap%
\pgfsetmiterjoin%
\pgfsetlinewidth{0.803000pt}%
\definecolor{currentstroke}{rgb}{0.000000,0.000000,0.000000}%
\pgfsetstrokecolor{currentstroke}%
\pgfsetdash{}{0pt}%
\pgfpathmoveto{\pgfqpoint{6.893870in}{3.443136in}}%
\pgfpathlineto{\pgfqpoint{6.893870in}{3.452323in}}%
\pgfpathlineto{\pgfqpoint{6.893870in}{5.785829in}}%
\pgfpathlineto{\pgfqpoint{6.893870in}{5.795016in}}%
\pgfpathlineto{\pgfqpoint{7.011464in}{5.795016in}}%
\pgfpathlineto{\pgfqpoint{7.011464in}{5.785829in}}%
\pgfpathlineto{\pgfqpoint{7.011464in}{3.452323in}}%
\pgfpathlineto{\pgfqpoint{7.011464in}{3.443136in}}%
\pgfpathclose%
\pgfusepath{stroke}%
\end{pgfscope}%
\begin{pgfscope}%
\pgfsetbuttcap%
\pgfsetmiterjoin%
\definecolor{currentfill}{rgb}{1.000000,1.000000,1.000000}%
\pgfsetfillcolor{currentfill}%
\pgfsetlinewidth{0.000000pt}%
\definecolor{currentstroke}{rgb}{0.000000,0.000000,0.000000}%
\pgfsetstrokecolor{currentstroke}%
\pgfsetstrokeopacity{0.000000}%
\pgfsetdash{}{0pt}%
\pgfpathmoveto{\pgfqpoint{0.717778in}{0.603554in}}%
\pgfpathlineto{\pgfqpoint{2.918806in}{0.603554in}}%
\pgfpathlineto{\pgfqpoint{2.918806in}{2.804582in}}%
\pgfpathlineto{\pgfqpoint{0.717778in}{2.804582in}}%
\pgfpathclose%
\pgfusepath{fill}%
\end{pgfscope}%
\begin{pgfscope}%
\pgfpathrectangle{\pgfqpoint{0.717778in}{0.603554in}}{\pgfqpoint{2.201028in}{2.201028in}}%
\pgfusepath{clip}%
\pgfsys@transformshift{0.717778in}{0.603554in}%
\pgftext[left,bottom]{\pgfimage[interpolate=true,width=2.210000in,height=2.210000in]{Figure2-img4.png}}%
\end{pgfscope}%
\begin{pgfscope}%
\pgfsetbuttcap%
\pgfsetroundjoin%
\definecolor{currentfill}{rgb}{0.000000,0.000000,0.000000}%
\pgfsetfillcolor{currentfill}%
\pgfsetlinewidth{0.803000pt}%
\definecolor{currentstroke}{rgb}{0.000000,0.000000,0.000000}%
\pgfsetstrokecolor{currentstroke}%
\pgfsetdash{}{0pt}%
\pgfsys@defobject{currentmarker}{\pgfqpoint{0.000000in}{-0.048611in}}{\pgfqpoint{0.000000in}{0.000000in}}{%
\pgfpathmoveto{\pgfqpoint{0.000000in}{0.000000in}}%
\pgfpathlineto{\pgfqpoint{0.000000in}{-0.048611in}}%
\pgfusepath{stroke,fill}%
}%
\begin{pgfscope}%
\pgfsys@transformshift{0.719923in}{0.603554in}%
\pgfsys@useobject{currentmarker}{}%
\end{pgfscope}%
\end{pgfscope}%
\begin{pgfscope}%
\definecolor{textcolor}{rgb}{0.000000,0.000000,0.000000}%
\pgfsetstrokecolor{textcolor}%
\pgfsetfillcolor{textcolor}%
\pgftext[x=0.719923in,y=0.506332in,,top]{\color{textcolor}\sffamily\fontsize{10.000000}{12.000000}\selectfont 0}%
\end{pgfscope}%
\begin{pgfscope}%
\pgfsetbuttcap%
\pgfsetroundjoin%
\definecolor{currentfill}{rgb}{0.000000,0.000000,0.000000}%
\pgfsetfillcolor{currentfill}%
\pgfsetlinewidth{0.803000pt}%
\definecolor{currentstroke}{rgb}{0.000000,0.000000,0.000000}%
\pgfsetstrokecolor{currentstroke}%
\pgfsetdash{}{0pt}%
\pgfsys@defobject{currentmarker}{\pgfqpoint{0.000000in}{-0.048611in}}{\pgfqpoint{0.000000in}{0.000000in}}{%
\pgfpathmoveto{\pgfqpoint{0.000000in}{0.000000in}}%
\pgfpathlineto{\pgfqpoint{0.000000in}{-0.048611in}}%
\pgfusepath{stroke,fill}%
}%
\begin{pgfscope}%
\pgfsys@transformshift{1.578024in}{0.603554in}%
\pgfsys@useobject{currentmarker}{}%
\end{pgfscope}%
\end{pgfscope}%
\begin{pgfscope}%
\definecolor{textcolor}{rgb}{0.000000,0.000000,0.000000}%
\pgfsetstrokecolor{textcolor}%
\pgfsetfillcolor{textcolor}%
\pgftext[x=1.578024in,y=0.506332in,,top]{\color{textcolor}\sffamily\fontsize{10.000000}{12.000000}\selectfont 200}%
\end{pgfscope}%
\begin{pgfscope}%
\pgfsetbuttcap%
\pgfsetroundjoin%
\definecolor{currentfill}{rgb}{0.000000,0.000000,0.000000}%
\pgfsetfillcolor{currentfill}%
\pgfsetlinewidth{0.803000pt}%
\definecolor{currentstroke}{rgb}{0.000000,0.000000,0.000000}%
\pgfsetstrokecolor{currentstroke}%
\pgfsetdash{}{0pt}%
\pgfsys@defobject{currentmarker}{\pgfqpoint{0.000000in}{-0.048611in}}{\pgfqpoint{0.000000in}{0.000000in}}{%
\pgfpathmoveto{\pgfqpoint{0.000000in}{0.000000in}}%
\pgfpathlineto{\pgfqpoint{0.000000in}{-0.048611in}}%
\pgfusepath{stroke,fill}%
}%
\begin{pgfscope}%
\pgfsys@transformshift{2.436124in}{0.603554in}%
\pgfsys@useobject{currentmarker}{}%
\end{pgfscope}%
\end{pgfscope}%
\begin{pgfscope}%
\definecolor{textcolor}{rgb}{0.000000,0.000000,0.000000}%
\pgfsetstrokecolor{textcolor}%
\pgfsetfillcolor{textcolor}%
\pgftext[x=2.436124in,y=0.506332in,,top]{\color{textcolor}\sffamily\fontsize{10.000000}{12.000000}\selectfont 400}%
\end{pgfscope}%
\begin{pgfscope}%
\definecolor{textcolor}{rgb}{0.000000,0.000000,0.000000}%
\pgfsetstrokecolor{textcolor}%
\pgfsetfillcolor{textcolor}%
\pgftext[x=1.818292in,y=0.316363in,,top]{\color{textcolor}\sffamily\fontsize{10.000000}{12.000000}\selectfont \(\displaystyle x\)}%
\end{pgfscope}%
\begin{pgfscope}%
\pgfsetbuttcap%
\pgfsetroundjoin%
\definecolor{currentfill}{rgb}{0.000000,0.000000,0.000000}%
\pgfsetfillcolor{currentfill}%
\pgfsetlinewidth{0.803000pt}%
\definecolor{currentstroke}{rgb}{0.000000,0.000000,0.000000}%
\pgfsetstrokecolor{currentstroke}%
\pgfsetdash{}{0pt}%
\pgfsys@defobject{currentmarker}{\pgfqpoint{-0.048611in}{0.000000in}}{\pgfqpoint{0.000000in}{0.000000in}}{%
\pgfpathmoveto{\pgfqpoint{0.000000in}{0.000000in}}%
\pgfpathlineto{\pgfqpoint{-0.048611in}{0.000000in}}%
\pgfusepath{stroke,fill}%
}%
\begin{pgfscope}%
\pgfsys@transformshift{0.717778in}{0.605699in}%
\pgfsys@useobject{currentmarker}{}%
\end{pgfscope}%
\end{pgfscope}%
\begin{pgfscope}%
\definecolor{textcolor}{rgb}{0.000000,0.000000,0.000000}%
\pgfsetstrokecolor{textcolor}%
\pgfsetfillcolor{textcolor}%
\pgftext[x=0.532190in,y=0.552938in,left,base]{\color{textcolor}\sffamily\fontsize{10.000000}{12.000000}\selectfont 0}%
\end{pgfscope}%
\begin{pgfscope}%
\pgfsetbuttcap%
\pgfsetroundjoin%
\definecolor{currentfill}{rgb}{0.000000,0.000000,0.000000}%
\pgfsetfillcolor{currentfill}%
\pgfsetlinewidth{0.803000pt}%
\definecolor{currentstroke}{rgb}{0.000000,0.000000,0.000000}%
\pgfsetstrokecolor{currentstroke}%
\pgfsetdash{}{0pt}%
\pgfsys@defobject{currentmarker}{\pgfqpoint{-0.048611in}{0.000000in}}{\pgfqpoint{0.000000in}{0.000000in}}{%
\pgfpathmoveto{\pgfqpoint{0.000000in}{0.000000in}}%
\pgfpathlineto{\pgfqpoint{-0.048611in}{0.000000in}}%
\pgfusepath{stroke,fill}%
}%
\begin{pgfscope}%
\pgfsys@transformshift{0.717778in}{1.034750in}%
\pgfsys@useobject{currentmarker}{}%
\end{pgfscope}%
\end{pgfscope}%
\begin{pgfscope}%
\definecolor{textcolor}{rgb}{0.000000,0.000000,0.000000}%
\pgfsetstrokecolor{textcolor}%
\pgfsetfillcolor{textcolor}%
\pgftext[x=0.355460in,y=0.981988in,left,base]{\color{textcolor}\sffamily\fontsize{10.000000}{12.000000}\selectfont 100}%
\end{pgfscope}%
\begin{pgfscope}%
\pgfsetbuttcap%
\pgfsetroundjoin%
\definecolor{currentfill}{rgb}{0.000000,0.000000,0.000000}%
\pgfsetfillcolor{currentfill}%
\pgfsetlinewidth{0.803000pt}%
\definecolor{currentstroke}{rgb}{0.000000,0.000000,0.000000}%
\pgfsetstrokecolor{currentstroke}%
\pgfsetdash{}{0pt}%
\pgfsys@defobject{currentmarker}{\pgfqpoint{-0.048611in}{0.000000in}}{\pgfqpoint{0.000000in}{0.000000in}}{%
\pgfpathmoveto{\pgfqpoint{0.000000in}{0.000000in}}%
\pgfpathlineto{\pgfqpoint{-0.048611in}{0.000000in}}%
\pgfusepath{stroke,fill}%
}%
\begin{pgfscope}%
\pgfsys@transformshift{0.717778in}{1.463800in}%
\pgfsys@useobject{currentmarker}{}%
\end{pgfscope}%
\end{pgfscope}%
\begin{pgfscope}%
\definecolor{textcolor}{rgb}{0.000000,0.000000,0.000000}%
\pgfsetstrokecolor{textcolor}%
\pgfsetfillcolor{textcolor}%
\pgftext[x=0.355460in,y=1.411038in,left,base]{\color{textcolor}\sffamily\fontsize{10.000000}{12.000000}\selectfont 200}%
\end{pgfscope}%
\begin{pgfscope}%
\pgfsetbuttcap%
\pgfsetroundjoin%
\definecolor{currentfill}{rgb}{0.000000,0.000000,0.000000}%
\pgfsetfillcolor{currentfill}%
\pgfsetlinewidth{0.803000pt}%
\definecolor{currentstroke}{rgb}{0.000000,0.000000,0.000000}%
\pgfsetstrokecolor{currentstroke}%
\pgfsetdash{}{0pt}%
\pgfsys@defobject{currentmarker}{\pgfqpoint{-0.048611in}{0.000000in}}{\pgfqpoint{0.000000in}{0.000000in}}{%
\pgfpathmoveto{\pgfqpoint{0.000000in}{0.000000in}}%
\pgfpathlineto{\pgfqpoint{-0.048611in}{0.000000in}}%
\pgfusepath{stroke,fill}%
}%
\begin{pgfscope}%
\pgfsys@transformshift{0.717778in}{1.892850in}%
\pgfsys@useobject{currentmarker}{}%
\end{pgfscope}%
\end{pgfscope}%
\begin{pgfscope}%
\definecolor{textcolor}{rgb}{0.000000,0.000000,0.000000}%
\pgfsetstrokecolor{textcolor}%
\pgfsetfillcolor{textcolor}%
\pgftext[x=0.355460in,y=1.840089in,left,base]{\color{textcolor}\sffamily\fontsize{10.000000}{12.000000}\selectfont 300}%
\end{pgfscope}%
\begin{pgfscope}%
\pgfsetbuttcap%
\pgfsetroundjoin%
\definecolor{currentfill}{rgb}{0.000000,0.000000,0.000000}%
\pgfsetfillcolor{currentfill}%
\pgfsetlinewidth{0.803000pt}%
\definecolor{currentstroke}{rgb}{0.000000,0.000000,0.000000}%
\pgfsetstrokecolor{currentstroke}%
\pgfsetdash{}{0pt}%
\pgfsys@defobject{currentmarker}{\pgfqpoint{-0.048611in}{0.000000in}}{\pgfqpoint{0.000000in}{0.000000in}}{%
\pgfpathmoveto{\pgfqpoint{0.000000in}{0.000000in}}%
\pgfpathlineto{\pgfqpoint{-0.048611in}{0.000000in}}%
\pgfusepath{stroke,fill}%
}%
\begin{pgfscope}%
\pgfsys@transformshift{0.717778in}{2.321900in}%
\pgfsys@useobject{currentmarker}{}%
\end{pgfscope}%
\end{pgfscope}%
\begin{pgfscope}%
\definecolor{textcolor}{rgb}{0.000000,0.000000,0.000000}%
\pgfsetstrokecolor{textcolor}%
\pgfsetfillcolor{textcolor}%
\pgftext[x=0.355460in,y=2.269139in,left,base]{\color{textcolor}\sffamily\fontsize{10.000000}{12.000000}\selectfont 400}%
\end{pgfscope}%
\begin{pgfscope}%
\pgfsetbuttcap%
\pgfsetroundjoin%
\definecolor{currentfill}{rgb}{0.000000,0.000000,0.000000}%
\pgfsetfillcolor{currentfill}%
\pgfsetlinewidth{0.803000pt}%
\definecolor{currentstroke}{rgb}{0.000000,0.000000,0.000000}%
\pgfsetstrokecolor{currentstroke}%
\pgfsetdash{}{0pt}%
\pgfsys@defobject{currentmarker}{\pgfqpoint{-0.048611in}{0.000000in}}{\pgfqpoint{0.000000in}{0.000000in}}{%
\pgfpathmoveto{\pgfqpoint{0.000000in}{0.000000in}}%
\pgfpathlineto{\pgfqpoint{-0.048611in}{0.000000in}}%
\pgfusepath{stroke,fill}%
}%
\begin{pgfscope}%
\pgfsys@transformshift{0.717778in}{2.750951in}%
\pgfsys@useobject{currentmarker}{}%
\end{pgfscope}%
\end{pgfscope}%
\begin{pgfscope}%
\definecolor{textcolor}{rgb}{0.000000,0.000000,0.000000}%
\pgfsetstrokecolor{textcolor}%
\pgfsetfillcolor{textcolor}%
\pgftext[x=0.355460in,y=2.698189in,left,base]{\color{textcolor}\sffamily\fontsize{10.000000}{12.000000}\selectfont 500}%
\end{pgfscope}%
\begin{pgfscope}%
\definecolor{textcolor}{rgb}{0.000000,0.000000,0.000000}%
\pgfsetstrokecolor{textcolor}%
\pgfsetfillcolor{textcolor}%
\pgftext[x=0.299904in,y=1.704068in,,bottom,rotate=90.000000]{\color{textcolor}\sffamily\fontsize{10.000000}{12.000000}\selectfont \(\displaystyle y\)}%
\end{pgfscope}%
\begin{pgfscope}%
\pgfsetrectcap%
\pgfsetmiterjoin%
\pgfsetlinewidth{0.803000pt}%
\definecolor{currentstroke}{rgb}{0.000000,0.000000,0.000000}%
\pgfsetstrokecolor{currentstroke}%
\pgfsetdash{}{0pt}%
\pgfpathmoveto{\pgfqpoint{0.717778in}{0.603554in}}%
\pgfpathlineto{\pgfqpoint{0.717778in}{2.804582in}}%
\pgfusepath{stroke}%
\end{pgfscope}%
\begin{pgfscope}%
\pgfsetrectcap%
\pgfsetmiterjoin%
\pgfsetlinewidth{0.803000pt}%
\definecolor{currentstroke}{rgb}{0.000000,0.000000,0.000000}%
\pgfsetstrokecolor{currentstroke}%
\pgfsetdash{}{0pt}%
\pgfpathmoveto{\pgfqpoint{2.918806in}{0.603554in}}%
\pgfpathlineto{\pgfqpoint{2.918806in}{2.804582in}}%
\pgfusepath{stroke}%
\end{pgfscope}%
\begin{pgfscope}%
\pgfsetrectcap%
\pgfsetmiterjoin%
\pgfsetlinewidth{0.803000pt}%
\definecolor{currentstroke}{rgb}{0.000000,0.000000,0.000000}%
\pgfsetstrokecolor{currentstroke}%
\pgfsetdash{}{0pt}%
\pgfpathmoveto{\pgfqpoint{0.717778in}{0.603554in}}%
\pgfpathlineto{\pgfqpoint{2.918806in}{0.603554in}}%
\pgfusepath{stroke}%
\end{pgfscope}%
\begin{pgfscope}%
\pgfsetrectcap%
\pgfsetmiterjoin%
\pgfsetlinewidth{0.803000pt}%
\definecolor{currentstroke}{rgb}{0.000000,0.000000,0.000000}%
\pgfsetstrokecolor{currentstroke}%
\pgfsetdash{}{0pt}%
\pgfpathmoveto{\pgfqpoint{0.717778in}{2.804582in}}%
\pgfpathlineto{\pgfqpoint{2.918806in}{2.804582in}}%
\pgfusepath{stroke}%
\end{pgfscope}%
\begin{pgfscope}%
\pgfpathrectangle{\pgfqpoint{3.056370in}{0.528128in}}{\pgfqpoint{0.117594in}{2.351880in}}%
\pgfusepath{clip}%
\pgfsetbuttcap%
\pgfsetmiterjoin%
\definecolor{currentfill}{rgb}{1.000000,1.000000,1.000000}%
\pgfsetfillcolor{currentfill}%
\pgfsetlinewidth{0.010037pt}%
\definecolor{currentstroke}{rgb}{1.000000,1.000000,1.000000}%
\pgfsetstrokecolor{currentstroke}%
\pgfsetdash{}{0pt}%
\pgfpathmoveto{\pgfqpoint{3.056370in}{0.528128in}}%
\pgfpathlineto{\pgfqpoint{3.056370in}{0.537315in}}%
\pgfpathlineto{\pgfqpoint{3.056370in}{2.870821in}}%
\pgfpathlineto{\pgfqpoint{3.056370in}{2.880008in}}%
\pgfpathlineto{\pgfqpoint{3.173964in}{2.880008in}}%
\pgfpathlineto{\pgfqpoint{3.173964in}{2.870821in}}%
\pgfpathlineto{\pgfqpoint{3.173964in}{0.537315in}}%
\pgfpathlineto{\pgfqpoint{3.173964in}{0.528128in}}%
\pgfpathclose%
\pgfusepath{stroke,fill}%
\end{pgfscope}%
\begin{pgfscope}%
\pgfsys@transformshift{3.060000in}{0.530000in}%
\pgftext[left,bottom]{\pgfimage[interpolate=true,width=0.110000in,height=2.350000in]{Figure2-img5.png}}%
\end{pgfscope}%
\begin{pgfscope}%
\pgfsetbuttcap%
\pgfsetroundjoin%
\definecolor{currentfill}{rgb}{0.000000,0.000000,0.000000}%
\pgfsetfillcolor{currentfill}%
\pgfsetlinewidth{0.803000pt}%
\definecolor{currentstroke}{rgb}{0.000000,0.000000,0.000000}%
\pgfsetstrokecolor{currentstroke}%
\pgfsetdash{}{0pt}%
\pgfsys@defobject{currentmarker}{\pgfqpoint{0.000000in}{0.000000in}}{\pgfqpoint{0.048611in}{0.000000in}}{%
\pgfpathmoveto{\pgfqpoint{0.000000in}{0.000000in}}%
\pgfpathlineto{\pgfqpoint{0.048611in}{0.000000in}}%
\pgfusepath{stroke,fill}%
}%
\begin{pgfscope}%
\pgfsys@transformshift{3.173964in}{0.767020in}%
\pgfsys@useobject{currentmarker}{}%
\end{pgfscope}%
\end{pgfscope}%
\begin{pgfscope}%
\definecolor{textcolor}{rgb}{0.000000,0.000000,0.000000}%
\pgfsetstrokecolor{textcolor}%
\pgfsetfillcolor{textcolor}%
\pgftext[x=3.271186in,y=0.714259in,left,base]{\color{textcolor}\sffamily\fontsize{10.000000}{12.000000}\selectfont −0.00002}%
\end{pgfscope}%
\begin{pgfscope}%
\pgfsetbuttcap%
\pgfsetroundjoin%
\definecolor{currentfill}{rgb}{0.000000,0.000000,0.000000}%
\pgfsetfillcolor{currentfill}%
\pgfsetlinewidth{0.803000pt}%
\definecolor{currentstroke}{rgb}{0.000000,0.000000,0.000000}%
\pgfsetstrokecolor{currentstroke}%
\pgfsetdash{}{0pt}%
\pgfsys@defobject{currentmarker}{\pgfqpoint{0.000000in}{0.000000in}}{\pgfqpoint{0.048611in}{0.000000in}}{%
\pgfpathmoveto{\pgfqpoint{0.000000in}{0.000000in}}%
\pgfpathlineto{\pgfqpoint{0.048611in}{0.000000in}}%
\pgfusepath{stroke,fill}%
}%
\begin{pgfscope}%
\pgfsys@transformshift{3.173964in}{1.235545in}%
\pgfsys@useobject{currentmarker}{}%
\end{pgfscope}%
\end{pgfscope}%
\begin{pgfscope}%
\definecolor{textcolor}{rgb}{0.000000,0.000000,0.000000}%
\pgfsetstrokecolor{textcolor}%
\pgfsetfillcolor{textcolor}%
\pgftext[x=3.271186in,y=1.182784in,left,base]{\color{textcolor}\sffamily\fontsize{10.000000}{12.000000}\selectfont −0.00001}%
\end{pgfscope}%
\begin{pgfscope}%
\pgfsetbuttcap%
\pgfsetroundjoin%
\definecolor{currentfill}{rgb}{0.000000,0.000000,0.000000}%
\pgfsetfillcolor{currentfill}%
\pgfsetlinewidth{0.803000pt}%
\definecolor{currentstroke}{rgb}{0.000000,0.000000,0.000000}%
\pgfsetstrokecolor{currentstroke}%
\pgfsetdash{}{0pt}%
\pgfsys@defobject{currentmarker}{\pgfqpoint{0.000000in}{0.000000in}}{\pgfqpoint{0.048611in}{0.000000in}}{%
\pgfpathmoveto{\pgfqpoint{0.000000in}{0.000000in}}%
\pgfpathlineto{\pgfqpoint{0.048611in}{0.000000in}}%
\pgfusepath{stroke,fill}%
}%
\begin{pgfscope}%
\pgfsys@transformshift{3.173964in}{1.704070in}%
\pgfsys@useobject{currentmarker}{}%
\end{pgfscope}%
\end{pgfscope}%
\begin{pgfscope}%
\definecolor{textcolor}{rgb}{0.000000,0.000000,0.000000}%
\pgfsetstrokecolor{textcolor}%
\pgfsetfillcolor{textcolor}%
\pgftext[x=3.271186in,y=1.651309in,left,base]{\color{textcolor}\sffamily\fontsize{10.000000}{12.000000}\selectfont 0.00000}%
\end{pgfscope}%
\begin{pgfscope}%
\pgfsetbuttcap%
\pgfsetroundjoin%
\definecolor{currentfill}{rgb}{0.000000,0.000000,0.000000}%
\pgfsetfillcolor{currentfill}%
\pgfsetlinewidth{0.803000pt}%
\definecolor{currentstroke}{rgb}{0.000000,0.000000,0.000000}%
\pgfsetstrokecolor{currentstroke}%
\pgfsetdash{}{0pt}%
\pgfsys@defobject{currentmarker}{\pgfqpoint{0.000000in}{0.000000in}}{\pgfqpoint{0.048611in}{0.000000in}}{%
\pgfpathmoveto{\pgfqpoint{0.000000in}{0.000000in}}%
\pgfpathlineto{\pgfqpoint{0.048611in}{0.000000in}}%
\pgfusepath{stroke,fill}%
}%
\begin{pgfscope}%
\pgfsys@transformshift{3.173964in}{2.172595in}%
\pgfsys@useobject{currentmarker}{}%
\end{pgfscope}%
\end{pgfscope}%
\begin{pgfscope}%
\definecolor{textcolor}{rgb}{0.000000,0.000000,0.000000}%
\pgfsetstrokecolor{textcolor}%
\pgfsetfillcolor{textcolor}%
\pgftext[x=3.271186in,y=2.119834in,left,base]{\color{textcolor}\sffamily\fontsize{10.000000}{12.000000}\selectfont 0.00001}%
\end{pgfscope}%
\begin{pgfscope}%
\pgfsetbuttcap%
\pgfsetroundjoin%
\definecolor{currentfill}{rgb}{0.000000,0.000000,0.000000}%
\pgfsetfillcolor{currentfill}%
\pgfsetlinewidth{0.803000pt}%
\definecolor{currentstroke}{rgb}{0.000000,0.000000,0.000000}%
\pgfsetstrokecolor{currentstroke}%
\pgfsetdash{}{0pt}%
\pgfsys@defobject{currentmarker}{\pgfqpoint{0.000000in}{0.000000in}}{\pgfqpoint{0.048611in}{0.000000in}}{%
\pgfpathmoveto{\pgfqpoint{0.000000in}{0.000000in}}%
\pgfpathlineto{\pgfqpoint{0.048611in}{0.000000in}}%
\pgfusepath{stroke,fill}%
}%
\begin{pgfscope}%
\pgfsys@transformshift{3.173964in}{2.641120in}%
\pgfsys@useobject{currentmarker}{}%
\end{pgfscope}%
\end{pgfscope}%
\begin{pgfscope}%
\definecolor{textcolor}{rgb}{0.000000,0.000000,0.000000}%
\pgfsetstrokecolor{textcolor}%
\pgfsetfillcolor{textcolor}%
\pgftext[x=3.271186in,y=2.588359in,left,base]{\color{textcolor}\sffamily\fontsize{10.000000}{12.000000}\selectfont 0.00002}%
\end{pgfscope}%
\begin{pgfscope}%
\pgfsetbuttcap%
\pgfsetmiterjoin%
\pgfsetlinewidth{0.803000pt}%
\definecolor{currentstroke}{rgb}{0.000000,0.000000,0.000000}%
\pgfsetstrokecolor{currentstroke}%
\pgfsetdash{}{0pt}%
\pgfpathmoveto{\pgfqpoint{3.056370in}{0.528128in}}%
\pgfpathlineto{\pgfqpoint{3.056370in}{0.537315in}}%
\pgfpathlineto{\pgfqpoint{3.056370in}{2.870821in}}%
\pgfpathlineto{\pgfqpoint{3.056370in}{2.880008in}}%
\pgfpathlineto{\pgfqpoint{3.173964in}{2.880008in}}%
\pgfpathlineto{\pgfqpoint{3.173964in}{2.870821in}}%
\pgfpathlineto{\pgfqpoint{3.173964in}{0.537315in}}%
\pgfpathlineto{\pgfqpoint{3.173964in}{0.528128in}}%
\pgfpathclose%
\pgfusepath{stroke}%
\end{pgfscope}%
\begin{pgfscope}%
\pgfsetbuttcap%
\pgfsetmiterjoin%
\definecolor{currentfill}{rgb}{1.000000,1.000000,1.000000}%
\pgfsetfillcolor{currentfill}%
\pgfsetlinewidth{0.000000pt}%
\definecolor{currentstroke}{rgb}{0.000000,0.000000,0.000000}%
\pgfsetstrokecolor{currentstroke}%
\pgfsetstrokeopacity{0.000000}%
\pgfsetdash{}{0pt}%
\pgfpathmoveto{\pgfqpoint{4.555278in}{0.603554in}}%
\pgfpathlineto{\pgfqpoint{6.756306in}{0.603554in}}%
\pgfpathlineto{\pgfqpoint{6.756306in}{2.804582in}}%
\pgfpathlineto{\pgfqpoint{4.555278in}{2.804582in}}%
\pgfpathclose%
\pgfusepath{fill}%
\end{pgfscope}%
\begin{pgfscope}%
\pgfpathrectangle{\pgfqpoint{4.555278in}{0.603554in}}{\pgfqpoint{2.201028in}{2.201028in}}%
\pgfusepath{clip}%
\pgfsys@transformshift{4.555278in}{0.603554in}%
\pgftext[left,bottom]{\pgfimage[interpolate=true,width=2.210000in,height=2.210000in]{Figure2-img6.png}}%
\end{pgfscope}%
\begin{pgfscope}%
\pgfsetbuttcap%
\pgfsetroundjoin%
\definecolor{currentfill}{rgb}{0.000000,0.000000,0.000000}%
\pgfsetfillcolor{currentfill}%
\pgfsetlinewidth{0.803000pt}%
\definecolor{currentstroke}{rgb}{0.000000,0.000000,0.000000}%
\pgfsetstrokecolor{currentstroke}%
\pgfsetdash{}{0pt}%
\pgfsys@defobject{currentmarker}{\pgfqpoint{0.000000in}{-0.048611in}}{\pgfqpoint{0.000000in}{0.000000in}}{%
\pgfpathmoveto{\pgfqpoint{0.000000in}{0.000000in}}%
\pgfpathlineto{\pgfqpoint{0.000000in}{-0.048611in}}%
\pgfusepath{stroke,fill}%
}%
\begin{pgfscope}%
\pgfsys@transformshift{4.588627in}{0.603554in}%
\pgfsys@useobject{currentmarker}{}%
\end{pgfscope}%
\end{pgfscope}%
\begin{pgfscope}%
\definecolor{textcolor}{rgb}{0.000000,0.000000,0.000000}%
\pgfsetstrokecolor{textcolor}%
\pgfsetfillcolor{textcolor}%
\pgftext[x=4.588627in,y=0.506332in,,top]{\color{textcolor}\sffamily\fontsize{10.000000}{12.000000}\selectfont 0}%
\end{pgfscope}%
\begin{pgfscope}%
\pgfsetbuttcap%
\pgfsetroundjoin%
\definecolor{currentfill}{rgb}{0.000000,0.000000,0.000000}%
\pgfsetfillcolor{currentfill}%
\pgfsetlinewidth{0.803000pt}%
\definecolor{currentstroke}{rgb}{0.000000,0.000000,0.000000}%
\pgfsetstrokecolor{currentstroke}%
\pgfsetdash{}{0pt}%
\pgfsys@defobject{currentmarker}{\pgfqpoint{0.000000in}{-0.048611in}}{\pgfqpoint{0.000000in}{0.000000in}}{%
\pgfpathmoveto{\pgfqpoint{0.000000in}{0.000000in}}%
\pgfpathlineto{\pgfqpoint{0.000000in}{-0.048611in}}%
\pgfusepath{stroke,fill}%
}%
\begin{pgfscope}%
\pgfsys@transformshift{5.255605in}{0.603554in}%
\pgfsys@useobject{currentmarker}{}%
\end{pgfscope}%
\end{pgfscope}%
\begin{pgfscope}%
\definecolor{textcolor}{rgb}{0.000000,0.000000,0.000000}%
\pgfsetstrokecolor{textcolor}%
\pgfsetfillcolor{textcolor}%
\pgftext[x=5.255605in,y=0.506332in,,top]{\color{textcolor}\sffamily\fontsize{10.000000}{12.000000}\selectfont 10}%
\end{pgfscope}%
\begin{pgfscope}%
\pgfsetbuttcap%
\pgfsetroundjoin%
\definecolor{currentfill}{rgb}{0.000000,0.000000,0.000000}%
\pgfsetfillcolor{currentfill}%
\pgfsetlinewidth{0.803000pt}%
\definecolor{currentstroke}{rgb}{0.000000,0.000000,0.000000}%
\pgfsetstrokecolor{currentstroke}%
\pgfsetdash{}{0pt}%
\pgfsys@defobject{currentmarker}{\pgfqpoint{0.000000in}{-0.048611in}}{\pgfqpoint{0.000000in}{0.000000in}}{%
\pgfpathmoveto{\pgfqpoint{0.000000in}{0.000000in}}%
\pgfpathlineto{\pgfqpoint{0.000000in}{-0.048611in}}%
\pgfusepath{stroke,fill}%
}%
\begin{pgfscope}%
\pgfsys@transformshift{5.922583in}{0.603554in}%
\pgfsys@useobject{currentmarker}{}%
\end{pgfscope}%
\end{pgfscope}%
\begin{pgfscope}%
\definecolor{textcolor}{rgb}{0.000000,0.000000,0.000000}%
\pgfsetstrokecolor{textcolor}%
\pgfsetfillcolor{textcolor}%
\pgftext[x=5.922583in,y=0.506332in,,top]{\color{textcolor}\sffamily\fontsize{10.000000}{12.000000}\selectfont 20}%
\end{pgfscope}%
\begin{pgfscope}%
\pgfsetbuttcap%
\pgfsetroundjoin%
\definecolor{currentfill}{rgb}{0.000000,0.000000,0.000000}%
\pgfsetfillcolor{currentfill}%
\pgfsetlinewidth{0.803000pt}%
\definecolor{currentstroke}{rgb}{0.000000,0.000000,0.000000}%
\pgfsetstrokecolor{currentstroke}%
\pgfsetdash{}{0pt}%
\pgfsys@defobject{currentmarker}{\pgfqpoint{0.000000in}{-0.048611in}}{\pgfqpoint{0.000000in}{0.000000in}}{%
\pgfpathmoveto{\pgfqpoint{0.000000in}{0.000000in}}%
\pgfpathlineto{\pgfqpoint{0.000000in}{-0.048611in}}%
\pgfusepath{stroke,fill}%
}%
\begin{pgfscope}%
\pgfsys@transformshift{6.589561in}{0.603554in}%
\pgfsys@useobject{currentmarker}{}%
\end{pgfscope}%
\end{pgfscope}%
\begin{pgfscope}%
\definecolor{textcolor}{rgb}{0.000000,0.000000,0.000000}%
\pgfsetstrokecolor{textcolor}%
\pgfsetfillcolor{textcolor}%
\pgftext[x=6.589561in,y=0.506332in,,top]{\color{textcolor}\sffamily\fontsize{10.000000}{12.000000}\selectfont 30}%
\end{pgfscope}%
\begin{pgfscope}%
\definecolor{textcolor}{rgb}{0.000000,0.000000,0.000000}%
\pgfsetstrokecolor{textcolor}%
\pgfsetfillcolor{textcolor}%
\pgftext[x=5.655792in,y=0.316363in,,top]{\color{textcolor}\sffamily\fontsize{10.000000}{12.000000}\selectfont \(\displaystyle x\)}%
\end{pgfscope}%
\begin{pgfscope}%
\pgfsetbuttcap%
\pgfsetroundjoin%
\definecolor{currentfill}{rgb}{0.000000,0.000000,0.000000}%
\pgfsetfillcolor{currentfill}%
\pgfsetlinewidth{0.803000pt}%
\definecolor{currentstroke}{rgb}{0.000000,0.000000,0.000000}%
\pgfsetstrokecolor{currentstroke}%
\pgfsetdash{}{0pt}%
\pgfsys@defobject{currentmarker}{\pgfqpoint{-0.048611in}{0.000000in}}{\pgfqpoint{0.000000in}{0.000000in}}{%
\pgfpathmoveto{\pgfqpoint{0.000000in}{0.000000in}}%
\pgfpathlineto{\pgfqpoint{-0.048611in}{0.000000in}}%
\pgfusepath{stroke,fill}%
}%
\begin{pgfscope}%
\pgfsys@transformshift{4.555278in}{0.636903in}%
\pgfsys@useobject{currentmarker}{}%
\end{pgfscope}%
\end{pgfscope}%
\begin{pgfscope}%
\definecolor{textcolor}{rgb}{0.000000,0.000000,0.000000}%
\pgfsetstrokecolor{textcolor}%
\pgfsetfillcolor{textcolor}%
\pgftext[x=4.369690in,y=0.584141in,left,base]{\color{textcolor}\sffamily\fontsize{10.000000}{12.000000}\selectfont 0}%
\end{pgfscope}%
\begin{pgfscope}%
\pgfsetbuttcap%
\pgfsetroundjoin%
\definecolor{currentfill}{rgb}{0.000000,0.000000,0.000000}%
\pgfsetfillcolor{currentfill}%
\pgfsetlinewidth{0.803000pt}%
\definecolor{currentstroke}{rgb}{0.000000,0.000000,0.000000}%
\pgfsetstrokecolor{currentstroke}%
\pgfsetdash{}{0pt}%
\pgfsys@defobject{currentmarker}{\pgfqpoint{-0.048611in}{0.000000in}}{\pgfqpoint{0.000000in}{0.000000in}}{%
\pgfpathmoveto{\pgfqpoint{0.000000in}{0.000000in}}%
\pgfpathlineto{\pgfqpoint{-0.048611in}{0.000000in}}%
\pgfusepath{stroke,fill}%
}%
\begin{pgfscope}%
\pgfsys@transformshift{4.555278in}{0.970392in}%
\pgfsys@useobject{currentmarker}{}%
\end{pgfscope}%
\end{pgfscope}%
\begin{pgfscope}%
\definecolor{textcolor}{rgb}{0.000000,0.000000,0.000000}%
\pgfsetstrokecolor{textcolor}%
\pgfsetfillcolor{textcolor}%
\pgftext[x=4.369690in,y=0.917630in,left,base]{\color{textcolor}\sffamily\fontsize{10.000000}{12.000000}\selectfont 5}%
\end{pgfscope}%
\begin{pgfscope}%
\pgfsetbuttcap%
\pgfsetroundjoin%
\definecolor{currentfill}{rgb}{0.000000,0.000000,0.000000}%
\pgfsetfillcolor{currentfill}%
\pgfsetlinewidth{0.803000pt}%
\definecolor{currentstroke}{rgb}{0.000000,0.000000,0.000000}%
\pgfsetstrokecolor{currentstroke}%
\pgfsetdash{}{0pt}%
\pgfsys@defobject{currentmarker}{\pgfqpoint{-0.048611in}{0.000000in}}{\pgfqpoint{0.000000in}{0.000000in}}{%
\pgfpathmoveto{\pgfqpoint{0.000000in}{0.000000in}}%
\pgfpathlineto{\pgfqpoint{-0.048611in}{0.000000in}}%
\pgfusepath{stroke,fill}%
}%
\begin{pgfscope}%
\pgfsys@transformshift{4.555278in}{1.303881in}%
\pgfsys@useobject{currentmarker}{}%
\end{pgfscope}%
\end{pgfscope}%
\begin{pgfscope}%
\definecolor{textcolor}{rgb}{0.000000,0.000000,0.000000}%
\pgfsetstrokecolor{textcolor}%
\pgfsetfillcolor{textcolor}%
\pgftext[x=4.281325in,y=1.251120in,left,base]{\color{textcolor}\sffamily\fontsize{10.000000}{12.000000}\selectfont 10}%
\end{pgfscope}%
\begin{pgfscope}%
\pgfsetbuttcap%
\pgfsetroundjoin%
\definecolor{currentfill}{rgb}{0.000000,0.000000,0.000000}%
\pgfsetfillcolor{currentfill}%
\pgfsetlinewidth{0.803000pt}%
\definecolor{currentstroke}{rgb}{0.000000,0.000000,0.000000}%
\pgfsetstrokecolor{currentstroke}%
\pgfsetdash{}{0pt}%
\pgfsys@defobject{currentmarker}{\pgfqpoint{-0.048611in}{0.000000in}}{\pgfqpoint{0.000000in}{0.000000in}}{%
\pgfpathmoveto{\pgfqpoint{0.000000in}{0.000000in}}%
\pgfpathlineto{\pgfqpoint{-0.048611in}{0.000000in}}%
\pgfusepath{stroke,fill}%
}%
\begin{pgfscope}%
\pgfsys@transformshift{4.555278in}{1.637370in}%
\pgfsys@useobject{currentmarker}{}%
\end{pgfscope}%
\end{pgfscope}%
\begin{pgfscope}%
\definecolor{textcolor}{rgb}{0.000000,0.000000,0.000000}%
\pgfsetstrokecolor{textcolor}%
\pgfsetfillcolor{textcolor}%
\pgftext[x=4.281325in,y=1.584609in,left,base]{\color{textcolor}\sffamily\fontsize{10.000000}{12.000000}\selectfont 15}%
\end{pgfscope}%
\begin{pgfscope}%
\pgfsetbuttcap%
\pgfsetroundjoin%
\definecolor{currentfill}{rgb}{0.000000,0.000000,0.000000}%
\pgfsetfillcolor{currentfill}%
\pgfsetlinewidth{0.803000pt}%
\definecolor{currentstroke}{rgb}{0.000000,0.000000,0.000000}%
\pgfsetstrokecolor{currentstroke}%
\pgfsetdash{}{0pt}%
\pgfsys@defobject{currentmarker}{\pgfqpoint{-0.048611in}{0.000000in}}{\pgfqpoint{0.000000in}{0.000000in}}{%
\pgfpathmoveto{\pgfqpoint{0.000000in}{0.000000in}}%
\pgfpathlineto{\pgfqpoint{-0.048611in}{0.000000in}}%
\pgfusepath{stroke,fill}%
}%
\begin{pgfscope}%
\pgfsys@transformshift{4.555278in}{1.970859in}%
\pgfsys@useobject{currentmarker}{}%
\end{pgfscope}%
\end{pgfscope}%
\begin{pgfscope}%
\definecolor{textcolor}{rgb}{0.000000,0.000000,0.000000}%
\pgfsetstrokecolor{textcolor}%
\pgfsetfillcolor{textcolor}%
\pgftext[x=4.281325in,y=1.918098in,left,base]{\color{textcolor}\sffamily\fontsize{10.000000}{12.000000}\selectfont 20}%
\end{pgfscope}%
\begin{pgfscope}%
\pgfsetbuttcap%
\pgfsetroundjoin%
\definecolor{currentfill}{rgb}{0.000000,0.000000,0.000000}%
\pgfsetfillcolor{currentfill}%
\pgfsetlinewidth{0.803000pt}%
\definecolor{currentstroke}{rgb}{0.000000,0.000000,0.000000}%
\pgfsetstrokecolor{currentstroke}%
\pgfsetdash{}{0pt}%
\pgfsys@defobject{currentmarker}{\pgfqpoint{-0.048611in}{0.000000in}}{\pgfqpoint{0.000000in}{0.000000in}}{%
\pgfpathmoveto{\pgfqpoint{0.000000in}{0.000000in}}%
\pgfpathlineto{\pgfqpoint{-0.048611in}{0.000000in}}%
\pgfusepath{stroke,fill}%
}%
\begin{pgfscope}%
\pgfsys@transformshift{4.555278in}{2.304348in}%
\pgfsys@useobject{currentmarker}{}%
\end{pgfscope}%
\end{pgfscope}%
\begin{pgfscope}%
\definecolor{textcolor}{rgb}{0.000000,0.000000,0.000000}%
\pgfsetstrokecolor{textcolor}%
\pgfsetfillcolor{textcolor}%
\pgftext[x=4.281325in,y=2.251587in,left,base]{\color{textcolor}\sffamily\fontsize{10.000000}{12.000000}\selectfont 25}%
\end{pgfscope}%
\begin{pgfscope}%
\pgfsetbuttcap%
\pgfsetroundjoin%
\definecolor{currentfill}{rgb}{0.000000,0.000000,0.000000}%
\pgfsetfillcolor{currentfill}%
\pgfsetlinewidth{0.803000pt}%
\definecolor{currentstroke}{rgb}{0.000000,0.000000,0.000000}%
\pgfsetstrokecolor{currentstroke}%
\pgfsetdash{}{0pt}%
\pgfsys@defobject{currentmarker}{\pgfqpoint{-0.048611in}{0.000000in}}{\pgfqpoint{0.000000in}{0.000000in}}{%
\pgfpathmoveto{\pgfqpoint{0.000000in}{0.000000in}}%
\pgfpathlineto{\pgfqpoint{-0.048611in}{0.000000in}}%
\pgfusepath{stroke,fill}%
}%
\begin{pgfscope}%
\pgfsys@transformshift{4.555278in}{2.637837in}%
\pgfsys@useobject{currentmarker}{}%
\end{pgfscope}%
\end{pgfscope}%
\begin{pgfscope}%
\definecolor{textcolor}{rgb}{0.000000,0.000000,0.000000}%
\pgfsetstrokecolor{textcolor}%
\pgfsetfillcolor{textcolor}%
\pgftext[x=4.281325in,y=2.585076in,left,base]{\color{textcolor}\sffamily\fontsize{10.000000}{12.000000}\selectfont 30}%
\end{pgfscope}%
\begin{pgfscope}%
\definecolor{textcolor}{rgb}{0.000000,0.000000,0.000000}%
\pgfsetstrokecolor{textcolor}%
\pgfsetfillcolor{textcolor}%
\pgftext[x=4.225769in,y=1.704068in,,bottom,rotate=90.000000]{\color{textcolor}\sffamily\fontsize{10.000000}{12.000000}\selectfont \(\displaystyle y\)}%
\end{pgfscope}%
\begin{pgfscope}%
\pgfsetrectcap%
\pgfsetmiterjoin%
\pgfsetlinewidth{0.803000pt}%
\definecolor{currentstroke}{rgb}{0.000000,0.000000,0.000000}%
\pgfsetstrokecolor{currentstroke}%
\pgfsetdash{}{0pt}%
\pgfpathmoveto{\pgfqpoint{4.555278in}{0.603554in}}%
\pgfpathlineto{\pgfqpoint{4.555278in}{2.804582in}}%
\pgfusepath{stroke}%
\end{pgfscope}%
\begin{pgfscope}%
\pgfsetrectcap%
\pgfsetmiterjoin%
\pgfsetlinewidth{0.803000pt}%
\definecolor{currentstroke}{rgb}{0.000000,0.000000,0.000000}%
\pgfsetstrokecolor{currentstroke}%
\pgfsetdash{}{0pt}%
\pgfpathmoveto{\pgfqpoint{6.756306in}{0.603554in}}%
\pgfpathlineto{\pgfqpoint{6.756306in}{2.804582in}}%
\pgfusepath{stroke}%
\end{pgfscope}%
\begin{pgfscope}%
\pgfsetrectcap%
\pgfsetmiterjoin%
\pgfsetlinewidth{0.803000pt}%
\definecolor{currentstroke}{rgb}{0.000000,0.000000,0.000000}%
\pgfsetstrokecolor{currentstroke}%
\pgfsetdash{}{0pt}%
\pgfpathmoveto{\pgfqpoint{4.555278in}{0.603554in}}%
\pgfpathlineto{\pgfqpoint{6.756306in}{0.603554in}}%
\pgfusepath{stroke}%
\end{pgfscope}%
\begin{pgfscope}%
\pgfsetrectcap%
\pgfsetmiterjoin%
\pgfsetlinewidth{0.803000pt}%
\definecolor{currentstroke}{rgb}{0.000000,0.000000,0.000000}%
\pgfsetstrokecolor{currentstroke}%
\pgfsetdash{}{0pt}%
\pgfpathmoveto{\pgfqpoint{4.555278in}{2.804582in}}%
\pgfpathlineto{\pgfqpoint{6.756306in}{2.804582in}}%
\pgfusepath{stroke}%
\end{pgfscope}%
\begin{pgfscope}%
\pgfpathrectangle{\pgfqpoint{6.893870in}{0.528128in}}{\pgfqpoint{0.117594in}{2.351880in}}%
\pgfusepath{clip}%
\pgfsetbuttcap%
\pgfsetmiterjoin%
\definecolor{currentfill}{rgb}{1.000000,1.000000,1.000000}%
\pgfsetfillcolor{currentfill}%
\pgfsetlinewidth{0.010037pt}%
\definecolor{currentstroke}{rgb}{1.000000,1.000000,1.000000}%
\pgfsetstrokecolor{currentstroke}%
\pgfsetdash{}{0pt}%
\pgfpathmoveto{\pgfqpoint{6.893870in}{0.528128in}}%
\pgfpathlineto{\pgfqpoint{6.893870in}{0.537315in}}%
\pgfpathlineto{\pgfqpoint{6.893870in}{2.870821in}}%
\pgfpathlineto{\pgfqpoint{6.893870in}{2.880008in}}%
\pgfpathlineto{\pgfqpoint{7.011464in}{2.880008in}}%
\pgfpathlineto{\pgfqpoint{7.011464in}{2.870821in}}%
\pgfpathlineto{\pgfqpoint{7.011464in}{0.537315in}}%
\pgfpathlineto{\pgfqpoint{7.011464in}{0.528128in}}%
\pgfpathclose%
\pgfusepath{stroke,fill}%
\end{pgfscope}%
\begin{pgfscope}%
\pgfsys@transformshift{6.890000in}{0.530000in}%
\pgftext[left,bottom]{\pgfimage[interpolate=true,width=0.120000in,height=2.350000in]{Figure2-img7.png}}%
\end{pgfscope}%
\begin{pgfscope}%
\pgfsetbuttcap%
\pgfsetroundjoin%
\definecolor{currentfill}{rgb}{0.000000,0.000000,0.000000}%
\pgfsetfillcolor{currentfill}%
\pgfsetlinewidth{0.803000pt}%
\definecolor{currentstroke}{rgb}{0.000000,0.000000,0.000000}%
\pgfsetstrokecolor{currentstroke}%
\pgfsetdash{}{0pt}%
\pgfsys@defobject{currentmarker}{\pgfqpoint{0.000000in}{0.000000in}}{\pgfqpoint{0.048611in}{0.000000in}}{%
\pgfpathmoveto{\pgfqpoint{0.000000in}{0.000000in}}%
\pgfpathlineto{\pgfqpoint{0.048611in}{0.000000in}}%
\pgfusepath{stroke,fill}%
}%
\begin{pgfscope}%
\pgfsys@transformshift{7.011464in}{0.594964in}%
\pgfsys@useobject{currentmarker}{}%
\end{pgfscope}%
\end{pgfscope}%
\begin{pgfscope}%
\definecolor{textcolor}{rgb}{0.000000,0.000000,0.000000}%
\pgfsetstrokecolor{textcolor}%
\pgfsetfillcolor{textcolor}%
\pgftext[x=7.108686in,y=0.542203in,left,base]{\color{textcolor}\sffamily\fontsize{10.000000}{12.000000}\selectfont −0.006}%
\end{pgfscope}%
\begin{pgfscope}%
\pgfsetbuttcap%
\pgfsetroundjoin%
\definecolor{currentfill}{rgb}{0.000000,0.000000,0.000000}%
\pgfsetfillcolor{currentfill}%
\pgfsetlinewidth{0.803000pt}%
\definecolor{currentstroke}{rgb}{0.000000,0.000000,0.000000}%
\pgfsetstrokecolor{currentstroke}%
\pgfsetdash{}{0pt}%
\pgfsys@defobject{currentmarker}{\pgfqpoint{0.000000in}{0.000000in}}{\pgfqpoint{0.048611in}{0.000000in}}{%
\pgfpathmoveto{\pgfqpoint{0.000000in}{0.000000in}}%
\pgfpathlineto{\pgfqpoint{0.048611in}{0.000000in}}%
\pgfusepath{stroke,fill}%
}%
\begin{pgfscope}%
\pgfsys@transformshift{7.011464in}{0.964683in}%
\pgfsys@useobject{currentmarker}{}%
\end{pgfscope}%
\end{pgfscope}%
\begin{pgfscope}%
\definecolor{textcolor}{rgb}{0.000000,0.000000,0.000000}%
\pgfsetstrokecolor{textcolor}%
\pgfsetfillcolor{textcolor}%
\pgftext[x=7.108686in,y=0.911922in,left,base]{\color{textcolor}\sffamily\fontsize{10.000000}{12.000000}\selectfont −0.004}%
\end{pgfscope}%
\begin{pgfscope}%
\pgfsetbuttcap%
\pgfsetroundjoin%
\definecolor{currentfill}{rgb}{0.000000,0.000000,0.000000}%
\pgfsetfillcolor{currentfill}%
\pgfsetlinewidth{0.803000pt}%
\definecolor{currentstroke}{rgb}{0.000000,0.000000,0.000000}%
\pgfsetstrokecolor{currentstroke}%
\pgfsetdash{}{0pt}%
\pgfsys@defobject{currentmarker}{\pgfqpoint{0.000000in}{0.000000in}}{\pgfqpoint{0.048611in}{0.000000in}}{%
\pgfpathmoveto{\pgfqpoint{0.000000in}{0.000000in}}%
\pgfpathlineto{\pgfqpoint{0.048611in}{0.000000in}}%
\pgfusepath{stroke,fill}%
}%
\begin{pgfscope}%
\pgfsys@transformshift{7.011464in}{1.334402in}%
\pgfsys@useobject{currentmarker}{}%
\end{pgfscope}%
\end{pgfscope}%
\begin{pgfscope}%
\definecolor{textcolor}{rgb}{0.000000,0.000000,0.000000}%
\pgfsetstrokecolor{textcolor}%
\pgfsetfillcolor{textcolor}%
\pgftext[x=7.108686in,y=1.281641in,left,base]{\color{textcolor}\sffamily\fontsize{10.000000}{12.000000}\selectfont −0.002}%
\end{pgfscope}%
\begin{pgfscope}%
\pgfsetbuttcap%
\pgfsetroundjoin%
\definecolor{currentfill}{rgb}{0.000000,0.000000,0.000000}%
\pgfsetfillcolor{currentfill}%
\pgfsetlinewidth{0.803000pt}%
\definecolor{currentstroke}{rgb}{0.000000,0.000000,0.000000}%
\pgfsetstrokecolor{currentstroke}%
\pgfsetdash{}{0pt}%
\pgfsys@defobject{currentmarker}{\pgfqpoint{0.000000in}{0.000000in}}{\pgfqpoint{0.048611in}{0.000000in}}{%
\pgfpathmoveto{\pgfqpoint{0.000000in}{0.000000in}}%
\pgfpathlineto{\pgfqpoint{0.048611in}{0.000000in}}%
\pgfusepath{stroke,fill}%
}%
\begin{pgfscope}%
\pgfsys@transformshift{7.011464in}{1.704121in}%
\pgfsys@useobject{currentmarker}{}%
\end{pgfscope}%
\end{pgfscope}%
\begin{pgfscope}%
\definecolor{textcolor}{rgb}{0.000000,0.000000,0.000000}%
\pgfsetstrokecolor{textcolor}%
\pgfsetfillcolor{textcolor}%
\pgftext[x=7.108686in,y=1.651359in,left,base]{\color{textcolor}\sffamily\fontsize{10.000000}{12.000000}\selectfont 0.000}%
\end{pgfscope}%
\begin{pgfscope}%
\pgfsetbuttcap%
\pgfsetroundjoin%
\definecolor{currentfill}{rgb}{0.000000,0.000000,0.000000}%
\pgfsetfillcolor{currentfill}%
\pgfsetlinewidth{0.803000pt}%
\definecolor{currentstroke}{rgb}{0.000000,0.000000,0.000000}%
\pgfsetstrokecolor{currentstroke}%
\pgfsetdash{}{0pt}%
\pgfsys@defobject{currentmarker}{\pgfqpoint{0.000000in}{0.000000in}}{\pgfqpoint{0.048611in}{0.000000in}}{%
\pgfpathmoveto{\pgfqpoint{0.000000in}{0.000000in}}%
\pgfpathlineto{\pgfqpoint{0.048611in}{0.000000in}}%
\pgfusepath{stroke,fill}%
}%
\begin{pgfscope}%
\pgfsys@transformshift{7.011464in}{2.073840in}%
\pgfsys@useobject{currentmarker}{}%
\end{pgfscope}%
\end{pgfscope}%
\begin{pgfscope}%
\definecolor{textcolor}{rgb}{0.000000,0.000000,0.000000}%
\pgfsetstrokecolor{textcolor}%
\pgfsetfillcolor{textcolor}%
\pgftext[x=7.108686in,y=2.021078in,left,base]{\color{textcolor}\sffamily\fontsize{10.000000}{12.000000}\selectfont 0.002}%
\end{pgfscope}%
\begin{pgfscope}%
\pgfsetbuttcap%
\pgfsetroundjoin%
\definecolor{currentfill}{rgb}{0.000000,0.000000,0.000000}%
\pgfsetfillcolor{currentfill}%
\pgfsetlinewidth{0.803000pt}%
\definecolor{currentstroke}{rgb}{0.000000,0.000000,0.000000}%
\pgfsetstrokecolor{currentstroke}%
\pgfsetdash{}{0pt}%
\pgfsys@defobject{currentmarker}{\pgfqpoint{0.000000in}{0.000000in}}{\pgfqpoint{0.048611in}{0.000000in}}{%
\pgfpathmoveto{\pgfqpoint{0.000000in}{0.000000in}}%
\pgfpathlineto{\pgfqpoint{0.048611in}{0.000000in}}%
\pgfusepath{stroke,fill}%
}%
\begin{pgfscope}%
\pgfsys@transformshift{7.011464in}{2.443559in}%
\pgfsys@useobject{currentmarker}{}%
\end{pgfscope}%
\end{pgfscope}%
\begin{pgfscope}%
\definecolor{textcolor}{rgb}{0.000000,0.000000,0.000000}%
\pgfsetstrokecolor{textcolor}%
\pgfsetfillcolor{textcolor}%
\pgftext[x=7.108686in,y=2.390797in,left,base]{\color{textcolor}\sffamily\fontsize{10.000000}{12.000000}\selectfont 0.004}%
\end{pgfscope}%
\begin{pgfscope}%
\pgfsetbuttcap%
\pgfsetroundjoin%
\definecolor{currentfill}{rgb}{0.000000,0.000000,0.000000}%
\pgfsetfillcolor{currentfill}%
\pgfsetlinewidth{0.803000pt}%
\definecolor{currentstroke}{rgb}{0.000000,0.000000,0.000000}%
\pgfsetstrokecolor{currentstroke}%
\pgfsetdash{}{0pt}%
\pgfsys@defobject{currentmarker}{\pgfqpoint{0.000000in}{0.000000in}}{\pgfqpoint{0.048611in}{0.000000in}}{%
\pgfpathmoveto{\pgfqpoint{0.000000in}{0.000000in}}%
\pgfpathlineto{\pgfqpoint{0.048611in}{0.000000in}}%
\pgfusepath{stroke,fill}%
}%
\begin{pgfscope}%
\pgfsys@transformshift{7.011464in}{2.813277in}%
\pgfsys@useobject{currentmarker}{}%
\end{pgfscope}%
\end{pgfscope}%
\begin{pgfscope}%
\definecolor{textcolor}{rgb}{0.000000,0.000000,0.000000}%
\pgfsetstrokecolor{textcolor}%
\pgfsetfillcolor{textcolor}%
\pgftext[x=7.108686in,y=2.760516in,left,base]{\color{textcolor}\sffamily\fontsize{10.000000}{12.000000}\selectfont 0.006}%
\end{pgfscope}%
\begin{pgfscope}%
\pgfsetbuttcap%
\pgfsetmiterjoin%
\pgfsetlinewidth{0.803000pt}%
\definecolor{currentstroke}{rgb}{0.000000,0.000000,0.000000}%
\pgfsetstrokecolor{currentstroke}%
\pgfsetdash{}{0pt}%
\pgfpathmoveto{\pgfqpoint{6.893870in}{0.528128in}}%
\pgfpathlineto{\pgfqpoint{6.893870in}{0.537315in}}%
\pgfpathlineto{\pgfqpoint{6.893870in}{2.870821in}}%
\pgfpathlineto{\pgfqpoint{6.893870in}{2.880008in}}%
\pgfpathlineto{\pgfqpoint{7.011464in}{2.880008in}}%
\pgfpathlineto{\pgfqpoint{7.011464in}{2.870821in}}%
\pgfpathlineto{\pgfqpoint{7.011464in}{0.537315in}}%
\pgfpathlineto{\pgfqpoint{7.011464in}{0.528128in}}%
\pgfpathclose%
\pgfusepath{stroke}%
\end{pgfscope}%
\end{pgfpicture}%
\makeatother%
\endgroup%
}
\caption{Heat map of solutions using triangle element for the first equation}
\label{Fig:Rect1}
}
{
\footnotesize Top left: $u_h$ with $ N = 256 $; top right: $ u_h - u $ at nodes with $ N = 256 $; bottom left: $ u_h - u $ at $\Omega$ with $ N = 256 $; bottom right: $ u_h - u $ at $\Omega$ with $ N = 16 $.
}
\end{figure}

We also graph the error curve with respect to different $N$ in Figure \ref{Fig:Err1}. Here $L^2$ error and $H^1$ (semi-norm) error are calculated by Simpson's quadrature. Numerical results are summarized in Table \ref{Tbl:SumTri1} and \ref{Tbl:SumRect1}.

\begin{figure}[htbp]
\centering
%% Creator: Matplotlib, PGF backend
%%
%% To include the figure in your LaTeX document, write
%%   \input{<filename>.pgf}
%%
%% Make sure the required packages are loaded in your preamble
%%   \usepackage{pgf}
%%
%% Figures using additional raster images can only be included by \input if
%% they are in the same directory as the main LaTeX file. For loading figures
%% from other directories you can use the `import` package
%%   \usepackage{import}
%% and then include the figures with
%%   \import{<path to file>}{<filename>.pgf}
%%
%% Matplotlib used the following preamble
%%   \usepackage{fontspec}
%%   \setmainfont{DejaVuSerif.ttf}[Path=/home/lzh/anaconda3/envs/numpde/lib/python3.7/site-packages/matplotlib/mpl-data/fonts/ttf/]
%%   \setsansfont{DejaVuSans.ttf}[Path=/home/lzh/anaconda3/envs/numpde/lib/python3.7/site-packages/matplotlib/mpl-data/fonts/ttf/]
%%   \setmonofont{DejaVuSansMono.ttf}[Path=/home/lzh/anaconda3/envs/numpde/lib/python3.7/site-packages/matplotlib/mpl-data/fonts/ttf/]
%%
\begingroup%
\makeatletter%
\begin{pgfpicture}%
\pgfpathrectangle{\pgfpointorigin}{\pgfqpoint{8.000000in}{12.000000in}}%
\pgfusepath{use as bounding box, clip}%
\begin{pgfscope}%
\pgfsetbuttcap%
\pgfsetmiterjoin%
\definecolor{currentfill}{rgb}{1.000000,1.000000,1.000000}%
\pgfsetfillcolor{currentfill}%
\pgfsetlinewidth{0.000000pt}%
\definecolor{currentstroke}{rgb}{1.000000,1.000000,1.000000}%
\pgfsetstrokecolor{currentstroke}%
\pgfsetdash{}{0pt}%
\pgfpathmoveto{\pgfqpoint{0.000000in}{0.000000in}}%
\pgfpathlineto{\pgfqpoint{8.000000in}{0.000000in}}%
\pgfpathlineto{\pgfqpoint{8.000000in}{12.000000in}}%
\pgfpathlineto{\pgfqpoint{0.000000in}{12.000000in}}%
\pgfpathclose%
\pgfusepath{fill}%
\end{pgfscope}%
\begin{pgfscope}%
\pgfsetbuttcap%
\pgfsetmiterjoin%
\definecolor{currentfill}{rgb}{1.000000,1.000000,1.000000}%
\pgfsetfillcolor{currentfill}%
\pgfsetlinewidth{0.000000pt}%
\definecolor{currentstroke}{rgb}{0.000000,0.000000,0.000000}%
\pgfsetstrokecolor{currentstroke}%
\pgfsetstrokeopacity{0.000000}%
\pgfsetdash{}{0pt}%
\pgfpathmoveto{\pgfqpoint{4.616667in}{8.480556in}}%
\pgfpathlineto{\pgfqpoint{7.801389in}{8.480556in}}%
\pgfpathlineto{\pgfqpoint{7.801389in}{11.627778in}}%
\pgfpathlineto{\pgfqpoint{4.616667in}{11.627778in}}%
\pgfpathclose%
\pgfusepath{fill}%
\end{pgfscope}%
\begin{pgfscope}%
\pgfsetbuttcap%
\pgfsetroundjoin%
\definecolor{currentfill}{rgb}{0.000000,0.000000,0.000000}%
\pgfsetfillcolor{currentfill}%
\pgfsetlinewidth{0.803000pt}%
\definecolor{currentstroke}{rgb}{0.000000,0.000000,0.000000}%
\pgfsetstrokecolor{currentstroke}%
\pgfsetdash{}{0pt}%
\pgfsys@defobject{currentmarker}{\pgfqpoint{0.000000in}{-0.048611in}}{\pgfqpoint{0.000000in}{0.000000in}}{%
\pgfpathmoveto{\pgfqpoint{0.000000in}{0.000000in}}%
\pgfpathlineto{\pgfqpoint{0.000000in}{-0.048611in}}%
\pgfusepath{stroke,fill}%
}%
\begin{pgfscope}%
\pgfsys@transformshift{4.761065in}{8.480556in}%
\pgfsys@useobject{currentmarker}{}%
\end{pgfscope}%
\end{pgfscope}%
\begin{pgfscope}%
\definecolor{textcolor}{rgb}{0.000000,0.000000,0.000000}%
\pgfsetstrokecolor{textcolor}%
\pgfsetfillcolor{textcolor}%
\pgftext[x=4.761065in,y=8.383333in,,top]{\color{textcolor}\sffamily\fontsize{10.000000}{12.000000}\selectfont −2}%
\end{pgfscope}%
\begin{pgfscope}%
\pgfsetbuttcap%
\pgfsetroundjoin%
\definecolor{currentfill}{rgb}{0.000000,0.000000,0.000000}%
\pgfsetfillcolor{currentfill}%
\pgfsetlinewidth{0.803000pt}%
\definecolor{currentstroke}{rgb}{0.000000,0.000000,0.000000}%
\pgfsetstrokecolor{currentstroke}%
\pgfsetdash{}{0pt}%
\pgfsys@defobject{currentmarker}{\pgfqpoint{0.000000in}{-0.048611in}}{\pgfqpoint{0.000000in}{0.000000in}}{%
\pgfpathmoveto{\pgfqpoint{0.000000in}{0.000000in}}%
\pgfpathlineto{\pgfqpoint{0.000000in}{-0.048611in}}%
\pgfusepath{stroke,fill}%
}%
\begin{pgfscope}%
\pgfsys@transformshift{5.485046in}{8.480556in}%
\pgfsys@useobject{currentmarker}{}%
\end{pgfscope}%
\end{pgfscope}%
\begin{pgfscope}%
\definecolor{textcolor}{rgb}{0.000000,0.000000,0.000000}%
\pgfsetstrokecolor{textcolor}%
\pgfsetfillcolor{textcolor}%
\pgftext[x=5.485046in,y=8.383333in,,top]{\color{textcolor}\sffamily\fontsize{10.000000}{12.000000}\selectfont −1}%
\end{pgfscope}%
\begin{pgfscope}%
\pgfsetbuttcap%
\pgfsetroundjoin%
\definecolor{currentfill}{rgb}{0.000000,0.000000,0.000000}%
\pgfsetfillcolor{currentfill}%
\pgfsetlinewidth{0.803000pt}%
\definecolor{currentstroke}{rgb}{0.000000,0.000000,0.000000}%
\pgfsetstrokecolor{currentstroke}%
\pgfsetdash{}{0pt}%
\pgfsys@defobject{currentmarker}{\pgfqpoint{0.000000in}{-0.048611in}}{\pgfqpoint{0.000000in}{0.000000in}}{%
\pgfpathmoveto{\pgfqpoint{0.000000in}{0.000000in}}%
\pgfpathlineto{\pgfqpoint{0.000000in}{-0.048611in}}%
\pgfusepath{stroke,fill}%
}%
\begin{pgfscope}%
\pgfsys@transformshift{6.209028in}{8.480556in}%
\pgfsys@useobject{currentmarker}{}%
\end{pgfscope}%
\end{pgfscope}%
\begin{pgfscope}%
\definecolor{textcolor}{rgb}{0.000000,0.000000,0.000000}%
\pgfsetstrokecolor{textcolor}%
\pgfsetfillcolor{textcolor}%
\pgftext[x=6.209028in,y=8.383333in,,top]{\color{textcolor}\sffamily\fontsize{10.000000}{12.000000}\selectfont 0}%
\end{pgfscope}%
\begin{pgfscope}%
\pgfsetbuttcap%
\pgfsetroundjoin%
\definecolor{currentfill}{rgb}{0.000000,0.000000,0.000000}%
\pgfsetfillcolor{currentfill}%
\pgfsetlinewidth{0.803000pt}%
\definecolor{currentstroke}{rgb}{0.000000,0.000000,0.000000}%
\pgfsetstrokecolor{currentstroke}%
\pgfsetdash{}{0pt}%
\pgfsys@defobject{currentmarker}{\pgfqpoint{0.000000in}{-0.048611in}}{\pgfqpoint{0.000000in}{0.000000in}}{%
\pgfpathmoveto{\pgfqpoint{0.000000in}{0.000000in}}%
\pgfpathlineto{\pgfqpoint{0.000000in}{-0.048611in}}%
\pgfusepath{stroke,fill}%
}%
\begin{pgfscope}%
\pgfsys@transformshift{6.933009in}{8.480556in}%
\pgfsys@useobject{currentmarker}{}%
\end{pgfscope}%
\end{pgfscope}%
\begin{pgfscope}%
\definecolor{textcolor}{rgb}{0.000000,0.000000,0.000000}%
\pgfsetstrokecolor{textcolor}%
\pgfsetfillcolor{textcolor}%
\pgftext[x=6.933009in,y=8.383333in,,top]{\color{textcolor}\sffamily\fontsize{10.000000}{12.000000}\selectfont 1}%
\end{pgfscope}%
\begin{pgfscope}%
\pgfsetbuttcap%
\pgfsetroundjoin%
\definecolor{currentfill}{rgb}{0.000000,0.000000,0.000000}%
\pgfsetfillcolor{currentfill}%
\pgfsetlinewidth{0.803000pt}%
\definecolor{currentstroke}{rgb}{0.000000,0.000000,0.000000}%
\pgfsetstrokecolor{currentstroke}%
\pgfsetdash{}{0pt}%
\pgfsys@defobject{currentmarker}{\pgfqpoint{0.000000in}{-0.048611in}}{\pgfqpoint{0.000000in}{0.000000in}}{%
\pgfpathmoveto{\pgfqpoint{0.000000in}{0.000000in}}%
\pgfpathlineto{\pgfqpoint{0.000000in}{-0.048611in}}%
\pgfusepath{stroke,fill}%
}%
\begin{pgfscope}%
\pgfsys@transformshift{7.656991in}{8.480556in}%
\pgfsys@useobject{currentmarker}{}%
\end{pgfscope}%
\end{pgfscope}%
\begin{pgfscope}%
\definecolor{textcolor}{rgb}{0.000000,0.000000,0.000000}%
\pgfsetstrokecolor{textcolor}%
\pgfsetfillcolor{textcolor}%
\pgftext[x=7.656991in,y=8.383333in,,top]{\color{textcolor}\sffamily\fontsize{10.000000}{12.000000}\selectfont 2}%
\end{pgfscope}%
\begin{pgfscope}%
\definecolor{textcolor}{rgb}{0.000000,0.000000,0.000000}%
\pgfsetstrokecolor{textcolor}%
\pgfsetfillcolor{textcolor}%
\pgftext[x=6.209028in,y=8.193365in,,top]{\color{textcolor}\sffamily\fontsize{10.000000}{12.000000}\selectfont \(\displaystyle x\)}%
\end{pgfscope}%
\begin{pgfscope}%
\pgfsetbuttcap%
\pgfsetroundjoin%
\definecolor{currentfill}{rgb}{0.000000,0.000000,0.000000}%
\pgfsetfillcolor{currentfill}%
\pgfsetlinewidth{0.803000pt}%
\definecolor{currentstroke}{rgb}{0.000000,0.000000,0.000000}%
\pgfsetstrokecolor{currentstroke}%
\pgfsetdash{}{0pt}%
\pgfsys@defobject{currentmarker}{\pgfqpoint{-0.048611in}{0.000000in}}{\pgfqpoint{0.000000in}{0.000000in}}{%
\pgfpathmoveto{\pgfqpoint{0.000000in}{0.000000in}}%
\pgfpathlineto{\pgfqpoint{-0.048611in}{0.000000in}}%
\pgfusepath{stroke,fill}%
}%
\begin{pgfscope}%
\pgfsys@transformshift{4.616667in}{8.623611in}%
\pgfsys@useobject{currentmarker}{}%
\end{pgfscope}%
\end{pgfscope}%
\begin{pgfscope}%
\definecolor{textcolor}{rgb}{0.000000,0.000000,0.000000}%
\pgfsetstrokecolor{textcolor}%
\pgfsetfillcolor{textcolor}%
\pgftext[x=4.298565in,y=8.570850in,left,base]{\color{textcolor}\sffamily\fontsize{10.000000}{12.000000}\selectfont 0.0}%
\end{pgfscope}%
\begin{pgfscope}%
\pgfsetbuttcap%
\pgfsetroundjoin%
\definecolor{currentfill}{rgb}{0.000000,0.000000,0.000000}%
\pgfsetfillcolor{currentfill}%
\pgfsetlinewidth{0.803000pt}%
\definecolor{currentstroke}{rgb}{0.000000,0.000000,0.000000}%
\pgfsetstrokecolor{currentstroke}%
\pgfsetdash{}{0pt}%
\pgfsys@defobject{currentmarker}{\pgfqpoint{-0.048611in}{0.000000in}}{\pgfqpoint{0.000000in}{0.000000in}}{%
\pgfpathmoveto{\pgfqpoint{0.000000in}{0.000000in}}%
\pgfpathlineto{\pgfqpoint{-0.048611in}{0.000000in}}%
\pgfusepath{stroke,fill}%
}%
\begin{pgfscope}%
\pgfsys@transformshift{4.616667in}{9.017495in}%
\pgfsys@useobject{currentmarker}{}%
\end{pgfscope}%
\end{pgfscope}%
\begin{pgfscope}%
\definecolor{textcolor}{rgb}{0.000000,0.000000,0.000000}%
\pgfsetstrokecolor{textcolor}%
\pgfsetfillcolor{textcolor}%
\pgftext[x=4.298565in,y=8.964733in,left,base]{\color{textcolor}\sffamily\fontsize{10.000000}{12.000000}\selectfont 0.2}%
\end{pgfscope}%
\begin{pgfscope}%
\pgfsetbuttcap%
\pgfsetroundjoin%
\definecolor{currentfill}{rgb}{0.000000,0.000000,0.000000}%
\pgfsetfillcolor{currentfill}%
\pgfsetlinewidth{0.803000pt}%
\definecolor{currentstroke}{rgb}{0.000000,0.000000,0.000000}%
\pgfsetstrokecolor{currentstroke}%
\pgfsetdash{}{0pt}%
\pgfsys@defobject{currentmarker}{\pgfqpoint{-0.048611in}{0.000000in}}{\pgfqpoint{0.000000in}{0.000000in}}{%
\pgfpathmoveto{\pgfqpoint{0.000000in}{0.000000in}}%
\pgfpathlineto{\pgfqpoint{-0.048611in}{0.000000in}}%
\pgfusepath{stroke,fill}%
}%
\begin{pgfscope}%
\pgfsys@transformshift{4.616667in}{9.411378in}%
\pgfsys@useobject{currentmarker}{}%
\end{pgfscope}%
\end{pgfscope}%
\begin{pgfscope}%
\definecolor{textcolor}{rgb}{0.000000,0.000000,0.000000}%
\pgfsetstrokecolor{textcolor}%
\pgfsetfillcolor{textcolor}%
\pgftext[x=4.298565in,y=9.358617in,left,base]{\color{textcolor}\sffamily\fontsize{10.000000}{12.000000}\selectfont 0.4}%
\end{pgfscope}%
\begin{pgfscope}%
\pgfsetbuttcap%
\pgfsetroundjoin%
\definecolor{currentfill}{rgb}{0.000000,0.000000,0.000000}%
\pgfsetfillcolor{currentfill}%
\pgfsetlinewidth{0.803000pt}%
\definecolor{currentstroke}{rgb}{0.000000,0.000000,0.000000}%
\pgfsetstrokecolor{currentstroke}%
\pgfsetdash{}{0pt}%
\pgfsys@defobject{currentmarker}{\pgfqpoint{-0.048611in}{0.000000in}}{\pgfqpoint{0.000000in}{0.000000in}}{%
\pgfpathmoveto{\pgfqpoint{0.000000in}{0.000000in}}%
\pgfpathlineto{\pgfqpoint{-0.048611in}{0.000000in}}%
\pgfusepath{stroke,fill}%
}%
\begin{pgfscope}%
\pgfsys@transformshift{4.616667in}{9.805262in}%
\pgfsys@useobject{currentmarker}{}%
\end{pgfscope}%
\end{pgfscope}%
\begin{pgfscope}%
\definecolor{textcolor}{rgb}{0.000000,0.000000,0.000000}%
\pgfsetstrokecolor{textcolor}%
\pgfsetfillcolor{textcolor}%
\pgftext[x=4.298565in,y=9.752501in,left,base]{\color{textcolor}\sffamily\fontsize{10.000000}{12.000000}\selectfont 0.6}%
\end{pgfscope}%
\begin{pgfscope}%
\pgfsetbuttcap%
\pgfsetroundjoin%
\definecolor{currentfill}{rgb}{0.000000,0.000000,0.000000}%
\pgfsetfillcolor{currentfill}%
\pgfsetlinewidth{0.803000pt}%
\definecolor{currentstroke}{rgb}{0.000000,0.000000,0.000000}%
\pgfsetstrokecolor{currentstroke}%
\pgfsetdash{}{0pt}%
\pgfsys@defobject{currentmarker}{\pgfqpoint{-0.048611in}{0.000000in}}{\pgfqpoint{0.000000in}{0.000000in}}{%
\pgfpathmoveto{\pgfqpoint{0.000000in}{0.000000in}}%
\pgfpathlineto{\pgfqpoint{-0.048611in}{0.000000in}}%
\pgfusepath{stroke,fill}%
}%
\begin{pgfscope}%
\pgfsys@transformshift{4.616667in}{10.199146in}%
\pgfsys@useobject{currentmarker}{}%
\end{pgfscope}%
\end{pgfscope}%
\begin{pgfscope}%
\definecolor{textcolor}{rgb}{0.000000,0.000000,0.000000}%
\pgfsetstrokecolor{textcolor}%
\pgfsetfillcolor{textcolor}%
\pgftext[x=4.298565in,y=10.146384in,left,base]{\color{textcolor}\sffamily\fontsize{10.000000}{12.000000}\selectfont 0.8}%
\end{pgfscope}%
\begin{pgfscope}%
\pgfsetbuttcap%
\pgfsetroundjoin%
\definecolor{currentfill}{rgb}{0.000000,0.000000,0.000000}%
\pgfsetfillcolor{currentfill}%
\pgfsetlinewidth{0.803000pt}%
\definecolor{currentstroke}{rgb}{0.000000,0.000000,0.000000}%
\pgfsetstrokecolor{currentstroke}%
\pgfsetdash{}{0pt}%
\pgfsys@defobject{currentmarker}{\pgfqpoint{-0.048611in}{0.000000in}}{\pgfqpoint{0.000000in}{0.000000in}}{%
\pgfpathmoveto{\pgfqpoint{0.000000in}{0.000000in}}%
\pgfpathlineto{\pgfqpoint{-0.048611in}{0.000000in}}%
\pgfusepath{stroke,fill}%
}%
\begin{pgfscope}%
\pgfsys@transformshift{4.616667in}{10.593029in}%
\pgfsys@useobject{currentmarker}{}%
\end{pgfscope}%
\end{pgfscope}%
\begin{pgfscope}%
\definecolor{textcolor}{rgb}{0.000000,0.000000,0.000000}%
\pgfsetstrokecolor{textcolor}%
\pgfsetfillcolor{textcolor}%
\pgftext[x=4.298565in,y=10.540268in,left,base]{\color{textcolor}\sffamily\fontsize{10.000000}{12.000000}\selectfont 1.0}%
\end{pgfscope}%
\begin{pgfscope}%
\pgfsetbuttcap%
\pgfsetroundjoin%
\definecolor{currentfill}{rgb}{0.000000,0.000000,0.000000}%
\pgfsetfillcolor{currentfill}%
\pgfsetlinewidth{0.803000pt}%
\definecolor{currentstroke}{rgb}{0.000000,0.000000,0.000000}%
\pgfsetstrokecolor{currentstroke}%
\pgfsetdash{}{0pt}%
\pgfsys@defobject{currentmarker}{\pgfqpoint{-0.048611in}{0.000000in}}{\pgfqpoint{0.000000in}{0.000000in}}{%
\pgfpathmoveto{\pgfqpoint{0.000000in}{0.000000in}}%
\pgfpathlineto{\pgfqpoint{-0.048611in}{0.000000in}}%
\pgfusepath{stroke,fill}%
}%
\begin{pgfscope}%
\pgfsys@transformshift{4.616667in}{10.986913in}%
\pgfsys@useobject{currentmarker}{}%
\end{pgfscope}%
\end{pgfscope}%
\begin{pgfscope}%
\definecolor{textcolor}{rgb}{0.000000,0.000000,0.000000}%
\pgfsetstrokecolor{textcolor}%
\pgfsetfillcolor{textcolor}%
\pgftext[x=4.298565in,y=10.934151in,left,base]{\color{textcolor}\sffamily\fontsize{10.000000}{12.000000}\selectfont 1.2}%
\end{pgfscope}%
\begin{pgfscope}%
\pgfsetbuttcap%
\pgfsetroundjoin%
\definecolor{currentfill}{rgb}{0.000000,0.000000,0.000000}%
\pgfsetfillcolor{currentfill}%
\pgfsetlinewidth{0.803000pt}%
\definecolor{currentstroke}{rgb}{0.000000,0.000000,0.000000}%
\pgfsetstrokecolor{currentstroke}%
\pgfsetdash{}{0pt}%
\pgfsys@defobject{currentmarker}{\pgfqpoint{-0.048611in}{0.000000in}}{\pgfqpoint{0.000000in}{0.000000in}}{%
\pgfpathmoveto{\pgfqpoint{0.000000in}{0.000000in}}%
\pgfpathlineto{\pgfqpoint{-0.048611in}{0.000000in}}%
\pgfusepath{stroke,fill}%
}%
\begin{pgfscope}%
\pgfsys@transformshift{4.616667in}{11.380797in}%
\pgfsys@useobject{currentmarker}{}%
\end{pgfscope}%
\end{pgfscope}%
\begin{pgfscope}%
\definecolor{textcolor}{rgb}{0.000000,0.000000,0.000000}%
\pgfsetstrokecolor{textcolor}%
\pgfsetfillcolor{textcolor}%
\pgftext[x=4.298565in,y=11.328035in,left,base]{\color{textcolor}\sffamily\fontsize{10.000000}{12.000000}\selectfont 1.4}%
\end{pgfscope}%
\begin{pgfscope}%
\definecolor{textcolor}{rgb}{0.000000,0.000000,0.000000}%
\pgfsetstrokecolor{textcolor}%
\pgfsetfillcolor{textcolor}%
\pgftext[x=4.243009in,y=10.054167in,,bottom,rotate=90.000000]{\color{textcolor}\sffamily\fontsize{10.000000}{12.000000}\selectfont \(\displaystyle U\)}%
\end{pgfscope}%
\begin{pgfscope}%
\pgfpathrectangle{\pgfqpoint{4.616667in}{8.480556in}}{\pgfqpoint{3.184722in}{3.147222in}}%
\pgfusepath{clip}%
\pgfsetrectcap%
\pgfsetroundjoin%
\pgfsetlinewidth{1.505625pt}%
\definecolor{currentstroke}{rgb}{0.121569,0.466667,0.705882}%
\pgfsetstrokecolor{currentstroke}%
\pgfsetdash{}{0pt}%
\pgfpathmoveto{\pgfqpoint{4.761427in}{10.593029in}}%
\pgfpathlineto{\pgfqpoint{6.235453in}{10.593037in}}%
\pgfpathlineto{\pgfqpoint{6.236177in}{10.591167in}}%
\pgfpathlineto{\pgfqpoint{6.237625in}{10.594385in}}%
\pgfpathlineto{\pgfqpoint{6.238349in}{10.590394in}}%
\pgfpathlineto{\pgfqpoint{6.239073in}{10.593220in}}%
\pgfpathlineto{\pgfqpoint{6.239797in}{10.596163in}}%
\pgfpathlineto{\pgfqpoint{6.240521in}{10.590567in}}%
\pgfpathlineto{\pgfqpoint{6.241245in}{10.590903in}}%
\pgfpathlineto{\pgfqpoint{6.241969in}{10.597576in}}%
\pgfpathlineto{\pgfqpoint{6.242693in}{10.592404in}}%
\pgfpathlineto{\pgfqpoint{6.243417in}{10.587858in}}%
\pgfpathlineto{\pgfqpoint{6.244141in}{10.597501in}}%
\pgfpathlineto{\pgfqpoint{6.244865in}{10.596216in}}%
\pgfpathlineto{\pgfqpoint{6.245589in}{10.585189in}}%
\pgfpathlineto{\pgfqpoint{6.246313in}{10.594670in}}%
\pgfpathlineto{\pgfqpoint{6.247037in}{10.601566in}}%
\pgfpathlineto{\pgfqpoint{6.247761in}{10.584882in}}%
\pgfpathlineto{\pgfqpoint{6.248485in}{10.588238in}}%
\pgfpathlineto{\pgfqpoint{6.249209in}{10.606490in}}%
\pgfpathlineto{\pgfqpoint{6.249933in}{10.589299in}}%
\pgfpathlineto{\pgfqpoint{6.250657in}{10.579001in}}%
\pgfpathlineto{\pgfqpoint{6.251381in}{10.607598in}}%
\pgfpathlineto{\pgfqpoint{6.252105in}{10.599886in}}%
\pgfpathlineto{\pgfqpoint{6.252829in}{10.569988in}}%
\pgfpathlineto{\pgfqpoint{6.253553in}{10.600857in}}%
\pgfpathlineto{\pgfqpoint{6.254277in}{10.615839in}}%
\pgfpathlineto{\pgfqpoint{6.255001in}{10.566838in}}%
\pgfpathlineto{\pgfqpoint{6.255725in}{10.583227in}}%
\pgfpathlineto{\pgfqpoint{6.256449in}{10.632171in}}%
\pgfpathlineto{\pgfqpoint{6.257897in}{10.555786in}}%
\pgfpathlineto{\pgfqpoint{6.258621in}{10.639145in}}%
\pgfpathlineto{\pgfqpoint{6.259344in}{10.605433in}}%
\pgfpathlineto{\pgfqpoint{6.260068in}{10.526300in}}%
\pgfpathlineto{\pgfqpoint{6.261516in}{10.652303in}}%
\pgfpathlineto{\pgfqpoint{6.262240in}{10.510620in}}%
\pgfpathlineto{\pgfqpoint{6.262964in}{10.576188in}}%
\pgfpathlineto{\pgfqpoint{6.263688in}{10.704823in}}%
\pgfpathlineto{\pgfqpoint{6.265136in}{10.495627in}}%
\pgfpathlineto{\pgfqpoint{6.265860in}{10.734884in}}%
\pgfpathlineto{\pgfqpoint{6.266584in}{10.606008in}}%
\pgfpathlineto{\pgfqpoint{6.267308in}{10.401763in}}%
\pgfpathlineto{\pgfqpoint{6.268756in}{10.741904in}}%
\pgfpathlineto{\pgfqpoint{6.269480in}{10.338547in}}%
\pgfpathlineto{\pgfqpoint{6.270204in}{10.568844in}}%
\pgfpathlineto{\pgfqpoint{6.270928in}{10.906519in}}%
\pgfpathlineto{\pgfqpoint{6.272376in}{10.328285in}}%
\pgfpathlineto{\pgfqpoint{6.273100in}{11.009962in}}%
\pgfpathlineto{\pgfqpoint{6.273824in}{10.574657in}}%
\pgfpathlineto{\pgfqpoint{6.274548in}{10.035241in}}%
\pgfpathlineto{\pgfqpoint{6.275996in}{10.986456in}}%
\pgfpathlineto{\pgfqpoint{6.276720in}{9.832790in}}%
\pgfpathlineto{\pgfqpoint{6.277444in}{10.417759in}}%
\pgfpathlineto{\pgfqpoint{6.278168in}{11.484722in}}%
\pgfpathlineto{\pgfqpoint{6.279616in}{9.559498in}}%
\pgfpathlineto{\pgfqpoint{6.280340in}{11.447629in}}%
\pgfpathlineto{\pgfqpoint{6.281064in}{10.972204in}}%
\pgfpathlineto{\pgfqpoint{6.283236in}{8.623611in}}%
\pgfpathlineto{\pgfqpoint{7.656629in}{8.623611in}}%
\pgfpathlineto{\pgfqpoint{7.656629in}{8.623611in}}%
\pgfusepath{stroke}%
\end{pgfscope}%
\begin{pgfscope}%
\pgfpathrectangle{\pgfqpoint{4.616667in}{8.480556in}}{\pgfqpoint{3.184722in}{3.147222in}}%
\pgfusepath{clip}%
\pgfsetrectcap%
\pgfsetroundjoin%
\pgfsetlinewidth{1.505625pt}%
\definecolor{currentstroke}{rgb}{1.000000,0.498039,0.054902}%
\pgfsetstrokecolor{currentstroke}%
\pgfsetdash{}{0pt}%
\pgfpathmoveto{\pgfqpoint{4.761427in}{10.593029in}}%
\pgfpathlineto{\pgfqpoint{6.237625in}{10.593874in}}%
\pgfpathlineto{\pgfqpoint{6.238349in}{10.591438in}}%
\pgfpathlineto{\pgfqpoint{6.239073in}{10.593112in}}%
\pgfpathlineto{\pgfqpoint{6.239797in}{10.594943in}}%
\pgfpathlineto{\pgfqpoint{6.241245in}{10.591700in}}%
\pgfpathlineto{\pgfqpoint{6.241969in}{10.595776in}}%
\pgfpathlineto{\pgfqpoint{6.242693in}{10.592705in}}%
\pgfpathlineto{\pgfqpoint{6.243417in}{10.589863in}}%
\pgfpathlineto{\pgfqpoint{6.244141in}{10.595691in}}%
\pgfpathlineto{\pgfqpoint{6.244865in}{10.595031in}}%
\pgfpathlineto{\pgfqpoint{6.245589in}{10.588280in}}%
\pgfpathlineto{\pgfqpoint{6.246313in}{10.593929in}}%
\pgfpathlineto{\pgfqpoint{6.247037in}{10.598255in}}%
\pgfpathlineto{\pgfqpoint{6.247761in}{10.588161in}}%
\pgfpathlineto{\pgfqpoint{6.248485in}{10.589999in}}%
\pgfpathlineto{\pgfqpoint{6.249209in}{10.601170in}}%
\pgfpathlineto{\pgfqpoint{6.249933in}{10.590921in}}%
\pgfpathlineto{\pgfqpoint{6.250657in}{10.584405in}}%
\pgfpathlineto{\pgfqpoint{6.251381in}{10.601726in}}%
\pgfpathlineto{\pgfqpoint{6.252105in}{10.597393in}}%
\pgfpathlineto{\pgfqpoint{6.252829in}{10.579015in}}%
\pgfpathlineto{\pgfqpoint{6.253553in}{10.597502in}}%
\pgfpathlineto{\pgfqpoint{6.254277in}{10.607011in}}%
\pgfpathlineto{\pgfqpoint{6.255001in}{10.577284in}}%
\pgfpathlineto{\pgfqpoint{6.255725in}{10.586695in}}%
\pgfpathlineto{\pgfqpoint{6.256449in}{10.616683in}}%
\pgfpathlineto{\pgfqpoint{6.257173in}{10.583577in}}%
\pgfpathlineto{\pgfqpoint{6.257897in}{10.569980in}}%
\pgfpathlineto{\pgfqpoint{6.258621in}{10.620541in}}%
\pgfpathlineto{\pgfqpoint{6.259344in}{10.601148in}}%
\pgfpathlineto{\pgfqpoint{6.260068in}{10.552134in}}%
\pgfpathlineto{\pgfqpoint{6.261516in}{10.629321in}}%
\pgfpathlineto{\pgfqpoint{6.262240in}{10.542987in}}%
\pgfpathlineto{\pgfqpoint{6.262964in}{10.581574in}}%
\pgfpathlineto{\pgfqpoint{6.263688in}{10.660151in}}%
\pgfpathlineto{\pgfqpoint{6.265136in}{10.531961in}}%
\pgfpathlineto{\pgfqpoint{6.265860in}{10.676885in}}%
\pgfpathlineto{\pgfqpoint{6.266584in}{10.602548in}}%
\pgfpathlineto{\pgfqpoint{6.267308in}{10.473703in}}%
\pgfpathlineto{\pgfqpoint{6.268756in}{10.683499in}}%
\pgfpathlineto{\pgfqpoint{6.269480in}{10.434816in}}%
\pgfpathlineto{\pgfqpoint{6.270204in}{10.573870in}}%
\pgfpathlineto{\pgfqpoint{6.270928in}{10.777194in}}%
\pgfpathlineto{\pgfqpoint{6.272376in}{10.421709in}}%
\pgfpathlineto{\pgfqpoint{6.273100in}{10.831473in}}%
\pgfpathlineto{\pgfqpoint{6.273824in}{10.586583in}}%
\pgfpathlineto{\pgfqpoint{6.274548in}{10.226898in}}%
\pgfpathlineto{\pgfqpoint{6.275996in}{10.826436in}}%
\pgfpathlineto{\pgfqpoint{6.276720in}{10.088625in}}%
\pgfpathlineto{\pgfqpoint{6.277444in}{10.459934in}}%
\pgfpathlineto{\pgfqpoint{6.278168in}{11.082632in}}%
\pgfpathlineto{\pgfqpoint{6.279616in}{9.840603in}}%
\pgfpathlineto{\pgfqpoint{6.280340in}{11.025062in}}%
\pgfpathlineto{\pgfqpoint{6.281064in}{10.843344in}}%
\pgfpathlineto{\pgfqpoint{6.283236in}{8.623613in}}%
\pgfpathlineto{\pgfqpoint{6.765408in}{8.623611in}}%
\pgfpathlineto{\pgfqpoint{7.656629in}{8.623611in}}%
\pgfpathlineto{\pgfqpoint{7.656629in}{8.623611in}}%
\pgfusepath{stroke}%
\end{pgfscope}%
\begin{pgfscope}%
\pgfpathrectangle{\pgfqpoint{4.616667in}{8.480556in}}{\pgfqpoint{3.184722in}{3.147222in}}%
\pgfusepath{clip}%
\pgfsetrectcap%
\pgfsetroundjoin%
\pgfsetlinewidth{1.505625pt}%
\definecolor{currentstroke}{rgb}{0.172549,0.627451,0.172549}%
\pgfsetstrokecolor{currentstroke}%
\pgfsetdash{}{0pt}%
\pgfpathmoveto{\pgfqpoint{4.761427in}{10.593029in}}%
\pgfpathlineto{\pgfqpoint{6.278168in}{10.592375in}}%
\pgfpathlineto{\pgfqpoint{6.278892in}{10.589421in}}%
\pgfpathlineto{\pgfqpoint{6.279616in}{10.573234in}}%
\pgfpathlineto{\pgfqpoint{6.280340in}{10.487561in}}%
\pgfpathlineto{\pgfqpoint{6.285408in}{8.623611in}}%
\pgfpathlineto{\pgfqpoint{7.656629in}{8.623611in}}%
\pgfpathlineto{\pgfqpoint{7.656629in}{8.623611in}}%
\pgfusepath{stroke}%
\end{pgfscope}%
\begin{pgfscope}%
\pgfpathrectangle{\pgfqpoint{4.616667in}{8.480556in}}{\pgfqpoint{3.184722in}{3.147222in}}%
\pgfusepath{clip}%
\pgfsetrectcap%
\pgfsetroundjoin%
\pgfsetlinewidth{1.505625pt}%
\definecolor{currentstroke}{rgb}{0.839216,0.152941,0.156863}%
\pgfsetstrokecolor{currentstroke}%
\pgfsetdash{}{0pt}%
\pgfpathmoveto{\pgfqpoint{4.761427in}{10.593029in}}%
\pgfpathlineto{\pgfqpoint{6.208666in}{10.593029in}}%
\pgfpathlineto{\pgfqpoint{6.210838in}{8.623611in}}%
\pgfpathlineto{\pgfqpoint{7.656629in}{8.623611in}}%
\pgfpathlineto{\pgfqpoint{7.656629in}{8.623611in}}%
\pgfusepath{stroke}%
\end{pgfscope}%
\begin{pgfscope}%
\pgfpathrectangle{\pgfqpoint{4.616667in}{8.480556in}}{\pgfqpoint{3.184722in}{3.147222in}}%
\pgfusepath{clip}%
\pgfsetrectcap%
\pgfsetroundjoin%
\pgfsetlinewidth{0.501875pt}%
\definecolor{currentstroke}{rgb}{0.000000,0.000000,0.000000}%
\pgfsetstrokecolor{currentstroke}%
\pgfsetdash{}{0pt}%
\pgfpathmoveto{\pgfqpoint{4.761427in}{10.593029in}}%
\pgfpathlineto{\pgfqpoint{6.281064in}{10.593029in}}%
\pgfpathlineto{\pgfqpoint{6.283236in}{8.623611in}}%
\pgfpathlineto{\pgfqpoint{7.656629in}{8.623611in}}%
\pgfpathlineto{\pgfqpoint{7.656629in}{8.623611in}}%
\pgfusepath{stroke}%
\end{pgfscope}%
\begin{pgfscope}%
\pgfsetrectcap%
\pgfsetmiterjoin%
\pgfsetlinewidth{0.803000pt}%
\definecolor{currentstroke}{rgb}{0.000000,0.000000,0.000000}%
\pgfsetstrokecolor{currentstroke}%
\pgfsetdash{}{0pt}%
\pgfpathmoveto{\pgfqpoint{4.616667in}{8.480556in}}%
\pgfpathlineto{\pgfqpoint{4.616667in}{11.627778in}}%
\pgfusepath{stroke}%
\end{pgfscope}%
\begin{pgfscope}%
\pgfsetrectcap%
\pgfsetmiterjoin%
\pgfsetlinewidth{0.803000pt}%
\definecolor{currentstroke}{rgb}{0.000000,0.000000,0.000000}%
\pgfsetstrokecolor{currentstroke}%
\pgfsetdash{}{0pt}%
\pgfpathmoveto{\pgfqpoint{7.801389in}{8.480556in}}%
\pgfpathlineto{\pgfqpoint{7.801389in}{11.627778in}}%
\pgfusepath{stroke}%
\end{pgfscope}%
\begin{pgfscope}%
\pgfsetrectcap%
\pgfsetmiterjoin%
\pgfsetlinewidth{0.803000pt}%
\definecolor{currentstroke}{rgb}{0.000000,0.000000,0.000000}%
\pgfsetstrokecolor{currentstroke}%
\pgfsetdash{}{0pt}%
\pgfpathmoveto{\pgfqpoint{4.616667in}{8.480556in}}%
\pgfpathlineto{\pgfqpoint{7.801389in}{8.480556in}}%
\pgfusepath{stroke}%
\end{pgfscope}%
\begin{pgfscope}%
\pgfsetrectcap%
\pgfsetmiterjoin%
\pgfsetlinewidth{0.803000pt}%
\definecolor{currentstroke}{rgb}{0.000000,0.000000,0.000000}%
\pgfsetstrokecolor{currentstroke}%
\pgfsetdash{}{0pt}%
\pgfpathmoveto{\pgfqpoint{4.616667in}{11.627778in}}%
\pgfpathlineto{\pgfqpoint{7.801389in}{11.627778in}}%
\pgfusepath{stroke}%
\end{pgfscope}%
\begin{pgfscope}%
\definecolor{textcolor}{rgb}{0.000000,0.000000,0.000000}%
\pgfsetstrokecolor{textcolor}%
\pgfsetfillcolor{textcolor}%
\pgftext[x=6.209028in,y=11.711111in,,base]{\color{textcolor}\sffamily\fontsize{12.000000}{14.400000}\selectfont \(\displaystyle  t = 0.2 \)}%
\end{pgfscope}%
\begin{pgfscope}%
\pgfsetbuttcap%
\pgfsetmiterjoin%
\definecolor{currentfill}{rgb}{1.000000,1.000000,1.000000}%
\pgfsetfillcolor{currentfill}%
\pgfsetlinewidth{0.000000pt}%
\definecolor{currentstroke}{rgb}{0.000000,0.000000,0.000000}%
\pgfsetstrokecolor{currentstroke}%
\pgfsetstrokeopacity{0.000000}%
\pgfsetdash{}{0pt}%
\pgfpathmoveto{\pgfqpoint{0.691667in}{4.530556in}}%
\pgfpathlineto{\pgfqpoint{3.876389in}{4.530556in}}%
\pgfpathlineto{\pgfqpoint{3.876389in}{7.677778in}}%
\pgfpathlineto{\pgfqpoint{0.691667in}{7.677778in}}%
\pgfpathclose%
\pgfusepath{fill}%
\end{pgfscope}%
\begin{pgfscope}%
\pgfsetbuttcap%
\pgfsetroundjoin%
\definecolor{currentfill}{rgb}{0.000000,0.000000,0.000000}%
\pgfsetfillcolor{currentfill}%
\pgfsetlinewidth{0.803000pt}%
\definecolor{currentstroke}{rgb}{0.000000,0.000000,0.000000}%
\pgfsetstrokecolor{currentstroke}%
\pgfsetdash{}{0pt}%
\pgfsys@defobject{currentmarker}{\pgfqpoint{0.000000in}{-0.048611in}}{\pgfqpoint{0.000000in}{0.000000in}}{%
\pgfpathmoveto{\pgfqpoint{0.000000in}{0.000000in}}%
\pgfpathlineto{\pgfqpoint{0.000000in}{-0.048611in}}%
\pgfusepath{stroke,fill}%
}%
\begin{pgfscope}%
\pgfsys@transformshift{0.836065in}{4.530556in}%
\pgfsys@useobject{currentmarker}{}%
\end{pgfscope}%
\end{pgfscope}%
\begin{pgfscope}%
\definecolor{textcolor}{rgb}{0.000000,0.000000,0.000000}%
\pgfsetstrokecolor{textcolor}%
\pgfsetfillcolor{textcolor}%
\pgftext[x=0.836065in,y=4.433333in,,top]{\color{textcolor}\sffamily\fontsize{10.000000}{12.000000}\selectfont −2}%
\end{pgfscope}%
\begin{pgfscope}%
\pgfsetbuttcap%
\pgfsetroundjoin%
\definecolor{currentfill}{rgb}{0.000000,0.000000,0.000000}%
\pgfsetfillcolor{currentfill}%
\pgfsetlinewidth{0.803000pt}%
\definecolor{currentstroke}{rgb}{0.000000,0.000000,0.000000}%
\pgfsetstrokecolor{currentstroke}%
\pgfsetdash{}{0pt}%
\pgfsys@defobject{currentmarker}{\pgfqpoint{0.000000in}{-0.048611in}}{\pgfqpoint{0.000000in}{0.000000in}}{%
\pgfpathmoveto{\pgfqpoint{0.000000in}{0.000000in}}%
\pgfpathlineto{\pgfqpoint{0.000000in}{-0.048611in}}%
\pgfusepath{stroke,fill}%
}%
\begin{pgfscope}%
\pgfsys@transformshift{1.560046in}{4.530556in}%
\pgfsys@useobject{currentmarker}{}%
\end{pgfscope}%
\end{pgfscope}%
\begin{pgfscope}%
\definecolor{textcolor}{rgb}{0.000000,0.000000,0.000000}%
\pgfsetstrokecolor{textcolor}%
\pgfsetfillcolor{textcolor}%
\pgftext[x=1.560046in,y=4.433333in,,top]{\color{textcolor}\sffamily\fontsize{10.000000}{12.000000}\selectfont −1}%
\end{pgfscope}%
\begin{pgfscope}%
\pgfsetbuttcap%
\pgfsetroundjoin%
\definecolor{currentfill}{rgb}{0.000000,0.000000,0.000000}%
\pgfsetfillcolor{currentfill}%
\pgfsetlinewidth{0.803000pt}%
\definecolor{currentstroke}{rgb}{0.000000,0.000000,0.000000}%
\pgfsetstrokecolor{currentstroke}%
\pgfsetdash{}{0pt}%
\pgfsys@defobject{currentmarker}{\pgfqpoint{0.000000in}{-0.048611in}}{\pgfqpoint{0.000000in}{0.000000in}}{%
\pgfpathmoveto{\pgfqpoint{0.000000in}{0.000000in}}%
\pgfpathlineto{\pgfqpoint{0.000000in}{-0.048611in}}%
\pgfusepath{stroke,fill}%
}%
\begin{pgfscope}%
\pgfsys@transformshift{2.284028in}{4.530556in}%
\pgfsys@useobject{currentmarker}{}%
\end{pgfscope}%
\end{pgfscope}%
\begin{pgfscope}%
\definecolor{textcolor}{rgb}{0.000000,0.000000,0.000000}%
\pgfsetstrokecolor{textcolor}%
\pgfsetfillcolor{textcolor}%
\pgftext[x=2.284028in,y=4.433333in,,top]{\color{textcolor}\sffamily\fontsize{10.000000}{12.000000}\selectfont 0}%
\end{pgfscope}%
\begin{pgfscope}%
\pgfsetbuttcap%
\pgfsetroundjoin%
\definecolor{currentfill}{rgb}{0.000000,0.000000,0.000000}%
\pgfsetfillcolor{currentfill}%
\pgfsetlinewidth{0.803000pt}%
\definecolor{currentstroke}{rgb}{0.000000,0.000000,0.000000}%
\pgfsetstrokecolor{currentstroke}%
\pgfsetdash{}{0pt}%
\pgfsys@defobject{currentmarker}{\pgfqpoint{0.000000in}{-0.048611in}}{\pgfqpoint{0.000000in}{0.000000in}}{%
\pgfpathmoveto{\pgfqpoint{0.000000in}{0.000000in}}%
\pgfpathlineto{\pgfqpoint{0.000000in}{-0.048611in}}%
\pgfusepath{stroke,fill}%
}%
\begin{pgfscope}%
\pgfsys@transformshift{3.008009in}{4.530556in}%
\pgfsys@useobject{currentmarker}{}%
\end{pgfscope}%
\end{pgfscope}%
\begin{pgfscope}%
\definecolor{textcolor}{rgb}{0.000000,0.000000,0.000000}%
\pgfsetstrokecolor{textcolor}%
\pgfsetfillcolor{textcolor}%
\pgftext[x=3.008009in,y=4.433333in,,top]{\color{textcolor}\sffamily\fontsize{10.000000}{12.000000}\selectfont 1}%
\end{pgfscope}%
\begin{pgfscope}%
\pgfsetbuttcap%
\pgfsetroundjoin%
\definecolor{currentfill}{rgb}{0.000000,0.000000,0.000000}%
\pgfsetfillcolor{currentfill}%
\pgfsetlinewidth{0.803000pt}%
\definecolor{currentstroke}{rgb}{0.000000,0.000000,0.000000}%
\pgfsetstrokecolor{currentstroke}%
\pgfsetdash{}{0pt}%
\pgfsys@defobject{currentmarker}{\pgfqpoint{0.000000in}{-0.048611in}}{\pgfqpoint{0.000000in}{0.000000in}}{%
\pgfpathmoveto{\pgfqpoint{0.000000in}{0.000000in}}%
\pgfpathlineto{\pgfqpoint{0.000000in}{-0.048611in}}%
\pgfusepath{stroke,fill}%
}%
\begin{pgfscope}%
\pgfsys@transformshift{3.731991in}{4.530556in}%
\pgfsys@useobject{currentmarker}{}%
\end{pgfscope}%
\end{pgfscope}%
\begin{pgfscope}%
\definecolor{textcolor}{rgb}{0.000000,0.000000,0.000000}%
\pgfsetstrokecolor{textcolor}%
\pgfsetfillcolor{textcolor}%
\pgftext[x=3.731991in,y=4.433333in,,top]{\color{textcolor}\sffamily\fontsize{10.000000}{12.000000}\selectfont 2}%
\end{pgfscope}%
\begin{pgfscope}%
\definecolor{textcolor}{rgb}{0.000000,0.000000,0.000000}%
\pgfsetstrokecolor{textcolor}%
\pgfsetfillcolor{textcolor}%
\pgftext[x=2.284028in,y=4.243365in,,top]{\color{textcolor}\sffamily\fontsize{10.000000}{12.000000}\selectfont \(\displaystyle x\)}%
\end{pgfscope}%
\begin{pgfscope}%
\pgfsetbuttcap%
\pgfsetroundjoin%
\definecolor{currentfill}{rgb}{0.000000,0.000000,0.000000}%
\pgfsetfillcolor{currentfill}%
\pgfsetlinewidth{0.803000pt}%
\definecolor{currentstroke}{rgb}{0.000000,0.000000,0.000000}%
\pgfsetstrokecolor{currentstroke}%
\pgfsetdash{}{0pt}%
\pgfsys@defobject{currentmarker}{\pgfqpoint{-0.048611in}{0.000000in}}{\pgfqpoint{0.000000in}{0.000000in}}{%
\pgfpathmoveto{\pgfqpoint{0.000000in}{0.000000in}}%
\pgfpathlineto{\pgfqpoint{-0.048611in}{0.000000in}}%
\pgfusepath{stroke,fill}%
}%
\begin{pgfscope}%
\pgfsys@transformshift{0.691667in}{4.673611in}%
\pgfsys@useobject{currentmarker}{}%
\end{pgfscope}%
\end{pgfscope}%
\begin{pgfscope}%
\definecolor{textcolor}{rgb}{0.000000,0.000000,0.000000}%
\pgfsetstrokecolor{textcolor}%
\pgfsetfillcolor{textcolor}%
\pgftext[x=0.373565in,y=4.620850in,left,base]{\color{textcolor}\sffamily\fontsize{10.000000}{12.000000}\selectfont 0.0}%
\end{pgfscope}%
\begin{pgfscope}%
\pgfsetbuttcap%
\pgfsetroundjoin%
\definecolor{currentfill}{rgb}{0.000000,0.000000,0.000000}%
\pgfsetfillcolor{currentfill}%
\pgfsetlinewidth{0.803000pt}%
\definecolor{currentstroke}{rgb}{0.000000,0.000000,0.000000}%
\pgfsetstrokecolor{currentstroke}%
\pgfsetdash{}{0pt}%
\pgfsys@defobject{currentmarker}{\pgfqpoint{-0.048611in}{0.000000in}}{\pgfqpoint{0.000000in}{0.000000in}}{%
\pgfpathmoveto{\pgfqpoint{0.000000in}{0.000000in}}%
\pgfpathlineto{\pgfqpoint{-0.048611in}{0.000000in}}%
\pgfusepath{stroke,fill}%
}%
\begin{pgfscope}%
\pgfsys@transformshift{0.691667in}{5.067495in}%
\pgfsys@useobject{currentmarker}{}%
\end{pgfscope}%
\end{pgfscope}%
\begin{pgfscope}%
\definecolor{textcolor}{rgb}{0.000000,0.000000,0.000000}%
\pgfsetstrokecolor{textcolor}%
\pgfsetfillcolor{textcolor}%
\pgftext[x=0.373565in,y=5.014733in,left,base]{\color{textcolor}\sffamily\fontsize{10.000000}{12.000000}\selectfont 0.2}%
\end{pgfscope}%
\begin{pgfscope}%
\pgfsetbuttcap%
\pgfsetroundjoin%
\definecolor{currentfill}{rgb}{0.000000,0.000000,0.000000}%
\pgfsetfillcolor{currentfill}%
\pgfsetlinewidth{0.803000pt}%
\definecolor{currentstroke}{rgb}{0.000000,0.000000,0.000000}%
\pgfsetstrokecolor{currentstroke}%
\pgfsetdash{}{0pt}%
\pgfsys@defobject{currentmarker}{\pgfqpoint{-0.048611in}{0.000000in}}{\pgfqpoint{0.000000in}{0.000000in}}{%
\pgfpathmoveto{\pgfqpoint{0.000000in}{0.000000in}}%
\pgfpathlineto{\pgfqpoint{-0.048611in}{0.000000in}}%
\pgfusepath{stroke,fill}%
}%
\begin{pgfscope}%
\pgfsys@transformshift{0.691667in}{5.461378in}%
\pgfsys@useobject{currentmarker}{}%
\end{pgfscope}%
\end{pgfscope}%
\begin{pgfscope}%
\definecolor{textcolor}{rgb}{0.000000,0.000000,0.000000}%
\pgfsetstrokecolor{textcolor}%
\pgfsetfillcolor{textcolor}%
\pgftext[x=0.373565in,y=5.408617in,left,base]{\color{textcolor}\sffamily\fontsize{10.000000}{12.000000}\selectfont 0.4}%
\end{pgfscope}%
\begin{pgfscope}%
\pgfsetbuttcap%
\pgfsetroundjoin%
\definecolor{currentfill}{rgb}{0.000000,0.000000,0.000000}%
\pgfsetfillcolor{currentfill}%
\pgfsetlinewidth{0.803000pt}%
\definecolor{currentstroke}{rgb}{0.000000,0.000000,0.000000}%
\pgfsetstrokecolor{currentstroke}%
\pgfsetdash{}{0pt}%
\pgfsys@defobject{currentmarker}{\pgfqpoint{-0.048611in}{0.000000in}}{\pgfqpoint{0.000000in}{0.000000in}}{%
\pgfpathmoveto{\pgfqpoint{0.000000in}{0.000000in}}%
\pgfpathlineto{\pgfqpoint{-0.048611in}{0.000000in}}%
\pgfusepath{stroke,fill}%
}%
\begin{pgfscope}%
\pgfsys@transformshift{0.691667in}{5.855262in}%
\pgfsys@useobject{currentmarker}{}%
\end{pgfscope}%
\end{pgfscope}%
\begin{pgfscope}%
\definecolor{textcolor}{rgb}{0.000000,0.000000,0.000000}%
\pgfsetstrokecolor{textcolor}%
\pgfsetfillcolor{textcolor}%
\pgftext[x=0.373565in,y=5.802500in,left,base]{\color{textcolor}\sffamily\fontsize{10.000000}{12.000000}\selectfont 0.6}%
\end{pgfscope}%
\begin{pgfscope}%
\pgfsetbuttcap%
\pgfsetroundjoin%
\definecolor{currentfill}{rgb}{0.000000,0.000000,0.000000}%
\pgfsetfillcolor{currentfill}%
\pgfsetlinewidth{0.803000pt}%
\definecolor{currentstroke}{rgb}{0.000000,0.000000,0.000000}%
\pgfsetstrokecolor{currentstroke}%
\pgfsetdash{}{0pt}%
\pgfsys@defobject{currentmarker}{\pgfqpoint{-0.048611in}{0.000000in}}{\pgfqpoint{0.000000in}{0.000000in}}{%
\pgfpathmoveto{\pgfqpoint{0.000000in}{0.000000in}}%
\pgfpathlineto{\pgfqpoint{-0.048611in}{0.000000in}}%
\pgfusepath{stroke,fill}%
}%
\begin{pgfscope}%
\pgfsys@transformshift{0.691667in}{6.249146in}%
\pgfsys@useobject{currentmarker}{}%
\end{pgfscope}%
\end{pgfscope}%
\begin{pgfscope}%
\definecolor{textcolor}{rgb}{0.000000,0.000000,0.000000}%
\pgfsetstrokecolor{textcolor}%
\pgfsetfillcolor{textcolor}%
\pgftext[x=0.373565in,y=6.196384in,left,base]{\color{textcolor}\sffamily\fontsize{10.000000}{12.000000}\selectfont 0.8}%
\end{pgfscope}%
\begin{pgfscope}%
\pgfsetbuttcap%
\pgfsetroundjoin%
\definecolor{currentfill}{rgb}{0.000000,0.000000,0.000000}%
\pgfsetfillcolor{currentfill}%
\pgfsetlinewidth{0.803000pt}%
\definecolor{currentstroke}{rgb}{0.000000,0.000000,0.000000}%
\pgfsetstrokecolor{currentstroke}%
\pgfsetdash{}{0pt}%
\pgfsys@defobject{currentmarker}{\pgfqpoint{-0.048611in}{0.000000in}}{\pgfqpoint{0.000000in}{0.000000in}}{%
\pgfpathmoveto{\pgfqpoint{0.000000in}{0.000000in}}%
\pgfpathlineto{\pgfqpoint{-0.048611in}{0.000000in}}%
\pgfusepath{stroke,fill}%
}%
\begin{pgfscope}%
\pgfsys@transformshift{0.691667in}{6.643029in}%
\pgfsys@useobject{currentmarker}{}%
\end{pgfscope}%
\end{pgfscope}%
\begin{pgfscope}%
\definecolor{textcolor}{rgb}{0.000000,0.000000,0.000000}%
\pgfsetstrokecolor{textcolor}%
\pgfsetfillcolor{textcolor}%
\pgftext[x=0.373565in,y=6.590268in,left,base]{\color{textcolor}\sffamily\fontsize{10.000000}{12.000000}\selectfont 1.0}%
\end{pgfscope}%
\begin{pgfscope}%
\pgfsetbuttcap%
\pgfsetroundjoin%
\definecolor{currentfill}{rgb}{0.000000,0.000000,0.000000}%
\pgfsetfillcolor{currentfill}%
\pgfsetlinewidth{0.803000pt}%
\definecolor{currentstroke}{rgb}{0.000000,0.000000,0.000000}%
\pgfsetstrokecolor{currentstroke}%
\pgfsetdash{}{0pt}%
\pgfsys@defobject{currentmarker}{\pgfqpoint{-0.048611in}{0.000000in}}{\pgfqpoint{0.000000in}{0.000000in}}{%
\pgfpathmoveto{\pgfqpoint{0.000000in}{0.000000in}}%
\pgfpathlineto{\pgfqpoint{-0.048611in}{0.000000in}}%
\pgfusepath{stroke,fill}%
}%
\begin{pgfscope}%
\pgfsys@transformshift{0.691667in}{7.036913in}%
\pgfsys@useobject{currentmarker}{}%
\end{pgfscope}%
\end{pgfscope}%
\begin{pgfscope}%
\definecolor{textcolor}{rgb}{0.000000,0.000000,0.000000}%
\pgfsetstrokecolor{textcolor}%
\pgfsetfillcolor{textcolor}%
\pgftext[x=0.373565in,y=6.984151in,left,base]{\color{textcolor}\sffamily\fontsize{10.000000}{12.000000}\selectfont 1.2}%
\end{pgfscope}%
\begin{pgfscope}%
\pgfsetbuttcap%
\pgfsetroundjoin%
\definecolor{currentfill}{rgb}{0.000000,0.000000,0.000000}%
\pgfsetfillcolor{currentfill}%
\pgfsetlinewidth{0.803000pt}%
\definecolor{currentstroke}{rgb}{0.000000,0.000000,0.000000}%
\pgfsetstrokecolor{currentstroke}%
\pgfsetdash{}{0pt}%
\pgfsys@defobject{currentmarker}{\pgfqpoint{-0.048611in}{0.000000in}}{\pgfqpoint{0.000000in}{0.000000in}}{%
\pgfpathmoveto{\pgfqpoint{0.000000in}{0.000000in}}%
\pgfpathlineto{\pgfqpoint{-0.048611in}{0.000000in}}%
\pgfusepath{stroke,fill}%
}%
\begin{pgfscope}%
\pgfsys@transformshift{0.691667in}{7.430797in}%
\pgfsys@useobject{currentmarker}{}%
\end{pgfscope}%
\end{pgfscope}%
\begin{pgfscope}%
\definecolor{textcolor}{rgb}{0.000000,0.000000,0.000000}%
\pgfsetstrokecolor{textcolor}%
\pgfsetfillcolor{textcolor}%
\pgftext[x=0.373565in,y=7.378035in,left,base]{\color{textcolor}\sffamily\fontsize{10.000000}{12.000000}\selectfont 1.4}%
\end{pgfscope}%
\begin{pgfscope}%
\definecolor{textcolor}{rgb}{0.000000,0.000000,0.000000}%
\pgfsetstrokecolor{textcolor}%
\pgfsetfillcolor{textcolor}%
\pgftext[x=0.318009in,y=6.104167in,,bottom,rotate=90.000000]{\color{textcolor}\sffamily\fontsize{10.000000}{12.000000}\selectfont \(\displaystyle U\)}%
\end{pgfscope}%
\begin{pgfscope}%
\pgfpathrectangle{\pgfqpoint{0.691667in}{4.530556in}}{\pgfqpoint{3.184722in}{3.147222in}}%
\pgfusepath{clip}%
\pgfsetrectcap%
\pgfsetroundjoin%
\pgfsetlinewidth{1.505625pt}%
\definecolor{currentstroke}{rgb}{0.121569,0.466667,0.705882}%
\pgfsetstrokecolor{currentstroke}%
\pgfsetdash{}{0pt}%
\pgfpathmoveto{\pgfqpoint{0.836427in}{6.643029in}}%
\pgfpathlineto{\pgfqpoint{2.382851in}{6.643034in}}%
\pgfpathlineto{\pgfqpoint{2.383575in}{6.641154in}}%
\pgfpathlineto{\pgfqpoint{2.385023in}{6.644398in}}%
\pgfpathlineto{\pgfqpoint{2.385747in}{6.640401in}}%
\pgfpathlineto{\pgfqpoint{2.386471in}{6.643210in}}%
\pgfpathlineto{\pgfqpoint{2.387195in}{6.646151in}}%
\pgfpathlineto{\pgfqpoint{2.387919in}{6.640572in}}%
\pgfpathlineto{\pgfqpoint{2.388643in}{6.640917in}}%
\pgfpathlineto{\pgfqpoint{2.389367in}{6.647577in}}%
\pgfpathlineto{\pgfqpoint{2.390091in}{6.642389in}}%
\pgfpathlineto{\pgfqpoint{2.390815in}{6.637851in}}%
\pgfpathlineto{\pgfqpoint{2.391539in}{6.647511in}}%
\pgfpathlineto{\pgfqpoint{2.392263in}{6.646228in}}%
\pgfpathlineto{\pgfqpoint{2.392987in}{6.635184in}}%
\pgfpathlineto{\pgfqpoint{2.393711in}{6.644656in}}%
\pgfpathlineto{\pgfqpoint{2.394435in}{6.651563in}}%
\pgfpathlineto{\pgfqpoint{2.395159in}{6.634894in}}%
\pgfpathlineto{\pgfqpoint{2.395883in}{6.638248in}}%
\pgfpathlineto{\pgfqpoint{2.396607in}{6.656484in}}%
\pgfpathlineto{\pgfqpoint{2.397331in}{6.639285in}}%
\pgfpathlineto{\pgfqpoint{2.398055in}{6.629000in}}%
\pgfpathlineto{\pgfqpoint{2.398779in}{6.657611in}}%
\pgfpathlineto{\pgfqpoint{2.399503in}{6.649893in}}%
\pgfpathlineto{\pgfqpoint{2.400227in}{6.619981in}}%
\pgfpathlineto{\pgfqpoint{2.400951in}{6.650845in}}%
\pgfpathlineto{\pgfqpoint{2.401675in}{6.665839in}}%
\pgfpathlineto{\pgfqpoint{2.402399in}{6.616849in}}%
\pgfpathlineto{\pgfqpoint{2.403123in}{6.633235in}}%
\pgfpathlineto{\pgfqpoint{2.403847in}{6.682166in}}%
\pgfpathlineto{\pgfqpoint{2.405295in}{6.605785in}}%
\pgfpathlineto{\pgfqpoint{2.406019in}{6.689154in}}%
\pgfpathlineto{\pgfqpoint{2.406743in}{6.655439in}}%
\pgfpathlineto{\pgfqpoint{2.407467in}{6.576297in}}%
\pgfpathlineto{\pgfqpoint{2.408915in}{6.702299in}}%
\pgfpathlineto{\pgfqpoint{2.409639in}{6.560625in}}%
\pgfpathlineto{\pgfqpoint{2.410363in}{6.626195in}}%
\pgfpathlineto{\pgfqpoint{2.411087in}{6.754823in}}%
\pgfpathlineto{\pgfqpoint{2.412534in}{6.545623in}}%
\pgfpathlineto{\pgfqpoint{2.413258in}{6.784886in}}%
\pgfpathlineto{\pgfqpoint{2.413982in}{6.656012in}}%
\pgfpathlineto{\pgfqpoint{2.414706in}{6.451765in}}%
\pgfpathlineto{\pgfqpoint{2.416154in}{6.791900in}}%
\pgfpathlineto{\pgfqpoint{2.416878in}{6.388545in}}%
\pgfpathlineto{\pgfqpoint{2.417602in}{6.618846in}}%
\pgfpathlineto{\pgfqpoint{2.418326in}{6.956521in}}%
\pgfpathlineto{\pgfqpoint{2.419774in}{6.378284in}}%
\pgfpathlineto{\pgfqpoint{2.420498in}{7.059961in}}%
\pgfpathlineto{\pgfqpoint{2.421222in}{6.624656in}}%
\pgfpathlineto{\pgfqpoint{2.421946in}{6.085242in}}%
\pgfpathlineto{\pgfqpoint{2.423394in}{7.036454in}}%
\pgfpathlineto{\pgfqpoint{2.424118in}{5.882789in}}%
\pgfpathlineto{\pgfqpoint{2.424842in}{6.467760in}}%
\pgfpathlineto{\pgfqpoint{2.425566in}{7.534722in}}%
\pgfpathlineto{\pgfqpoint{2.427014in}{5.609499in}}%
\pgfpathlineto{\pgfqpoint{2.427738in}{7.497630in}}%
\pgfpathlineto{\pgfqpoint{2.428462in}{7.022202in}}%
\pgfpathlineto{\pgfqpoint{2.430634in}{4.673611in}}%
\pgfpathlineto{\pgfqpoint{3.731629in}{4.673611in}}%
\pgfpathlineto{\pgfqpoint{3.731629in}{4.673611in}}%
\pgfusepath{stroke}%
\end{pgfscope}%
\begin{pgfscope}%
\pgfpathrectangle{\pgfqpoint{0.691667in}{4.530556in}}{\pgfqpoint{3.184722in}{3.147222in}}%
\pgfusepath{clip}%
\pgfsetrectcap%
\pgfsetroundjoin%
\pgfsetlinewidth{1.505625pt}%
\definecolor{currentstroke}{rgb}{1.000000,0.498039,0.054902}%
\pgfsetstrokecolor{currentstroke}%
\pgfsetdash{}{0pt}%
\pgfpathmoveto{\pgfqpoint{0.836427in}{6.643029in}}%
\pgfpathlineto{\pgfqpoint{2.385023in}{6.643881in}}%
\pgfpathlineto{\pgfqpoint{2.385747in}{6.641442in}}%
\pgfpathlineto{\pgfqpoint{2.386471in}{6.643106in}}%
\pgfpathlineto{\pgfqpoint{2.387195in}{6.644936in}}%
\pgfpathlineto{\pgfqpoint{2.388643in}{6.641708in}}%
\pgfpathlineto{\pgfqpoint{2.389367in}{6.645776in}}%
\pgfpathlineto{\pgfqpoint{2.390091in}{6.642697in}}%
\pgfpathlineto{\pgfqpoint{2.390815in}{6.639859in}}%
\pgfpathlineto{\pgfqpoint{2.391539in}{6.645697in}}%
\pgfpathlineto{\pgfqpoint{2.392263in}{6.645038in}}%
\pgfpathlineto{\pgfqpoint{2.392987in}{6.638278in}}%
\pgfpathlineto{\pgfqpoint{2.393711in}{6.643921in}}%
\pgfpathlineto{\pgfqpoint{2.394435in}{6.648253in}}%
\pgfpathlineto{\pgfqpoint{2.395159in}{6.638168in}}%
\pgfpathlineto{\pgfqpoint{2.395883in}{6.640005in}}%
\pgfpathlineto{\pgfqpoint{2.396607in}{6.651167in}}%
\pgfpathlineto{\pgfqpoint{2.397331in}{6.640913in}}%
\pgfpathlineto{\pgfqpoint{2.398055in}{6.634404in}}%
\pgfpathlineto{\pgfqpoint{2.398779in}{6.651734in}}%
\pgfpathlineto{\pgfqpoint{2.399503in}{6.647398in}}%
\pgfpathlineto{\pgfqpoint{2.400227in}{6.629011in}}%
\pgfpathlineto{\pgfqpoint{2.400951in}{6.647496in}}%
\pgfpathlineto{\pgfqpoint{2.401675in}{6.657010in}}%
\pgfpathlineto{\pgfqpoint{2.402399in}{6.627290in}}%
\pgfpathlineto{\pgfqpoint{2.403123in}{6.636699in}}%
\pgfpathlineto{\pgfqpoint{2.403847in}{6.666680in}}%
\pgfpathlineto{\pgfqpoint{2.404571in}{6.633570in}}%
\pgfpathlineto{\pgfqpoint{2.405295in}{6.619979in}}%
\pgfpathlineto{\pgfqpoint{2.406019in}{6.670546in}}%
\pgfpathlineto{\pgfqpoint{2.406743in}{6.651152in}}%
\pgfpathlineto{\pgfqpoint{2.407467in}{6.602133in}}%
\pgfpathlineto{\pgfqpoint{2.408915in}{6.679319in}}%
\pgfpathlineto{\pgfqpoint{2.409639in}{6.592990in}}%
\pgfpathlineto{\pgfqpoint{2.410363in}{6.631578in}}%
\pgfpathlineto{\pgfqpoint{2.411087in}{6.710151in}}%
\pgfpathlineto{\pgfqpoint{2.412534in}{6.581958in}}%
\pgfpathlineto{\pgfqpoint{2.413258in}{6.726886in}}%
\pgfpathlineto{\pgfqpoint{2.413982in}{6.652550in}}%
\pgfpathlineto{\pgfqpoint{2.414706in}{6.523704in}}%
\pgfpathlineto{\pgfqpoint{2.416154in}{6.733497in}}%
\pgfpathlineto{\pgfqpoint{2.416878in}{6.484815in}}%
\pgfpathlineto{\pgfqpoint{2.417602in}{6.623871in}}%
\pgfpathlineto{\pgfqpoint{2.418326in}{6.827195in}}%
\pgfpathlineto{\pgfqpoint{2.419774in}{6.471709in}}%
\pgfpathlineto{\pgfqpoint{2.420498in}{6.881472in}}%
\pgfpathlineto{\pgfqpoint{2.421222in}{6.636583in}}%
\pgfpathlineto{\pgfqpoint{2.421946in}{6.276899in}}%
\pgfpathlineto{\pgfqpoint{2.423394in}{6.876435in}}%
\pgfpathlineto{\pgfqpoint{2.424118in}{6.138624in}}%
\pgfpathlineto{\pgfqpoint{2.424842in}{6.509935in}}%
\pgfpathlineto{\pgfqpoint{2.425566in}{7.132632in}}%
\pgfpathlineto{\pgfqpoint{2.427014in}{5.890603in}}%
\pgfpathlineto{\pgfqpoint{2.427738in}{7.075063in}}%
\pgfpathlineto{\pgfqpoint{2.428462in}{6.893344in}}%
\pgfpathlineto{\pgfqpoint{2.430634in}{4.673613in}}%
\pgfpathlineto{\pgfqpoint{2.912806in}{4.673611in}}%
\pgfpathlineto{\pgfqpoint{3.731629in}{4.673611in}}%
\pgfpathlineto{\pgfqpoint{3.731629in}{4.673611in}}%
\pgfusepath{stroke}%
\end{pgfscope}%
\begin{pgfscope}%
\pgfpathrectangle{\pgfqpoint{0.691667in}{4.530556in}}{\pgfqpoint{3.184722in}{3.147222in}}%
\pgfusepath{clip}%
\pgfsetrectcap%
\pgfsetroundjoin%
\pgfsetlinewidth{1.505625pt}%
\definecolor{currentstroke}{rgb}{0.172549,0.627451,0.172549}%
\pgfsetstrokecolor{currentstroke}%
\pgfsetdash{}{0pt}%
\pgfpathmoveto{\pgfqpoint{0.836427in}{6.643029in}}%
\pgfpathlineto{\pgfqpoint{2.425566in}{6.642375in}}%
\pgfpathlineto{\pgfqpoint{2.426290in}{6.639421in}}%
\pgfpathlineto{\pgfqpoint{2.427014in}{6.623234in}}%
\pgfpathlineto{\pgfqpoint{2.427738in}{6.537561in}}%
\pgfpathlineto{\pgfqpoint{2.432806in}{4.673611in}}%
\pgfpathlineto{\pgfqpoint{3.731629in}{4.673611in}}%
\pgfpathlineto{\pgfqpoint{3.731629in}{4.673611in}}%
\pgfusepath{stroke}%
\end{pgfscope}%
\begin{pgfscope}%
\pgfpathrectangle{\pgfqpoint{0.691667in}{4.530556in}}{\pgfqpoint{3.184722in}{3.147222in}}%
\pgfusepath{clip}%
\pgfsetrectcap%
\pgfsetroundjoin%
\pgfsetlinewidth{1.505625pt}%
\definecolor{currentstroke}{rgb}{0.839216,0.152941,0.156863}%
\pgfsetstrokecolor{currentstroke}%
\pgfsetdash{}{0pt}%
\pgfpathmoveto{\pgfqpoint{0.836427in}{6.643029in}}%
\pgfpathlineto{\pgfqpoint{2.283666in}{6.643029in}}%
\pgfpathlineto{\pgfqpoint{2.285838in}{4.673611in}}%
\pgfpathlineto{\pgfqpoint{3.731629in}{4.673611in}}%
\pgfpathlineto{\pgfqpoint{3.731629in}{4.673611in}}%
\pgfusepath{stroke}%
\end{pgfscope}%
\begin{pgfscope}%
\pgfpathrectangle{\pgfqpoint{0.691667in}{4.530556in}}{\pgfqpoint{3.184722in}{3.147222in}}%
\pgfusepath{clip}%
\pgfsetrectcap%
\pgfsetroundjoin%
\pgfsetlinewidth{0.501875pt}%
\definecolor{currentstroke}{rgb}{0.000000,0.000000,0.000000}%
\pgfsetstrokecolor{currentstroke}%
\pgfsetdash{}{0pt}%
\pgfpathmoveto{\pgfqpoint{0.836427in}{6.643029in}}%
\pgfpathlineto{\pgfqpoint{2.428462in}{6.643029in}}%
\pgfpathlineto{\pgfqpoint{2.430634in}{4.673611in}}%
\pgfpathlineto{\pgfqpoint{3.731629in}{4.673611in}}%
\pgfpathlineto{\pgfqpoint{3.731629in}{4.673611in}}%
\pgfusepath{stroke}%
\end{pgfscope}%
\begin{pgfscope}%
\pgfsetrectcap%
\pgfsetmiterjoin%
\pgfsetlinewidth{0.803000pt}%
\definecolor{currentstroke}{rgb}{0.000000,0.000000,0.000000}%
\pgfsetstrokecolor{currentstroke}%
\pgfsetdash{}{0pt}%
\pgfpathmoveto{\pgfqpoint{0.691667in}{4.530556in}}%
\pgfpathlineto{\pgfqpoint{0.691667in}{7.677778in}}%
\pgfusepath{stroke}%
\end{pgfscope}%
\begin{pgfscope}%
\pgfsetrectcap%
\pgfsetmiterjoin%
\pgfsetlinewidth{0.803000pt}%
\definecolor{currentstroke}{rgb}{0.000000,0.000000,0.000000}%
\pgfsetstrokecolor{currentstroke}%
\pgfsetdash{}{0pt}%
\pgfpathmoveto{\pgfqpoint{3.876389in}{4.530556in}}%
\pgfpathlineto{\pgfqpoint{3.876389in}{7.677778in}}%
\pgfusepath{stroke}%
\end{pgfscope}%
\begin{pgfscope}%
\pgfsetrectcap%
\pgfsetmiterjoin%
\pgfsetlinewidth{0.803000pt}%
\definecolor{currentstroke}{rgb}{0.000000,0.000000,0.000000}%
\pgfsetstrokecolor{currentstroke}%
\pgfsetdash{}{0pt}%
\pgfpathmoveto{\pgfqpoint{0.691667in}{4.530556in}}%
\pgfpathlineto{\pgfqpoint{3.876389in}{4.530556in}}%
\pgfusepath{stroke}%
\end{pgfscope}%
\begin{pgfscope}%
\pgfsetrectcap%
\pgfsetmiterjoin%
\pgfsetlinewidth{0.803000pt}%
\definecolor{currentstroke}{rgb}{0.000000,0.000000,0.000000}%
\pgfsetstrokecolor{currentstroke}%
\pgfsetdash{}{0pt}%
\pgfpathmoveto{\pgfqpoint{0.691667in}{7.677778in}}%
\pgfpathlineto{\pgfqpoint{3.876389in}{7.677778in}}%
\pgfusepath{stroke}%
\end{pgfscope}%
\begin{pgfscope}%
\definecolor{textcolor}{rgb}{0.000000,0.000000,0.000000}%
\pgfsetstrokecolor{textcolor}%
\pgfsetfillcolor{textcolor}%
\pgftext[x=2.284028in,y=7.761111in,,base]{\color{textcolor}\sffamily\fontsize{12.000000}{14.400000}\selectfont \(\displaystyle  t = 0.4 \)}%
\end{pgfscope}%
\begin{pgfscope}%
\pgfsetbuttcap%
\pgfsetmiterjoin%
\definecolor{currentfill}{rgb}{1.000000,1.000000,1.000000}%
\pgfsetfillcolor{currentfill}%
\pgfsetlinewidth{0.000000pt}%
\definecolor{currentstroke}{rgb}{0.000000,0.000000,0.000000}%
\pgfsetstrokecolor{currentstroke}%
\pgfsetstrokeopacity{0.000000}%
\pgfsetdash{}{0pt}%
\pgfpathmoveto{\pgfqpoint{4.616667in}{4.530556in}}%
\pgfpathlineto{\pgfqpoint{7.801389in}{4.530556in}}%
\pgfpathlineto{\pgfqpoint{7.801389in}{7.677778in}}%
\pgfpathlineto{\pgfqpoint{4.616667in}{7.677778in}}%
\pgfpathclose%
\pgfusepath{fill}%
\end{pgfscope}%
\begin{pgfscope}%
\pgfsetbuttcap%
\pgfsetroundjoin%
\definecolor{currentfill}{rgb}{0.000000,0.000000,0.000000}%
\pgfsetfillcolor{currentfill}%
\pgfsetlinewidth{0.803000pt}%
\definecolor{currentstroke}{rgb}{0.000000,0.000000,0.000000}%
\pgfsetstrokecolor{currentstroke}%
\pgfsetdash{}{0pt}%
\pgfsys@defobject{currentmarker}{\pgfqpoint{0.000000in}{-0.048611in}}{\pgfqpoint{0.000000in}{0.000000in}}{%
\pgfpathmoveto{\pgfqpoint{0.000000in}{0.000000in}}%
\pgfpathlineto{\pgfqpoint{0.000000in}{-0.048611in}}%
\pgfusepath{stroke,fill}%
}%
\begin{pgfscope}%
\pgfsys@transformshift{4.761065in}{4.530556in}%
\pgfsys@useobject{currentmarker}{}%
\end{pgfscope}%
\end{pgfscope}%
\begin{pgfscope}%
\definecolor{textcolor}{rgb}{0.000000,0.000000,0.000000}%
\pgfsetstrokecolor{textcolor}%
\pgfsetfillcolor{textcolor}%
\pgftext[x=4.761065in,y=4.433333in,,top]{\color{textcolor}\sffamily\fontsize{10.000000}{12.000000}\selectfont −2}%
\end{pgfscope}%
\begin{pgfscope}%
\pgfsetbuttcap%
\pgfsetroundjoin%
\definecolor{currentfill}{rgb}{0.000000,0.000000,0.000000}%
\pgfsetfillcolor{currentfill}%
\pgfsetlinewidth{0.803000pt}%
\definecolor{currentstroke}{rgb}{0.000000,0.000000,0.000000}%
\pgfsetstrokecolor{currentstroke}%
\pgfsetdash{}{0pt}%
\pgfsys@defobject{currentmarker}{\pgfqpoint{0.000000in}{-0.048611in}}{\pgfqpoint{0.000000in}{0.000000in}}{%
\pgfpathmoveto{\pgfqpoint{0.000000in}{0.000000in}}%
\pgfpathlineto{\pgfqpoint{0.000000in}{-0.048611in}}%
\pgfusepath{stroke,fill}%
}%
\begin{pgfscope}%
\pgfsys@transformshift{5.485046in}{4.530556in}%
\pgfsys@useobject{currentmarker}{}%
\end{pgfscope}%
\end{pgfscope}%
\begin{pgfscope}%
\definecolor{textcolor}{rgb}{0.000000,0.000000,0.000000}%
\pgfsetstrokecolor{textcolor}%
\pgfsetfillcolor{textcolor}%
\pgftext[x=5.485046in,y=4.433333in,,top]{\color{textcolor}\sffamily\fontsize{10.000000}{12.000000}\selectfont −1}%
\end{pgfscope}%
\begin{pgfscope}%
\pgfsetbuttcap%
\pgfsetroundjoin%
\definecolor{currentfill}{rgb}{0.000000,0.000000,0.000000}%
\pgfsetfillcolor{currentfill}%
\pgfsetlinewidth{0.803000pt}%
\definecolor{currentstroke}{rgb}{0.000000,0.000000,0.000000}%
\pgfsetstrokecolor{currentstroke}%
\pgfsetdash{}{0pt}%
\pgfsys@defobject{currentmarker}{\pgfqpoint{0.000000in}{-0.048611in}}{\pgfqpoint{0.000000in}{0.000000in}}{%
\pgfpathmoveto{\pgfqpoint{0.000000in}{0.000000in}}%
\pgfpathlineto{\pgfqpoint{0.000000in}{-0.048611in}}%
\pgfusepath{stroke,fill}%
}%
\begin{pgfscope}%
\pgfsys@transformshift{6.209028in}{4.530556in}%
\pgfsys@useobject{currentmarker}{}%
\end{pgfscope}%
\end{pgfscope}%
\begin{pgfscope}%
\definecolor{textcolor}{rgb}{0.000000,0.000000,0.000000}%
\pgfsetstrokecolor{textcolor}%
\pgfsetfillcolor{textcolor}%
\pgftext[x=6.209028in,y=4.433333in,,top]{\color{textcolor}\sffamily\fontsize{10.000000}{12.000000}\selectfont 0}%
\end{pgfscope}%
\begin{pgfscope}%
\pgfsetbuttcap%
\pgfsetroundjoin%
\definecolor{currentfill}{rgb}{0.000000,0.000000,0.000000}%
\pgfsetfillcolor{currentfill}%
\pgfsetlinewidth{0.803000pt}%
\definecolor{currentstroke}{rgb}{0.000000,0.000000,0.000000}%
\pgfsetstrokecolor{currentstroke}%
\pgfsetdash{}{0pt}%
\pgfsys@defobject{currentmarker}{\pgfqpoint{0.000000in}{-0.048611in}}{\pgfqpoint{0.000000in}{0.000000in}}{%
\pgfpathmoveto{\pgfqpoint{0.000000in}{0.000000in}}%
\pgfpathlineto{\pgfqpoint{0.000000in}{-0.048611in}}%
\pgfusepath{stroke,fill}%
}%
\begin{pgfscope}%
\pgfsys@transformshift{6.933009in}{4.530556in}%
\pgfsys@useobject{currentmarker}{}%
\end{pgfscope}%
\end{pgfscope}%
\begin{pgfscope}%
\definecolor{textcolor}{rgb}{0.000000,0.000000,0.000000}%
\pgfsetstrokecolor{textcolor}%
\pgfsetfillcolor{textcolor}%
\pgftext[x=6.933009in,y=4.433333in,,top]{\color{textcolor}\sffamily\fontsize{10.000000}{12.000000}\selectfont 1}%
\end{pgfscope}%
\begin{pgfscope}%
\pgfsetbuttcap%
\pgfsetroundjoin%
\definecolor{currentfill}{rgb}{0.000000,0.000000,0.000000}%
\pgfsetfillcolor{currentfill}%
\pgfsetlinewidth{0.803000pt}%
\definecolor{currentstroke}{rgb}{0.000000,0.000000,0.000000}%
\pgfsetstrokecolor{currentstroke}%
\pgfsetdash{}{0pt}%
\pgfsys@defobject{currentmarker}{\pgfqpoint{0.000000in}{-0.048611in}}{\pgfqpoint{0.000000in}{0.000000in}}{%
\pgfpathmoveto{\pgfqpoint{0.000000in}{0.000000in}}%
\pgfpathlineto{\pgfqpoint{0.000000in}{-0.048611in}}%
\pgfusepath{stroke,fill}%
}%
\begin{pgfscope}%
\pgfsys@transformshift{7.656991in}{4.530556in}%
\pgfsys@useobject{currentmarker}{}%
\end{pgfscope}%
\end{pgfscope}%
\begin{pgfscope}%
\definecolor{textcolor}{rgb}{0.000000,0.000000,0.000000}%
\pgfsetstrokecolor{textcolor}%
\pgfsetfillcolor{textcolor}%
\pgftext[x=7.656991in,y=4.433333in,,top]{\color{textcolor}\sffamily\fontsize{10.000000}{12.000000}\selectfont 2}%
\end{pgfscope}%
\begin{pgfscope}%
\definecolor{textcolor}{rgb}{0.000000,0.000000,0.000000}%
\pgfsetstrokecolor{textcolor}%
\pgfsetfillcolor{textcolor}%
\pgftext[x=6.209028in,y=4.243365in,,top]{\color{textcolor}\sffamily\fontsize{10.000000}{12.000000}\selectfont \(\displaystyle x\)}%
\end{pgfscope}%
\begin{pgfscope}%
\pgfsetbuttcap%
\pgfsetroundjoin%
\definecolor{currentfill}{rgb}{0.000000,0.000000,0.000000}%
\pgfsetfillcolor{currentfill}%
\pgfsetlinewidth{0.803000pt}%
\definecolor{currentstroke}{rgb}{0.000000,0.000000,0.000000}%
\pgfsetstrokecolor{currentstroke}%
\pgfsetdash{}{0pt}%
\pgfsys@defobject{currentmarker}{\pgfqpoint{-0.048611in}{0.000000in}}{\pgfqpoint{0.000000in}{0.000000in}}{%
\pgfpathmoveto{\pgfqpoint{0.000000in}{0.000000in}}%
\pgfpathlineto{\pgfqpoint{-0.048611in}{0.000000in}}%
\pgfusepath{stroke,fill}%
}%
\begin{pgfscope}%
\pgfsys@transformshift{4.616667in}{4.673611in}%
\pgfsys@useobject{currentmarker}{}%
\end{pgfscope}%
\end{pgfscope}%
\begin{pgfscope}%
\definecolor{textcolor}{rgb}{0.000000,0.000000,0.000000}%
\pgfsetstrokecolor{textcolor}%
\pgfsetfillcolor{textcolor}%
\pgftext[x=4.298565in,y=4.620850in,left,base]{\color{textcolor}\sffamily\fontsize{10.000000}{12.000000}\selectfont 0.0}%
\end{pgfscope}%
\begin{pgfscope}%
\pgfsetbuttcap%
\pgfsetroundjoin%
\definecolor{currentfill}{rgb}{0.000000,0.000000,0.000000}%
\pgfsetfillcolor{currentfill}%
\pgfsetlinewidth{0.803000pt}%
\definecolor{currentstroke}{rgb}{0.000000,0.000000,0.000000}%
\pgfsetstrokecolor{currentstroke}%
\pgfsetdash{}{0pt}%
\pgfsys@defobject{currentmarker}{\pgfqpoint{-0.048611in}{0.000000in}}{\pgfqpoint{0.000000in}{0.000000in}}{%
\pgfpathmoveto{\pgfqpoint{0.000000in}{0.000000in}}%
\pgfpathlineto{\pgfqpoint{-0.048611in}{0.000000in}}%
\pgfusepath{stroke,fill}%
}%
\begin{pgfscope}%
\pgfsys@transformshift{4.616667in}{5.067495in}%
\pgfsys@useobject{currentmarker}{}%
\end{pgfscope}%
\end{pgfscope}%
\begin{pgfscope}%
\definecolor{textcolor}{rgb}{0.000000,0.000000,0.000000}%
\pgfsetstrokecolor{textcolor}%
\pgfsetfillcolor{textcolor}%
\pgftext[x=4.298565in,y=5.014733in,left,base]{\color{textcolor}\sffamily\fontsize{10.000000}{12.000000}\selectfont 0.2}%
\end{pgfscope}%
\begin{pgfscope}%
\pgfsetbuttcap%
\pgfsetroundjoin%
\definecolor{currentfill}{rgb}{0.000000,0.000000,0.000000}%
\pgfsetfillcolor{currentfill}%
\pgfsetlinewidth{0.803000pt}%
\definecolor{currentstroke}{rgb}{0.000000,0.000000,0.000000}%
\pgfsetstrokecolor{currentstroke}%
\pgfsetdash{}{0pt}%
\pgfsys@defobject{currentmarker}{\pgfqpoint{-0.048611in}{0.000000in}}{\pgfqpoint{0.000000in}{0.000000in}}{%
\pgfpathmoveto{\pgfqpoint{0.000000in}{0.000000in}}%
\pgfpathlineto{\pgfqpoint{-0.048611in}{0.000000in}}%
\pgfusepath{stroke,fill}%
}%
\begin{pgfscope}%
\pgfsys@transformshift{4.616667in}{5.461378in}%
\pgfsys@useobject{currentmarker}{}%
\end{pgfscope}%
\end{pgfscope}%
\begin{pgfscope}%
\definecolor{textcolor}{rgb}{0.000000,0.000000,0.000000}%
\pgfsetstrokecolor{textcolor}%
\pgfsetfillcolor{textcolor}%
\pgftext[x=4.298565in,y=5.408617in,left,base]{\color{textcolor}\sffamily\fontsize{10.000000}{12.000000}\selectfont 0.4}%
\end{pgfscope}%
\begin{pgfscope}%
\pgfsetbuttcap%
\pgfsetroundjoin%
\definecolor{currentfill}{rgb}{0.000000,0.000000,0.000000}%
\pgfsetfillcolor{currentfill}%
\pgfsetlinewidth{0.803000pt}%
\definecolor{currentstroke}{rgb}{0.000000,0.000000,0.000000}%
\pgfsetstrokecolor{currentstroke}%
\pgfsetdash{}{0pt}%
\pgfsys@defobject{currentmarker}{\pgfqpoint{-0.048611in}{0.000000in}}{\pgfqpoint{0.000000in}{0.000000in}}{%
\pgfpathmoveto{\pgfqpoint{0.000000in}{0.000000in}}%
\pgfpathlineto{\pgfqpoint{-0.048611in}{0.000000in}}%
\pgfusepath{stroke,fill}%
}%
\begin{pgfscope}%
\pgfsys@transformshift{4.616667in}{5.855262in}%
\pgfsys@useobject{currentmarker}{}%
\end{pgfscope}%
\end{pgfscope}%
\begin{pgfscope}%
\definecolor{textcolor}{rgb}{0.000000,0.000000,0.000000}%
\pgfsetstrokecolor{textcolor}%
\pgfsetfillcolor{textcolor}%
\pgftext[x=4.298565in,y=5.802500in,left,base]{\color{textcolor}\sffamily\fontsize{10.000000}{12.000000}\selectfont 0.6}%
\end{pgfscope}%
\begin{pgfscope}%
\pgfsetbuttcap%
\pgfsetroundjoin%
\definecolor{currentfill}{rgb}{0.000000,0.000000,0.000000}%
\pgfsetfillcolor{currentfill}%
\pgfsetlinewidth{0.803000pt}%
\definecolor{currentstroke}{rgb}{0.000000,0.000000,0.000000}%
\pgfsetstrokecolor{currentstroke}%
\pgfsetdash{}{0pt}%
\pgfsys@defobject{currentmarker}{\pgfqpoint{-0.048611in}{0.000000in}}{\pgfqpoint{0.000000in}{0.000000in}}{%
\pgfpathmoveto{\pgfqpoint{0.000000in}{0.000000in}}%
\pgfpathlineto{\pgfqpoint{-0.048611in}{0.000000in}}%
\pgfusepath{stroke,fill}%
}%
\begin{pgfscope}%
\pgfsys@transformshift{4.616667in}{6.249146in}%
\pgfsys@useobject{currentmarker}{}%
\end{pgfscope}%
\end{pgfscope}%
\begin{pgfscope}%
\definecolor{textcolor}{rgb}{0.000000,0.000000,0.000000}%
\pgfsetstrokecolor{textcolor}%
\pgfsetfillcolor{textcolor}%
\pgftext[x=4.298565in,y=6.196384in,left,base]{\color{textcolor}\sffamily\fontsize{10.000000}{12.000000}\selectfont 0.8}%
\end{pgfscope}%
\begin{pgfscope}%
\pgfsetbuttcap%
\pgfsetroundjoin%
\definecolor{currentfill}{rgb}{0.000000,0.000000,0.000000}%
\pgfsetfillcolor{currentfill}%
\pgfsetlinewidth{0.803000pt}%
\definecolor{currentstroke}{rgb}{0.000000,0.000000,0.000000}%
\pgfsetstrokecolor{currentstroke}%
\pgfsetdash{}{0pt}%
\pgfsys@defobject{currentmarker}{\pgfqpoint{-0.048611in}{0.000000in}}{\pgfqpoint{0.000000in}{0.000000in}}{%
\pgfpathmoveto{\pgfqpoint{0.000000in}{0.000000in}}%
\pgfpathlineto{\pgfqpoint{-0.048611in}{0.000000in}}%
\pgfusepath{stroke,fill}%
}%
\begin{pgfscope}%
\pgfsys@transformshift{4.616667in}{6.643029in}%
\pgfsys@useobject{currentmarker}{}%
\end{pgfscope}%
\end{pgfscope}%
\begin{pgfscope}%
\definecolor{textcolor}{rgb}{0.000000,0.000000,0.000000}%
\pgfsetstrokecolor{textcolor}%
\pgfsetfillcolor{textcolor}%
\pgftext[x=4.298565in,y=6.590268in,left,base]{\color{textcolor}\sffamily\fontsize{10.000000}{12.000000}\selectfont 1.0}%
\end{pgfscope}%
\begin{pgfscope}%
\pgfsetbuttcap%
\pgfsetroundjoin%
\definecolor{currentfill}{rgb}{0.000000,0.000000,0.000000}%
\pgfsetfillcolor{currentfill}%
\pgfsetlinewidth{0.803000pt}%
\definecolor{currentstroke}{rgb}{0.000000,0.000000,0.000000}%
\pgfsetstrokecolor{currentstroke}%
\pgfsetdash{}{0pt}%
\pgfsys@defobject{currentmarker}{\pgfqpoint{-0.048611in}{0.000000in}}{\pgfqpoint{0.000000in}{0.000000in}}{%
\pgfpathmoveto{\pgfqpoint{0.000000in}{0.000000in}}%
\pgfpathlineto{\pgfqpoint{-0.048611in}{0.000000in}}%
\pgfusepath{stroke,fill}%
}%
\begin{pgfscope}%
\pgfsys@transformshift{4.616667in}{7.036913in}%
\pgfsys@useobject{currentmarker}{}%
\end{pgfscope}%
\end{pgfscope}%
\begin{pgfscope}%
\definecolor{textcolor}{rgb}{0.000000,0.000000,0.000000}%
\pgfsetstrokecolor{textcolor}%
\pgfsetfillcolor{textcolor}%
\pgftext[x=4.298565in,y=6.984151in,left,base]{\color{textcolor}\sffamily\fontsize{10.000000}{12.000000}\selectfont 1.2}%
\end{pgfscope}%
\begin{pgfscope}%
\pgfsetbuttcap%
\pgfsetroundjoin%
\definecolor{currentfill}{rgb}{0.000000,0.000000,0.000000}%
\pgfsetfillcolor{currentfill}%
\pgfsetlinewidth{0.803000pt}%
\definecolor{currentstroke}{rgb}{0.000000,0.000000,0.000000}%
\pgfsetstrokecolor{currentstroke}%
\pgfsetdash{}{0pt}%
\pgfsys@defobject{currentmarker}{\pgfqpoint{-0.048611in}{0.000000in}}{\pgfqpoint{0.000000in}{0.000000in}}{%
\pgfpathmoveto{\pgfqpoint{0.000000in}{0.000000in}}%
\pgfpathlineto{\pgfqpoint{-0.048611in}{0.000000in}}%
\pgfusepath{stroke,fill}%
}%
\begin{pgfscope}%
\pgfsys@transformshift{4.616667in}{7.430797in}%
\pgfsys@useobject{currentmarker}{}%
\end{pgfscope}%
\end{pgfscope}%
\begin{pgfscope}%
\definecolor{textcolor}{rgb}{0.000000,0.000000,0.000000}%
\pgfsetstrokecolor{textcolor}%
\pgfsetfillcolor{textcolor}%
\pgftext[x=4.298565in,y=7.378035in,left,base]{\color{textcolor}\sffamily\fontsize{10.000000}{12.000000}\selectfont 1.4}%
\end{pgfscope}%
\begin{pgfscope}%
\definecolor{textcolor}{rgb}{0.000000,0.000000,0.000000}%
\pgfsetstrokecolor{textcolor}%
\pgfsetfillcolor{textcolor}%
\pgftext[x=4.243009in,y=6.104167in,,bottom,rotate=90.000000]{\color{textcolor}\sffamily\fontsize{10.000000}{12.000000}\selectfont \(\displaystyle U\)}%
\end{pgfscope}%
\begin{pgfscope}%
\pgfpathrectangle{\pgfqpoint{4.616667in}{4.530556in}}{\pgfqpoint{3.184722in}{3.147222in}}%
\pgfusepath{clip}%
\pgfsetrectcap%
\pgfsetroundjoin%
\pgfsetlinewidth{1.505625pt}%
\definecolor{currentstroke}{rgb}{0.121569,0.466667,0.705882}%
\pgfsetstrokecolor{currentstroke}%
\pgfsetdash{}{0pt}%
\pgfpathmoveto{\pgfqpoint{4.761427in}{6.643029in}}%
\pgfpathlineto{\pgfqpoint{6.380249in}{6.643034in}}%
\pgfpathlineto{\pgfqpoint{6.380973in}{6.641154in}}%
\pgfpathlineto{\pgfqpoint{6.382421in}{6.644398in}}%
\pgfpathlineto{\pgfqpoint{6.383145in}{6.640401in}}%
\pgfpathlineto{\pgfqpoint{6.383869in}{6.643211in}}%
\pgfpathlineto{\pgfqpoint{6.384593in}{6.646151in}}%
\pgfpathlineto{\pgfqpoint{6.385317in}{6.640571in}}%
\pgfpathlineto{\pgfqpoint{6.386041in}{6.640917in}}%
\pgfpathlineto{\pgfqpoint{6.386765in}{6.647577in}}%
\pgfpathlineto{\pgfqpoint{6.387489in}{6.642390in}}%
\pgfpathlineto{\pgfqpoint{6.388213in}{6.637851in}}%
\pgfpathlineto{\pgfqpoint{6.388937in}{6.647511in}}%
\pgfpathlineto{\pgfqpoint{6.389661in}{6.646228in}}%
\pgfpathlineto{\pgfqpoint{6.390385in}{6.635184in}}%
\pgfpathlineto{\pgfqpoint{6.391109in}{6.644656in}}%
\pgfpathlineto{\pgfqpoint{6.391833in}{6.651563in}}%
\pgfpathlineto{\pgfqpoint{6.392557in}{6.634894in}}%
\pgfpathlineto{\pgfqpoint{6.393281in}{6.638248in}}%
\pgfpathlineto{\pgfqpoint{6.394005in}{6.656483in}}%
\pgfpathlineto{\pgfqpoint{6.394729in}{6.639286in}}%
\pgfpathlineto{\pgfqpoint{6.395453in}{6.629000in}}%
\pgfpathlineto{\pgfqpoint{6.396177in}{6.657611in}}%
\pgfpathlineto{\pgfqpoint{6.396901in}{6.649893in}}%
\pgfpathlineto{\pgfqpoint{6.397625in}{6.619981in}}%
\pgfpathlineto{\pgfqpoint{6.398349in}{6.650845in}}%
\pgfpathlineto{\pgfqpoint{6.399073in}{6.665839in}}%
\pgfpathlineto{\pgfqpoint{6.399797in}{6.616849in}}%
\pgfpathlineto{\pgfqpoint{6.400521in}{6.633235in}}%
\pgfpathlineto{\pgfqpoint{6.401245in}{6.682166in}}%
\pgfpathlineto{\pgfqpoint{6.402693in}{6.605785in}}%
\pgfpathlineto{\pgfqpoint{6.403417in}{6.689154in}}%
\pgfpathlineto{\pgfqpoint{6.404141in}{6.655439in}}%
\pgfpathlineto{\pgfqpoint{6.404865in}{6.576296in}}%
\pgfpathlineto{\pgfqpoint{6.406313in}{6.702299in}}%
\pgfpathlineto{\pgfqpoint{6.407037in}{6.560625in}}%
\pgfpathlineto{\pgfqpoint{6.407761in}{6.626195in}}%
\pgfpathlineto{\pgfqpoint{6.408485in}{6.754823in}}%
\pgfpathlineto{\pgfqpoint{6.409933in}{6.545623in}}%
\pgfpathlineto{\pgfqpoint{6.410657in}{6.784886in}}%
\pgfpathlineto{\pgfqpoint{6.411381in}{6.656013in}}%
\pgfpathlineto{\pgfqpoint{6.412105in}{6.451765in}}%
\pgfpathlineto{\pgfqpoint{6.413553in}{6.791900in}}%
\pgfpathlineto{\pgfqpoint{6.414277in}{6.388545in}}%
\pgfpathlineto{\pgfqpoint{6.415001in}{6.618847in}}%
\pgfpathlineto{\pgfqpoint{6.415724in}{6.956521in}}%
\pgfpathlineto{\pgfqpoint{6.417172in}{6.378284in}}%
\pgfpathlineto{\pgfqpoint{6.417896in}{7.059961in}}%
\pgfpathlineto{\pgfqpoint{6.418620in}{6.624656in}}%
\pgfpathlineto{\pgfqpoint{6.419344in}{6.085242in}}%
\pgfpathlineto{\pgfqpoint{6.420792in}{7.036454in}}%
\pgfpathlineto{\pgfqpoint{6.421516in}{5.882789in}}%
\pgfpathlineto{\pgfqpoint{6.422240in}{6.467760in}}%
\pgfpathlineto{\pgfqpoint{6.422964in}{7.534722in}}%
\pgfpathlineto{\pgfqpoint{6.424412in}{5.609499in}}%
\pgfpathlineto{\pgfqpoint{6.425136in}{7.497630in}}%
\pgfpathlineto{\pgfqpoint{6.425860in}{7.022202in}}%
\pgfpathlineto{\pgfqpoint{6.428032in}{4.673611in}}%
\pgfpathlineto{\pgfqpoint{7.656629in}{4.673611in}}%
\pgfpathlineto{\pgfqpoint{7.656629in}{4.673611in}}%
\pgfusepath{stroke}%
\end{pgfscope}%
\begin{pgfscope}%
\pgfpathrectangle{\pgfqpoint{4.616667in}{4.530556in}}{\pgfqpoint{3.184722in}{3.147222in}}%
\pgfusepath{clip}%
\pgfsetrectcap%
\pgfsetroundjoin%
\pgfsetlinewidth{1.505625pt}%
\definecolor{currentstroke}{rgb}{1.000000,0.498039,0.054902}%
\pgfsetstrokecolor{currentstroke}%
\pgfsetdash{}{0pt}%
\pgfpathmoveto{\pgfqpoint{4.761427in}{6.643029in}}%
\pgfpathlineto{\pgfqpoint{6.382421in}{6.643881in}}%
\pgfpathlineto{\pgfqpoint{6.383145in}{6.641442in}}%
\pgfpathlineto{\pgfqpoint{6.383869in}{6.643106in}}%
\pgfpathlineto{\pgfqpoint{6.384593in}{6.644936in}}%
\pgfpathlineto{\pgfqpoint{6.386041in}{6.641708in}}%
\pgfpathlineto{\pgfqpoint{6.386765in}{6.645776in}}%
\pgfpathlineto{\pgfqpoint{6.387489in}{6.642697in}}%
\pgfpathlineto{\pgfqpoint{6.388213in}{6.639859in}}%
\pgfpathlineto{\pgfqpoint{6.388937in}{6.645697in}}%
\pgfpathlineto{\pgfqpoint{6.389661in}{6.645038in}}%
\pgfpathlineto{\pgfqpoint{6.390385in}{6.638278in}}%
\pgfpathlineto{\pgfqpoint{6.391109in}{6.643921in}}%
\pgfpathlineto{\pgfqpoint{6.391833in}{6.648253in}}%
\pgfpathlineto{\pgfqpoint{6.392557in}{6.638168in}}%
\pgfpathlineto{\pgfqpoint{6.393281in}{6.640004in}}%
\pgfpathlineto{\pgfqpoint{6.394005in}{6.651167in}}%
\pgfpathlineto{\pgfqpoint{6.394729in}{6.640913in}}%
\pgfpathlineto{\pgfqpoint{6.395453in}{6.634404in}}%
\pgfpathlineto{\pgfqpoint{6.396177in}{6.651734in}}%
\pgfpathlineto{\pgfqpoint{6.396901in}{6.647398in}}%
\pgfpathlineto{\pgfqpoint{6.397625in}{6.629011in}}%
\pgfpathlineto{\pgfqpoint{6.398349in}{6.647496in}}%
\pgfpathlineto{\pgfqpoint{6.399073in}{6.657010in}}%
\pgfpathlineto{\pgfqpoint{6.399797in}{6.627290in}}%
\pgfpathlineto{\pgfqpoint{6.400521in}{6.636699in}}%
\pgfpathlineto{\pgfqpoint{6.401245in}{6.666680in}}%
\pgfpathlineto{\pgfqpoint{6.401969in}{6.633570in}}%
\pgfpathlineto{\pgfqpoint{6.402693in}{6.619979in}}%
\pgfpathlineto{\pgfqpoint{6.403417in}{6.670546in}}%
\pgfpathlineto{\pgfqpoint{6.404141in}{6.651152in}}%
\pgfpathlineto{\pgfqpoint{6.404865in}{6.602133in}}%
\pgfpathlineto{\pgfqpoint{6.406313in}{6.679319in}}%
\pgfpathlineto{\pgfqpoint{6.407037in}{6.592990in}}%
\pgfpathlineto{\pgfqpoint{6.407761in}{6.631578in}}%
\pgfpathlineto{\pgfqpoint{6.408485in}{6.710151in}}%
\pgfpathlineto{\pgfqpoint{6.409933in}{6.581958in}}%
\pgfpathlineto{\pgfqpoint{6.410657in}{6.726886in}}%
\pgfpathlineto{\pgfqpoint{6.411381in}{6.652550in}}%
\pgfpathlineto{\pgfqpoint{6.412105in}{6.523704in}}%
\pgfpathlineto{\pgfqpoint{6.413553in}{6.733497in}}%
\pgfpathlineto{\pgfqpoint{6.414277in}{6.484815in}}%
\pgfpathlineto{\pgfqpoint{6.415001in}{6.623871in}}%
\pgfpathlineto{\pgfqpoint{6.415724in}{6.827195in}}%
\pgfpathlineto{\pgfqpoint{6.417172in}{6.471709in}}%
\pgfpathlineto{\pgfqpoint{6.417896in}{6.881472in}}%
\pgfpathlineto{\pgfqpoint{6.418620in}{6.636583in}}%
\pgfpathlineto{\pgfqpoint{6.419344in}{6.276899in}}%
\pgfpathlineto{\pgfqpoint{6.420792in}{6.876435in}}%
\pgfpathlineto{\pgfqpoint{6.421516in}{6.138624in}}%
\pgfpathlineto{\pgfqpoint{6.422240in}{6.509935in}}%
\pgfpathlineto{\pgfqpoint{6.422964in}{7.132632in}}%
\pgfpathlineto{\pgfqpoint{6.424412in}{5.890603in}}%
\pgfpathlineto{\pgfqpoint{6.425136in}{7.075063in}}%
\pgfpathlineto{\pgfqpoint{6.425860in}{6.893344in}}%
\pgfpathlineto{\pgfqpoint{6.428032in}{4.673613in}}%
\pgfpathlineto{\pgfqpoint{6.910204in}{4.673611in}}%
\pgfpathlineto{\pgfqpoint{7.656629in}{4.673611in}}%
\pgfpathlineto{\pgfqpoint{7.656629in}{4.673611in}}%
\pgfusepath{stroke}%
\end{pgfscope}%
\begin{pgfscope}%
\pgfpathrectangle{\pgfqpoint{4.616667in}{4.530556in}}{\pgfqpoint{3.184722in}{3.147222in}}%
\pgfusepath{clip}%
\pgfsetrectcap%
\pgfsetroundjoin%
\pgfsetlinewidth{1.505625pt}%
\definecolor{currentstroke}{rgb}{0.172549,0.627451,0.172549}%
\pgfsetstrokecolor{currentstroke}%
\pgfsetdash{}{0pt}%
\pgfpathmoveto{\pgfqpoint{4.761427in}{6.643029in}}%
\pgfpathlineto{\pgfqpoint{6.422964in}{6.642375in}}%
\pgfpathlineto{\pgfqpoint{6.423688in}{6.639421in}}%
\pgfpathlineto{\pgfqpoint{6.424412in}{6.623234in}}%
\pgfpathlineto{\pgfqpoint{6.425136in}{6.537561in}}%
\pgfpathlineto{\pgfqpoint{6.430204in}{4.673611in}}%
\pgfpathlineto{\pgfqpoint{7.656629in}{4.673611in}}%
\pgfpathlineto{\pgfqpoint{7.656629in}{4.673611in}}%
\pgfusepath{stroke}%
\end{pgfscope}%
\begin{pgfscope}%
\pgfpathrectangle{\pgfqpoint{4.616667in}{4.530556in}}{\pgfqpoint{3.184722in}{3.147222in}}%
\pgfusepath{clip}%
\pgfsetrectcap%
\pgfsetroundjoin%
\pgfsetlinewidth{1.505625pt}%
\definecolor{currentstroke}{rgb}{0.839216,0.152941,0.156863}%
\pgfsetstrokecolor{currentstroke}%
\pgfsetdash{}{0pt}%
\pgfpathmoveto{\pgfqpoint{4.761427in}{6.643029in}}%
\pgfpathlineto{\pgfqpoint{6.208666in}{6.643029in}}%
\pgfpathlineto{\pgfqpoint{6.210838in}{4.673611in}}%
\pgfpathlineto{\pgfqpoint{7.656629in}{4.673611in}}%
\pgfpathlineto{\pgfqpoint{7.656629in}{4.673611in}}%
\pgfusepath{stroke}%
\end{pgfscope}%
\begin{pgfscope}%
\pgfpathrectangle{\pgfqpoint{4.616667in}{4.530556in}}{\pgfqpoint{3.184722in}{3.147222in}}%
\pgfusepath{clip}%
\pgfsetrectcap%
\pgfsetroundjoin%
\pgfsetlinewidth{0.501875pt}%
\definecolor{currentstroke}{rgb}{0.000000,0.000000,0.000000}%
\pgfsetstrokecolor{currentstroke}%
\pgfsetdash{}{0pt}%
\pgfpathmoveto{\pgfqpoint{4.761427in}{6.643029in}}%
\pgfpathlineto{\pgfqpoint{6.425860in}{6.643029in}}%
\pgfpathlineto{\pgfqpoint{6.428032in}{4.673611in}}%
\pgfpathlineto{\pgfqpoint{7.656629in}{4.673611in}}%
\pgfpathlineto{\pgfqpoint{7.656629in}{4.673611in}}%
\pgfusepath{stroke}%
\end{pgfscope}%
\begin{pgfscope}%
\pgfsetrectcap%
\pgfsetmiterjoin%
\pgfsetlinewidth{0.803000pt}%
\definecolor{currentstroke}{rgb}{0.000000,0.000000,0.000000}%
\pgfsetstrokecolor{currentstroke}%
\pgfsetdash{}{0pt}%
\pgfpathmoveto{\pgfqpoint{4.616667in}{4.530556in}}%
\pgfpathlineto{\pgfqpoint{4.616667in}{7.677778in}}%
\pgfusepath{stroke}%
\end{pgfscope}%
\begin{pgfscope}%
\pgfsetrectcap%
\pgfsetmiterjoin%
\pgfsetlinewidth{0.803000pt}%
\definecolor{currentstroke}{rgb}{0.000000,0.000000,0.000000}%
\pgfsetstrokecolor{currentstroke}%
\pgfsetdash{}{0pt}%
\pgfpathmoveto{\pgfqpoint{7.801389in}{4.530556in}}%
\pgfpathlineto{\pgfqpoint{7.801389in}{7.677778in}}%
\pgfusepath{stroke}%
\end{pgfscope}%
\begin{pgfscope}%
\pgfsetrectcap%
\pgfsetmiterjoin%
\pgfsetlinewidth{0.803000pt}%
\definecolor{currentstroke}{rgb}{0.000000,0.000000,0.000000}%
\pgfsetstrokecolor{currentstroke}%
\pgfsetdash{}{0pt}%
\pgfpathmoveto{\pgfqpoint{4.616667in}{4.530556in}}%
\pgfpathlineto{\pgfqpoint{7.801389in}{4.530556in}}%
\pgfusepath{stroke}%
\end{pgfscope}%
\begin{pgfscope}%
\pgfsetrectcap%
\pgfsetmiterjoin%
\pgfsetlinewidth{0.803000pt}%
\definecolor{currentstroke}{rgb}{0.000000,0.000000,0.000000}%
\pgfsetstrokecolor{currentstroke}%
\pgfsetdash{}{0pt}%
\pgfpathmoveto{\pgfqpoint{4.616667in}{7.677778in}}%
\pgfpathlineto{\pgfqpoint{7.801389in}{7.677778in}}%
\pgfusepath{stroke}%
\end{pgfscope}%
\begin{pgfscope}%
\definecolor{textcolor}{rgb}{0.000000,0.000000,0.000000}%
\pgfsetstrokecolor{textcolor}%
\pgfsetfillcolor{textcolor}%
\pgftext[x=6.209028in,y=7.761111in,,base]{\color{textcolor}\sffamily\fontsize{12.000000}{14.400000}\selectfont \(\displaystyle  t = 0.6 \)}%
\end{pgfscope}%
\begin{pgfscope}%
\pgfsetbuttcap%
\pgfsetmiterjoin%
\definecolor{currentfill}{rgb}{1.000000,1.000000,1.000000}%
\pgfsetfillcolor{currentfill}%
\pgfsetlinewidth{0.000000pt}%
\definecolor{currentstroke}{rgb}{0.000000,0.000000,0.000000}%
\pgfsetstrokecolor{currentstroke}%
\pgfsetstrokeopacity{0.000000}%
\pgfsetdash{}{0pt}%
\pgfpathmoveto{\pgfqpoint{0.691667in}{0.580556in}}%
\pgfpathlineto{\pgfqpoint{3.876389in}{0.580556in}}%
\pgfpathlineto{\pgfqpoint{3.876389in}{3.727778in}}%
\pgfpathlineto{\pgfqpoint{0.691667in}{3.727778in}}%
\pgfpathclose%
\pgfusepath{fill}%
\end{pgfscope}%
\begin{pgfscope}%
\pgfsetbuttcap%
\pgfsetroundjoin%
\definecolor{currentfill}{rgb}{0.000000,0.000000,0.000000}%
\pgfsetfillcolor{currentfill}%
\pgfsetlinewidth{0.803000pt}%
\definecolor{currentstroke}{rgb}{0.000000,0.000000,0.000000}%
\pgfsetstrokecolor{currentstroke}%
\pgfsetdash{}{0pt}%
\pgfsys@defobject{currentmarker}{\pgfqpoint{0.000000in}{-0.048611in}}{\pgfqpoint{0.000000in}{0.000000in}}{%
\pgfpathmoveto{\pgfqpoint{0.000000in}{0.000000in}}%
\pgfpathlineto{\pgfqpoint{0.000000in}{-0.048611in}}%
\pgfusepath{stroke,fill}%
}%
\begin{pgfscope}%
\pgfsys@transformshift{0.836065in}{0.580556in}%
\pgfsys@useobject{currentmarker}{}%
\end{pgfscope}%
\end{pgfscope}%
\begin{pgfscope}%
\definecolor{textcolor}{rgb}{0.000000,0.000000,0.000000}%
\pgfsetstrokecolor{textcolor}%
\pgfsetfillcolor{textcolor}%
\pgftext[x=0.836065in,y=0.483333in,,top]{\color{textcolor}\sffamily\fontsize{10.000000}{12.000000}\selectfont −2}%
\end{pgfscope}%
\begin{pgfscope}%
\pgfsetbuttcap%
\pgfsetroundjoin%
\definecolor{currentfill}{rgb}{0.000000,0.000000,0.000000}%
\pgfsetfillcolor{currentfill}%
\pgfsetlinewidth{0.803000pt}%
\definecolor{currentstroke}{rgb}{0.000000,0.000000,0.000000}%
\pgfsetstrokecolor{currentstroke}%
\pgfsetdash{}{0pt}%
\pgfsys@defobject{currentmarker}{\pgfqpoint{0.000000in}{-0.048611in}}{\pgfqpoint{0.000000in}{0.000000in}}{%
\pgfpathmoveto{\pgfqpoint{0.000000in}{0.000000in}}%
\pgfpathlineto{\pgfqpoint{0.000000in}{-0.048611in}}%
\pgfusepath{stroke,fill}%
}%
\begin{pgfscope}%
\pgfsys@transformshift{1.560046in}{0.580556in}%
\pgfsys@useobject{currentmarker}{}%
\end{pgfscope}%
\end{pgfscope}%
\begin{pgfscope}%
\definecolor{textcolor}{rgb}{0.000000,0.000000,0.000000}%
\pgfsetstrokecolor{textcolor}%
\pgfsetfillcolor{textcolor}%
\pgftext[x=1.560046in,y=0.483333in,,top]{\color{textcolor}\sffamily\fontsize{10.000000}{12.000000}\selectfont −1}%
\end{pgfscope}%
\begin{pgfscope}%
\pgfsetbuttcap%
\pgfsetroundjoin%
\definecolor{currentfill}{rgb}{0.000000,0.000000,0.000000}%
\pgfsetfillcolor{currentfill}%
\pgfsetlinewidth{0.803000pt}%
\definecolor{currentstroke}{rgb}{0.000000,0.000000,0.000000}%
\pgfsetstrokecolor{currentstroke}%
\pgfsetdash{}{0pt}%
\pgfsys@defobject{currentmarker}{\pgfqpoint{0.000000in}{-0.048611in}}{\pgfqpoint{0.000000in}{0.000000in}}{%
\pgfpathmoveto{\pgfqpoint{0.000000in}{0.000000in}}%
\pgfpathlineto{\pgfqpoint{0.000000in}{-0.048611in}}%
\pgfusepath{stroke,fill}%
}%
\begin{pgfscope}%
\pgfsys@transformshift{2.284028in}{0.580556in}%
\pgfsys@useobject{currentmarker}{}%
\end{pgfscope}%
\end{pgfscope}%
\begin{pgfscope}%
\definecolor{textcolor}{rgb}{0.000000,0.000000,0.000000}%
\pgfsetstrokecolor{textcolor}%
\pgfsetfillcolor{textcolor}%
\pgftext[x=2.284028in,y=0.483333in,,top]{\color{textcolor}\sffamily\fontsize{10.000000}{12.000000}\selectfont 0}%
\end{pgfscope}%
\begin{pgfscope}%
\pgfsetbuttcap%
\pgfsetroundjoin%
\definecolor{currentfill}{rgb}{0.000000,0.000000,0.000000}%
\pgfsetfillcolor{currentfill}%
\pgfsetlinewidth{0.803000pt}%
\definecolor{currentstroke}{rgb}{0.000000,0.000000,0.000000}%
\pgfsetstrokecolor{currentstroke}%
\pgfsetdash{}{0pt}%
\pgfsys@defobject{currentmarker}{\pgfqpoint{0.000000in}{-0.048611in}}{\pgfqpoint{0.000000in}{0.000000in}}{%
\pgfpathmoveto{\pgfqpoint{0.000000in}{0.000000in}}%
\pgfpathlineto{\pgfqpoint{0.000000in}{-0.048611in}}%
\pgfusepath{stroke,fill}%
}%
\begin{pgfscope}%
\pgfsys@transformshift{3.008009in}{0.580556in}%
\pgfsys@useobject{currentmarker}{}%
\end{pgfscope}%
\end{pgfscope}%
\begin{pgfscope}%
\definecolor{textcolor}{rgb}{0.000000,0.000000,0.000000}%
\pgfsetstrokecolor{textcolor}%
\pgfsetfillcolor{textcolor}%
\pgftext[x=3.008009in,y=0.483333in,,top]{\color{textcolor}\sffamily\fontsize{10.000000}{12.000000}\selectfont 1}%
\end{pgfscope}%
\begin{pgfscope}%
\pgfsetbuttcap%
\pgfsetroundjoin%
\definecolor{currentfill}{rgb}{0.000000,0.000000,0.000000}%
\pgfsetfillcolor{currentfill}%
\pgfsetlinewidth{0.803000pt}%
\definecolor{currentstroke}{rgb}{0.000000,0.000000,0.000000}%
\pgfsetstrokecolor{currentstroke}%
\pgfsetdash{}{0pt}%
\pgfsys@defobject{currentmarker}{\pgfqpoint{0.000000in}{-0.048611in}}{\pgfqpoint{0.000000in}{0.000000in}}{%
\pgfpathmoveto{\pgfqpoint{0.000000in}{0.000000in}}%
\pgfpathlineto{\pgfqpoint{0.000000in}{-0.048611in}}%
\pgfusepath{stroke,fill}%
}%
\begin{pgfscope}%
\pgfsys@transformshift{3.731991in}{0.580556in}%
\pgfsys@useobject{currentmarker}{}%
\end{pgfscope}%
\end{pgfscope}%
\begin{pgfscope}%
\definecolor{textcolor}{rgb}{0.000000,0.000000,0.000000}%
\pgfsetstrokecolor{textcolor}%
\pgfsetfillcolor{textcolor}%
\pgftext[x=3.731991in,y=0.483333in,,top]{\color{textcolor}\sffamily\fontsize{10.000000}{12.000000}\selectfont 2}%
\end{pgfscope}%
\begin{pgfscope}%
\definecolor{textcolor}{rgb}{0.000000,0.000000,0.000000}%
\pgfsetstrokecolor{textcolor}%
\pgfsetfillcolor{textcolor}%
\pgftext[x=2.284028in,y=0.293365in,,top]{\color{textcolor}\sffamily\fontsize{10.000000}{12.000000}\selectfont \(\displaystyle x\)}%
\end{pgfscope}%
\begin{pgfscope}%
\pgfsetbuttcap%
\pgfsetroundjoin%
\definecolor{currentfill}{rgb}{0.000000,0.000000,0.000000}%
\pgfsetfillcolor{currentfill}%
\pgfsetlinewidth{0.803000pt}%
\definecolor{currentstroke}{rgb}{0.000000,0.000000,0.000000}%
\pgfsetstrokecolor{currentstroke}%
\pgfsetdash{}{0pt}%
\pgfsys@defobject{currentmarker}{\pgfqpoint{-0.048611in}{0.000000in}}{\pgfqpoint{0.000000in}{0.000000in}}{%
\pgfpathmoveto{\pgfqpoint{0.000000in}{0.000000in}}%
\pgfpathlineto{\pgfqpoint{-0.048611in}{0.000000in}}%
\pgfusepath{stroke,fill}%
}%
\begin{pgfscope}%
\pgfsys@transformshift{0.691667in}{0.723611in}%
\pgfsys@useobject{currentmarker}{}%
\end{pgfscope}%
\end{pgfscope}%
\begin{pgfscope}%
\definecolor{textcolor}{rgb}{0.000000,0.000000,0.000000}%
\pgfsetstrokecolor{textcolor}%
\pgfsetfillcolor{textcolor}%
\pgftext[x=0.373565in,y=0.670850in,left,base]{\color{textcolor}\sffamily\fontsize{10.000000}{12.000000}\selectfont 0.0}%
\end{pgfscope}%
\begin{pgfscope}%
\pgfsetbuttcap%
\pgfsetroundjoin%
\definecolor{currentfill}{rgb}{0.000000,0.000000,0.000000}%
\pgfsetfillcolor{currentfill}%
\pgfsetlinewidth{0.803000pt}%
\definecolor{currentstroke}{rgb}{0.000000,0.000000,0.000000}%
\pgfsetstrokecolor{currentstroke}%
\pgfsetdash{}{0pt}%
\pgfsys@defobject{currentmarker}{\pgfqpoint{-0.048611in}{0.000000in}}{\pgfqpoint{0.000000in}{0.000000in}}{%
\pgfpathmoveto{\pgfqpoint{0.000000in}{0.000000in}}%
\pgfpathlineto{\pgfqpoint{-0.048611in}{0.000000in}}%
\pgfusepath{stroke,fill}%
}%
\begin{pgfscope}%
\pgfsys@transformshift{0.691667in}{1.117495in}%
\pgfsys@useobject{currentmarker}{}%
\end{pgfscope}%
\end{pgfscope}%
\begin{pgfscope}%
\definecolor{textcolor}{rgb}{0.000000,0.000000,0.000000}%
\pgfsetstrokecolor{textcolor}%
\pgfsetfillcolor{textcolor}%
\pgftext[x=0.373565in,y=1.064733in,left,base]{\color{textcolor}\sffamily\fontsize{10.000000}{12.000000}\selectfont 0.2}%
\end{pgfscope}%
\begin{pgfscope}%
\pgfsetbuttcap%
\pgfsetroundjoin%
\definecolor{currentfill}{rgb}{0.000000,0.000000,0.000000}%
\pgfsetfillcolor{currentfill}%
\pgfsetlinewidth{0.803000pt}%
\definecolor{currentstroke}{rgb}{0.000000,0.000000,0.000000}%
\pgfsetstrokecolor{currentstroke}%
\pgfsetdash{}{0pt}%
\pgfsys@defobject{currentmarker}{\pgfqpoint{-0.048611in}{0.000000in}}{\pgfqpoint{0.000000in}{0.000000in}}{%
\pgfpathmoveto{\pgfqpoint{0.000000in}{0.000000in}}%
\pgfpathlineto{\pgfqpoint{-0.048611in}{0.000000in}}%
\pgfusepath{stroke,fill}%
}%
\begin{pgfscope}%
\pgfsys@transformshift{0.691667in}{1.511378in}%
\pgfsys@useobject{currentmarker}{}%
\end{pgfscope}%
\end{pgfscope}%
\begin{pgfscope}%
\definecolor{textcolor}{rgb}{0.000000,0.000000,0.000000}%
\pgfsetstrokecolor{textcolor}%
\pgfsetfillcolor{textcolor}%
\pgftext[x=0.373565in,y=1.458617in,left,base]{\color{textcolor}\sffamily\fontsize{10.000000}{12.000000}\selectfont 0.4}%
\end{pgfscope}%
\begin{pgfscope}%
\pgfsetbuttcap%
\pgfsetroundjoin%
\definecolor{currentfill}{rgb}{0.000000,0.000000,0.000000}%
\pgfsetfillcolor{currentfill}%
\pgfsetlinewidth{0.803000pt}%
\definecolor{currentstroke}{rgb}{0.000000,0.000000,0.000000}%
\pgfsetstrokecolor{currentstroke}%
\pgfsetdash{}{0pt}%
\pgfsys@defobject{currentmarker}{\pgfqpoint{-0.048611in}{0.000000in}}{\pgfqpoint{0.000000in}{0.000000in}}{%
\pgfpathmoveto{\pgfqpoint{0.000000in}{0.000000in}}%
\pgfpathlineto{\pgfqpoint{-0.048611in}{0.000000in}}%
\pgfusepath{stroke,fill}%
}%
\begin{pgfscope}%
\pgfsys@transformshift{0.691667in}{1.905262in}%
\pgfsys@useobject{currentmarker}{}%
\end{pgfscope}%
\end{pgfscope}%
\begin{pgfscope}%
\definecolor{textcolor}{rgb}{0.000000,0.000000,0.000000}%
\pgfsetstrokecolor{textcolor}%
\pgfsetfillcolor{textcolor}%
\pgftext[x=0.373565in,y=1.852500in,left,base]{\color{textcolor}\sffamily\fontsize{10.000000}{12.000000}\selectfont 0.6}%
\end{pgfscope}%
\begin{pgfscope}%
\pgfsetbuttcap%
\pgfsetroundjoin%
\definecolor{currentfill}{rgb}{0.000000,0.000000,0.000000}%
\pgfsetfillcolor{currentfill}%
\pgfsetlinewidth{0.803000pt}%
\definecolor{currentstroke}{rgb}{0.000000,0.000000,0.000000}%
\pgfsetstrokecolor{currentstroke}%
\pgfsetdash{}{0pt}%
\pgfsys@defobject{currentmarker}{\pgfqpoint{-0.048611in}{0.000000in}}{\pgfqpoint{0.000000in}{0.000000in}}{%
\pgfpathmoveto{\pgfqpoint{0.000000in}{0.000000in}}%
\pgfpathlineto{\pgfqpoint{-0.048611in}{0.000000in}}%
\pgfusepath{stroke,fill}%
}%
\begin{pgfscope}%
\pgfsys@transformshift{0.691667in}{2.299146in}%
\pgfsys@useobject{currentmarker}{}%
\end{pgfscope}%
\end{pgfscope}%
\begin{pgfscope}%
\definecolor{textcolor}{rgb}{0.000000,0.000000,0.000000}%
\pgfsetstrokecolor{textcolor}%
\pgfsetfillcolor{textcolor}%
\pgftext[x=0.373565in,y=2.246384in,left,base]{\color{textcolor}\sffamily\fontsize{10.000000}{12.000000}\selectfont 0.8}%
\end{pgfscope}%
\begin{pgfscope}%
\pgfsetbuttcap%
\pgfsetroundjoin%
\definecolor{currentfill}{rgb}{0.000000,0.000000,0.000000}%
\pgfsetfillcolor{currentfill}%
\pgfsetlinewidth{0.803000pt}%
\definecolor{currentstroke}{rgb}{0.000000,0.000000,0.000000}%
\pgfsetstrokecolor{currentstroke}%
\pgfsetdash{}{0pt}%
\pgfsys@defobject{currentmarker}{\pgfqpoint{-0.048611in}{0.000000in}}{\pgfqpoint{0.000000in}{0.000000in}}{%
\pgfpathmoveto{\pgfqpoint{0.000000in}{0.000000in}}%
\pgfpathlineto{\pgfqpoint{-0.048611in}{0.000000in}}%
\pgfusepath{stroke,fill}%
}%
\begin{pgfscope}%
\pgfsys@transformshift{0.691667in}{2.693029in}%
\pgfsys@useobject{currentmarker}{}%
\end{pgfscope}%
\end{pgfscope}%
\begin{pgfscope}%
\definecolor{textcolor}{rgb}{0.000000,0.000000,0.000000}%
\pgfsetstrokecolor{textcolor}%
\pgfsetfillcolor{textcolor}%
\pgftext[x=0.373565in,y=2.640268in,left,base]{\color{textcolor}\sffamily\fontsize{10.000000}{12.000000}\selectfont 1.0}%
\end{pgfscope}%
\begin{pgfscope}%
\pgfsetbuttcap%
\pgfsetroundjoin%
\definecolor{currentfill}{rgb}{0.000000,0.000000,0.000000}%
\pgfsetfillcolor{currentfill}%
\pgfsetlinewidth{0.803000pt}%
\definecolor{currentstroke}{rgb}{0.000000,0.000000,0.000000}%
\pgfsetstrokecolor{currentstroke}%
\pgfsetdash{}{0pt}%
\pgfsys@defobject{currentmarker}{\pgfqpoint{-0.048611in}{0.000000in}}{\pgfqpoint{0.000000in}{0.000000in}}{%
\pgfpathmoveto{\pgfqpoint{0.000000in}{0.000000in}}%
\pgfpathlineto{\pgfqpoint{-0.048611in}{0.000000in}}%
\pgfusepath{stroke,fill}%
}%
\begin{pgfscope}%
\pgfsys@transformshift{0.691667in}{3.086913in}%
\pgfsys@useobject{currentmarker}{}%
\end{pgfscope}%
\end{pgfscope}%
\begin{pgfscope}%
\definecolor{textcolor}{rgb}{0.000000,0.000000,0.000000}%
\pgfsetstrokecolor{textcolor}%
\pgfsetfillcolor{textcolor}%
\pgftext[x=0.373565in,y=3.034151in,left,base]{\color{textcolor}\sffamily\fontsize{10.000000}{12.000000}\selectfont 1.2}%
\end{pgfscope}%
\begin{pgfscope}%
\pgfsetbuttcap%
\pgfsetroundjoin%
\definecolor{currentfill}{rgb}{0.000000,0.000000,0.000000}%
\pgfsetfillcolor{currentfill}%
\pgfsetlinewidth{0.803000pt}%
\definecolor{currentstroke}{rgb}{0.000000,0.000000,0.000000}%
\pgfsetstrokecolor{currentstroke}%
\pgfsetdash{}{0pt}%
\pgfsys@defobject{currentmarker}{\pgfqpoint{-0.048611in}{0.000000in}}{\pgfqpoint{0.000000in}{0.000000in}}{%
\pgfpathmoveto{\pgfqpoint{0.000000in}{0.000000in}}%
\pgfpathlineto{\pgfqpoint{-0.048611in}{0.000000in}}%
\pgfusepath{stroke,fill}%
}%
\begin{pgfscope}%
\pgfsys@transformshift{0.691667in}{3.480797in}%
\pgfsys@useobject{currentmarker}{}%
\end{pgfscope}%
\end{pgfscope}%
\begin{pgfscope}%
\definecolor{textcolor}{rgb}{0.000000,0.000000,0.000000}%
\pgfsetstrokecolor{textcolor}%
\pgfsetfillcolor{textcolor}%
\pgftext[x=0.373565in,y=3.428035in,left,base]{\color{textcolor}\sffamily\fontsize{10.000000}{12.000000}\selectfont 1.4}%
\end{pgfscope}%
\begin{pgfscope}%
\definecolor{textcolor}{rgb}{0.000000,0.000000,0.000000}%
\pgfsetstrokecolor{textcolor}%
\pgfsetfillcolor{textcolor}%
\pgftext[x=0.318009in,y=2.154167in,,bottom,rotate=90.000000]{\color{textcolor}\sffamily\fontsize{10.000000}{12.000000}\selectfont \(\displaystyle U\)}%
\end{pgfscope}%
\begin{pgfscope}%
\pgfpathrectangle{\pgfqpoint{0.691667in}{0.580556in}}{\pgfqpoint{3.184722in}{3.147222in}}%
\pgfusepath{clip}%
\pgfsetrectcap%
\pgfsetroundjoin%
\pgfsetlinewidth{1.505625pt}%
\definecolor{currentstroke}{rgb}{0.121569,0.466667,0.705882}%
\pgfsetstrokecolor{currentstroke}%
\pgfsetdash{}{0pt}%
\pgfpathmoveto{\pgfqpoint{0.836427in}{2.693029in}}%
\pgfpathlineto{\pgfqpoint{2.527648in}{2.693034in}}%
\pgfpathlineto{\pgfqpoint{2.528372in}{2.691154in}}%
\pgfpathlineto{\pgfqpoint{2.529819in}{2.694398in}}%
\pgfpathlineto{\pgfqpoint{2.530543in}{2.690401in}}%
\pgfpathlineto{\pgfqpoint{2.531267in}{2.693211in}}%
\pgfpathlineto{\pgfqpoint{2.531991in}{2.696151in}}%
\pgfpathlineto{\pgfqpoint{2.532715in}{2.690571in}}%
\pgfpathlineto{\pgfqpoint{2.533439in}{2.690917in}}%
\pgfpathlineto{\pgfqpoint{2.534163in}{2.697577in}}%
\pgfpathlineto{\pgfqpoint{2.534887in}{2.692390in}}%
\pgfpathlineto{\pgfqpoint{2.535611in}{2.687851in}}%
\pgfpathlineto{\pgfqpoint{2.536335in}{2.697511in}}%
\pgfpathlineto{\pgfqpoint{2.537059in}{2.696228in}}%
\pgfpathlineto{\pgfqpoint{2.537783in}{2.685184in}}%
\pgfpathlineto{\pgfqpoint{2.538507in}{2.694656in}}%
\pgfpathlineto{\pgfqpoint{2.539231in}{2.701563in}}%
\pgfpathlineto{\pgfqpoint{2.539955in}{2.684894in}}%
\pgfpathlineto{\pgfqpoint{2.540679in}{2.688248in}}%
\pgfpathlineto{\pgfqpoint{2.541403in}{2.706483in}}%
\pgfpathlineto{\pgfqpoint{2.542127in}{2.689286in}}%
\pgfpathlineto{\pgfqpoint{2.542851in}{2.679000in}}%
\pgfpathlineto{\pgfqpoint{2.543575in}{2.707611in}}%
\pgfpathlineto{\pgfqpoint{2.544299in}{2.699893in}}%
\pgfpathlineto{\pgfqpoint{2.545023in}{2.669981in}}%
\pgfpathlineto{\pgfqpoint{2.545747in}{2.700845in}}%
\pgfpathlineto{\pgfqpoint{2.546471in}{2.715839in}}%
\pgfpathlineto{\pgfqpoint{2.547195in}{2.666849in}}%
\pgfpathlineto{\pgfqpoint{2.547919in}{2.683235in}}%
\pgfpathlineto{\pgfqpoint{2.548643in}{2.732166in}}%
\pgfpathlineto{\pgfqpoint{2.550091in}{2.655785in}}%
\pgfpathlineto{\pgfqpoint{2.550815in}{2.739154in}}%
\pgfpathlineto{\pgfqpoint{2.551539in}{2.705439in}}%
\pgfpathlineto{\pgfqpoint{2.552263in}{2.626296in}}%
\pgfpathlineto{\pgfqpoint{2.553711in}{2.752299in}}%
\pgfpathlineto{\pgfqpoint{2.554435in}{2.610625in}}%
\pgfpathlineto{\pgfqpoint{2.555159in}{2.676195in}}%
\pgfpathlineto{\pgfqpoint{2.555883in}{2.804823in}}%
\pgfpathlineto{\pgfqpoint{2.557331in}{2.595623in}}%
\pgfpathlineto{\pgfqpoint{2.558055in}{2.834886in}}%
\pgfpathlineto{\pgfqpoint{2.558779in}{2.706013in}}%
\pgfpathlineto{\pgfqpoint{2.559503in}{2.501765in}}%
\pgfpathlineto{\pgfqpoint{2.560951in}{2.841900in}}%
\pgfpathlineto{\pgfqpoint{2.561675in}{2.438545in}}%
\pgfpathlineto{\pgfqpoint{2.562399in}{2.668847in}}%
\pgfpathlineto{\pgfqpoint{2.563123in}{3.006521in}}%
\pgfpathlineto{\pgfqpoint{2.564571in}{2.428284in}}%
\pgfpathlineto{\pgfqpoint{2.565295in}{3.109961in}}%
\pgfpathlineto{\pgfqpoint{2.566019in}{2.674656in}}%
\pgfpathlineto{\pgfqpoint{2.566743in}{2.135242in}}%
\pgfpathlineto{\pgfqpoint{2.568191in}{3.086454in}}%
\pgfpathlineto{\pgfqpoint{2.568914in}{1.932789in}}%
\pgfpathlineto{\pgfqpoint{2.569638in}{2.517760in}}%
\pgfpathlineto{\pgfqpoint{2.570362in}{3.584722in}}%
\pgfpathlineto{\pgfqpoint{2.571810in}{1.659499in}}%
\pgfpathlineto{\pgfqpoint{2.572534in}{3.547630in}}%
\pgfpathlineto{\pgfqpoint{2.573258in}{3.072202in}}%
\pgfpathlineto{\pgfqpoint{2.575430in}{0.723611in}}%
\pgfpathlineto{\pgfqpoint{3.731629in}{0.723611in}}%
\pgfpathlineto{\pgfqpoint{3.731629in}{0.723611in}}%
\pgfusepath{stroke}%
\end{pgfscope}%
\begin{pgfscope}%
\pgfpathrectangle{\pgfqpoint{0.691667in}{0.580556in}}{\pgfqpoint{3.184722in}{3.147222in}}%
\pgfusepath{clip}%
\pgfsetrectcap%
\pgfsetroundjoin%
\pgfsetlinewidth{1.505625pt}%
\definecolor{currentstroke}{rgb}{1.000000,0.498039,0.054902}%
\pgfsetstrokecolor{currentstroke}%
\pgfsetdash{}{0pt}%
\pgfpathmoveto{\pgfqpoint{0.836427in}{2.693029in}}%
\pgfpathlineto{\pgfqpoint{2.529819in}{2.693881in}}%
\pgfpathlineto{\pgfqpoint{2.530543in}{2.691442in}}%
\pgfpathlineto{\pgfqpoint{2.531267in}{2.693106in}}%
\pgfpathlineto{\pgfqpoint{2.531991in}{2.694936in}}%
\pgfpathlineto{\pgfqpoint{2.533439in}{2.691708in}}%
\pgfpathlineto{\pgfqpoint{2.534163in}{2.695776in}}%
\pgfpathlineto{\pgfqpoint{2.534887in}{2.692697in}}%
\pgfpathlineto{\pgfqpoint{2.535611in}{2.689859in}}%
\pgfpathlineto{\pgfqpoint{2.536335in}{2.695697in}}%
\pgfpathlineto{\pgfqpoint{2.537059in}{2.695038in}}%
\pgfpathlineto{\pgfqpoint{2.537783in}{2.688278in}}%
\pgfpathlineto{\pgfqpoint{2.538507in}{2.693921in}}%
\pgfpathlineto{\pgfqpoint{2.539231in}{2.698253in}}%
\pgfpathlineto{\pgfqpoint{2.539955in}{2.688168in}}%
\pgfpathlineto{\pgfqpoint{2.540679in}{2.690004in}}%
\pgfpathlineto{\pgfqpoint{2.541403in}{2.701167in}}%
\pgfpathlineto{\pgfqpoint{2.542127in}{2.690913in}}%
\pgfpathlineto{\pgfqpoint{2.542851in}{2.684404in}}%
\pgfpathlineto{\pgfqpoint{2.543575in}{2.701734in}}%
\pgfpathlineto{\pgfqpoint{2.544299in}{2.697398in}}%
\pgfpathlineto{\pgfqpoint{2.545023in}{2.679011in}}%
\pgfpathlineto{\pgfqpoint{2.545747in}{2.697496in}}%
\pgfpathlineto{\pgfqpoint{2.546471in}{2.707010in}}%
\pgfpathlineto{\pgfqpoint{2.547195in}{2.677290in}}%
\pgfpathlineto{\pgfqpoint{2.547919in}{2.686699in}}%
\pgfpathlineto{\pgfqpoint{2.548643in}{2.716680in}}%
\pgfpathlineto{\pgfqpoint{2.549367in}{2.683570in}}%
\pgfpathlineto{\pgfqpoint{2.550091in}{2.669979in}}%
\pgfpathlineto{\pgfqpoint{2.550815in}{2.720546in}}%
\pgfpathlineto{\pgfqpoint{2.551539in}{2.701152in}}%
\pgfpathlineto{\pgfqpoint{2.552263in}{2.652133in}}%
\pgfpathlineto{\pgfqpoint{2.553711in}{2.729319in}}%
\pgfpathlineto{\pgfqpoint{2.554435in}{2.642990in}}%
\pgfpathlineto{\pgfqpoint{2.555159in}{2.681578in}}%
\pgfpathlineto{\pgfqpoint{2.555883in}{2.760151in}}%
\pgfpathlineto{\pgfqpoint{2.557331in}{2.631958in}}%
\pgfpathlineto{\pgfqpoint{2.558055in}{2.776886in}}%
\pgfpathlineto{\pgfqpoint{2.558779in}{2.702550in}}%
\pgfpathlineto{\pgfqpoint{2.559503in}{2.573704in}}%
\pgfpathlineto{\pgfqpoint{2.560951in}{2.783497in}}%
\pgfpathlineto{\pgfqpoint{2.561675in}{2.534815in}}%
\pgfpathlineto{\pgfqpoint{2.562399in}{2.673871in}}%
\pgfpathlineto{\pgfqpoint{2.563123in}{2.877195in}}%
\pgfpathlineto{\pgfqpoint{2.564571in}{2.521709in}}%
\pgfpathlineto{\pgfqpoint{2.565295in}{2.931472in}}%
\pgfpathlineto{\pgfqpoint{2.566019in}{2.686583in}}%
\pgfpathlineto{\pgfqpoint{2.566743in}{2.326899in}}%
\pgfpathlineto{\pgfqpoint{2.568191in}{2.926435in}}%
\pgfpathlineto{\pgfqpoint{2.568914in}{2.188624in}}%
\pgfpathlineto{\pgfqpoint{2.569638in}{2.559935in}}%
\pgfpathlineto{\pgfqpoint{2.570362in}{3.182632in}}%
\pgfpathlineto{\pgfqpoint{2.571810in}{1.940603in}}%
\pgfpathlineto{\pgfqpoint{2.572534in}{3.125063in}}%
\pgfpathlineto{\pgfqpoint{2.573258in}{2.943344in}}%
\pgfpathlineto{\pgfqpoint{2.575430in}{0.723613in}}%
\pgfpathlineto{\pgfqpoint{3.057602in}{0.723611in}}%
\pgfpathlineto{\pgfqpoint{3.731629in}{0.723611in}}%
\pgfpathlineto{\pgfqpoint{3.731629in}{0.723611in}}%
\pgfusepath{stroke}%
\end{pgfscope}%
\begin{pgfscope}%
\pgfpathrectangle{\pgfqpoint{0.691667in}{0.580556in}}{\pgfqpoint{3.184722in}{3.147222in}}%
\pgfusepath{clip}%
\pgfsetrectcap%
\pgfsetroundjoin%
\pgfsetlinewidth{1.505625pt}%
\definecolor{currentstroke}{rgb}{0.172549,0.627451,0.172549}%
\pgfsetstrokecolor{currentstroke}%
\pgfsetdash{}{0pt}%
\pgfpathmoveto{\pgfqpoint{0.836427in}{2.693029in}}%
\pgfpathlineto{\pgfqpoint{2.570362in}{2.692375in}}%
\pgfpathlineto{\pgfqpoint{2.571086in}{2.689421in}}%
\pgfpathlineto{\pgfqpoint{2.571810in}{2.673234in}}%
\pgfpathlineto{\pgfqpoint{2.572534in}{2.587561in}}%
\pgfpathlineto{\pgfqpoint{2.577602in}{0.723611in}}%
\pgfpathlineto{\pgfqpoint{3.731629in}{0.723611in}}%
\pgfpathlineto{\pgfqpoint{3.731629in}{0.723611in}}%
\pgfusepath{stroke}%
\end{pgfscope}%
\begin{pgfscope}%
\pgfpathrectangle{\pgfqpoint{0.691667in}{0.580556in}}{\pgfqpoint{3.184722in}{3.147222in}}%
\pgfusepath{clip}%
\pgfsetrectcap%
\pgfsetroundjoin%
\pgfsetlinewidth{1.505625pt}%
\definecolor{currentstroke}{rgb}{0.839216,0.152941,0.156863}%
\pgfsetstrokecolor{currentstroke}%
\pgfsetdash{}{0pt}%
\pgfpathmoveto{\pgfqpoint{0.836427in}{2.693029in}}%
\pgfpathlineto{\pgfqpoint{2.283666in}{2.693029in}}%
\pgfpathlineto{\pgfqpoint{2.285838in}{0.723611in}}%
\pgfpathlineto{\pgfqpoint{3.731629in}{0.723611in}}%
\pgfpathlineto{\pgfqpoint{3.731629in}{0.723611in}}%
\pgfusepath{stroke}%
\end{pgfscope}%
\begin{pgfscope}%
\pgfpathrectangle{\pgfqpoint{0.691667in}{0.580556in}}{\pgfqpoint{3.184722in}{3.147222in}}%
\pgfusepath{clip}%
\pgfsetrectcap%
\pgfsetroundjoin%
\pgfsetlinewidth{0.501875pt}%
\definecolor{currentstroke}{rgb}{0.000000,0.000000,0.000000}%
\pgfsetstrokecolor{currentstroke}%
\pgfsetdash{}{0pt}%
\pgfpathmoveto{\pgfqpoint{0.836427in}{2.693029in}}%
\pgfpathlineto{\pgfqpoint{2.573258in}{2.693029in}}%
\pgfpathlineto{\pgfqpoint{2.575430in}{0.723611in}}%
\pgfpathlineto{\pgfqpoint{3.731629in}{0.723611in}}%
\pgfpathlineto{\pgfqpoint{3.731629in}{0.723611in}}%
\pgfusepath{stroke}%
\end{pgfscope}%
\begin{pgfscope}%
\pgfsetrectcap%
\pgfsetmiterjoin%
\pgfsetlinewidth{0.803000pt}%
\definecolor{currentstroke}{rgb}{0.000000,0.000000,0.000000}%
\pgfsetstrokecolor{currentstroke}%
\pgfsetdash{}{0pt}%
\pgfpathmoveto{\pgfqpoint{0.691667in}{0.580556in}}%
\pgfpathlineto{\pgfqpoint{0.691667in}{3.727778in}}%
\pgfusepath{stroke}%
\end{pgfscope}%
\begin{pgfscope}%
\pgfsetrectcap%
\pgfsetmiterjoin%
\pgfsetlinewidth{0.803000pt}%
\definecolor{currentstroke}{rgb}{0.000000,0.000000,0.000000}%
\pgfsetstrokecolor{currentstroke}%
\pgfsetdash{}{0pt}%
\pgfpathmoveto{\pgfqpoint{3.876389in}{0.580556in}}%
\pgfpathlineto{\pgfqpoint{3.876389in}{3.727778in}}%
\pgfusepath{stroke}%
\end{pgfscope}%
\begin{pgfscope}%
\pgfsetrectcap%
\pgfsetmiterjoin%
\pgfsetlinewidth{0.803000pt}%
\definecolor{currentstroke}{rgb}{0.000000,0.000000,0.000000}%
\pgfsetstrokecolor{currentstroke}%
\pgfsetdash{}{0pt}%
\pgfpathmoveto{\pgfqpoint{0.691667in}{0.580556in}}%
\pgfpathlineto{\pgfqpoint{3.876389in}{0.580556in}}%
\pgfusepath{stroke}%
\end{pgfscope}%
\begin{pgfscope}%
\pgfsetrectcap%
\pgfsetmiterjoin%
\pgfsetlinewidth{0.803000pt}%
\definecolor{currentstroke}{rgb}{0.000000,0.000000,0.000000}%
\pgfsetstrokecolor{currentstroke}%
\pgfsetdash{}{0pt}%
\pgfpathmoveto{\pgfqpoint{0.691667in}{3.727778in}}%
\pgfpathlineto{\pgfqpoint{3.876389in}{3.727778in}}%
\pgfusepath{stroke}%
\end{pgfscope}%
\begin{pgfscope}%
\definecolor{textcolor}{rgb}{0.000000,0.000000,0.000000}%
\pgfsetstrokecolor{textcolor}%
\pgfsetfillcolor{textcolor}%
\pgftext[x=2.284028in,y=3.811111in,,base]{\color{textcolor}\sffamily\fontsize{12.000000}{14.400000}\selectfont \(\displaystyle  t = 0.8 \)}%
\end{pgfscope}%
\begin{pgfscope}%
\pgfsetbuttcap%
\pgfsetmiterjoin%
\definecolor{currentfill}{rgb}{1.000000,1.000000,1.000000}%
\pgfsetfillcolor{currentfill}%
\pgfsetlinewidth{0.000000pt}%
\definecolor{currentstroke}{rgb}{0.000000,0.000000,0.000000}%
\pgfsetstrokecolor{currentstroke}%
\pgfsetstrokeopacity{0.000000}%
\pgfsetdash{}{0pt}%
\pgfpathmoveto{\pgfqpoint{4.616667in}{0.580556in}}%
\pgfpathlineto{\pgfqpoint{7.801389in}{0.580556in}}%
\pgfpathlineto{\pgfqpoint{7.801389in}{3.727778in}}%
\pgfpathlineto{\pgfqpoint{4.616667in}{3.727778in}}%
\pgfpathclose%
\pgfusepath{fill}%
\end{pgfscope}%
\begin{pgfscope}%
\pgfsetbuttcap%
\pgfsetroundjoin%
\definecolor{currentfill}{rgb}{0.000000,0.000000,0.000000}%
\pgfsetfillcolor{currentfill}%
\pgfsetlinewidth{0.803000pt}%
\definecolor{currentstroke}{rgb}{0.000000,0.000000,0.000000}%
\pgfsetstrokecolor{currentstroke}%
\pgfsetdash{}{0pt}%
\pgfsys@defobject{currentmarker}{\pgfqpoint{0.000000in}{-0.048611in}}{\pgfqpoint{0.000000in}{0.000000in}}{%
\pgfpathmoveto{\pgfqpoint{0.000000in}{0.000000in}}%
\pgfpathlineto{\pgfqpoint{0.000000in}{-0.048611in}}%
\pgfusepath{stroke,fill}%
}%
\begin{pgfscope}%
\pgfsys@transformshift{4.761065in}{0.580556in}%
\pgfsys@useobject{currentmarker}{}%
\end{pgfscope}%
\end{pgfscope}%
\begin{pgfscope}%
\definecolor{textcolor}{rgb}{0.000000,0.000000,0.000000}%
\pgfsetstrokecolor{textcolor}%
\pgfsetfillcolor{textcolor}%
\pgftext[x=4.761065in,y=0.483333in,,top]{\color{textcolor}\sffamily\fontsize{10.000000}{12.000000}\selectfont −2}%
\end{pgfscope}%
\begin{pgfscope}%
\pgfsetbuttcap%
\pgfsetroundjoin%
\definecolor{currentfill}{rgb}{0.000000,0.000000,0.000000}%
\pgfsetfillcolor{currentfill}%
\pgfsetlinewidth{0.803000pt}%
\definecolor{currentstroke}{rgb}{0.000000,0.000000,0.000000}%
\pgfsetstrokecolor{currentstroke}%
\pgfsetdash{}{0pt}%
\pgfsys@defobject{currentmarker}{\pgfqpoint{0.000000in}{-0.048611in}}{\pgfqpoint{0.000000in}{0.000000in}}{%
\pgfpathmoveto{\pgfqpoint{0.000000in}{0.000000in}}%
\pgfpathlineto{\pgfqpoint{0.000000in}{-0.048611in}}%
\pgfusepath{stroke,fill}%
}%
\begin{pgfscope}%
\pgfsys@transformshift{5.485046in}{0.580556in}%
\pgfsys@useobject{currentmarker}{}%
\end{pgfscope}%
\end{pgfscope}%
\begin{pgfscope}%
\definecolor{textcolor}{rgb}{0.000000,0.000000,0.000000}%
\pgfsetstrokecolor{textcolor}%
\pgfsetfillcolor{textcolor}%
\pgftext[x=5.485046in,y=0.483333in,,top]{\color{textcolor}\sffamily\fontsize{10.000000}{12.000000}\selectfont −1}%
\end{pgfscope}%
\begin{pgfscope}%
\pgfsetbuttcap%
\pgfsetroundjoin%
\definecolor{currentfill}{rgb}{0.000000,0.000000,0.000000}%
\pgfsetfillcolor{currentfill}%
\pgfsetlinewidth{0.803000pt}%
\definecolor{currentstroke}{rgb}{0.000000,0.000000,0.000000}%
\pgfsetstrokecolor{currentstroke}%
\pgfsetdash{}{0pt}%
\pgfsys@defobject{currentmarker}{\pgfqpoint{0.000000in}{-0.048611in}}{\pgfqpoint{0.000000in}{0.000000in}}{%
\pgfpathmoveto{\pgfqpoint{0.000000in}{0.000000in}}%
\pgfpathlineto{\pgfqpoint{0.000000in}{-0.048611in}}%
\pgfusepath{stroke,fill}%
}%
\begin{pgfscope}%
\pgfsys@transformshift{6.209028in}{0.580556in}%
\pgfsys@useobject{currentmarker}{}%
\end{pgfscope}%
\end{pgfscope}%
\begin{pgfscope}%
\definecolor{textcolor}{rgb}{0.000000,0.000000,0.000000}%
\pgfsetstrokecolor{textcolor}%
\pgfsetfillcolor{textcolor}%
\pgftext[x=6.209028in,y=0.483333in,,top]{\color{textcolor}\sffamily\fontsize{10.000000}{12.000000}\selectfont 0}%
\end{pgfscope}%
\begin{pgfscope}%
\pgfsetbuttcap%
\pgfsetroundjoin%
\definecolor{currentfill}{rgb}{0.000000,0.000000,0.000000}%
\pgfsetfillcolor{currentfill}%
\pgfsetlinewidth{0.803000pt}%
\definecolor{currentstroke}{rgb}{0.000000,0.000000,0.000000}%
\pgfsetstrokecolor{currentstroke}%
\pgfsetdash{}{0pt}%
\pgfsys@defobject{currentmarker}{\pgfqpoint{0.000000in}{-0.048611in}}{\pgfqpoint{0.000000in}{0.000000in}}{%
\pgfpathmoveto{\pgfqpoint{0.000000in}{0.000000in}}%
\pgfpathlineto{\pgfqpoint{0.000000in}{-0.048611in}}%
\pgfusepath{stroke,fill}%
}%
\begin{pgfscope}%
\pgfsys@transformshift{6.933009in}{0.580556in}%
\pgfsys@useobject{currentmarker}{}%
\end{pgfscope}%
\end{pgfscope}%
\begin{pgfscope}%
\definecolor{textcolor}{rgb}{0.000000,0.000000,0.000000}%
\pgfsetstrokecolor{textcolor}%
\pgfsetfillcolor{textcolor}%
\pgftext[x=6.933009in,y=0.483333in,,top]{\color{textcolor}\sffamily\fontsize{10.000000}{12.000000}\selectfont 1}%
\end{pgfscope}%
\begin{pgfscope}%
\pgfsetbuttcap%
\pgfsetroundjoin%
\definecolor{currentfill}{rgb}{0.000000,0.000000,0.000000}%
\pgfsetfillcolor{currentfill}%
\pgfsetlinewidth{0.803000pt}%
\definecolor{currentstroke}{rgb}{0.000000,0.000000,0.000000}%
\pgfsetstrokecolor{currentstroke}%
\pgfsetdash{}{0pt}%
\pgfsys@defobject{currentmarker}{\pgfqpoint{0.000000in}{-0.048611in}}{\pgfqpoint{0.000000in}{0.000000in}}{%
\pgfpathmoveto{\pgfqpoint{0.000000in}{0.000000in}}%
\pgfpathlineto{\pgfqpoint{0.000000in}{-0.048611in}}%
\pgfusepath{stroke,fill}%
}%
\begin{pgfscope}%
\pgfsys@transformshift{7.656991in}{0.580556in}%
\pgfsys@useobject{currentmarker}{}%
\end{pgfscope}%
\end{pgfscope}%
\begin{pgfscope}%
\definecolor{textcolor}{rgb}{0.000000,0.000000,0.000000}%
\pgfsetstrokecolor{textcolor}%
\pgfsetfillcolor{textcolor}%
\pgftext[x=7.656991in,y=0.483333in,,top]{\color{textcolor}\sffamily\fontsize{10.000000}{12.000000}\selectfont 2}%
\end{pgfscope}%
\begin{pgfscope}%
\definecolor{textcolor}{rgb}{0.000000,0.000000,0.000000}%
\pgfsetstrokecolor{textcolor}%
\pgfsetfillcolor{textcolor}%
\pgftext[x=6.209028in,y=0.293365in,,top]{\color{textcolor}\sffamily\fontsize{10.000000}{12.000000}\selectfont \(\displaystyle x\)}%
\end{pgfscope}%
\begin{pgfscope}%
\pgfsetbuttcap%
\pgfsetroundjoin%
\definecolor{currentfill}{rgb}{0.000000,0.000000,0.000000}%
\pgfsetfillcolor{currentfill}%
\pgfsetlinewidth{0.803000pt}%
\definecolor{currentstroke}{rgb}{0.000000,0.000000,0.000000}%
\pgfsetstrokecolor{currentstroke}%
\pgfsetdash{}{0pt}%
\pgfsys@defobject{currentmarker}{\pgfqpoint{-0.048611in}{0.000000in}}{\pgfqpoint{0.000000in}{0.000000in}}{%
\pgfpathmoveto{\pgfqpoint{0.000000in}{0.000000in}}%
\pgfpathlineto{\pgfqpoint{-0.048611in}{0.000000in}}%
\pgfusepath{stroke,fill}%
}%
\begin{pgfscope}%
\pgfsys@transformshift{4.616667in}{0.723611in}%
\pgfsys@useobject{currentmarker}{}%
\end{pgfscope}%
\end{pgfscope}%
\begin{pgfscope}%
\definecolor{textcolor}{rgb}{0.000000,0.000000,0.000000}%
\pgfsetstrokecolor{textcolor}%
\pgfsetfillcolor{textcolor}%
\pgftext[x=4.298565in,y=0.670850in,left,base]{\color{textcolor}\sffamily\fontsize{10.000000}{12.000000}\selectfont 0.0}%
\end{pgfscope}%
\begin{pgfscope}%
\pgfsetbuttcap%
\pgfsetroundjoin%
\definecolor{currentfill}{rgb}{0.000000,0.000000,0.000000}%
\pgfsetfillcolor{currentfill}%
\pgfsetlinewidth{0.803000pt}%
\definecolor{currentstroke}{rgb}{0.000000,0.000000,0.000000}%
\pgfsetstrokecolor{currentstroke}%
\pgfsetdash{}{0pt}%
\pgfsys@defobject{currentmarker}{\pgfqpoint{-0.048611in}{0.000000in}}{\pgfqpoint{0.000000in}{0.000000in}}{%
\pgfpathmoveto{\pgfqpoint{0.000000in}{0.000000in}}%
\pgfpathlineto{\pgfqpoint{-0.048611in}{0.000000in}}%
\pgfusepath{stroke,fill}%
}%
\begin{pgfscope}%
\pgfsys@transformshift{4.616667in}{1.117495in}%
\pgfsys@useobject{currentmarker}{}%
\end{pgfscope}%
\end{pgfscope}%
\begin{pgfscope}%
\definecolor{textcolor}{rgb}{0.000000,0.000000,0.000000}%
\pgfsetstrokecolor{textcolor}%
\pgfsetfillcolor{textcolor}%
\pgftext[x=4.298565in,y=1.064733in,left,base]{\color{textcolor}\sffamily\fontsize{10.000000}{12.000000}\selectfont 0.2}%
\end{pgfscope}%
\begin{pgfscope}%
\pgfsetbuttcap%
\pgfsetroundjoin%
\definecolor{currentfill}{rgb}{0.000000,0.000000,0.000000}%
\pgfsetfillcolor{currentfill}%
\pgfsetlinewidth{0.803000pt}%
\definecolor{currentstroke}{rgb}{0.000000,0.000000,0.000000}%
\pgfsetstrokecolor{currentstroke}%
\pgfsetdash{}{0pt}%
\pgfsys@defobject{currentmarker}{\pgfqpoint{-0.048611in}{0.000000in}}{\pgfqpoint{0.000000in}{0.000000in}}{%
\pgfpathmoveto{\pgfqpoint{0.000000in}{0.000000in}}%
\pgfpathlineto{\pgfqpoint{-0.048611in}{0.000000in}}%
\pgfusepath{stroke,fill}%
}%
\begin{pgfscope}%
\pgfsys@transformshift{4.616667in}{1.511378in}%
\pgfsys@useobject{currentmarker}{}%
\end{pgfscope}%
\end{pgfscope}%
\begin{pgfscope}%
\definecolor{textcolor}{rgb}{0.000000,0.000000,0.000000}%
\pgfsetstrokecolor{textcolor}%
\pgfsetfillcolor{textcolor}%
\pgftext[x=4.298565in,y=1.458617in,left,base]{\color{textcolor}\sffamily\fontsize{10.000000}{12.000000}\selectfont 0.4}%
\end{pgfscope}%
\begin{pgfscope}%
\pgfsetbuttcap%
\pgfsetroundjoin%
\definecolor{currentfill}{rgb}{0.000000,0.000000,0.000000}%
\pgfsetfillcolor{currentfill}%
\pgfsetlinewidth{0.803000pt}%
\definecolor{currentstroke}{rgb}{0.000000,0.000000,0.000000}%
\pgfsetstrokecolor{currentstroke}%
\pgfsetdash{}{0pt}%
\pgfsys@defobject{currentmarker}{\pgfqpoint{-0.048611in}{0.000000in}}{\pgfqpoint{0.000000in}{0.000000in}}{%
\pgfpathmoveto{\pgfqpoint{0.000000in}{0.000000in}}%
\pgfpathlineto{\pgfqpoint{-0.048611in}{0.000000in}}%
\pgfusepath{stroke,fill}%
}%
\begin{pgfscope}%
\pgfsys@transformshift{4.616667in}{1.905262in}%
\pgfsys@useobject{currentmarker}{}%
\end{pgfscope}%
\end{pgfscope}%
\begin{pgfscope}%
\definecolor{textcolor}{rgb}{0.000000,0.000000,0.000000}%
\pgfsetstrokecolor{textcolor}%
\pgfsetfillcolor{textcolor}%
\pgftext[x=4.298565in,y=1.852500in,left,base]{\color{textcolor}\sffamily\fontsize{10.000000}{12.000000}\selectfont 0.6}%
\end{pgfscope}%
\begin{pgfscope}%
\pgfsetbuttcap%
\pgfsetroundjoin%
\definecolor{currentfill}{rgb}{0.000000,0.000000,0.000000}%
\pgfsetfillcolor{currentfill}%
\pgfsetlinewidth{0.803000pt}%
\definecolor{currentstroke}{rgb}{0.000000,0.000000,0.000000}%
\pgfsetstrokecolor{currentstroke}%
\pgfsetdash{}{0pt}%
\pgfsys@defobject{currentmarker}{\pgfqpoint{-0.048611in}{0.000000in}}{\pgfqpoint{0.000000in}{0.000000in}}{%
\pgfpathmoveto{\pgfqpoint{0.000000in}{0.000000in}}%
\pgfpathlineto{\pgfqpoint{-0.048611in}{0.000000in}}%
\pgfusepath{stroke,fill}%
}%
\begin{pgfscope}%
\pgfsys@transformshift{4.616667in}{2.299146in}%
\pgfsys@useobject{currentmarker}{}%
\end{pgfscope}%
\end{pgfscope}%
\begin{pgfscope}%
\definecolor{textcolor}{rgb}{0.000000,0.000000,0.000000}%
\pgfsetstrokecolor{textcolor}%
\pgfsetfillcolor{textcolor}%
\pgftext[x=4.298565in,y=2.246384in,left,base]{\color{textcolor}\sffamily\fontsize{10.000000}{12.000000}\selectfont 0.8}%
\end{pgfscope}%
\begin{pgfscope}%
\pgfsetbuttcap%
\pgfsetroundjoin%
\definecolor{currentfill}{rgb}{0.000000,0.000000,0.000000}%
\pgfsetfillcolor{currentfill}%
\pgfsetlinewidth{0.803000pt}%
\definecolor{currentstroke}{rgb}{0.000000,0.000000,0.000000}%
\pgfsetstrokecolor{currentstroke}%
\pgfsetdash{}{0pt}%
\pgfsys@defobject{currentmarker}{\pgfqpoint{-0.048611in}{0.000000in}}{\pgfqpoint{0.000000in}{0.000000in}}{%
\pgfpathmoveto{\pgfqpoint{0.000000in}{0.000000in}}%
\pgfpathlineto{\pgfqpoint{-0.048611in}{0.000000in}}%
\pgfusepath{stroke,fill}%
}%
\begin{pgfscope}%
\pgfsys@transformshift{4.616667in}{2.693029in}%
\pgfsys@useobject{currentmarker}{}%
\end{pgfscope}%
\end{pgfscope}%
\begin{pgfscope}%
\definecolor{textcolor}{rgb}{0.000000,0.000000,0.000000}%
\pgfsetstrokecolor{textcolor}%
\pgfsetfillcolor{textcolor}%
\pgftext[x=4.298565in,y=2.640268in,left,base]{\color{textcolor}\sffamily\fontsize{10.000000}{12.000000}\selectfont 1.0}%
\end{pgfscope}%
\begin{pgfscope}%
\pgfsetbuttcap%
\pgfsetroundjoin%
\definecolor{currentfill}{rgb}{0.000000,0.000000,0.000000}%
\pgfsetfillcolor{currentfill}%
\pgfsetlinewidth{0.803000pt}%
\definecolor{currentstroke}{rgb}{0.000000,0.000000,0.000000}%
\pgfsetstrokecolor{currentstroke}%
\pgfsetdash{}{0pt}%
\pgfsys@defobject{currentmarker}{\pgfqpoint{-0.048611in}{0.000000in}}{\pgfqpoint{0.000000in}{0.000000in}}{%
\pgfpathmoveto{\pgfqpoint{0.000000in}{0.000000in}}%
\pgfpathlineto{\pgfqpoint{-0.048611in}{0.000000in}}%
\pgfusepath{stroke,fill}%
}%
\begin{pgfscope}%
\pgfsys@transformshift{4.616667in}{3.086913in}%
\pgfsys@useobject{currentmarker}{}%
\end{pgfscope}%
\end{pgfscope}%
\begin{pgfscope}%
\definecolor{textcolor}{rgb}{0.000000,0.000000,0.000000}%
\pgfsetstrokecolor{textcolor}%
\pgfsetfillcolor{textcolor}%
\pgftext[x=4.298565in,y=3.034151in,left,base]{\color{textcolor}\sffamily\fontsize{10.000000}{12.000000}\selectfont 1.2}%
\end{pgfscope}%
\begin{pgfscope}%
\pgfsetbuttcap%
\pgfsetroundjoin%
\definecolor{currentfill}{rgb}{0.000000,0.000000,0.000000}%
\pgfsetfillcolor{currentfill}%
\pgfsetlinewidth{0.803000pt}%
\definecolor{currentstroke}{rgb}{0.000000,0.000000,0.000000}%
\pgfsetstrokecolor{currentstroke}%
\pgfsetdash{}{0pt}%
\pgfsys@defobject{currentmarker}{\pgfqpoint{-0.048611in}{0.000000in}}{\pgfqpoint{0.000000in}{0.000000in}}{%
\pgfpathmoveto{\pgfqpoint{0.000000in}{0.000000in}}%
\pgfpathlineto{\pgfqpoint{-0.048611in}{0.000000in}}%
\pgfusepath{stroke,fill}%
}%
\begin{pgfscope}%
\pgfsys@transformshift{4.616667in}{3.480797in}%
\pgfsys@useobject{currentmarker}{}%
\end{pgfscope}%
\end{pgfscope}%
\begin{pgfscope}%
\definecolor{textcolor}{rgb}{0.000000,0.000000,0.000000}%
\pgfsetstrokecolor{textcolor}%
\pgfsetfillcolor{textcolor}%
\pgftext[x=4.298565in,y=3.428035in,left,base]{\color{textcolor}\sffamily\fontsize{10.000000}{12.000000}\selectfont 1.4}%
\end{pgfscope}%
\begin{pgfscope}%
\definecolor{textcolor}{rgb}{0.000000,0.000000,0.000000}%
\pgfsetstrokecolor{textcolor}%
\pgfsetfillcolor{textcolor}%
\pgftext[x=4.243009in,y=2.154167in,,bottom,rotate=90.000000]{\color{textcolor}\sffamily\fontsize{10.000000}{12.000000}\selectfont \(\displaystyle U\)}%
\end{pgfscope}%
\begin{pgfscope}%
\pgfpathrectangle{\pgfqpoint{4.616667in}{0.580556in}}{\pgfqpoint{3.184722in}{3.147222in}}%
\pgfusepath{clip}%
\pgfsetrectcap%
\pgfsetroundjoin%
\pgfsetlinewidth{1.505625pt}%
\definecolor{currentstroke}{rgb}{0.121569,0.466667,0.705882}%
\pgfsetstrokecolor{currentstroke}%
\pgfsetdash{}{0pt}%
\pgfpathmoveto{\pgfqpoint{4.761427in}{2.693029in}}%
\pgfpathlineto{\pgfqpoint{6.525046in}{2.693034in}}%
\pgfpathlineto{\pgfqpoint{6.525770in}{2.691154in}}%
\pgfpathlineto{\pgfqpoint{6.527218in}{2.694398in}}%
\pgfpathlineto{\pgfqpoint{6.527942in}{2.690401in}}%
\pgfpathlineto{\pgfqpoint{6.528666in}{2.693211in}}%
\pgfpathlineto{\pgfqpoint{6.529390in}{2.696151in}}%
\pgfpathlineto{\pgfqpoint{6.530114in}{2.690571in}}%
\pgfpathlineto{\pgfqpoint{6.530838in}{2.690917in}}%
\pgfpathlineto{\pgfqpoint{6.531562in}{2.697577in}}%
\pgfpathlineto{\pgfqpoint{6.532286in}{2.692390in}}%
\pgfpathlineto{\pgfqpoint{6.533009in}{2.687851in}}%
\pgfpathlineto{\pgfqpoint{6.533733in}{2.697511in}}%
\pgfpathlineto{\pgfqpoint{6.534457in}{2.696228in}}%
\pgfpathlineto{\pgfqpoint{6.535181in}{2.685184in}}%
\pgfpathlineto{\pgfqpoint{6.535905in}{2.694656in}}%
\pgfpathlineto{\pgfqpoint{6.536629in}{2.701563in}}%
\pgfpathlineto{\pgfqpoint{6.537353in}{2.684894in}}%
\pgfpathlineto{\pgfqpoint{6.538077in}{2.688248in}}%
\pgfpathlineto{\pgfqpoint{6.538801in}{2.706483in}}%
\pgfpathlineto{\pgfqpoint{6.539525in}{2.689286in}}%
\pgfpathlineto{\pgfqpoint{6.540249in}{2.679000in}}%
\pgfpathlineto{\pgfqpoint{6.540973in}{2.707611in}}%
\pgfpathlineto{\pgfqpoint{6.541697in}{2.699893in}}%
\pgfpathlineto{\pgfqpoint{6.542421in}{2.669981in}}%
\pgfpathlineto{\pgfqpoint{6.543145in}{2.700845in}}%
\pgfpathlineto{\pgfqpoint{6.543869in}{2.715839in}}%
\pgfpathlineto{\pgfqpoint{6.544593in}{2.666849in}}%
\pgfpathlineto{\pgfqpoint{6.545317in}{2.683235in}}%
\pgfpathlineto{\pgfqpoint{6.546041in}{2.732166in}}%
\pgfpathlineto{\pgfqpoint{6.547489in}{2.655785in}}%
\pgfpathlineto{\pgfqpoint{6.548213in}{2.739154in}}%
\pgfpathlineto{\pgfqpoint{6.548937in}{2.705439in}}%
\pgfpathlineto{\pgfqpoint{6.549661in}{2.626296in}}%
\pgfpathlineto{\pgfqpoint{6.551109in}{2.752299in}}%
\pgfpathlineto{\pgfqpoint{6.551833in}{2.610625in}}%
\pgfpathlineto{\pgfqpoint{6.552557in}{2.676195in}}%
\pgfpathlineto{\pgfqpoint{6.553281in}{2.804823in}}%
\pgfpathlineto{\pgfqpoint{6.554729in}{2.595623in}}%
\pgfpathlineto{\pgfqpoint{6.555453in}{2.834886in}}%
\pgfpathlineto{\pgfqpoint{6.556177in}{2.706013in}}%
\pgfpathlineto{\pgfqpoint{6.556901in}{2.501765in}}%
\pgfpathlineto{\pgfqpoint{6.558349in}{2.841900in}}%
\pgfpathlineto{\pgfqpoint{6.559073in}{2.438545in}}%
\pgfpathlineto{\pgfqpoint{6.559797in}{2.668847in}}%
\pgfpathlineto{\pgfqpoint{6.560521in}{3.006521in}}%
\pgfpathlineto{\pgfqpoint{6.561969in}{2.428284in}}%
\pgfpathlineto{\pgfqpoint{6.562693in}{3.109961in}}%
\pgfpathlineto{\pgfqpoint{6.563417in}{2.674656in}}%
\pgfpathlineto{\pgfqpoint{6.564141in}{2.135242in}}%
\pgfpathlineto{\pgfqpoint{6.565589in}{3.086454in}}%
\pgfpathlineto{\pgfqpoint{6.566313in}{1.932789in}}%
\pgfpathlineto{\pgfqpoint{6.567037in}{2.517760in}}%
\pgfpathlineto{\pgfqpoint{6.567761in}{3.584722in}}%
\pgfpathlineto{\pgfqpoint{6.569209in}{1.659499in}}%
\pgfpathlineto{\pgfqpoint{6.569933in}{3.547630in}}%
\pgfpathlineto{\pgfqpoint{6.570657in}{3.072202in}}%
\pgfpathlineto{\pgfqpoint{6.572828in}{0.723611in}}%
\pgfpathlineto{\pgfqpoint{7.656629in}{0.723611in}}%
\pgfpathlineto{\pgfqpoint{7.656629in}{0.723611in}}%
\pgfusepath{stroke}%
\end{pgfscope}%
\begin{pgfscope}%
\pgfpathrectangle{\pgfqpoint{4.616667in}{0.580556in}}{\pgfqpoint{3.184722in}{3.147222in}}%
\pgfusepath{clip}%
\pgfsetrectcap%
\pgfsetroundjoin%
\pgfsetlinewidth{1.505625pt}%
\definecolor{currentstroke}{rgb}{1.000000,0.498039,0.054902}%
\pgfsetstrokecolor{currentstroke}%
\pgfsetdash{}{0pt}%
\pgfpathmoveto{\pgfqpoint{4.761427in}{2.693029in}}%
\pgfpathlineto{\pgfqpoint{6.527218in}{2.693881in}}%
\pgfpathlineto{\pgfqpoint{6.527942in}{2.691442in}}%
\pgfpathlineto{\pgfqpoint{6.528666in}{2.693106in}}%
\pgfpathlineto{\pgfqpoint{6.529390in}{2.694936in}}%
\pgfpathlineto{\pgfqpoint{6.530838in}{2.691708in}}%
\pgfpathlineto{\pgfqpoint{6.531562in}{2.695776in}}%
\pgfpathlineto{\pgfqpoint{6.532286in}{2.692697in}}%
\pgfpathlineto{\pgfqpoint{6.533009in}{2.689859in}}%
\pgfpathlineto{\pgfqpoint{6.533733in}{2.695697in}}%
\pgfpathlineto{\pgfqpoint{6.534457in}{2.695038in}}%
\pgfpathlineto{\pgfqpoint{6.535181in}{2.688278in}}%
\pgfpathlineto{\pgfqpoint{6.535905in}{2.693921in}}%
\pgfpathlineto{\pgfqpoint{6.536629in}{2.698253in}}%
\pgfpathlineto{\pgfqpoint{6.537353in}{2.688168in}}%
\pgfpathlineto{\pgfqpoint{6.538077in}{2.690004in}}%
\pgfpathlineto{\pgfqpoint{6.538801in}{2.701167in}}%
\pgfpathlineto{\pgfqpoint{6.539525in}{2.690913in}}%
\pgfpathlineto{\pgfqpoint{6.540249in}{2.684404in}}%
\pgfpathlineto{\pgfqpoint{6.540973in}{2.701734in}}%
\pgfpathlineto{\pgfqpoint{6.541697in}{2.697398in}}%
\pgfpathlineto{\pgfqpoint{6.542421in}{2.679011in}}%
\pgfpathlineto{\pgfqpoint{6.543145in}{2.697496in}}%
\pgfpathlineto{\pgfqpoint{6.543869in}{2.707010in}}%
\pgfpathlineto{\pgfqpoint{6.544593in}{2.677290in}}%
\pgfpathlineto{\pgfqpoint{6.545317in}{2.686699in}}%
\pgfpathlineto{\pgfqpoint{6.546041in}{2.716680in}}%
\pgfpathlineto{\pgfqpoint{6.546765in}{2.683570in}}%
\pgfpathlineto{\pgfqpoint{6.547489in}{2.669979in}}%
\pgfpathlineto{\pgfqpoint{6.548213in}{2.720546in}}%
\pgfpathlineto{\pgfqpoint{6.548937in}{2.701152in}}%
\pgfpathlineto{\pgfqpoint{6.549661in}{2.652133in}}%
\pgfpathlineto{\pgfqpoint{6.551109in}{2.729319in}}%
\pgfpathlineto{\pgfqpoint{6.551833in}{2.642990in}}%
\pgfpathlineto{\pgfqpoint{6.552557in}{2.681578in}}%
\pgfpathlineto{\pgfqpoint{6.553281in}{2.760151in}}%
\pgfpathlineto{\pgfqpoint{6.554729in}{2.631958in}}%
\pgfpathlineto{\pgfqpoint{6.555453in}{2.776886in}}%
\pgfpathlineto{\pgfqpoint{6.556177in}{2.702550in}}%
\pgfpathlineto{\pgfqpoint{6.556901in}{2.573704in}}%
\pgfpathlineto{\pgfqpoint{6.558349in}{2.783497in}}%
\pgfpathlineto{\pgfqpoint{6.559073in}{2.534815in}}%
\pgfpathlineto{\pgfqpoint{6.559797in}{2.673871in}}%
\pgfpathlineto{\pgfqpoint{6.560521in}{2.877195in}}%
\pgfpathlineto{\pgfqpoint{6.561969in}{2.521709in}}%
\pgfpathlineto{\pgfqpoint{6.562693in}{2.931472in}}%
\pgfpathlineto{\pgfqpoint{6.563417in}{2.686583in}}%
\pgfpathlineto{\pgfqpoint{6.564141in}{2.326899in}}%
\pgfpathlineto{\pgfqpoint{6.565589in}{2.926435in}}%
\pgfpathlineto{\pgfqpoint{6.566313in}{2.188624in}}%
\pgfpathlineto{\pgfqpoint{6.567037in}{2.559935in}}%
\pgfpathlineto{\pgfqpoint{6.567761in}{3.182632in}}%
\pgfpathlineto{\pgfqpoint{6.569209in}{1.940603in}}%
\pgfpathlineto{\pgfqpoint{6.569933in}{3.125063in}}%
\pgfpathlineto{\pgfqpoint{6.570657in}{2.943344in}}%
\pgfpathlineto{\pgfqpoint{6.572828in}{0.723613in}}%
\pgfpathlineto{\pgfqpoint{7.055000in}{0.723611in}}%
\pgfpathlineto{\pgfqpoint{7.656629in}{0.723611in}}%
\pgfpathlineto{\pgfqpoint{7.656629in}{0.723611in}}%
\pgfusepath{stroke}%
\end{pgfscope}%
\begin{pgfscope}%
\pgfpathrectangle{\pgfqpoint{4.616667in}{0.580556in}}{\pgfqpoint{3.184722in}{3.147222in}}%
\pgfusepath{clip}%
\pgfsetrectcap%
\pgfsetroundjoin%
\pgfsetlinewidth{1.505625pt}%
\definecolor{currentstroke}{rgb}{0.172549,0.627451,0.172549}%
\pgfsetstrokecolor{currentstroke}%
\pgfsetdash{}{0pt}%
\pgfpathmoveto{\pgfqpoint{4.761427in}{2.693029in}}%
\pgfpathlineto{\pgfqpoint{6.567761in}{2.692375in}}%
\pgfpathlineto{\pgfqpoint{6.568485in}{2.689421in}}%
\pgfpathlineto{\pgfqpoint{6.569209in}{2.673234in}}%
\pgfpathlineto{\pgfqpoint{6.569933in}{2.587561in}}%
\pgfpathlineto{\pgfqpoint{6.575000in}{0.723611in}}%
\pgfpathlineto{\pgfqpoint{7.656629in}{0.723611in}}%
\pgfpathlineto{\pgfqpoint{7.656629in}{0.723611in}}%
\pgfusepath{stroke}%
\end{pgfscope}%
\begin{pgfscope}%
\pgfpathrectangle{\pgfqpoint{4.616667in}{0.580556in}}{\pgfqpoint{3.184722in}{3.147222in}}%
\pgfusepath{clip}%
\pgfsetrectcap%
\pgfsetroundjoin%
\pgfsetlinewidth{1.505625pt}%
\definecolor{currentstroke}{rgb}{0.839216,0.152941,0.156863}%
\pgfsetstrokecolor{currentstroke}%
\pgfsetdash{}{0pt}%
\pgfpathmoveto{\pgfqpoint{4.761427in}{2.693029in}}%
\pgfpathlineto{\pgfqpoint{6.208666in}{2.693029in}}%
\pgfpathlineto{\pgfqpoint{6.210838in}{0.723611in}}%
\pgfpathlineto{\pgfqpoint{7.656629in}{0.723611in}}%
\pgfpathlineto{\pgfqpoint{7.656629in}{0.723611in}}%
\pgfusepath{stroke}%
\end{pgfscope}%
\begin{pgfscope}%
\pgfpathrectangle{\pgfqpoint{4.616667in}{0.580556in}}{\pgfqpoint{3.184722in}{3.147222in}}%
\pgfusepath{clip}%
\pgfsetrectcap%
\pgfsetroundjoin%
\pgfsetlinewidth{0.501875pt}%
\definecolor{currentstroke}{rgb}{0.000000,0.000000,0.000000}%
\pgfsetstrokecolor{currentstroke}%
\pgfsetdash{}{0pt}%
\pgfpathmoveto{\pgfqpoint{4.761427in}{2.693029in}}%
\pgfpathlineto{\pgfqpoint{6.570657in}{2.693029in}}%
\pgfpathlineto{\pgfqpoint{6.572828in}{0.723611in}}%
\pgfpathlineto{\pgfqpoint{7.656629in}{0.723611in}}%
\pgfpathlineto{\pgfqpoint{7.656629in}{0.723611in}}%
\pgfusepath{stroke}%
\end{pgfscope}%
\begin{pgfscope}%
\pgfsetrectcap%
\pgfsetmiterjoin%
\pgfsetlinewidth{0.803000pt}%
\definecolor{currentstroke}{rgb}{0.000000,0.000000,0.000000}%
\pgfsetstrokecolor{currentstroke}%
\pgfsetdash{}{0pt}%
\pgfpathmoveto{\pgfqpoint{4.616667in}{0.580556in}}%
\pgfpathlineto{\pgfqpoint{4.616667in}{3.727778in}}%
\pgfusepath{stroke}%
\end{pgfscope}%
\begin{pgfscope}%
\pgfsetrectcap%
\pgfsetmiterjoin%
\pgfsetlinewidth{0.803000pt}%
\definecolor{currentstroke}{rgb}{0.000000,0.000000,0.000000}%
\pgfsetstrokecolor{currentstroke}%
\pgfsetdash{}{0pt}%
\pgfpathmoveto{\pgfqpoint{7.801389in}{0.580556in}}%
\pgfpathlineto{\pgfqpoint{7.801389in}{3.727778in}}%
\pgfusepath{stroke}%
\end{pgfscope}%
\begin{pgfscope}%
\pgfsetrectcap%
\pgfsetmiterjoin%
\pgfsetlinewidth{0.803000pt}%
\definecolor{currentstroke}{rgb}{0.000000,0.000000,0.000000}%
\pgfsetstrokecolor{currentstroke}%
\pgfsetdash{}{0pt}%
\pgfpathmoveto{\pgfqpoint{4.616667in}{0.580556in}}%
\pgfpathlineto{\pgfqpoint{7.801389in}{0.580556in}}%
\pgfusepath{stroke}%
\end{pgfscope}%
\begin{pgfscope}%
\pgfsetrectcap%
\pgfsetmiterjoin%
\pgfsetlinewidth{0.803000pt}%
\definecolor{currentstroke}{rgb}{0.000000,0.000000,0.000000}%
\pgfsetstrokecolor{currentstroke}%
\pgfsetdash{}{0pt}%
\pgfpathmoveto{\pgfqpoint{4.616667in}{3.727778in}}%
\pgfpathlineto{\pgfqpoint{7.801389in}{3.727778in}}%
\pgfusepath{stroke}%
\end{pgfscope}%
\begin{pgfscope}%
\definecolor{textcolor}{rgb}{0.000000,0.000000,0.000000}%
\pgfsetstrokecolor{textcolor}%
\pgfsetfillcolor{textcolor}%
\pgftext[x=6.209028in,y=3.811111in,,base]{\color{textcolor}\sffamily\fontsize{12.000000}{14.400000}\selectfont \(\displaystyle  t = 1.0 \)}%
\end{pgfscope}%
\begin{pgfscope}%
\pgfsetbuttcap%
\pgfsetmiterjoin%
\definecolor{currentfill}{rgb}{1.000000,1.000000,1.000000}%
\pgfsetfillcolor{currentfill}%
\pgfsetlinewidth{0.000000pt}%
\definecolor{currentstroke}{rgb}{0.000000,0.000000,0.000000}%
\pgfsetstrokecolor{currentstroke}%
\pgfsetstrokeopacity{0.000000}%
\pgfsetdash{}{0pt}%
\pgfpathmoveto{\pgfqpoint{0.691667in}{8.480556in}}%
\pgfpathlineto{\pgfqpoint{3.876389in}{8.480556in}}%
\pgfpathlineto{\pgfqpoint{3.876389in}{11.627778in}}%
\pgfpathlineto{\pgfqpoint{0.691667in}{11.627778in}}%
\pgfpathclose%
\pgfusepath{fill}%
\end{pgfscope}%
\begin{pgfscope}%
\pgfsetbuttcap%
\pgfsetroundjoin%
\definecolor{currentfill}{rgb}{0.000000,0.000000,0.000000}%
\pgfsetfillcolor{currentfill}%
\pgfsetlinewidth{0.803000pt}%
\definecolor{currentstroke}{rgb}{0.000000,0.000000,0.000000}%
\pgfsetstrokecolor{currentstroke}%
\pgfsetdash{}{0pt}%
\pgfsys@defobject{currentmarker}{\pgfqpoint{0.000000in}{-0.048611in}}{\pgfqpoint{0.000000in}{0.000000in}}{%
\pgfpathmoveto{\pgfqpoint{0.000000in}{0.000000in}}%
\pgfpathlineto{\pgfqpoint{0.000000in}{-0.048611in}}%
\pgfusepath{stroke,fill}%
}%
\begin{pgfscope}%
\pgfsys@transformshift{0.836065in}{8.480556in}%
\pgfsys@useobject{currentmarker}{}%
\end{pgfscope}%
\end{pgfscope}%
\begin{pgfscope}%
\definecolor{textcolor}{rgb}{0.000000,0.000000,0.000000}%
\pgfsetstrokecolor{textcolor}%
\pgfsetfillcolor{textcolor}%
\pgftext[x=0.836065in,y=8.383333in,,top]{\color{textcolor}\sffamily\fontsize{10.000000}{12.000000}\selectfont −2}%
\end{pgfscope}%
\begin{pgfscope}%
\pgfsetbuttcap%
\pgfsetroundjoin%
\definecolor{currentfill}{rgb}{0.000000,0.000000,0.000000}%
\pgfsetfillcolor{currentfill}%
\pgfsetlinewidth{0.803000pt}%
\definecolor{currentstroke}{rgb}{0.000000,0.000000,0.000000}%
\pgfsetstrokecolor{currentstroke}%
\pgfsetdash{}{0pt}%
\pgfsys@defobject{currentmarker}{\pgfqpoint{0.000000in}{-0.048611in}}{\pgfqpoint{0.000000in}{0.000000in}}{%
\pgfpathmoveto{\pgfqpoint{0.000000in}{0.000000in}}%
\pgfpathlineto{\pgfqpoint{0.000000in}{-0.048611in}}%
\pgfusepath{stroke,fill}%
}%
\begin{pgfscope}%
\pgfsys@transformshift{1.560046in}{8.480556in}%
\pgfsys@useobject{currentmarker}{}%
\end{pgfscope}%
\end{pgfscope}%
\begin{pgfscope}%
\definecolor{textcolor}{rgb}{0.000000,0.000000,0.000000}%
\pgfsetstrokecolor{textcolor}%
\pgfsetfillcolor{textcolor}%
\pgftext[x=1.560046in,y=8.383333in,,top]{\color{textcolor}\sffamily\fontsize{10.000000}{12.000000}\selectfont −1}%
\end{pgfscope}%
\begin{pgfscope}%
\pgfsetbuttcap%
\pgfsetroundjoin%
\definecolor{currentfill}{rgb}{0.000000,0.000000,0.000000}%
\pgfsetfillcolor{currentfill}%
\pgfsetlinewidth{0.803000pt}%
\definecolor{currentstroke}{rgb}{0.000000,0.000000,0.000000}%
\pgfsetstrokecolor{currentstroke}%
\pgfsetdash{}{0pt}%
\pgfsys@defobject{currentmarker}{\pgfqpoint{0.000000in}{-0.048611in}}{\pgfqpoint{0.000000in}{0.000000in}}{%
\pgfpathmoveto{\pgfqpoint{0.000000in}{0.000000in}}%
\pgfpathlineto{\pgfqpoint{0.000000in}{-0.048611in}}%
\pgfusepath{stroke,fill}%
}%
\begin{pgfscope}%
\pgfsys@transformshift{2.284028in}{8.480556in}%
\pgfsys@useobject{currentmarker}{}%
\end{pgfscope}%
\end{pgfscope}%
\begin{pgfscope}%
\definecolor{textcolor}{rgb}{0.000000,0.000000,0.000000}%
\pgfsetstrokecolor{textcolor}%
\pgfsetfillcolor{textcolor}%
\pgftext[x=2.284028in,y=8.383333in,,top]{\color{textcolor}\sffamily\fontsize{10.000000}{12.000000}\selectfont 0}%
\end{pgfscope}%
\begin{pgfscope}%
\pgfsetbuttcap%
\pgfsetroundjoin%
\definecolor{currentfill}{rgb}{0.000000,0.000000,0.000000}%
\pgfsetfillcolor{currentfill}%
\pgfsetlinewidth{0.803000pt}%
\definecolor{currentstroke}{rgb}{0.000000,0.000000,0.000000}%
\pgfsetstrokecolor{currentstroke}%
\pgfsetdash{}{0pt}%
\pgfsys@defobject{currentmarker}{\pgfqpoint{0.000000in}{-0.048611in}}{\pgfqpoint{0.000000in}{0.000000in}}{%
\pgfpathmoveto{\pgfqpoint{0.000000in}{0.000000in}}%
\pgfpathlineto{\pgfqpoint{0.000000in}{-0.048611in}}%
\pgfusepath{stroke,fill}%
}%
\begin{pgfscope}%
\pgfsys@transformshift{3.008009in}{8.480556in}%
\pgfsys@useobject{currentmarker}{}%
\end{pgfscope}%
\end{pgfscope}%
\begin{pgfscope}%
\definecolor{textcolor}{rgb}{0.000000,0.000000,0.000000}%
\pgfsetstrokecolor{textcolor}%
\pgfsetfillcolor{textcolor}%
\pgftext[x=3.008009in,y=8.383333in,,top]{\color{textcolor}\sffamily\fontsize{10.000000}{12.000000}\selectfont 1}%
\end{pgfscope}%
\begin{pgfscope}%
\pgfsetbuttcap%
\pgfsetroundjoin%
\definecolor{currentfill}{rgb}{0.000000,0.000000,0.000000}%
\pgfsetfillcolor{currentfill}%
\pgfsetlinewidth{0.803000pt}%
\definecolor{currentstroke}{rgb}{0.000000,0.000000,0.000000}%
\pgfsetstrokecolor{currentstroke}%
\pgfsetdash{}{0pt}%
\pgfsys@defobject{currentmarker}{\pgfqpoint{0.000000in}{-0.048611in}}{\pgfqpoint{0.000000in}{0.000000in}}{%
\pgfpathmoveto{\pgfqpoint{0.000000in}{0.000000in}}%
\pgfpathlineto{\pgfqpoint{0.000000in}{-0.048611in}}%
\pgfusepath{stroke,fill}%
}%
\begin{pgfscope}%
\pgfsys@transformshift{3.731991in}{8.480556in}%
\pgfsys@useobject{currentmarker}{}%
\end{pgfscope}%
\end{pgfscope}%
\begin{pgfscope}%
\definecolor{textcolor}{rgb}{0.000000,0.000000,0.000000}%
\pgfsetstrokecolor{textcolor}%
\pgfsetfillcolor{textcolor}%
\pgftext[x=3.731991in,y=8.383333in,,top]{\color{textcolor}\sffamily\fontsize{10.000000}{12.000000}\selectfont 2}%
\end{pgfscope}%
\begin{pgfscope}%
\definecolor{textcolor}{rgb}{0.000000,0.000000,0.000000}%
\pgfsetstrokecolor{textcolor}%
\pgfsetfillcolor{textcolor}%
\pgftext[x=2.284028in,y=8.193365in,,top]{\color{textcolor}\sffamily\fontsize{10.000000}{12.000000}\selectfont \(\displaystyle x\)}%
\end{pgfscope}%
\begin{pgfscope}%
\pgfsetbuttcap%
\pgfsetroundjoin%
\definecolor{currentfill}{rgb}{0.000000,0.000000,0.000000}%
\pgfsetfillcolor{currentfill}%
\pgfsetlinewidth{0.803000pt}%
\definecolor{currentstroke}{rgb}{0.000000,0.000000,0.000000}%
\pgfsetstrokecolor{currentstroke}%
\pgfsetdash{}{0pt}%
\pgfsys@defobject{currentmarker}{\pgfqpoint{-0.048611in}{0.000000in}}{\pgfqpoint{0.000000in}{0.000000in}}{%
\pgfpathmoveto{\pgfqpoint{0.000000in}{0.000000in}}%
\pgfpathlineto{\pgfqpoint{-0.048611in}{0.000000in}}%
\pgfusepath{stroke,fill}%
}%
\begin{pgfscope}%
\pgfsys@transformshift{0.691667in}{8.623611in}%
\pgfsys@useobject{currentmarker}{}%
\end{pgfscope}%
\end{pgfscope}%
\begin{pgfscope}%
\definecolor{textcolor}{rgb}{0.000000,0.000000,0.000000}%
\pgfsetstrokecolor{textcolor}%
\pgfsetfillcolor{textcolor}%
\pgftext[x=0.373565in,y=8.570850in,left,base]{\color{textcolor}\sffamily\fontsize{10.000000}{12.000000}\selectfont 0.0}%
\end{pgfscope}%
\begin{pgfscope}%
\pgfsetbuttcap%
\pgfsetroundjoin%
\definecolor{currentfill}{rgb}{0.000000,0.000000,0.000000}%
\pgfsetfillcolor{currentfill}%
\pgfsetlinewidth{0.803000pt}%
\definecolor{currentstroke}{rgb}{0.000000,0.000000,0.000000}%
\pgfsetstrokecolor{currentstroke}%
\pgfsetdash{}{0pt}%
\pgfsys@defobject{currentmarker}{\pgfqpoint{-0.048611in}{0.000000in}}{\pgfqpoint{0.000000in}{0.000000in}}{%
\pgfpathmoveto{\pgfqpoint{0.000000in}{0.000000in}}%
\pgfpathlineto{\pgfqpoint{-0.048611in}{0.000000in}}%
\pgfusepath{stroke,fill}%
}%
\begin{pgfscope}%
\pgfsys@transformshift{0.691667in}{9.195833in}%
\pgfsys@useobject{currentmarker}{}%
\end{pgfscope}%
\end{pgfscope}%
\begin{pgfscope}%
\definecolor{textcolor}{rgb}{0.000000,0.000000,0.000000}%
\pgfsetstrokecolor{textcolor}%
\pgfsetfillcolor{textcolor}%
\pgftext[x=0.373565in,y=9.143072in,left,base]{\color{textcolor}\sffamily\fontsize{10.000000}{12.000000}\selectfont 0.2}%
\end{pgfscope}%
\begin{pgfscope}%
\pgfsetbuttcap%
\pgfsetroundjoin%
\definecolor{currentfill}{rgb}{0.000000,0.000000,0.000000}%
\pgfsetfillcolor{currentfill}%
\pgfsetlinewidth{0.803000pt}%
\definecolor{currentstroke}{rgb}{0.000000,0.000000,0.000000}%
\pgfsetstrokecolor{currentstroke}%
\pgfsetdash{}{0pt}%
\pgfsys@defobject{currentmarker}{\pgfqpoint{-0.048611in}{0.000000in}}{\pgfqpoint{0.000000in}{0.000000in}}{%
\pgfpathmoveto{\pgfqpoint{0.000000in}{0.000000in}}%
\pgfpathlineto{\pgfqpoint{-0.048611in}{0.000000in}}%
\pgfusepath{stroke,fill}%
}%
\begin{pgfscope}%
\pgfsys@transformshift{0.691667in}{9.768056in}%
\pgfsys@useobject{currentmarker}{}%
\end{pgfscope}%
\end{pgfscope}%
\begin{pgfscope}%
\definecolor{textcolor}{rgb}{0.000000,0.000000,0.000000}%
\pgfsetstrokecolor{textcolor}%
\pgfsetfillcolor{textcolor}%
\pgftext[x=0.373565in,y=9.715294in,left,base]{\color{textcolor}\sffamily\fontsize{10.000000}{12.000000}\selectfont 0.4}%
\end{pgfscope}%
\begin{pgfscope}%
\pgfsetbuttcap%
\pgfsetroundjoin%
\definecolor{currentfill}{rgb}{0.000000,0.000000,0.000000}%
\pgfsetfillcolor{currentfill}%
\pgfsetlinewidth{0.803000pt}%
\definecolor{currentstroke}{rgb}{0.000000,0.000000,0.000000}%
\pgfsetstrokecolor{currentstroke}%
\pgfsetdash{}{0pt}%
\pgfsys@defobject{currentmarker}{\pgfqpoint{-0.048611in}{0.000000in}}{\pgfqpoint{0.000000in}{0.000000in}}{%
\pgfpathmoveto{\pgfqpoint{0.000000in}{0.000000in}}%
\pgfpathlineto{\pgfqpoint{-0.048611in}{0.000000in}}%
\pgfusepath{stroke,fill}%
}%
\begin{pgfscope}%
\pgfsys@transformshift{0.691667in}{10.340278in}%
\pgfsys@useobject{currentmarker}{}%
\end{pgfscope}%
\end{pgfscope}%
\begin{pgfscope}%
\definecolor{textcolor}{rgb}{0.000000,0.000000,0.000000}%
\pgfsetstrokecolor{textcolor}%
\pgfsetfillcolor{textcolor}%
\pgftext[x=0.373565in,y=10.287516in,left,base]{\color{textcolor}\sffamily\fontsize{10.000000}{12.000000}\selectfont 0.6}%
\end{pgfscope}%
\begin{pgfscope}%
\pgfsetbuttcap%
\pgfsetroundjoin%
\definecolor{currentfill}{rgb}{0.000000,0.000000,0.000000}%
\pgfsetfillcolor{currentfill}%
\pgfsetlinewidth{0.803000pt}%
\definecolor{currentstroke}{rgb}{0.000000,0.000000,0.000000}%
\pgfsetstrokecolor{currentstroke}%
\pgfsetdash{}{0pt}%
\pgfsys@defobject{currentmarker}{\pgfqpoint{-0.048611in}{0.000000in}}{\pgfqpoint{0.000000in}{0.000000in}}{%
\pgfpathmoveto{\pgfqpoint{0.000000in}{0.000000in}}%
\pgfpathlineto{\pgfqpoint{-0.048611in}{0.000000in}}%
\pgfusepath{stroke,fill}%
}%
\begin{pgfscope}%
\pgfsys@transformshift{0.691667in}{10.912500in}%
\pgfsys@useobject{currentmarker}{}%
\end{pgfscope}%
\end{pgfscope}%
\begin{pgfscope}%
\definecolor{textcolor}{rgb}{0.000000,0.000000,0.000000}%
\pgfsetstrokecolor{textcolor}%
\pgfsetfillcolor{textcolor}%
\pgftext[x=0.373565in,y=10.859738in,left,base]{\color{textcolor}\sffamily\fontsize{10.000000}{12.000000}\selectfont 0.8}%
\end{pgfscope}%
\begin{pgfscope}%
\pgfsetbuttcap%
\pgfsetroundjoin%
\definecolor{currentfill}{rgb}{0.000000,0.000000,0.000000}%
\pgfsetfillcolor{currentfill}%
\pgfsetlinewidth{0.803000pt}%
\definecolor{currentstroke}{rgb}{0.000000,0.000000,0.000000}%
\pgfsetstrokecolor{currentstroke}%
\pgfsetdash{}{0pt}%
\pgfsys@defobject{currentmarker}{\pgfqpoint{-0.048611in}{0.000000in}}{\pgfqpoint{0.000000in}{0.000000in}}{%
\pgfpathmoveto{\pgfqpoint{0.000000in}{0.000000in}}%
\pgfpathlineto{\pgfqpoint{-0.048611in}{0.000000in}}%
\pgfusepath{stroke,fill}%
}%
\begin{pgfscope}%
\pgfsys@transformshift{0.691667in}{11.484722in}%
\pgfsys@useobject{currentmarker}{}%
\end{pgfscope}%
\end{pgfscope}%
\begin{pgfscope}%
\definecolor{textcolor}{rgb}{0.000000,0.000000,0.000000}%
\pgfsetstrokecolor{textcolor}%
\pgfsetfillcolor{textcolor}%
\pgftext[x=0.373565in,y=11.431961in,left,base]{\color{textcolor}\sffamily\fontsize{10.000000}{12.000000}\selectfont 1.0}%
\end{pgfscope}%
\begin{pgfscope}%
\definecolor{textcolor}{rgb}{0.000000,0.000000,0.000000}%
\pgfsetstrokecolor{textcolor}%
\pgfsetfillcolor{textcolor}%
\pgftext[x=0.318009in,y=10.054167in,,bottom,rotate=90.000000]{\color{textcolor}\sffamily\fontsize{10.000000}{12.000000}\selectfont \(\displaystyle U\)}%
\end{pgfscope}%
\begin{pgfscope}%
\pgfpathrectangle{\pgfqpoint{0.691667in}{8.480556in}}{\pgfqpoint{3.184722in}{3.147222in}}%
\pgfusepath{clip}%
\pgfsetrectcap%
\pgfsetroundjoin%
\pgfsetlinewidth{0.501875pt}%
\definecolor{currentstroke}{rgb}{0.000000,0.000000,0.000000}%
\pgfsetstrokecolor{currentstroke}%
\pgfsetdash{}{0pt}%
\pgfpathmoveto{\pgfqpoint{0.836427in}{11.484722in}}%
\pgfpathlineto{\pgfqpoint{2.283666in}{11.484722in}}%
\pgfpathlineto{\pgfqpoint{2.285838in}{8.623611in}}%
\pgfpathlineto{\pgfqpoint{3.731629in}{8.623611in}}%
\pgfpathlineto{\pgfqpoint{3.731629in}{8.623611in}}%
\pgfusepath{stroke}%
\end{pgfscope}%
\begin{pgfscope}%
\pgfsetrectcap%
\pgfsetmiterjoin%
\pgfsetlinewidth{0.803000pt}%
\definecolor{currentstroke}{rgb}{0.000000,0.000000,0.000000}%
\pgfsetstrokecolor{currentstroke}%
\pgfsetdash{}{0pt}%
\pgfpathmoveto{\pgfqpoint{0.691667in}{8.480556in}}%
\pgfpathlineto{\pgfqpoint{0.691667in}{11.627778in}}%
\pgfusepath{stroke}%
\end{pgfscope}%
\begin{pgfscope}%
\pgfsetrectcap%
\pgfsetmiterjoin%
\pgfsetlinewidth{0.803000pt}%
\definecolor{currentstroke}{rgb}{0.000000,0.000000,0.000000}%
\pgfsetstrokecolor{currentstroke}%
\pgfsetdash{}{0pt}%
\pgfpathmoveto{\pgfqpoint{3.876389in}{8.480556in}}%
\pgfpathlineto{\pgfqpoint{3.876389in}{11.627778in}}%
\pgfusepath{stroke}%
\end{pgfscope}%
\begin{pgfscope}%
\pgfsetrectcap%
\pgfsetmiterjoin%
\pgfsetlinewidth{0.803000pt}%
\definecolor{currentstroke}{rgb}{0.000000,0.000000,0.000000}%
\pgfsetstrokecolor{currentstroke}%
\pgfsetdash{}{0pt}%
\pgfpathmoveto{\pgfqpoint{0.691667in}{8.480556in}}%
\pgfpathlineto{\pgfqpoint{3.876389in}{8.480556in}}%
\pgfusepath{stroke}%
\end{pgfscope}%
\begin{pgfscope}%
\pgfsetrectcap%
\pgfsetmiterjoin%
\pgfsetlinewidth{0.803000pt}%
\definecolor{currentstroke}{rgb}{0.000000,0.000000,0.000000}%
\pgfsetstrokecolor{currentstroke}%
\pgfsetdash{}{0pt}%
\pgfpathmoveto{\pgfqpoint{0.691667in}{11.627778in}}%
\pgfpathlineto{\pgfqpoint{3.876389in}{11.627778in}}%
\pgfusepath{stroke}%
\end{pgfscope}%
\begin{pgfscope}%
\definecolor{textcolor}{rgb}{0.000000,0.000000,0.000000}%
\pgfsetstrokecolor{textcolor}%
\pgfsetfillcolor{textcolor}%
\pgftext[x=2.284028in,y=11.711111in,,base]{\color{textcolor}\sffamily\fontsize{12.000000}{14.400000}\selectfont \(\displaystyle  t = 0.0 \)}%
\end{pgfscope}%
\begin{pgfscope}%
\pgfsetbuttcap%
\pgfsetmiterjoin%
\definecolor{currentfill}{rgb}{1.000000,1.000000,1.000000}%
\pgfsetfillcolor{currentfill}%
\pgfsetfillopacity{0.800000}%
\pgfsetlinewidth{1.003750pt}%
\definecolor{currentstroke}{rgb}{0.800000,0.800000,0.800000}%
\pgfsetstrokecolor{currentstroke}%
\pgfsetstrokeopacity{0.800000}%
\pgfsetdash{}{0pt}%
\pgfpathmoveto{\pgfqpoint{1.566737in}{9.523690in}}%
\pgfpathlineto{\pgfqpoint{3.779167in}{9.523690in}}%
\pgfpathquadraticcurveto{\pgfqpoint{3.806944in}{9.523690in}}{\pgfqpoint{3.806944in}{9.551468in}}%
\pgfpathlineto{\pgfqpoint{3.806944in}{10.556865in}}%
\pgfpathquadraticcurveto{\pgfqpoint{3.806944in}{10.584643in}}{\pgfqpoint{3.779167in}{10.584643in}}%
\pgfpathlineto{\pgfqpoint{1.566737in}{10.584643in}}%
\pgfpathquadraticcurveto{\pgfqpoint{1.538959in}{10.584643in}}{\pgfqpoint{1.538959in}{10.556865in}}%
\pgfpathlineto{\pgfqpoint{1.538959in}{9.551468in}}%
\pgfpathquadraticcurveto{\pgfqpoint{1.538959in}{9.523690in}}{\pgfqpoint{1.566737in}{9.523690in}}%
\pgfpathclose%
\pgfusepath{stroke,fill}%
\end{pgfscope}%
\begin{pgfscope}%
\pgfsetrectcap%
\pgfsetroundjoin%
\pgfsetlinewidth{1.505625pt}%
\definecolor{currentstroke}{rgb}{0.121569,0.466667,0.705882}%
\pgfsetstrokecolor{currentstroke}%
\pgfsetdash{}{0pt}%
\pgfpathmoveto{\pgfqpoint{1.594515in}{10.472176in}}%
\pgfpathlineto{\pgfqpoint{1.872293in}{10.472176in}}%
\pgfusepath{stroke}%
\end{pgfscope}%
\begin{pgfscope}%
\definecolor{textcolor}{rgb}{0.000000,0.000000,0.000000}%
\pgfsetstrokecolor{textcolor}%
\pgfsetfillcolor{textcolor}%
\pgftext[x=1.983404in,y=10.423564in,left,base]{\color{textcolor}\sffamily\fontsize{10.000000}{12.000000}\selectfont Richtmyer}%
\end{pgfscope}%
\begin{pgfscope}%
\pgfsetrectcap%
\pgfsetroundjoin%
\pgfsetlinewidth{1.505625pt}%
\definecolor{currentstroke}{rgb}{1.000000,0.498039,0.054902}%
\pgfsetstrokecolor{currentstroke}%
\pgfsetdash{}{0pt}%
\pgfpathmoveto{\pgfqpoint{1.594515in}{10.268318in}}%
\pgfpathlineto{\pgfqpoint{1.872293in}{10.268318in}}%
\pgfusepath{stroke}%
\end{pgfscope}%
\begin{pgfscope}%
\definecolor{textcolor}{rgb}{0.000000,0.000000,0.000000}%
\pgfsetstrokecolor{textcolor}%
\pgfsetfillcolor{textcolor}%
\pgftext[x=1.983404in,y=10.219707in,left,base]{\color{textcolor}\sffamily\fontsize{10.000000}{12.000000}\selectfont Lax--Wendroff}%
\end{pgfscope}%
\begin{pgfscope}%
\pgfsetrectcap%
\pgfsetroundjoin%
\pgfsetlinewidth{1.505625pt}%
\definecolor{currentstroke}{rgb}{0.172549,0.627451,0.172549}%
\pgfsetstrokecolor{currentstroke}%
\pgfsetdash{}{0pt}%
\pgfpathmoveto{\pgfqpoint{1.594515in}{10.064461in}}%
\pgfpathlineto{\pgfqpoint{1.872293in}{10.064461in}}%
\pgfusepath{stroke}%
\end{pgfscope}%
\begin{pgfscope}%
\definecolor{textcolor}{rgb}{0.000000,0.000000,0.000000}%
\pgfsetstrokecolor{textcolor}%
\pgfsetfillcolor{textcolor}%
\pgftext[x=1.983404in,y=10.015850in,left,base]{\color{textcolor}\sffamily\fontsize{10.000000}{12.000000}\selectfont Conservative upwind}%
\end{pgfscope}%
\begin{pgfscope}%
\pgfsetrectcap%
\pgfsetroundjoin%
\pgfsetlinewidth{1.505625pt}%
\definecolor{currentstroke}{rgb}{0.839216,0.152941,0.156863}%
\pgfsetstrokecolor{currentstroke}%
\pgfsetdash{}{0pt}%
\pgfpathmoveto{\pgfqpoint{1.594515in}{9.860604in}}%
\pgfpathlineto{\pgfqpoint{1.872293in}{9.860604in}}%
\pgfusepath{stroke}%
\end{pgfscope}%
\begin{pgfscope}%
\definecolor{textcolor}{rgb}{0.000000,0.000000,0.000000}%
\pgfsetstrokecolor{textcolor}%
\pgfsetfillcolor{textcolor}%
\pgftext[x=1.983404in,y=9.811993in,left,base]{\color{textcolor}\sffamily\fontsize{10.000000}{12.000000}\selectfont Non-conservative upwind}%
\end{pgfscope}%
\begin{pgfscope}%
\pgfsetrectcap%
\pgfsetroundjoin%
\pgfsetlinewidth{0.501875pt}%
\definecolor{currentstroke}{rgb}{0.000000,0.000000,0.000000}%
\pgfsetstrokecolor{currentstroke}%
\pgfsetdash{}{0pt}%
\pgfpathmoveto{\pgfqpoint{1.594515in}{9.656747in}}%
\pgfpathlineto{\pgfqpoint{1.872293in}{9.656747in}}%
\pgfusepath{stroke}%
\end{pgfscope}%
\begin{pgfscope}%
\definecolor{textcolor}{rgb}{0.000000,0.000000,0.000000}%
\pgfsetstrokecolor{textcolor}%
\pgfsetfillcolor{textcolor}%
\pgftext[x=1.983404in,y=9.608136in,left,base]{\color{textcolor}\sffamily\fontsize{10.000000}{12.000000}\selectfont Analytical}%
\end{pgfscope}%
\end{pgfpicture}%
\makeatother%
\endgroup%

\caption{Error for the first equation with respect to $N$}
\label{Fig:Err1}
\end{figure}

\begin{table}[htbp]
\centering
\begin{tabular}{|c|c|c|c|c|c|}
\hline
$N$ & $L^{\infty}$ error & $L^2$ error & $H^1$ error & Time (\Si{s}) & \#Iterations \\
\hline
\input{Table1.tbl}
\end{tabular}
\caption{Numerical results using triangle elements for the first equation}
\label{Tbl:SumTri1}
\end{table}

\begin{table}[htbp]
\centering
\begin{tabular}{|c|c|c|c|c|c|}
\hline
$N$ & $L^{\infty}$ error & $L^2$ error & $H^1$ error & Time (\Si{s}) & \#Iterations \\
\hline
\input{Table2.tbl}
\end{tabular}
\caption{Numerical results using rectangle elements for the first equation}
\label{Tbl:SumRect1}
\end{table}

It can be directly verify the second-order convergence of $L^{\infty}$ and $L^2$ norm, as well as the first-order convergence under $H^1$ semi-norm. We discover that the rectangle elements commits smaller error, and is slightly faster since the number of iterations is smaller. However, we use na\"ive conjugate gradient method here, and the speed can be increased using some other modern methods.

\subsection{Second equation}

We solve the equation and plot the solution, the errors at the node, and the errors of the whole function in Figure \ref{Fig:Tri2} and \ref{Fig:Rect2}.

\begin{figure}[htbp]
{
\centering
\scalebox{0.7}{%% Creator: Matplotlib, PGF backend
%%
%% To include the figure in your LaTeX document, write
%%   \input{<filename>.pgf}
%%
%% Make sure the required packages are loaded in your preamble
%%   \usepackage{pgf}
%%
%% Figures using additional raster images can only be included by \input if
%% they are in the same directory as the main LaTeX file. For loading figures
%% from other directories you can use the `import` package
%%   \usepackage{import}
%% and then include the figures with
%%   \import{<path to file>}{<filename>.pgf}
%%
%% Matplotlib used the following preamble
%%   \usepackage{fontspec}
%%   \setmainfont{DejaVuSerif.ttf}[Path=/home/lzh/anaconda3/envs/numpde/lib/python3.7/site-packages/matplotlib/mpl-data/fonts/ttf/]
%%   \setsansfont{DejaVuSans.ttf}[Path=/home/lzh/anaconda3/envs/numpde/lib/python3.7/site-packages/matplotlib/mpl-data/fonts/ttf/]
%%   \setmonofont{DejaVuSansMono.ttf}[Path=/home/lzh/anaconda3/envs/numpde/lib/python3.7/site-packages/matplotlib/mpl-data/fonts/ttf/]
%%
\begingroup%
\makeatletter%
\begin{pgfpicture}%
\pgfpathrectangle{\pgfpointorigin}{\pgfqpoint{8.000000in}{12.000000in}}%
\pgfusepath{use as bounding box, clip}%
\begin{pgfscope}%
\pgfsetbuttcap%
\pgfsetmiterjoin%
\definecolor{currentfill}{rgb}{1.000000,1.000000,1.000000}%
\pgfsetfillcolor{currentfill}%
\pgfsetlinewidth{0.000000pt}%
\definecolor{currentstroke}{rgb}{1.000000,1.000000,1.000000}%
\pgfsetstrokecolor{currentstroke}%
\pgfsetdash{}{0pt}%
\pgfpathmoveto{\pgfqpoint{0.000000in}{0.000000in}}%
\pgfpathlineto{\pgfqpoint{8.000000in}{0.000000in}}%
\pgfpathlineto{\pgfqpoint{8.000000in}{12.000000in}}%
\pgfpathlineto{\pgfqpoint{0.000000in}{12.000000in}}%
\pgfpathclose%
\pgfusepath{fill}%
\end{pgfscope}%
\begin{pgfscope}%
\pgfsetbuttcap%
\pgfsetmiterjoin%
\definecolor{currentfill}{rgb}{1.000000,1.000000,1.000000}%
\pgfsetfillcolor{currentfill}%
\pgfsetlinewidth{0.000000pt}%
\definecolor{currentstroke}{rgb}{0.000000,0.000000,0.000000}%
\pgfsetstrokecolor{currentstroke}%
\pgfsetstrokeopacity{0.000000}%
\pgfsetdash{}{0pt}%
\pgfpathmoveto{\pgfqpoint{4.616667in}{8.480556in}}%
\pgfpathlineto{\pgfqpoint{7.801389in}{8.480556in}}%
\pgfpathlineto{\pgfqpoint{7.801389in}{11.627778in}}%
\pgfpathlineto{\pgfqpoint{4.616667in}{11.627778in}}%
\pgfpathclose%
\pgfusepath{fill}%
\end{pgfscope}%
\begin{pgfscope}%
\pgfsetbuttcap%
\pgfsetroundjoin%
\definecolor{currentfill}{rgb}{0.000000,0.000000,0.000000}%
\pgfsetfillcolor{currentfill}%
\pgfsetlinewidth{0.803000pt}%
\definecolor{currentstroke}{rgb}{0.000000,0.000000,0.000000}%
\pgfsetstrokecolor{currentstroke}%
\pgfsetdash{}{0pt}%
\pgfsys@defobject{currentmarker}{\pgfqpoint{0.000000in}{-0.048611in}}{\pgfqpoint{0.000000in}{0.000000in}}{%
\pgfpathmoveto{\pgfqpoint{0.000000in}{0.000000in}}%
\pgfpathlineto{\pgfqpoint{0.000000in}{-0.048611in}}%
\pgfusepath{stroke,fill}%
}%
\begin{pgfscope}%
\pgfsys@transformshift{4.761391in}{8.480556in}%
\pgfsys@useobject{currentmarker}{}%
\end{pgfscope}%
\end{pgfscope}%
\begin{pgfscope}%
\definecolor{textcolor}{rgb}{0.000000,0.000000,0.000000}%
\pgfsetstrokecolor{textcolor}%
\pgfsetfillcolor{textcolor}%
\pgftext[x=4.761391in,y=8.383333in,,top]{\color{textcolor}\sffamily\fontsize{10.000000}{12.000000}\selectfont −2}%
\end{pgfscope}%
\begin{pgfscope}%
\pgfsetbuttcap%
\pgfsetroundjoin%
\definecolor{currentfill}{rgb}{0.000000,0.000000,0.000000}%
\pgfsetfillcolor{currentfill}%
\pgfsetlinewidth{0.803000pt}%
\definecolor{currentstroke}{rgb}{0.000000,0.000000,0.000000}%
\pgfsetstrokecolor{currentstroke}%
\pgfsetdash{}{0pt}%
\pgfsys@defobject{currentmarker}{\pgfqpoint{0.000000in}{-0.048611in}}{\pgfqpoint{0.000000in}{0.000000in}}{%
\pgfpathmoveto{\pgfqpoint{0.000000in}{0.000000in}}%
\pgfpathlineto{\pgfqpoint{0.000000in}{-0.048611in}}%
\pgfusepath{stroke,fill}%
}%
\begin{pgfscope}%
\pgfsys@transformshift{5.485209in}{8.480556in}%
\pgfsys@useobject{currentmarker}{}%
\end{pgfscope}%
\end{pgfscope}%
\begin{pgfscope}%
\definecolor{textcolor}{rgb}{0.000000,0.000000,0.000000}%
\pgfsetstrokecolor{textcolor}%
\pgfsetfillcolor{textcolor}%
\pgftext[x=5.485209in,y=8.383333in,,top]{\color{textcolor}\sffamily\fontsize{10.000000}{12.000000}\selectfont −1}%
\end{pgfscope}%
\begin{pgfscope}%
\pgfsetbuttcap%
\pgfsetroundjoin%
\definecolor{currentfill}{rgb}{0.000000,0.000000,0.000000}%
\pgfsetfillcolor{currentfill}%
\pgfsetlinewidth{0.803000pt}%
\definecolor{currentstroke}{rgb}{0.000000,0.000000,0.000000}%
\pgfsetstrokecolor{currentstroke}%
\pgfsetdash{}{0pt}%
\pgfsys@defobject{currentmarker}{\pgfqpoint{0.000000in}{-0.048611in}}{\pgfqpoint{0.000000in}{0.000000in}}{%
\pgfpathmoveto{\pgfqpoint{0.000000in}{0.000000in}}%
\pgfpathlineto{\pgfqpoint{0.000000in}{-0.048611in}}%
\pgfusepath{stroke,fill}%
}%
\begin{pgfscope}%
\pgfsys@transformshift{6.209028in}{8.480556in}%
\pgfsys@useobject{currentmarker}{}%
\end{pgfscope}%
\end{pgfscope}%
\begin{pgfscope}%
\definecolor{textcolor}{rgb}{0.000000,0.000000,0.000000}%
\pgfsetstrokecolor{textcolor}%
\pgfsetfillcolor{textcolor}%
\pgftext[x=6.209028in,y=8.383333in,,top]{\color{textcolor}\sffamily\fontsize{10.000000}{12.000000}\selectfont 0}%
\end{pgfscope}%
\begin{pgfscope}%
\pgfsetbuttcap%
\pgfsetroundjoin%
\definecolor{currentfill}{rgb}{0.000000,0.000000,0.000000}%
\pgfsetfillcolor{currentfill}%
\pgfsetlinewidth{0.803000pt}%
\definecolor{currentstroke}{rgb}{0.000000,0.000000,0.000000}%
\pgfsetstrokecolor{currentstroke}%
\pgfsetdash{}{0pt}%
\pgfsys@defobject{currentmarker}{\pgfqpoint{0.000000in}{-0.048611in}}{\pgfqpoint{0.000000in}{0.000000in}}{%
\pgfpathmoveto{\pgfqpoint{0.000000in}{0.000000in}}%
\pgfpathlineto{\pgfqpoint{0.000000in}{-0.048611in}}%
\pgfusepath{stroke,fill}%
}%
\begin{pgfscope}%
\pgfsys@transformshift{6.932846in}{8.480556in}%
\pgfsys@useobject{currentmarker}{}%
\end{pgfscope}%
\end{pgfscope}%
\begin{pgfscope}%
\definecolor{textcolor}{rgb}{0.000000,0.000000,0.000000}%
\pgfsetstrokecolor{textcolor}%
\pgfsetfillcolor{textcolor}%
\pgftext[x=6.932846in,y=8.383333in,,top]{\color{textcolor}\sffamily\fontsize{10.000000}{12.000000}\selectfont 1}%
\end{pgfscope}%
\begin{pgfscope}%
\pgfsetbuttcap%
\pgfsetroundjoin%
\definecolor{currentfill}{rgb}{0.000000,0.000000,0.000000}%
\pgfsetfillcolor{currentfill}%
\pgfsetlinewidth{0.803000pt}%
\definecolor{currentstroke}{rgb}{0.000000,0.000000,0.000000}%
\pgfsetstrokecolor{currentstroke}%
\pgfsetdash{}{0pt}%
\pgfsys@defobject{currentmarker}{\pgfqpoint{0.000000in}{-0.048611in}}{\pgfqpoint{0.000000in}{0.000000in}}{%
\pgfpathmoveto{\pgfqpoint{0.000000in}{0.000000in}}%
\pgfpathlineto{\pgfqpoint{0.000000in}{-0.048611in}}%
\pgfusepath{stroke,fill}%
}%
\begin{pgfscope}%
\pgfsys@transformshift{7.656665in}{8.480556in}%
\pgfsys@useobject{currentmarker}{}%
\end{pgfscope}%
\end{pgfscope}%
\begin{pgfscope}%
\definecolor{textcolor}{rgb}{0.000000,0.000000,0.000000}%
\pgfsetstrokecolor{textcolor}%
\pgfsetfillcolor{textcolor}%
\pgftext[x=7.656665in,y=8.383333in,,top]{\color{textcolor}\sffamily\fontsize{10.000000}{12.000000}\selectfont 2}%
\end{pgfscope}%
\begin{pgfscope}%
\definecolor{textcolor}{rgb}{0.000000,0.000000,0.000000}%
\pgfsetstrokecolor{textcolor}%
\pgfsetfillcolor{textcolor}%
\pgftext[x=6.209028in,y=8.193365in,,top]{\color{textcolor}\sffamily\fontsize{10.000000}{12.000000}\selectfont \(\displaystyle x\)}%
\end{pgfscope}%
\begin{pgfscope}%
\pgfsetbuttcap%
\pgfsetroundjoin%
\definecolor{currentfill}{rgb}{0.000000,0.000000,0.000000}%
\pgfsetfillcolor{currentfill}%
\pgfsetlinewidth{0.803000pt}%
\definecolor{currentstroke}{rgb}{0.000000,0.000000,0.000000}%
\pgfsetstrokecolor{currentstroke}%
\pgfsetdash{}{0pt}%
\pgfsys@defobject{currentmarker}{\pgfqpoint{-0.048611in}{0.000000in}}{\pgfqpoint{0.000000in}{0.000000in}}{%
\pgfpathmoveto{\pgfqpoint{0.000000in}{0.000000in}}%
\pgfpathlineto{\pgfqpoint{-0.048611in}{0.000000in}}%
\pgfusepath{stroke,fill}%
}%
\begin{pgfscope}%
\pgfsys@transformshift{4.616667in}{8.623611in}%
\pgfsys@useobject{currentmarker}{}%
\end{pgfscope}%
\end{pgfscope}%
\begin{pgfscope}%
\definecolor{textcolor}{rgb}{0.000000,0.000000,0.000000}%
\pgfsetstrokecolor{textcolor}%
\pgfsetfillcolor{textcolor}%
\pgftext[x=4.298565in,y=8.570850in,left,base]{\color{textcolor}\sffamily\fontsize{10.000000}{12.000000}\selectfont 0.0}%
\end{pgfscope}%
\begin{pgfscope}%
\pgfsetbuttcap%
\pgfsetroundjoin%
\definecolor{currentfill}{rgb}{0.000000,0.000000,0.000000}%
\pgfsetfillcolor{currentfill}%
\pgfsetlinewidth{0.803000pt}%
\definecolor{currentstroke}{rgb}{0.000000,0.000000,0.000000}%
\pgfsetstrokecolor{currentstroke}%
\pgfsetdash{}{0pt}%
\pgfsys@defobject{currentmarker}{\pgfqpoint{-0.048611in}{0.000000in}}{\pgfqpoint{0.000000in}{0.000000in}}{%
\pgfpathmoveto{\pgfqpoint{0.000000in}{0.000000in}}%
\pgfpathlineto{\pgfqpoint{-0.048611in}{0.000000in}}%
\pgfusepath{stroke,fill}%
}%
\begin{pgfscope}%
\pgfsys@transformshift{4.616667in}{9.017495in}%
\pgfsys@useobject{currentmarker}{}%
\end{pgfscope}%
\end{pgfscope}%
\begin{pgfscope}%
\definecolor{textcolor}{rgb}{0.000000,0.000000,0.000000}%
\pgfsetstrokecolor{textcolor}%
\pgfsetfillcolor{textcolor}%
\pgftext[x=4.298565in,y=8.964733in,left,base]{\color{textcolor}\sffamily\fontsize{10.000000}{12.000000}\selectfont 0.2}%
\end{pgfscope}%
\begin{pgfscope}%
\pgfsetbuttcap%
\pgfsetroundjoin%
\definecolor{currentfill}{rgb}{0.000000,0.000000,0.000000}%
\pgfsetfillcolor{currentfill}%
\pgfsetlinewidth{0.803000pt}%
\definecolor{currentstroke}{rgb}{0.000000,0.000000,0.000000}%
\pgfsetstrokecolor{currentstroke}%
\pgfsetdash{}{0pt}%
\pgfsys@defobject{currentmarker}{\pgfqpoint{-0.048611in}{0.000000in}}{\pgfqpoint{0.000000in}{0.000000in}}{%
\pgfpathmoveto{\pgfqpoint{0.000000in}{0.000000in}}%
\pgfpathlineto{\pgfqpoint{-0.048611in}{0.000000in}}%
\pgfusepath{stroke,fill}%
}%
\begin{pgfscope}%
\pgfsys@transformshift{4.616667in}{9.411378in}%
\pgfsys@useobject{currentmarker}{}%
\end{pgfscope}%
\end{pgfscope}%
\begin{pgfscope}%
\definecolor{textcolor}{rgb}{0.000000,0.000000,0.000000}%
\pgfsetstrokecolor{textcolor}%
\pgfsetfillcolor{textcolor}%
\pgftext[x=4.298565in,y=9.358617in,left,base]{\color{textcolor}\sffamily\fontsize{10.000000}{12.000000}\selectfont 0.4}%
\end{pgfscope}%
\begin{pgfscope}%
\pgfsetbuttcap%
\pgfsetroundjoin%
\definecolor{currentfill}{rgb}{0.000000,0.000000,0.000000}%
\pgfsetfillcolor{currentfill}%
\pgfsetlinewidth{0.803000pt}%
\definecolor{currentstroke}{rgb}{0.000000,0.000000,0.000000}%
\pgfsetstrokecolor{currentstroke}%
\pgfsetdash{}{0pt}%
\pgfsys@defobject{currentmarker}{\pgfqpoint{-0.048611in}{0.000000in}}{\pgfqpoint{0.000000in}{0.000000in}}{%
\pgfpathmoveto{\pgfqpoint{0.000000in}{0.000000in}}%
\pgfpathlineto{\pgfqpoint{-0.048611in}{0.000000in}}%
\pgfusepath{stroke,fill}%
}%
\begin{pgfscope}%
\pgfsys@transformshift{4.616667in}{9.805262in}%
\pgfsys@useobject{currentmarker}{}%
\end{pgfscope}%
\end{pgfscope}%
\begin{pgfscope}%
\definecolor{textcolor}{rgb}{0.000000,0.000000,0.000000}%
\pgfsetstrokecolor{textcolor}%
\pgfsetfillcolor{textcolor}%
\pgftext[x=4.298565in,y=9.752500in,left,base]{\color{textcolor}\sffamily\fontsize{10.000000}{12.000000}\selectfont 0.6}%
\end{pgfscope}%
\begin{pgfscope}%
\pgfsetbuttcap%
\pgfsetroundjoin%
\definecolor{currentfill}{rgb}{0.000000,0.000000,0.000000}%
\pgfsetfillcolor{currentfill}%
\pgfsetlinewidth{0.803000pt}%
\definecolor{currentstroke}{rgb}{0.000000,0.000000,0.000000}%
\pgfsetstrokecolor{currentstroke}%
\pgfsetdash{}{0pt}%
\pgfsys@defobject{currentmarker}{\pgfqpoint{-0.048611in}{0.000000in}}{\pgfqpoint{0.000000in}{0.000000in}}{%
\pgfpathmoveto{\pgfqpoint{0.000000in}{0.000000in}}%
\pgfpathlineto{\pgfqpoint{-0.048611in}{0.000000in}}%
\pgfusepath{stroke,fill}%
}%
\begin{pgfscope}%
\pgfsys@transformshift{4.616667in}{10.199146in}%
\pgfsys@useobject{currentmarker}{}%
\end{pgfscope}%
\end{pgfscope}%
\begin{pgfscope}%
\definecolor{textcolor}{rgb}{0.000000,0.000000,0.000000}%
\pgfsetstrokecolor{textcolor}%
\pgfsetfillcolor{textcolor}%
\pgftext[x=4.298565in,y=10.146384in,left,base]{\color{textcolor}\sffamily\fontsize{10.000000}{12.000000}\selectfont 0.8}%
\end{pgfscope}%
\begin{pgfscope}%
\pgfsetbuttcap%
\pgfsetroundjoin%
\definecolor{currentfill}{rgb}{0.000000,0.000000,0.000000}%
\pgfsetfillcolor{currentfill}%
\pgfsetlinewidth{0.803000pt}%
\definecolor{currentstroke}{rgb}{0.000000,0.000000,0.000000}%
\pgfsetstrokecolor{currentstroke}%
\pgfsetdash{}{0pt}%
\pgfsys@defobject{currentmarker}{\pgfqpoint{-0.048611in}{0.000000in}}{\pgfqpoint{0.000000in}{0.000000in}}{%
\pgfpathmoveto{\pgfqpoint{0.000000in}{0.000000in}}%
\pgfpathlineto{\pgfqpoint{-0.048611in}{0.000000in}}%
\pgfusepath{stroke,fill}%
}%
\begin{pgfscope}%
\pgfsys@transformshift{4.616667in}{10.593029in}%
\pgfsys@useobject{currentmarker}{}%
\end{pgfscope}%
\end{pgfscope}%
\begin{pgfscope}%
\definecolor{textcolor}{rgb}{0.000000,0.000000,0.000000}%
\pgfsetstrokecolor{textcolor}%
\pgfsetfillcolor{textcolor}%
\pgftext[x=4.298565in,y=10.540268in,left,base]{\color{textcolor}\sffamily\fontsize{10.000000}{12.000000}\selectfont 1.0}%
\end{pgfscope}%
\begin{pgfscope}%
\pgfsetbuttcap%
\pgfsetroundjoin%
\definecolor{currentfill}{rgb}{0.000000,0.000000,0.000000}%
\pgfsetfillcolor{currentfill}%
\pgfsetlinewidth{0.803000pt}%
\definecolor{currentstroke}{rgb}{0.000000,0.000000,0.000000}%
\pgfsetstrokecolor{currentstroke}%
\pgfsetdash{}{0pt}%
\pgfsys@defobject{currentmarker}{\pgfqpoint{-0.048611in}{0.000000in}}{\pgfqpoint{0.000000in}{0.000000in}}{%
\pgfpathmoveto{\pgfqpoint{0.000000in}{0.000000in}}%
\pgfpathlineto{\pgfqpoint{-0.048611in}{0.000000in}}%
\pgfusepath{stroke,fill}%
}%
\begin{pgfscope}%
\pgfsys@transformshift{4.616667in}{10.986913in}%
\pgfsys@useobject{currentmarker}{}%
\end{pgfscope}%
\end{pgfscope}%
\begin{pgfscope}%
\definecolor{textcolor}{rgb}{0.000000,0.000000,0.000000}%
\pgfsetstrokecolor{textcolor}%
\pgfsetfillcolor{textcolor}%
\pgftext[x=4.298565in,y=10.934151in,left,base]{\color{textcolor}\sffamily\fontsize{10.000000}{12.000000}\selectfont 1.2}%
\end{pgfscope}%
\begin{pgfscope}%
\pgfsetbuttcap%
\pgfsetroundjoin%
\definecolor{currentfill}{rgb}{0.000000,0.000000,0.000000}%
\pgfsetfillcolor{currentfill}%
\pgfsetlinewidth{0.803000pt}%
\definecolor{currentstroke}{rgb}{0.000000,0.000000,0.000000}%
\pgfsetstrokecolor{currentstroke}%
\pgfsetdash{}{0pt}%
\pgfsys@defobject{currentmarker}{\pgfqpoint{-0.048611in}{0.000000in}}{\pgfqpoint{0.000000in}{0.000000in}}{%
\pgfpathmoveto{\pgfqpoint{0.000000in}{0.000000in}}%
\pgfpathlineto{\pgfqpoint{-0.048611in}{0.000000in}}%
\pgfusepath{stroke,fill}%
}%
\begin{pgfscope}%
\pgfsys@transformshift{4.616667in}{11.380797in}%
\pgfsys@useobject{currentmarker}{}%
\end{pgfscope}%
\end{pgfscope}%
\begin{pgfscope}%
\definecolor{textcolor}{rgb}{0.000000,0.000000,0.000000}%
\pgfsetstrokecolor{textcolor}%
\pgfsetfillcolor{textcolor}%
\pgftext[x=4.298565in,y=11.328035in,left,base]{\color{textcolor}\sffamily\fontsize{10.000000}{12.000000}\selectfont 1.4}%
\end{pgfscope}%
\begin{pgfscope}%
\definecolor{textcolor}{rgb}{0.000000,0.000000,0.000000}%
\pgfsetstrokecolor{textcolor}%
\pgfsetfillcolor{textcolor}%
\pgftext[x=4.243009in,y=10.054167in,,bottom,rotate=90.000000]{\color{textcolor}\sffamily\fontsize{10.000000}{12.000000}\selectfont \(\displaystyle U\)}%
\end{pgfscope}%
\begin{pgfscope}%
\pgfpathrectangle{\pgfqpoint{4.616667in}{8.480556in}}{\pgfqpoint{3.184722in}{3.147222in}}%
\pgfusepath{clip}%
\pgfsetrectcap%
\pgfsetroundjoin%
\pgfsetlinewidth{1.505625pt}%
\definecolor{currentstroke}{rgb}{0.121569,0.466667,0.705882}%
\pgfsetstrokecolor{currentstroke}%
\pgfsetdash{}{0pt}%
\pgfpathmoveto{\pgfqpoint{4.761427in}{10.593029in}}%
\pgfpathlineto{\pgfqpoint{6.276813in}{10.593034in}}%
\pgfpathlineto{\pgfqpoint{6.278044in}{10.584894in}}%
\pgfpathlineto{\pgfqpoint{6.277971in}{10.601563in}}%
\pgfpathlineto{\pgfqpoint{6.278116in}{10.588248in}}%
\pgfpathlineto{\pgfqpoint{6.279274in}{10.526296in}}%
\pgfpathlineto{\pgfqpoint{6.279419in}{10.652299in}}%
\pgfpathlineto{\pgfqpoint{6.280722in}{10.035242in}}%
\pgfpathlineto{\pgfqpoint{6.280577in}{11.009961in}}%
\pgfpathlineto{\pgfqpoint{6.280794in}{10.900437in}}%
\pgfpathlineto{\pgfqpoint{6.281084in}{11.484722in}}%
\pgfpathlineto{\pgfqpoint{6.280939in}{9.832789in}}%
\pgfpathlineto{\pgfqpoint{6.281156in}{9.988285in}}%
\pgfpathlineto{\pgfqpoint{6.281301in}{11.447630in}}%
\pgfpathlineto{\pgfqpoint{6.282893in}{8.623611in}}%
\pgfpathlineto{\pgfqpoint{7.656629in}{8.623611in}}%
\pgfpathlineto{\pgfqpoint{7.656629in}{8.623611in}}%
\pgfusepath{stroke}%
\end{pgfscope}%
\begin{pgfscope}%
\pgfpathrectangle{\pgfqpoint{4.616667in}{8.480556in}}{\pgfqpoint{3.184722in}{3.147222in}}%
\pgfusepath{clip}%
\pgfsetrectcap%
\pgfsetroundjoin%
\pgfsetlinewidth{1.505625pt}%
\definecolor{currentstroke}{rgb}{1.000000,0.498039,0.054902}%
\pgfsetstrokecolor{currentstroke}%
\pgfsetdash{}{0pt}%
\pgfpathmoveto{\pgfqpoint{4.761427in}{10.593029in}}%
\pgfpathlineto{\pgfqpoint{6.277031in}{10.593881in}}%
\pgfpathlineto{\pgfqpoint{6.278189in}{10.601167in}}%
\pgfpathlineto{\pgfqpoint{6.278333in}{10.584404in}}%
\pgfpathlineto{\pgfqpoint{6.279636in}{10.660151in}}%
\pgfpathlineto{\pgfqpoint{6.279492in}{10.542990in}}%
\pgfpathlineto{\pgfqpoint{6.279709in}{10.555916in}}%
\pgfpathlineto{\pgfqpoint{6.280360in}{10.777195in}}%
\pgfpathlineto{\pgfqpoint{6.280505in}{10.421709in}}%
\pgfpathlineto{\pgfqpoint{6.281084in}{11.082632in}}%
\pgfpathlineto{\pgfqpoint{6.281663in}{8.623611in}}%
\pgfpathlineto{\pgfqpoint{7.656629in}{8.623611in}}%
\pgfpathlineto{\pgfqpoint{7.656629in}{8.623611in}}%
\pgfusepath{stroke}%
\end{pgfscope}%
\begin{pgfscope}%
\pgfpathrectangle{\pgfqpoint{4.616667in}{8.480556in}}{\pgfqpoint{3.184722in}{3.147222in}}%
\pgfusepath{clip}%
\pgfsetrectcap%
\pgfsetroundjoin%
\pgfsetlinewidth{1.505625pt}%
\definecolor{currentstroke}{rgb}{0.172549,0.627451,0.172549}%
\pgfsetstrokecolor{currentstroke}%
\pgfsetdash{}{0pt}%
\pgfpathmoveto{\pgfqpoint{4.761427in}{10.593029in}}%
\pgfpathlineto{\pgfqpoint{6.281084in}{10.592375in}}%
\pgfpathlineto{\pgfqpoint{6.281229in}{10.573234in}}%
\pgfpathlineto{\pgfqpoint{6.284413in}{8.623611in}}%
\pgfpathlineto{\pgfqpoint{7.656629in}{8.623611in}}%
\pgfpathlineto{\pgfqpoint{7.656629in}{8.623611in}}%
\pgfusepath{stroke}%
\end{pgfscope}%
\begin{pgfscope}%
\pgfpathrectangle{\pgfqpoint{4.616667in}{8.480556in}}{\pgfqpoint{3.184722in}{3.147222in}}%
\pgfusepath{clip}%
\pgfsetrectcap%
\pgfsetroundjoin%
\pgfsetlinewidth{1.505625pt}%
\definecolor{currentstroke}{rgb}{0.839216,0.152941,0.156863}%
\pgfsetstrokecolor{currentstroke}%
\pgfsetdash{}{0pt}%
\pgfpathmoveto{\pgfqpoint{4.761427in}{10.593029in}}%
\pgfpathlineto{\pgfqpoint{6.208992in}{10.593029in}}%
\pgfpathlineto{\pgfqpoint{6.210584in}{8.623611in}}%
\pgfpathlineto{\pgfqpoint{7.656629in}{8.623611in}}%
\pgfpathlineto{\pgfqpoint{7.656629in}{8.623611in}}%
\pgfusepath{stroke}%
\end{pgfscope}%
\begin{pgfscope}%
\pgfpathrectangle{\pgfqpoint{4.616667in}{8.480556in}}{\pgfqpoint{3.184722in}{3.147222in}}%
\pgfusepath{clip}%
\pgfsetrectcap%
\pgfsetroundjoin%
\pgfsetlinewidth{0.501875pt}%
\definecolor{currentstroke}{rgb}{0.000000,0.000000,0.000000}%
\pgfsetstrokecolor{currentstroke}%
\pgfsetdash{}{0pt}%
\pgfpathmoveto{\pgfqpoint{4.761427in}{10.593029in}}%
\pgfpathlineto{\pgfqpoint{6.281373in}{10.593029in}}%
\pgfpathlineto{\pgfqpoint{6.282966in}{8.623611in}}%
\pgfpathlineto{\pgfqpoint{7.656629in}{8.623611in}}%
\pgfpathlineto{\pgfqpoint{7.656629in}{8.623611in}}%
\pgfusepath{stroke}%
\end{pgfscope}%
\begin{pgfscope}%
\pgfsetrectcap%
\pgfsetmiterjoin%
\pgfsetlinewidth{0.803000pt}%
\definecolor{currentstroke}{rgb}{0.000000,0.000000,0.000000}%
\pgfsetstrokecolor{currentstroke}%
\pgfsetdash{}{0pt}%
\pgfpathmoveto{\pgfqpoint{4.616667in}{8.480556in}}%
\pgfpathlineto{\pgfqpoint{4.616667in}{11.627778in}}%
\pgfusepath{stroke}%
\end{pgfscope}%
\begin{pgfscope}%
\pgfsetrectcap%
\pgfsetmiterjoin%
\pgfsetlinewidth{0.803000pt}%
\definecolor{currentstroke}{rgb}{0.000000,0.000000,0.000000}%
\pgfsetstrokecolor{currentstroke}%
\pgfsetdash{}{0pt}%
\pgfpathmoveto{\pgfqpoint{7.801389in}{8.480556in}}%
\pgfpathlineto{\pgfqpoint{7.801389in}{11.627778in}}%
\pgfusepath{stroke}%
\end{pgfscope}%
\begin{pgfscope}%
\pgfsetrectcap%
\pgfsetmiterjoin%
\pgfsetlinewidth{0.803000pt}%
\definecolor{currentstroke}{rgb}{0.000000,0.000000,0.000000}%
\pgfsetstrokecolor{currentstroke}%
\pgfsetdash{}{0pt}%
\pgfpathmoveto{\pgfqpoint{4.616667in}{8.480556in}}%
\pgfpathlineto{\pgfqpoint{7.801389in}{8.480556in}}%
\pgfusepath{stroke}%
\end{pgfscope}%
\begin{pgfscope}%
\pgfsetrectcap%
\pgfsetmiterjoin%
\pgfsetlinewidth{0.803000pt}%
\definecolor{currentstroke}{rgb}{0.000000,0.000000,0.000000}%
\pgfsetstrokecolor{currentstroke}%
\pgfsetdash{}{0pt}%
\pgfpathmoveto{\pgfqpoint{4.616667in}{11.627778in}}%
\pgfpathlineto{\pgfqpoint{7.801389in}{11.627778in}}%
\pgfusepath{stroke}%
\end{pgfscope}%
\begin{pgfscope}%
\definecolor{textcolor}{rgb}{0.000000,0.000000,0.000000}%
\pgfsetstrokecolor{textcolor}%
\pgfsetfillcolor{textcolor}%
\pgftext[x=6.209028in,y=11.711111in,,base]{\color{textcolor}\sffamily\fontsize{12.000000}{14.400000}\selectfont \(\displaystyle  t = 0.2 \)}%
\end{pgfscope}%
\begin{pgfscope}%
\pgfsetbuttcap%
\pgfsetmiterjoin%
\definecolor{currentfill}{rgb}{1.000000,1.000000,1.000000}%
\pgfsetfillcolor{currentfill}%
\pgfsetlinewidth{0.000000pt}%
\definecolor{currentstroke}{rgb}{0.000000,0.000000,0.000000}%
\pgfsetstrokecolor{currentstroke}%
\pgfsetstrokeopacity{0.000000}%
\pgfsetdash{}{0pt}%
\pgfpathmoveto{\pgfqpoint{0.691667in}{4.530556in}}%
\pgfpathlineto{\pgfqpoint{3.876389in}{4.530556in}}%
\pgfpathlineto{\pgfqpoint{3.876389in}{7.677778in}}%
\pgfpathlineto{\pgfqpoint{0.691667in}{7.677778in}}%
\pgfpathclose%
\pgfusepath{fill}%
\end{pgfscope}%
\begin{pgfscope}%
\pgfsetbuttcap%
\pgfsetroundjoin%
\definecolor{currentfill}{rgb}{0.000000,0.000000,0.000000}%
\pgfsetfillcolor{currentfill}%
\pgfsetlinewidth{0.803000pt}%
\definecolor{currentstroke}{rgb}{0.000000,0.000000,0.000000}%
\pgfsetstrokecolor{currentstroke}%
\pgfsetdash{}{0pt}%
\pgfsys@defobject{currentmarker}{\pgfqpoint{0.000000in}{-0.048611in}}{\pgfqpoint{0.000000in}{0.000000in}}{%
\pgfpathmoveto{\pgfqpoint{0.000000in}{0.000000in}}%
\pgfpathlineto{\pgfqpoint{0.000000in}{-0.048611in}}%
\pgfusepath{stroke,fill}%
}%
\begin{pgfscope}%
\pgfsys@transformshift{0.836391in}{4.530556in}%
\pgfsys@useobject{currentmarker}{}%
\end{pgfscope}%
\end{pgfscope}%
\begin{pgfscope}%
\definecolor{textcolor}{rgb}{0.000000,0.000000,0.000000}%
\pgfsetstrokecolor{textcolor}%
\pgfsetfillcolor{textcolor}%
\pgftext[x=0.836391in,y=4.433333in,,top]{\color{textcolor}\sffamily\fontsize{10.000000}{12.000000}\selectfont −2}%
\end{pgfscope}%
\begin{pgfscope}%
\pgfsetbuttcap%
\pgfsetroundjoin%
\definecolor{currentfill}{rgb}{0.000000,0.000000,0.000000}%
\pgfsetfillcolor{currentfill}%
\pgfsetlinewidth{0.803000pt}%
\definecolor{currentstroke}{rgb}{0.000000,0.000000,0.000000}%
\pgfsetstrokecolor{currentstroke}%
\pgfsetdash{}{0pt}%
\pgfsys@defobject{currentmarker}{\pgfqpoint{0.000000in}{-0.048611in}}{\pgfqpoint{0.000000in}{0.000000in}}{%
\pgfpathmoveto{\pgfqpoint{0.000000in}{0.000000in}}%
\pgfpathlineto{\pgfqpoint{0.000000in}{-0.048611in}}%
\pgfusepath{stroke,fill}%
}%
\begin{pgfscope}%
\pgfsys@transformshift{1.560209in}{4.530556in}%
\pgfsys@useobject{currentmarker}{}%
\end{pgfscope}%
\end{pgfscope}%
\begin{pgfscope}%
\definecolor{textcolor}{rgb}{0.000000,0.000000,0.000000}%
\pgfsetstrokecolor{textcolor}%
\pgfsetfillcolor{textcolor}%
\pgftext[x=1.560209in,y=4.433333in,,top]{\color{textcolor}\sffamily\fontsize{10.000000}{12.000000}\selectfont −1}%
\end{pgfscope}%
\begin{pgfscope}%
\pgfsetbuttcap%
\pgfsetroundjoin%
\definecolor{currentfill}{rgb}{0.000000,0.000000,0.000000}%
\pgfsetfillcolor{currentfill}%
\pgfsetlinewidth{0.803000pt}%
\definecolor{currentstroke}{rgb}{0.000000,0.000000,0.000000}%
\pgfsetstrokecolor{currentstroke}%
\pgfsetdash{}{0pt}%
\pgfsys@defobject{currentmarker}{\pgfqpoint{0.000000in}{-0.048611in}}{\pgfqpoint{0.000000in}{0.000000in}}{%
\pgfpathmoveto{\pgfqpoint{0.000000in}{0.000000in}}%
\pgfpathlineto{\pgfqpoint{0.000000in}{-0.048611in}}%
\pgfusepath{stroke,fill}%
}%
\begin{pgfscope}%
\pgfsys@transformshift{2.284028in}{4.530556in}%
\pgfsys@useobject{currentmarker}{}%
\end{pgfscope}%
\end{pgfscope}%
\begin{pgfscope}%
\definecolor{textcolor}{rgb}{0.000000,0.000000,0.000000}%
\pgfsetstrokecolor{textcolor}%
\pgfsetfillcolor{textcolor}%
\pgftext[x=2.284028in,y=4.433333in,,top]{\color{textcolor}\sffamily\fontsize{10.000000}{12.000000}\selectfont 0}%
\end{pgfscope}%
\begin{pgfscope}%
\pgfsetbuttcap%
\pgfsetroundjoin%
\definecolor{currentfill}{rgb}{0.000000,0.000000,0.000000}%
\pgfsetfillcolor{currentfill}%
\pgfsetlinewidth{0.803000pt}%
\definecolor{currentstroke}{rgb}{0.000000,0.000000,0.000000}%
\pgfsetstrokecolor{currentstroke}%
\pgfsetdash{}{0pt}%
\pgfsys@defobject{currentmarker}{\pgfqpoint{0.000000in}{-0.048611in}}{\pgfqpoint{0.000000in}{0.000000in}}{%
\pgfpathmoveto{\pgfqpoint{0.000000in}{0.000000in}}%
\pgfpathlineto{\pgfqpoint{0.000000in}{-0.048611in}}%
\pgfusepath{stroke,fill}%
}%
\begin{pgfscope}%
\pgfsys@transformshift{3.007846in}{4.530556in}%
\pgfsys@useobject{currentmarker}{}%
\end{pgfscope}%
\end{pgfscope}%
\begin{pgfscope}%
\definecolor{textcolor}{rgb}{0.000000,0.000000,0.000000}%
\pgfsetstrokecolor{textcolor}%
\pgfsetfillcolor{textcolor}%
\pgftext[x=3.007846in,y=4.433333in,,top]{\color{textcolor}\sffamily\fontsize{10.000000}{12.000000}\selectfont 1}%
\end{pgfscope}%
\begin{pgfscope}%
\pgfsetbuttcap%
\pgfsetroundjoin%
\definecolor{currentfill}{rgb}{0.000000,0.000000,0.000000}%
\pgfsetfillcolor{currentfill}%
\pgfsetlinewidth{0.803000pt}%
\definecolor{currentstroke}{rgb}{0.000000,0.000000,0.000000}%
\pgfsetstrokecolor{currentstroke}%
\pgfsetdash{}{0pt}%
\pgfsys@defobject{currentmarker}{\pgfqpoint{0.000000in}{-0.048611in}}{\pgfqpoint{0.000000in}{0.000000in}}{%
\pgfpathmoveto{\pgfqpoint{0.000000in}{0.000000in}}%
\pgfpathlineto{\pgfqpoint{0.000000in}{-0.048611in}}%
\pgfusepath{stroke,fill}%
}%
\begin{pgfscope}%
\pgfsys@transformshift{3.731665in}{4.530556in}%
\pgfsys@useobject{currentmarker}{}%
\end{pgfscope}%
\end{pgfscope}%
\begin{pgfscope}%
\definecolor{textcolor}{rgb}{0.000000,0.000000,0.000000}%
\pgfsetstrokecolor{textcolor}%
\pgfsetfillcolor{textcolor}%
\pgftext[x=3.731665in,y=4.433333in,,top]{\color{textcolor}\sffamily\fontsize{10.000000}{12.000000}\selectfont 2}%
\end{pgfscope}%
\begin{pgfscope}%
\definecolor{textcolor}{rgb}{0.000000,0.000000,0.000000}%
\pgfsetstrokecolor{textcolor}%
\pgfsetfillcolor{textcolor}%
\pgftext[x=2.284028in,y=4.243365in,,top]{\color{textcolor}\sffamily\fontsize{10.000000}{12.000000}\selectfont \(\displaystyle x\)}%
\end{pgfscope}%
\begin{pgfscope}%
\pgfsetbuttcap%
\pgfsetroundjoin%
\definecolor{currentfill}{rgb}{0.000000,0.000000,0.000000}%
\pgfsetfillcolor{currentfill}%
\pgfsetlinewidth{0.803000pt}%
\definecolor{currentstroke}{rgb}{0.000000,0.000000,0.000000}%
\pgfsetstrokecolor{currentstroke}%
\pgfsetdash{}{0pt}%
\pgfsys@defobject{currentmarker}{\pgfqpoint{-0.048611in}{0.000000in}}{\pgfqpoint{0.000000in}{0.000000in}}{%
\pgfpathmoveto{\pgfqpoint{0.000000in}{0.000000in}}%
\pgfpathlineto{\pgfqpoint{-0.048611in}{0.000000in}}%
\pgfusepath{stroke,fill}%
}%
\begin{pgfscope}%
\pgfsys@transformshift{0.691667in}{4.673611in}%
\pgfsys@useobject{currentmarker}{}%
\end{pgfscope}%
\end{pgfscope}%
\begin{pgfscope}%
\definecolor{textcolor}{rgb}{0.000000,0.000000,0.000000}%
\pgfsetstrokecolor{textcolor}%
\pgfsetfillcolor{textcolor}%
\pgftext[x=0.373565in,y=4.620850in,left,base]{\color{textcolor}\sffamily\fontsize{10.000000}{12.000000}\selectfont 0.0}%
\end{pgfscope}%
\begin{pgfscope}%
\pgfsetbuttcap%
\pgfsetroundjoin%
\definecolor{currentfill}{rgb}{0.000000,0.000000,0.000000}%
\pgfsetfillcolor{currentfill}%
\pgfsetlinewidth{0.803000pt}%
\definecolor{currentstroke}{rgb}{0.000000,0.000000,0.000000}%
\pgfsetstrokecolor{currentstroke}%
\pgfsetdash{}{0pt}%
\pgfsys@defobject{currentmarker}{\pgfqpoint{-0.048611in}{0.000000in}}{\pgfqpoint{0.000000in}{0.000000in}}{%
\pgfpathmoveto{\pgfqpoint{0.000000in}{0.000000in}}%
\pgfpathlineto{\pgfqpoint{-0.048611in}{0.000000in}}%
\pgfusepath{stroke,fill}%
}%
\begin{pgfscope}%
\pgfsys@transformshift{0.691667in}{5.067495in}%
\pgfsys@useobject{currentmarker}{}%
\end{pgfscope}%
\end{pgfscope}%
\begin{pgfscope}%
\definecolor{textcolor}{rgb}{0.000000,0.000000,0.000000}%
\pgfsetstrokecolor{textcolor}%
\pgfsetfillcolor{textcolor}%
\pgftext[x=0.373565in,y=5.014733in,left,base]{\color{textcolor}\sffamily\fontsize{10.000000}{12.000000}\selectfont 0.2}%
\end{pgfscope}%
\begin{pgfscope}%
\pgfsetbuttcap%
\pgfsetroundjoin%
\definecolor{currentfill}{rgb}{0.000000,0.000000,0.000000}%
\pgfsetfillcolor{currentfill}%
\pgfsetlinewidth{0.803000pt}%
\definecolor{currentstroke}{rgb}{0.000000,0.000000,0.000000}%
\pgfsetstrokecolor{currentstroke}%
\pgfsetdash{}{0pt}%
\pgfsys@defobject{currentmarker}{\pgfqpoint{-0.048611in}{0.000000in}}{\pgfqpoint{0.000000in}{0.000000in}}{%
\pgfpathmoveto{\pgfqpoint{0.000000in}{0.000000in}}%
\pgfpathlineto{\pgfqpoint{-0.048611in}{0.000000in}}%
\pgfusepath{stroke,fill}%
}%
\begin{pgfscope}%
\pgfsys@transformshift{0.691667in}{5.461378in}%
\pgfsys@useobject{currentmarker}{}%
\end{pgfscope}%
\end{pgfscope}%
\begin{pgfscope}%
\definecolor{textcolor}{rgb}{0.000000,0.000000,0.000000}%
\pgfsetstrokecolor{textcolor}%
\pgfsetfillcolor{textcolor}%
\pgftext[x=0.373565in,y=5.408617in,left,base]{\color{textcolor}\sffamily\fontsize{10.000000}{12.000000}\selectfont 0.4}%
\end{pgfscope}%
\begin{pgfscope}%
\pgfsetbuttcap%
\pgfsetroundjoin%
\definecolor{currentfill}{rgb}{0.000000,0.000000,0.000000}%
\pgfsetfillcolor{currentfill}%
\pgfsetlinewidth{0.803000pt}%
\definecolor{currentstroke}{rgb}{0.000000,0.000000,0.000000}%
\pgfsetstrokecolor{currentstroke}%
\pgfsetdash{}{0pt}%
\pgfsys@defobject{currentmarker}{\pgfqpoint{-0.048611in}{0.000000in}}{\pgfqpoint{0.000000in}{0.000000in}}{%
\pgfpathmoveto{\pgfqpoint{0.000000in}{0.000000in}}%
\pgfpathlineto{\pgfqpoint{-0.048611in}{0.000000in}}%
\pgfusepath{stroke,fill}%
}%
\begin{pgfscope}%
\pgfsys@transformshift{0.691667in}{5.855262in}%
\pgfsys@useobject{currentmarker}{}%
\end{pgfscope}%
\end{pgfscope}%
\begin{pgfscope}%
\definecolor{textcolor}{rgb}{0.000000,0.000000,0.000000}%
\pgfsetstrokecolor{textcolor}%
\pgfsetfillcolor{textcolor}%
\pgftext[x=0.373565in,y=5.802500in,left,base]{\color{textcolor}\sffamily\fontsize{10.000000}{12.000000}\selectfont 0.6}%
\end{pgfscope}%
\begin{pgfscope}%
\pgfsetbuttcap%
\pgfsetroundjoin%
\definecolor{currentfill}{rgb}{0.000000,0.000000,0.000000}%
\pgfsetfillcolor{currentfill}%
\pgfsetlinewidth{0.803000pt}%
\definecolor{currentstroke}{rgb}{0.000000,0.000000,0.000000}%
\pgfsetstrokecolor{currentstroke}%
\pgfsetdash{}{0pt}%
\pgfsys@defobject{currentmarker}{\pgfqpoint{-0.048611in}{0.000000in}}{\pgfqpoint{0.000000in}{0.000000in}}{%
\pgfpathmoveto{\pgfqpoint{0.000000in}{0.000000in}}%
\pgfpathlineto{\pgfqpoint{-0.048611in}{0.000000in}}%
\pgfusepath{stroke,fill}%
}%
\begin{pgfscope}%
\pgfsys@transformshift{0.691667in}{6.249146in}%
\pgfsys@useobject{currentmarker}{}%
\end{pgfscope}%
\end{pgfscope}%
\begin{pgfscope}%
\definecolor{textcolor}{rgb}{0.000000,0.000000,0.000000}%
\pgfsetstrokecolor{textcolor}%
\pgfsetfillcolor{textcolor}%
\pgftext[x=0.373565in,y=6.196384in,left,base]{\color{textcolor}\sffamily\fontsize{10.000000}{12.000000}\selectfont 0.8}%
\end{pgfscope}%
\begin{pgfscope}%
\pgfsetbuttcap%
\pgfsetroundjoin%
\definecolor{currentfill}{rgb}{0.000000,0.000000,0.000000}%
\pgfsetfillcolor{currentfill}%
\pgfsetlinewidth{0.803000pt}%
\definecolor{currentstroke}{rgb}{0.000000,0.000000,0.000000}%
\pgfsetstrokecolor{currentstroke}%
\pgfsetdash{}{0pt}%
\pgfsys@defobject{currentmarker}{\pgfqpoint{-0.048611in}{0.000000in}}{\pgfqpoint{0.000000in}{0.000000in}}{%
\pgfpathmoveto{\pgfqpoint{0.000000in}{0.000000in}}%
\pgfpathlineto{\pgfqpoint{-0.048611in}{0.000000in}}%
\pgfusepath{stroke,fill}%
}%
\begin{pgfscope}%
\pgfsys@transformshift{0.691667in}{6.643029in}%
\pgfsys@useobject{currentmarker}{}%
\end{pgfscope}%
\end{pgfscope}%
\begin{pgfscope}%
\definecolor{textcolor}{rgb}{0.000000,0.000000,0.000000}%
\pgfsetstrokecolor{textcolor}%
\pgfsetfillcolor{textcolor}%
\pgftext[x=0.373565in,y=6.590268in,left,base]{\color{textcolor}\sffamily\fontsize{10.000000}{12.000000}\selectfont 1.0}%
\end{pgfscope}%
\begin{pgfscope}%
\pgfsetbuttcap%
\pgfsetroundjoin%
\definecolor{currentfill}{rgb}{0.000000,0.000000,0.000000}%
\pgfsetfillcolor{currentfill}%
\pgfsetlinewidth{0.803000pt}%
\definecolor{currentstroke}{rgb}{0.000000,0.000000,0.000000}%
\pgfsetstrokecolor{currentstroke}%
\pgfsetdash{}{0pt}%
\pgfsys@defobject{currentmarker}{\pgfqpoint{-0.048611in}{0.000000in}}{\pgfqpoint{0.000000in}{0.000000in}}{%
\pgfpathmoveto{\pgfqpoint{0.000000in}{0.000000in}}%
\pgfpathlineto{\pgfqpoint{-0.048611in}{0.000000in}}%
\pgfusepath{stroke,fill}%
}%
\begin{pgfscope}%
\pgfsys@transformshift{0.691667in}{7.036913in}%
\pgfsys@useobject{currentmarker}{}%
\end{pgfscope}%
\end{pgfscope}%
\begin{pgfscope}%
\definecolor{textcolor}{rgb}{0.000000,0.000000,0.000000}%
\pgfsetstrokecolor{textcolor}%
\pgfsetfillcolor{textcolor}%
\pgftext[x=0.373565in,y=6.984151in,left,base]{\color{textcolor}\sffamily\fontsize{10.000000}{12.000000}\selectfont 1.2}%
\end{pgfscope}%
\begin{pgfscope}%
\pgfsetbuttcap%
\pgfsetroundjoin%
\definecolor{currentfill}{rgb}{0.000000,0.000000,0.000000}%
\pgfsetfillcolor{currentfill}%
\pgfsetlinewidth{0.803000pt}%
\definecolor{currentstroke}{rgb}{0.000000,0.000000,0.000000}%
\pgfsetstrokecolor{currentstroke}%
\pgfsetdash{}{0pt}%
\pgfsys@defobject{currentmarker}{\pgfqpoint{-0.048611in}{0.000000in}}{\pgfqpoint{0.000000in}{0.000000in}}{%
\pgfpathmoveto{\pgfqpoint{0.000000in}{0.000000in}}%
\pgfpathlineto{\pgfqpoint{-0.048611in}{0.000000in}}%
\pgfusepath{stroke,fill}%
}%
\begin{pgfscope}%
\pgfsys@transformshift{0.691667in}{7.430797in}%
\pgfsys@useobject{currentmarker}{}%
\end{pgfscope}%
\end{pgfscope}%
\begin{pgfscope}%
\definecolor{textcolor}{rgb}{0.000000,0.000000,0.000000}%
\pgfsetstrokecolor{textcolor}%
\pgfsetfillcolor{textcolor}%
\pgftext[x=0.373565in,y=7.378035in,left,base]{\color{textcolor}\sffamily\fontsize{10.000000}{12.000000}\selectfont 1.4}%
\end{pgfscope}%
\begin{pgfscope}%
\definecolor{textcolor}{rgb}{0.000000,0.000000,0.000000}%
\pgfsetstrokecolor{textcolor}%
\pgfsetfillcolor{textcolor}%
\pgftext[x=0.318009in,y=6.104167in,,bottom,rotate=90.000000]{\color{textcolor}\sffamily\fontsize{10.000000}{12.000000}\selectfont \(\displaystyle U\)}%
\end{pgfscope}%
\begin{pgfscope}%
\pgfpathrectangle{\pgfqpoint{0.691667in}{4.530556in}}{\pgfqpoint{3.184722in}{3.147222in}}%
\pgfusepath{clip}%
\pgfsetrectcap%
\pgfsetroundjoin%
\pgfsetlinewidth{1.505625pt}%
\definecolor{currentstroke}{rgb}{0.121569,0.466667,0.705882}%
\pgfsetstrokecolor{currentstroke}%
\pgfsetdash{}{0pt}%
\pgfpathmoveto{\pgfqpoint{0.836427in}{6.643029in}}%
\pgfpathlineto{\pgfqpoint{2.424195in}{6.643034in}}%
\pgfpathlineto{\pgfqpoint{2.425426in}{6.634894in}}%
\pgfpathlineto{\pgfqpoint{2.425353in}{6.651563in}}%
\pgfpathlineto{\pgfqpoint{2.425498in}{6.638248in}}%
\pgfpathlineto{\pgfqpoint{2.426656in}{6.576296in}}%
\pgfpathlineto{\pgfqpoint{2.426801in}{6.702299in}}%
\pgfpathlineto{\pgfqpoint{2.428104in}{6.085242in}}%
\pgfpathlineto{\pgfqpoint{2.427959in}{7.059961in}}%
\pgfpathlineto{\pgfqpoint{2.428176in}{6.950437in}}%
\pgfpathlineto{\pgfqpoint{2.428466in}{7.534722in}}%
\pgfpathlineto{\pgfqpoint{2.428321in}{5.882789in}}%
\pgfpathlineto{\pgfqpoint{2.428538in}{6.038285in}}%
\pgfpathlineto{\pgfqpoint{2.428683in}{7.497630in}}%
\pgfpathlineto{\pgfqpoint{2.430275in}{4.673611in}}%
\pgfpathlineto{\pgfqpoint{3.731629in}{4.673611in}}%
\pgfpathlineto{\pgfqpoint{3.731629in}{4.673611in}}%
\pgfusepath{stroke}%
\end{pgfscope}%
\begin{pgfscope}%
\pgfpathrectangle{\pgfqpoint{0.691667in}{4.530556in}}{\pgfqpoint{3.184722in}{3.147222in}}%
\pgfusepath{clip}%
\pgfsetrectcap%
\pgfsetroundjoin%
\pgfsetlinewidth{1.505625pt}%
\definecolor{currentstroke}{rgb}{1.000000,0.498039,0.054902}%
\pgfsetstrokecolor{currentstroke}%
\pgfsetdash{}{0pt}%
\pgfpathmoveto{\pgfqpoint{0.836427in}{6.643029in}}%
\pgfpathlineto{\pgfqpoint{2.424412in}{6.643881in}}%
\pgfpathlineto{\pgfqpoint{2.425571in}{6.651167in}}%
\pgfpathlineto{\pgfqpoint{2.425715in}{6.634404in}}%
\pgfpathlineto{\pgfqpoint{2.427018in}{6.710151in}}%
\pgfpathlineto{\pgfqpoint{2.426873in}{6.592990in}}%
\pgfpathlineto{\pgfqpoint{2.427091in}{6.605916in}}%
\pgfpathlineto{\pgfqpoint{2.427742in}{6.827195in}}%
\pgfpathlineto{\pgfqpoint{2.427887in}{6.471709in}}%
\pgfpathlineto{\pgfqpoint{2.428466in}{7.132632in}}%
\pgfpathlineto{\pgfqpoint{2.429045in}{4.673611in}}%
\pgfpathlineto{\pgfqpoint{3.731629in}{4.673611in}}%
\pgfpathlineto{\pgfqpoint{3.731629in}{4.673611in}}%
\pgfusepath{stroke}%
\end{pgfscope}%
\begin{pgfscope}%
\pgfpathrectangle{\pgfqpoint{0.691667in}{4.530556in}}{\pgfqpoint{3.184722in}{3.147222in}}%
\pgfusepath{clip}%
\pgfsetrectcap%
\pgfsetroundjoin%
\pgfsetlinewidth{1.505625pt}%
\definecolor{currentstroke}{rgb}{0.172549,0.627451,0.172549}%
\pgfsetstrokecolor{currentstroke}%
\pgfsetdash{}{0pt}%
\pgfpathmoveto{\pgfqpoint{0.836427in}{6.643029in}}%
\pgfpathlineto{\pgfqpoint{2.428466in}{6.642375in}}%
\pgfpathlineto{\pgfqpoint{2.428611in}{6.623234in}}%
\pgfpathlineto{\pgfqpoint{2.431795in}{4.673611in}}%
\pgfpathlineto{\pgfqpoint{3.731629in}{4.673611in}}%
\pgfpathlineto{\pgfqpoint{3.731629in}{4.673611in}}%
\pgfusepath{stroke}%
\end{pgfscope}%
\begin{pgfscope}%
\pgfpathrectangle{\pgfqpoint{0.691667in}{4.530556in}}{\pgfqpoint{3.184722in}{3.147222in}}%
\pgfusepath{clip}%
\pgfsetrectcap%
\pgfsetroundjoin%
\pgfsetlinewidth{1.505625pt}%
\definecolor{currentstroke}{rgb}{0.839216,0.152941,0.156863}%
\pgfsetstrokecolor{currentstroke}%
\pgfsetdash{}{0pt}%
\pgfpathmoveto{\pgfqpoint{0.836427in}{6.643029in}}%
\pgfpathlineto{\pgfqpoint{2.283992in}{6.643029in}}%
\pgfpathlineto{\pgfqpoint{2.285584in}{4.673611in}}%
\pgfpathlineto{\pgfqpoint{3.731629in}{4.673611in}}%
\pgfpathlineto{\pgfqpoint{3.731629in}{4.673611in}}%
\pgfusepath{stroke}%
\end{pgfscope}%
\begin{pgfscope}%
\pgfpathrectangle{\pgfqpoint{0.691667in}{4.530556in}}{\pgfqpoint{3.184722in}{3.147222in}}%
\pgfusepath{clip}%
\pgfsetrectcap%
\pgfsetroundjoin%
\pgfsetlinewidth{0.501875pt}%
\definecolor{currentstroke}{rgb}{0.000000,0.000000,0.000000}%
\pgfsetstrokecolor{currentstroke}%
\pgfsetdash{}{0pt}%
\pgfpathmoveto{\pgfqpoint{0.836427in}{6.643029in}}%
\pgfpathlineto{\pgfqpoint{2.428755in}{6.643029in}}%
\pgfpathlineto{\pgfqpoint{2.430348in}{4.673611in}}%
\pgfpathlineto{\pgfqpoint{3.731629in}{4.673611in}}%
\pgfpathlineto{\pgfqpoint{3.731629in}{4.673611in}}%
\pgfusepath{stroke}%
\end{pgfscope}%
\begin{pgfscope}%
\pgfsetrectcap%
\pgfsetmiterjoin%
\pgfsetlinewidth{0.803000pt}%
\definecolor{currentstroke}{rgb}{0.000000,0.000000,0.000000}%
\pgfsetstrokecolor{currentstroke}%
\pgfsetdash{}{0pt}%
\pgfpathmoveto{\pgfqpoint{0.691667in}{4.530556in}}%
\pgfpathlineto{\pgfqpoint{0.691667in}{7.677778in}}%
\pgfusepath{stroke}%
\end{pgfscope}%
\begin{pgfscope}%
\pgfsetrectcap%
\pgfsetmiterjoin%
\pgfsetlinewidth{0.803000pt}%
\definecolor{currentstroke}{rgb}{0.000000,0.000000,0.000000}%
\pgfsetstrokecolor{currentstroke}%
\pgfsetdash{}{0pt}%
\pgfpathmoveto{\pgfqpoint{3.876389in}{4.530556in}}%
\pgfpathlineto{\pgfqpoint{3.876389in}{7.677778in}}%
\pgfusepath{stroke}%
\end{pgfscope}%
\begin{pgfscope}%
\pgfsetrectcap%
\pgfsetmiterjoin%
\pgfsetlinewidth{0.803000pt}%
\definecolor{currentstroke}{rgb}{0.000000,0.000000,0.000000}%
\pgfsetstrokecolor{currentstroke}%
\pgfsetdash{}{0pt}%
\pgfpathmoveto{\pgfqpoint{0.691667in}{4.530556in}}%
\pgfpathlineto{\pgfqpoint{3.876389in}{4.530556in}}%
\pgfusepath{stroke}%
\end{pgfscope}%
\begin{pgfscope}%
\pgfsetrectcap%
\pgfsetmiterjoin%
\pgfsetlinewidth{0.803000pt}%
\definecolor{currentstroke}{rgb}{0.000000,0.000000,0.000000}%
\pgfsetstrokecolor{currentstroke}%
\pgfsetdash{}{0pt}%
\pgfpathmoveto{\pgfqpoint{0.691667in}{7.677778in}}%
\pgfpathlineto{\pgfqpoint{3.876389in}{7.677778in}}%
\pgfusepath{stroke}%
\end{pgfscope}%
\begin{pgfscope}%
\definecolor{textcolor}{rgb}{0.000000,0.000000,0.000000}%
\pgfsetstrokecolor{textcolor}%
\pgfsetfillcolor{textcolor}%
\pgftext[x=2.284028in,y=7.761111in,,base]{\color{textcolor}\sffamily\fontsize{12.000000}{14.400000}\selectfont \(\displaystyle  t = 0.4 \)}%
\end{pgfscope}%
\begin{pgfscope}%
\pgfsetbuttcap%
\pgfsetmiterjoin%
\definecolor{currentfill}{rgb}{1.000000,1.000000,1.000000}%
\pgfsetfillcolor{currentfill}%
\pgfsetlinewidth{0.000000pt}%
\definecolor{currentstroke}{rgb}{0.000000,0.000000,0.000000}%
\pgfsetstrokecolor{currentstroke}%
\pgfsetstrokeopacity{0.000000}%
\pgfsetdash{}{0pt}%
\pgfpathmoveto{\pgfqpoint{4.616667in}{4.530556in}}%
\pgfpathlineto{\pgfqpoint{7.801389in}{4.530556in}}%
\pgfpathlineto{\pgfqpoint{7.801389in}{7.677778in}}%
\pgfpathlineto{\pgfqpoint{4.616667in}{7.677778in}}%
\pgfpathclose%
\pgfusepath{fill}%
\end{pgfscope}%
\begin{pgfscope}%
\pgfsetbuttcap%
\pgfsetroundjoin%
\definecolor{currentfill}{rgb}{0.000000,0.000000,0.000000}%
\pgfsetfillcolor{currentfill}%
\pgfsetlinewidth{0.803000pt}%
\definecolor{currentstroke}{rgb}{0.000000,0.000000,0.000000}%
\pgfsetstrokecolor{currentstroke}%
\pgfsetdash{}{0pt}%
\pgfsys@defobject{currentmarker}{\pgfqpoint{0.000000in}{-0.048611in}}{\pgfqpoint{0.000000in}{0.000000in}}{%
\pgfpathmoveto{\pgfqpoint{0.000000in}{0.000000in}}%
\pgfpathlineto{\pgfqpoint{0.000000in}{-0.048611in}}%
\pgfusepath{stroke,fill}%
}%
\begin{pgfscope}%
\pgfsys@transformshift{4.761391in}{4.530556in}%
\pgfsys@useobject{currentmarker}{}%
\end{pgfscope}%
\end{pgfscope}%
\begin{pgfscope}%
\definecolor{textcolor}{rgb}{0.000000,0.000000,0.000000}%
\pgfsetstrokecolor{textcolor}%
\pgfsetfillcolor{textcolor}%
\pgftext[x=4.761391in,y=4.433333in,,top]{\color{textcolor}\sffamily\fontsize{10.000000}{12.000000}\selectfont −2}%
\end{pgfscope}%
\begin{pgfscope}%
\pgfsetbuttcap%
\pgfsetroundjoin%
\definecolor{currentfill}{rgb}{0.000000,0.000000,0.000000}%
\pgfsetfillcolor{currentfill}%
\pgfsetlinewidth{0.803000pt}%
\definecolor{currentstroke}{rgb}{0.000000,0.000000,0.000000}%
\pgfsetstrokecolor{currentstroke}%
\pgfsetdash{}{0pt}%
\pgfsys@defobject{currentmarker}{\pgfqpoint{0.000000in}{-0.048611in}}{\pgfqpoint{0.000000in}{0.000000in}}{%
\pgfpathmoveto{\pgfqpoint{0.000000in}{0.000000in}}%
\pgfpathlineto{\pgfqpoint{0.000000in}{-0.048611in}}%
\pgfusepath{stroke,fill}%
}%
\begin{pgfscope}%
\pgfsys@transformshift{5.485209in}{4.530556in}%
\pgfsys@useobject{currentmarker}{}%
\end{pgfscope}%
\end{pgfscope}%
\begin{pgfscope}%
\definecolor{textcolor}{rgb}{0.000000,0.000000,0.000000}%
\pgfsetstrokecolor{textcolor}%
\pgfsetfillcolor{textcolor}%
\pgftext[x=5.485209in,y=4.433333in,,top]{\color{textcolor}\sffamily\fontsize{10.000000}{12.000000}\selectfont −1}%
\end{pgfscope}%
\begin{pgfscope}%
\pgfsetbuttcap%
\pgfsetroundjoin%
\definecolor{currentfill}{rgb}{0.000000,0.000000,0.000000}%
\pgfsetfillcolor{currentfill}%
\pgfsetlinewidth{0.803000pt}%
\definecolor{currentstroke}{rgb}{0.000000,0.000000,0.000000}%
\pgfsetstrokecolor{currentstroke}%
\pgfsetdash{}{0pt}%
\pgfsys@defobject{currentmarker}{\pgfqpoint{0.000000in}{-0.048611in}}{\pgfqpoint{0.000000in}{0.000000in}}{%
\pgfpathmoveto{\pgfqpoint{0.000000in}{0.000000in}}%
\pgfpathlineto{\pgfqpoint{0.000000in}{-0.048611in}}%
\pgfusepath{stroke,fill}%
}%
\begin{pgfscope}%
\pgfsys@transformshift{6.209028in}{4.530556in}%
\pgfsys@useobject{currentmarker}{}%
\end{pgfscope}%
\end{pgfscope}%
\begin{pgfscope}%
\definecolor{textcolor}{rgb}{0.000000,0.000000,0.000000}%
\pgfsetstrokecolor{textcolor}%
\pgfsetfillcolor{textcolor}%
\pgftext[x=6.209028in,y=4.433333in,,top]{\color{textcolor}\sffamily\fontsize{10.000000}{12.000000}\selectfont 0}%
\end{pgfscope}%
\begin{pgfscope}%
\pgfsetbuttcap%
\pgfsetroundjoin%
\definecolor{currentfill}{rgb}{0.000000,0.000000,0.000000}%
\pgfsetfillcolor{currentfill}%
\pgfsetlinewidth{0.803000pt}%
\definecolor{currentstroke}{rgb}{0.000000,0.000000,0.000000}%
\pgfsetstrokecolor{currentstroke}%
\pgfsetdash{}{0pt}%
\pgfsys@defobject{currentmarker}{\pgfqpoint{0.000000in}{-0.048611in}}{\pgfqpoint{0.000000in}{0.000000in}}{%
\pgfpathmoveto{\pgfqpoint{0.000000in}{0.000000in}}%
\pgfpathlineto{\pgfqpoint{0.000000in}{-0.048611in}}%
\pgfusepath{stroke,fill}%
}%
\begin{pgfscope}%
\pgfsys@transformshift{6.932846in}{4.530556in}%
\pgfsys@useobject{currentmarker}{}%
\end{pgfscope}%
\end{pgfscope}%
\begin{pgfscope}%
\definecolor{textcolor}{rgb}{0.000000,0.000000,0.000000}%
\pgfsetstrokecolor{textcolor}%
\pgfsetfillcolor{textcolor}%
\pgftext[x=6.932846in,y=4.433333in,,top]{\color{textcolor}\sffamily\fontsize{10.000000}{12.000000}\selectfont 1}%
\end{pgfscope}%
\begin{pgfscope}%
\pgfsetbuttcap%
\pgfsetroundjoin%
\definecolor{currentfill}{rgb}{0.000000,0.000000,0.000000}%
\pgfsetfillcolor{currentfill}%
\pgfsetlinewidth{0.803000pt}%
\definecolor{currentstroke}{rgb}{0.000000,0.000000,0.000000}%
\pgfsetstrokecolor{currentstroke}%
\pgfsetdash{}{0pt}%
\pgfsys@defobject{currentmarker}{\pgfqpoint{0.000000in}{-0.048611in}}{\pgfqpoint{0.000000in}{0.000000in}}{%
\pgfpathmoveto{\pgfqpoint{0.000000in}{0.000000in}}%
\pgfpathlineto{\pgfqpoint{0.000000in}{-0.048611in}}%
\pgfusepath{stroke,fill}%
}%
\begin{pgfscope}%
\pgfsys@transformshift{7.656665in}{4.530556in}%
\pgfsys@useobject{currentmarker}{}%
\end{pgfscope}%
\end{pgfscope}%
\begin{pgfscope}%
\definecolor{textcolor}{rgb}{0.000000,0.000000,0.000000}%
\pgfsetstrokecolor{textcolor}%
\pgfsetfillcolor{textcolor}%
\pgftext[x=7.656665in,y=4.433333in,,top]{\color{textcolor}\sffamily\fontsize{10.000000}{12.000000}\selectfont 2}%
\end{pgfscope}%
\begin{pgfscope}%
\definecolor{textcolor}{rgb}{0.000000,0.000000,0.000000}%
\pgfsetstrokecolor{textcolor}%
\pgfsetfillcolor{textcolor}%
\pgftext[x=6.209028in,y=4.243365in,,top]{\color{textcolor}\sffamily\fontsize{10.000000}{12.000000}\selectfont \(\displaystyle x\)}%
\end{pgfscope}%
\begin{pgfscope}%
\pgfsetbuttcap%
\pgfsetroundjoin%
\definecolor{currentfill}{rgb}{0.000000,0.000000,0.000000}%
\pgfsetfillcolor{currentfill}%
\pgfsetlinewidth{0.803000pt}%
\definecolor{currentstroke}{rgb}{0.000000,0.000000,0.000000}%
\pgfsetstrokecolor{currentstroke}%
\pgfsetdash{}{0pt}%
\pgfsys@defobject{currentmarker}{\pgfqpoint{-0.048611in}{0.000000in}}{\pgfqpoint{0.000000in}{0.000000in}}{%
\pgfpathmoveto{\pgfqpoint{0.000000in}{0.000000in}}%
\pgfpathlineto{\pgfqpoint{-0.048611in}{0.000000in}}%
\pgfusepath{stroke,fill}%
}%
\begin{pgfscope}%
\pgfsys@transformshift{4.616667in}{4.673611in}%
\pgfsys@useobject{currentmarker}{}%
\end{pgfscope}%
\end{pgfscope}%
\begin{pgfscope}%
\definecolor{textcolor}{rgb}{0.000000,0.000000,0.000000}%
\pgfsetstrokecolor{textcolor}%
\pgfsetfillcolor{textcolor}%
\pgftext[x=4.298565in,y=4.620850in,left,base]{\color{textcolor}\sffamily\fontsize{10.000000}{12.000000}\selectfont 0.0}%
\end{pgfscope}%
\begin{pgfscope}%
\pgfsetbuttcap%
\pgfsetroundjoin%
\definecolor{currentfill}{rgb}{0.000000,0.000000,0.000000}%
\pgfsetfillcolor{currentfill}%
\pgfsetlinewidth{0.803000pt}%
\definecolor{currentstroke}{rgb}{0.000000,0.000000,0.000000}%
\pgfsetstrokecolor{currentstroke}%
\pgfsetdash{}{0pt}%
\pgfsys@defobject{currentmarker}{\pgfqpoint{-0.048611in}{0.000000in}}{\pgfqpoint{0.000000in}{0.000000in}}{%
\pgfpathmoveto{\pgfqpoint{0.000000in}{0.000000in}}%
\pgfpathlineto{\pgfqpoint{-0.048611in}{0.000000in}}%
\pgfusepath{stroke,fill}%
}%
\begin{pgfscope}%
\pgfsys@transformshift{4.616667in}{5.067495in}%
\pgfsys@useobject{currentmarker}{}%
\end{pgfscope}%
\end{pgfscope}%
\begin{pgfscope}%
\definecolor{textcolor}{rgb}{0.000000,0.000000,0.000000}%
\pgfsetstrokecolor{textcolor}%
\pgfsetfillcolor{textcolor}%
\pgftext[x=4.298565in,y=5.014733in,left,base]{\color{textcolor}\sffamily\fontsize{10.000000}{12.000000}\selectfont 0.2}%
\end{pgfscope}%
\begin{pgfscope}%
\pgfsetbuttcap%
\pgfsetroundjoin%
\definecolor{currentfill}{rgb}{0.000000,0.000000,0.000000}%
\pgfsetfillcolor{currentfill}%
\pgfsetlinewidth{0.803000pt}%
\definecolor{currentstroke}{rgb}{0.000000,0.000000,0.000000}%
\pgfsetstrokecolor{currentstroke}%
\pgfsetdash{}{0pt}%
\pgfsys@defobject{currentmarker}{\pgfqpoint{-0.048611in}{0.000000in}}{\pgfqpoint{0.000000in}{0.000000in}}{%
\pgfpathmoveto{\pgfqpoint{0.000000in}{0.000000in}}%
\pgfpathlineto{\pgfqpoint{-0.048611in}{0.000000in}}%
\pgfusepath{stroke,fill}%
}%
\begin{pgfscope}%
\pgfsys@transformshift{4.616667in}{5.461378in}%
\pgfsys@useobject{currentmarker}{}%
\end{pgfscope}%
\end{pgfscope}%
\begin{pgfscope}%
\definecolor{textcolor}{rgb}{0.000000,0.000000,0.000000}%
\pgfsetstrokecolor{textcolor}%
\pgfsetfillcolor{textcolor}%
\pgftext[x=4.298565in,y=5.408617in,left,base]{\color{textcolor}\sffamily\fontsize{10.000000}{12.000000}\selectfont 0.4}%
\end{pgfscope}%
\begin{pgfscope}%
\pgfsetbuttcap%
\pgfsetroundjoin%
\definecolor{currentfill}{rgb}{0.000000,0.000000,0.000000}%
\pgfsetfillcolor{currentfill}%
\pgfsetlinewidth{0.803000pt}%
\definecolor{currentstroke}{rgb}{0.000000,0.000000,0.000000}%
\pgfsetstrokecolor{currentstroke}%
\pgfsetdash{}{0pt}%
\pgfsys@defobject{currentmarker}{\pgfqpoint{-0.048611in}{0.000000in}}{\pgfqpoint{0.000000in}{0.000000in}}{%
\pgfpathmoveto{\pgfqpoint{0.000000in}{0.000000in}}%
\pgfpathlineto{\pgfqpoint{-0.048611in}{0.000000in}}%
\pgfusepath{stroke,fill}%
}%
\begin{pgfscope}%
\pgfsys@transformshift{4.616667in}{5.855262in}%
\pgfsys@useobject{currentmarker}{}%
\end{pgfscope}%
\end{pgfscope}%
\begin{pgfscope}%
\definecolor{textcolor}{rgb}{0.000000,0.000000,0.000000}%
\pgfsetstrokecolor{textcolor}%
\pgfsetfillcolor{textcolor}%
\pgftext[x=4.298565in,y=5.802500in,left,base]{\color{textcolor}\sffamily\fontsize{10.000000}{12.000000}\selectfont 0.6}%
\end{pgfscope}%
\begin{pgfscope}%
\pgfsetbuttcap%
\pgfsetroundjoin%
\definecolor{currentfill}{rgb}{0.000000,0.000000,0.000000}%
\pgfsetfillcolor{currentfill}%
\pgfsetlinewidth{0.803000pt}%
\definecolor{currentstroke}{rgb}{0.000000,0.000000,0.000000}%
\pgfsetstrokecolor{currentstroke}%
\pgfsetdash{}{0pt}%
\pgfsys@defobject{currentmarker}{\pgfqpoint{-0.048611in}{0.000000in}}{\pgfqpoint{0.000000in}{0.000000in}}{%
\pgfpathmoveto{\pgfqpoint{0.000000in}{0.000000in}}%
\pgfpathlineto{\pgfqpoint{-0.048611in}{0.000000in}}%
\pgfusepath{stroke,fill}%
}%
\begin{pgfscope}%
\pgfsys@transformshift{4.616667in}{6.249146in}%
\pgfsys@useobject{currentmarker}{}%
\end{pgfscope}%
\end{pgfscope}%
\begin{pgfscope}%
\definecolor{textcolor}{rgb}{0.000000,0.000000,0.000000}%
\pgfsetstrokecolor{textcolor}%
\pgfsetfillcolor{textcolor}%
\pgftext[x=4.298565in,y=6.196384in,left,base]{\color{textcolor}\sffamily\fontsize{10.000000}{12.000000}\selectfont 0.8}%
\end{pgfscope}%
\begin{pgfscope}%
\pgfsetbuttcap%
\pgfsetroundjoin%
\definecolor{currentfill}{rgb}{0.000000,0.000000,0.000000}%
\pgfsetfillcolor{currentfill}%
\pgfsetlinewidth{0.803000pt}%
\definecolor{currentstroke}{rgb}{0.000000,0.000000,0.000000}%
\pgfsetstrokecolor{currentstroke}%
\pgfsetdash{}{0pt}%
\pgfsys@defobject{currentmarker}{\pgfqpoint{-0.048611in}{0.000000in}}{\pgfqpoint{0.000000in}{0.000000in}}{%
\pgfpathmoveto{\pgfqpoint{0.000000in}{0.000000in}}%
\pgfpathlineto{\pgfqpoint{-0.048611in}{0.000000in}}%
\pgfusepath{stroke,fill}%
}%
\begin{pgfscope}%
\pgfsys@transformshift{4.616667in}{6.643029in}%
\pgfsys@useobject{currentmarker}{}%
\end{pgfscope}%
\end{pgfscope}%
\begin{pgfscope}%
\definecolor{textcolor}{rgb}{0.000000,0.000000,0.000000}%
\pgfsetstrokecolor{textcolor}%
\pgfsetfillcolor{textcolor}%
\pgftext[x=4.298565in,y=6.590268in,left,base]{\color{textcolor}\sffamily\fontsize{10.000000}{12.000000}\selectfont 1.0}%
\end{pgfscope}%
\begin{pgfscope}%
\pgfsetbuttcap%
\pgfsetroundjoin%
\definecolor{currentfill}{rgb}{0.000000,0.000000,0.000000}%
\pgfsetfillcolor{currentfill}%
\pgfsetlinewidth{0.803000pt}%
\definecolor{currentstroke}{rgb}{0.000000,0.000000,0.000000}%
\pgfsetstrokecolor{currentstroke}%
\pgfsetdash{}{0pt}%
\pgfsys@defobject{currentmarker}{\pgfqpoint{-0.048611in}{0.000000in}}{\pgfqpoint{0.000000in}{0.000000in}}{%
\pgfpathmoveto{\pgfqpoint{0.000000in}{0.000000in}}%
\pgfpathlineto{\pgfqpoint{-0.048611in}{0.000000in}}%
\pgfusepath{stroke,fill}%
}%
\begin{pgfscope}%
\pgfsys@transformshift{4.616667in}{7.036913in}%
\pgfsys@useobject{currentmarker}{}%
\end{pgfscope}%
\end{pgfscope}%
\begin{pgfscope}%
\definecolor{textcolor}{rgb}{0.000000,0.000000,0.000000}%
\pgfsetstrokecolor{textcolor}%
\pgfsetfillcolor{textcolor}%
\pgftext[x=4.298565in,y=6.984151in,left,base]{\color{textcolor}\sffamily\fontsize{10.000000}{12.000000}\selectfont 1.2}%
\end{pgfscope}%
\begin{pgfscope}%
\pgfsetbuttcap%
\pgfsetroundjoin%
\definecolor{currentfill}{rgb}{0.000000,0.000000,0.000000}%
\pgfsetfillcolor{currentfill}%
\pgfsetlinewidth{0.803000pt}%
\definecolor{currentstroke}{rgb}{0.000000,0.000000,0.000000}%
\pgfsetstrokecolor{currentstroke}%
\pgfsetdash{}{0pt}%
\pgfsys@defobject{currentmarker}{\pgfqpoint{-0.048611in}{0.000000in}}{\pgfqpoint{0.000000in}{0.000000in}}{%
\pgfpathmoveto{\pgfqpoint{0.000000in}{0.000000in}}%
\pgfpathlineto{\pgfqpoint{-0.048611in}{0.000000in}}%
\pgfusepath{stroke,fill}%
}%
\begin{pgfscope}%
\pgfsys@transformshift{4.616667in}{7.430797in}%
\pgfsys@useobject{currentmarker}{}%
\end{pgfscope}%
\end{pgfscope}%
\begin{pgfscope}%
\definecolor{textcolor}{rgb}{0.000000,0.000000,0.000000}%
\pgfsetstrokecolor{textcolor}%
\pgfsetfillcolor{textcolor}%
\pgftext[x=4.298565in,y=7.378035in,left,base]{\color{textcolor}\sffamily\fontsize{10.000000}{12.000000}\selectfont 1.4}%
\end{pgfscope}%
\begin{pgfscope}%
\definecolor{textcolor}{rgb}{0.000000,0.000000,0.000000}%
\pgfsetstrokecolor{textcolor}%
\pgfsetfillcolor{textcolor}%
\pgftext[x=4.243009in,y=6.104167in,,bottom,rotate=90.000000]{\color{textcolor}\sffamily\fontsize{10.000000}{12.000000}\selectfont \(\displaystyle U\)}%
\end{pgfscope}%
\begin{pgfscope}%
\pgfpathrectangle{\pgfqpoint{4.616667in}{4.530556in}}{\pgfqpoint{3.184722in}{3.147222in}}%
\pgfusepath{clip}%
\pgfsetrectcap%
\pgfsetroundjoin%
\pgfsetlinewidth{1.505625pt}%
\definecolor{currentstroke}{rgb}{0.121569,0.466667,0.705882}%
\pgfsetstrokecolor{currentstroke}%
\pgfsetdash{}{0pt}%
\pgfpathmoveto{\pgfqpoint{4.761427in}{6.643029in}}%
\pgfpathlineto{\pgfqpoint{6.421577in}{6.643034in}}%
\pgfpathlineto{\pgfqpoint{6.422808in}{6.634894in}}%
\pgfpathlineto{\pgfqpoint{6.422735in}{6.651563in}}%
\pgfpathlineto{\pgfqpoint{6.422880in}{6.638248in}}%
\pgfpathlineto{\pgfqpoint{6.424038in}{6.576296in}}%
\pgfpathlineto{\pgfqpoint{6.424183in}{6.702299in}}%
\pgfpathlineto{\pgfqpoint{6.425486in}{6.085242in}}%
\pgfpathlineto{\pgfqpoint{6.425341in}{7.059961in}}%
\pgfpathlineto{\pgfqpoint{6.425558in}{6.950437in}}%
\pgfpathlineto{\pgfqpoint{6.425848in}{7.534722in}}%
\pgfpathlineto{\pgfqpoint{6.425703in}{5.882789in}}%
\pgfpathlineto{\pgfqpoint{6.425920in}{6.038285in}}%
\pgfpathlineto{\pgfqpoint{6.426065in}{7.497630in}}%
\pgfpathlineto{\pgfqpoint{6.427657in}{4.673611in}}%
\pgfpathlineto{\pgfqpoint{7.656629in}{4.673611in}}%
\pgfpathlineto{\pgfqpoint{7.656629in}{4.673611in}}%
\pgfusepath{stroke}%
\end{pgfscope}%
\begin{pgfscope}%
\pgfpathrectangle{\pgfqpoint{4.616667in}{4.530556in}}{\pgfqpoint{3.184722in}{3.147222in}}%
\pgfusepath{clip}%
\pgfsetrectcap%
\pgfsetroundjoin%
\pgfsetlinewidth{1.505625pt}%
\definecolor{currentstroke}{rgb}{1.000000,0.498039,0.054902}%
\pgfsetstrokecolor{currentstroke}%
\pgfsetdash{}{0pt}%
\pgfpathmoveto{\pgfqpoint{4.761427in}{6.643029in}}%
\pgfpathlineto{\pgfqpoint{6.421794in}{6.643881in}}%
\pgfpathlineto{\pgfqpoint{6.422952in}{6.651167in}}%
\pgfpathlineto{\pgfqpoint{6.423097in}{6.634404in}}%
\pgfpathlineto{\pgfqpoint{6.424400in}{6.710151in}}%
\pgfpathlineto{\pgfqpoint{6.424255in}{6.592990in}}%
\pgfpathlineto{\pgfqpoint{6.424472in}{6.605916in}}%
\pgfpathlineto{\pgfqpoint{6.425124in}{6.827195in}}%
\pgfpathlineto{\pgfqpoint{6.425269in}{6.471709in}}%
\pgfpathlineto{\pgfqpoint{6.425848in}{7.132632in}}%
\pgfpathlineto{\pgfqpoint{6.426427in}{4.673611in}}%
\pgfpathlineto{\pgfqpoint{7.656629in}{4.673611in}}%
\pgfpathlineto{\pgfqpoint{7.656629in}{4.673611in}}%
\pgfusepath{stroke}%
\end{pgfscope}%
\begin{pgfscope}%
\pgfpathrectangle{\pgfqpoint{4.616667in}{4.530556in}}{\pgfqpoint{3.184722in}{3.147222in}}%
\pgfusepath{clip}%
\pgfsetrectcap%
\pgfsetroundjoin%
\pgfsetlinewidth{1.505625pt}%
\definecolor{currentstroke}{rgb}{0.172549,0.627451,0.172549}%
\pgfsetstrokecolor{currentstroke}%
\pgfsetdash{}{0pt}%
\pgfpathmoveto{\pgfqpoint{4.761427in}{6.643029in}}%
\pgfpathlineto{\pgfqpoint{6.425848in}{6.642375in}}%
\pgfpathlineto{\pgfqpoint{6.425992in}{6.623234in}}%
\pgfpathlineto{\pgfqpoint{6.429177in}{4.673611in}}%
\pgfpathlineto{\pgfqpoint{7.656629in}{4.673611in}}%
\pgfpathlineto{\pgfqpoint{7.656629in}{4.673611in}}%
\pgfusepath{stroke}%
\end{pgfscope}%
\begin{pgfscope}%
\pgfpathrectangle{\pgfqpoint{4.616667in}{4.530556in}}{\pgfqpoint{3.184722in}{3.147222in}}%
\pgfusepath{clip}%
\pgfsetrectcap%
\pgfsetroundjoin%
\pgfsetlinewidth{1.505625pt}%
\definecolor{currentstroke}{rgb}{0.839216,0.152941,0.156863}%
\pgfsetstrokecolor{currentstroke}%
\pgfsetdash{}{0pt}%
\pgfpathmoveto{\pgfqpoint{4.761427in}{6.643029in}}%
\pgfpathlineto{\pgfqpoint{6.208992in}{6.643029in}}%
\pgfpathlineto{\pgfqpoint{6.210584in}{4.673611in}}%
\pgfpathlineto{\pgfqpoint{7.656629in}{4.673611in}}%
\pgfpathlineto{\pgfqpoint{7.656629in}{4.673611in}}%
\pgfusepath{stroke}%
\end{pgfscope}%
\begin{pgfscope}%
\pgfpathrectangle{\pgfqpoint{4.616667in}{4.530556in}}{\pgfqpoint{3.184722in}{3.147222in}}%
\pgfusepath{clip}%
\pgfsetrectcap%
\pgfsetroundjoin%
\pgfsetlinewidth{0.501875pt}%
\definecolor{currentstroke}{rgb}{0.000000,0.000000,0.000000}%
\pgfsetstrokecolor{currentstroke}%
\pgfsetdash{}{0pt}%
\pgfpathmoveto{\pgfqpoint{4.761427in}{6.643029in}}%
\pgfpathlineto{\pgfqpoint{6.426137in}{6.643029in}}%
\pgfpathlineto{\pgfqpoint{6.427730in}{4.673611in}}%
\pgfpathlineto{\pgfqpoint{7.656629in}{4.673611in}}%
\pgfpathlineto{\pgfqpoint{7.656629in}{4.673611in}}%
\pgfusepath{stroke}%
\end{pgfscope}%
\begin{pgfscope}%
\pgfsetrectcap%
\pgfsetmiterjoin%
\pgfsetlinewidth{0.803000pt}%
\definecolor{currentstroke}{rgb}{0.000000,0.000000,0.000000}%
\pgfsetstrokecolor{currentstroke}%
\pgfsetdash{}{0pt}%
\pgfpathmoveto{\pgfqpoint{4.616667in}{4.530556in}}%
\pgfpathlineto{\pgfqpoint{4.616667in}{7.677778in}}%
\pgfusepath{stroke}%
\end{pgfscope}%
\begin{pgfscope}%
\pgfsetrectcap%
\pgfsetmiterjoin%
\pgfsetlinewidth{0.803000pt}%
\definecolor{currentstroke}{rgb}{0.000000,0.000000,0.000000}%
\pgfsetstrokecolor{currentstroke}%
\pgfsetdash{}{0pt}%
\pgfpathmoveto{\pgfqpoint{7.801389in}{4.530556in}}%
\pgfpathlineto{\pgfqpoint{7.801389in}{7.677778in}}%
\pgfusepath{stroke}%
\end{pgfscope}%
\begin{pgfscope}%
\pgfsetrectcap%
\pgfsetmiterjoin%
\pgfsetlinewidth{0.803000pt}%
\definecolor{currentstroke}{rgb}{0.000000,0.000000,0.000000}%
\pgfsetstrokecolor{currentstroke}%
\pgfsetdash{}{0pt}%
\pgfpathmoveto{\pgfqpoint{4.616667in}{4.530556in}}%
\pgfpathlineto{\pgfqpoint{7.801389in}{4.530556in}}%
\pgfusepath{stroke}%
\end{pgfscope}%
\begin{pgfscope}%
\pgfsetrectcap%
\pgfsetmiterjoin%
\pgfsetlinewidth{0.803000pt}%
\definecolor{currentstroke}{rgb}{0.000000,0.000000,0.000000}%
\pgfsetstrokecolor{currentstroke}%
\pgfsetdash{}{0pt}%
\pgfpathmoveto{\pgfqpoint{4.616667in}{7.677778in}}%
\pgfpathlineto{\pgfqpoint{7.801389in}{7.677778in}}%
\pgfusepath{stroke}%
\end{pgfscope}%
\begin{pgfscope}%
\definecolor{textcolor}{rgb}{0.000000,0.000000,0.000000}%
\pgfsetstrokecolor{textcolor}%
\pgfsetfillcolor{textcolor}%
\pgftext[x=6.209028in,y=7.761111in,,base]{\color{textcolor}\sffamily\fontsize{12.000000}{14.400000}\selectfont \(\displaystyle  t = 0.6 \)}%
\end{pgfscope}%
\begin{pgfscope}%
\pgfsetbuttcap%
\pgfsetmiterjoin%
\definecolor{currentfill}{rgb}{1.000000,1.000000,1.000000}%
\pgfsetfillcolor{currentfill}%
\pgfsetlinewidth{0.000000pt}%
\definecolor{currentstroke}{rgb}{0.000000,0.000000,0.000000}%
\pgfsetstrokecolor{currentstroke}%
\pgfsetstrokeopacity{0.000000}%
\pgfsetdash{}{0pt}%
\pgfpathmoveto{\pgfqpoint{0.691667in}{0.580556in}}%
\pgfpathlineto{\pgfqpoint{3.876389in}{0.580556in}}%
\pgfpathlineto{\pgfqpoint{3.876389in}{3.727778in}}%
\pgfpathlineto{\pgfqpoint{0.691667in}{3.727778in}}%
\pgfpathclose%
\pgfusepath{fill}%
\end{pgfscope}%
\begin{pgfscope}%
\pgfsetbuttcap%
\pgfsetroundjoin%
\definecolor{currentfill}{rgb}{0.000000,0.000000,0.000000}%
\pgfsetfillcolor{currentfill}%
\pgfsetlinewidth{0.803000pt}%
\definecolor{currentstroke}{rgb}{0.000000,0.000000,0.000000}%
\pgfsetstrokecolor{currentstroke}%
\pgfsetdash{}{0pt}%
\pgfsys@defobject{currentmarker}{\pgfqpoint{0.000000in}{-0.048611in}}{\pgfqpoint{0.000000in}{0.000000in}}{%
\pgfpathmoveto{\pgfqpoint{0.000000in}{0.000000in}}%
\pgfpathlineto{\pgfqpoint{0.000000in}{-0.048611in}}%
\pgfusepath{stroke,fill}%
}%
\begin{pgfscope}%
\pgfsys@transformshift{0.836391in}{0.580556in}%
\pgfsys@useobject{currentmarker}{}%
\end{pgfscope}%
\end{pgfscope}%
\begin{pgfscope}%
\definecolor{textcolor}{rgb}{0.000000,0.000000,0.000000}%
\pgfsetstrokecolor{textcolor}%
\pgfsetfillcolor{textcolor}%
\pgftext[x=0.836391in,y=0.483333in,,top]{\color{textcolor}\sffamily\fontsize{10.000000}{12.000000}\selectfont −2}%
\end{pgfscope}%
\begin{pgfscope}%
\pgfsetbuttcap%
\pgfsetroundjoin%
\definecolor{currentfill}{rgb}{0.000000,0.000000,0.000000}%
\pgfsetfillcolor{currentfill}%
\pgfsetlinewidth{0.803000pt}%
\definecolor{currentstroke}{rgb}{0.000000,0.000000,0.000000}%
\pgfsetstrokecolor{currentstroke}%
\pgfsetdash{}{0pt}%
\pgfsys@defobject{currentmarker}{\pgfqpoint{0.000000in}{-0.048611in}}{\pgfqpoint{0.000000in}{0.000000in}}{%
\pgfpathmoveto{\pgfqpoint{0.000000in}{0.000000in}}%
\pgfpathlineto{\pgfqpoint{0.000000in}{-0.048611in}}%
\pgfusepath{stroke,fill}%
}%
\begin{pgfscope}%
\pgfsys@transformshift{1.560209in}{0.580556in}%
\pgfsys@useobject{currentmarker}{}%
\end{pgfscope}%
\end{pgfscope}%
\begin{pgfscope}%
\definecolor{textcolor}{rgb}{0.000000,0.000000,0.000000}%
\pgfsetstrokecolor{textcolor}%
\pgfsetfillcolor{textcolor}%
\pgftext[x=1.560209in,y=0.483333in,,top]{\color{textcolor}\sffamily\fontsize{10.000000}{12.000000}\selectfont −1}%
\end{pgfscope}%
\begin{pgfscope}%
\pgfsetbuttcap%
\pgfsetroundjoin%
\definecolor{currentfill}{rgb}{0.000000,0.000000,0.000000}%
\pgfsetfillcolor{currentfill}%
\pgfsetlinewidth{0.803000pt}%
\definecolor{currentstroke}{rgb}{0.000000,0.000000,0.000000}%
\pgfsetstrokecolor{currentstroke}%
\pgfsetdash{}{0pt}%
\pgfsys@defobject{currentmarker}{\pgfqpoint{0.000000in}{-0.048611in}}{\pgfqpoint{0.000000in}{0.000000in}}{%
\pgfpathmoveto{\pgfqpoint{0.000000in}{0.000000in}}%
\pgfpathlineto{\pgfqpoint{0.000000in}{-0.048611in}}%
\pgfusepath{stroke,fill}%
}%
\begin{pgfscope}%
\pgfsys@transformshift{2.284028in}{0.580556in}%
\pgfsys@useobject{currentmarker}{}%
\end{pgfscope}%
\end{pgfscope}%
\begin{pgfscope}%
\definecolor{textcolor}{rgb}{0.000000,0.000000,0.000000}%
\pgfsetstrokecolor{textcolor}%
\pgfsetfillcolor{textcolor}%
\pgftext[x=2.284028in,y=0.483333in,,top]{\color{textcolor}\sffamily\fontsize{10.000000}{12.000000}\selectfont 0}%
\end{pgfscope}%
\begin{pgfscope}%
\pgfsetbuttcap%
\pgfsetroundjoin%
\definecolor{currentfill}{rgb}{0.000000,0.000000,0.000000}%
\pgfsetfillcolor{currentfill}%
\pgfsetlinewidth{0.803000pt}%
\definecolor{currentstroke}{rgb}{0.000000,0.000000,0.000000}%
\pgfsetstrokecolor{currentstroke}%
\pgfsetdash{}{0pt}%
\pgfsys@defobject{currentmarker}{\pgfqpoint{0.000000in}{-0.048611in}}{\pgfqpoint{0.000000in}{0.000000in}}{%
\pgfpathmoveto{\pgfqpoint{0.000000in}{0.000000in}}%
\pgfpathlineto{\pgfqpoint{0.000000in}{-0.048611in}}%
\pgfusepath{stroke,fill}%
}%
\begin{pgfscope}%
\pgfsys@transformshift{3.007846in}{0.580556in}%
\pgfsys@useobject{currentmarker}{}%
\end{pgfscope}%
\end{pgfscope}%
\begin{pgfscope}%
\definecolor{textcolor}{rgb}{0.000000,0.000000,0.000000}%
\pgfsetstrokecolor{textcolor}%
\pgfsetfillcolor{textcolor}%
\pgftext[x=3.007846in,y=0.483333in,,top]{\color{textcolor}\sffamily\fontsize{10.000000}{12.000000}\selectfont 1}%
\end{pgfscope}%
\begin{pgfscope}%
\pgfsetbuttcap%
\pgfsetroundjoin%
\definecolor{currentfill}{rgb}{0.000000,0.000000,0.000000}%
\pgfsetfillcolor{currentfill}%
\pgfsetlinewidth{0.803000pt}%
\definecolor{currentstroke}{rgb}{0.000000,0.000000,0.000000}%
\pgfsetstrokecolor{currentstroke}%
\pgfsetdash{}{0pt}%
\pgfsys@defobject{currentmarker}{\pgfqpoint{0.000000in}{-0.048611in}}{\pgfqpoint{0.000000in}{0.000000in}}{%
\pgfpathmoveto{\pgfqpoint{0.000000in}{0.000000in}}%
\pgfpathlineto{\pgfqpoint{0.000000in}{-0.048611in}}%
\pgfusepath{stroke,fill}%
}%
\begin{pgfscope}%
\pgfsys@transformshift{3.731665in}{0.580556in}%
\pgfsys@useobject{currentmarker}{}%
\end{pgfscope}%
\end{pgfscope}%
\begin{pgfscope}%
\definecolor{textcolor}{rgb}{0.000000,0.000000,0.000000}%
\pgfsetstrokecolor{textcolor}%
\pgfsetfillcolor{textcolor}%
\pgftext[x=3.731665in,y=0.483333in,,top]{\color{textcolor}\sffamily\fontsize{10.000000}{12.000000}\selectfont 2}%
\end{pgfscope}%
\begin{pgfscope}%
\definecolor{textcolor}{rgb}{0.000000,0.000000,0.000000}%
\pgfsetstrokecolor{textcolor}%
\pgfsetfillcolor{textcolor}%
\pgftext[x=2.284028in,y=0.293365in,,top]{\color{textcolor}\sffamily\fontsize{10.000000}{12.000000}\selectfont \(\displaystyle x\)}%
\end{pgfscope}%
\begin{pgfscope}%
\pgfsetbuttcap%
\pgfsetroundjoin%
\definecolor{currentfill}{rgb}{0.000000,0.000000,0.000000}%
\pgfsetfillcolor{currentfill}%
\pgfsetlinewidth{0.803000pt}%
\definecolor{currentstroke}{rgb}{0.000000,0.000000,0.000000}%
\pgfsetstrokecolor{currentstroke}%
\pgfsetdash{}{0pt}%
\pgfsys@defobject{currentmarker}{\pgfqpoint{-0.048611in}{0.000000in}}{\pgfqpoint{0.000000in}{0.000000in}}{%
\pgfpathmoveto{\pgfqpoint{0.000000in}{0.000000in}}%
\pgfpathlineto{\pgfqpoint{-0.048611in}{0.000000in}}%
\pgfusepath{stroke,fill}%
}%
\begin{pgfscope}%
\pgfsys@transformshift{0.691667in}{0.723611in}%
\pgfsys@useobject{currentmarker}{}%
\end{pgfscope}%
\end{pgfscope}%
\begin{pgfscope}%
\definecolor{textcolor}{rgb}{0.000000,0.000000,0.000000}%
\pgfsetstrokecolor{textcolor}%
\pgfsetfillcolor{textcolor}%
\pgftext[x=0.373565in,y=0.670850in,left,base]{\color{textcolor}\sffamily\fontsize{10.000000}{12.000000}\selectfont 0.0}%
\end{pgfscope}%
\begin{pgfscope}%
\pgfsetbuttcap%
\pgfsetroundjoin%
\definecolor{currentfill}{rgb}{0.000000,0.000000,0.000000}%
\pgfsetfillcolor{currentfill}%
\pgfsetlinewidth{0.803000pt}%
\definecolor{currentstroke}{rgb}{0.000000,0.000000,0.000000}%
\pgfsetstrokecolor{currentstroke}%
\pgfsetdash{}{0pt}%
\pgfsys@defobject{currentmarker}{\pgfqpoint{-0.048611in}{0.000000in}}{\pgfqpoint{0.000000in}{0.000000in}}{%
\pgfpathmoveto{\pgfqpoint{0.000000in}{0.000000in}}%
\pgfpathlineto{\pgfqpoint{-0.048611in}{0.000000in}}%
\pgfusepath{stroke,fill}%
}%
\begin{pgfscope}%
\pgfsys@transformshift{0.691667in}{1.117495in}%
\pgfsys@useobject{currentmarker}{}%
\end{pgfscope}%
\end{pgfscope}%
\begin{pgfscope}%
\definecolor{textcolor}{rgb}{0.000000,0.000000,0.000000}%
\pgfsetstrokecolor{textcolor}%
\pgfsetfillcolor{textcolor}%
\pgftext[x=0.373565in,y=1.064733in,left,base]{\color{textcolor}\sffamily\fontsize{10.000000}{12.000000}\selectfont 0.2}%
\end{pgfscope}%
\begin{pgfscope}%
\pgfsetbuttcap%
\pgfsetroundjoin%
\definecolor{currentfill}{rgb}{0.000000,0.000000,0.000000}%
\pgfsetfillcolor{currentfill}%
\pgfsetlinewidth{0.803000pt}%
\definecolor{currentstroke}{rgb}{0.000000,0.000000,0.000000}%
\pgfsetstrokecolor{currentstroke}%
\pgfsetdash{}{0pt}%
\pgfsys@defobject{currentmarker}{\pgfqpoint{-0.048611in}{0.000000in}}{\pgfqpoint{0.000000in}{0.000000in}}{%
\pgfpathmoveto{\pgfqpoint{0.000000in}{0.000000in}}%
\pgfpathlineto{\pgfqpoint{-0.048611in}{0.000000in}}%
\pgfusepath{stroke,fill}%
}%
\begin{pgfscope}%
\pgfsys@transformshift{0.691667in}{1.511378in}%
\pgfsys@useobject{currentmarker}{}%
\end{pgfscope}%
\end{pgfscope}%
\begin{pgfscope}%
\definecolor{textcolor}{rgb}{0.000000,0.000000,0.000000}%
\pgfsetstrokecolor{textcolor}%
\pgfsetfillcolor{textcolor}%
\pgftext[x=0.373565in,y=1.458617in,left,base]{\color{textcolor}\sffamily\fontsize{10.000000}{12.000000}\selectfont 0.4}%
\end{pgfscope}%
\begin{pgfscope}%
\pgfsetbuttcap%
\pgfsetroundjoin%
\definecolor{currentfill}{rgb}{0.000000,0.000000,0.000000}%
\pgfsetfillcolor{currentfill}%
\pgfsetlinewidth{0.803000pt}%
\definecolor{currentstroke}{rgb}{0.000000,0.000000,0.000000}%
\pgfsetstrokecolor{currentstroke}%
\pgfsetdash{}{0pt}%
\pgfsys@defobject{currentmarker}{\pgfqpoint{-0.048611in}{0.000000in}}{\pgfqpoint{0.000000in}{0.000000in}}{%
\pgfpathmoveto{\pgfqpoint{0.000000in}{0.000000in}}%
\pgfpathlineto{\pgfqpoint{-0.048611in}{0.000000in}}%
\pgfusepath{stroke,fill}%
}%
\begin{pgfscope}%
\pgfsys@transformshift{0.691667in}{1.905262in}%
\pgfsys@useobject{currentmarker}{}%
\end{pgfscope}%
\end{pgfscope}%
\begin{pgfscope}%
\definecolor{textcolor}{rgb}{0.000000,0.000000,0.000000}%
\pgfsetstrokecolor{textcolor}%
\pgfsetfillcolor{textcolor}%
\pgftext[x=0.373565in,y=1.852500in,left,base]{\color{textcolor}\sffamily\fontsize{10.000000}{12.000000}\selectfont 0.6}%
\end{pgfscope}%
\begin{pgfscope}%
\pgfsetbuttcap%
\pgfsetroundjoin%
\definecolor{currentfill}{rgb}{0.000000,0.000000,0.000000}%
\pgfsetfillcolor{currentfill}%
\pgfsetlinewidth{0.803000pt}%
\definecolor{currentstroke}{rgb}{0.000000,0.000000,0.000000}%
\pgfsetstrokecolor{currentstroke}%
\pgfsetdash{}{0pt}%
\pgfsys@defobject{currentmarker}{\pgfqpoint{-0.048611in}{0.000000in}}{\pgfqpoint{0.000000in}{0.000000in}}{%
\pgfpathmoveto{\pgfqpoint{0.000000in}{0.000000in}}%
\pgfpathlineto{\pgfqpoint{-0.048611in}{0.000000in}}%
\pgfusepath{stroke,fill}%
}%
\begin{pgfscope}%
\pgfsys@transformshift{0.691667in}{2.299146in}%
\pgfsys@useobject{currentmarker}{}%
\end{pgfscope}%
\end{pgfscope}%
\begin{pgfscope}%
\definecolor{textcolor}{rgb}{0.000000,0.000000,0.000000}%
\pgfsetstrokecolor{textcolor}%
\pgfsetfillcolor{textcolor}%
\pgftext[x=0.373565in,y=2.246384in,left,base]{\color{textcolor}\sffamily\fontsize{10.000000}{12.000000}\selectfont 0.8}%
\end{pgfscope}%
\begin{pgfscope}%
\pgfsetbuttcap%
\pgfsetroundjoin%
\definecolor{currentfill}{rgb}{0.000000,0.000000,0.000000}%
\pgfsetfillcolor{currentfill}%
\pgfsetlinewidth{0.803000pt}%
\definecolor{currentstroke}{rgb}{0.000000,0.000000,0.000000}%
\pgfsetstrokecolor{currentstroke}%
\pgfsetdash{}{0pt}%
\pgfsys@defobject{currentmarker}{\pgfqpoint{-0.048611in}{0.000000in}}{\pgfqpoint{0.000000in}{0.000000in}}{%
\pgfpathmoveto{\pgfqpoint{0.000000in}{0.000000in}}%
\pgfpathlineto{\pgfqpoint{-0.048611in}{0.000000in}}%
\pgfusepath{stroke,fill}%
}%
\begin{pgfscope}%
\pgfsys@transformshift{0.691667in}{2.693029in}%
\pgfsys@useobject{currentmarker}{}%
\end{pgfscope}%
\end{pgfscope}%
\begin{pgfscope}%
\definecolor{textcolor}{rgb}{0.000000,0.000000,0.000000}%
\pgfsetstrokecolor{textcolor}%
\pgfsetfillcolor{textcolor}%
\pgftext[x=0.373565in,y=2.640268in,left,base]{\color{textcolor}\sffamily\fontsize{10.000000}{12.000000}\selectfont 1.0}%
\end{pgfscope}%
\begin{pgfscope}%
\pgfsetbuttcap%
\pgfsetroundjoin%
\definecolor{currentfill}{rgb}{0.000000,0.000000,0.000000}%
\pgfsetfillcolor{currentfill}%
\pgfsetlinewidth{0.803000pt}%
\definecolor{currentstroke}{rgb}{0.000000,0.000000,0.000000}%
\pgfsetstrokecolor{currentstroke}%
\pgfsetdash{}{0pt}%
\pgfsys@defobject{currentmarker}{\pgfqpoint{-0.048611in}{0.000000in}}{\pgfqpoint{0.000000in}{0.000000in}}{%
\pgfpathmoveto{\pgfqpoint{0.000000in}{0.000000in}}%
\pgfpathlineto{\pgfqpoint{-0.048611in}{0.000000in}}%
\pgfusepath{stroke,fill}%
}%
\begin{pgfscope}%
\pgfsys@transformshift{0.691667in}{3.086913in}%
\pgfsys@useobject{currentmarker}{}%
\end{pgfscope}%
\end{pgfscope}%
\begin{pgfscope}%
\definecolor{textcolor}{rgb}{0.000000,0.000000,0.000000}%
\pgfsetstrokecolor{textcolor}%
\pgfsetfillcolor{textcolor}%
\pgftext[x=0.373565in,y=3.034151in,left,base]{\color{textcolor}\sffamily\fontsize{10.000000}{12.000000}\selectfont 1.2}%
\end{pgfscope}%
\begin{pgfscope}%
\pgfsetbuttcap%
\pgfsetroundjoin%
\definecolor{currentfill}{rgb}{0.000000,0.000000,0.000000}%
\pgfsetfillcolor{currentfill}%
\pgfsetlinewidth{0.803000pt}%
\definecolor{currentstroke}{rgb}{0.000000,0.000000,0.000000}%
\pgfsetstrokecolor{currentstroke}%
\pgfsetdash{}{0pt}%
\pgfsys@defobject{currentmarker}{\pgfqpoint{-0.048611in}{0.000000in}}{\pgfqpoint{0.000000in}{0.000000in}}{%
\pgfpathmoveto{\pgfqpoint{0.000000in}{0.000000in}}%
\pgfpathlineto{\pgfqpoint{-0.048611in}{0.000000in}}%
\pgfusepath{stroke,fill}%
}%
\begin{pgfscope}%
\pgfsys@transformshift{0.691667in}{3.480797in}%
\pgfsys@useobject{currentmarker}{}%
\end{pgfscope}%
\end{pgfscope}%
\begin{pgfscope}%
\definecolor{textcolor}{rgb}{0.000000,0.000000,0.000000}%
\pgfsetstrokecolor{textcolor}%
\pgfsetfillcolor{textcolor}%
\pgftext[x=0.373565in,y=3.428035in,left,base]{\color{textcolor}\sffamily\fontsize{10.000000}{12.000000}\selectfont 1.4}%
\end{pgfscope}%
\begin{pgfscope}%
\definecolor{textcolor}{rgb}{0.000000,0.000000,0.000000}%
\pgfsetstrokecolor{textcolor}%
\pgfsetfillcolor{textcolor}%
\pgftext[x=0.318009in,y=2.154167in,,bottom,rotate=90.000000]{\color{textcolor}\sffamily\fontsize{10.000000}{12.000000}\selectfont \(\displaystyle U\)}%
\end{pgfscope}%
\begin{pgfscope}%
\pgfpathrectangle{\pgfqpoint{0.691667in}{0.580556in}}{\pgfqpoint{3.184722in}{3.147222in}}%
\pgfusepath{clip}%
\pgfsetrectcap%
\pgfsetroundjoin%
\pgfsetlinewidth{1.505625pt}%
\definecolor{currentstroke}{rgb}{0.121569,0.466667,0.705882}%
\pgfsetstrokecolor{currentstroke}%
\pgfsetdash{}{0pt}%
\pgfpathmoveto{\pgfqpoint{0.836427in}{2.693029in}}%
\pgfpathlineto{\pgfqpoint{2.568959in}{2.693034in}}%
\pgfpathlineto{\pgfqpoint{2.570189in}{2.684894in}}%
\pgfpathlineto{\pgfqpoint{2.570117in}{2.701563in}}%
\pgfpathlineto{\pgfqpoint{2.570262in}{2.688248in}}%
\pgfpathlineto{\pgfqpoint{2.571420in}{2.626296in}}%
\pgfpathlineto{\pgfqpoint{2.571565in}{2.752299in}}%
\pgfpathlineto{\pgfqpoint{2.572868in}{2.135242in}}%
\pgfpathlineto{\pgfqpoint{2.572723in}{3.109961in}}%
\pgfpathlineto{\pgfqpoint{2.572940in}{3.000437in}}%
\pgfpathlineto{\pgfqpoint{2.573229in}{3.584722in}}%
\pgfpathlineto{\pgfqpoint{2.573085in}{1.932789in}}%
\pgfpathlineto{\pgfqpoint{2.573302in}{2.088285in}}%
\pgfpathlineto{\pgfqpoint{2.573447in}{3.547630in}}%
\pgfpathlineto{\pgfqpoint{2.575039in}{0.723611in}}%
\pgfpathlineto{\pgfqpoint{3.731629in}{0.723611in}}%
\pgfpathlineto{\pgfqpoint{3.731629in}{0.723611in}}%
\pgfusepath{stroke}%
\end{pgfscope}%
\begin{pgfscope}%
\pgfpathrectangle{\pgfqpoint{0.691667in}{0.580556in}}{\pgfqpoint{3.184722in}{3.147222in}}%
\pgfusepath{clip}%
\pgfsetrectcap%
\pgfsetroundjoin%
\pgfsetlinewidth{1.505625pt}%
\definecolor{currentstroke}{rgb}{1.000000,0.498039,0.054902}%
\pgfsetstrokecolor{currentstroke}%
\pgfsetdash{}{0pt}%
\pgfpathmoveto{\pgfqpoint{0.836427in}{2.693029in}}%
\pgfpathlineto{\pgfqpoint{2.569176in}{2.693881in}}%
\pgfpathlineto{\pgfqpoint{2.570334in}{2.701167in}}%
\pgfpathlineto{\pgfqpoint{2.570479in}{2.684404in}}%
\pgfpathlineto{\pgfqpoint{2.571782in}{2.760151in}}%
\pgfpathlineto{\pgfqpoint{2.571637in}{2.642990in}}%
\pgfpathlineto{\pgfqpoint{2.571854in}{2.655916in}}%
\pgfpathlineto{\pgfqpoint{2.572506in}{2.877195in}}%
\pgfpathlineto{\pgfqpoint{2.572650in}{2.521709in}}%
\pgfpathlineto{\pgfqpoint{2.573229in}{3.182632in}}%
\pgfpathlineto{\pgfqpoint{2.573809in}{0.723611in}}%
\pgfpathlineto{\pgfqpoint{3.731629in}{0.723611in}}%
\pgfpathlineto{\pgfqpoint{3.731629in}{0.723611in}}%
\pgfusepath{stroke}%
\end{pgfscope}%
\begin{pgfscope}%
\pgfpathrectangle{\pgfqpoint{0.691667in}{0.580556in}}{\pgfqpoint{3.184722in}{3.147222in}}%
\pgfusepath{clip}%
\pgfsetrectcap%
\pgfsetroundjoin%
\pgfsetlinewidth{1.505625pt}%
\definecolor{currentstroke}{rgb}{0.172549,0.627451,0.172549}%
\pgfsetstrokecolor{currentstroke}%
\pgfsetdash{}{0pt}%
\pgfpathmoveto{\pgfqpoint{0.836427in}{2.693029in}}%
\pgfpathlineto{\pgfqpoint{2.573229in}{2.692375in}}%
\pgfpathlineto{\pgfqpoint{2.573374in}{2.673234in}}%
\pgfpathlineto{\pgfqpoint{2.576559in}{0.723611in}}%
\pgfpathlineto{\pgfqpoint{3.731629in}{0.723611in}}%
\pgfpathlineto{\pgfqpoint{3.731629in}{0.723611in}}%
\pgfusepath{stroke}%
\end{pgfscope}%
\begin{pgfscope}%
\pgfpathrectangle{\pgfqpoint{0.691667in}{0.580556in}}{\pgfqpoint{3.184722in}{3.147222in}}%
\pgfusepath{clip}%
\pgfsetrectcap%
\pgfsetroundjoin%
\pgfsetlinewidth{1.505625pt}%
\definecolor{currentstroke}{rgb}{0.839216,0.152941,0.156863}%
\pgfsetstrokecolor{currentstroke}%
\pgfsetdash{}{0pt}%
\pgfpathmoveto{\pgfqpoint{0.836427in}{2.693029in}}%
\pgfpathlineto{\pgfqpoint{2.283992in}{2.693029in}}%
\pgfpathlineto{\pgfqpoint{2.285584in}{0.723611in}}%
\pgfpathlineto{\pgfqpoint{3.731629in}{0.723611in}}%
\pgfpathlineto{\pgfqpoint{3.731629in}{0.723611in}}%
\pgfusepath{stroke}%
\end{pgfscope}%
\begin{pgfscope}%
\pgfpathrectangle{\pgfqpoint{0.691667in}{0.580556in}}{\pgfqpoint{3.184722in}{3.147222in}}%
\pgfusepath{clip}%
\pgfsetrectcap%
\pgfsetroundjoin%
\pgfsetlinewidth{0.501875pt}%
\definecolor{currentstroke}{rgb}{0.000000,0.000000,0.000000}%
\pgfsetstrokecolor{currentstroke}%
\pgfsetdash{}{0pt}%
\pgfpathmoveto{\pgfqpoint{0.836427in}{2.693029in}}%
\pgfpathlineto{\pgfqpoint{2.573519in}{2.693029in}}%
\pgfpathlineto{\pgfqpoint{2.575111in}{0.723611in}}%
\pgfpathlineto{\pgfqpoint{3.731629in}{0.723611in}}%
\pgfpathlineto{\pgfqpoint{3.731629in}{0.723611in}}%
\pgfusepath{stroke}%
\end{pgfscope}%
\begin{pgfscope}%
\pgfsetrectcap%
\pgfsetmiterjoin%
\pgfsetlinewidth{0.803000pt}%
\definecolor{currentstroke}{rgb}{0.000000,0.000000,0.000000}%
\pgfsetstrokecolor{currentstroke}%
\pgfsetdash{}{0pt}%
\pgfpathmoveto{\pgfqpoint{0.691667in}{0.580556in}}%
\pgfpathlineto{\pgfqpoint{0.691667in}{3.727778in}}%
\pgfusepath{stroke}%
\end{pgfscope}%
\begin{pgfscope}%
\pgfsetrectcap%
\pgfsetmiterjoin%
\pgfsetlinewidth{0.803000pt}%
\definecolor{currentstroke}{rgb}{0.000000,0.000000,0.000000}%
\pgfsetstrokecolor{currentstroke}%
\pgfsetdash{}{0pt}%
\pgfpathmoveto{\pgfqpoint{3.876389in}{0.580556in}}%
\pgfpathlineto{\pgfqpoint{3.876389in}{3.727778in}}%
\pgfusepath{stroke}%
\end{pgfscope}%
\begin{pgfscope}%
\pgfsetrectcap%
\pgfsetmiterjoin%
\pgfsetlinewidth{0.803000pt}%
\definecolor{currentstroke}{rgb}{0.000000,0.000000,0.000000}%
\pgfsetstrokecolor{currentstroke}%
\pgfsetdash{}{0pt}%
\pgfpathmoveto{\pgfqpoint{0.691667in}{0.580556in}}%
\pgfpathlineto{\pgfqpoint{3.876389in}{0.580556in}}%
\pgfusepath{stroke}%
\end{pgfscope}%
\begin{pgfscope}%
\pgfsetrectcap%
\pgfsetmiterjoin%
\pgfsetlinewidth{0.803000pt}%
\definecolor{currentstroke}{rgb}{0.000000,0.000000,0.000000}%
\pgfsetstrokecolor{currentstroke}%
\pgfsetdash{}{0pt}%
\pgfpathmoveto{\pgfqpoint{0.691667in}{3.727778in}}%
\pgfpathlineto{\pgfqpoint{3.876389in}{3.727778in}}%
\pgfusepath{stroke}%
\end{pgfscope}%
\begin{pgfscope}%
\definecolor{textcolor}{rgb}{0.000000,0.000000,0.000000}%
\pgfsetstrokecolor{textcolor}%
\pgfsetfillcolor{textcolor}%
\pgftext[x=2.284028in,y=3.811111in,,base]{\color{textcolor}\sffamily\fontsize{12.000000}{14.400000}\selectfont \(\displaystyle  t = 0.8 \)}%
\end{pgfscope}%
\begin{pgfscope}%
\pgfsetbuttcap%
\pgfsetmiterjoin%
\definecolor{currentfill}{rgb}{1.000000,1.000000,1.000000}%
\pgfsetfillcolor{currentfill}%
\pgfsetlinewidth{0.000000pt}%
\definecolor{currentstroke}{rgb}{0.000000,0.000000,0.000000}%
\pgfsetstrokecolor{currentstroke}%
\pgfsetstrokeopacity{0.000000}%
\pgfsetdash{}{0pt}%
\pgfpathmoveto{\pgfqpoint{4.616667in}{0.580556in}}%
\pgfpathlineto{\pgfqpoint{7.801389in}{0.580556in}}%
\pgfpathlineto{\pgfqpoint{7.801389in}{3.727778in}}%
\pgfpathlineto{\pgfqpoint{4.616667in}{3.727778in}}%
\pgfpathclose%
\pgfusepath{fill}%
\end{pgfscope}%
\begin{pgfscope}%
\pgfsetbuttcap%
\pgfsetroundjoin%
\definecolor{currentfill}{rgb}{0.000000,0.000000,0.000000}%
\pgfsetfillcolor{currentfill}%
\pgfsetlinewidth{0.803000pt}%
\definecolor{currentstroke}{rgb}{0.000000,0.000000,0.000000}%
\pgfsetstrokecolor{currentstroke}%
\pgfsetdash{}{0pt}%
\pgfsys@defobject{currentmarker}{\pgfqpoint{0.000000in}{-0.048611in}}{\pgfqpoint{0.000000in}{0.000000in}}{%
\pgfpathmoveto{\pgfqpoint{0.000000in}{0.000000in}}%
\pgfpathlineto{\pgfqpoint{0.000000in}{-0.048611in}}%
\pgfusepath{stroke,fill}%
}%
\begin{pgfscope}%
\pgfsys@transformshift{4.761391in}{0.580556in}%
\pgfsys@useobject{currentmarker}{}%
\end{pgfscope}%
\end{pgfscope}%
\begin{pgfscope}%
\definecolor{textcolor}{rgb}{0.000000,0.000000,0.000000}%
\pgfsetstrokecolor{textcolor}%
\pgfsetfillcolor{textcolor}%
\pgftext[x=4.761391in,y=0.483333in,,top]{\color{textcolor}\sffamily\fontsize{10.000000}{12.000000}\selectfont −2}%
\end{pgfscope}%
\begin{pgfscope}%
\pgfsetbuttcap%
\pgfsetroundjoin%
\definecolor{currentfill}{rgb}{0.000000,0.000000,0.000000}%
\pgfsetfillcolor{currentfill}%
\pgfsetlinewidth{0.803000pt}%
\definecolor{currentstroke}{rgb}{0.000000,0.000000,0.000000}%
\pgfsetstrokecolor{currentstroke}%
\pgfsetdash{}{0pt}%
\pgfsys@defobject{currentmarker}{\pgfqpoint{0.000000in}{-0.048611in}}{\pgfqpoint{0.000000in}{0.000000in}}{%
\pgfpathmoveto{\pgfqpoint{0.000000in}{0.000000in}}%
\pgfpathlineto{\pgfqpoint{0.000000in}{-0.048611in}}%
\pgfusepath{stroke,fill}%
}%
\begin{pgfscope}%
\pgfsys@transformshift{5.485209in}{0.580556in}%
\pgfsys@useobject{currentmarker}{}%
\end{pgfscope}%
\end{pgfscope}%
\begin{pgfscope}%
\definecolor{textcolor}{rgb}{0.000000,0.000000,0.000000}%
\pgfsetstrokecolor{textcolor}%
\pgfsetfillcolor{textcolor}%
\pgftext[x=5.485209in,y=0.483333in,,top]{\color{textcolor}\sffamily\fontsize{10.000000}{12.000000}\selectfont −1}%
\end{pgfscope}%
\begin{pgfscope}%
\pgfsetbuttcap%
\pgfsetroundjoin%
\definecolor{currentfill}{rgb}{0.000000,0.000000,0.000000}%
\pgfsetfillcolor{currentfill}%
\pgfsetlinewidth{0.803000pt}%
\definecolor{currentstroke}{rgb}{0.000000,0.000000,0.000000}%
\pgfsetstrokecolor{currentstroke}%
\pgfsetdash{}{0pt}%
\pgfsys@defobject{currentmarker}{\pgfqpoint{0.000000in}{-0.048611in}}{\pgfqpoint{0.000000in}{0.000000in}}{%
\pgfpathmoveto{\pgfqpoint{0.000000in}{0.000000in}}%
\pgfpathlineto{\pgfqpoint{0.000000in}{-0.048611in}}%
\pgfusepath{stroke,fill}%
}%
\begin{pgfscope}%
\pgfsys@transformshift{6.209028in}{0.580556in}%
\pgfsys@useobject{currentmarker}{}%
\end{pgfscope}%
\end{pgfscope}%
\begin{pgfscope}%
\definecolor{textcolor}{rgb}{0.000000,0.000000,0.000000}%
\pgfsetstrokecolor{textcolor}%
\pgfsetfillcolor{textcolor}%
\pgftext[x=6.209028in,y=0.483333in,,top]{\color{textcolor}\sffamily\fontsize{10.000000}{12.000000}\selectfont 0}%
\end{pgfscope}%
\begin{pgfscope}%
\pgfsetbuttcap%
\pgfsetroundjoin%
\definecolor{currentfill}{rgb}{0.000000,0.000000,0.000000}%
\pgfsetfillcolor{currentfill}%
\pgfsetlinewidth{0.803000pt}%
\definecolor{currentstroke}{rgb}{0.000000,0.000000,0.000000}%
\pgfsetstrokecolor{currentstroke}%
\pgfsetdash{}{0pt}%
\pgfsys@defobject{currentmarker}{\pgfqpoint{0.000000in}{-0.048611in}}{\pgfqpoint{0.000000in}{0.000000in}}{%
\pgfpathmoveto{\pgfqpoint{0.000000in}{0.000000in}}%
\pgfpathlineto{\pgfqpoint{0.000000in}{-0.048611in}}%
\pgfusepath{stroke,fill}%
}%
\begin{pgfscope}%
\pgfsys@transformshift{6.932846in}{0.580556in}%
\pgfsys@useobject{currentmarker}{}%
\end{pgfscope}%
\end{pgfscope}%
\begin{pgfscope}%
\definecolor{textcolor}{rgb}{0.000000,0.000000,0.000000}%
\pgfsetstrokecolor{textcolor}%
\pgfsetfillcolor{textcolor}%
\pgftext[x=6.932846in,y=0.483333in,,top]{\color{textcolor}\sffamily\fontsize{10.000000}{12.000000}\selectfont 1}%
\end{pgfscope}%
\begin{pgfscope}%
\pgfsetbuttcap%
\pgfsetroundjoin%
\definecolor{currentfill}{rgb}{0.000000,0.000000,0.000000}%
\pgfsetfillcolor{currentfill}%
\pgfsetlinewidth{0.803000pt}%
\definecolor{currentstroke}{rgb}{0.000000,0.000000,0.000000}%
\pgfsetstrokecolor{currentstroke}%
\pgfsetdash{}{0pt}%
\pgfsys@defobject{currentmarker}{\pgfqpoint{0.000000in}{-0.048611in}}{\pgfqpoint{0.000000in}{0.000000in}}{%
\pgfpathmoveto{\pgfqpoint{0.000000in}{0.000000in}}%
\pgfpathlineto{\pgfqpoint{0.000000in}{-0.048611in}}%
\pgfusepath{stroke,fill}%
}%
\begin{pgfscope}%
\pgfsys@transformshift{7.656665in}{0.580556in}%
\pgfsys@useobject{currentmarker}{}%
\end{pgfscope}%
\end{pgfscope}%
\begin{pgfscope}%
\definecolor{textcolor}{rgb}{0.000000,0.000000,0.000000}%
\pgfsetstrokecolor{textcolor}%
\pgfsetfillcolor{textcolor}%
\pgftext[x=7.656665in,y=0.483333in,,top]{\color{textcolor}\sffamily\fontsize{10.000000}{12.000000}\selectfont 2}%
\end{pgfscope}%
\begin{pgfscope}%
\definecolor{textcolor}{rgb}{0.000000,0.000000,0.000000}%
\pgfsetstrokecolor{textcolor}%
\pgfsetfillcolor{textcolor}%
\pgftext[x=6.209028in,y=0.293365in,,top]{\color{textcolor}\sffamily\fontsize{10.000000}{12.000000}\selectfont \(\displaystyle x\)}%
\end{pgfscope}%
\begin{pgfscope}%
\pgfsetbuttcap%
\pgfsetroundjoin%
\definecolor{currentfill}{rgb}{0.000000,0.000000,0.000000}%
\pgfsetfillcolor{currentfill}%
\pgfsetlinewidth{0.803000pt}%
\definecolor{currentstroke}{rgb}{0.000000,0.000000,0.000000}%
\pgfsetstrokecolor{currentstroke}%
\pgfsetdash{}{0pt}%
\pgfsys@defobject{currentmarker}{\pgfqpoint{-0.048611in}{0.000000in}}{\pgfqpoint{0.000000in}{0.000000in}}{%
\pgfpathmoveto{\pgfqpoint{0.000000in}{0.000000in}}%
\pgfpathlineto{\pgfqpoint{-0.048611in}{0.000000in}}%
\pgfusepath{stroke,fill}%
}%
\begin{pgfscope}%
\pgfsys@transformshift{4.616667in}{0.723611in}%
\pgfsys@useobject{currentmarker}{}%
\end{pgfscope}%
\end{pgfscope}%
\begin{pgfscope}%
\definecolor{textcolor}{rgb}{0.000000,0.000000,0.000000}%
\pgfsetstrokecolor{textcolor}%
\pgfsetfillcolor{textcolor}%
\pgftext[x=4.298565in,y=0.670850in,left,base]{\color{textcolor}\sffamily\fontsize{10.000000}{12.000000}\selectfont 0.0}%
\end{pgfscope}%
\begin{pgfscope}%
\pgfsetbuttcap%
\pgfsetroundjoin%
\definecolor{currentfill}{rgb}{0.000000,0.000000,0.000000}%
\pgfsetfillcolor{currentfill}%
\pgfsetlinewidth{0.803000pt}%
\definecolor{currentstroke}{rgb}{0.000000,0.000000,0.000000}%
\pgfsetstrokecolor{currentstroke}%
\pgfsetdash{}{0pt}%
\pgfsys@defobject{currentmarker}{\pgfqpoint{-0.048611in}{0.000000in}}{\pgfqpoint{0.000000in}{0.000000in}}{%
\pgfpathmoveto{\pgfqpoint{0.000000in}{0.000000in}}%
\pgfpathlineto{\pgfqpoint{-0.048611in}{0.000000in}}%
\pgfusepath{stroke,fill}%
}%
\begin{pgfscope}%
\pgfsys@transformshift{4.616667in}{1.117495in}%
\pgfsys@useobject{currentmarker}{}%
\end{pgfscope}%
\end{pgfscope}%
\begin{pgfscope}%
\definecolor{textcolor}{rgb}{0.000000,0.000000,0.000000}%
\pgfsetstrokecolor{textcolor}%
\pgfsetfillcolor{textcolor}%
\pgftext[x=4.298565in,y=1.064733in,left,base]{\color{textcolor}\sffamily\fontsize{10.000000}{12.000000}\selectfont 0.2}%
\end{pgfscope}%
\begin{pgfscope}%
\pgfsetbuttcap%
\pgfsetroundjoin%
\definecolor{currentfill}{rgb}{0.000000,0.000000,0.000000}%
\pgfsetfillcolor{currentfill}%
\pgfsetlinewidth{0.803000pt}%
\definecolor{currentstroke}{rgb}{0.000000,0.000000,0.000000}%
\pgfsetstrokecolor{currentstroke}%
\pgfsetdash{}{0pt}%
\pgfsys@defobject{currentmarker}{\pgfqpoint{-0.048611in}{0.000000in}}{\pgfqpoint{0.000000in}{0.000000in}}{%
\pgfpathmoveto{\pgfqpoint{0.000000in}{0.000000in}}%
\pgfpathlineto{\pgfqpoint{-0.048611in}{0.000000in}}%
\pgfusepath{stroke,fill}%
}%
\begin{pgfscope}%
\pgfsys@transformshift{4.616667in}{1.511378in}%
\pgfsys@useobject{currentmarker}{}%
\end{pgfscope}%
\end{pgfscope}%
\begin{pgfscope}%
\definecolor{textcolor}{rgb}{0.000000,0.000000,0.000000}%
\pgfsetstrokecolor{textcolor}%
\pgfsetfillcolor{textcolor}%
\pgftext[x=4.298565in,y=1.458617in,left,base]{\color{textcolor}\sffamily\fontsize{10.000000}{12.000000}\selectfont 0.4}%
\end{pgfscope}%
\begin{pgfscope}%
\pgfsetbuttcap%
\pgfsetroundjoin%
\definecolor{currentfill}{rgb}{0.000000,0.000000,0.000000}%
\pgfsetfillcolor{currentfill}%
\pgfsetlinewidth{0.803000pt}%
\definecolor{currentstroke}{rgb}{0.000000,0.000000,0.000000}%
\pgfsetstrokecolor{currentstroke}%
\pgfsetdash{}{0pt}%
\pgfsys@defobject{currentmarker}{\pgfqpoint{-0.048611in}{0.000000in}}{\pgfqpoint{0.000000in}{0.000000in}}{%
\pgfpathmoveto{\pgfqpoint{0.000000in}{0.000000in}}%
\pgfpathlineto{\pgfqpoint{-0.048611in}{0.000000in}}%
\pgfusepath{stroke,fill}%
}%
\begin{pgfscope}%
\pgfsys@transformshift{4.616667in}{1.905262in}%
\pgfsys@useobject{currentmarker}{}%
\end{pgfscope}%
\end{pgfscope}%
\begin{pgfscope}%
\definecolor{textcolor}{rgb}{0.000000,0.000000,0.000000}%
\pgfsetstrokecolor{textcolor}%
\pgfsetfillcolor{textcolor}%
\pgftext[x=4.298565in,y=1.852500in,left,base]{\color{textcolor}\sffamily\fontsize{10.000000}{12.000000}\selectfont 0.6}%
\end{pgfscope}%
\begin{pgfscope}%
\pgfsetbuttcap%
\pgfsetroundjoin%
\definecolor{currentfill}{rgb}{0.000000,0.000000,0.000000}%
\pgfsetfillcolor{currentfill}%
\pgfsetlinewidth{0.803000pt}%
\definecolor{currentstroke}{rgb}{0.000000,0.000000,0.000000}%
\pgfsetstrokecolor{currentstroke}%
\pgfsetdash{}{0pt}%
\pgfsys@defobject{currentmarker}{\pgfqpoint{-0.048611in}{0.000000in}}{\pgfqpoint{0.000000in}{0.000000in}}{%
\pgfpathmoveto{\pgfqpoint{0.000000in}{0.000000in}}%
\pgfpathlineto{\pgfqpoint{-0.048611in}{0.000000in}}%
\pgfusepath{stroke,fill}%
}%
\begin{pgfscope}%
\pgfsys@transformshift{4.616667in}{2.299146in}%
\pgfsys@useobject{currentmarker}{}%
\end{pgfscope}%
\end{pgfscope}%
\begin{pgfscope}%
\definecolor{textcolor}{rgb}{0.000000,0.000000,0.000000}%
\pgfsetstrokecolor{textcolor}%
\pgfsetfillcolor{textcolor}%
\pgftext[x=4.298565in,y=2.246384in,left,base]{\color{textcolor}\sffamily\fontsize{10.000000}{12.000000}\selectfont 0.8}%
\end{pgfscope}%
\begin{pgfscope}%
\pgfsetbuttcap%
\pgfsetroundjoin%
\definecolor{currentfill}{rgb}{0.000000,0.000000,0.000000}%
\pgfsetfillcolor{currentfill}%
\pgfsetlinewidth{0.803000pt}%
\definecolor{currentstroke}{rgb}{0.000000,0.000000,0.000000}%
\pgfsetstrokecolor{currentstroke}%
\pgfsetdash{}{0pt}%
\pgfsys@defobject{currentmarker}{\pgfqpoint{-0.048611in}{0.000000in}}{\pgfqpoint{0.000000in}{0.000000in}}{%
\pgfpathmoveto{\pgfqpoint{0.000000in}{0.000000in}}%
\pgfpathlineto{\pgfqpoint{-0.048611in}{0.000000in}}%
\pgfusepath{stroke,fill}%
}%
\begin{pgfscope}%
\pgfsys@transformshift{4.616667in}{2.693029in}%
\pgfsys@useobject{currentmarker}{}%
\end{pgfscope}%
\end{pgfscope}%
\begin{pgfscope}%
\definecolor{textcolor}{rgb}{0.000000,0.000000,0.000000}%
\pgfsetstrokecolor{textcolor}%
\pgfsetfillcolor{textcolor}%
\pgftext[x=4.298565in,y=2.640268in,left,base]{\color{textcolor}\sffamily\fontsize{10.000000}{12.000000}\selectfont 1.0}%
\end{pgfscope}%
\begin{pgfscope}%
\pgfsetbuttcap%
\pgfsetroundjoin%
\definecolor{currentfill}{rgb}{0.000000,0.000000,0.000000}%
\pgfsetfillcolor{currentfill}%
\pgfsetlinewidth{0.803000pt}%
\definecolor{currentstroke}{rgb}{0.000000,0.000000,0.000000}%
\pgfsetstrokecolor{currentstroke}%
\pgfsetdash{}{0pt}%
\pgfsys@defobject{currentmarker}{\pgfqpoint{-0.048611in}{0.000000in}}{\pgfqpoint{0.000000in}{0.000000in}}{%
\pgfpathmoveto{\pgfqpoint{0.000000in}{0.000000in}}%
\pgfpathlineto{\pgfqpoint{-0.048611in}{0.000000in}}%
\pgfusepath{stroke,fill}%
}%
\begin{pgfscope}%
\pgfsys@transformshift{4.616667in}{3.086913in}%
\pgfsys@useobject{currentmarker}{}%
\end{pgfscope}%
\end{pgfscope}%
\begin{pgfscope}%
\definecolor{textcolor}{rgb}{0.000000,0.000000,0.000000}%
\pgfsetstrokecolor{textcolor}%
\pgfsetfillcolor{textcolor}%
\pgftext[x=4.298565in,y=3.034151in,left,base]{\color{textcolor}\sffamily\fontsize{10.000000}{12.000000}\selectfont 1.2}%
\end{pgfscope}%
\begin{pgfscope}%
\pgfsetbuttcap%
\pgfsetroundjoin%
\definecolor{currentfill}{rgb}{0.000000,0.000000,0.000000}%
\pgfsetfillcolor{currentfill}%
\pgfsetlinewidth{0.803000pt}%
\definecolor{currentstroke}{rgb}{0.000000,0.000000,0.000000}%
\pgfsetstrokecolor{currentstroke}%
\pgfsetdash{}{0pt}%
\pgfsys@defobject{currentmarker}{\pgfqpoint{-0.048611in}{0.000000in}}{\pgfqpoint{0.000000in}{0.000000in}}{%
\pgfpathmoveto{\pgfqpoint{0.000000in}{0.000000in}}%
\pgfpathlineto{\pgfqpoint{-0.048611in}{0.000000in}}%
\pgfusepath{stroke,fill}%
}%
\begin{pgfscope}%
\pgfsys@transformshift{4.616667in}{3.480797in}%
\pgfsys@useobject{currentmarker}{}%
\end{pgfscope}%
\end{pgfscope}%
\begin{pgfscope}%
\definecolor{textcolor}{rgb}{0.000000,0.000000,0.000000}%
\pgfsetstrokecolor{textcolor}%
\pgfsetfillcolor{textcolor}%
\pgftext[x=4.298565in,y=3.428035in,left,base]{\color{textcolor}\sffamily\fontsize{10.000000}{12.000000}\selectfont 1.4}%
\end{pgfscope}%
\begin{pgfscope}%
\definecolor{textcolor}{rgb}{0.000000,0.000000,0.000000}%
\pgfsetstrokecolor{textcolor}%
\pgfsetfillcolor{textcolor}%
\pgftext[x=4.243009in,y=2.154167in,,bottom,rotate=90.000000]{\color{textcolor}\sffamily\fontsize{10.000000}{12.000000}\selectfont \(\displaystyle U\)}%
\end{pgfscope}%
\begin{pgfscope}%
\pgfpathrectangle{\pgfqpoint{4.616667in}{0.580556in}}{\pgfqpoint{3.184722in}{3.147222in}}%
\pgfusepath{clip}%
\pgfsetrectcap%
\pgfsetroundjoin%
\pgfsetlinewidth{1.505625pt}%
\definecolor{currentstroke}{rgb}{0.121569,0.466667,0.705882}%
\pgfsetstrokecolor{currentstroke}%
\pgfsetdash{}{0pt}%
\pgfpathmoveto{\pgfqpoint{4.761427in}{2.693029in}}%
\pgfpathlineto{\pgfqpoint{6.566341in}{2.693034in}}%
\pgfpathlineto{\pgfqpoint{6.567571in}{2.684894in}}%
\pgfpathlineto{\pgfqpoint{6.567499in}{2.701563in}}%
\pgfpathlineto{\pgfqpoint{6.567644in}{2.688248in}}%
\pgfpathlineto{\pgfqpoint{6.568802in}{2.626296in}}%
\pgfpathlineto{\pgfqpoint{6.568947in}{2.752299in}}%
\pgfpathlineto{\pgfqpoint{6.570249in}{2.135242in}}%
\pgfpathlineto{\pgfqpoint{6.570105in}{3.109961in}}%
\pgfpathlineto{\pgfqpoint{6.570322in}{3.000437in}}%
\pgfpathlineto{\pgfqpoint{6.570611in}{3.584722in}}%
\pgfpathlineto{\pgfqpoint{6.570467in}{1.932789in}}%
\pgfpathlineto{\pgfqpoint{6.570684in}{2.088285in}}%
\pgfpathlineto{\pgfqpoint{6.570829in}{3.547630in}}%
\pgfpathlineto{\pgfqpoint{6.572421in}{0.723611in}}%
\pgfpathlineto{\pgfqpoint{7.656629in}{0.723611in}}%
\pgfpathlineto{\pgfqpoint{7.656629in}{0.723611in}}%
\pgfusepath{stroke}%
\end{pgfscope}%
\begin{pgfscope}%
\pgfpathrectangle{\pgfqpoint{4.616667in}{0.580556in}}{\pgfqpoint{3.184722in}{3.147222in}}%
\pgfusepath{clip}%
\pgfsetrectcap%
\pgfsetroundjoin%
\pgfsetlinewidth{1.505625pt}%
\definecolor{currentstroke}{rgb}{1.000000,0.498039,0.054902}%
\pgfsetstrokecolor{currentstroke}%
\pgfsetdash{}{0pt}%
\pgfpathmoveto{\pgfqpoint{4.761427in}{2.693029in}}%
\pgfpathlineto{\pgfqpoint{6.566558in}{2.693881in}}%
\pgfpathlineto{\pgfqpoint{6.567716in}{2.701167in}}%
\pgfpathlineto{\pgfqpoint{6.567861in}{2.684404in}}%
\pgfpathlineto{\pgfqpoint{6.569164in}{2.760151in}}%
\pgfpathlineto{\pgfqpoint{6.569019in}{2.642990in}}%
\pgfpathlineto{\pgfqpoint{6.569236in}{2.655916in}}%
\pgfpathlineto{\pgfqpoint{6.569888in}{2.877195in}}%
\pgfpathlineto{\pgfqpoint{6.570032in}{2.521709in}}%
\pgfpathlineto{\pgfqpoint{6.570611in}{3.182632in}}%
\pgfpathlineto{\pgfqpoint{6.571190in}{0.723611in}}%
\pgfpathlineto{\pgfqpoint{7.656629in}{0.723611in}}%
\pgfpathlineto{\pgfqpoint{7.656629in}{0.723611in}}%
\pgfusepath{stroke}%
\end{pgfscope}%
\begin{pgfscope}%
\pgfpathrectangle{\pgfqpoint{4.616667in}{0.580556in}}{\pgfqpoint{3.184722in}{3.147222in}}%
\pgfusepath{clip}%
\pgfsetrectcap%
\pgfsetroundjoin%
\pgfsetlinewidth{1.505625pt}%
\definecolor{currentstroke}{rgb}{0.172549,0.627451,0.172549}%
\pgfsetstrokecolor{currentstroke}%
\pgfsetdash{}{0pt}%
\pgfpathmoveto{\pgfqpoint{4.761427in}{2.693029in}}%
\pgfpathlineto{\pgfqpoint{6.570611in}{2.692375in}}%
\pgfpathlineto{\pgfqpoint{6.570756in}{2.673234in}}%
\pgfpathlineto{\pgfqpoint{6.573941in}{0.723611in}}%
\pgfpathlineto{\pgfqpoint{7.656629in}{0.723611in}}%
\pgfpathlineto{\pgfqpoint{7.656629in}{0.723611in}}%
\pgfusepath{stroke}%
\end{pgfscope}%
\begin{pgfscope}%
\pgfpathrectangle{\pgfqpoint{4.616667in}{0.580556in}}{\pgfqpoint{3.184722in}{3.147222in}}%
\pgfusepath{clip}%
\pgfsetrectcap%
\pgfsetroundjoin%
\pgfsetlinewidth{1.505625pt}%
\definecolor{currentstroke}{rgb}{0.839216,0.152941,0.156863}%
\pgfsetstrokecolor{currentstroke}%
\pgfsetdash{}{0pt}%
\pgfpathmoveto{\pgfqpoint{4.761427in}{2.693029in}}%
\pgfpathlineto{\pgfqpoint{6.208992in}{2.693029in}}%
\pgfpathlineto{\pgfqpoint{6.210584in}{0.723611in}}%
\pgfpathlineto{\pgfqpoint{7.656629in}{0.723611in}}%
\pgfpathlineto{\pgfqpoint{7.656629in}{0.723611in}}%
\pgfusepath{stroke}%
\end{pgfscope}%
\begin{pgfscope}%
\pgfpathrectangle{\pgfqpoint{4.616667in}{0.580556in}}{\pgfqpoint{3.184722in}{3.147222in}}%
\pgfusepath{clip}%
\pgfsetrectcap%
\pgfsetroundjoin%
\pgfsetlinewidth{0.501875pt}%
\definecolor{currentstroke}{rgb}{0.000000,0.000000,0.000000}%
\pgfsetstrokecolor{currentstroke}%
\pgfsetdash{}{0pt}%
\pgfpathmoveto{\pgfqpoint{4.761427in}{2.693029in}}%
\pgfpathlineto{\pgfqpoint{6.570901in}{2.693029in}}%
\pgfpathlineto{\pgfqpoint{6.572493in}{0.723611in}}%
\pgfpathlineto{\pgfqpoint{7.656629in}{0.723611in}}%
\pgfpathlineto{\pgfqpoint{7.656629in}{0.723611in}}%
\pgfusepath{stroke}%
\end{pgfscope}%
\begin{pgfscope}%
\pgfsetrectcap%
\pgfsetmiterjoin%
\pgfsetlinewidth{0.803000pt}%
\definecolor{currentstroke}{rgb}{0.000000,0.000000,0.000000}%
\pgfsetstrokecolor{currentstroke}%
\pgfsetdash{}{0pt}%
\pgfpathmoveto{\pgfqpoint{4.616667in}{0.580556in}}%
\pgfpathlineto{\pgfqpoint{4.616667in}{3.727778in}}%
\pgfusepath{stroke}%
\end{pgfscope}%
\begin{pgfscope}%
\pgfsetrectcap%
\pgfsetmiterjoin%
\pgfsetlinewidth{0.803000pt}%
\definecolor{currentstroke}{rgb}{0.000000,0.000000,0.000000}%
\pgfsetstrokecolor{currentstroke}%
\pgfsetdash{}{0pt}%
\pgfpathmoveto{\pgfqpoint{7.801389in}{0.580556in}}%
\pgfpathlineto{\pgfqpoint{7.801389in}{3.727778in}}%
\pgfusepath{stroke}%
\end{pgfscope}%
\begin{pgfscope}%
\pgfsetrectcap%
\pgfsetmiterjoin%
\pgfsetlinewidth{0.803000pt}%
\definecolor{currentstroke}{rgb}{0.000000,0.000000,0.000000}%
\pgfsetstrokecolor{currentstroke}%
\pgfsetdash{}{0pt}%
\pgfpathmoveto{\pgfqpoint{4.616667in}{0.580556in}}%
\pgfpathlineto{\pgfqpoint{7.801389in}{0.580556in}}%
\pgfusepath{stroke}%
\end{pgfscope}%
\begin{pgfscope}%
\pgfsetrectcap%
\pgfsetmiterjoin%
\pgfsetlinewidth{0.803000pt}%
\definecolor{currentstroke}{rgb}{0.000000,0.000000,0.000000}%
\pgfsetstrokecolor{currentstroke}%
\pgfsetdash{}{0pt}%
\pgfpathmoveto{\pgfqpoint{4.616667in}{3.727778in}}%
\pgfpathlineto{\pgfqpoint{7.801389in}{3.727778in}}%
\pgfusepath{stroke}%
\end{pgfscope}%
\begin{pgfscope}%
\definecolor{textcolor}{rgb}{0.000000,0.000000,0.000000}%
\pgfsetstrokecolor{textcolor}%
\pgfsetfillcolor{textcolor}%
\pgftext[x=6.209028in,y=3.811111in,,base]{\color{textcolor}\sffamily\fontsize{12.000000}{14.400000}\selectfont \(\displaystyle  t = 1.0 \)}%
\end{pgfscope}%
\begin{pgfscope}%
\pgfsetbuttcap%
\pgfsetmiterjoin%
\definecolor{currentfill}{rgb}{1.000000,1.000000,1.000000}%
\pgfsetfillcolor{currentfill}%
\pgfsetlinewidth{0.000000pt}%
\definecolor{currentstroke}{rgb}{0.000000,0.000000,0.000000}%
\pgfsetstrokecolor{currentstroke}%
\pgfsetstrokeopacity{0.000000}%
\pgfsetdash{}{0pt}%
\pgfpathmoveto{\pgfqpoint{0.691667in}{8.480556in}}%
\pgfpathlineto{\pgfqpoint{3.876389in}{8.480556in}}%
\pgfpathlineto{\pgfqpoint{3.876389in}{11.627778in}}%
\pgfpathlineto{\pgfqpoint{0.691667in}{11.627778in}}%
\pgfpathclose%
\pgfusepath{fill}%
\end{pgfscope}%
\begin{pgfscope}%
\pgfsetbuttcap%
\pgfsetroundjoin%
\definecolor{currentfill}{rgb}{0.000000,0.000000,0.000000}%
\pgfsetfillcolor{currentfill}%
\pgfsetlinewidth{0.803000pt}%
\definecolor{currentstroke}{rgb}{0.000000,0.000000,0.000000}%
\pgfsetstrokecolor{currentstroke}%
\pgfsetdash{}{0pt}%
\pgfsys@defobject{currentmarker}{\pgfqpoint{0.000000in}{-0.048611in}}{\pgfqpoint{0.000000in}{0.000000in}}{%
\pgfpathmoveto{\pgfqpoint{0.000000in}{0.000000in}}%
\pgfpathlineto{\pgfqpoint{0.000000in}{-0.048611in}}%
\pgfusepath{stroke,fill}%
}%
\begin{pgfscope}%
\pgfsys@transformshift{0.836391in}{8.480556in}%
\pgfsys@useobject{currentmarker}{}%
\end{pgfscope}%
\end{pgfscope}%
\begin{pgfscope}%
\definecolor{textcolor}{rgb}{0.000000,0.000000,0.000000}%
\pgfsetstrokecolor{textcolor}%
\pgfsetfillcolor{textcolor}%
\pgftext[x=0.836391in,y=8.383333in,,top]{\color{textcolor}\sffamily\fontsize{10.000000}{12.000000}\selectfont −2}%
\end{pgfscope}%
\begin{pgfscope}%
\pgfsetbuttcap%
\pgfsetroundjoin%
\definecolor{currentfill}{rgb}{0.000000,0.000000,0.000000}%
\pgfsetfillcolor{currentfill}%
\pgfsetlinewidth{0.803000pt}%
\definecolor{currentstroke}{rgb}{0.000000,0.000000,0.000000}%
\pgfsetstrokecolor{currentstroke}%
\pgfsetdash{}{0pt}%
\pgfsys@defobject{currentmarker}{\pgfqpoint{0.000000in}{-0.048611in}}{\pgfqpoint{0.000000in}{0.000000in}}{%
\pgfpathmoveto{\pgfqpoint{0.000000in}{0.000000in}}%
\pgfpathlineto{\pgfqpoint{0.000000in}{-0.048611in}}%
\pgfusepath{stroke,fill}%
}%
\begin{pgfscope}%
\pgfsys@transformshift{1.560209in}{8.480556in}%
\pgfsys@useobject{currentmarker}{}%
\end{pgfscope}%
\end{pgfscope}%
\begin{pgfscope}%
\definecolor{textcolor}{rgb}{0.000000,0.000000,0.000000}%
\pgfsetstrokecolor{textcolor}%
\pgfsetfillcolor{textcolor}%
\pgftext[x=1.560209in,y=8.383333in,,top]{\color{textcolor}\sffamily\fontsize{10.000000}{12.000000}\selectfont −1}%
\end{pgfscope}%
\begin{pgfscope}%
\pgfsetbuttcap%
\pgfsetroundjoin%
\definecolor{currentfill}{rgb}{0.000000,0.000000,0.000000}%
\pgfsetfillcolor{currentfill}%
\pgfsetlinewidth{0.803000pt}%
\definecolor{currentstroke}{rgb}{0.000000,0.000000,0.000000}%
\pgfsetstrokecolor{currentstroke}%
\pgfsetdash{}{0pt}%
\pgfsys@defobject{currentmarker}{\pgfqpoint{0.000000in}{-0.048611in}}{\pgfqpoint{0.000000in}{0.000000in}}{%
\pgfpathmoveto{\pgfqpoint{0.000000in}{0.000000in}}%
\pgfpathlineto{\pgfqpoint{0.000000in}{-0.048611in}}%
\pgfusepath{stroke,fill}%
}%
\begin{pgfscope}%
\pgfsys@transformshift{2.284028in}{8.480556in}%
\pgfsys@useobject{currentmarker}{}%
\end{pgfscope}%
\end{pgfscope}%
\begin{pgfscope}%
\definecolor{textcolor}{rgb}{0.000000,0.000000,0.000000}%
\pgfsetstrokecolor{textcolor}%
\pgfsetfillcolor{textcolor}%
\pgftext[x=2.284028in,y=8.383333in,,top]{\color{textcolor}\sffamily\fontsize{10.000000}{12.000000}\selectfont 0}%
\end{pgfscope}%
\begin{pgfscope}%
\pgfsetbuttcap%
\pgfsetroundjoin%
\definecolor{currentfill}{rgb}{0.000000,0.000000,0.000000}%
\pgfsetfillcolor{currentfill}%
\pgfsetlinewidth{0.803000pt}%
\definecolor{currentstroke}{rgb}{0.000000,0.000000,0.000000}%
\pgfsetstrokecolor{currentstroke}%
\pgfsetdash{}{0pt}%
\pgfsys@defobject{currentmarker}{\pgfqpoint{0.000000in}{-0.048611in}}{\pgfqpoint{0.000000in}{0.000000in}}{%
\pgfpathmoveto{\pgfqpoint{0.000000in}{0.000000in}}%
\pgfpathlineto{\pgfqpoint{0.000000in}{-0.048611in}}%
\pgfusepath{stroke,fill}%
}%
\begin{pgfscope}%
\pgfsys@transformshift{3.007846in}{8.480556in}%
\pgfsys@useobject{currentmarker}{}%
\end{pgfscope}%
\end{pgfscope}%
\begin{pgfscope}%
\definecolor{textcolor}{rgb}{0.000000,0.000000,0.000000}%
\pgfsetstrokecolor{textcolor}%
\pgfsetfillcolor{textcolor}%
\pgftext[x=3.007846in,y=8.383333in,,top]{\color{textcolor}\sffamily\fontsize{10.000000}{12.000000}\selectfont 1}%
\end{pgfscope}%
\begin{pgfscope}%
\pgfsetbuttcap%
\pgfsetroundjoin%
\definecolor{currentfill}{rgb}{0.000000,0.000000,0.000000}%
\pgfsetfillcolor{currentfill}%
\pgfsetlinewidth{0.803000pt}%
\definecolor{currentstroke}{rgb}{0.000000,0.000000,0.000000}%
\pgfsetstrokecolor{currentstroke}%
\pgfsetdash{}{0pt}%
\pgfsys@defobject{currentmarker}{\pgfqpoint{0.000000in}{-0.048611in}}{\pgfqpoint{0.000000in}{0.000000in}}{%
\pgfpathmoveto{\pgfqpoint{0.000000in}{0.000000in}}%
\pgfpathlineto{\pgfqpoint{0.000000in}{-0.048611in}}%
\pgfusepath{stroke,fill}%
}%
\begin{pgfscope}%
\pgfsys@transformshift{3.731665in}{8.480556in}%
\pgfsys@useobject{currentmarker}{}%
\end{pgfscope}%
\end{pgfscope}%
\begin{pgfscope}%
\definecolor{textcolor}{rgb}{0.000000,0.000000,0.000000}%
\pgfsetstrokecolor{textcolor}%
\pgfsetfillcolor{textcolor}%
\pgftext[x=3.731665in,y=8.383333in,,top]{\color{textcolor}\sffamily\fontsize{10.000000}{12.000000}\selectfont 2}%
\end{pgfscope}%
\begin{pgfscope}%
\definecolor{textcolor}{rgb}{0.000000,0.000000,0.000000}%
\pgfsetstrokecolor{textcolor}%
\pgfsetfillcolor{textcolor}%
\pgftext[x=2.284028in,y=8.193365in,,top]{\color{textcolor}\sffamily\fontsize{10.000000}{12.000000}\selectfont \(\displaystyle x\)}%
\end{pgfscope}%
\begin{pgfscope}%
\pgfsetbuttcap%
\pgfsetroundjoin%
\definecolor{currentfill}{rgb}{0.000000,0.000000,0.000000}%
\pgfsetfillcolor{currentfill}%
\pgfsetlinewidth{0.803000pt}%
\definecolor{currentstroke}{rgb}{0.000000,0.000000,0.000000}%
\pgfsetstrokecolor{currentstroke}%
\pgfsetdash{}{0pt}%
\pgfsys@defobject{currentmarker}{\pgfqpoint{-0.048611in}{0.000000in}}{\pgfqpoint{0.000000in}{0.000000in}}{%
\pgfpathmoveto{\pgfqpoint{0.000000in}{0.000000in}}%
\pgfpathlineto{\pgfqpoint{-0.048611in}{0.000000in}}%
\pgfusepath{stroke,fill}%
}%
\begin{pgfscope}%
\pgfsys@transformshift{0.691667in}{8.623611in}%
\pgfsys@useobject{currentmarker}{}%
\end{pgfscope}%
\end{pgfscope}%
\begin{pgfscope}%
\definecolor{textcolor}{rgb}{0.000000,0.000000,0.000000}%
\pgfsetstrokecolor{textcolor}%
\pgfsetfillcolor{textcolor}%
\pgftext[x=0.373565in,y=8.570850in,left,base]{\color{textcolor}\sffamily\fontsize{10.000000}{12.000000}\selectfont 0.0}%
\end{pgfscope}%
\begin{pgfscope}%
\pgfsetbuttcap%
\pgfsetroundjoin%
\definecolor{currentfill}{rgb}{0.000000,0.000000,0.000000}%
\pgfsetfillcolor{currentfill}%
\pgfsetlinewidth{0.803000pt}%
\definecolor{currentstroke}{rgb}{0.000000,0.000000,0.000000}%
\pgfsetstrokecolor{currentstroke}%
\pgfsetdash{}{0pt}%
\pgfsys@defobject{currentmarker}{\pgfqpoint{-0.048611in}{0.000000in}}{\pgfqpoint{0.000000in}{0.000000in}}{%
\pgfpathmoveto{\pgfqpoint{0.000000in}{0.000000in}}%
\pgfpathlineto{\pgfqpoint{-0.048611in}{0.000000in}}%
\pgfusepath{stroke,fill}%
}%
\begin{pgfscope}%
\pgfsys@transformshift{0.691667in}{9.195833in}%
\pgfsys@useobject{currentmarker}{}%
\end{pgfscope}%
\end{pgfscope}%
\begin{pgfscope}%
\definecolor{textcolor}{rgb}{0.000000,0.000000,0.000000}%
\pgfsetstrokecolor{textcolor}%
\pgfsetfillcolor{textcolor}%
\pgftext[x=0.373565in,y=9.143072in,left,base]{\color{textcolor}\sffamily\fontsize{10.000000}{12.000000}\selectfont 0.2}%
\end{pgfscope}%
\begin{pgfscope}%
\pgfsetbuttcap%
\pgfsetroundjoin%
\definecolor{currentfill}{rgb}{0.000000,0.000000,0.000000}%
\pgfsetfillcolor{currentfill}%
\pgfsetlinewidth{0.803000pt}%
\definecolor{currentstroke}{rgb}{0.000000,0.000000,0.000000}%
\pgfsetstrokecolor{currentstroke}%
\pgfsetdash{}{0pt}%
\pgfsys@defobject{currentmarker}{\pgfqpoint{-0.048611in}{0.000000in}}{\pgfqpoint{0.000000in}{0.000000in}}{%
\pgfpathmoveto{\pgfqpoint{0.000000in}{0.000000in}}%
\pgfpathlineto{\pgfqpoint{-0.048611in}{0.000000in}}%
\pgfusepath{stroke,fill}%
}%
\begin{pgfscope}%
\pgfsys@transformshift{0.691667in}{9.768056in}%
\pgfsys@useobject{currentmarker}{}%
\end{pgfscope}%
\end{pgfscope}%
\begin{pgfscope}%
\definecolor{textcolor}{rgb}{0.000000,0.000000,0.000000}%
\pgfsetstrokecolor{textcolor}%
\pgfsetfillcolor{textcolor}%
\pgftext[x=0.373565in,y=9.715294in,left,base]{\color{textcolor}\sffamily\fontsize{10.000000}{12.000000}\selectfont 0.4}%
\end{pgfscope}%
\begin{pgfscope}%
\pgfsetbuttcap%
\pgfsetroundjoin%
\definecolor{currentfill}{rgb}{0.000000,0.000000,0.000000}%
\pgfsetfillcolor{currentfill}%
\pgfsetlinewidth{0.803000pt}%
\definecolor{currentstroke}{rgb}{0.000000,0.000000,0.000000}%
\pgfsetstrokecolor{currentstroke}%
\pgfsetdash{}{0pt}%
\pgfsys@defobject{currentmarker}{\pgfqpoint{-0.048611in}{0.000000in}}{\pgfqpoint{0.000000in}{0.000000in}}{%
\pgfpathmoveto{\pgfqpoint{0.000000in}{0.000000in}}%
\pgfpathlineto{\pgfqpoint{-0.048611in}{0.000000in}}%
\pgfusepath{stroke,fill}%
}%
\begin{pgfscope}%
\pgfsys@transformshift{0.691667in}{10.340278in}%
\pgfsys@useobject{currentmarker}{}%
\end{pgfscope}%
\end{pgfscope}%
\begin{pgfscope}%
\definecolor{textcolor}{rgb}{0.000000,0.000000,0.000000}%
\pgfsetstrokecolor{textcolor}%
\pgfsetfillcolor{textcolor}%
\pgftext[x=0.373565in,y=10.287516in,left,base]{\color{textcolor}\sffamily\fontsize{10.000000}{12.000000}\selectfont 0.6}%
\end{pgfscope}%
\begin{pgfscope}%
\pgfsetbuttcap%
\pgfsetroundjoin%
\definecolor{currentfill}{rgb}{0.000000,0.000000,0.000000}%
\pgfsetfillcolor{currentfill}%
\pgfsetlinewidth{0.803000pt}%
\definecolor{currentstroke}{rgb}{0.000000,0.000000,0.000000}%
\pgfsetstrokecolor{currentstroke}%
\pgfsetdash{}{0pt}%
\pgfsys@defobject{currentmarker}{\pgfqpoint{-0.048611in}{0.000000in}}{\pgfqpoint{0.000000in}{0.000000in}}{%
\pgfpathmoveto{\pgfqpoint{0.000000in}{0.000000in}}%
\pgfpathlineto{\pgfqpoint{-0.048611in}{0.000000in}}%
\pgfusepath{stroke,fill}%
}%
\begin{pgfscope}%
\pgfsys@transformshift{0.691667in}{10.912500in}%
\pgfsys@useobject{currentmarker}{}%
\end{pgfscope}%
\end{pgfscope}%
\begin{pgfscope}%
\definecolor{textcolor}{rgb}{0.000000,0.000000,0.000000}%
\pgfsetstrokecolor{textcolor}%
\pgfsetfillcolor{textcolor}%
\pgftext[x=0.373565in,y=10.859738in,left,base]{\color{textcolor}\sffamily\fontsize{10.000000}{12.000000}\selectfont 0.8}%
\end{pgfscope}%
\begin{pgfscope}%
\pgfsetbuttcap%
\pgfsetroundjoin%
\definecolor{currentfill}{rgb}{0.000000,0.000000,0.000000}%
\pgfsetfillcolor{currentfill}%
\pgfsetlinewidth{0.803000pt}%
\definecolor{currentstroke}{rgb}{0.000000,0.000000,0.000000}%
\pgfsetstrokecolor{currentstroke}%
\pgfsetdash{}{0pt}%
\pgfsys@defobject{currentmarker}{\pgfqpoint{-0.048611in}{0.000000in}}{\pgfqpoint{0.000000in}{0.000000in}}{%
\pgfpathmoveto{\pgfqpoint{0.000000in}{0.000000in}}%
\pgfpathlineto{\pgfqpoint{-0.048611in}{0.000000in}}%
\pgfusepath{stroke,fill}%
}%
\begin{pgfscope}%
\pgfsys@transformshift{0.691667in}{11.484722in}%
\pgfsys@useobject{currentmarker}{}%
\end{pgfscope}%
\end{pgfscope}%
\begin{pgfscope}%
\definecolor{textcolor}{rgb}{0.000000,0.000000,0.000000}%
\pgfsetstrokecolor{textcolor}%
\pgfsetfillcolor{textcolor}%
\pgftext[x=0.373565in,y=11.431961in,left,base]{\color{textcolor}\sffamily\fontsize{10.000000}{12.000000}\selectfont 1.0}%
\end{pgfscope}%
\begin{pgfscope}%
\definecolor{textcolor}{rgb}{0.000000,0.000000,0.000000}%
\pgfsetstrokecolor{textcolor}%
\pgfsetfillcolor{textcolor}%
\pgftext[x=0.318009in,y=10.054167in,,bottom,rotate=90.000000]{\color{textcolor}\sffamily\fontsize{10.000000}{12.000000}\selectfont \(\displaystyle U\)}%
\end{pgfscope}%
\begin{pgfscope}%
\pgfpathrectangle{\pgfqpoint{0.691667in}{8.480556in}}{\pgfqpoint{3.184722in}{3.147222in}}%
\pgfusepath{clip}%
\pgfsetrectcap%
\pgfsetroundjoin%
\pgfsetlinewidth{0.501875pt}%
\definecolor{currentstroke}{rgb}{0.000000,0.000000,0.000000}%
\pgfsetstrokecolor{currentstroke}%
\pgfsetdash{}{0pt}%
\pgfpathmoveto{\pgfqpoint{0.836427in}{11.484722in}}%
\pgfpathlineto{\pgfqpoint{2.283992in}{11.484722in}}%
\pgfpathlineto{\pgfqpoint{2.285584in}{8.623611in}}%
\pgfpathlineto{\pgfqpoint{3.731629in}{8.623611in}}%
\pgfpathlineto{\pgfqpoint{3.731629in}{8.623611in}}%
\pgfusepath{stroke}%
\end{pgfscope}%
\begin{pgfscope}%
\pgfsetrectcap%
\pgfsetmiterjoin%
\pgfsetlinewidth{0.803000pt}%
\definecolor{currentstroke}{rgb}{0.000000,0.000000,0.000000}%
\pgfsetstrokecolor{currentstroke}%
\pgfsetdash{}{0pt}%
\pgfpathmoveto{\pgfqpoint{0.691667in}{8.480556in}}%
\pgfpathlineto{\pgfqpoint{0.691667in}{11.627778in}}%
\pgfusepath{stroke}%
\end{pgfscope}%
\begin{pgfscope}%
\pgfsetrectcap%
\pgfsetmiterjoin%
\pgfsetlinewidth{0.803000pt}%
\definecolor{currentstroke}{rgb}{0.000000,0.000000,0.000000}%
\pgfsetstrokecolor{currentstroke}%
\pgfsetdash{}{0pt}%
\pgfpathmoveto{\pgfqpoint{3.876389in}{8.480556in}}%
\pgfpathlineto{\pgfqpoint{3.876389in}{11.627778in}}%
\pgfusepath{stroke}%
\end{pgfscope}%
\begin{pgfscope}%
\pgfsetrectcap%
\pgfsetmiterjoin%
\pgfsetlinewidth{0.803000pt}%
\definecolor{currentstroke}{rgb}{0.000000,0.000000,0.000000}%
\pgfsetstrokecolor{currentstroke}%
\pgfsetdash{}{0pt}%
\pgfpathmoveto{\pgfqpoint{0.691667in}{8.480556in}}%
\pgfpathlineto{\pgfqpoint{3.876389in}{8.480556in}}%
\pgfusepath{stroke}%
\end{pgfscope}%
\begin{pgfscope}%
\pgfsetrectcap%
\pgfsetmiterjoin%
\pgfsetlinewidth{0.803000pt}%
\definecolor{currentstroke}{rgb}{0.000000,0.000000,0.000000}%
\pgfsetstrokecolor{currentstroke}%
\pgfsetdash{}{0pt}%
\pgfpathmoveto{\pgfqpoint{0.691667in}{11.627778in}}%
\pgfpathlineto{\pgfqpoint{3.876389in}{11.627778in}}%
\pgfusepath{stroke}%
\end{pgfscope}%
\begin{pgfscope}%
\definecolor{textcolor}{rgb}{0.000000,0.000000,0.000000}%
\pgfsetstrokecolor{textcolor}%
\pgfsetfillcolor{textcolor}%
\pgftext[x=2.284028in,y=11.711111in,,base]{\color{textcolor}\sffamily\fontsize{12.000000}{14.400000}\selectfont \(\displaystyle  t = 0.0 \)}%
\end{pgfscope}%
\begin{pgfscope}%
\pgfsetbuttcap%
\pgfsetmiterjoin%
\definecolor{currentfill}{rgb}{1.000000,1.000000,1.000000}%
\pgfsetfillcolor{currentfill}%
\pgfsetfillopacity{0.800000}%
\pgfsetlinewidth{1.003750pt}%
\definecolor{currentstroke}{rgb}{0.800000,0.800000,0.800000}%
\pgfsetstrokecolor{currentstroke}%
\pgfsetstrokeopacity{0.800000}%
\pgfsetdash{}{0pt}%
\pgfpathmoveto{\pgfqpoint{1.566737in}{9.523690in}}%
\pgfpathlineto{\pgfqpoint{3.779167in}{9.523690in}}%
\pgfpathquadraticcurveto{\pgfqpoint{3.806944in}{9.523690in}}{\pgfqpoint{3.806944in}{9.551468in}}%
\pgfpathlineto{\pgfqpoint{3.806944in}{10.556865in}}%
\pgfpathquadraticcurveto{\pgfqpoint{3.806944in}{10.584643in}}{\pgfqpoint{3.779167in}{10.584643in}}%
\pgfpathlineto{\pgfqpoint{1.566737in}{10.584643in}}%
\pgfpathquadraticcurveto{\pgfqpoint{1.538959in}{10.584643in}}{\pgfqpoint{1.538959in}{10.556865in}}%
\pgfpathlineto{\pgfqpoint{1.538959in}{9.551468in}}%
\pgfpathquadraticcurveto{\pgfqpoint{1.538959in}{9.523690in}}{\pgfqpoint{1.566737in}{9.523690in}}%
\pgfpathclose%
\pgfusepath{stroke,fill}%
\end{pgfscope}%
\begin{pgfscope}%
\pgfsetrectcap%
\pgfsetroundjoin%
\pgfsetlinewidth{1.505625pt}%
\definecolor{currentstroke}{rgb}{0.121569,0.466667,0.705882}%
\pgfsetstrokecolor{currentstroke}%
\pgfsetdash{}{0pt}%
\pgfpathmoveto{\pgfqpoint{1.594515in}{10.472176in}}%
\pgfpathlineto{\pgfqpoint{1.872293in}{10.472176in}}%
\pgfusepath{stroke}%
\end{pgfscope}%
\begin{pgfscope}%
\definecolor{textcolor}{rgb}{0.000000,0.000000,0.000000}%
\pgfsetstrokecolor{textcolor}%
\pgfsetfillcolor{textcolor}%
\pgftext[x=1.983404in,y=10.423564in,left,base]{\color{textcolor}\sffamily\fontsize{10.000000}{12.000000}\selectfont Richtmyer}%
\end{pgfscope}%
\begin{pgfscope}%
\pgfsetrectcap%
\pgfsetroundjoin%
\pgfsetlinewidth{1.505625pt}%
\definecolor{currentstroke}{rgb}{1.000000,0.498039,0.054902}%
\pgfsetstrokecolor{currentstroke}%
\pgfsetdash{}{0pt}%
\pgfpathmoveto{\pgfqpoint{1.594515in}{10.268318in}}%
\pgfpathlineto{\pgfqpoint{1.872293in}{10.268318in}}%
\pgfusepath{stroke}%
\end{pgfscope}%
\begin{pgfscope}%
\definecolor{textcolor}{rgb}{0.000000,0.000000,0.000000}%
\pgfsetstrokecolor{textcolor}%
\pgfsetfillcolor{textcolor}%
\pgftext[x=1.983404in,y=10.219707in,left,base]{\color{textcolor}\sffamily\fontsize{10.000000}{12.000000}\selectfont Lax--Wendroff}%
\end{pgfscope}%
\begin{pgfscope}%
\pgfsetrectcap%
\pgfsetroundjoin%
\pgfsetlinewidth{1.505625pt}%
\definecolor{currentstroke}{rgb}{0.172549,0.627451,0.172549}%
\pgfsetstrokecolor{currentstroke}%
\pgfsetdash{}{0pt}%
\pgfpathmoveto{\pgfqpoint{1.594515in}{10.064461in}}%
\pgfpathlineto{\pgfqpoint{1.872293in}{10.064461in}}%
\pgfusepath{stroke}%
\end{pgfscope}%
\begin{pgfscope}%
\definecolor{textcolor}{rgb}{0.000000,0.000000,0.000000}%
\pgfsetstrokecolor{textcolor}%
\pgfsetfillcolor{textcolor}%
\pgftext[x=1.983404in,y=10.015850in,left,base]{\color{textcolor}\sffamily\fontsize{10.000000}{12.000000}\selectfont Conservative upwind}%
\end{pgfscope}%
\begin{pgfscope}%
\pgfsetrectcap%
\pgfsetroundjoin%
\pgfsetlinewidth{1.505625pt}%
\definecolor{currentstroke}{rgb}{0.839216,0.152941,0.156863}%
\pgfsetstrokecolor{currentstroke}%
\pgfsetdash{}{0pt}%
\pgfpathmoveto{\pgfqpoint{1.594515in}{9.860604in}}%
\pgfpathlineto{\pgfqpoint{1.872293in}{9.860604in}}%
\pgfusepath{stroke}%
\end{pgfscope}%
\begin{pgfscope}%
\definecolor{textcolor}{rgb}{0.000000,0.000000,0.000000}%
\pgfsetstrokecolor{textcolor}%
\pgfsetfillcolor{textcolor}%
\pgftext[x=1.983404in,y=9.811993in,left,base]{\color{textcolor}\sffamily\fontsize{10.000000}{12.000000}\selectfont Non-conservative upwind}%
\end{pgfscope}%
\begin{pgfscope}%
\pgfsetrectcap%
\pgfsetroundjoin%
\pgfsetlinewidth{0.501875pt}%
\definecolor{currentstroke}{rgb}{0.000000,0.000000,0.000000}%
\pgfsetstrokecolor{currentstroke}%
\pgfsetdash{}{0pt}%
\pgfpathmoveto{\pgfqpoint{1.594515in}{9.656747in}}%
\pgfpathlineto{\pgfqpoint{1.872293in}{9.656747in}}%
\pgfusepath{stroke}%
\end{pgfscope}%
\begin{pgfscope}%
\definecolor{textcolor}{rgb}{0.000000,0.000000,0.000000}%
\pgfsetstrokecolor{textcolor}%
\pgfsetfillcolor{textcolor}%
\pgftext[x=1.983404in,y=9.608136in,left,base]{\color{textcolor}\sffamily\fontsize{10.000000}{12.000000}\selectfont Analytical}%
\end{pgfscope}%
\end{pgfpicture}%
\makeatother%
\endgroup%
}
\caption{Heat map of solutions using triangle element for the second equation}
\label{Fig:Tri2}
}
{
\footnotesize Top left: $u_h$ with $ N = 256 $; top right: $ u_h - u $ at nodes with $ N = 256 $; bottom left: $ u_h - u $ at $\Omega$ with $ N = 256 $; bottom right: $ u_h - u $ at $\Omega$ with $ N = 16 $.
}
\end{figure}

\begin{figure}[htbp]
{
\centering
\scalebox{0.7}{%% Creator: Matplotlib, PGF backend
%%
%% To include the figure in your LaTeX document, write
%%   \input{<filename>.pgf}
%%
%% Make sure the required packages are loaded in your preamble
%%   \usepackage{pgf}
%%
%% Figures using additional raster images can only be included by \input if
%% they are in the same directory as the main LaTeX file. For loading figures
%% from other directories you can use the `import` package
%%   \usepackage{import}
%% and then include the figures with
%%   \import{<path to file>}{<filename>.pgf}
%%
%% Matplotlib used the following preamble
%%   \usepackage{fontspec}
%%   \setmainfont{DejaVuSerif.ttf}[Path=/home/lzh/anaconda3/envs/numpde/lib/python3.7/site-packages/matplotlib/mpl-data/fonts/ttf/]
%%   \setsansfont{DejaVuSans.ttf}[Path=/home/lzh/anaconda3/envs/numpde/lib/python3.7/site-packages/matplotlib/mpl-data/fonts/ttf/]
%%   \setmonofont{DejaVuSansMono.ttf}[Path=/home/lzh/anaconda3/envs/numpde/lib/python3.7/site-packages/matplotlib/mpl-data/fonts/ttf/]
%%
\begingroup%
\makeatletter%
\begin{pgfpicture}%
\pgfpathrectangle{\pgfpointorigin}{\pgfqpoint{8.000000in}{12.000000in}}%
\pgfusepath{use as bounding box, clip}%
\begin{pgfscope}%
\pgfsetbuttcap%
\pgfsetmiterjoin%
\definecolor{currentfill}{rgb}{1.000000,1.000000,1.000000}%
\pgfsetfillcolor{currentfill}%
\pgfsetlinewidth{0.000000pt}%
\definecolor{currentstroke}{rgb}{1.000000,1.000000,1.000000}%
\pgfsetstrokecolor{currentstroke}%
\pgfsetdash{}{0pt}%
\pgfpathmoveto{\pgfqpoint{0.000000in}{0.000000in}}%
\pgfpathlineto{\pgfqpoint{8.000000in}{0.000000in}}%
\pgfpathlineto{\pgfqpoint{8.000000in}{12.000000in}}%
\pgfpathlineto{\pgfqpoint{0.000000in}{12.000000in}}%
\pgfpathclose%
\pgfusepath{fill}%
\end{pgfscope}%
\begin{pgfscope}%
\pgfsetbuttcap%
\pgfsetmiterjoin%
\definecolor{currentfill}{rgb}{1.000000,1.000000,1.000000}%
\pgfsetfillcolor{currentfill}%
\pgfsetlinewidth{0.000000pt}%
\definecolor{currentstroke}{rgb}{0.000000,0.000000,0.000000}%
\pgfsetstrokecolor{currentstroke}%
\pgfsetstrokeopacity{0.000000}%
\pgfsetdash{}{0pt}%
\pgfpathmoveto{\pgfqpoint{4.660937in}{8.480556in}}%
\pgfpathlineto{\pgfqpoint{7.801389in}{8.480556in}}%
\pgfpathlineto{\pgfqpoint{7.801389in}{11.627778in}}%
\pgfpathlineto{\pgfqpoint{4.660937in}{11.627778in}}%
\pgfpathclose%
\pgfusepath{fill}%
\end{pgfscope}%
\begin{pgfscope}%
\pgfsetbuttcap%
\pgfsetroundjoin%
\definecolor{currentfill}{rgb}{0.000000,0.000000,0.000000}%
\pgfsetfillcolor{currentfill}%
\pgfsetlinewidth{0.803000pt}%
\definecolor{currentstroke}{rgb}{0.000000,0.000000,0.000000}%
\pgfsetstrokecolor{currentstroke}%
\pgfsetdash{}{0pt}%
\pgfsys@defobject{currentmarker}{\pgfqpoint{0.000000in}{-0.048611in}}{\pgfqpoint{0.000000in}{0.000000in}}{%
\pgfpathmoveto{\pgfqpoint{0.000000in}{0.000000in}}%
\pgfpathlineto{\pgfqpoint{0.000000in}{-0.048611in}}%
\pgfusepath{stroke,fill}%
}%
\begin{pgfscope}%
\pgfsys@transformshift{4.779491in}{8.480556in}%
\pgfsys@useobject{currentmarker}{}%
\end{pgfscope}%
\end{pgfscope}%
\begin{pgfscope}%
\definecolor{textcolor}{rgb}{0.000000,0.000000,0.000000}%
\pgfsetstrokecolor{textcolor}%
\pgfsetfillcolor{textcolor}%
\pgftext[x=4.779491in,y=8.383333in,,top]{\color{textcolor}\sffamily\fontsize{10.000000}{12.000000}\selectfont −3}%
\end{pgfscope}%
\begin{pgfscope}%
\pgfsetbuttcap%
\pgfsetroundjoin%
\definecolor{currentfill}{rgb}{0.000000,0.000000,0.000000}%
\pgfsetfillcolor{currentfill}%
\pgfsetlinewidth{0.803000pt}%
\definecolor{currentstroke}{rgb}{0.000000,0.000000,0.000000}%
\pgfsetstrokecolor{currentstroke}%
\pgfsetdash{}{0pt}%
\pgfsys@defobject{currentmarker}{\pgfqpoint{0.000000in}{-0.048611in}}{\pgfqpoint{0.000000in}{0.000000in}}{%
\pgfpathmoveto{\pgfqpoint{0.000000in}{0.000000in}}%
\pgfpathlineto{\pgfqpoint{0.000000in}{-0.048611in}}%
\pgfusepath{stroke,fill}%
}%
\begin{pgfscope}%
\pgfsys@transformshift{5.263382in}{8.480556in}%
\pgfsys@useobject{currentmarker}{}%
\end{pgfscope}%
\end{pgfscope}%
\begin{pgfscope}%
\definecolor{textcolor}{rgb}{0.000000,0.000000,0.000000}%
\pgfsetstrokecolor{textcolor}%
\pgfsetfillcolor{textcolor}%
\pgftext[x=5.263382in,y=8.383333in,,top]{\color{textcolor}\sffamily\fontsize{10.000000}{12.000000}\selectfont −2}%
\end{pgfscope}%
\begin{pgfscope}%
\pgfsetbuttcap%
\pgfsetroundjoin%
\definecolor{currentfill}{rgb}{0.000000,0.000000,0.000000}%
\pgfsetfillcolor{currentfill}%
\pgfsetlinewidth{0.803000pt}%
\definecolor{currentstroke}{rgb}{0.000000,0.000000,0.000000}%
\pgfsetstrokecolor{currentstroke}%
\pgfsetdash{}{0pt}%
\pgfsys@defobject{currentmarker}{\pgfqpoint{0.000000in}{-0.048611in}}{\pgfqpoint{0.000000in}{0.000000in}}{%
\pgfpathmoveto{\pgfqpoint{0.000000in}{0.000000in}}%
\pgfpathlineto{\pgfqpoint{0.000000in}{-0.048611in}}%
\pgfusepath{stroke,fill}%
}%
\begin{pgfscope}%
\pgfsys@transformshift{5.747272in}{8.480556in}%
\pgfsys@useobject{currentmarker}{}%
\end{pgfscope}%
\end{pgfscope}%
\begin{pgfscope}%
\definecolor{textcolor}{rgb}{0.000000,0.000000,0.000000}%
\pgfsetstrokecolor{textcolor}%
\pgfsetfillcolor{textcolor}%
\pgftext[x=5.747272in,y=8.383333in,,top]{\color{textcolor}\sffamily\fontsize{10.000000}{12.000000}\selectfont −1}%
\end{pgfscope}%
\begin{pgfscope}%
\pgfsetbuttcap%
\pgfsetroundjoin%
\definecolor{currentfill}{rgb}{0.000000,0.000000,0.000000}%
\pgfsetfillcolor{currentfill}%
\pgfsetlinewidth{0.803000pt}%
\definecolor{currentstroke}{rgb}{0.000000,0.000000,0.000000}%
\pgfsetstrokecolor{currentstroke}%
\pgfsetdash{}{0pt}%
\pgfsys@defobject{currentmarker}{\pgfqpoint{0.000000in}{-0.048611in}}{\pgfqpoint{0.000000in}{0.000000in}}{%
\pgfpathmoveto{\pgfqpoint{0.000000in}{0.000000in}}%
\pgfpathlineto{\pgfqpoint{0.000000in}{-0.048611in}}%
\pgfusepath{stroke,fill}%
}%
\begin{pgfscope}%
\pgfsys@transformshift{6.231163in}{8.480556in}%
\pgfsys@useobject{currentmarker}{}%
\end{pgfscope}%
\end{pgfscope}%
\begin{pgfscope}%
\definecolor{textcolor}{rgb}{0.000000,0.000000,0.000000}%
\pgfsetstrokecolor{textcolor}%
\pgfsetfillcolor{textcolor}%
\pgftext[x=6.231163in,y=8.383333in,,top]{\color{textcolor}\sffamily\fontsize{10.000000}{12.000000}\selectfont 0}%
\end{pgfscope}%
\begin{pgfscope}%
\pgfsetbuttcap%
\pgfsetroundjoin%
\definecolor{currentfill}{rgb}{0.000000,0.000000,0.000000}%
\pgfsetfillcolor{currentfill}%
\pgfsetlinewidth{0.803000pt}%
\definecolor{currentstroke}{rgb}{0.000000,0.000000,0.000000}%
\pgfsetstrokecolor{currentstroke}%
\pgfsetdash{}{0pt}%
\pgfsys@defobject{currentmarker}{\pgfqpoint{0.000000in}{-0.048611in}}{\pgfqpoint{0.000000in}{0.000000in}}{%
\pgfpathmoveto{\pgfqpoint{0.000000in}{0.000000in}}%
\pgfpathlineto{\pgfqpoint{0.000000in}{-0.048611in}}%
\pgfusepath{stroke,fill}%
}%
\begin{pgfscope}%
\pgfsys@transformshift{6.715054in}{8.480556in}%
\pgfsys@useobject{currentmarker}{}%
\end{pgfscope}%
\end{pgfscope}%
\begin{pgfscope}%
\definecolor{textcolor}{rgb}{0.000000,0.000000,0.000000}%
\pgfsetstrokecolor{textcolor}%
\pgfsetfillcolor{textcolor}%
\pgftext[x=6.715054in,y=8.383333in,,top]{\color{textcolor}\sffamily\fontsize{10.000000}{12.000000}\selectfont 1}%
\end{pgfscope}%
\begin{pgfscope}%
\pgfsetbuttcap%
\pgfsetroundjoin%
\definecolor{currentfill}{rgb}{0.000000,0.000000,0.000000}%
\pgfsetfillcolor{currentfill}%
\pgfsetlinewidth{0.803000pt}%
\definecolor{currentstroke}{rgb}{0.000000,0.000000,0.000000}%
\pgfsetstrokecolor{currentstroke}%
\pgfsetdash{}{0pt}%
\pgfsys@defobject{currentmarker}{\pgfqpoint{0.000000in}{-0.048611in}}{\pgfqpoint{0.000000in}{0.000000in}}{%
\pgfpathmoveto{\pgfqpoint{0.000000in}{0.000000in}}%
\pgfpathlineto{\pgfqpoint{0.000000in}{-0.048611in}}%
\pgfusepath{stroke,fill}%
}%
\begin{pgfscope}%
\pgfsys@transformshift{7.198945in}{8.480556in}%
\pgfsys@useobject{currentmarker}{}%
\end{pgfscope}%
\end{pgfscope}%
\begin{pgfscope}%
\definecolor{textcolor}{rgb}{0.000000,0.000000,0.000000}%
\pgfsetstrokecolor{textcolor}%
\pgfsetfillcolor{textcolor}%
\pgftext[x=7.198945in,y=8.383333in,,top]{\color{textcolor}\sffamily\fontsize{10.000000}{12.000000}\selectfont 2}%
\end{pgfscope}%
\begin{pgfscope}%
\pgfsetbuttcap%
\pgfsetroundjoin%
\definecolor{currentfill}{rgb}{0.000000,0.000000,0.000000}%
\pgfsetfillcolor{currentfill}%
\pgfsetlinewidth{0.803000pt}%
\definecolor{currentstroke}{rgb}{0.000000,0.000000,0.000000}%
\pgfsetstrokecolor{currentstroke}%
\pgfsetdash{}{0pt}%
\pgfsys@defobject{currentmarker}{\pgfqpoint{0.000000in}{-0.048611in}}{\pgfqpoint{0.000000in}{0.000000in}}{%
\pgfpathmoveto{\pgfqpoint{0.000000in}{0.000000in}}%
\pgfpathlineto{\pgfqpoint{0.000000in}{-0.048611in}}%
\pgfusepath{stroke,fill}%
}%
\begin{pgfscope}%
\pgfsys@transformshift{7.682836in}{8.480556in}%
\pgfsys@useobject{currentmarker}{}%
\end{pgfscope}%
\end{pgfscope}%
\begin{pgfscope}%
\definecolor{textcolor}{rgb}{0.000000,0.000000,0.000000}%
\pgfsetstrokecolor{textcolor}%
\pgfsetfillcolor{textcolor}%
\pgftext[x=7.682836in,y=8.383333in,,top]{\color{textcolor}\sffamily\fontsize{10.000000}{12.000000}\selectfont 3}%
\end{pgfscope}%
\begin{pgfscope}%
\definecolor{textcolor}{rgb}{0.000000,0.000000,0.000000}%
\pgfsetstrokecolor{textcolor}%
\pgfsetfillcolor{textcolor}%
\pgftext[x=6.231163in,y=8.193365in,,top]{\color{textcolor}\sffamily\fontsize{10.000000}{12.000000}\selectfont \(\displaystyle x\)}%
\end{pgfscope}%
\begin{pgfscope}%
\pgfsetbuttcap%
\pgfsetroundjoin%
\definecolor{currentfill}{rgb}{0.000000,0.000000,0.000000}%
\pgfsetfillcolor{currentfill}%
\pgfsetlinewidth{0.803000pt}%
\definecolor{currentstroke}{rgb}{0.000000,0.000000,0.000000}%
\pgfsetstrokecolor{currentstroke}%
\pgfsetdash{}{0pt}%
\pgfsys@defobject{currentmarker}{\pgfqpoint{-0.048611in}{0.000000in}}{\pgfqpoint{0.000000in}{0.000000in}}{%
\pgfpathmoveto{\pgfqpoint{0.000000in}{0.000000in}}%
\pgfpathlineto{\pgfqpoint{-0.048611in}{0.000000in}}%
\pgfusepath{stroke,fill}%
}%
\begin{pgfscope}%
\pgfsys@transformshift{4.660937in}{8.623611in}%
\pgfsys@useobject{currentmarker}{}%
\end{pgfscope}%
\end{pgfscope}%
\begin{pgfscope}%
\definecolor{textcolor}{rgb}{0.000000,0.000000,0.000000}%
\pgfsetstrokecolor{textcolor}%
\pgfsetfillcolor{textcolor}%
\pgftext[x=4.342836in,y=8.570850in,left,base]{\color{textcolor}\sffamily\fontsize{10.000000}{12.000000}\selectfont 0.0}%
\end{pgfscope}%
\begin{pgfscope}%
\pgfsetbuttcap%
\pgfsetroundjoin%
\definecolor{currentfill}{rgb}{0.000000,0.000000,0.000000}%
\pgfsetfillcolor{currentfill}%
\pgfsetlinewidth{0.803000pt}%
\definecolor{currentstroke}{rgb}{0.000000,0.000000,0.000000}%
\pgfsetstrokecolor{currentstroke}%
\pgfsetdash{}{0pt}%
\pgfsys@defobject{currentmarker}{\pgfqpoint{-0.048611in}{0.000000in}}{\pgfqpoint{0.000000in}{0.000000in}}{%
\pgfpathmoveto{\pgfqpoint{0.000000in}{0.000000in}}%
\pgfpathlineto{\pgfqpoint{-0.048611in}{0.000000in}}%
\pgfusepath{stroke,fill}%
}%
\begin{pgfscope}%
\pgfsys@transformshift{4.660937in}{9.193811in}%
\pgfsys@useobject{currentmarker}{}%
\end{pgfscope}%
\end{pgfscope}%
\begin{pgfscope}%
\definecolor{textcolor}{rgb}{0.000000,0.000000,0.000000}%
\pgfsetstrokecolor{textcolor}%
\pgfsetfillcolor{textcolor}%
\pgftext[x=4.342836in,y=9.141049in,left,base]{\color{textcolor}\sffamily\fontsize{10.000000}{12.000000}\selectfont 0.2}%
\end{pgfscope}%
\begin{pgfscope}%
\pgfsetbuttcap%
\pgfsetroundjoin%
\definecolor{currentfill}{rgb}{0.000000,0.000000,0.000000}%
\pgfsetfillcolor{currentfill}%
\pgfsetlinewidth{0.803000pt}%
\definecolor{currentstroke}{rgb}{0.000000,0.000000,0.000000}%
\pgfsetstrokecolor{currentstroke}%
\pgfsetdash{}{0pt}%
\pgfsys@defobject{currentmarker}{\pgfqpoint{-0.048611in}{0.000000in}}{\pgfqpoint{0.000000in}{0.000000in}}{%
\pgfpathmoveto{\pgfqpoint{0.000000in}{0.000000in}}%
\pgfpathlineto{\pgfqpoint{-0.048611in}{0.000000in}}%
\pgfusepath{stroke,fill}%
}%
\begin{pgfscope}%
\pgfsys@transformshift{4.660937in}{9.764010in}%
\pgfsys@useobject{currentmarker}{}%
\end{pgfscope}%
\end{pgfscope}%
\begin{pgfscope}%
\definecolor{textcolor}{rgb}{0.000000,0.000000,0.000000}%
\pgfsetstrokecolor{textcolor}%
\pgfsetfillcolor{textcolor}%
\pgftext[x=4.342836in,y=9.711248in,left,base]{\color{textcolor}\sffamily\fontsize{10.000000}{12.000000}\selectfont 0.4}%
\end{pgfscope}%
\begin{pgfscope}%
\pgfsetbuttcap%
\pgfsetroundjoin%
\definecolor{currentfill}{rgb}{0.000000,0.000000,0.000000}%
\pgfsetfillcolor{currentfill}%
\pgfsetlinewidth{0.803000pt}%
\definecolor{currentstroke}{rgb}{0.000000,0.000000,0.000000}%
\pgfsetstrokecolor{currentstroke}%
\pgfsetdash{}{0pt}%
\pgfsys@defobject{currentmarker}{\pgfqpoint{-0.048611in}{0.000000in}}{\pgfqpoint{0.000000in}{0.000000in}}{%
\pgfpathmoveto{\pgfqpoint{0.000000in}{0.000000in}}%
\pgfpathlineto{\pgfqpoint{-0.048611in}{0.000000in}}%
\pgfusepath{stroke,fill}%
}%
\begin{pgfscope}%
\pgfsys@transformshift{4.660937in}{10.334209in}%
\pgfsys@useobject{currentmarker}{}%
\end{pgfscope}%
\end{pgfscope}%
\begin{pgfscope}%
\definecolor{textcolor}{rgb}{0.000000,0.000000,0.000000}%
\pgfsetstrokecolor{textcolor}%
\pgfsetfillcolor{textcolor}%
\pgftext[x=4.342836in,y=10.281448in,left,base]{\color{textcolor}\sffamily\fontsize{10.000000}{12.000000}\selectfont 0.6}%
\end{pgfscope}%
\begin{pgfscope}%
\pgfsetbuttcap%
\pgfsetroundjoin%
\definecolor{currentfill}{rgb}{0.000000,0.000000,0.000000}%
\pgfsetfillcolor{currentfill}%
\pgfsetlinewidth{0.803000pt}%
\definecolor{currentstroke}{rgb}{0.000000,0.000000,0.000000}%
\pgfsetstrokecolor{currentstroke}%
\pgfsetdash{}{0pt}%
\pgfsys@defobject{currentmarker}{\pgfqpoint{-0.048611in}{0.000000in}}{\pgfqpoint{0.000000in}{0.000000in}}{%
\pgfpathmoveto{\pgfqpoint{0.000000in}{0.000000in}}%
\pgfpathlineto{\pgfqpoint{-0.048611in}{0.000000in}}%
\pgfusepath{stroke,fill}%
}%
\begin{pgfscope}%
\pgfsys@transformshift{4.660937in}{10.904409in}%
\pgfsys@useobject{currentmarker}{}%
\end{pgfscope}%
\end{pgfscope}%
\begin{pgfscope}%
\definecolor{textcolor}{rgb}{0.000000,0.000000,0.000000}%
\pgfsetstrokecolor{textcolor}%
\pgfsetfillcolor{textcolor}%
\pgftext[x=4.342836in,y=10.851647in,left,base]{\color{textcolor}\sffamily\fontsize{10.000000}{12.000000}\selectfont 0.8}%
\end{pgfscope}%
\begin{pgfscope}%
\pgfsetbuttcap%
\pgfsetroundjoin%
\definecolor{currentfill}{rgb}{0.000000,0.000000,0.000000}%
\pgfsetfillcolor{currentfill}%
\pgfsetlinewidth{0.803000pt}%
\definecolor{currentstroke}{rgb}{0.000000,0.000000,0.000000}%
\pgfsetstrokecolor{currentstroke}%
\pgfsetdash{}{0pt}%
\pgfsys@defobject{currentmarker}{\pgfqpoint{-0.048611in}{0.000000in}}{\pgfqpoint{0.000000in}{0.000000in}}{%
\pgfpathmoveto{\pgfqpoint{0.000000in}{0.000000in}}%
\pgfpathlineto{\pgfqpoint{-0.048611in}{0.000000in}}%
\pgfusepath{stroke,fill}%
}%
\begin{pgfscope}%
\pgfsys@transformshift{4.660937in}{11.474608in}%
\pgfsys@useobject{currentmarker}{}%
\end{pgfscope}%
\end{pgfscope}%
\begin{pgfscope}%
\definecolor{textcolor}{rgb}{0.000000,0.000000,0.000000}%
\pgfsetstrokecolor{textcolor}%
\pgfsetfillcolor{textcolor}%
\pgftext[x=4.342836in,y=11.421847in,left,base]{\color{textcolor}\sffamily\fontsize{10.000000}{12.000000}\selectfont 1.0}%
\end{pgfscope}%
\begin{pgfscope}%
\definecolor{textcolor}{rgb}{0.000000,0.000000,0.000000}%
\pgfsetstrokecolor{textcolor}%
\pgfsetfillcolor{textcolor}%
\pgftext[x=4.287280in,y=10.054167in,,bottom,rotate=90.000000]{\color{textcolor}\sffamily\fontsize{10.000000}{12.000000}\selectfont \(\displaystyle U\)}%
\end{pgfscope}%
\begin{pgfscope}%
\pgfpathrectangle{\pgfqpoint{4.660937in}{8.480556in}}{\pgfqpoint{3.140451in}{3.147222in}}%
\pgfusepath{clip}%
\pgfsetrectcap%
\pgfsetroundjoin%
\pgfsetlinewidth{1.505625pt}%
\definecolor{currentstroke}{rgb}{0.121569,0.466667,0.705882}%
\pgfsetstrokecolor{currentstroke}%
\pgfsetdash{}{0pt}%
\pgfpathmoveto{\pgfqpoint{4.803685in}{11.474609in}}%
\pgfpathlineto{\pgfqpoint{4.852074in}{11.474606in}}%
\pgfpathlineto{\pgfqpoint{4.900463in}{11.474615in}}%
\pgfpathlineto{\pgfqpoint{4.948853in}{11.474593in}}%
\pgfpathlineto{\pgfqpoint{4.997242in}{11.474639in}}%
\pgfpathlineto{\pgfqpoint{5.045631in}{11.474559in}}%
\pgfpathlineto{\pgfqpoint{5.094020in}{11.474663in}}%
\pgfpathlineto{\pgfqpoint{5.142409in}{11.474593in}}%
\pgfpathlineto{\pgfqpoint{5.190798in}{11.474533in}}%
\pgfpathlineto{\pgfqpoint{5.239187in}{11.474750in}}%
\pgfpathlineto{\pgfqpoint{5.287576in}{11.474569in}}%
\pgfpathlineto{\pgfqpoint{5.335965in}{11.474393in}}%
\pgfpathlineto{\pgfqpoint{5.384354in}{11.474834in}}%
\pgfpathlineto{\pgfqpoint{5.432743in}{11.474878in}}%
\pgfpathlineto{\pgfqpoint{5.481132in}{11.474154in}}%
\pgfpathlineto{\pgfqpoint{5.529522in}{11.474135in}}%
\pgfpathlineto{\pgfqpoint{5.577911in}{11.475275in}}%
\pgfpathlineto{\pgfqpoint{5.626300in}{11.475765in}}%
\pgfpathlineto{\pgfqpoint{5.674689in}{11.474280in}}%
\pgfpathlineto{\pgfqpoint{5.723078in}{11.472216in}}%
\pgfpathlineto{\pgfqpoint{5.771467in}{11.472323in}}%
\pgfpathlineto{\pgfqpoint{5.819856in}{11.475853in}}%
\pgfpathlineto{\pgfqpoint{5.868245in}{11.481237in}}%
\pgfpathlineto{\pgfqpoint{5.916634in}{11.484712in}}%
\pgfpathlineto{\pgfqpoint{5.965023in}{11.481626in}}%
\pgfpathlineto{\pgfqpoint{6.013412in}{11.467380in}}%
\pgfpathlineto{\pgfqpoint{6.061801in}{11.437724in}}%
\pgfpathlineto{\pgfqpoint{6.110190in}{11.388621in}}%
\pgfpathlineto{\pgfqpoint{6.158580in}{11.315938in}}%
\pgfpathlineto{\pgfqpoint{6.206969in}{11.215223in}}%
\pgfpathlineto{\pgfqpoint{6.255358in}{11.081757in}}%
\pgfpathlineto{\pgfqpoint{6.303747in}{10.911229in}}%
\pgfpathlineto{\pgfqpoint{6.352136in}{10.701470in}}%
\pgfpathlineto{\pgfqpoint{6.400525in}{10.455299in}}%
\pgfpathlineto{\pgfqpoint{6.448914in}{10.183089in}}%
\pgfpathlineto{\pgfqpoint{6.497303in}{9.902285in}}%
\pgfpathlineto{\pgfqpoint{6.545692in}{9.632701in}}%
\pgfpathlineto{\pgfqpoint{6.594081in}{9.390355in}}%
\pgfpathlineto{\pgfqpoint{6.642470in}{9.184061in}}%
\pgfpathlineto{\pgfqpoint{6.690859in}{9.015931in}}%
\pgfpathlineto{\pgfqpoint{6.739249in}{8.883858in}}%
\pgfpathlineto{\pgfqpoint{6.787638in}{8.783886in}}%
\pgfpathlineto{\pgfqpoint{6.836027in}{8.711678in}}%
\pgfpathlineto{\pgfqpoint{6.884416in}{8.663199in}}%
\pgfpathlineto{\pgfqpoint{6.932805in}{8.634946in}}%
\pgfpathlineto{\pgfqpoint{6.981194in}{8.623974in}}%
\pgfpathlineto{\pgfqpoint{7.029583in}{8.623611in}}%
\pgfpathlineto{\pgfqpoint{7.077972in}{8.623611in}}%
\pgfpathlineto{\pgfqpoint{7.126361in}{8.623611in}}%
\pgfpathlineto{\pgfqpoint{7.174750in}{8.623611in}}%
\pgfpathlineto{\pgfqpoint{7.223139in}{8.623611in}}%
\pgfpathlineto{\pgfqpoint{7.271528in}{8.623611in}}%
\pgfpathlineto{\pgfqpoint{7.319918in}{8.623611in}}%
\pgfpathlineto{\pgfqpoint{7.368307in}{8.623611in}}%
\pgfpathlineto{\pgfqpoint{7.416696in}{8.623611in}}%
\pgfpathlineto{\pgfqpoint{7.465085in}{8.623611in}}%
\pgfpathlineto{\pgfqpoint{7.513474in}{8.623611in}}%
\pgfpathlineto{\pgfqpoint{7.561863in}{8.623611in}}%
\pgfpathlineto{\pgfqpoint{7.610252in}{8.623611in}}%
\pgfpathlineto{\pgfqpoint{7.658641in}{8.623611in}}%
\pgfusepath{stroke}%
\end{pgfscope}%
\begin{pgfscope}%
\pgfpathrectangle{\pgfqpoint{4.660937in}{8.480556in}}{\pgfqpoint{3.140451in}{3.147222in}}%
\pgfusepath{clip}%
\pgfsetrectcap%
\pgfsetroundjoin%
\pgfsetlinewidth{1.505625pt}%
\definecolor{currentstroke}{rgb}{1.000000,0.498039,0.054902}%
\pgfsetstrokecolor{currentstroke}%
\pgfsetdash{}{0pt}%
\pgfpathmoveto{\pgfqpoint{4.803685in}{11.474609in}}%
\pgfpathlineto{\pgfqpoint{4.852074in}{11.474606in}}%
\pgfpathlineto{\pgfqpoint{4.900463in}{11.474615in}}%
\pgfpathlineto{\pgfqpoint{4.948853in}{11.474593in}}%
\pgfpathlineto{\pgfqpoint{4.997242in}{11.474639in}}%
\pgfpathlineto{\pgfqpoint{5.045631in}{11.474559in}}%
\pgfpathlineto{\pgfqpoint{5.094020in}{11.474663in}}%
\pgfpathlineto{\pgfqpoint{5.142409in}{11.474593in}}%
\pgfpathlineto{\pgfqpoint{5.190798in}{11.474533in}}%
\pgfpathlineto{\pgfqpoint{5.239187in}{11.474751in}}%
\pgfpathlineto{\pgfqpoint{5.287576in}{11.474569in}}%
\pgfpathlineto{\pgfqpoint{5.335965in}{11.474392in}}%
\pgfpathlineto{\pgfqpoint{5.384354in}{11.474835in}}%
\pgfpathlineto{\pgfqpoint{5.432743in}{11.474878in}}%
\pgfpathlineto{\pgfqpoint{5.481132in}{11.474152in}}%
\pgfpathlineto{\pgfqpoint{5.529522in}{11.474134in}}%
\pgfpathlineto{\pgfqpoint{5.577911in}{11.475278in}}%
\pgfpathlineto{\pgfqpoint{5.626300in}{11.475767in}}%
\pgfpathlineto{\pgfqpoint{5.674689in}{11.474274in}}%
\pgfpathlineto{\pgfqpoint{5.723078in}{11.472204in}}%
\pgfpathlineto{\pgfqpoint{5.771467in}{11.472321in}}%
\pgfpathlineto{\pgfqpoint{5.819856in}{11.475878in}}%
\pgfpathlineto{\pgfqpoint{5.868245in}{11.481280in}}%
\pgfpathlineto{\pgfqpoint{5.916634in}{11.484722in}}%
\pgfpathlineto{\pgfqpoint{5.965023in}{11.481527in}}%
\pgfpathlineto{\pgfqpoint{6.013412in}{11.467082in}}%
\pgfpathlineto{\pgfqpoint{6.061801in}{11.437136in}}%
\pgfpathlineto{\pgfqpoint{6.110190in}{11.387643in}}%
\pgfpathlineto{\pgfqpoint{6.158580in}{11.314466in}}%
\pgfpathlineto{\pgfqpoint{6.206969in}{11.213163in}}%
\pgfpathlineto{\pgfqpoint{6.255358in}{11.079083in}}%
\pgfpathlineto{\pgfqpoint{6.303747in}{10.908101in}}%
\pgfpathlineto{\pgfqpoint{6.352136in}{10.698369in}}%
\pgfpathlineto{\pgfqpoint{6.400525in}{10.453016in}}%
\pgfpathlineto{\pgfqpoint{6.448914in}{10.182377in}}%
\pgfpathlineto{\pgfqpoint{6.497303in}{9.903382in}}%
\pgfpathlineto{\pgfqpoint{6.545692in}{9.635170in}}%
\pgfpathlineto{\pgfqpoint{6.594081in}{9.393421in}}%
\pgfpathlineto{\pgfqpoint{6.642470in}{9.187050in}}%
\pgfpathlineto{\pgfqpoint{6.690859in}{9.018465in}}%
\pgfpathlineto{\pgfqpoint{6.739249in}{8.885815in}}%
\pgfpathlineto{\pgfqpoint{6.787638in}{8.785291in}}%
\pgfpathlineto{\pgfqpoint{6.836027in}{8.712614in}}%
\pgfpathlineto{\pgfqpoint{6.884416in}{8.663758in}}%
\pgfpathlineto{\pgfqpoint{6.932805in}{8.635213in}}%
\pgfpathlineto{\pgfqpoint{6.981194in}{8.624025in}}%
\pgfpathlineto{\pgfqpoint{7.029583in}{8.623611in}}%
\pgfpathlineto{\pgfqpoint{7.077972in}{8.623611in}}%
\pgfpathlineto{\pgfqpoint{7.126361in}{8.623611in}}%
\pgfpathlineto{\pgfqpoint{7.174750in}{8.623611in}}%
\pgfpathlineto{\pgfqpoint{7.223139in}{8.623611in}}%
\pgfpathlineto{\pgfqpoint{7.271528in}{8.623611in}}%
\pgfpathlineto{\pgfqpoint{7.319918in}{8.623611in}}%
\pgfpathlineto{\pgfqpoint{7.368307in}{8.623611in}}%
\pgfpathlineto{\pgfqpoint{7.416696in}{8.623611in}}%
\pgfpathlineto{\pgfqpoint{7.465085in}{8.623611in}}%
\pgfpathlineto{\pgfqpoint{7.513474in}{8.623611in}}%
\pgfpathlineto{\pgfqpoint{7.561863in}{8.623611in}}%
\pgfpathlineto{\pgfqpoint{7.610252in}{8.623611in}}%
\pgfpathlineto{\pgfqpoint{7.658641in}{8.623611in}}%
\pgfusepath{stroke}%
\end{pgfscope}%
\begin{pgfscope}%
\pgfpathrectangle{\pgfqpoint{4.660937in}{8.480556in}}{\pgfqpoint{3.140451in}{3.147222in}}%
\pgfusepath{clip}%
\pgfsetrectcap%
\pgfsetroundjoin%
\pgfsetlinewidth{1.505625pt}%
\definecolor{currentstroke}{rgb}{0.172549,0.627451,0.172549}%
\pgfsetstrokecolor{currentstroke}%
\pgfsetdash{}{0pt}%
\pgfpathmoveto{\pgfqpoint{4.803685in}{11.474608in}}%
\pgfpathlineto{\pgfqpoint{4.852074in}{11.474608in}}%
\pgfpathlineto{\pgfqpoint{4.900463in}{11.474608in}}%
\pgfpathlineto{\pgfqpoint{4.948853in}{11.474608in}}%
\pgfpathlineto{\pgfqpoint{4.997242in}{11.474608in}}%
\pgfpathlineto{\pgfqpoint{5.045631in}{11.474608in}}%
\pgfpathlineto{\pgfqpoint{5.094020in}{11.474608in}}%
\pgfpathlineto{\pgfqpoint{5.142409in}{11.474608in}}%
\pgfpathlineto{\pgfqpoint{5.190798in}{11.474608in}}%
\pgfpathlineto{\pgfqpoint{5.239187in}{11.474608in}}%
\pgfpathlineto{\pgfqpoint{5.287576in}{11.474608in}}%
\pgfpathlineto{\pgfqpoint{5.335965in}{11.474608in}}%
\pgfpathlineto{\pgfqpoint{5.384354in}{11.474608in}}%
\pgfpathlineto{\pgfqpoint{5.432743in}{11.474608in}}%
\pgfpathlineto{\pgfqpoint{5.481132in}{11.474608in}}%
\pgfpathlineto{\pgfqpoint{5.529522in}{11.474608in}}%
\pgfpathlineto{\pgfqpoint{5.577911in}{11.474604in}}%
\pgfpathlineto{\pgfqpoint{5.626300in}{11.474577in}}%
\pgfpathlineto{\pgfqpoint{5.674689in}{11.474460in}}%
\pgfpathlineto{\pgfqpoint{5.723078in}{11.474060in}}%
\pgfpathlineto{\pgfqpoint{5.771467in}{11.472951in}}%
\pgfpathlineto{\pgfqpoint{5.819856in}{11.470343in}}%
\pgfpathlineto{\pgfqpoint{5.868245in}{11.464964in}}%
\pgfpathlineto{\pgfqpoint{5.916634in}{11.455012in}}%
\pgfpathlineto{\pgfqpoint{5.965023in}{11.438170in}}%
\pgfpathlineto{\pgfqpoint{6.013412in}{11.411685in}}%
\pgfpathlineto{\pgfqpoint{6.061801in}{11.372477in}}%
\pgfpathlineto{\pgfqpoint{6.110190in}{11.317251in}}%
\pgfpathlineto{\pgfqpoint{6.158580in}{11.242598in}}%
\pgfpathlineto{\pgfqpoint{6.206969in}{11.145123in}}%
\pgfpathlineto{\pgfqpoint{6.255358in}{11.021637in}}%
\pgfpathlineto{\pgfqpoint{6.303747in}{10.869515in}}%
\pgfpathlineto{\pgfqpoint{6.352136in}{10.687323in}}%
\pgfpathlineto{\pgfqpoint{6.400525in}{10.475779in}}%
\pgfpathlineto{\pgfqpoint{6.448914in}{10.238950in}}%
\pgfpathlineto{\pgfqpoint{6.497303in}{9.985237in}}%
\pgfpathlineto{\pgfqpoint{6.545692in}{9.727316in}}%
\pgfpathlineto{\pgfqpoint{6.594081in}{9.480242in}}%
\pgfpathlineto{\pgfqpoint{6.642470in}{9.257942in}}%
\pgfpathlineto{\pgfqpoint{6.690859in}{9.069779in}}%
\pgfpathlineto{\pgfqpoint{6.739249in}{8.919197in}}%
\pgfpathlineto{\pgfqpoint{6.787638in}{8.804820in}}%
\pgfpathlineto{\pgfqpoint{6.836027in}{8.722665in}}%
\pgfpathlineto{\pgfqpoint{6.884416in}{8.668014in}}%
\pgfpathlineto{\pgfqpoint{6.932805in}{8.636438in}}%
\pgfpathlineto{\pgfqpoint{6.981194in}{8.624152in}}%
\pgfpathlineto{\pgfqpoint{7.029583in}{8.623611in}}%
\pgfpathlineto{\pgfqpoint{7.077972in}{8.623611in}}%
\pgfpathlineto{\pgfqpoint{7.126361in}{8.623611in}}%
\pgfpathlineto{\pgfqpoint{7.174750in}{8.623611in}}%
\pgfpathlineto{\pgfqpoint{7.223139in}{8.623611in}}%
\pgfpathlineto{\pgfqpoint{7.271528in}{8.623611in}}%
\pgfpathlineto{\pgfqpoint{7.319918in}{8.623611in}}%
\pgfpathlineto{\pgfqpoint{7.368307in}{8.623611in}}%
\pgfpathlineto{\pgfqpoint{7.416696in}{8.623611in}}%
\pgfpathlineto{\pgfqpoint{7.465085in}{8.623611in}}%
\pgfpathlineto{\pgfqpoint{7.513474in}{8.623611in}}%
\pgfpathlineto{\pgfqpoint{7.561863in}{8.623611in}}%
\pgfpathlineto{\pgfqpoint{7.610252in}{8.623611in}}%
\pgfpathlineto{\pgfqpoint{7.658641in}{8.623611in}}%
\pgfusepath{stroke}%
\end{pgfscope}%
\begin{pgfscope}%
\pgfpathrectangle{\pgfqpoint{4.660937in}{8.480556in}}{\pgfqpoint{3.140451in}{3.147222in}}%
\pgfusepath{clip}%
\pgfsetrectcap%
\pgfsetroundjoin%
\pgfsetlinewidth{1.505625pt}%
\definecolor{currentstroke}{rgb}{0.839216,0.152941,0.156863}%
\pgfsetstrokecolor{currentstroke}%
\pgfsetdash{}{0pt}%
\pgfpathmoveto{\pgfqpoint{4.803685in}{11.474608in}}%
\pgfpathlineto{\pgfqpoint{4.852074in}{11.474608in}}%
\pgfpathlineto{\pgfqpoint{4.900463in}{11.474608in}}%
\pgfpathlineto{\pgfqpoint{4.948853in}{11.474608in}}%
\pgfpathlineto{\pgfqpoint{4.997242in}{11.474608in}}%
\pgfpathlineto{\pgfqpoint{5.045631in}{11.474608in}}%
\pgfpathlineto{\pgfqpoint{5.094020in}{11.474608in}}%
\pgfpathlineto{\pgfqpoint{5.142409in}{11.474608in}}%
\pgfpathlineto{\pgfqpoint{5.190798in}{11.474608in}}%
\pgfpathlineto{\pgfqpoint{5.239187in}{11.474608in}}%
\pgfpathlineto{\pgfqpoint{5.287576in}{11.474608in}}%
\pgfpathlineto{\pgfqpoint{5.335965in}{11.474608in}}%
\pgfpathlineto{\pgfqpoint{5.384354in}{11.474608in}}%
\pgfpathlineto{\pgfqpoint{5.432743in}{11.474608in}}%
\pgfpathlineto{\pgfqpoint{5.481132in}{11.474608in}}%
\pgfpathlineto{\pgfqpoint{5.529522in}{11.474608in}}%
\pgfpathlineto{\pgfqpoint{5.577911in}{11.474604in}}%
\pgfpathlineto{\pgfqpoint{5.626300in}{11.474577in}}%
\pgfpathlineto{\pgfqpoint{5.674689in}{11.474459in}}%
\pgfpathlineto{\pgfqpoint{5.723078in}{11.474055in}}%
\pgfpathlineto{\pgfqpoint{5.771467in}{11.472932in}}%
\pgfpathlineto{\pgfqpoint{5.819856in}{11.470279in}}%
\pgfpathlineto{\pgfqpoint{5.868245in}{11.464781in}}%
\pgfpathlineto{\pgfqpoint{5.916634in}{11.454548in}}%
\pgfpathlineto{\pgfqpoint{5.965023in}{11.437119in}}%
\pgfpathlineto{\pgfqpoint{6.013412in}{11.409523in}}%
\pgfpathlineto{\pgfqpoint{6.061801in}{11.368385in}}%
\pgfpathlineto{\pgfqpoint{6.110190in}{11.310035in}}%
\pgfpathlineto{\pgfqpoint{6.158580in}{11.230643in}}%
\pgfpathlineto{\pgfqpoint{6.206969in}{11.126388in}}%
\pgfpathlineto{\pgfqpoint{6.255358in}{10.993759in}}%
\pgfpathlineto{\pgfqpoint{6.303747in}{10.830107in}}%
\pgfpathlineto{\pgfqpoint{6.352136in}{10.634577in}}%
\pgfpathlineto{\pgfqpoint{6.400525in}{10.409410in}}%
\pgfpathlineto{\pgfqpoint{6.448914in}{10.161268in}}%
\pgfpathlineto{\pgfqpoint{6.497303in}{9.901691in}}%
\pgfpathlineto{\pgfqpoint{6.545692in}{9.645674in}}%
\pgfpathlineto{\pgfqpoint{6.594081in}{9.408241in}}%
\pgfpathlineto{\pgfqpoint{6.642470in}{9.200646in}}%
\pgfpathlineto{\pgfqpoint{6.690859in}{9.028414in}}%
\pgfpathlineto{\pgfqpoint{6.739249in}{8.891926in}}%
\pgfpathlineto{\pgfqpoint{6.787638in}{8.788397in}}%
\pgfpathlineto{\pgfqpoint{6.836027in}{8.713771in}}%
\pgfpathlineto{\pgfqpoint{6.884416in}{8.663888in}}%
\pgfpathlineto{\pgfqpoint{6.932805in}{8.635004in}}%
\pgfpathlineto{\pgfqpoint{6.981194in}{8.623931in}}%
\pgfpathlineto{\pgfqpoint{7.029583in}{8.623611in}}%
\pgfpathlineto{\pgfqpoint{7.077972in}{8.623611in}}%
\pgfpathlineto{\pgfqpoint{7.126361in}{8.623611in}}%
\pgfpathlineto{\pgfqpoint{7.174750in}{8.623611in}}%
\pgfpathlineto{\pgfqpoint{7.223139in}{8.623611in}}%
\pgfpathlineto{\pgfqpoint{7.271528in}{8.623611in}}%
\pgfpathlineto{\pgfqpoint{7.319918in}{8.623611in}}%
\pgfpathlineto{\pgfqpoint{7.368307in}{8.623611in}}%
\pgfpathlineto{\pgfqpoint{7.416696in}{8.623611in}}%
\pgfpathlineto{\pgfqpoint{7.465085in}{8.623611in}}%
\pgfpathlineto{\pgfqpoint{7.513474in}{8.623611in}}%
\pgfpathlineto{\pgfqpoint{7.561863in}{8.623611in}}%
\pgfpathlineto{\pgfqpoint{7.610252in}{8.623611in}}%
\pgfpathlineto{\pgfqpoint{7.658641in}{8.623611in}}%
\pgfusepath{stroke}%
\end{pgfscope}%
\begin{pgfscope}%
\pgfpathrectangle{\pgfqpoint{4.660937in}{8.480556in}}{\pgfqpoint{3.140451in}{3.147222in}}%
\pgfusepath{clip}%
\pgfsetrectcap%
\pgfsetroundjoin%
\pgfsetlinewidth{0.501875pt}%
\definecolor{currentstroke}{rgb}{0.000000,0.000000,0.000000}%
\pgfsetstrokecolor{currentstroke}%
\pgfsetdash{}{0pt}%
\pgfpathmoveto{\pgfqpoint{4.803685in}{11.474608in}}%
\pgfpathlineto{\pgfqpoint{4.852074in}{11.474608in}}%
\pgfpathlineto{\pgfqpoint{4.900463in}{11.474608in}}%
\pgfpathlineto{\pgfqpoint{4.948853in}{11.474608in}}%
\pgfpathlineto{\pgfqpoint{4.997242in}{11.474608in}}%
\pgfpathlineto{\pgfqpoint{5.045631in}{11.474608in}}%
\pgfpathlineto{\pgfqpoint{5.094020in}{11.474608in}}%
\pgfpathlineto{\pgfqpoint{5.142409in}{11.474608in}}%
\pgfpathlineto{\pgfqpoint{5.190798in}{11.474608in}}%
\pgfpathlineto{\pgfqpoint{5.239187in}{11.474608in}}%
\pgfpathlineto{\pgfqpoint{5.287576in}{11.474608in}}%
\pgfpathlineto{\pgfqpoint{5.335965in}{11.474608in}}%
\pgfpathlineto{\pgfqpoint{5.384354in}{11.474608in}}%
\pgfpathlineto{\pgfqpoint{5.432743in}{11.474608in}}%
\pgfpathlineto{\pgfqpoint{5.481132in}{11.474608in}}%
\pgfpathlineto{\pgfqpoint{5.529522in}{11.474608in}}%
\pgfpathlineto{\pgfqpoint{5.577911in}{11.474608in}}%
\pgfpathlineto{\pgfqpoint{5.626300in}{11.474608in}}%
\pgfpathlineto{\pgfqpoint{5.674689in}{11.474608in}}%
\pgfpathlineto{\pgfqpoint{5.723078in}{11.474608in}}%
\pgfpathlineto{\pgfqpoint{5.771467in}{11.474608in}}%
\pgfpathlineto{\pgfqpoint{5.819856in}{11.474608in}}%
\pgfpathlineto{\pgfqpoint{5.868245in}{11.474608in}}%
\pgfpathlineto{\pgfqpoint{5.916634in}{11.474608in}}%
\pgfpathlineto{\pgfqpoint{5.965023in}{11.474297in}}%
\pgfpathlineto{\pgfqpoint{6.013412in}{11.463543in}}%
\pgfpathlineto{\pgfqpoint{6.061801in}{11.435600in}}%
\pgfpathlineto{\pgfqpoint{6.110190in}{11.387528in}}%
\pgfpathlineto{\pgfqpoint{6.158580in}{11.315821in}}%
\pgfpathlineto{\pgfqpoint{6.206969in}{11.216403in}}%
\pgfpathlineto{\pgfqpoint{6.255358in}{11.084787in}}%
\pgfpathlineto{\pgfqpoint{6.303747in}{10.916677in}}%
\pgfpathlineto{\pgfqpoint{6.352136in}{10.709395in}}%
\pgfpathlineto{\pgfqpoint{6.400525in}{10.464526in}}%
\pgfpathlineto{\pgfqpoint{6.448914in}{10.191187in}}%
\pgfpathlineto{\pgfqpoint{6.497303in}{9.907032in}}%
\pgfpathlineto{\pgfqpoint{6.545692in}{9.633693in}}%
\pgfpathlineto{\pgfqpoint{6.594081in}{9.388824in}}%
\pgfpathlineto{\pgfqpoint{6.642470in}{9.181542in}}%
\pgfpathlineto{\pgfqpoint{6.690859in}{9.013432in}}%
\pgfpathlineto{\pgfqpoint{6.739249in}{8.881817in}}%
\pgfpathlineto{\pgfqpoint{6.787638in}{8.782398in}}%
\pgfpathlineto{\pgfqpoint{6.836027in}{8.710691in}}%
\pgfpathlineto{\pgfqpoint{6.884416in}{8.662619in}}%
\pgfpathlineto{\pgfqpoint{6.932805in}{8.634676in}}%
\pgfpathlineto{\pgfqpoint{6.981194in}{8.623923in}}%
\pgfpathlineto{\pgfqpoint{7.029583in}{8.623611in}}%
\pgfpathlineto{\pgfqpoint{7.077972in}{8.623611in}}%
\pgfpathlineto{\pgfqpoint{7.126361in}{8.623611in}}%
\pgfpathlineto{\pgfqpoint{7.174750in}{8.623611in}}%
\pgfpathlineto{\pgfqpoint{7.223139in}{8.623611in}}%
\pgfpathlineto{\pgfqpoint{7.271528in}{8.623611in}}%
\pgfpathlineto{\pgfqpoint{7.319918in}{8.623611in}}%
\pgfpathlineto{\pgfqpoint{7.368307in}{8.623611in}}%
\pgfpathlineto{\pgfqpoint{7.416696in}{8.623611in}}%
\pgfpathlineto{\pgfqpoint{7.465085in}{8.623611in}}%
\pgfpathlineto{\pgfqpoint{7.513474in}{8.623611in}}%
\pgfpathlineto{\pgfqpoint{7.561863in}{8.623611in}}%
\pgfpathlineto{\pgfqpoint{7.610252in}{8.623611in}}%
\pgfpathlineto{\pgfqpoint{7.658641in}{8.623611in}}%
\pgfusepath{stroke}%
\end{pgfscope}%
\begin{pgfscope}%
\pgfsetrectcap%
\pgfsetmiterjoin%
\pgfsetlinewidth{0.803000pt}%
\definecolor{currentstroke}{rgb}{0.000000,0.000000,0.000000}%
\pgfsetstrokecolor{currentstroke}%
\pgfsetdash{}{0pt}%
\pgfpathmoveto{\pgfqpoint{4.660937in}{8.480556in}}%
\pgfpathlineto{\pgfqpoint{4.660937in}{11.627778in}}%
\pgfusepath{stroke}%
\end{pgfscope}%
\begin{pgfscope}%
\pgfsetrectcap%
\pgfsetmiterjoin%
\pgfsetlinewidth{0.803000pt}%
\definecolor{currentstroke}{rgb}{0.000000,0.000000,0.000000}%
\pgfsetstrokecolor{currentstroke}%
\pgfsetdash{}{0pt}%
\pgfpathmoveto{\pgfqpoint{7.801389in}{8.480556in}}%
\pgfpathlineto{\pgfqpoint{7.801389in}{11.627778in}}%
\pgfusepath{stroke}%
\end{pgfscope}%
\begin{pgfscope}%
\pgfsetrectcap%
\pgfsetmiterjoin%
\pgfsetlinewidth{0.803000pt}%
\definecolor{currentstroke}{rgb}{0.000000,0.000000,0.000000}%
\pgfsetstrokecolor{currentstroke}%
\pgfsetdash{}{0pt}%
\pgfpathmoveto{\pgfqpoint{4.660937in}{8.480556in}}%
\pgfpathlineto{\pgfqpoint{7.801389in}{8.480556in}}%
\pgfusepath{stroke}%
\end{pgfscope}%
\begin{pgfscope}%
\pgfsetrectcap%
\pgfsetmiterjoin%
\pgfsetlinewidth{0.803000pt}%
\definecolor{currentstroke}{rgb}{0.000000,0.000000,0.000000}%
\pgfsetstrokecolor{currentstroke}%
\pgfsetdash{}{0pt}%
\pgfpathmoveto{\pgfqpoint{4.660937in}{11.627778in}}%
\pgfpathlineto{\pgfqpoint{7.801389in}{11.627778in}}%
\pgfusepath{stroke}%
\end{pgfscope}%
\begin{pgfscope}%
\definecolor{textcolor}{rgb}{0.000000,0.000000,0.000000}%
\pgfsetstrokecolor{textcolor}%
\pgfsetfillcolor{textcolor}%
\pgftext[x=6.231163in,y=11.711111in,,base]{\color{textcolor}\sffamily\fontsize{12.000000}{14.400000}\selectfont \(\displaystyle  t = 1.0 \)}%
\end{pgfscope}%
\begin{pgfscope}%
\pgfsetbuttcap%
\pgfsetmiterjoin%
\definecolor{currentfill}{rgb}{1.000000,1.000000,1.000000}%
\pgfsetfillcolor{currentfill}%
\pgfsetlinewidth{0.000000pt}%
\definecolor{currentstroke}{rgb}{0.000000,0.000000,0.000000}%
\pgfsetstrokecolor{currentstroke}%
\pgfsetstrokeopacity{0.000000}%
\pgfsetdash{}{0pt}%
\pgfpathmoveto{\pgfqpoint{0.691667in}{4.530556in}}%
\pgfpathlineto{\pgfqpoint{3.832118in}{4.530556in}}%
\pgfpathlineto{\pgfqpoint{3.832118in}{7.677778in}}%
\pgfpathlineto{\pgfqpoint{0.691667in}{7.677778in}}%
\pgfpathclose%
\pgfusepath{fill}%
\end{pgfscope}%
\begin{pgfscope}%
\pgfsetbuttcap%
\pgfsetroundjoin%
\definecolor{currentfill}{rgb}{0.000000,0.000000,0.000000}%
\pgfsetfillcolor{currentfill}%
\pgfsetlinewidth{0.803000pt}%
\definecolor{currentstroke}{rgb}{0.000000,0.000000,0.000000}%
\pgfsetstrokecolor{currentstroke}%
\pgfsetdash{}{0pt}%
\pgfsys@defobject{currentmarker}{\pgfqpoint{0.000000in}{-0.048611in}}{\pgfqpoint{0.000000in}{0.000000in}}{%
\pgfpathmoveto{\pgfqpoint{0.000000in}{0.000000in}}%
\pgfpathlineto{\pgfqpoint{0.000000in}{-0.048611in}}%
\pgfusepath{stroke,fill}%
}%
\begin{pgfscope}%
\pgfsys@transformshift{0.810220in}{4.530556in}%
\pgfsys@useobject{currentmarker}{}%
\end{pgfscope}%
\end{pgfscope}%
\begin{pgfscope}%
\definecolor{textcolor}{rgb}{0.000000,0.000000,0.000000}%
\pgfsetstrokecolor{textcolor}%
\pgfsetfillcolor{textcolor}%
\pgftext[x=0.810220in,y=4.433333in,,top]{\color{textcolor}\sffamily\fontsize{10.000000}{12.000000}\selectfont −3}%
\end{pgfscope}%
\begin{pgfscope}%
\pgfsetbuttcap%
\pgfsetroundjoin%
\definecolor{currentfill}{rgb}{0.000000,0.000000,0.000000}%
\pgfsetfillcolor{currentfill}%
\pgfsetlinewidth{0.803000pt}%
\definecolor{currentstroke}{rgb}{0.000000,0.000000,0.000000}%
\pgfsetstrokecolor{currentstroke}%
\pgfsetdash{}{0pt}%
\pgfsys@defobject{currentmarker}{\pgfqpoint{0.000000in}{-0.048611in}}{\pgfqpoint{0.000000in}{0.000000in}}{%
\pgfpathmoveto{\pgfqpoint{0.000000in}{0.000000in}}%
\pgfpathlineto{\pgfqpoint{0.000000in}{-0.048611in}}%
\pgfusepath{stroke,fill}%
}%
\begin{pgfscope}%
\pgfsys@transformshift{1.294111in}{4.530556in}%
\pgfsys@useobject{currentmarker}{}%
\end{pgfscope}%
\end{pgfscope}%
\begin{pgfscope}%
\definecolor{textcolor}{rgb}{0.000000,0.000000,0.000000}%
\pgfsetstrokecolor{textcolor}%
\pgfsetfillcolor{textcolor}%
\pgftext[x=1.294111in,y=4.433333in,,top]{\color{textcolor}\sffamily\fontsize{10.000000}{12.000000}\selectfont −2}%
\end{pgfscope}%
\begin{pgfscope}%
\pgfsetbuttcap%
\pgfsetroundjoin%
\definecolor{currentfill}{rgb}{0.000000,0.000000,0.000000}%
\pgfsetfillcolor{currentfill}%
\pgfsetlinewidth{0.803000pt}%
\definecolor{currentstroke}{rgb}{0.000000,0.000000,0.000000}%
\pgfsetstrokecolor{currentstroke}%
\pgfsetdash{}{0pt}%
\pgfsys@defobject{currentmarker}{\pgfqpoint{0.000000in}{-0.048611in}}{\pgfqpoint{0.000000in}{0.000000in}}{%
\pgfpathmoveto{\pgfqpoint{0.000000in}{0.000000in}}%
\pgfpathlineto{\pgfqpoint{0.000000in}{-0.048611in}}%
\pgfusepath{stroke,fill}%
}%
\begin{pgfscope}%
\pgfsys@transformshift{1.778002in}{4.530556in}%
\pgfsys@useobject{currentmarker}{}%
\end{pgfscope}%
\end{pgfscope}%
\begin{pgfscope}%
\definecolor{textcolor}{rgb}{0.000000,0.000000,0.000000}%
\pgfsetstrokecolor{textcolor}%
\pgfsetfillcolor{textcolor}%
\pgftext[x=1.778002in,y=4.433333in,,top]{\color{textcolor}\sffamily\fontsize{10.000000}{12.000000}\selectfont −1}%
\end{pgfscope}%
\begin{pgfscope}%
\pgfsetbuttcap%
\pgfsetroundjoin%
\definecolor{currentfill}{rgb}{0.000000,0.000000,0.000000}%
\pgfsetfillcolor{currentfill}%
\pgfsetlinewidth{0.803000pt}%
\definecolor{currentstroke}{rgb}{0.000000,0.000000,0.000000}%
\pgfsetstrokecolor{currentstroke}%
\pgfsetdash{}{0pt}%
\pgfsys@defobject{currentmarker}{\pgfqpoint{0.000000in}{-0.048611in}}{\pgfqpoint{0.000000in}{0.000000in}}{%
\pgfpathmoveto{\pgfqpoint{0.000000in}{0.000000in}}%
\pgfpathlineto{\pgfqpoint{0.000000in}{-0.048611in}}%
\pgfusepath{stroke,fill}%
}%
\begin{pgfscope}%
\pgfsys@transformshift{2.261892in}{4.530556in}%
\pgfsys@useobject{currentmarker}{}%
\end{pgfscope}%
\end{pgfscope}%
\begin{pgfscope}%
\definecolor{textcolor}{rgb}{0.000000,0.000000,0.000000}%
\pgfsetstrokecolor{textcolor}%
\pgfsetfillcolor{textcolor}%
\pgftext[x=2.261892in,y=4.433333in,,top]{\color{textcolor}\sffamily\fontsize{10.000000}{12.000000}\selectfont 0}%
\end{pgfscope}%
\begin{pgfscope}%
\pgfsetbuttcap%
\pgfsetroundjoin%
\definecolor{currentfill}{rgb}{0.000000,0.000000,0.000000}%
\pgfsetfillcolor{currentfill}%
\pgfsetlinewidth{0.803000pt}%
\definecolor{currentstroke}{rgb}{0.000000,0.000000,0.000000}%
\pgfsetstrokecolor{currentstroke}%
\pgfsetdash{}{0pt}%
\pgfsys@defobject{currentmarker}{\pgfqpoint{0.000000in}{-0.048611in}}{\pgfqpoint{0.000000in}{0.000000in}}{%
\pgfpathmoveto{\pgfqpoint{0.000000in}{0.000000in}}%
\pgfpathlineto{\pgfqpoint{0.000000in}{-0.048611in}}%
\pgfusepath{stroke,fill}%
}%
\begin{pgfscope}%
\pgfsys@transformshift{2.745783in}{4.530556in}%
\pgfsys@useobject{currentmarker}{}%
\end{pgfscope}%
\end{pgfscope}%
\begin{pgfscope}%
\definecolor{textcolor}{rgb}{0.000000,0.000000,0.000000}%
\pgfsetstrokecolor{textcolor}%
\pgfsetfillcolor{textcolor}%
\pgftext[x=2.745783in,y=4.433333in,,top]{\color{textcolor}\sffamily\fontsize{10.000000}{12.000000}\selectfont 1}%
\end{pgfscope}%
\begin{pgfscope}%
\pgfsetbuttcap%
\pgfsetroundjoin%
\definecolor{currentfill}{rgb}{0.000000,0.000000,0.000000}%
\pgfsetfillcolor{currentfill}%
\pgfsetlinewidth{0.803000pt}%
\definecolor{currentstroke}{rgb}{0.000000,0.000000,0.000000}%
\pgfsetstrokecolor{currentstroke}%
\pgfsetdash{}{0pt}%
\pgfsys@defobject{currentmarker}{\pgfqpoint{0.000000in}{-0.048611in}}{\pgfqpoint{0.000000in}{0.000000in}}{%
\pgfpathmoveto{\pgfqpoint{0.000000in}{0.000000in}}%
\pgfpathlineto{\pgfqpoint{0.000000in}{-0.048611in}}%
\pgfusepath{stroke,fill}%
}%
\begin{pgfscope}%
\pgfsys@transformshift{3.229674in}{4.530556in}%
\pgfsys@useobject{currentmarker}{}%
\end{pgfscope}%
\end{pgfscope}%
\begin{pgfscope}%
\definecolor{textcolor}{rgb}{0.000000,0.000000,0.000000}%
\pgfsetstrokecolor{textcolor}%
\pgfsetfillcolor{textcolor}%
\pgftext[x=3.229674in,y=4.433333in,,top]{\color{textcolor}\sffamily\fontsize{10.000000}{12.000000}\selectfont 2}%
\end{pgfscope}%
\begin{pgfscope}%
\pgfsetbuttcap%
\pgfsetroundjoin%
\definecolor{currentfill}{rgb}{0.000000,0.000000,0.000000}%
\pgfsetfillcolor{currentfill}%
\pgfsetlinewidth{0.803000pt}%
\definecolor{currentstroke}{rgb}{0.000000,0.000000,0.000000}%
\pgfsetstrokecolor{currentstroke}%
\pgfsetdash{}{0pt}%
\pgfsys@defobject{currentmarker}{\pgfqpoint{0.000000in}{-0.048611in}}{\pgfqpoint{0.000000in}{0.000000in}}{%
\pgfpathmoveto{\pgfqpoint{0.000000in}{0.000000in}}%
\pgfpathlineto{\pgfqpoint{0.000000in}{-0.048611in}}%
\pgfusepath{stroke,fill}%
}%
\begin{pgfscope}%
\pgfsys@transformshift{3.713565in}{4.530556in}%
\pgfsys@useobject{currentmarker}{}%
\end{pgfscope}%
\end{pgfscope}%
\begin{pgfscope}%
\definecolor{textcolor}{rgb}{0.000000,0.000000,0.000000}%
\pgfsetstrokecolor{textcolor}%
\pgfsetfillcolor{textcolor}%
\pgftext[x=3.713565in,y=4.433333in,,top]{\color{textcolor}\sffamily\fontsize{10.000000}{12.000000}\selectfont 3}%
\end{pgfscope}%
\begin{pgfscope}%
\definecolor{textcolor}{rgb}{0.000000,0.000000,0.000000}%
\pgfsetstrokecolor{textcolor}%
\pgfsetfillcolor{textcolor}%
\pgftext[x=2.261892in,y=4.243365in,,top]{\color{textcolor}\sffamily\fontsize{10.000000}{12.000000}\selectfont \(\displaystyle x\)}%
\end{pgfscope}%
\begin{pgfscope}%
\pgfsetbuttcap%
\pgfsetroundjoin%
\definecolor{currentfill}{rgb}{0.000000,0.000000,0.000000}%
\pgfsetfillcolor{currentfill}%
\pgfsetlinewidth{0.803000pt}%
\definecolor{currentstroke}{rgb}{0.000000,0.000000,0.000000}%
\pgfsetstrokecolor{currentstroke}%
\pgfsetdash{}{0pt}%
\pgfsys@defobject{currentmarker}{\pgfqpoint{-0.048611in}{0.000000in}}{\pgfqpoint{0.000000in}{0.000000in}}{%
\pgfpathmoveto{\pgfqpoint{0.000000in}{0.000000in}}%
\pgfpathlineto{\pgfqpoint{-0.048611in}{0.000000in}}%
\pgfusepath{stroke,fill}%
}%
\begin{pgfscope}%
\pgfsys@transformshift{0.691667in}{4.673611in}%
\pgfsys@useobject{currentmarker}{}%
\end{pgfscope}%
\end{pgfscope}%
\begin{pgfscope}%
\definecolor{textcolor}{rgb}{0.000000,0.000000,0.000000}%
\pgfsetstrokecolor{textcolor}%
\pgfsetfillcolor{textcolor}%
\pgftext[x=0.373565in,y=4.620850in,left,base]{\color{textcolor}\sffamily\fontsize{10.000000}{12.000000}\selectfont 0.0}%
\end{pgfscope}%
\begin{pgfscope}%
\pgfsetbuttcap%
\pgfsetroundjoin%
\definecolor{currentfill}{rgb}{0.000000,0.000000,0.000000}%
\pgfsetfillcolor{currentfill}%
\pgfsetlinewidth{0.803000pt}%
\definecolor{currentstroke}{rgb}{0.000000,0.000000,0.000000}%
\pgfsetstrokecolor{currentstroke}%
\pgfsetdash{}{0pt}%
\pgfsys@defobject{currentmarker}{\pgfqpoint{-0.048611in}{0.000000in}}{\pgfqpoint{0.000000in}{0.000000in}}{%
\pgfpathmoveto{\pgfqpoint{0.000000in}{0.000000in}}%
\pgfpathlineto{\pgfqpoint{-0.048611in}{0.000000in}}%
\pgfusepath{stroke,fill}%
}%
\begin{pgfscope}%
\pgfsys@transformshift{0.691667in}{5.235486in}%
\pgfsys@useobject{currentmarker}{}%
\end{pgfscope}%
\end{pgfscope}%
\begin{pgfscope}%
\definecolor{textcolor}{rgb}{0.000000,0.000000,0.000000}%
\pgfsetstrokecolor{textcolor}%
\pgfsetfillcolor{textcolor}%
\pgftext[x=0.373565in,y=5.182725in,left,base]{\color{textcolor}\sffamily\fontsize{10.000000}{12.000000}\selectfont 0.2}%
\end{pgfscope}%
\begin{pgfscope}%
\pgfsetbuttcap%
\pgfsetroundjoin%
\definecolor{currentfill}{rgb}{0.000000,0.000000,0.000000}%
\pgfsetfillcolor{currentfill}%
\pgfsetlinewidth{0.803000pt}%
\definecolor{currentstroke}{rgb}{0.000000,0.000000,0.000000}%
\pgfsetstrokecolor{currentstroke}%
\pgfsetdash{}{0pt}%
\pgfsys@defobject{currentmarker}{\pgfqpoint{-0.048611in}{0.000000in}}{\pgfqpoint{0.000000in}{0.000000in}}{%
\pgfpathmoveto{\pgfqpoint{0.000000in}{0.000000in}}%
\pgfpathlineto{\pgfqpoint{-0.048611in}{0.000000in}}%
\pgfusepath{stroke,fill}%
}%
\begin{pgfscope}%
\pgfsys@transformshift{0.691667in}{5.797361in}%
\pgfsys@useobject{currentmarker}{}%
\end{pgfscope}%
\end{pgfscope}%
\begin{pgfscope}%
\definecolor{textcolor}{rgb}{0.000000,0.000000,0.000000}%
\pgfsetstrokecolor{textcolor}%
\pgfsetfillcolor{textcolor}%
\pgftext[x=0.373565in,y=5.744600in,left,base]{\color{textcolor}\sffamily\fontsize{10.000000}{12.000000}\selectfont 0.4}%
\end{pgfscope}%
\begin{pgfscope}%
\pgfsetbuttcap%
\pgfsetroundjoin%
\definecolor{currentfill}{rgb}{0.000000,0.000000,0.000000}%
\pgfsetfillcolor{currentfill}%
\pgfsetlinewidth{0.803000pt}%
\definecolor{currentstroke}{rgb}{0.000000,0.000000,0.000000}%
\pgfsetstrokecolor{currentstroke}%
\pgfsetdash{}{0pt}%
\pgfsys@defobject{currentmarker}{\pgfqpoint{-0.048611in}{0.000000in}}{\pgfqpoint{0.000000in}{0.000000in}}{%
\pgfpathmoveto{\pgfqpoint{0.000000in}{0.000000in}}%
\pgfpathlineto{\pgfqpoint{-0.048611in}{0.000000in}}%
\pgfusepath{stroke,fill}%
}%
\begin{pgfscope}%
\pgfsys@transformshift{0.691667in}{6.359236in}%
\pgfsys@useobject{currentmarker}{}%
\end{pgfscope}%
\end{pgfscope}%
\begin{pgfscope}%
\definecolor{textcolor}{rgb}{0.000000,0.000000,0.000000}%
\pgfsetstrokecolor{textcolor}%
\pgfsetfillcolor{textcolor}%
\pgftext[x=0.373565in,y=6.306475in,left,base]{\color{textcolor}\sffamily\fontsize{10.000000}{12.000000}\selectfont 0.6}%
\end{pgfscope}%
\begin{pgfscope}%
\pgfsetbuttcap%
\pgfsetroundjoin%
\definecolor{currentfill}{rgb}{0.000000,0.000000,0.000000}%
\pgfsetfillcolor{currentfill}%
\pgfsetlinewidth{0.803000pt}%
\definecolor{currentstroke}{rgb}{0.000000,0.000000,0.000000}%
\pgfsetstrokecolor{currentstroke}%
\pgfsetdash{}{0pt}%
\pgfsys@defobject{currentmarker}{\pgfqpoint{-0.048611in}{0.000000in}}{\pgfqpoint{0.000000in}{0.000000in}}{%
\pgfpathmoveto{\pgfqpoint{0.000000in}{0.000000in}}%
\pgfpathlineto{\pgfqpoint{-0.048611in}{0.000000in}}%
\pgfusepath{stroke,fill}%
}%
\begin{pgfscope}%
\pgfsys@transformshift{0.691667in}{6.921111in}%
\pgfsys@useobject{currentmarker}{}%
\end{pgfscope}%
\end{pgfscope}%
\begin{pgfscope}%
\definecolor{textcolor}{rgb}{0.000000,0.000000,0.000000}%
\pgfsetstrokecolor{textcolor}%
\pgfsetfillcolor{textcolor}%
\pgftext[x=0.373565in,y=6.868350in,left,base]{\color{textcolor}\sffamily\fontsize{10.000000}{12.000000}\selectfont 0.8}%
\end{pgfscope}%
\begin{pgfscope}%
\pgfsetbuttcap%
\pgfsetroundjoin%
\definecolor{currentfill}{rgb}{0.000000,0.000000,0.000000}%
\pgfsetfillcolor{currentfill}%
\pgfsetlinewidth{0.803000pt}%
\definecolor{currentstroke}{rgb}{0.000000,0.000000,0.000000}%
\pgfsetstrokecolor{currentstroke}%
\pgfsetdash{}{0pt}%
\pgfsys@defobject{currentmarker}{\pgfqpoint{-0.048611in}{0.000000in}}{\pgfqpoint{0.000000in}{0.000000in}}{%
\pgfpathmoveto{\pgfqpoint{0.000000in}{0.000000in}}%
\pgfpathlineto{\pgfqpoint{-0.048611in}{0.000000in}}%
\pgfusepath{stroke,fill}%
}%
\begin{pgfscope}%
\pgfsys@transformshift{0.691667in}{7.482987in}%
\pgfsys@useobject{currentmarker}{}%
\end{pgfscope}%
\end{pgfscope}%
\begin{pgfscope}%
\definecolor{textcolor}{rgb}{0.000000,0.000000,0.000000}%
\pgfsetstrokecolor{textcolor}%
\pgfsetfillcolor{textcolor}%
\pgftext[x=0.373565in,y=7.430225in,left,base]{\color{textcolor}\sffamily\fontsize{10.000000}{12.000000}\selectfont 1.0}%
\end{pgfscope}%
\begin{pgfscope}%
\definecolor{textcolor}{rgb}{0.000000,0.000000,0.000000}%
\pgfsetstrokecolor{textcolor}%
\pgfsetfillcolor{textcolor}%
\pgftext[x=0.318009in,y=6.104167in,,bottom,rotate=90.000000]{\color{textcolor}\sffamily\fontsize{10.000000}{12.000000}\selectfont \(\displaystyle U\)}%
\end{pgfscope}%
\begin{pgfscope}%
\pgfpathrectangle{\pgfqpoint{0.691667in}{4.530556in}}{\pgfqpoint{3.140451in}{3.147222in}}%
\pgfusepath{clip}%
\pgfsetrectcap%
\pgfsetroundjoin%
\pgfsetlinewidth{1.505625pt}%
\definecolor{currentstroke}{rgb}{0.121569,0.466667,0.705882}%
\pgfsetstrokecolor{currentstroke}%
\pgfsetdash{}{0pt}%
\pgfpathmoveto{\pgfqpoint{0.834414in}{7.483077in}}%
\pgfpathlineto{\pgfqpoint{0.882804in}{7.483110in}}%
\pgfpathlineto{\pgfqpoint{0.931193in}{7.483091in}}%
\pgfpathlineto{\pgfqpoint{0.979582in}{7.483046in}}%
\pgfpathlineto{\pgfqpoint{1.027971in}{7.483073in}}%
\pgfpathlineto{\pgfqpoint{1.076360in}{7.483029in}}%
\pgfpathlineto{\pgfqpoint{1.124749in}{7.482962in}}%
\pgfpathlineto{\pgfqpoint{1.173138in}{7.483037in}}%
\pgfpathlineto{\pgfqpoint{1.221527in}{7.483034in}}%
\pgfpathlineto{\pgfqpoint{1.269916in}{7.482903in}}%
\pgfpathlineto{\pgfqpoint{1.318305in}{7.482975in}}%
\pgfpathlineto{\pgfqpoint{1.366694in}{7.483122in}}%
\pgfpathlineto{\pgfqpoint{1.415083in}{7.482961in}}%
\pgfpathlineto{\pgfqpoint{1.463473in}{7.482792in}}%
\pgfpathlineto{\pgfqpoint{1.511862in}{7.483042in}}%
\pgfpathlineto{\pgfqpoint{1.560251in}{7.483273in}}%
\pgfpathlineto{\pgfqpoint{1.608640in}{7.482946in}}%
\pgfpathlineto{\pgfqpoint{1.657029in}{7.482560in}}%
\pgfpathlineto{\pgfqpoint{1.705418in}{7.482885in}}%
\pgfpathlineto{\pgfqpoint{1.753807in}{7.483554in}}%
\pgfpathlineto{\pgfqpoint{1.802196in}{7.483476in}}%
\pgfpathlineto{\pgfqpoint{1.850585in}{7.482517in}}%
\pgfpathlineto{\pgfqpoint{1.898974in}{7.481884in}}%
\pgfpathlineto{\pgfqpoint{1.947363in}{7.482622in}}%
\pgfpathlineto{\pgfqpoint{1.995752in}{7.484212in}}%
\pgfpathlineto{\pgfqpoint{2.044141in}{7.484975in}}%
\pgfpathlineto{\pgfqpoint{2.092531in}{7.483742in}}%
\pgfpathlineto{\pgfqpoint{2.140920in}{7.481057in}}%
\pgfpathlineto{\pgfqpoint{2.189309in}{7.478844in}}%
\pgfpathlineto{\pgfqpoint{2.237698in}{7.479049in}}%
\pgfpathlineto{\pgfqpoint{2.286087in}{7.482770in}}%
\pgfpathlineto{\pgfqpoint{2.334476in}{7.489376in}}%
\pgfpathlineto{\pgfqpoint{2.382865in}{7.494639in}}%
\pgfpathlineto{\pgfqpoint{2.431254in}{7.499906in}}%
\pgfpathlineto{\pgfqpoint{2.479643in}{7.509491in}}%
\pgfpathlineto{\pgfqpoint{2.528032in}{7.471132in}}%
\pgfpathlineto{\pgfqpoint{2.576421in}{7.401302in}}%
\pgfpathlineto{\pgfqpoint{2.624810in}{7.534722in}}%
\pgfpathlineto{\pgfqpoint{2.673200in}{7.451657in}}%
\pgfpathlineto{\pgfqpoint{2.721589in}{6.385707in}}%
\pgfpathlineto{\pgfqpoint{2.769978in}{5.347388in}}%
\pgfpathlineto{\pgfqpoint{2.818367in}{4.939459in}}%
\pgfpathlineto{\pgfqpoint{2.866756in}{4.790864in}}%
\pgfpathlineto{\pgfqpoint{2.915145in}{4.720340in}}%
\pgfpathlineto{\pgfqpoint{2.963534in}{4.686028in}}%
\pgfpathlineto{\pgfqpoint{3.011923in}{4.674034in}}%
\pgfpathlineto{\pgfqpoint{3.060312in}{4.673611in}}%
\pgfpathlineto{\pgfqpoint{3.108701in}{4.673611in}}%
\pgfpathlineto{\pgfqpoint{3.157090in}{4.673611in}}%
\pgfpathlineto{\pgfqpoint{3.205479in}{4.673611in}}%
\pgfpathlineto{\pgfqpoint{3.253869in}{4.673611in}}%
\pgfpathlineto{\pgfqpoint{3.302258in}{4.673611in}}%
\pgfpathlineto{\pgfqpoint{3.350647in}{4.673611in}}%
\pgfpathlineto{\pgfqpoint{3.399036in}{4.673611in}}%
\pgfpathlineto{\pgfqpoint{3.447425in}{4.673611in}}%
\pgfpathlineto{\pgfqpoint{3.495814in}{4.673611in}}%
\pgfpathlineto{\pgfqpoint{3.544203in}{4.673611in}}%
\pgfpathlineto{\pgfqpoint{3.592592in}{4.673611in}}%
\pgfpathlineto{\pgfqpoint{3.640981in}{4.673611in}}%
\pgfpathlineto{\pgfqpoint{3.689370in}{4.673611in}}%
\pgfusepath{stroke}%
\end{pgfscope}%
\begin{pgfscope}%
\pgfpathrectangle{\pgfqpoint{0.691667in}{4.530556in}}{\pgfqpoint{3.140451in}{3.147222in}}%
\pgfusepath{clip}%
\pgfsetrectcap%
\pgfsetroundjoin%
\pgfsetlinewidth{1.505625pt}%
\definecolor{currentstroke}{rgb}{1.000000,0.498039,0.054902}%
\pgfsetstrokecolor{currentstroke}%
\pgfsetdash{}{0pt}%
\pgfpathmoveto{\pgfqpoint{0.834414in}{7.483077in}}%
\pgfpathlineto{\pgfqpoint{0.882804in}{7.483111in}}%
\pgfpathlineto{\pgfqpoint{0.931193in}{7.483091in}}%
\pgfpathlineto{\pgfqpoint{0.979582in}{7.483046in}}%
\pgfpathlineto{\pgfqpoint{1.027971in}{7.483073in}}%
\pgfpathlineto{\pgfqpoint{1.076360in}{7.483029in}}%
\pgfpathlineto{\pgfqpoint{1.124749in}{7.482962in}}%
\pgfpathlineto{\pgfqpoint{1.173138in}{7.483037in}}%
\pgfpathlineto{\pgfqpoint{1.221527in}{7.483034in}}%
\pgfpathlineto{\pgfqpoint{1.269916in}{7.482903in}}%
\pgfpathlineto{\pgfqpoint{1.318305in}{7.482975in}}%
\pgfpathlineto{\pgfqpoint{1.366694in}{7.483122in}}%
\pgfpathlineto{\pgfqpoint{1.415083in}{7.482961in}}%
\pgfpathlineto{\pgfqpoint{1.463473in}{7.482792in}}%
\pgfpathlineto{\pgfqpoint{1.511862in}{7.483042in}}%
\pgfpathlineto{\pgfqpoint{1.560251in}{7.483273in}}%
\pgfpathlineto{\pgfqpoint{1.608640in}{7.482946in}}%
\pgfpathlineto{\pgfqpoint{1.657029in}{7.482559in}}%
\pgfpathlineto{\pgfqpoint{1.705418in}{7.482885in}}%
\pgfpathlineto{\pgfqpoint{1.753807in}{7.483556in}}%
\pgfpathlineto{\pgfqpoint{1.802196in}{7.483477in}}%
\pgfpathlineto{\pgfqpoint{1.850585in}{7.482514in}}%
\pgfpathlineto{\pgfqpoint{1.898974in}{7.481879in}}%
\pgfpathlineto{\pgfqpoint{1.947363in}{7.482623in}}%
\pgfpathlineto{\pgfqpoint{1.995752in}{7.484221in}}%
\pgfpathlineto{\pgfqpoint{2.044141in}{7.484984in}}%
\pgfpathlineto{\pgfqpoint{2.092531in}{7.483740in}}%
\pgfpathlineto{\pgfqpoint{2.140920in}{7.481037in}}%
\pgfpathlineto{\pgfqpoint{2.189309in}{7.478804in}}%
\pgfpathlineto{\pgfqpoint{2.237698in}{7.479050in}}%
\pgfpathlineto{\pgfqpoint{2.286087in}{7.482849in}}%
\pgfpathlineto{\pgfqpoint{2.334476in}{7.489192in}}%
\pgfpathlineto{\pgfqpoint{2.382865in}{7.494990in}}%
\pgfpathlineto{\pgfqpoint{2.431254in}{7.501780in}}%
\pgfpathlineto{\pgfqpoint{2.479643in}{7.504507in}}%
\pgfpathlineto{\pgfqpoint{2.528032in}{7.466814in}}%
\pgfpathlineto{\pgfqpoint{2.576421in}{7.433525in}}%
\pgfpathlineto{\pgfqpoint{2.624810in}{7.512814in}}%
\pgfpathlineto{\pgfqpoint{2.673200in}{7.314949in}}%
\pgfpathlineto{\pgfqpoint{2.721589in}{6.405053in}}%
\pgfpathlineto{\pgfqpoint{2.769978in}{5.427910in}}%
\pgfpathlineto{\pgfqpoint{2.818367in}{4.964283in}}%
\pgfpathlineto{\pgfqpoint{2.866756in}{4.796915in}}%
\pgfpathlineto{\pgfqpoint{2.915145in}{4.722385in}}%
\pgfpathlineto{\pgfqpoint{2.963534in}{4.686743in}}%
\pgfpathlineto{\pgfqpoint{3.011923in}{4.674152in}}%
\pgfpathlineto{\pgfqpoint{3.060312in}{4.673611in}}%
\pgfpathlineto{\pgfqpoint{3.108701in}{4.673611in}}%
\pgfpathlineto{\pgfqpoint{3.157090in}{4.673611in}}%
\pgfpathlineto{\pgfqpoint{3.205479in}{4.673611in}}%
\pgfpathlineto{\pgfqpoint{3.253869in}{4.673611in}}%
\pgfpathlineto{\pgfqpoint{3.302258in}{4.673611in}}%
\pgfpathlineto{\pgfqpoint{3.350647in}{4.673611in}}%
\pgfpathlineto{\pgfqpoint{3.399036in}{4.673611in}}%
\pgfpathlineto{\pgfqpoint{3.447425in}{4.673611in}}%
\pgfpathlineto{\pgfqpoint{3.495814in}{4.673611in}}%
\pgfpathlineto{\pgfqpoint{3.544203in}{4.673611in}}%
\pgfpathlineto{\pgfqpoint{3.592592in}{4.673611in}}%
\pgfpathlineto{\pgfqpoint{3.640981in}{4.673611in}}%
\pgfpathlineto{\pgfqpoint{3.689370in}{4.673611in}}%
\pgfusepath{stroke}%
\end{pgfscope}%
\begin{pgfscope}%
\pgfpathrectangle{\pgfqpoint{0.691667in}{4.530556in}}{\pgfqpoint{3.140451in}{3.147222in}}%
\pgfusepath{clip}%
\pgfsetrectcap%
\pgfsetroundjoin%
\pgfsetlinewidth{1.505625pt}%
\definecolor{currentstroke}{rgb}{0.172549,0.627451,0.172549}%
\pgfsetstrokecolor{currentstroke}%
\pgfsetdash{}{0pt}%
\pgfpathmoveto{\pgfqpoint{0.834414in}{7.482987in}}%
\pgfpathlineto{\pgfqpoint{0.882804in}{7.482987in}}%
\pgfpathlineto{\pgfqpoint{0.931193in}{7.482987in}}%
\pgfpathlineto{\pgfqpoint{0.979582in}{7.482987in}}%
\pgfpathlineto{\pgfqpoint{1.027971in}{7.482987in}}%
\pgfpathlineto{\pgfqpoint{1.076360in}{7.482987in}}%
\pgfpathlineto{\pgfqpoint{1.124749in}{7.482987in}}%
\pgfpathlineto{\pgfqpoint{1.173138in}{7.482987in}}%
\pgfpathlineto{\pgfqpoint{1.221527in}{7.482987in}}%
\pgfpathlineto{\pgfqpoint{1.269916in}{7.482987in}}%
\pgfpathlineto{\pgfqpoint{1.318305in}{7.482987in}}%
\pgfpathlineto{\pgfqpoint{1.366694in}{7.482987in}}%
\pgfpathlineto{\pgfqpoint{1.415083in}{7.482987in}}%
\pgfpathlineto{\pgfqpoint{1.463473in}{7.482987in}}%
\pgfpathlineto{\pgfqpoint{1.511862in}{7.482987in}}%
\pgfpathlineto{\pgfqpoint{1.560251in}{7.482987in}}%
\pgfpathlineto{\pgfqpoint{1.608640in}{7.482987in}}%
\pgfpathlineto{\pgfqpoint{1.657029in}{7.482987in}}%
\pgfpathlineto{\pgfqpoint{1.705418in}{7.482986in}}%
\pgfpathlineto{\pgfqpoint{1.753807in}{7.482986in}}%
\pgfpathlineto{\pgfqpoint{1.802196in}{7.482986in}}%
\pgfpathlineto{\pgfqpoint{1.850585in}{7.482984in}}%
\pgfpathlineto{\pgfqpoint{1.898974in}{7.482976in}}%
\pgfpathlineto{\pgfqpoint{1.947363in}{7.482951in}}%
\pgfpathlineto{\pgfqpoint{1.995752in}{7.482883in}}%
\pgfpathlineto{\pgfqpoint{2.044141in}{7.482715in}}%
\pgfpathlineto{\pgfqpoint{2.092531in}{7.482329in}}%
\pgfpathlineto{\pgfqpoint{2.140920in}{7.481515in}}%
\pgfpathlineto{\pgfqpoint{2.189309in}{7.479909in}}%
\pgfpathlineto{\pgfqpoint{2.237698in}{7.476923in}}%
\pgfpathlineto{\pgfqpoint{2.286087in}{7.471659in}}%
\pgfpathlineto{\pgfqpoint{2.334476in}{7.462776in}}%
\pgfpathlineto{\pgfqpoint{2.382865in}{7.448328in}}%
\pgfpathlineto{\pgfqpoint{2.431254in}{7.425481in}}%
\pgfpathlineto{\pgfqpoint{2.479643in}{7.390050in}}%
\pgfpathlineto{\pgfqpoint{2.528032in}{7.335622in}}%
\pgfpathlineto{\pgfqpoint{2.576421in}{7.251841in}}%
\pgfpathlineto{\pgfqpoint{2.624810in}{7.120968in}}%
\pgfpathlineto{\pgfqpoint{2.673200in}{6.911396in}}%
\pgfpathlineto{\pgfqpoint{2.721589in}{6.570178in}}%
\pgfpathlineto{\pgfqpoint{2.769978in}{6.042938in}}%
\pgfpathlineto{\pgfqpoint{2.818367in}{5.402339in}}%
\pgfpathlineto{\pgfqpoint{2.866756in}{4.934893in}}%
\pgfpathlineto{\pgfqpoint{2.915145in}{4.749008in}}%
\pgfpathlineto{\pgfqpoint{2.963534in}{4.691221in}}%
\pgfpathlineto{\pgfqpoint{3.011923in}{4.674529in}}%
\pgfpathlineto{\pgfqpoint{3.060312in}{4.673612in}}%
\pgfpathlineto{\pgfqpoint{3.108701in}{4.673611in}}%
\pgfpathlineto{\pgfqpoint{3.157090in}{4.673611in}}%
\pgfpathlineto{\pgfqpoint{3.205479in}{4.673611in}}%
\pgfpathlineto{\pgfqpoint{3.253869in}{4.673611in}}%
\pgfpathlineto{\pgfqpoint{3.302258in}{4.673611in}}%
\pgfpathlineto{\pgfqpoint{3.350647in}{4.673611in}}%
\pgfpathlineto{\pgfqpoint{3.399036in}{4.673611in}}%
\pgfpathlineto{\pgfqpoint{3.447425in}{4.673611in}}%
\pgfpathlineto{\pgfqpoint{3.495814in}{4.673611in}}%
\pgfpathlineto{\pgfqpoint{3.544203in}{4.673611in}}%
\pgfpathlineto{\pgfqpoint{3.592592in}{4.673611in}}%
\pgfpathlineto{\pgfqpoint{3.640981in}{4.673611in}}%
\pgfpathlineto{\pgfqpoint{3.689370in}{4.673611in}}%
\pgfusepath{stroke}%
\end{pgfscope}%
\begin{pgfscope}%
\pgfpathrectangle{\pgfqpoint{0.691667in}{4.530556in}}{\pgfqpoint{3.140451in}{3.147222in}}%
\pgfusepath{clip}%
\pgfsetrectcap%
\pgfsetroundjoin%
\pgfsetlinewidth{1.505625pt}%
\definecolor{currentstroke}{rgb}{0.839216,0.152941,0.156863}%
\pgfsetstrokecolor{currentstroke}%
\pgfsetdash{}{0pt}%
\pgfpathmoveto{\pgfqpoint{0.834414in}{7.482987in}}%
\pgfpathlineto{\pgfqpoint{0.882804in}{7.482987in}}%
\pgfpathlineto{\pgfqpoint{0.931193in}{7.482987in}}%
\pgfpathlineto{\pgfqpoint{0.979582in}{7.482987in}}%
\pgfpathlineto{\pgfqpoint{1.027971in}{7.482987in}}%
\pgfpathlineto{\pgfqpoint{1.076360in}{7.482987in}}%
\pgfpathlineto{\pgfqpoint{1.124749in}{7.482987in}}%
\pgfpathlineto{\pgfqpoint{1.173138in}{7.482987in}}%
\pgfpathlineto{\pgfqpoint{1.221527in}{7.482987in}}%
\pgfpathlineto{\pgfqpoint{1.269916in}{7.482987in}}%
\pgfpathlineto{\pgfqpoint{1.318305in}{7.482987in}}%
\pgfpathlineto{\pgfqpoint{1.366694in}{7.482987in}}%
\pgfpathlineto{\pgfqpoint{1.415083in}{7.482987in}}%
\pgfpathlineto{\pgfqpoint{1.463473in}{7.482987in}}%
\pgfpathlineto{\pgfqpoint{1.511862in}{7.482987in}}%
\pgfpathlineto{\pgfqpoint{1.560251in}{7.482987in}}%
\pgfpathlineto{\pgfqpoint{1.608640in}{7.482987in}}%
\pgfpathlineto{\pgfqpoint{1.657029in}{7.482987in}}%
\pgfpathlineto{\pgfqpoint{1.705418in}{7.482986in}}%
\pgfpathlineto{\pgfqpoint{1.753807in}{7.482986in}}%
\pgfpathlineto{\pgfqpoint{1.802196in}{7.482986in}}%
\pgfpathlineto{\pgfqpoint{1.850585in}{7.482984in}}%
\pgfpathlineto{\pgfqpoint{1.898974in}{7.482975in}}%
\pgfpathlineto{\pgfqpoint{1.947363in}{7.482951in}}%
\pgfpathlineto{\pgfqpoint{1.995752in}{7.482882in}}%
\pgfpathlineto{\pgfqpoint{2.044141in}{7.482710in}}%
\pgfpathlineto{\pgfqpoint{2.092531in}{7.482316in}}%
\pgfpathlineto{\pgfqpoint{2.140920in}{7.481478in}}%
\pgfpathlineto{\pgfqpoint{2.189309in}{7.479815in}}%
\pgfpathlineto{\pgfqpoint{2.237698in}{7.476697in}}%
\pgfpathlineto{\pgfqpoint{2.286087in}{7.471142in}}%
\pgfpathlineto{\pgfqpoint{2.334476in}{7.461643in}}%
\pgfpathlineto{\pgfqpoint{2.382865in}{7.445920in}}%
\pgfpathlineto{\pgfqpoint{2.431254in}{7.420473in}}%
\pgfpathlineto{\pgfqpoint{2.479643in}{7.379728in}}%
\pgfpathlineto{\pgfqpoint{2.528032in}{7.314211in}}%
\pgfpathlineto{\pgfqpoint{2.576421in}{7.206370in}}%
\pgfpathlineto{\pgfqpoint{2.624810in}{7.020765in}}%
\pgfpathlineto{\pgfqpoint{2.673200in}{6.686334in}}%
\pgfpathlineto{\pgfqpoint{2.721589in}{6.115722in}}%
\pgfpathlineto{\pgfqpoint{2.769978in}{5.431523in}}%
\pgfpathlineto{\pgfqpoint{2.818367in}{4.990754in}}%
\pgfpathlineto{\pgfqpoint{2.866756in}{4.803764in}}%
\pgfpathlineto{\pgfqpoint{2.915145in}{4.722924in}}%
\pgfpathlineto{\pgfqpoint{2.963534in}{4.686193in}}%
\pgfpathlineto{\pgfqpoint{3.011923in}{4.673940in}}%
\pgfpathlineto{\pgfqpoint{3.060312in}{4.673611in}}%
\pgfpathlineto{\pgfqpoint{3.108701in}{4.673611in}}%
\pgfpathlineto{\pgfqpoint{3.157090in}{4.673611in}}%
\pgfpathlineto{\pgfqpoint{3.205479in}{4.673611in}}%
\pgfpathlineto{\pgfqpoint{3.253869in}{4.673611in}}%
\pgfpathlineto{\pgfqpoint{3.302258in}{4.673611in}}%
\pgfpathlineto{\pgfqpoint{3.350647in}{4.673611in}}%
\pgfpathlineto{\pgfqpoint{3.399036in}{4.673611in}}%
\pgfpathlineto{\pgfqpoint{3.447425in}{4.673611in}}%
\pgfpathlineto{\pgfqpoint{3.495814in}{4.673611in}}%
\pgfpathlineto{\pgfqpoint{3.544203in}{4.673611in}}%
\pgfpathlineto{\pgfqpoint{3.592592in}{4.673611in}}%
\pgfpathlineto{\pgfqpoint{3.640981in}{4.673611in}}%
\pgfpathlineto{\pgfqpoint{3.689370in}{4.673611in}}%
\pgfusepath{stroke}%
\end{pgfscope}%
\begin{pgfscope}%
\pgfpathrectangle{\pgfqpoint{0.691667in}{4.530556in}}{\pgfqpoint{3.140451in}{3.147222in}}%
\pgfusepath{clip}%
\pgfsetrectcap%
\pgfsetroundjoin%
\pgfsetlinewidth{0.501875pt}%
\definecolor{currentstroke}{rgb}{0.000000,0.000000,0.000000}%
\pgfsetstrokecolor{currentstroke}%
\pgfsetdash{}{0pt}%
\pgfpathmoveto{\pgfqpoint{0.834414in}{7.482987in}}%
\pgfpathlineto{\pgfqpoint{0.882804in}{7.482987in}}%
\pgfpathlineto{\pgfqpoint{0.931193in}{7.482987in}}%
\pgfpathlineto{\pgfqpoint{0.979582in}{7.482987in}}%
\pgfpathlineto{\pgfqpoint{1.027971in}{7.482987in}}%
\pgfpathlineto{\pgfqpoint{1.076360in}{7.482987in}}%
\pgfpathlineto{\pgfqpoint{1.124749in}{7.482987in}}%
\pgfpathlineto{\pgfqpoint{1.173138in}{7.482987in}}%
\pgfpathlineto{\pgfqpoint{1.221527in}{7.482987in}}%
\pgfpathlineto{\pgfqpoint{1.269916in}{7.482987in}}%
\pgfpathlineto{\pgfqpoint{1.318305in}{7.482987in}}%
\pgfpathlineto{\pgfqpoint{1.366694in}{7.482987in}}%
\pgfpathlineto{\pgfqpoint{1.415083in}{7.482987in}}%
\pgfpathlineto{\pgfqpoint{1.463473in}{7.482987in}}%
\pgfpathlineto{\pgfqpoint{1.511862in}{7.482987in}}%
\pgfpathlineto{\pgfqpoint{1.560251in}{7.482987in}}%
\pgfpathlineto{\pgfqpoint{1.608640in}{7.482987in}}%
\pgfpathlineto{\pgfqpoint{1.657029in}{7.482987in}}%
\pgfpathlineto{\pgfqpoint{1.705418in}{7.482987in}}%
\pgfpathlineto{\pgfqpoint{1.753807in}{7.482987in}}%
\pgfpathlineto{\pgfqpoint{1.802196in}{7.482987in}}%
\pgfpathlineto{\pgfqpoint{1.850585in}{7.482987in}}%
\pgfpathlineto{\pgfqpoint{1.898974in}{7.482987in}}%
\pgfpathlineto{\pgfqpoint{1.947363in}{7.482987in}}%
\pgfpathlineto{\pgfqpoint{1.995752in}{7.482987in}}%
\pgfpathlineto{\pgfqpoint{2.044141in}{7.482987in}}%
\pgfpathlineto{\pgfqpoint{2.092531in}{7.482987in}}%
\pgfpathlineto{\pgfqpoint{2.140920in}{7.482987in}}%
\pgfpathlineto{\pgfqpoint{2.189309in}{7.482987in}}%
\pgfpathlineto{\pgfqpoint{2.237698in}{7.482987in}}%
\pgfpathlineto{\pgfqpoint{2.286087in}{7.482987in}}%
\pgfpathlineto{\pgfqpoint{2.334476in}{7.482987in}}%
\pgfpathlineto{\pgfqpoint{2.382865in}{7.482987in}}%
\pgfpathlineto{\pgfqpoint{2.431254in}{7.482987in}}%
\pgfpathlineto{\pgfqpoint{2.479643in}{7.482676in}}%
\pgfpathlineto{\pgfqpoint{2.528032in}{7.471293in}}%
\pgfpathlineto{\pgfqpoint{2.576421in}{7.438443in}}%
\pgfpathlineto{\pgfqpoint{2.624810in}{7.372339in}}%
\pgfpathlineto{\pgfqpoint{2.673200in}{7.245765in}}%
\pgfpathlineto{\pgfqpoint{2.721589in}{6.955570in}}%
\pgfpathlineto{\pgfqpoint{2.769978in}{5.201027in}}%
\pgfpathlineto{\pgfqpoint{2.818367in}{4.910833in}}%
\pgfpathlineto{\pgfqpoint{2.866756in}{4.784259in}}%
\pgfpathlineto{\pgfqpoint{2.915145in}{4.718155in}}%
\pgfpathlineto{\pgfqpoint{2.963534in}{4.685304in}}%
\pgfpathlineto{\pgfqpoint{3.011923in}{4.673921in}}%
\pgfpathlineto{\pgfqpoint{3.060312in}{4.673611in}}%
\pgfpathlineto{\pgfqpoint{3.108701in}{4.673611in}}%
\pgfpathlineto{\pgfqpoint{3.157090in}{4.673611in}}%
\pgfpathlineto{\pgfqpoint{3.205479in}{4.673611in}}%
\pgfpathlineto{\pgfqpoint{3.253869in}{4.673611in}}%
\pgfpathlineto{\pgfqpoint{3.302258in}{4.673611in}}%
\pgfpathlineto{\pgfqpoint{3.350647in}{4.673611in}}%
\pgfpathlineto{\pgfqpoint{3.399036in}{4.673611in}}%
\pgfpathlineto{\pgfqpoint{3.447425in}{4.673611in}}%
\pgfpathlineto{\pgfqpoint{3.495814in}{4.673611in}}%
\pgfpathlineto{\pgfqpoint{3.544203in}{4.673611in}}%
\pgfpathlineto{\pgfqpoint{3.592592in}{4.673611in}}%
\pgfpathlineto{\pgfqpoint{3.640981in}{4.673611in}}%
\pgfpathlineto{\pgfqpoint{3.689370in}{4.673611in}}%
\pgfusepath{stroke}%
\end{pgfscope}%
\begin{pgfscope}%
\pgfsetrectcap%
\pgfsetmiterjoin%
\pgfsetlinewidth{0.803000pt}%
\definecolor{currentstroke}{rgb}{0.000000,0.000000,0.000000}%
\pgfsetstrokecolor{currentstroke}%
\pgfsetdash{}{0pt}%
\pgfpathmoveto{\pgfqpoint{0.691667in}{4.530556in}}%
\pgfpathlineto{\pgfqpoint{0.691667in}{7.677778in}}%
\pgfusepath{stroke}%
\end{pgfscope}%
\begin{pgfscope}%
\pgfsetrectcap%
\pgfsetmiterjoin%
\pgfsetlinewidth{0.803000pt}%
\definecolor{currentstroke}{rgb}{0.000000,0.000000,0.000000}%
\pgfsetstrokecolor{currentstroke}%
\pgfsetdash{}{0pt}%
\pgfpathmoveto{\pgfqpoint{3.832118in}{4.530556in}}%
\pgfpathlineto{\pgfqpoint{3.832118in}{7.677778in}}%
\pgfusepath{stroke}%
\end{pgfscope}%
\begin{pgfscope}%
\pgfsetrectcap%
\pgfsetmiterjoin%
\pgfsetlinewidth{0.803000pt}%
\definecolor{currentstroke}{rgb}{0.000000,0.000000,0.000000}%
\pgfsetstrokecolor{currentstroke}%
\pgfsetdash{}{0pt}%
\pgfpathmoveto{\pgfqpoint{0.691667in}{4.530556in}}%
\pgfpathlineto{\pgfqpoint{3.832118in}{4.530556in}}%
\pgfusepath{stroke}%
\end{pgfscope}%
\begin{pgfscope}%
\pgfsetrectcap%
\pgfsetmiterjoin%
\pgfsetlinewidth{0.803000pt}%
\definecolor{currentstroke}{rgb}{0.000000,0.000000,0.000000}%
\pgfsetstrokecolor{currentstroke}%
\pgfsetdash{}{0pt}%
\pgfpathmoveto{\pgfqpoint{0.691667in}{7.677778in}}%
\pgfpathlineto{\pgfqpoint{3.832118in}{7.677778in}}%
\pgfusepath{stroke}%
\end{pgfscope}%
\begin{pgfscope}%
\definecolor{textcolor}{rgb}{0.000000,0.000000,0.000000}%
\pgfsetstrokecolor{textcolor}%
\pgfsetfillcolor{textcolor}%
\pgftext[x=2.261892in,y=7.761111in,,base]{\color{textcolor}\sffamily\fontsize{12.000000}{14.400000}\selectfont \(\displaystyle  t = 2.0 \)}%
\end{pgfscope}%
\begin{pgfscope}%
\pgfsetbuttcap%
\pgfsetmiterjoin%
\definecolor{currentfill}{rgb}{1.000000,1.000000,1.000000}%
\pgfsetfillcolor{currentfill}%
\pgfsetlinewidth{0.000000pt}%
\definecolor{currentstroke}{rgb}{0.000000,0.000000,0.000000}%
\pgfsetstrokecolor{currentstroke}%
\pgfsetstrokeopacity{0.000000}%
\pgfsetdash{}{0pt}%
\pgfpathmoveto{\pgfqpoint{4.660937in}{4.530556in}}%
\pgfpathlineto{\pgfqpoint{7.801389in}{4.530556in}}%
\pgfpathlineto{\pgfqpoint{7.801389in}{7.677778in}}%
\pgfpathlineto{\pgfqpoint{4.660937in}{7.677778in}}%
\pgfpathclose%
\pgfusepath{fill}%
\end{pgfscope}%
\begin{pgfscope}%
\pgfsetbuttcap%
\pgfsetroundjoin%
\definecolor{currentfill}{rgb}{0.000000,0.000000,0.000000}%
\pgfsetfillcolor{currentfill}%
\pgfsetlinewidth{0.803000pt}%
\definecolor{currentstroke}{rgb}{0.000000,0.000000,0.000000}%
\pgfsetstrokecolor{currentstroke}%
\pgfsetdash{}{0pt}%
\pgfsys@defobject{currentmarker}{\pgfqpoint{0.000000in}{-0.048611in}}{\pgfqpoint{0.000000in}{0.000000in}}{%
\pgfpathmoveto{\pgfqpoint{0.000000in}{0.000000in}}%
\pgfpathlineto{\pgfqpoint{0.000000in}{-0.048611in}}%
\pgfusepath{stroke,fill}%
}%
\begin{pgfscope}%
\pgfsys@transformshift{4.779491in}{4.530556in}%
\pgfsys@useobject{currentmarker}{}%
\end{pgfscope}%
\end{pgfscope}%
\begin{pgfscope}%
\definecolor{textcolor}{rgb}{0.000000,0.000000,0.000000}%
\pgfsetstrokecolor{textcolor}%
\pgfsetfillcolor{textcolor}%
\pgftext[x=4.779491in,y=4.433333in,,top]{\color{textcolor}\sffamily\fontsize{10.000000}{12.000000}\selectfont −3}%
\end{pgfscope}%
\begin{pgfscope}%
\pgfsetbuttcap%
\pgfsetroundjoin%
\definecolor{currentfill}{rgb}{0.000000,0.000000,0.000000}%
\pgfsetfillcolor{currentfill}%
\pgfsetlinewidth{0.803000pt}%
\definecolor{currentstroke}{rgb}{0.000000,0.000000,0.000000}%
\pgfsetstrokecolor{currentstroke}%
\pgfsetdash{}{0pt}%
\pgfsys@defobject{currentmarker}{\pgfqpoint{0.000000in}{-0.048611in}}{\pgfqpoint{0.000000in}{0.000000in}}{%
\pgfpathmoveto{\pgfqpoint{0.000000in}{0.000000in}}%
\pgfpathlineto{\pgfqpoint{0.000000in}{-0.048611in}}%
\pgfusepath{stroke,fill}%
}%
\begin{pgfscope}%
\pgfsys@transformshift{5.263382in}{4.530556in}%
\pgfsys@useobject{currentmarker}{}%
\end{pgfscope}%
\end{pgfscope}%
\begin{pgfscope}%
\definecolor{textcolor}{rgb}{0.000000,0.000000,0.000000}%
\pgfsetstrokecolor{textcolor}%
\pgfsetfillcolor{textcolor}%
\pgftext[x=5.263382in,y=4.433333in,,top]{\color{textcolor}\sffamily\fontsize{10.000000}{12.000000}\selectfont −2}%
\end{pgfscope}%
\begin{pgfscope}%
\pgfsetbuttcap%
\pgfsetroundjoin%
\definecolor{currentfill}{rgb}{0.000000,0.000000,0.000000}%
\pgfsetfillcolor{currentfill}%
\pgfsetlinewidth{0.803000pt}%
\definecolor{currentstroke}{rgb}{0.000000,0.000000,0.000000}%
\pgfsetstrokecolor{currentstroke}%
\pgfsetdash{}{0pt}%
\pgfsys@defobject{currentmarker}{\pgfqpoint{0.000000in}{-0.048611in}}{\pgfqpoint{0.000000in}{0.000000in}}{%
\pgfpathmoveto{\pgfqpoint{0.000000in}{0.000000in}}%
\pgfpathlineto{\pgfqpoint{0.000000in}{-0.048611in}}%
\pgfusepath{stroke,fill}%
}%
\begin{pgfscope}%
\pgfsys@transformshift{5.747272in}{4.530556in}%
\pgfsys@useobject{currentmarker}{}%
\end{pgfscope}%
\end{pgfscope}%
\begin{pgfscope}%
\definecolor{textcolor}{rgb}{0.000000,0.000000,0.000000}%
\pgfsetstrokecolor{textcolor}%
\pgfsetfillcolor{textcolor}%
\pgftext[x=5.747272in,y=4.433333in,,top]{\color{textcolor}\sffamily\fontsize{10.000000}{12.000000}\selectfont −1}%
\end{pgfscope}%
\begin{pgfscope}%
\pgfsetbuttcap%
\pgfsetroundjoin%
\definecolor{currentfill}{rgb}{0.000000,0.000000,0.000000}%
\pgfsetfillcolor{currentfill}%
\pgfsetlinewidth{0.803000pt}%
\definecolor{currentstroke}{rgb}{0.000000,0.000000,0.000000}%
\pgfsetstrokecolor{currentstroke}%
\pgfsetdash{}{0pt}%
\pgfsys@defobject{currentmarker}{\pgfqpoint{0.000000in}{-0.048611in}}{\pgfqpoint{0.000000in}{0.000000in}}{%
\pgfpathmoveto{\pgfqpoint{0.000000in}{0.000000in}}%
\pgfpathlineto{\pgfqpoint{0.000000in}{-0.048611in}}%
\pgfusepath{stroke,fill}%
}%
\begin{pgfscope}%
\pgfsys@transformshift{6.231163in}{4.530556in}%
\pgfsys@useobject{currentmarker}{}%
\end{pgfscope}%
\end{pgfscope}%
\begin{pgfscope}%
\definecolor{textcolor}{rgb}{0.000000,0.000000,0.000000}%
\pgfsetstrokecolor{textcolor}%
\pgfsetfillcolor{textcolor}%
\pgftext[x=6.231163in,y=4.433333in,,top]{\color{textcolor}\sffamily\fontsize{10.000000}{12.000000}\selectfont 0}%
\end{pgfscope}%
\begin{pgfscope}%
\pgfsetbuttcap%
\pgfsetroundjoin%
\definecolor{currentfill}{rgb}{0.000000,0.000000,0.000000}%
\pgfsetfillcolor{currentfill}%
\pgfsetlinewidth{0.803000pt}%
\definecolor{currentstroke}{rgb}{0.000000,0.000000,0.000000}%
\pgfsetstrokecolor{currentstroke}%
\pgfsetdash{}{0pt}%
\pgfsys@defobject{currentmarker}{\pgfqpoint{0.000000in}{-0.048611in}}{\pgfqpoint{0.000000in}{0.000000in}}{%
\pgfpathmoveto{\pgfqpoint{0.000000in}{0.000000in}}%
\pgfpathlineto{\pgfqpoint{0.000000in}{-0.048611in}}%
\pgfusepath{stroke,fill}%
}%
\begin{pgfscope}%
\pgfsys@transformshift{6.715054in}{4.530556in}%
\pgfsys@useobject{currentmarker}{}%
\end{pgfscope}%
\end{pgfscope}%
\begin{pgfscope}%
\definecolor{textcolor}{rgb}{0.000000,0.000000,0.000000}%
\pgfsetstrokecolor{textcolor}%
\pgfsetfillcolor{textcolor}%
\pgftext[x=6.715054in,y=4.433333in,,top]{\color{textcolor}\sffamily\fontsize{10.000000}{12.000000}\selectfont 1}%
\end{pgfscope}%
\begin{pgfscope}%
\pgfsetbuttcap%
\pgfsetroundjoin%
\definecolor{currentfill}{rgb}{0.000000,0.000000,0.000000}%
\pgfsetfillcolor{currentfill}%
\pgfsetlinewidth{0.803000pt}%
\definecolor{currentstroke}{rgb}{0.000000,0.000000,0.000000}%
\pgfsetstrokecolor{currentstroke}%
\pgfsetdash{}{0pt}%
\pgfsys@defobject{currentmarker}{\pgfqpoint{0.000000in}{-0.048611in}}{\pgfqpoint{0.000000in}{0.000000in}}{%
\pgfpathmoveto{\pgfqpoint{0.000000in}{0.000000in}}%
\pgfpathlineto{\pgfqpoint{0.000000in}{-0.048611in}}%
\pgfusepath{stroke,fill}%
}%
\begin{pgfscope}%
\pgfsys@transformshift{7.198945in}{4.530556in}%
\pgfsys@useobject{currentmarker}{}%
\end{pgfscope}%
\end{pgfscope}%
\begin{pgfscope}%
\definecolor{textcolor}{rgb}{0.000000,0.000000,0.000000}%
\pgfsetstrokecolor{textcolor}%
\pgfsetfillcolor{textcolor}%
\pgftext[x=7.198945in,y=4.433333in,,top]{\color{textcolor}\sffamily\fontsize{10.000000}{12.000000}\selectfont 2}%
\end{pgfscope}%
\begin{pgfscope}%
\pgfsetbuttcap%
\pgfsetroundjoin%
\definecolor{currentfill}{rgb}{0.000000,0.000000,0.000000}%
\pgfsetfillcolor{currentfill}%
\pgfsetlinewidth{0.803000pt}%
\definecolor{currentstroke}{rgb}{0.000000,0.000000,0.000000}%
\pgfsetstrokecolor{currentstroke}%
\pgfsetdash{}{0pt}%
\pgfsys@defobject{currentmarker}{\pgfqpoint{0.000000in}{-0.048611in}}{\pgfqpoint{0.000000in}{0.000000in}}{%
\pgfpathmoveto{\pgfqpoint{0.000000in}{0.000000in}}%
\pgfpathlineto{\pgfqpoint{0.000000in}{-0.048611in}}%
\pgfusepath{stroke,fill}%
}%
\begin{pgfscope}%
\pgfsys@transformshift{7.682836in}{4.530556in}%
\pgfsys@useobject{currentmarker}{}%
\end{pgfscope}%
\end{pgfscope}%
\begin{pgfscope}%
\definecolor{textcolor}{rgb}{0.000000,0.000000,0.000000}%
\pgfsetstrokecolor{textcolor}%
\pgfsetfillcolor{textcolor}%
\pgftext[x=7.682836in,y=4.433333in,,top]{\color{textcolor}\sffamily\fontsize{10.000000}{12.000000}\selectfont 3}%
\end{pgfscope}%
\begin{pgfscope}%
\definecolor{textcolor}{rgb}{0.000000,0.000000,0.000000}%
\pgfsetstrokecolor{textcolor}%
\pgfsetfillcolor{textcolor}%
\pgftext[x=6.231163in,y=4.243365in,,top]{\color{textcolor}\sffamily\fontsize{10.000000}{12.000000}\selectfont \(\displaystyle x\)}%
\end{pgfscope}%
\begin{pgfscope}%
\pgfsetbuttcap%
\pgfsetroundjoin%
\definecolor{currentfill}{rgb}{0.000000,0.000000,0.000000}%
\pgfsetfillcolor{currentfill}%
\pgfsetlinewidth{0.803000pt}%
\definecolor{currentstroke}{rgb}{0.000000,0.000000,0.000000}%
\pgfsetstrokecolor{currentstroke}%
\pgfsetdash{}{0pt}%
\pgfsys@defobject{currentmarker}{\pgfqpoint{-0.048611in}{0.000000in}}{\pgfqpoint{0.000000in}{0.000000in}}{%
\pgfpathmoveto{\pgfqpoint{0.000000in}{0.000000in}}%
\pgfpathlineto{\pgfqpoint{-0.048611in}{0.000000in}}%
\pgfusepath{stroke,fill}%
}%
\begin{pgfscope}%
\pgfsys@transformshift{4.660937in}{4.673611in}%
\pgfsys@useobject{currentmarker}{}%
\end{pgfscope}%
\end{pgfscope}%
\begin{pgfscope}%
\definecolor{textcolor}{rgb}{0.000000,0.000000,0.000000}%
\pgfsetstrokecolor{textcolor}%
\pgfsetfillcolor{textcolor}%
\pgftext[x=4.254470in,y=4.620850in,left,base]{\color{textcolor}\sffamily\fontsize{10.000000}{12.000000}\selectfont 0.00}%
\end{pgfscope}%
\begin{pgfscope}%
\pgfsetbuttcap%
\pgfsetroundjoin%
\definecolor{currentfill}{rgb}{0.000000,0.000000,0.000000}%
\pgfsetfillcolor{currentfill}%
\pgfsetlinewidth{0.803000pt}%
\definecolor{currentstroke}{rgb}{0.000000,0.000000,0.000000}%
\pgfsetstrokecolor{currentstroke}%
\pgfsetdash{}{0pt}%
\pgfsys@defobject{currentmarker}{\pgfqpoint{-0.048611in}{0.000000in}}{\pgfqpoint{0.000000in}{0.000000in}}{%
\pgfpathmoveto{\pgfqpoint{0.000000in}{0.000000in}}%
\pgfpathlineto{\pgfqpoint{-0.048611in}{0.000000in}}%
\pgfusepath{stroke,fill}%
}%
\begin{pgfscope}%
\pgfsys@transformshift{4.660937in}{5.092614in}%
\pgfsys@useobject{currentmarker}{}%
\end{pgfscope}%
\end{pgfscope}%
\begin{pgfscope}%
\definecolor{textcolor}{rgb}{0.000000,0.000000,0.000000}%
\pgfsetstrokecolor{textcolor}%
\pgfsetfillcolor{textcolor}%
\pgftext[x=4.254470in,y=5.039853in,left,base]{\color{textcolor}\sffamily\fontsize{10.000000}{12.000000}\selectfont 0.25}%
\end{pgfscope}%
\begin{pgfscope}%
\pgfsetbuttcap%
\pgfsetroundjoin%
\definecolor{currentfill}{rgb}{0.000000,0.000000,0.000000}%
\pgfsetfillcolor{currentfill}%
\pgfsetlinewidth{0.803000pt}%
\definecolor{currentstroke}{rgb}{0.000000,0.000000,0.000000}%
\pgfsetstrokecolor{currentstroke}%
\pgfsetdash{}{0pt}%
\pgfsys@defobject{currentmarker}{\pgfqpoint{-0.048611in}{0.000000in}}{\pgfqpoint{0.000000in}{0.000000in}}{%
\pgfpathmoveto{\pgfqpoint{0.000000in}{0.000000in}}%
\pgfpathlineto{\pgfqpoint{-0.048611in}{0.000000in}}%
\pgfusepath{stroke,fill}%
}%
\begin{pgfscope}%
\pgfsys@transformshift{4.660937in}{5.511617in}%
\pgfsys@useobject{currentmarker}{}%
\end{pgfscope}%
\end{pgfscope}%
\begin{pgfscope}%
\definecolor{textcolor}{rgb}{0.000000,0.000000,0.000000}%
\pgfsetstrokecolor{textcolor}%
\pgfsetfillcolor{textcolor}%
\pgftext[x=4.254470in,y=5.458856in,left,base]{\color{textcolor}\sffamily\fontsize{10.000000}{12.000000}\selectfont 0.50}%
\end{pgfscope}%
\begin{pgfscope}%
\pgfsetbuttcap%
\pgfsetroundjoin%
\definecolor{currentfill}{rgb}{0.000000,0.000000,0.000000}%
\pgfsetfillcolor{currentfill}%
\pgfsetlinewidth{0.803000pt}%
\definecolor{currentstroke}{rgb}{0.000000,0.000000,0.000000}%
\pgfsetstrokecolor{currentstroke}%
\pgfsetdash{}{0pt}%
\pgfsys@defobject{currentmarker}{\pgfqpoint{-0.048611in}{0.000000in}}{\pgfqpoint{0.000000in}{0.000000in}}{%
\pgfpathmoveto{\pgfqpoint{0.000000in}{0.000000in}}%
\pgfpathlineto{\pgfqpoint{-0.048611in}{0.000000in}}%
\pgfusepath{stroke,fill}%
}%
\begin{pgfscope}%
\pgfsys@transformshift{4.660937in}{5.930620in}%
\pgfsys@useobject{currentmarker}{}%
\end{pgfscope}%
\end{pgfscope}%
\begin{pgfscope}%
\definecolor{textcolor}{rgb}{0.000000,0.000000,0.000000}%
\pgfsetstrokecolor{textcolor}%
\pgfsetfillcolor{textcolor}%
\pgftext[x=4.254470in,y=5.877859in,left,base]{\color{textcolor}\sffamily\fontsize{10.000000}{12.000000}\selectfont 0.75}%
\end{pgfscope}%
\begin{pgfscope}%
\pgfsetbuttcap%
\pgfsetroundjoin%
\definecolor{currentfill}{rgb}{0.000000,0.000000,0.000000}%
\pgfsetfillcolor{currentfill}%
\pgfsetlinewidth{0.803000pt}%
\definecolor{currentstroke}{rgb}{0.000000,0.000000,0.000000}%
\pgfsetstrokecolor{currentstroke}%
\pgfsetdash{}{0pt}%
\pgfsys@defobject{currentmarker}{\pgfqpoint{-0.048611in}{0.000000in}}{\pgfqpoint{0.000000in}{0.000000in}}{%
\pgfpathmoveto{\pgfqpoint{0.000000in}{0.000000in}}%
\pgfpathlineto{\pgfqpoint{-0.048611in}{0.000000in}}%
\pgfusepath{stroke,fill}%
}%
\begin{pgfscope}%
\pgfsys@transformshift{4.660937in}{6.349623in}%
\pgfsys@useobject{currentmarker}{}%
\end{pgfscope}%
\end{pgfscope}%
\begin{pgfscope}%
\definecolor{textcolor}{rgb}{0.000000,0.000000,0.000000}%
\pgfsetstrokecolor{textcolor}%
\pgfsetfillcolor{textcolor}%
\pgftext[x=4.254470in,y=6.296861in,left,base]{\color{textcolor}\sffamily\fontsize{10.000000}{12.000000}\selectfont 1.00}%
\end{pgfscope}%
\begin{pgfscope}%
\pgfsetbuttcap%
\pgfsetroundjoin%
\definecolor{currentfill}{rgb}{0.000000,0.000000,0.000000}%
\pgfsetfillcolor{currentfill}%
\pgfsetlinewidth{0.803000pt}%
\definecolor{currentstroke}{rgb}{0.000000,0.000000,0.000000}%
\pgfsetstrokecolor{currentstroke}%
\pgfsetdash{}{0pt}%
\pgfsys@defobject{currentmarker}{\pgfqpoint{-0.048611in}{0.000000in}}{\pgfqpoint{0.000000in}{0.000000in}}{%
\pgfpathmoveto{\pgfqpoint{0.000000in}{0.000000in}}%
\pgfpathlineto{\pgfqpoint{-0.048611in}{0.000000in}}%
\pgfusepath{stroke,fill}%
}%
\begin{pgfscope}%
\pgfsys@transformshift{4.660937in}{6.768626in}%
\pgfsys@useobject{currentmarker}{}%
\end{pgfscope}%
\end{pgfscope}%
\begin{pgfscope}%
\definecolor{textcolor}{rgb}{0.000000,0.000000,0.000000}%
\pgfsetstrokecolor{textcolor}%
\pgfsetfillcolor{textcolor}%
\pgftext[x=4.254470in,y=6.715864in,left,base]{\color{textcolor}\sffamily\fontsize{10.000000}{12.000000}\selectfont 1.25}%
\end{pgfscope}%
\begin{pgfscope}%
\pgfsetbuttcap%
\pgfsetroundjoin%
\definecolor{currentfill}{rgb}{0.000000,0.000000,0.000000}%
\pgfsetfillcolor{currentfill}%
\pgfsetlinewidth{0.803000pt}%
\definecolor{currentstroke}{rgb}{0.000000,0.000000,0.000000}%
\pgfsetstrokecolor{currentstroke}%
\pgfsetdash{}{0pt}%
\pgfsys@defobject{currentmarker}{\pgfqpoint{-0.048611in}{0.000000in}}{\pgfqpoint{0.000000in}{0.000000in}}{%
\pgfpathmoveto{\pgfqpoint{0.000000in}{0.000000in}}%
\pgfpathlineto{\pgfqpoint{-0.048611in}{0.000000in}}%
\pgfusepath{stroke,fill}%
}%
\begin{pgfscope}%
\pgfsys@transformshift{4.660937in}{7.187629in}%
\pgfsys@useobject{currentmarker}{}%
\end{pgfscope}%
\end{pgfscope}%
\begin{pgfscope}%
\definecolor{textcolor}{rgb}{0.000000,0.000000,0.000000}%
\pgfsetstrokecolor{textcolor}%
\pgfsetfillcolor{textcolor}%
\pgftext[x=4.254470in,y=7.134867in,left,base]{\color{textcolor}\sffamily\fontsize{10.000000}{12.000000}\selectfont 1.50}%
\end{pgfscope}%
\begin{pgfscope}%
\pgfsetbuttcap%
\pgfsetroundjoin%
\definecolor{currentfill}{rgb}{0.000000,0.000000,0.000000}%
\pgfsetfillcolor{currentfill}%
\pgfsetlinewidth{0.803000pt}%
\definecolor{currentstroke}{rgb}{0.000000,0.000000,0.000000}%
\pgfsetstrokecolor{currentstroke}%
\pgfsetdash{}{0pt}%
\pgfsys@defobject{currentmarker}{\pgfqpoint{-0.048611in}{0.000000in}}{\pgfqpoint{0.000000in}{0.000000in}}{%
\pgfpathmoveto{\pgfqpoint{0.000000in}{0.000000in}}%
\pgfpathlineto{\pgfqpoint{-0.048611in}{0.000000in}}%
\pgfusepath{stroke,fill}%
}%
\begin{pgfscope}%
\pgfsys@transformshift{4.660937in}{7.606632in}%
\pgfsys@useobject{currentmarker}{}%
\end{pgfscope}%
\end{pgfscope}%
\begin{pgfscope}%
\definecolor{textcolor}{rgb}{0.000000,0.000000,0.000000}%
\pgfsetstrokecolor{textcolor}%
\pgfsetfillcolor{textcolor}%
\pgftext[x=4.254470in,y=7.553870in,left,base]{\color{textcolor}\sffamily\fontsize{10.000000}{12.000000}\selectfont 1.75}%
\end{pgfscope}%
\begin{pgfscope}%
\definecolor{textcolor}{rgb}{0.000000,0.000000,0.000000}%
\pgfsetstrokecolor{textcolor}%
\pgfsetfillcolor{textcolor}%
\pgftext[x=4.198915in,y=6.104167in,,bottom,rotate=90.000000]{\color{textcolor}\sffamily\fontsize{10.000000}{12.000000}\selectfont \(\displaystyle U\)}%
\end{pgfscope}%
\begin{pgfscope}%
\pgfpathrectangle{\pgfqpoint{4.660937in}{4.530556in}}{\pgfqpoint{3.140451in}{3.147222in}}%
\pgfusepath{clip}%
\pgfsetrectcap%
\pgfsetroundjoin%
\pgfsetlinewidth{1.505625pt}%
\definecolor{currentstroke}{rgb}{0.121569,0.466667,0.705882}%
\pgfsetstrokecolor{currentstroke}%
\pgfsetdash{}{0pt}%
\pgfpathmoveto{\pgfqpoint{4.803685in}{6.349666in}}%
\pgfpathlineto{\pgfqpoint{4.852074in}{6.349665in}}%
\pgfpathlineto{\pgfqpoint{4.900463in}{6.349685in}}%
\pgfpathlineto{\pgfqpoint{4.948853in}{6.349687in}}%
\pgfpathlineto{\pgfqpoint{4.997242in}{6.349659in}}%
\pgfpathlineto{\pgfqpoint{5.045631in}{6.349660in}}%
\pgfpathlineto{\pgfqpoint{5.094020in}{6.349676in}}%
\pgfpathlineto{\pgfqpoint{5.142409in}{6.349653in}}%
\pgfpathlineto{\pgfqpoint{5.190798in}{6.349652in}}%
\pgfpathlineto{\pgfqpoint{5.239187in}{6.349705in}}%
\pgfpathlineto{\pgfqpoint{5.287576in}{6.349709in}}%
\pgfpathlineto{\pgfqpoint{5.335965in}{6.349648in}}%
\pgfpathlineto{\pgfqpoint{5.384354in}{6.349656in}}%
\pgfpathlineto{\pgfqpoint{5.432743in}{6.349723in}}%
\pgfpathlineto{\pgfqpoint{5.481132in}{6.349683in}}%
\pgfpathlineto{\pgfqpoint{5.529522in}{6.349569in}}%
\pgfpathlineto{\pgfqpoint{5.577911in}{6.349592in}}%
\pgfpathlineto{\pgfqpoint{5.626300in}{6.349728in}}%
\pgfpathlineto{\pgfqpoint{5.674689in}{6.349714in}}%
\pgfpathlineto{\pgfqpoint{5.723078in}{6.349523in}}%
\pgfpathlineto{\pgfqpoint{5.771467in}{6.349466in}}%
\pgfpathlineto{\pgfqpoint{5.819856in}{6.349688in}}%
\pgfpathlineto{\pgfqpoint{5.868245in}{6.349873in}}%
\pgfpathlineto{\pgfqpoint{5.916634in}{6.349682in}}%
\pgfpathlineto{\pgfqpoint{5.965023in}{6.349359in}}%
\pgfpathlineto{\pgfqpoint{6.013412in}{6.349309in}}%
\pgfpathlineto{\pgfqpoint{6.061801in}{6.349604in}}%
\pgfpathlineto{\pgfqpoint{6.110190in}{6.350528in}}%
\pgfpathlineto{\pgfqpoint{6.158580in}{6.349726in}}%
\pgfpathlineto{\pgfqpoint{6.206969in}{6.348086in}}%
\pgfpathlineto{\pgfqpoint{6.255358in}{6.352909in}}%
\pgfpathlineto{\pgfqpoint{6.303747in}{6.345276in}}%
\pgfpathlineto{\pgfqpoint{6.352136in}{6.340891in}}%
\pgfpathlineto{\pgfqpoint{6.400525in}{6.381279in}}%
\pgfpathlineto{\pgfqpoint{6.448914in}{6.330070in}}%
\pgfpathlineto{\pgfqpoint{6.497303in}{6.288904in}}%
\pgfpathlineto{\pgfqpoint{6.545692in}{6.490375in}}%
\pgfpathlineto{\pgfqpoint{6.594081in}{6.324098in}}%
\pgfpathlineto{\pgfqpoint{6.642470in}{6.035318in}}%
\pgfpathlineto{\pgfqpoint{6.690859in}{6.666894in}}%
\pgfpathlineto{\pgfqpoint{6.739249in}{6.643012in}}%
\pgfpathlineto{\pgfqpoint{6.787638in}{5.557991in}}%
\pgfpathlineto{\pgfqpoint{6.836027in}{6.187275in}}%
\pgfpathlineto{\pgfqpoint{6.884416in}{7.534722in}}%
\pgfpathlineto{\pgfqpoint{6.932805in}{5.735676in}}%
\pgfpathlineto{\pgfqpoint{6.981194in}{4.704163in}}%
\pgfpathlineto{\pgfqpoint{7.029583in}{4.673620in}}%
\pgfpathlineto{\pgfqpoint{7.077972in}{4.673611in}}%
\pgfpathlineto{\pgfqpoint{7.126361in}{4.673611in}}%
\pgfpathlineto{\pgfqpoint{7.174750in}{4.673611in}}%
\pgfpathlineto{\pgfqpoint{7.223139in}{4.673611in}}%
\pgfpathlineto{\pgfqpoint{7.271528in}{4.673611in}}%
\pgfpathlineto{\pgfqpoint{7.319918in}{4.673611in}}%
\pgfpathlineto{\pgfqpoint{7.368307in}{4.673611in}}%
\pgfpathlineto{\pgfqpoint{7.416696in}{4.673611in}}%
\pgfpathlineto{\pgfqpoint{7.465085in}{4.673611in}}%
\pgfpathlineto{\pgfqpoint{7.513474in}{4.673611in}}%
\pgfpathlineto{\pgfqpoint{7.561863in}{4.673611in}}%
\pgfpathlineto{\pgfqpoint{7.610252in}{4.673611in}}%
\pgfpathlineto{\pgfqpoint{7.658641in}{4.673611in}}%
\pgfusepath{stroke}%
\end{pgfscope}%
\begin{pgfscope}%
\pgfpathrectangle{\pgfqpoint{4.660937in}{4.530556in}}{\pgfqpoint{3.140451in}{3.147222in}}%
\pgfusepath{clip}%
\pgfsetrectcap%
\pgfsetroundjoin%
\pgfsetlinewidth{1.505625pt}%
\definecolor{currentstroke}{rgb}{1.000000,0.498039,0.054902}%
\pgfsetstrokecolor{currentstroke}%
\pgfsetdash{}{0pt}%
\pgfpathmoveto{\pgfqpoint{4.803685in}{6.349666in}}%
\pgfpathlineto{\pgfqpoint{4.852074in}{6.349665in}}%
\pgfpathlineto{\pgfqpoint{4.900463in}{6.349685in}}%
\pgfpathlineto{\pgfqpoint{4.948853in}{6.349687in}}%
\pgfpathlineto{\pgfqpoint{4.997242in}{6.349659in}}%
\pgfpathlineto{\pgfqpoint{5.045631in}{6.349660in}}%
\pgfpathlineto{\pgfqpoint{5.094020in}{6.349676in}}%
\pgfpathlineto{\pgfqpoint{5.142409in}{6.349653in}}%
\pgfpathlineto{\pgfqpoint{5.190798in}{6.349652in}}%
\pgfpathlineto{\pgfqpoint{5.239187in}{6.349705in}}%
\pgfpathlineto{\pgfqpoint{5.287576in}{6.349709in}}%
\pgfpathlineto{\pgfqpoint{5.335965in}{6.349648in}}%
\pgfpathlineto{\pgfqpoint{5.384354in}{6.349656in}}%
\pgfpathlineto{\pgfqpoint{5.432743in}{6.349723in}}%
\pgfpathlineto{\pgfqpoint{5.481132in}{6.349683in}}%
\pgfpathlineto{\pgfqpoint{5.529522in}{6.349569in}}%
\pgfpathlineto{\pgfqpoint{5.577911in}{6.349592in}}%
\pgfpathlineto{\pgfqpoint{5.626300in}{6.349728in}}%
\pgfpathlineto{\pgfqpoint{5.674689in}{6.349714in}}%
\pgfpathlineto{\pgfqpoint{5.723078in}{6.349522in}}%
\pgfpathlineto{\pgfqpoint{5.771467in}{6.349466in}}%
\pgfpathlineto{\pgfqpoint{5.819856in}{6.349689in}}%
\pgfpathlineto{\pgfqpoint{5.868245in}{6.349872in}}%
\pgfpathlineto{\pgfqpoint{5.916634in}{6.349689in}}%
\pgfpathlineto{\pgfqpoint{5.965023in}{6.349353in}}%
\pgfpathlineto{\pgfqpoint{6.013412in}{6.349283in}}%
\pgfpathlineto{\pgfqpoint{6.061801in}{6.349710in}}%
\pgfpathlineto{\pgfqpoint{6.110190in}{6.350375in}}%
\pgfpathlineto{\pgfqpoint{6.158580in}{6.349599in}}%
\pgfpathlineto{\pgfqpoint{6.206969in}{6.349158in}}%
\pgfpathlineto{\pgfqpoint{6.255358in}{6.350941in}}%
\pgfpathlineto{\pgfqpoint{6.303747in}{6.345187in}}%
\pgfpathlineto{\pgfqpoint{6.352136in}{6.348427in}}%
\pgfpathlineto{\pgfqpoint{6.400525in}{6.367398in}}%
\pgfpathlineto{\pgfqpoint{6.448914in}{6.330372in}}%
\pgfpathlineto{\pgfqpoint{6.497303in}{6.328000in}}%
\pgfpathlineto{\pgfqpoint{6.545692in}{6.433311in}}%
\pgfpathlineto{\pgfqpoint{6.594081in}{6.305185in}}%
\pgfpathlineto{\pgfqpoint{6.642470in}{6.173940in}}%
\pgfpathlineto{\pgfqpoint{6.690859in}{6.563060in}}%
\pgfpathlineto{\pgfqpoint{6.739249in}{6.492592in}}%
\pgfpathlineto{\pgfqpoint{6.787638in}{5.800306in}}%
\pgfpathlineto{\pgfqpoint{6.836027in}{6.299733in}}%
\pgfpathlineto{\pgfqpoint{6.884416in}{7.026197in}}%
\pgfpathlineto{\pgfqpoint{6.932805in}{5.952734in}}%
\pgfpathlineto{\pgfqpoint{6.981194in}{4.800138in}}%
\pgfpathlineto{\pgfqpoint{7.029583in}{4.674082in}}%
\pgfpathlineto{\pgfqpoint{7.077972in}{4.673611in}}%
\pgfpathlineto{\pgfqpoint{7.126361in}{4.673611in}}%
\pgfpathlineto{\pgfqpoint{7.174750in}{4.673611in}}%
\pgfpathlineto{\pgfqpoint{7.223139in}{4.673611in}}%
\pgfpathlineto{\pgfqpoint{7.271528in}{4.673611in}}%
\pgfpathlineto{\pgfqpoint{7.319918in}{4.673611in}}%
\pgfpathlineto{\pgfqpoint{7.368307in}{4.673611in}}%
\pgfpathlineto{\pgfqpoint{7.416696in}{4.673611in}}%
\pgfpathlineto{\pgfqpoint{7.465085in}{4.673611in}}%
\pgfpathlineto{\pgfqpoint{7.513474in}{4.673611in}}%
\pgfpathlineto{\pgfqpoint{7.561863in}{4.673611in}}%
\pgfpathlineto{\pgfqpoint{7.610252in}{4.673611in}}%
\pgfpathlineto{\pgfqpoint{7.658641in}{4.673611in}}%
\pgfusepath{stroke}%
\end{pgfscope}%
\begin{pgfscope}%
\pgfpathrectangle{\pgfqpoint{4.660937in}{4.530556in}}{\pgfqpoint{3.140451in}{3.147222in}}%
\pgfusepath{clip}%
\pgfsetrectcap%
\pgfsetroundjoin%
\pgfsetlinewidth{1.505625pt}%
\definecolor{currentstroke}{rgb}{0.172549,0.627451,0.172549}%
\pgfsetstrokecolor{currentstroke}%
\pgfsetdash{}{0pt}%
\pgfpathmoveto{\pgfqpoint{4.803685in}{6.349623in}}%
\pgfpathlineto{\pgfqpoint{4.852074in}{6.349623in}}%
\pgfpathlineto{\pgfqpoint{4.900463in}{6.349623in}}%
\pgfpathlineto{\pgfqpoint{4.948853in}{6.349623in}}%
\pgfpathlineto{\pgfqpoint{4.997242in}{6.349623in}}%
\pgfpathlineto{\pgfqpoint{5.045631in}{6.349623in}}%
\pgfpathlineto{\pgfqpoint{5.094020in}{6.349623in}}%
\pgfpathlineto{\pgfqpoint{5.142409in}{6.349623in}}%
\pgfpathlineto{\pgfqpoint{5.190798in}{6.349623in}}%
\pgfpathlineto{\pgfqpoint{5.239187in}{6.349623in}}%
\pgfpathlineto{\pgfqpoint{5.287576in}{6.349623in}}%
\pgfpathlineto{\pgfqpoint{5.335965in}{6.349623in}}%
\pgfpathlineto{\pgfqpoint{5.384354in}{6.349623in}}%
\pgfpathlineto{\pgfqpoint{5.432743in}{6.349623in}}%
\pgfpathlineto{\pgfqpoint{5.481132in}{6.349623in}}%
\pgfpathlineto{\pgfqpoint{5.529522in}{6.349623in}}%
\pgfpathlineto{\pgfqpoint{5.577911in}{6.349623in}}%
\pgfpathlineto{\pgfqpoint{5.626300in}{6.349623in}}%
\pgfpathlineto{\pgfqpoint{5.674689in}{6.349623in}}%
\pgfpathlineto{\pgfqpoint{5.723078in}{6.349623in}}%
\pgfpathlineto{\pgfqpoint{5.771467in}{6.349623in}}%
\pgfpathlineto{\pgfqpoint{5.819856in}{6.349623in}}%
\pgfpathlineto{\pgfqpoint{5.868245in}{6.349623in}}%
\pgfpathlineto{\pgfqpoint{5.916634in}{6.349623in}}%
\pgfpathlineto{\pgfqpoint{5.965023in}{6.349623in}}%
\pgfpathlineto{\pgfqpoint{6.013412in}{6.349623in}}%
\pgfpathlineto{\pgfqpoint{6.061801in}{6.349623in}}%
\pgfpathlineto{\pgfqpoint{6.110190in}{6.349622in}}%
\pgfpathlineto{\pgfqpoint{6.158580in}{6.349619in}}%
\pgfpathlineto{\pgfqpoint{6.206969in}{6.349612in}}%
\pgfpathlineto{\pgfqpoint{6.255358in}{6.349595in}}%
\pgfpathlineto{\pgfqpoint{6.303747in}{6.349558in}}%
\pgfpathlineto{\pgfqpoint{6.352136in}{6.349479in}}%
\pgfpathlineto{\pgfqpoint{6.400525in}{6.349318in}}%
\pgfpathlineto{\pgfqpoint{6.448914in}{6.349009in}}%
\pgfpathlineto{\pgfqpoint{6.497303in}{6.348434in}}%
\pgfpathlineto{\pgfqpoint{6.545692in}{6.347404in}}%
\pgfpathlineto{\pgfqpoint{6.594081in}{6.345611in}}%
\pgfpathlineto{\pgfqpoint{6.642470in}{6.342557in}}%
\pgfpathlineto{\pgfqpoint{6.690859in}{6.337404in}}%
\pgfpathlineto{\pgfqpoint{6.739249in}{6.328637in}}%
\pgfpathlineto{\pgfqpoint{6.787638in}{6.313063in}}%
\pgfpathlineto{\pgfqpoint{6.836027in}{6.281985in}}%
\pgfpathlineto{\pgfqpoint{6.884416in}{6.203445in}}%
\pgfpathlineto{\pgfqpoint{6.932805in}{5.950640in}}%
\pgfpathlineto{\pgfqpoint{6.981194in}{5.300857in}}%
\pgfpathlineto{\pgfqpoint{7.029583in}{4.744226in}}%
\pgfpathlineto{\pgfqpoint{7.077972in}{4.673976in}}%
\pgfpathlineto{\pgfqpoint{7.126361in}{4.673611in}}%
\pgfpathlineto{\pgfqpoint{7.174750in}{4.673611in}}%
\pgfpathlineto{\pgfqpoint{7.223139in}{4.673611in}}%
\pgfpathlineto{\pgfqpoint{7.271528in}{4.673611in}}%
\pgfpathlineto{\pgfqpoint{7.319918in}{4.673611in}}%
\pgfpathlineto{\pgfqpoint{7.368307in}{4.673611in}}%
\pgfpathlineto{\pgfqpoint{7.416696in}{4.673611in}}%
\pgfpathlineto{\pgfqpoint{7.465085in}{4.673611in}}%
\pgfpathlineto{\pgfqpoint{7.513474in}{4.673611in}}%
\pgfpathlineto{\pgfqpoint{7.561863in}{4.673611in}}%
\pgfpathlineto{\pgfqpoint{7.610252in}{4.673611in}}%
\pgfpathlineto{\pgfqpoint{7.658641in}{4.673611in}}%
\pgfusepath{stroke}%
\end{pgfscope}%
\begin{pgfscope}%
\pgfpathrectangle{\pgfqpoint{4.660937in}{4.530556in}}{\pgfqpoint{3.140451in}{3.147222in}}%
\pgfusepath{clip}%
\pgfsetrectcap%
\pgfsetroundjoin%
\pgfsetlinewidth{1.505625pt}%
\definecolor{currentstroke}{rgb}{0.839216,0.152941,0.156863}%
\pgfsetstrokecolor{currentstroke}%
\pgfsetdash{}{0pt}%
\pgfpathmoveto{\pgfqpoint{4.803685in}{6.349623in}}%
\pgfpathlineto{\pgfqpoint{4.852074in}{6.349623in}}%
\pgfpathlineto{\pgfqpoint{4.900463in}{6.349623in}}%
\pgfpathlineto{\pgfqpoint{4.948853in}{6.349623in}}%
\pgfpathlineto{\pgfqpoint{4.997242in}{6.349623in}}%
\pgfpathlineto{\pgfqpoint{5.045631in}{6.349623in}}%
\pgfpathlineto{\pgfqpoint{5.094020in}{6.349623in}}%
\pgfpathlineto{\pgfqpoint{5.142409in}{6.349623in}}%
\pgfpathlineto{\pgfqpoint{5.190798in}{6.349623in}}%
\pgfpathlineto{\pgfqpoint{5.239187in}{6.349623in}}%
\pgfpathlineto{\pgfqpoint{5.287576in}{6.349623in}}%
\pgfpathlineto{\pgfqpoint{5.335965in}{6.349623in}}%
\pgfpathlineto{\pgfqpoint{5.384354in}{6.349623in}}%
\pgfpathlineto{\pgfqpoint{5.432743in}{6.349623in}}%
\pgfpathlineto{\pgfqpoint{5.481132in}{6.349623in}}%
\pgfpathlineto{\pgfqpoint{5.529522in}{6.349623in}}%
\pgfpathlineto{\pgfqpoint{5.577911in}{6.349623in}}%
\pgfpathlineto{\pgfqpoint{5.626300in}{6.349623in}}%
\pgfpathlineto{\pgfqpoint{5.674689in}{6.349623in}}%
\pgfpathlineto{\pgfqpoint{5.723078in}{6.349623in}}%
\pgfpathlineto{\pgfqpoint{5.771467in}{6.349623in}}%
\pgfpathlineto{\pgfqpoint{5.819856in}{6.349623in}}%
\pgfpathlineto{\pgfqpoint{5.868245in}{6.349623in}}%
\pgfpathlineto{\pgfqpoint{5.916634in}{6.349623in}}%
\pgfpathlineto{\pgfqpoint{5.965023in}{6.349623in}}%
\pgfpathlineto{\pgfqpoint{6.013412in}{6.349623in}}%
\pgfpathlineto{\pgfqpoint{6.061801in}{6.349623in}}%
\pgfpathlineto{\pgfqpoint{6.110190in}{6.349622in}}%
\pgfpathlineto{\pgfqpoint{6.158580in}{6.349619in}}%
\pgfpathlineto{\pgfqpoint{6.206969in}{6.349612in}}%
\pgfpathlineto{\pgfqpoint{6.255358in}{6.349595in}}%
\pgfpathlineto{\pgfqpoint{6.303747in}{6.349556in}}%
\pgfpathlineto{\pgfqpoint{6.352136in}{6.349475in}}%
\pgfpathlineto{\pgfqpoint{6.400525in}{6.349308in}}%
\pgfpathlineto{\pgfqpoint{6.448914in}{6.348984in}}%
\pgfpathlineto{\pgfqpoint{6.497303in}{6.348375in}}%
\pgfpathlineto{\pgfqpoint{6.545692in}{6.347269in}}%
\pgfpathlineto{\pgfqpoint{6.594081in}{6.345302in}}%
\pgfpathlineto{\pgfqpoint{6.642470in}{6.341836in}}%
\pgfpathlineto{\pgfqpoint{6.690859in}{6.335612in}}%
\pgfpathlineto{\pgfqpoint{6.739249in}{6.323307in}}%
\pgfpathlineto{\pgfqpoint{6.787638in}{6.287619in}}%
\pgfpathlineto{\pgfqpoint{6.836027in}{6.006339in}}%
\pgfpathlineto{\pgfqpoint{6.884416in}{4.862234in}}%
\pgfpathlineto{\pgfqpoint{6.932805in}{4.683313in}}%
\pgfpathlineto{\pgfqpoint{6.981194in}{4.673817in}}%
\pgfpathlineto{\pgfqpoint{7.029583in}{4.673611in}}%
\pgfpathlineto{\pgfqpoint{7.077972in}{4.673611in}}%
\pgfpathlineto{\pgfqpoint{7.126361in}{4.673611in}}%
\pgfpathlineto{\pgfqpoint{7.174750in}{4.673611in}}%
\pgfpathlineto{\pgfqpoint{7.223139in}{4.673611in}}%
\pgfpathlineto{\pgfqpoint{7.271528in}{4.673611in}}%
\pgfpathlineto{\pgfqpoint{7.319918in}{4.673611in}}%
\pgfpathlineto{\pgfqpoint{7.368307in}{4.673611in}}%
\pgfpathlineto{\pgfqpoint{7.416696in}{4.673611in}}%
\pgfpathlineto{\pgfqpoint{7.465085in}{4.673611in}}%
\pgfpathlineto{\pgfqpoint{7.513474in}{4.673611in}}%
\pgfpathlineto{\pgfqpoint{7.561863in}{4.673611in}}%
\pgfpathlineto{\pgfqpoint{7.610252in}{4.673611in}}%
\pgfpathlineto{\pgfqpoint{7.658641in}{4.673611in}}%
\pgfusepath{stroke}%
\end{pgfscope}%
\begin{pgfscope}%
\pgfpathrectangle{\pgfqpoint{4.660937in}{4.530556in}}{\pgfqpoint{3.140451in}{3.147222in}}%
\pgfusepath{clip}%
\pgfsetrectcap%
\pgfsetroundjoin%
\pgfsetlinewidth{0.501875pt}%
\definecolor{currentstroke}{rgb}{0.000000,0.000000,0.000000}%
\pgfsetstrokecolor{currentstroke}%
\pgfsetdash{}{0pt}%
\pgfpathmoveto{\pgfqpoint{4.803685in}{6.349623in}}%
\pgfpathlineto{\pgfqpoint{4.852074in}{6.349623in}}%
\pgfpathlineto{\pgfqpoint{4.900463in}{6.349623in}}%
\pgfpathlineto{\pgfqpoint{4.948853in}{6.349623in}}%
\pgfpathlineto{\pgfqpoint{4.997242in}{6.349623in}}%
\pgfpathlineto{\pgfqpoint{5.045631in}{6.349623in}}%
\pgfpathlineto{\pgfqpoint{5.094020in}{6.349623in}}%
\pgfpathlineto{\pgfqpoint{5.142409in}{6.349623in}}%
\pgfpathlineto{\pgfqpoint{5.190798in}{6.349623in}}%
\pgfpathlineto{\pgfqpoint{5.239187in}{6.349623in}}%
\pgfpathlineto{\pgfqpoint{5.287576in}{6.349623in}}%
\pgfpathlineto{\pgfqpoint{5.335965in}{6.349623in}}%
\pgfpathlineto{\pgfqpoint{5.384354in}{6.349623in}}%
\pgfpathlineto{\pgfqpoint{5.432743in}{6.349623in}}%
\pgfpathlineto{\pgfqpoint{5.481132in}{6.349623in}}%
\pgfpathlineto{\pgfqpoint{5.529522in}{6.349623in}}%
\pgfpathlineto{\pgfqpoint{5.577911in}{6.349623in}}%
\pgfpathlineto{\pgfqpoint{5.626300in}{6.349623in}}%
\pgfpathlineto{\pgfqpoint{5.674689in}{6.349623in}}%
\pgfpathlineto{\pgfqpoint{5.723078in}{6.349623in}}%
\pgfpathlineto{\pgfqpoint{5.771467in}{6.349623in}}%
\pgfpathlineto{\pgfqpoint{5.819856in}{6.349623in}}%
\pgfpathlineto{\pgfqpoint{5.868245in}{6.349623in}}%
\pgfpathlineto{\pgfqpoint{5.916634in}{6.349623in}}%
\pgfpathlineto{\pgfqpoint{5.965023in}{6.349623in}}%
\pgfpathlineto{\pgfqpoint{6.013412in}{6.349623in}}%
\pgfpathlineto{\pgfqpoint{6.061801in}{6.349623in}}%
\pgfpathlineto{\pgfqpoint{6.110190in}{6.349623in}}%
\pgfpathlineto{\pgfqpoint{6.158580in}{6.349623in}}%
\pgfpathlineto{\pgfqpoint{6.206969in}{6.349623in}}%
\pgfpathlineto{\pgfqpoint{6.255358in}{6.349623in}}%
\pgfpathlineto{\pgfqpoint{6.303747in}{6.349623in}}%
\pgfpathlineto{\pgfqpoint{6.352136in}{6.349623in}}%
\pgfpathlineto{\pgfqpoint{6.400525in}{6.349623in}}%
\pgfpathlineto{\pgfqpoint{6.448914in}{6.349623in}}%
\pgfpathlineto{\pgfqpoint{6.497303in}{6.349623in}}%
\pgfpathlineto{\pgfqpoint{6.545692in}{6.349623in}}%
\pgfpathlineto{\pgfqpoint{6.594081in}{6.349623in}}%
\pgfpathlineto{\pgfqpoint{6.642470in}{6.349623in}}%
\pgfpathlineto{\pgfqpoint{6.690859in}{6.349623in}}%
\pgfpathlineto{\pgfqpoint{6.739249in}{6.349623in}}%
\pgfpathlineto{\pgfqpoint{6.787638in}{6.349623in}}%
\pgfpathlineto{\pgfqpoint{6.836027in}{6.349623in}}%
\pgfpathlineto{\pgfqpoint{6.884416in}{6.349623in}}%
\pgfpathlineto{\pgfqpoint{6.932805in}{6.349436in}}%
\pgfpathlineto{\pgfqpoint{6.981194in}{4.673798in}}%
\pgfpathlineto{\pgfqpoint{7.029583in}{4.673611in}}%
\pgfpathlineto{\pgfqpoint{7.077972in}{4.673611in}}%
\pgfpathlineto{\pgfqpoint{7.126361in}{4.673611in}}%
\pgfpathlineto{\pgfqpoint{7.174750in}{4.673611in}}%
\pgfpathlineto{\pgfqpoint{7.223139in}{4.673611in}}%
\pgfpathlineto{\pgfqpoint{7.271528in}{4.673611in}}%
\pgfpathlineto{\pgfqpoint{7.319918in}{4.673611in}}%
\pgfpathlineto{\pgfqpoint{7.368307in}{4.673611in}}%
\pgfpathlineto{\pgfqpoint{7.416696in}{4.673611in}}%
\pgfpathlineto{\pgfqpoint{7.465085in}{4.673611in}}%
\pgfpathlineto{\pgfqpoint{7.513474in}{4.673611in}}%
\pgfpathlineto{\pgfqpoint{7.561863in}{4.673611in}}%
\pgfpathlineto{\pgfqpoint{7.610252in}{4.673611in}}%
\pgfpathlineto{\pgfqpoint{7.658641in}{4.673611in}}%
\pgfusepath{stroke}%
\end{pgfscope}%
\begin{pgfscope}%
\pgfsetrectcap%
\pgfsetmiterjoin%
\pgfsetlinewidth{0.803000pt}%
\definecolor{currentstroke}{rgb}{0.000000,0.000000,0.000000}%
\pgfsetstrokecolor{currentstroke}%
\pgfsetdash{}{0pt}%
\pgfpathmoveto{\pgfqpoint{4.660937in}{4.530556in}}%
\pgfpathlineto{\pgfqpoint{4.660937in}{7.677778in}}%
\pgfusepath{stroke}%
\end{pgfscope}%
\begin{pgfscope}%
\pgfsetrectcap%
\pgfsetmiterjoin%
\pgfsetlinewidth{0.803000pt}%
\definecolor{currentstroke}{rgb}{0.000000,0.000000,0.000000}%
\pgfsetstrokecolor{currentstroke}%
\pgfsetdash{}{0pt}%
\pgfpathmoveto{\pgfqpoint{7.801389in}{4.530556in}}%
\pgfpathlineto{\pgfqpoint{7.801389in}{7.677778in}}%
\pgfusepath{stroke}%
\end{pgfscope}%
\begin{pgfscope}%
\pgfsetrectcap%
\pgfsetmiterjoin%
\pgfsetlinewidth{0.803000pt}%
\definecolor{currentstroke}{rgb}{0.000000,0.000000,0.000000}%
\pgfsetstrokecolor{currentstroke}%
\pgfsetdash{}{0pt}%
\pgfpathmoveto{\pgfqpoint{4.660937in}{4.530556in}}%
\pgfpathlineto{\pgfqpoint{7.801389in}{4.530556in}}%
\pgfusepath{stroke}%
\end{pgfscope}%
\begin{pgfscope}%
\pgfsetrectcap%
\pgfsetmiterjoin%
\pgfsetlinewidth{0.803000pt}%
\definecolor{currentstroke}{rgb}{0.000000,0.000000,0.000000}%
\pgfsetstrokecolor{currentstroke}%
\pgfsetdash{}{0pt}%
\pgfpathmoveto{\pgfqpoint{4.660937in}{7.677778in}}%
\pgfpathlineto{\pgfqpoint{7.801389in}{7.677778in}}%
\pgfusepath{stroke}%
\end{pgfscope}%
\begin{pgfscope}%
\definecolor{textcolor}{rgb}{0.000000,0.000000,0.000000}%
\pgfsetstrokecolor{textcolor}%
\pgfsetfillcolor{textcolor}%
\pgftext[x=6.231163in,y=7.761111in,,base]{\color{textcolor}\sffamily\fontsize{12.000000}{14.400000}\selectfont \(\displaystyle  t = 3.0 \)}%
\end{pgfscope}%
\begin{pgfscope}%
\pgfsetbuttcap%
\pgfsetmiterjoin%
\definecolor{currentfill}{rgb}{1.000000,1.000000,1.000000}%
\pgfsetfillcolor{currentfill}%
\pgfsetlinewidth{0.000000pt}%
\definecolor{currentstroke}{rgb}{0.000000,0.000000,0.000000}%
\pgfsetstrokecolor{currentstroke}%
\pgfsetstrokeopacity{0.000000}%
\pgfsetdash{}{0pt}%
\pgfpathmoveto{\pgfqpoint{0.691667in}{0.580556in}}%
\pgfpathlineto{\pgfqpoint{3.832118in}{0.580556in}}%
\pgfpathlineto{\pgfqpoint{3.832118in}{3.727778in}}%
\pgfpathlineto{\pgfqpoint{0.691667in}{3.727778in}}%
\pgfpathclose%
\pgfusepath{fill}%
\end{pgfscope}%
\begin{pgfscope}%
\pgfsetbuttcap%
\pgfsetroundjoin%
\definecolor{currentfill}{rgb}{0.000000,0.000000,0.000000}%
\pgfsetfillcolor{currentfill}%
\pgfsetlinewidth{0.803000pt}%
\definecolor{currentstroke}{rgb}{0.000000,0.000000,0.000000}%
\pgfsetstrokecolor{currentstroke}%
\pgfsetdash{}{0pt}%
\pgfsys@defobject{currentmarker}{\pgfqpoint{0.000000in}{-0.048611in}}{\pgfqpoint{0.000000in}{0.000000in}}{%
\pgfpathmoveto{\pgfqpoint{0.000000in}{0.000000in}}%
\pgfpathlineto{\pgfqpoint{0.000000in}{-0.048611in}}%
\pgfusepath{stroke,fill}%
}%
\begin{pgfscope}%
\pgfsys@transformshift{0.810220in}{0.580556in}%
\pgfsys@useobject{currentmarker}{}%
\end{pgfscope}%
\end{pgfscope}%
\begin{pgfscope}%
\definecolor{textcolor}{rgb}{0.000000,0.000000,0.000000}%
\pgfsetstrokecolor{textcolor}%
\pgfsetfillcolor{textcolor}%
\pgftext[x=0.810220in,y=0.483333in,,top]{\color{textcolor}\sffamily\fontsize{10.000000}{12.000000}\selectfont −3}%
\end{pgfscope}%
\begin{pgfscope}%
\pgfsetbuttcap%
\pgfsetroundjoin%
\definecolor{currentfill}{rgb}{0.000000,0.000000,0.000000}%
\pgfsetfillcolor{currentfill}%
\pgfsetlinewidth{0.803000pt}%
\definecolor{currentstroke}{rgb}{0.000000,0.000000,0.000000}%
\pgfsetstrokecolor{currentstroke}%
\pgfsetdash{}{0pt}%
\pgfsys@defobject{currentmarker}{\pgfqpoint{0.000000in}{-0.048611in}}{\pgfqpoint{0.000000in}{0.000000in}}{%
\pgfpathmoveto{\pgfqpoint{0.000000in}{0.000000in}}%
\pgfpathlineto{\pgfqpoint{0.000000in}{-0.048611in}}%
\pgfusepath{stroke,fill}%
}%
\begin{pgfscope}%
\pgfsys@transformshift{1.294111in}{0.580556in}%
\pgfsys@useobject{currentmarker}{}%
\end{pgfscope}%
\end{pgfscope}%
\begin{pgfscope}%
\definecolor{textcolor}{rgb}{0.000000,0.000000,0.000000}%
\pgfsetstrokecolor{textcolor}%
\pgfsetfillcolor{textcolor}%
\pgftext[x=1.294111in,y=0.483333in,,top]{\color{textcolor}\sffamily\fontsize{10.000000}{12.000000}\selectfont −2}%
\end{pgfscope}%
\begin{pgfscope}%
\pgfsetbuttcap%
\pgfsetroundjoin%
\definecolor{currentfill}{rgb}{0.000000,0.000000,0.000000}%
\pgfsetfillcolor{currentfill}%
\pgfsetlinewidth{0.803000pt}%
\definecolor{currentstroke}{rgb}{0.000000,0.000000,0.000000}%
\pgfsetstrokecolor{currentstroke}%
\pgfsetdash{}{0pt}%
\pgfsys@defobject{currentmarker}{\pgfqpoint{0.000000in}{-0.048611in}}{\pgfqpoint{0.000000in}{0.000000in}}{%
\pgfpathmoveto{\pgfqpoint{0.000000in}{0.000000in}}%
\pgfpathlineto{\pgfqpoint{0.000000in}{-0.048611in}}%
\pgfusepath{stroke,fill}%
}%
\begin{pgfscope}%
\pgfsys@transformshift{1.778002in}{0.580556in}%
\pgfsys@useobject{currentmarker}{}%
\end{pgfscope}%
\end{pgfscope}%
\begin{pgfscope}%
\definecolor{textcolor}{rgb}{0.000000,0.000000,0.000000}%
\pgfsetstrokecolor{textcolor}%
\pgfsetfillcolor{textcolor}%
\pgftext[x=1.778002in,y=0.483333in,,top]{\color{textcolor}\sffamily\fontsize{10.000000}{12.000000}\selectfont −1}%
\end{pgfscope}%
\begin{pgfscope}%
\pgfsetbuttcap%
\pgfsetroundjoin%
\definecolor{currentfill}{rgb}{0.000000,0.000000,0.000000}%
\pgfsetfillcolor{currentfill}%
\pgfsetlinewidth{0.803000pt}%
\definecolor{currentstroke}{rgb}{0.000000,0.000000,0.000000}%
\pgfsetstrokecolor{currentstroke}%
\pgfsetdash{}{0pt}%
\pgfsys@defobject{currentmarker}{\pgfqpoint{0.000000in}{-0.048611in}}{\pgfqpoint{0.000000in}{0.000000in}}{%
\pgfpathmoveto{\pgfqpoint{0.000000in}{0.000000in}}%
\pgfpathlineto{\pgfqpoint{0.000000in}{-0.048611in}}%
\pgfusepath{stroke,fill}%
}%
\begin{pgfscope}%
\pgfsys@transformshift{2.261892in}{0.580556in}%
\pgfsys@useobject{currentmarker}{}%
\end{pgfscope}%
\end{pgfscope}%
\begin{pgfscope}%
\definecolor{textcolor}{rgb}{0.000000,0.000000,0.000000}%
\pgfsetstrokecolor{textcolor}%
\pgfsetfillcolor{textcolor}%
\pgftext[x=2.261892in,y=0.483333in,,top]{\color{textcolor}\sffamily\fontsize{10.000000}{12.000000}\selectfont 0}%
\end{pgfscope}%
\begin{pgfscope}%
\pgfsetbuttcap%
\pgfsetroundjoin%
\definecolor{currentfill}{rgb}{0.000000,0.000000,0.000000}%
\pgfsetfillcolor{currentfill}%
\pgfsetlinewidth{0.803000pt}%
\definecolor{currentstroke}{rgb}{0.000000,0.000000,0.000000}%
\pgfsetstrokecolor{currentstroke}%
\pgfsetdash{}{0pt}%
\pgfsys@defobject{currentmarker}{\pgfqpoint{0.000000in}{-0.048611in}}{\pgfqpoint{0.000000in}{0.000000in}}{%
\pgfpathmoveto{\pgfqpoint{0.000000in}{0.000000in}}%
\pgfpathlineto{\pgfqpoint{0.000000in}{-0.048611in}}%
\pgfusepath{stroke,fill}%
}%
\begin{pgfscope}%
\pgfsys@transformshift{2.745783in}{0.580556in}%
\pgfsys@useobject{currentmarker}{}%
\end{pgfscope}%
\end{pgfscope}%
\begin{pgfscope}%
\definecolor{textcolor}{rgb}{0.000000,0.000000,0.000000}%
\pgfsetstrokecolor{textcolor}%
\pgfsetfillcolor{textcolor}%
\pgftext[x=2.745783in,y=0.483333in,,top]{\color{textcolor}\sffamily\fontsize{10.000000}{12.000000}\selectfont 1}%
\end{pgfscope}%
\begin{pgfscope}%
\pgfsetbuttcap%
\pgfsetroundjoin%
\definecolor{currentfill}{rgb}{0.000000,0.000000,0.000000}%
\pgfsetfillcolor{currentfill}%
\pgfsetlinewidth{0.803000pt}%
\definecolor{currentstroke}{rgb}{0.000000,0.000000,0.000000}%
\pgfsetstrokecolor{currentstroke}%
\pgfsetdash{}{0pt}%
\pgfsys@defobject{currentmarker}{\pgfqpoint{0.000000in}{-0.048611in}}{\pgfqpoint{0.000000in}{0.000000in}}{%
\pgfpathmoveto{\pgfqpoint{0.000000in}{0.000000in}}%
\pgfpathlineto{\pgfqpoint{0.000000in}{-0.048611in}}%
\pgfusepath{stroke,fill}%
}%
\begin{pgfscope}%
\pgfsys@transformshift{3.229674in}{0.580556in}%
\pgfsys@useobject{currentmarker}{}%
\end{pgfscope}%
\end{pgfscope}%
\begin{pgfscope}%
\definecolor{textcolor}{rgb}{0.000000,0.000000,0.000000}%
\pgfsetstrokecolor{textcolor}%
\pgfsetfillcolor{textcolor}%
\pgftext[x=3.229674in,y=0.483333in,,top]{\color{textcolor}\sffamily\fontsize{10.000000}{12.000000}\selectfont 2}%
\end{pgfscope}%
\begin{pgfscope}%
\pgfsetbuttcap%
\pgfsetroundjoin%
\definecolor{currentfill}{rgb}{0.000000,0.000000,0.000000}%
\pgfsetfillcolor{currentfill}%
\pgfsetlinewidth{0.803000pt}%
\definecolor{currentstroke}{rgb}{0.000000,0.000000,0.000000}%
\pgfsetstrokecolor{currentstroke}%
\pgfsetdash{}{0pt}%
\pgfsys@defobject{currentmarker}{\pgfqpoint{0.000000in}{-0.048611in}}{\pgfqpoint{0.000000in}{0.000000in}}{%
\pgfpathmoveto{\pgfqpoint{0.000000in}{0.000000in}}%
\pgfpathlineto{\pgfqpoint{0.000000in}{-0.048611in}}%
\pgfusepath{stroke,fill}%
}%
\begin{pgfscope}%
\pgfsys@transformshift{3.713565in}{0.580556in}%
\pgfsys@useobject{currentmarker}{}%
\end{pgfscope}%
\end{pgfscope}%
\begin{pgfscope}%
\definecolor{textcolor}{rgb}{0.000000,0.000000,0.000000}%
\pgfsetstrokecolor{textcolor}%
\pgfsetfillcolor{textcolor}%
\pgftext[x=3.713565in,y=0.483333in,,top]{\color{textcolor}\sffamily\fontsize{10.000000}{12.000000}\selectfont 3}%
\end{pgfscope}%
\begin{pgfscope}%
\definecolor{textcolor}{rgb}{0.000000,0.000000,0.000000}%
\pgfsetstrokecolor{textcolor}%
\pgfsetfillcolor{textcolor}%
\pgftext[x=2.261892in,y=0.293365in,,top]{\color{textcolor}\sffamily\fontsize{10.000000}{12.000000}\selectfont \(\displaystyle x\)}%
\end{pgfscope}%
\begin{pgfscope}%
\pgfsetbuttcap%
\pgfsetroundjoin%
\definecolor{currentfill}{rgb}{0.000000,0.000000,0.000000}%
\pgfsetfillcolor{currentfill}%
\pgfsetlinewidth{0.803000pt}%
\definecolor{currentstroke}{rgb}{0.000000,0.000000,0.000000}%
\pgfsetstrokecolor{currentstroke}%
\pgfsetdash{}{0pt}%
\pgfsys@defobject{currentmarker}{\pgfqpoint{-0.048611in}{0.000000in}}{\pgfqpoint{0.000000in}{0.000000in}}{%
\pgfpathmoveto{\pgfqpoint{0.000000in}{0.000000in}}%
\pgfpathlineto{\pgfqpoint{-0.048611in}{0.000000in}}%
\pgfusepath{stroke,fill}%
}%
\begin{pgfscope}%
\pgfsys@transformshift{0.691667in}{0.723611in}%
\pgfsys@useobject{currentmarker}{}%
\end{pgfscope}%
\end{pgfscope}%
\begin{pgfscope}%
\definecolor{textcolor}{rgb}{0.000000,0.000000,0.000000}%
\pgfsetstrokecolor{textcolor}%
\pgfsetfillcolor{textcolor}%
\pgftext[x=0.373565in,y=0.670850in,left,base]{\color{textcolor}\sffamily\fontsize{10.000000}{12.000000}\selectfont 0.0}%
\end{pgfscope}%
\begin{pgfscope}%
\pgfsetbuttcap%
\pgfsetroundjoin%
\definecolor{currentfill}{rgb}{0.000000,0.000000,0.000000}%
\pgfsetfillcolor{currentfill}%
\pgfsetlinewidth{0.803000pt}%
\definecolor{currentstroke}{rgb}{0.000000,0.000000,0.000000}%
\pgfsetstrokecolor{currentstroke}%
\pgfsetdash{}{0pt}%
\pgfsys@defobject{currentmarker}{\pgfqpoint{-0.048611in}{0.000000in}}{\pgfqpoint{0.000000in}{0.000000in}}{%
\pgfpathmoveto{\pgfqpoint{0.000000in}{0.000000in}}%
\pgfpathlineto{\pgfqpoint{-0.048611in}{0.000000in}}%
\pgfusepath{stroke,fill}%
}%
\begin{pgfscope}%
\pgfsys@transformshift{0.691667in}{1.078362in}%
\pgfsys@useobject{currentmarker}{}%
\end{pgfscope}%
\end{pgfscope}%
\begin{pgfscope}%
\definecolor{textcolor}{rgb}{0.000000,0.000000,0.000000}%
\pgfsetstrokecolor{textcolor}%
\pgfsetfillcolor{textcolor}%
\pgftext[x=0.373565in,y=1.025601in,left,base]{\color{textcolor}\sffamily\fontsize{10.000000}{12.000000}\selectfont 0.2}%
\end{pgfscope}%
\begin{pgfscope}%
\pgfsetbuttcap%
\pgfsetroundjoin%
\definecolor{currentfill}{rgb}{0.000000,0.000000,0.000000}%
\pgfsetfillcolor{currentfill}%
\pgfsetlinewidth{0.803000pt}%
\definecolor{currentstroke}{rgb}{0.000000,0.000000,0.000000}%
\pgfsetstrokecolor{currentstroke}%
\pgfsetdash{}{0pt}%
\pgfsys@defobject{currentmarker}{\pgfqpoint{-0.048611in}{0.000000in}}{\pgfqpoint{0.000000in}{0.000000in}}{%
\pgfpathmoveto{\pgfqpoint{0.000000in}{0.000000in}}%
\pgfpathlineto{\pgfqpoint{-0.048611in}{0.000000in}}%
\pgfusepath{stroke,fill}%
}%
\begin{pgfscope}%
\pgfsys@transformshift{0.691667in}{1.433113in}%
\pgfsys@useobject{currentmarker}{}%
\end{pgfscope}%
\end{pgfscope}%
\begin{pgfscope}%
\definecolor{textcolor}{rgb}{0.000000,0.000000,0.000000}%
\pgfsetstrokecolor{textcolor}%
\pgfsetfillcolor{textcolor}%
\pgftext[x=0.373565in,y=1.380352in,left,base]{\color{textcolor}\sffamily\fontsize{10.000000}{12.000000}\selectfont 0.4}%
\end{pgfscope}%
\begin{pgfscope}%
\pgfsetbuttcap%
\pgfsetroundjoin%
\definecolor{currentfill}{rgb}{0.000000,0.000000,0.000000}%
\pgfsetfillcolor{currentfill}%
\pgfsetlinewidth{0.803000pt}%
\definecolor{currentstroke}{rgb}{0.000000,0.000000,0.000000}%
\pgfsetstrokecolor{currentstroke}%
\pgfsetdash{}{0pt}%
\pgfsys@defobject{currentmarker}{\pgfqpoint{-0.048611in}{0.000000in}}{\pgfqpoint{0.000000in}{0.000000in}}{%
\pgfpathmoveto{\pgfqpoint{0.000000in}{0.000000in}}%
\pgfpathlineto{\pgfqpoint{-0.048611in}{0.000000in}}%
\pgfusepath{stroke,fill}%
}%
\begin{pgfscope}%
\pgfsys@transformshift{0.691667in}{1.787865in}%
\pgfsys@useobject{currentmarker}{}%
\end{pgfscope}%
\end{pgfscope}%
\begin{pgfscope}%
\definecolor{textcolor}{rgb}{0.000000,0.000000,0.000000}%
\pgfsetstrokecolor{textcolor}%
\pgfsetfillcolor{textcolor}%
\pgftext[x=0.373565in,y=1.735103in,left,base]{\color{textcolor}\sffamily\fontsize{10.000000}{12.000000}\selectfont 0.6}%
\end{pgfscope}%
\begin{pgfscope}%
\pgfsetbuttcap%
\pgfsetroundjoin%
\definecolor{currentfill}{rgb}{0.000000,0.000000,0.000000}%
\pgfsetfillcolor{currentfill}%
\pgfsetlinewidth{0.803000pt}%
\definecolor{currentstroke}{rgb}{0.000000,0.000000,0.000000}%
\pgfsetstrokecolor{currentstroke}%
\pgfsetdash{}{0pt}%
\pgfsys@defobject{currentmarker}{\pgfqpoint{-0.048611in}{0.000000in}}{\pgfqpoint{0.000000in}{0.000000in}}{%
\pgfpathmoveto{\pgfqpoint{0.000000in}{0.000000in}}%
\pgfpathlineto{\pgfqpoint{-0.048611in}{0.000000in}}%
\pgfusepath{stroke,fill}%
}%
\begin{pgfscope}%
\pgfsys@transformshift{0.691667in}{2.142616in}%
\pgfsys@useobject{currentmarker}{}%
\end{pgfscope}%
\end{pgfscope}%
\begin{pgfscope}%
\definecolor{textcolor}{rgb}{0.000000,0.000000,0.000000}%
\pgfsetstrokecolor{textcolor}%
\pgfsetfillcolor{textcolor}%
\pgftext[x=0.373565in,y=2.089854in,left,base]{\color{textcolor}\sffamily\fontsize{10.000000}{12.000000}\selectfont 0.8}%
\end{pgfscope}%
\begin{pgfscope}%
\pgfsetbuttcap%
\pgfsetroundjoin%
\definecolor{currentfill}{rgb}{0.000000,0.000000,0.000000}%
\pgfsetfillcolor{currentfill}%
\pgfsetlinewidth{0.803000pt}%
\definecolor{currentstroke}{rgb}{0.000000,0.000000,0.000000}%
\pgfsetstrokecolor{currentstroke}%
\pgfsetdash{}{0pt}%
\pgfsys@defobject{currentmarker}{\pgfqpoint{-0.048611in}{0.000000in}}{\pgfqpoint{0.000000in}{0.000000in}}{%
\pgfpathmoveto{\pgfqpoint{0.000000in}{0.000000in}}%
\pgfpathlineto{\pgfqpoint{-0.048611in}{0.000000in}}%
\pgfusepath{stroke,fill}%
}%
\begin{pgfscope}%
\pgfsys@transformshift{0.691667in}{2.497367in}%
\pgfsys@useobject{currentmarker}{}%
\end{pgfscope}%
\end{pgfscope}%
\begin{pgfscope}%
\definecolor{textcolor}{rgb}{0.000000,0.000000,0.000000}%
\pgfsetstrokecolor{textcolor}%
\pgfsetfillcolor{textcolor}%
\pgftext[x=0.373565in,y=2.444605in,left,base]{\color{textcolor}\sffamily\fontsize{10.000000}{12.000000}\selectfont 1.0}%
\end{pgfscope}%
\begin{pgfscope}%
\pgfsetbuttcap%
\pgfsetroundjoin%
\definecolor{currentfill}{rgb}{0.000000,0.000000,0.000000}%
\pgfsetfillcolor{currentfill}%
\pgfsetlinewidth{0.803000pt}%
\definecolor{currentstroke}{rgb}{0.000000,0.000000,0.000000}%
\pgfsetstrokecolor{currentstroke}%
\pgfsetdash{}{0pt}%
\pgfsys@defobject{currentmarker}{\pgfqpoint{-0.048611in}{0.000000in}}{\pgfqpoint{0.000000in}{0.000000in}}{%
\pgfpathmoveto{\pgfqpoint{0.000000in}{0.000000in}}%
\pgfpathlineto{\pgfqpoint{-0.048611in}{0.000000in}}%
\pgfusepath{stroke,fill}%
}%
\begin{pgfscope}%
\pgfsys@transformshift{0.691667in}{2.852118in}%
\pgfsys@useobject{currentmarker}{}%
\end{pgfscope}%
\end{pgfscope}%
\begin{pgfscope}%
\definecolor{textcolor}{rgb}{0.000000,0.000000,0.000000}%
\pgfsetstrokecolor{textcolor}%
\pgfsetfillcolor{textcolor}%
\pgftext[x=0.373565in,y=2.799356in,left,base]{\color{textcolor}\sffamily\fontsize{10.000000}{12.000000}\selectfont 1.2}%
\end{pgfscope}%
\begin{pgfscope}%
\pgfsetbuttcap%
\pgfsetroundjoin%
\definecolor{currentfill}{rgb}{0.000000,0.000000,0.000000}%
\pgfsetfillcolor{currentfill}%
\pgfsetlinewidth{0.803000pt}%
\definecolor{currentstroke}{rgb}{0.000000,0.000000,0.000000}%
\pgfsetstrokecolor{currentstroke}%
\pgfsetdash{}{0pt}%
\pgfsys@defobject{currentmarker}{\pgfqpoint{-0.048611in}{0.000000in}}{\pgfqpoint{0.000000in}{0.000000in}}{%
\pgfpathmoveto{\pgfqpoint{0.000000in}{0.000000in}}%
\pgfpathlineto{\pgfqpoint{-0.048611in}{0.000000in}}%
\pgfusepath{stroke,fill}%
}%
\begin{pgfscope}%
\pgfsys@transformshift{0.691667in}{3.206869in}%
\pgfsys@useobject{currentmarker}{}%
\end{pgfscope}%
\end{pgfscope}%
\begin{pgfscope}%
\definecolor{textcolor}{rgb}{0.000000,0.000000,0.000000}%
\pgfsetstrokecolor{textcolor}%
\pgfsetfillcolor{textcolor}%
\pgftext[x=0.373565in,y=3.154108in,left,base]{\color{textcolor}\sffamily\fontsize{10.000000}{12.000000}\selectfont 1.4}%
\end{pgfscope}%
\begin{pgfscope}%
\pgfsetbuttcap%
\pgfsetroundjoin%
\definecolor{currentfill}{rgb}{0.000000,0.000000,0.000000}%
\pgfsetfillcolor{currentfill}%
\pgfsetlinewidth{0.803000pt}%
\definecolor{currentstroke}{rgb}{0.000000,0.000000,0.000000}%
\pgfsetstrokecolor{currentstroke}%
\pgfsetdash{}{0pt}%
\pgfsys@defobject{currentmarker}{\pgfqpoint{-0.048611in}{0.000000in}}{\pgfqpoint{0.000000in}{0.000000in}}{%
\pgfpathmoveto{\pgfqpoint{0.000000in}{0.000000in}}%
\pgfpathlineto{\pgfqpoint{-0.048611in}{0.000000in}}%
\pgfusepath{stroke,fill}%
}%
\begin{pgfscope}%
\pgfsys@transformshift{0.691667in}{3.561620in}%
\pgfsys@useobject{currentmarker}{}%
\end{pgfscope}%
\end{pgfscope}%
\begin{pgfscope}%
\definecolor{textcolor}{rgb}{0.000000,0.000000,0.000000}%
\pgfsetstrokecolor{textcolor}%
\pgfsetfillcolor{textcolor}%
\pgftext[x=0.373565in,y=3.508859in,left,base]{\color{textcolor}\sffamily\fontsize{10.000000}{12.000000}\selectfont 1.6}%
\end{pgfscope}%
\begin{pgfscope}%
\definecolor{textcolor}{rgb}{0.000000,0.000000,0.000000}%
\pgfsetstrokecolor{textcolor}%
\pgfsetfillcolor{textcolor}%
\pgftext[x=0.318009in,y=2.154167in,,bottom,rotate=90.000000]{\color{textcolor}\sffamily\fontsize{10.000000}{12.000000}\selectfont \(\displaystyle U\)}%
\end{pgfscope}%
\begin{pgfscope}%
\pgfpathrectangle{\pgfqpoint{0.691667in}{0.580556in}}{\pgfqpoint{3.140451in}{3.147222in}}%
\pgfusepath{clip}%
\pgfsetrectcap%
\pgfsetroundjoin%
\pgfsetlinewidth{1.505625pt}%
\definecolor{currentstroke}{rgb}{0.121569,0.466667,0.705882}%
\pgfsetstrokecolor{currentstroke}%
\pgfsetdash{}{0pt}%
\pgfpathmoveto{\pgfqpoint{0.834414in}{2.497420in}}%
\pgfpathlineto{\pgfqpoint{0.882804in}{2.497419in}}%
\pgfpathlineto{\pgfqpoint{0.931193in}{2.497408in}}%
\pgfpathlineto{\pgfqpoint{0.979582in}{2.497410in}}%
\pgfpathlineto{\pgfqpoint{1.027971in}{2.497430in}}%
\pgfpathlineto{\pgfqpoint{1.076360in}{2.497428in}}%
\pgfpathlineto{\pgfqpoint{1.124749in}{2.497407in}}%
\pgfpathlineto{\pgfqpoint{1.173138in}{2.497411in}}%
\pgfpathlineto{\pgfqpoint{1.221527in}{2.497424in}}%
\pgfpathlineto{\pgfqpoint{1.269916in}{2.497409in}}%
\pgfpathlineto{\pgfqpoint{1.318305in}{2.497403in}}%
\pgfpathlineto{\pgfqpoint{1.366694in}{2.497439in}}%
\pgfpathlineto{\pgfqpoint{1.415083in}{2.497450in}}%
\pgfpathlineto{\pgfqpoint{1.463473in}{2.497398in}}%
\pgfpathlineto{\pgfqpoint{1.511862in}{2.497374in}}%
\pgfpathlineto{\pgfqpoint{1.560251in}{2.497429in}}%
\pgfpathlineto{\pgfqpoint{1.608640in}{2.497451in}}%
\pgfpathlineto{\pgfqpoint{1.657029in}{2.497398in}}%
\pgfpathlineto{\pgfqpoint{1.705418in}{2.497362in}}%
\pgfpathlineto{\pgfqpoint{1.753807in}{2.497419in}}%
\pgfpathlineto{\pgfqpoint{1.802196in}{2.497613in}}%
\pgfpathlineto{\pgfqpoint{1.850585in}{2.497327in}}%
\pgfpathlineto{\pgfqpoint{1.898974in}{2.497253in}}%
\pgfpathlineto{\pgfqpoint{1.947363in}{2.497907in}}%
\pgfpathlineto{\pgfqpoint{1.995752in}{2.496322in}}%
\pgfpathlineto{\pgfqpoint{2.044141in}{2.498387in}}%
\pgfpathlineto{\pgfqpoint{2.092531in}{2.499285in}}%
\pgfpathlineto{\pgfqpoint{2.140920in}{2.490654in}}%
\pgfpathlineto{\pgfqpoint{2.189309in}{2.504147in}}%
\pgfpathlineto{\pgfqpoint{2.237698in}{2.502411in}}%
\pgfpathlineto{\pgfqpoint{2.286087in}{2.472641in}}%
\pgfpathlineto{\pgfqpoint{2.334476in}{2.524360in}}%
\pgfpathlineto{\pgfqpoint{2.382865in}{2.514157in}}%
\pgfpathlineto{\pgfqpoint{2.431254in}{2.417538in}}%
\pgfpathlineto{\pgfqpoint{2.479643in}{2.557205in}}%
\pgfpathlineto{\pgfqpoint{2.528032in}{2.581443in}}%
\pgfpathlineto{\pgfqpoint{2.576421in}{2.303806in}}%
\pgfpathlineto{\pgfqpoint{2.624810in}{2.524528in}}%
\pgfpathlineto{\pgfqpoint{2.673200in}{2.786207in}}%
\pgfpathlineto{\pgfqpoint{2.721589in}{2.214481in}}%
\pgfpathlineto{\pgfqpoint{2.769978in}{2.234094in}}%
\pgfpathlineto{\pgfqpoint{2.818367in}{3.041111in}}%
\pgfpathlineto{\pgfqpoint{2.866756in}{2.466344in}}%
\pgfpathlineto{\pgfqpoint{2.915145in}{1.749633in}}%
\pgfpathlineto{\pgfqpoint{2.963534in}{2.784192in}}%
\pgfpathlineto{\pgfqpoint{3.011923in}{3.200015in}}%
\pgfpathlineto{\pgfqpoint{3.060312in}{1.607685in}}%
\pgfpathlineto{\pgfqpoint{3.108701in}{1.761023in}}%
\pgfpathlineto{\pgfqpoint{3.157090in}{3.584722in}}%
\pgfpathlineto{\pgfqpoint{3.205479in}{2.502434in}}%
\pgfpathlineto{\pgfqpoint{3.253869in}{0.835930in}}%
\pgfpathlineto{\pgfqpoint{3.302258in}{0.723752in}}%
\pgfpathlineto{\pgfqpoint{3.350647in}{0.723611in}}%
\pgfpathlineto{\pgfqpoint{3.399036in}{0.723611in}}%
\pgfpathlineto{\pgfqpoint{3.447425in}{0.723611in}}%
\pgfpathlineto{\pgfqpoint{3.495814in}{0.723611in}}%
\pgfpathlineto{\pgfqpoint{3.544203in}{0.723611in}}%
\pgfpathlineto{\pgfqpoint{3.592592in}{0.723611in}}%
\pgfpathlineto{\pgfqpoint{3.640981in}{0.723611in}}%
\pgfpathlineto{\pgfqpoint{3.689370in}{0.723611in}}%
\pgfusepath{stroke}%
\end{pgfscope}%
\begin{pgfscope}%
\pgfpathrectangle{\pgfqpoint{0.691667in}{0.580556in}}{\pgfqpoint{3.140451in}{3.147222in}}%
\pgfusepath{clip}%
\pgfsetrectcap%
\pgfsetroundjoin%
\pgfsetlinewidth{1.505625pt}%
\definecolor{currentstroke}{rgb}{1.000000,0.498039,0.054902}%
\pgfsetstrokecolor{currentstroke}%
\pgfsetdash{}{0pt}%
\pgfpathmoveto{\pgfqpoint{0.834414in}{2.497420in}}%
\pgfpathlineto{\pgfqpoint{0.882804in}{2.497419in}}%
\pgfpathlineto{\pgfqpoint{0.931193in}{2.497408in}}%
\pgfpathlineto{\pgfqpoint{0.979582in}{2.497410in}}%
\pgfpathlineto{\pgfqpoint{1.027971in}{2.497430in}}%
\pgfpathlineto{\pgfqpoint{1.076360in}{2.497428in}}%
\pgfpathlineto{\pgfqpoint{1.124749in}{2.497407in}}%
\pgfpathlineto{\pgfqpoint{1.173138in}{2.497411in}}%
\pgfpathlineto{\pgfqpoint{1.221527in}{2.497424in}}%
\pgfpathlineto{\pgfqpoint{1.269916in}{2.497409in}}%
\pgfpathlineto{\pgfqpoint{1.318305in}{2.497403in}}%
\pgfpathlineto{\pgfqpoint{1.366694in}{2.497439in}}%
\pgfpathlineto{\pgfqpoint{1.415083in}{2.497450in}}%
\pgfpathlineto{\pgfqpoint{1.463473in}{2.497398in}}%
\pgfpathlineto{\pgfqpoint{1.511862in}{2.497374in}}%
\pgfpathlineto{\pgfqpoint{1.560251in}{2.497428in}}%
\pgfpathlineto{\pgfqpoint{1.608640in}{2.497453in}}%
\pgfpathlineto{\pgfqpoint{1.657029in}{2.497396in}}%
\pgfpathlineto{\pgfqpoint{1.705418in}{2.497355in}}%
\pgfpathlineto{\pgfqpoint{1.753807in}{2.497446in}}%
\pgfpathlineto{\pgfqpoint{1.802196in}{2.497564in}}%
\pgfpathlineto{\pgfqpoint{1.850585in}{2.497357in}}%
\pgfpathlineto{\pgfqpoint{1.898974in}{2.497370in}}%
\pgfpathlineto{\pgfqpoint{1.947363in}{2.497497in}}%
\pgfpathlineto{\pgfqpoint{1.995752in}{2.496918in}}%
\pgfpathlineto{\pgfqpoint{2.044141in}{2.498339in}}%
\pgfpathlineto{\pgfqpoint{2.092531in}{2.497598in}}%
\pgfpathlineto{\pgfqpoint{2.140920in}{2.494215in}}%
\pgfpathlineto{\pgfqpoint{2.189309in}{2.501835in}}%
\pgfpathlineto{\pgfqpoint{2.237698in}{2.497669in}}%
\pgfpathlineto{\pgfqpoint{2.286087in}{2.485837in}}%
\pgfpathlineto{\pgfqpoint{2.334476in}{2.514620in}}%
\pgfpathlineto{\pgfqpoint{2.382865in}{2.500452in}}%
\pgfpathlineto{\pgfqpoint{2.431254in}{2.454399in}}%
\pgfpathlineto{\pgfqpoint{2.479643in}{2.538609in}}%
\pgfpathlineto{\pgfqpoint{2.528032in}{2.536056in}}%
\pgfpathlineto{\pgfqpoint{2.576421in}{2.379923in}}%
\pgfpathlineto{\pgfqpoint{2.624810in}{2.528955in}}%
\pgfpathlineto{\pgfqpoint{2.673200in}{2.659634in}}%
\pgfpathlineto{\pgfqpoint{2.721589in}{2.307155in}}%
\pgfpathlineto{\pgfqpoint{2.769978in}{2.347201in}}%
\pgfpathlineto{\pgfqpoint{2.818367in}{2.824228in}}%
\pgfpathlineto{\pgfqpoint{2.866756in}{2.468160in}}%
\pgfpathlineto{\pgfqpoint{2.915145in}{1.984790in}}%
\pgfpathlineto{\pgfqpoint{2.963534in}{2.687758in}}%
\pgfpathlineto{\pgfqpoint{3.011923in}{2.910230in}}%
\pgfpathlineto{\pgfqpoint{3.060312in}{1.865639in}}%
\pgfpathlineto{\pgfqpoint{3.108701in}{1.979183in}}%
\pgfpathlineto{\pgfqpoint{3.157090in}{3.101100in}}%
\pgfpathlineto{\pgfqpoint{3.205479in}{2.543460in}}%
\pgfpathlineto{\pgfqpoint{3.253869in}{1.047477in}}%
\pgfpathlineto{\pgfqpoint{3.302258in}{0.727362in}}%
\pgfpathlineto{\pgfqpoint{3.350647in}{0.723611in}}%
\pgfpathlineto{\pgfqpoint{3.399036in}{0.723611in}}%
\pgfpathlineto{\pgfqpoint{3.447425in}{0.723611in}}%
\pgfpathlineto{\pgfqpoint{3.495814in}{0.723611in}}%
\pgfpathlineto{\pgfqpoint{3.544203in}{0.723611in}}%
\pgfpathlineto{\pgfqpoint{3.592592in}{0.723611in}}%
\pgfpathlineto{\pgfqpoint{3.640981in}{0.723611in}}%
\pgfpathlineto{\pgfqpoint{3.689370in}{0.723611in}}%
\pgfusepath{stroke}%
\end{pgfscope}%
\begin{pgfscope}%
\pgfpathrectangle{\pgfqpoint{0.691667in}{0.580556in}}{\pgfqpoint{3.140451in}{3.147222in}}%
\pgfusepath{clip}%
\pgfsetrectcap%
\pgfsetroundjoin%
\pgfsetlinewidth{1.505625pt}%
\definecolor{currentstroke}{rgb}{0.172549,0.627451,0.172549}%
\pgfsetstrokecolor{currentstroke}%
\pgfsetdash{}{0pt}%
\pgfpathmoveto{\pgfqpoint{0.834414in}{2.497367in}}%
\pgfpathlineto{\pgfqpoint{0.882804in}{2.497367in}}%
\pgfpathlineto{\pgfqpoint{0.931193in}{2.497367in}}%
\pgfpathlineto{\pgfqpoint{0.979582in}{2.497367in}}%
\pgfpathlineto{\pgfqpoint{1.027971in}{2.497367in}}%
\pgfpathlineto{\pgfqpoint{1.076360in}{2.497367in}}%
\pgfpathlineto{\pgfqpoint{1.124749in}{2.497367in}}%
\pgfpathlineto{\pgfqpoint{1.173138in}{2.497367in}}%
\pgfpathlineto{\pgfqpoint{1.221527in}{2.497367in}}%
\pgfpathlineto{\pgfqpoint{1.269916in}{2.497367in}}%
\pgfpathlineto{\pgfqpoint{1.318305in}{2.497367in}}%
\pgfpathlineto{\pgfqpoint{1.366694in}{2.497367in}}%
\pgfpathlineto{\pgfqpoint{1.415083in}{2.497367in}}%
\pgfpathlineto{\pgfqpoint{1.463473in}{2.497367in}}%
\pgfpathlineto{\pgfqpoint{1.511862in}{2.497367in}}%
\pgfpathlineto{\pgfqpoint{1.560251in}{2.497367in}}%
\pgfpathlineto{\pgfqpoint{1.608640in}{2.497367in}}%
\pgfpathlineto{\pgfqpoint{1.657029in}{2.497367in}}%
\pgfpathlineto{\pgfqpoint{1.705418in}{2.497367in}}%
\pgfpathlineto{\pgfqpoint{1.753807in}{2.497367in}}%
\pgfpathlineto{\pgfqpoint{1.802196in}{2.497367in}}%
\pgfpathlineto{\pgfqpoint{1.850585in}{2.497367in}}%
\pgfpathlineto{\pgfqpoint{1.898974in}{2.497367in}}%
\pgfpathlineto{\pgfqpoint{1.947363in}{2.497367in}}%
\pgfpathlineto{\pgfqpoint{1.995752in}{2.497367in}}%
\pgfpathlineto{\pgfqpoint{2.044141in}{2.497367in}}%
\pgfpathlineto{\pgfqpoint{2.092531in}{2.497367in}}%
\pgfpathlineto{\pgfqpoint{2.140920in}{2.497367in}}%
\pgfpathlineto{\pgfqpoint{2.189309in}{2.497367in}}%
\pgfpathlineto{\pgfqpoint{2.237698in}{2.497367in}}%
\pgfpathlineto{\pgfqpoint{2.286087in}{2.497367in}}%
\pgfpathlineto{\pgfqpoint{2.334476in}{2.497367in}}%
\pgfpathlineto{\pgfqpoint{2.382865in}{2.497366in}}%
\pgfpathlineto{\pgfqpoint{2.431254in}{2.497366in}}%
\pgfpathlineto{\pgfqpoint{2.479643in}{2.497365in}}%
\pgfpathlineto{\pgfqpoint{2.528032in}{2.497362in}}%
\pgfpathlineto{\pgfqpoint{2.576421in}{2.497355in}}%
\pgfpathlineto{\pgfqpoint{2.624810in}{2.497341in}}%
\pgfpathlineto{\pgfqpoint{2.673200in}{2.497312in}}%
\pgfpathlineto{\pgfqpoint{2.721589in}{2.497255in}}%
\pgfpathlineto{\pgfqpoint{2.769978in}{2.497145in}}%
\pgfpathlineto{\pgfqpoint{2.818367in}{2.496942in}}%
\pgfpathlineto{\pgfqpoint{2.866756in}{2.496574in}}%
\pgfpathlineto{\pgfqpoint{2.915145in}{2.495914in}}%
\pgfpathlineto{\pgfqpoint{2.963534in}{2.494725in}}%
\pgfpathlineto{\pgfqpoint{3.011923in}{2.492438in}}%
\pgfpathlineto{\pgfqpoint{3.060312in}{2.487088in}}%
\pgfpathlineto{\pgfqpoint{3.108701in}{2.469489in}}%
\pgfpathlineto{\pgfqpoint{3.157090in}{2.392880in}}%
\pgfpathlineto{\pgfqpoint{3.205479in}{2.065140in}}%
\pgfpathlineto{\pgfqpoint{3.253869in}{1.269988in}}%
\pgfpathlineto{\pgfqpoint{3.302258in}{0.762701in}}%
\pgfpathlineto{\pgfqpoint{3.350647in}{0.723690in}}%
\pgfpathlineto{\pgfqpoint{3.399036in}{0.723611in}}%
\pgfpathlineto{\pgfqpoint{3.447425in}{0.723611in}}%
\pgfpathlineto{\pgfqpoint{3.495814in}{0.723611in}}%
\pgfpathlineto{\pgfqpoint{3.544203in}{0.723611in}}%
\pgfpathlineto{\pgfqpoint{3.592592in}{0.723611in}}%
\pgfpathlineto{\pgfqpoint{3.640981in}{0.723611in}}%
\pgfpathlineto{\pgfqpoint{3.689370in}{0.723611in}}%
\pgfusepath{stroke}%
\end{pgfscope}%
\begin{pgfscope}%
\pgfpathrectangle{\pgfqpoint{0.691667in}{0.580556in}}{\pgfqpoint{3.140451in}{3.147222in}}%
\pgfusepath{clip}%
\pgfsetrectcap%
\pgfsetroundjoin%
\pgfsetlinewidth{1.505625pt}%
\definecolor{currentstroke}{rgb}{0.839216,0.152941,0.156863}%
\pgfsetstrokecolor{currentstroke}%
\pgfsetdash{}{0pt}%
\pgfpathmoveto{\pgfqpoint{0.834414in}{2.497367in}}%
\pgfpathlineto{\pgfqpoint{0.882804in}{2.497367in}}%
\pgfpathlineto{\pgfqpoint{0.931193in}{2.497367in}}%
\pgfpathlineto{\pgfqpoint{0.979582in}{2.497367in}}%
\pgfpathlineto{\pgfqpoint{1.027971in}{2.497367in}}%
\pgfpathlineto{\pgfqpoint{1.076360in}{2.497367in}}%
\pgfpathlineto{\pgfqpoint{1.124749in}{2.497367in}}%
\pgfpathlineto{\pgfqpoint{1.173138in}{2.497367in}}%
\pgfpathlineto{\pgfqpoint{1.221527in}{2.497367in}}%
\pgfpathlineto{\pgfqpoint{1.269916in}{2.497367in}}%
\pgfpathlineto{\pgfqpoint{1.318305in}{2.497367in}}%
\pgfpathlineto{\pgfqpoint{1.366694in}{2.497367in}}%
\pgfpathlineto{\pgfqpoint{1.415083in}{2.497367in}}%
\pgfpathlineto{\pgfqpoint{1.463473in}{2.497367in}}%
\pgfpathlineto{\pgfqpoint{1.511862in}{2.497367in}}%
\pgfpathlineto{\pgfqpoint{1.560251in}{2.497367in}}%
\pgfpathlineto{\pgfqpoint{1.608640in}{2.497367in}}%
\pgfpathlineto{\pgfqpoint{1.657029in}{2.497367in}}%
\pgfpathlineto{\pgfqpoint{1.705418in}{2.497367in}}%
\pgfpathlineto{\pgfqpoint{1.753807in}{2.497367in}}%
\pgfpathlineto{\pgfqpoint{1.802196in}{2.497367in}}%
\pgfpathlineto{\pgfqpoint{1.850585in}{2.497367in}}%
\pgfpathlineto{\pgfqpoint{1.898974in}{2.497367in}}%
\pgfpathlineto{\pgfqpoint{1.947363in}{2.497367in}}%
\pgfpathlineto{\pgfqpoint{1.995752in}{2.497367in}}%
\pgfpathlineto{\pgfqpoint{2.044141in}{2.497367in}}%
\pgfpathlineto{\pgfqpoint{2.092531in}{2.497367in}}%
\pgfpathlineto{\pgfqpoint{2.140920in}{2.497367in}}%
\pgfpathlineto{\pgfqpoint{2.189309in}{2.497367in}}%
\pgfpathlineto{\pgfqpoint{2.237698in}{2.497367in}}%
\pgfpathlineto{\pgfqpoint{2.286087in}{2.497367in}}%
\pgfpathlineto{\pgfqpoint{2.334476in}{2.497367in}}%
\pgfpathlineto{\pgfqpoint{2.382865in}{2.497366in}}%
\pgfpathlineto{\pgfqpoint{2.431254in}{2.497366in}}%
\pgfpathlineto{\pgfqpoint{2.479643in}{2.497365in}}%
\pgfpathlineto{\pgfqpoint{2.528032in}{2.497362in}}%
\pgfpathlineto{\pgfqpoint{2.576421in}{2.497355in}}%
\pgfpathlineto{\pgfqpoint{2.624810in}{2.497340in}}%
\pgfpathlineto{\pgfqpoint{2.673200in}{2.497310in}}%
\pgfpathlineto{\pgfqpoint{2.721589in}{2.497250in}}%
\pgfpathlineto{\pgfqpoint{2.769978in}{2.497133in}}%
\pgfpathlineto{\pgfqpoint{2.818367in}{2.496913in}}%
\pgfpathlineto{\pgfqpoint{2.866756in}{2.496488in}}%
\pgfpathlineto{\pgfqpoint{2.915145in}{2.494958in}}%
\pgfpathlineto{\pgfqpoint{2.963534in}{2.313721in}}%
\pgfpathlineto{\pgfqpoint{3.011923in}{0.725554in}}%
\pgfpathlineto{\pgfqpoint{3.060312in}{0.723611in}}%
\pgfpathlineto{\pgfqpoint{3.108701in}{0.723611in}}%
\pgfpathlineto{\pgfqpoint{3.157090in}{0.723611in}}%
\pgfpathlineto{\pgfqpoint{3.205479in}{0.723611in}}%
\pgfpathlineto{\pgfqpoint{3.253869in}{0.723611in}}%
\pgfpathlineto{\pgfqpoint{3.302258in}{0.723611in}}%
\pgfpathlineto{\pgfqpoint{3.350647in}{0.723611in}}%
\pgfpathlineto{\pgfqpoint{3.399036in}{0.723611in}}%
\pgfpathlineto{\pgfqpoint{3.447425in}{0.723611in}}%
\pgfpathlineto{\pgfqpoint{3.495814in}{0.723611in}}%
\pgfpathlineto{\pgfqpoint{3.544203in}{0.723611in}}%
\pgfpathlineto{\pgfqpoint{3.592592in}{0.723611in}}%
\pgfpathlineto{\pgfqpoint{3.640981in}{0.723611in}}%
\pgfpathlineto{\pgfqpoint{3.689370in}{0.723611in}}%
\pgfusepath{stroke}%
\end{pgfscope}%
\begin{pgfscope}%
\pgfpathrectangle{\pgfqpoint{0.691667in}{0.580556in}}{\pgfqpoint{3.140451in}{3.147222in}}%
\pgfusepath{clip}%
\pgfsetrectcap%
\pgfsetroundjoin%
\pgfsetlinewidth{0.501875pt}%
\definecolor{currentstroke}{rgb}{0.000000,0.000000,0.000000}%
\pgfsetstrokecolor{currentstroke}%
\pgfsetdash{}{0pt}%
\pgfpathmoveto{\pgfqpoint{0.834414in}{2.497367in}}%
\pgfpathlineto{\pgfqpoint{0.882804in}{2.497367in}}%
\pgfpathlineto{\pgfqpoint{0.931193in}{2.497367in}}%
\pgfpathlineto{\pgfqpoint{0.979582in}{2.497367in}}%
\pgfpathlineto{\pgfqpoint{1.027971in}{2.497367in}}%
\pgfpathlineto{\pgfqpoint{1.076360in}{2.497367in}}%
\pgfpathlineto{\pgfqpoint{1.124749in}{2.497367in}}%
\pgfpathlineto{\pgfqpoint{1.173138in}{2.497367in}}%
\pgfpathlineto{\pgfqpoint{1.221527in}{2.497367in}}%
\pgfpathlineto{\pgfqpoint{1.269916in}{2.497367in}}%
\pgfpathlineto{\pgfqpoint{1.318305in}{2.497367in}}%
\pgfpathlineto{\pgfqpoint{1.366694in}{2.497367in}}%
\pgfpathlineto{\pgfqpoint{1.415083in}{2.497367in}}%
\pgfpathlineto{\pgfqpoint{1.463473in}{2.497367in}}%
\pgfpathlineto{\pgfqpoint{1.511862in}{2.497367in}}%
\pgfpathlineto{\pgfqpoint{1.560251in}{2.497367in}}%
\pgfpathlineto{\pgfqpoint{1.608640in}{2.497367in}}%
\pgfpathlineto{\pgfqpoint{1.657029in}{2.497367in}}%
\pgfpathlineto{\pgfqpoint{1.705418in}{2.497367in}}%
\pgfpathlineto{\pgfqpoint{1.753807in}{2.497367in}}%
\pgfpathlineto{\pgfqpoint{1.802196in}{2.497367in}}%
\pgfpathlineto{\pgfqpoint{1.850585in}{2.497367in}}%
\pgfpathlineto{\pgfqpoint{1.898974in}{2.497367in}}%
\pgfpathlineto{\pgfqpoint{1.947363in}{2.497367in}}%
\pgfpathlineto{\pgfqpoint{1.995752in}{2.497367in}}%
\pgfpathlineto{\pgfqpoint{2.044141in}{2.497367in}}%
\pgfpathlineto{\pgfqpoint{2.092531in}{2.497367in}}%
\pgfpathlineto{\pgfqpoint{2.140920in}{2.497367in}}%
\pgfpathlineto{\pgfqpoint{2.189309in}{2.497367in}}%
\pgfpathlineto{\pgfqpoint{2.237698in}{2.497367in}}%
\pgfpathlineto{\pgfqpoint{2.286087in}{2.497367in}}%
\pgfpathlineto{\pgfqpoint{2.334476in}{2.497367in}}%
\pgfpathlineto{\pgfqpoint{2.382865in}{2.497367in}}%
\pgfpathlineto{\pgfqpoint{2.431254in}{2.497367in}}%
\pgfpathlineto{\pgfqpoint{2.479643in}{2.497367in}}%
\pgfpathlineto{\pgfqpoint{2.528032in}{2.497367in}}%
\pgfpathlineto{\pgfqpoint{2.576421in}{2.497367in}}%
\pgfpathlineto{\pgfqpoint{2.624810in}{2.497367in}}%
\pgfpathlineto{\pgfqpoint{2.673200in}{2.497367in}}%
\pgfpathlineto{\pgfqpoint{2.721589in}{2.497367in}}%
\pgfpathlineto{\pgfqpoint{2.769978in}{2.497367in}}%
\pgfpathlineto{\pgfqpoint{2.818367in}{2.497367in}}%
\pgfpathlineto{\pgfqpoint{2.866756in}{2.497367in}}%
\pgfpathlineto{\pgfqpoint{2.915145in}{2.497367in}}%
\pgfpathlineto{\pgfqpoint{2.963534in}{2.497367in}}%
\pgfpathlineto{\pgfqpoint{3.011923in}{2.497367in}}%
\pgfpathlineto{\pgfqpoint{3.060312in}{2.497367in}}%
\pgfpathlineto{\pgfqpoint{3.108701in}{2.497367in}}%
\pgfpathlineto{\pgfqpoint{3.157090in}{2.497367in}}%
\pgfpathlineto{\pgfqpoint{3.205479in}{2.497367in}}%
\pgfpathlineto{\pgfqpoint{3.253869in}{0.723611in}}%
\pgfpathlineto{\pgfqpoint{3.302258in}{0.723611in}}%
\pgfpathlineto{\pgfqpoint{3.350647in}{0.723611in}}%
\pgfpathlineto{\pgfqpoint{3.399036in}{0.723611in}}%
\pgfpathlineto{\pgfqpoint{3.447425in}{0.723611in}}%
\pgfpathlineto{\pgfqpoint{3.495814in}{0.723611in}}%
\pgfpathlineto{\pgfqpoint{3.544203in}{0.723611in}}%
\pgfpathlineto{\pgfqpoint{3.592592in}{0.723611in}}%
\pgfpathlineto{\pgfqpoint{3.640981in}{0.723611in}}%
\pgfpathlineto{\pgfqpoint{3.689370in}{0.723611in}}%
\pgfusepath{stroke}%
\end{pgfscope}%
\begin{pgfscope}%
\pgfsetrectcap%
\pgfsetmiterjoin%
\pgfsetlinewidth{0.803000pt}%
\definecolor{currentstroke}{rgb}{0.000000,0.000000,0.000000}%
\pgfsetstrokecolor{currentstroke}%
\pgfsetdash{}{0pt}%
\pgfpathmoveto{\pgfqpoint{0.691667in}{0.580556in}}%
\pgfpathlineto{\pgfqpoint{0.691667in}{3.727778in}}%
\pgfusepath{stroke}%
\end{pgfscope}%
\begin{pgfscope}%
\pgfsetrectcap%
\pgfsetmiterjoin%
\pgfsetlinewidth{0.803000pt}%
\definecolor{currentstroke}{rgb}{0.000000,0.000000,0.000000}%
\pgfsetstrokecolor{currentstroke}%
\pgfsetdash{}{0pt}%
\pgfpathmoveto{\pgfqpoint{3.832118in}{0.580556in}}%
\pgfpathlineto{\pgfqpoint{3.832118in}{3.727778in}}%
\pgfusepath{stroke}%
\end{pgfscope}%
\begin{pgfscope}%
\pgfsetrectcap%
\pgfsetmiterjoin%
\pgfsetlinewidth{0.803000pt}%
\definecolor{currentstroke}{rgb}{0.000000,0.000000,0.000000}%
\pgfsetstrokecolor{currentstroke}%
\pgfsetdash{}{0pt}%
\pgfpathmoveto{\pgfqpoint{0.691667in}{0.580556in}}%
\pgfpathlineto{\pgfqpoint{3.832118in}{0.580556in}}%
\pgfusepath{stroke}%
\end{pgfscope}%
\begin{pgfscope}%
\pgfsetrectcap%
\pgfsetmiterjoin%
\pgfsetlinewidth{0.803000pt}%
\definecolor{currentstroke}{rgb}{0.000000,0.000000,0.000000}%
\pgfsetstrokecolor{currentstroke}%
\pgfsetdash{}{0pt}%
\pgfpathmoveto{\pgfqpoint{0.691667in}{3.727778in}}%
\pgfpathlineto{\pgfqpoint{3.832118in}{3.727778in}}%
\pgfusepath{stroke}%
\end{pgfscope}%
\begin{pgfscope}%
\definecolor{textcolor}{rgb}{0.000000,0.000000,0.000000}%
\pgfsetstrokecolor{textcolor}%
\pgfsetfillcolor{textcolor}%
\pgftext[x=2.261892in,y=3.811111in,,base]{\color{textcolor}\sffamily\fontsize{12.000000}{14.400000}\selectfont \(\displaystyle  t = 4.0 \)}%
\end{pgfscope}%
\begin{pgfscope}%
\pgfsetbuttcap%
\pgfsetmiterjoin%
\definecolor{currentfill}{rgb}{1.000000,1.000000,1.000000}%
\pgfsetfillcolor{currentfill}%
\pgfsetlinewidth{0.000000pt}%
\definecolor{currentstroke}{rgb}{0.000000,0.000000,0.000000}%
\pgfsetstrokecolor{currentstroke}%
\pgfsetstrokeopacity{0.000000}%
\pgfsetdash{}{0pt}%
\pgfpathmoveto{\pgfqpoint{4.660937in}{0.580556in}}%
\pgfpathlineto{\pgfqpoint{7.801389in}{0.580556in}}%
\pgfpathlineto{\pgfqpoint{7.801389in}{3.727778in}}%
\pgfpathlineto{\pgfqpoint{4.660937in}{3.727778in}}%
\pgfpathclose%
\pgfusepath{fill}%
\end{pgfscope}%
\begin{pgfscope}%
\pgfsetbuttcap%
\pgfsetroundjoin%
\definecolor{currentfill}{rgb}{0.000000,0.000000,0.000000}%
\pgfsetfillcolor{currentfill}%
\pgfsetlinewidth{0.803000pt}%
\definecolor{currentstroke}{rgb}{0.000000,0.000000,0.000000}%
\pgfsetstrokecolor{currentstroke}%
\pgfsetdash{}{0pt}%
\pgfsys@defobject{currentmarker}{\pgfqpoint{0.000000in}{-0.048611in}}{\pgfqpoint{0.000000in}{0.000000in}}{%
\pgfpathmoveto{\pgfqpoint{0.000000in}{0.000000in}}%
\pgfpathlineto{\pgfqpoint{0.000000in}{-0.048611in}}%
\pgfusepath{stroke,fill}%
}%
\begin{pgfscope}%
\pgfsys@transformshift{4.779491in}{0.580556in}%
\pgfsys@useobject{currentmarker}{}%
\end{pgfscope}%
\end{pgfscope}%
\begin{pgfscope}%
\definecolor{textcolor}{rgb}{0.000000,0.000000,0.000000}%
\pgfsetstrokecolor{textcolor}%
\pgfsetfillcolor{textcolor}%
\pgftext[x=4.779491in,y=0.483333in,,top]{\color{textcolor}\sffamily\fontsize{10.000000}{12.000000}\selectfont −3}%
\end{pgfscope}%
\begin{pgfscope}%
\pgfsetbuttcap%
\pgfsetroundjoin%
\definecolor{currentfill}{rgb}{0.000000,0.000000,0.000000}%
\pgfsetfillcolor{currentfill}%
\pgfsetlinewidth{0.803000pt}%
\definecolor{currentstroke}{rgb}{0.000000,0.000000,0.000000}%
\pgfsetstrokecolor{currentstroke}%
\pgfsetdash{}{0pt}%
\pgfsys@defobject{currentmarker}{\pgfqpoint{0.000000in}{-0.048611in}}{\pgfqpoint{0.000000in}{0.000000in}}{%
\pgfpathmoveto{\pgfqpoint{0.000000in}{0.000000in}}%
\pgfpathlineto{\pgfqpoint{0.000000in}{-0.048611in}}%
\pgfusepath{stroke,fill}%
}%
\begin{pgfscope}%
\pgfsys@transformshift{5.263382in}{0.580556in}%
\pgfsys@useobject{currentmarker}{}%
\end{pgfscope}%
\end{pgfscope}%
\begin{pgfscope}%
\definecolor{textcolor}{rgb}{0.000000,0.000000,0.000000}%
\pgfsetstrokecolor{textcolor}%
\pgfsetfillcolor{textcolor}%
\pgftext[x=5.263382in,y=0.483333in,,top]{\color{textcolor}\sffamily\fontsize{10.000000}{12.000000}\selectfont −2}%
\end{pgfscope}%
\begin{pgfscope}%
\pgfsetbuttcap%
\pgfsetroundjoin%
\definecolor{currentfill}{rgb}{0.000000,0.000000,0.000000}%
\pgfsetfillcolor{currentfill}%
\pgfsetlinewidth{0.803000pt}%
\definecolor{currentstroke}{rgb}{0.000000,0.000000,0.000000}%
\pgfsetstrokecolor{currentstroke}%
\pgfsetdash{}{0pt}%
\pgfsys@defobject{currentmarker}{\pgfqpoint{0.000000in}{-0.048611in}}{\pgfqpoint{0.000000in}{0.000000in}}{%
\pgfpathmoveto{\pgfqpoint{0.000000in}{0.000000in}}%
\pgfpathlineto{\pgfqpoint{0.000000in}{-0.048611in}}%
\pgfusepath{stroke,fill}%
}%
\begin{pgfscope}%
\pgfsys@transformshift{5.747272in}{0.580556in}%
\pgfsys@useobject{currentmarker}{}%
\end{pgfscope}%
\end{pgfscope}%
\begin{pgfscope}%
\definecolor{textcolor}{rgb}{0.000000,0.000000,0.000000}%
\pgfsetstrokecolor{textcolor}%
\pgfsetfillcolor{textcolor}%
\pgftext[x=5.747272in,y=0.483333in,,top]{\color{textcolor}\sffamily\fontsize{10.000000}{12.000000}\selectfont −1}%
\end{pgfscope}%
\begin{pgfscope}%
\pgfsetbuttcap%
\pgfsetroundjoin%
\definecolor{currentfill}{rgb}{0.000000,0.000000,0.000000}%
\pgfsetfillcolor{currentfill}%
\pgfsetlinewidth{0.803000pt}%
\definecolor{currentstroke}{rgb}{0.000000,0.000000,0.000000}%
\pgfsetstrokecolor{currentstroke}%
\pgfsetdash{}{0pt}%
\pgfsys@defobject{currentmarker}{\pgfqpoint{0.000000in}{-0.048611in}}{\pgfqpoint{0.000000in}{0.000000in}}{%
\pgfpathmoveto{\pgfqpoint{0.000000in}{0.000000in}}%
\pgfpathlineto{\pgfqpoint{0.000000in}{-0.048611in}}%
\pgfusepath{stroke,fill}%
}%
\begin{pgfscope}%
\pgfsys@transformshift{6.231163in}{0.580556in}%
\pgfsys@useobject{currentmarker}{}%
\end{pgfscope}%
\end{pgfscope}%
\begin{pgfscope}%
\definecolor{textcolor}{rgb}{0.000000,0.000000,0.000000}%
\pgfsetstrokecolor{textcolor}%
\pgfsetfillcolor{textcolor}%
\pgftext[x=6.231163in,y=0.483333in,,top]{\color{textcolor}\sffamily\fontsize{10.000000}{12.000000}\selectfont 0}%
\end{pgfscope}%
\begin{pgfscope}%
\pgfsetbuttcap%
\pgfsetroundjoin%
\definecolor{currentfill}{rgb}{0.000000,0.000000,0.000000}%
\pgfsetfillcolor{currentfill}%
\pgfsetlinewidth{0.803000pt}%
\definecolor{currentstroke}{rgb}{0.000000,0.000000,0.000000}%
\pgfsetstrokecolor{currentstroke}%
\pgfsetdash{}{0pt}%
\pgfsys@defobject{currentmarker}{\pgfqpoint{0.000000in}{-0.048611in}}{\pgfqpoint{0.000000in}{0.000000in}}{%
\pgfpathmoveto{\pgfqpoint{0.000000in}{0.000000in}}%
\pgfpathlineto{\pgfqpoint{0.000000in}{-0.048611in}}%
\pgfusepath{stroke,fill}%
}%
\begin{pgfscope}%
\pgfsys@transformshift{6.715054in}{0.580556in}%
\pgfsys@useobject{currentmarker}{}%
\end{pgfscope}%
\end{pgfscope}%
\begin{pgfscope}%
\definecolor{textcolor}{rgb}{0.000000,0.000000,0.000000}%
\pgfsetstrokecolor{textcolor}%
\pgfsetfillcolor{textcolor}%
\pgftext[x=6.715054in,y=0.483333in,,top]{\color{textcolor}\sffamily\fontsize{10.000000}{12.000000}\selectfont 1}%
\end{pgfscope}%
\begin{pgfscope}%
\pgfsetbuttcap%
\pgfsetroundjoin%
\definecolor{currentfill}{rgb}{0.000000,0.000000,0.000000}%
\pgfsetfillcolor{currentfill}%
\pgfsetlinewidth{0.803000pt}%
\definecolor{currentstroke}{rgb}{0.000000,0.000000,0.000000}%
\pgfsetstrokecolor{currentstroke}%
\pgfsetdash{}{0pt}%
\pgfsys@defobject{currentmarker}{\pgfqpoint{0.000000in}{-0.048611in}}{\pgfqpoint{0.000000in}{0.000000in}}{%
\pgfpathmoveto{\pgfqpoint{0.000000in}{0.000000in}}%
\pgfpathlineto{\pgfqpoint{0.000000in}{-0.048611in}}%
\pgfusepath{stroke,fill}%
}%
\begin{pgfscope}%
\pgfsys@transformshift{7.198945in}{0.580556in}%
\pgfsys@useobject{currentmarker}{}%
\end{pgfscope}%
\end{pgfscope}%
\begin{pgfscope}%
\definecolor{textcolor}{rgb}{0.000000,0.000000,0.000000}%
\pgfsetstrokecolor{textcolor}%
\pgfsetfillcolor{textcolor}%
\pgftext[x=7.198945in,y=0.483333in,,top]{\color{textcolor}\sffamily\fontsize{10.000000}{12.000000}\selectfont 2}%
\end{pgfscope}%
\begin{pgfscope}%
\pgfsetbuttcap%
\pgfsetroundjoin%
\definecolor{currentfill}{rgb}{0.000000,0.000000,0.000000}%
\pgfsetfillcolor{currentfill}%
\pgfsetlinewidth{0.803000pt}%
\definecolor{currentstroke}{rgb}{0.000000,0.000000,0.000000}%
\pgfsetstrokecolor{currentstroke}%
\pgfsetdash{}{0pt}%
\pgfsys@defobject{currentmarker}{\pgfqpoint{0.000000in}{-0.048611in}}{\pgfqpoint{0.000000in}{0.000000in}}{%
\pgfpathmoveto{\pgfqpoint{0.000000in}{0.000000in}}%
\pgfpathlineto{\pgfqpoint{0.000000in}{-0.048611in}}%
\pgfusepath{stroke,fill}%
}%
\begin{pgfscope}%
\pgfsys@transformshift{7.682836in}{0.580556in}%
\pgfsys@useobject{currentmarker}{}%
\end{pgfscope}%
\end{pgfscope}%
\begin{pgfscope}%
\definecolor{textcolor}{rgb}{0.000000,0.000000,0.000000}%
\pgfsetstrokecolor{textcolor}%
\pgfsetfillcolor{textcolor}%
\pgftext[x=7.682836in,y=0.483333in,,top]{\color{textcolor}\sffamily\fontsize{10.000000}{12.000000}\selectfont 3}%
\end{pgfscope}%
\begin{pgfscope}%
\definecolor{textcolor}{rgb}{0.000000,0.000000,0.000000}%
\pgfsetstrokecolor{textcolor}%
\pgfsetfillcolor{textcolor}%
\pgftext[x=6.231163in,y=0.293365in,,top]{\color{textcolor}\sffamily\fontsize{10.000000}{12.000000}\selectfont \(\displaystyle x\)}%
\end{pgfscope}%
\begin{pgfscope}%
\pgfsetbuttcap%
\pgfsetroundjoin%
\definecolor{currentfill}{rgb}{0.000000,0.000000,0.000000}%
\pgfsetfillcolor{currentfill}%
\pgfsetlinewidth{0.803000pt}%
\definecolor{currentstroke}{rgb}{0.000000,0.000000,0.000000}%
\pgfsetstrokecolor{currentstroke}%
\pgfsetdash{}{0pt}%
\pgfsys@defobject{currentmarker}{\pgfqpoint{-0.048611in}{0.000000in}}{\pgfqpoint{0.000000in}{0.000000in}}{%
\pgfpathmoveto{\pgfqpoint{0.000000in}{0.000000in}}%
\pgfpathlineto{\pgfqpoint{-0.048611in}{0.000000in}}%
\pgfusepath{stroke,fill}%
}%
\begin{pgfscope}%
\pgfsys@transformshift{4.660937in}{0.723611in}%
\pgfsys@useobject{currentmarker}{}%
\end{pgfscope}%
\end{pgfscope}%
\begin{pgfscope}%
\definecolor{textcolor}{rgb}{0.000000,0.000000,0.000000}%
\pgfsetstrokecolor{textcolor}%
\pgfsetfillcolor{textcolor}%
\pgftext[x=4.342836in,y=0.670850in,left,base]{\color{textcolor}\sffamily\fontsize{10.000000}{12.000000}\selectfont 0.0}%
\end{pgfscope}%
\begin{pgfscope}%
\pgfsetbuttcap%
\pgfsetroundjoin%
\definecolor{currentfill}{rgb}{0.000000,0.000000,0.000000}%
\pgfsetfillcolor{currentfill}%
\pgfsetlinewidth{0.803000pt}%
\definecolor{currentstroke}{rgb}{0.000000,0.000000,0.000000}%
\pgfsetstrokecolor{currentstroke}%
\pgfsetdash{}{0pt}%
\pgfsys@defobject{currentmarker}{\pgfqpoint{-0.048611in}{0.000000in}}{\pgfqpoint{0.000000in}{0.000000in}}{%
\pgfpathmoveto{\pgfqpoint{0.000000in}{0.000000in}}%
\pgfpathlineto{\pgfqpoint{-0.048611in}{0.000000in}}%
\pgfusepath{stroke,fill}%
}%
\begin{pgfscope}%
\pgfsys@transformshift{4.660937in}{1.108617in}%
\pgfsys@useobject{currentmarker}{}%
\end{pgfscope}%
\end{pgfscope}%
\begin{pgfscope}%
\definecolor{textcolor}{rgb}{0.000000,0.000000,0.000000}%
\pgfsetstrokecolor{textcolor}%
\pgfsetfillcolor{textcolor}%
\pgftext[x=4.342836in,y=1.055856in,left,base]{\color{textcolor}\sffamily\fontsize{10.000000}{12.000000}\selectfont 0.2}%
\end{pgfscope}%
\begin{pgfscope}%
\pgfsetbuttcap%
\pgfsetroundjoin%
\definecolor{currentfill}{rgb}{0.000000,0.000000,0.000000}%
\pgfsetfillcolor{currentfill}%
\pgfsetlinewidth{0.803000pt}%
\definecolor{currentstroke}{rgb}{0.000000,0.000000,0.000000}%
\pgfsetstrokecolor{currentstroke}%
\pgfsetdash{}{0pt}%
\pgfsys@defobject{currentmarker}{\pgfqpoint{-0.048611in}{0.000000in}}{\pgfqpoint{0.000000in}{0.000000in}}{%
\pgfpathmoveto{\pgfqpoint{0.000000in}{0.000000in}}%
\pgfpathlineto{\pgfqpoint{-0.048611in}{0.000000in}}%
\pgfusepath{stroke,fill}%
}%
\begin{pgfscope}%
\pgfsys@transformshift{4.660937in}{1.493623in}%
\pgfsys@useobject{currentmarker}{}%
\end{pgfscope}%
\end{pgfscope}%
\begin{pgfscope}%
\definecolor{textcolor}{rgb}{0.000000,0.000000,0.000000}%
\pgfsetstrokecolor{textcolor}%
\pgfsetfillcolor{textcolor}%
\pgftext[x=4.342836in,y=1.440862in,left,base]{\color{textcolor}\sffamily\fontsize{10.000000}{12.000000}\selectfont 0.4}%
\end{pgfscope}%
\begin{pgfscope}%
\pgfsetbuttcap%
\pgfsetroundjoin%
\definecolor{currentfill}{rgb}{0.000000,0.000000,0.000000}%
\pgfsetfillcolor{currentfill}%
\pgfsetlinewidth{0.803000pt}%
\definecolor{currentstroke}{rgb}{0.000000,0.000000,0.000000}%
\pgfsetstrokecolor{currentstroke}%
\pgfsetdash{}{0pt}%
\pgfsys@defobject{currentmarker}{\pgfqpoint{-0.048611in}{0.000000in}}{\pgfqpoint{0.000000in}{0.000000in}}{%
\pgfpathmoveto{\pgfqpoint{0.000000in}{0.000000in}}%
\pgfpathlineto{\pgfqpoint{-0.048611in}{0.000000in}}%
\pgfusepath{stroke,fill}%
}%
\begin{pgfscope}%
\pgfsys@transformshift{4.660937in}{1.878629in}%
\pgfsys@useobject{currentmarker}{}%
\end{pgfscope}%
\end{pgfscope}%
\begin{pgfscope}%
\definecolor{textcolor}{rgb}{0.000000,0.000000,0.000000}%
\pgfsetstrokecolor{textcolor}%
\pgfsetfillcolor{textcolor}%
\pgftext[x=4.342836in,y=1.825868in,left,base]{\color{textcolor}\sffamily\fontsize{10.000000}{12.000000}\selectfont 0.6}%
\end{pgfscope}%
\begin{pgfscope}%
\pgfsetbuttcap%
\pgfsetroundjoin%
\definecolor{currentfill}{rgb}{0.000000,0.000000,0.000000}%
\pgfsetfillcolor{currentfill}%
\pgfsetlinewidth{0.803000pt}%
\definecolor{currentstroke}{rgb}{0.000000,0.000000,0.000000}%
\pgfsetstrokecolor{currentstroke}%
\pgfsetdash{}{0pt}%
\pgfsys@defobject{currentmarker}{\pgfqpoint{-0.048611in}{0.000000in}}{\pgfqpoint{0.000000in}{0.000000in}}{%
\pgfpathmoveto{\pgfqpoint{0.000000in}{0.000000in}}%
\pgfpathlineto{\pgfqpoint{-0.048611in}{0.000000in}}%
\pgfusepath{stroke,fill}%
}%
\begin{pgfscope}%
\pgfsys@transformshift{4.660937in}{2.263635in}%
\pgfsys@useobject{currentmarker}{}%
\end{pgfscope}%
\end{pgfscope}%
\begin{pgfscope}%
\definecolor{textcolor}{rgb}{0.000000,0.000000,0.000000}%
\pgfsetstrokecolor{textcolor}%
\pgfsetfillcolor{textcolor}%
\pgftext[x=4.342836in,y=2.210874in,left,base]{\color{textcolor}\sffamily\fontsize{10.000000}{12.000000}\selectfont 0.8}%
\end{pgfscope}%
\begin{pgfscope}%
\pgfsetbuttcap%
\pgfsetroundjoin%
\definecolor{currentfill}{rgb}{0.000000,0.000000,0.000000}%
\pgfsetfillcolor{currentfill}%
\pgfsetlinewidth{0.803000pt}%
\definecolor{currentstroke}{rgb}{0.000000,0.000000,0.000000}%
\pgfsetstrokecolor{currentstroke}%
\pgfsetdash{}{0pt}%
\pgfsys@defobject{currentmarker}{\pgfqpoint{-0.048611in}{0.000000in}}{\pgfqpoint{0.000000in}{0.000000in}}{%
\pgfpathmoveto{\pgfqpoint{0.000000in}{0.000000in}}%
\pgfpathlineto{\pgfqpoint{-0.048611in}{0.000000in}}%
\pgfusepath{stroke,fill}%
}%
\begin{pgfscope}%
\pgfsys@transformshift{4.660937in}{2.648641in}%
\pgfsys@useobject{currentmarker}{}%
\end{pgfscope}%
\end{pgfscope}%
\begin{pgfscope}%
\definecolor{textcolor}{rgb}{0.000000,0.000000,0.000000}%
\pgfsetstrokecolor{textcolor}%
\pgfsetfillcolor{textcolor}%
\pgftext[x=4.342836in,y=2.595880in,left,base]{\color{textcolor}\sffamily\fontsize{10.000000}{12.000000}\selectfont 1.0}%
\end{pgfscope}%
\begin{pgfscope}%
\pgfsetbuttcap%
\pgfsetroundjoin%
\definecolor{currentfill}{rgb}{0.000000,0.000000,0.000000}%
\pgfsetfillcolor{currentfill}%
\pgfsetlinewidth{0.803000pt}%
\definecolor{currentstroke}{rgb}{0.000000,0.000000,0.000000}%
\pgfsetstrokecolor{currentstroke}%
\pgfsetdash{}{0pt}%
\pgfsys@defobject{currentmarker}{\pgfqpoint{-0.048611in}{0.000000in}}{\pgfqpoint{0.000000in}{0.000000in}}{%
\pgfpathmoveto{\pgfqpoint{0.000000in}{0.000000in}}%
\pgfpathlineto{\pgfqpoint{-0.048611in}{0.000000in}}%
\pgfusepath{stroke,fill}%
}%
\begin{pgfscope}%
\pgfsys@transformshift{4.660937in}{3.033647in}%
\pgfsys@useobject{currentmarker}{}%
\end{pgfscope}%
\end{pgfscope}%
\begin{pgfscope}%
\definecolor{textcolor}{rgb}{0.000000,0.000000,0.000000}%
\pgfsetstrokecolor{textcolor}%
\pgfsetfillcolor{textcolor}%
\pgftext[x=4.342836in,y=2.980886in,left,base]{\color{textcolor}\sffamily\fontsize{10.000000}{12.000000}\selectfont 1.2}%
\end{pgfscope}%
\begin{pgfscope}%
\pgfsetbuttcap%
\pgfsetroundjoin%
\definecolor{currentfill}{rgb}{0.000000,0.000000,0.000000}%
\pgfsetfillcolor{currentfill}%
\pgfsetlinewidth{0.803000pt}%
\definecolor{currentstroke}{rgb}{0.000000,0.000000,0.000000}%
\pgfsetstrokecolor{currentstroke}%
\pgfsetdash{}{0pt}%
\pgfsys@defobject{currentmarker}{\pgfqpoint{-0.048611in}{0.000000in}}{\pgfqpoint{0.000000in}{0.000000in}}{%
\pgfpathmoveto{\pgfqpoint{0.000000in}{0.000000in}}%
\pgfpathlineto{\pgfqpoint{-0.048611in}{0.000000in}}%
\pgfusepath{stroke,fill}%
}%
\begin{pgfscope}%
\pgfsys@transformshift{4.660937in}{3.418653in}%
\pgfsys@useobject{currentmarker}{}%
\end{pgfscope}%
\end{pgfscope}%
\begin{pgfscope}%
\definecolor{textcolor}{rgb}{0.000000,0.000000,0.000000}%
\pgfsetstrokecolor{textcolor}%
\pgfsetfillcolor{textcolor}%
\pgftext[x=4.342836in,y=3.365892in,left,base]{\color{textcolor}\sffamily\fontsize{10.000000}{12.000000}\selectfont 1.4}%
\end{pgfscope}%
\begin{pgfscope}%
\definecolor{textcolor}{rgb}{0.000000,0.000000,0.000000}%
\pgfsetstrokecolor{textcolor}%
\pgfsetfillcolor{textcolor}%
\pgftext[x=4.287280in,y=2.154167in,,bottom,rotate=90.000000]{\color{textcolor}\sffamily\fontsize{10.000000}{12.000000}\selectfont \(\displaystyle U\)}%
\end{pgfscope}%
\begin{pgfscope}%
\pgfpathrectangle{\pgfqpoint{4.660937in}{0.580556in}}{\pgfqpoint{3.140451in}{3.147222in}}%
\pgfusepath{clip}%
\pgfsetrectcap%
\pgfsetroundjoin%
\pgfsetlinewidth{1.505625pt}%
\definecolor{currentstroke}{rgb}{0.121569,0.466667,0.705882}%
\pgfsetstrokecolor{currentstroke}%
\pgfsetdash{}{0pt}%
\pgfpathmoveto{\pgfqpoint{4.803685in}{2.648695in}}%
\pgfpathlineto{\pgfqpoint{4.852074in}{2.648695in}}%
\pgfpathlineto{\pgfqpoint{4.900463in}{2.648701in}}%
\pgfpathlineto{\pgfqpoint{4.948853in}{2.648701in}}%
\pgfpathlineto{\pgfqpoint{4.997242in}{2.648689in}}%
\pgfpathlineto{\pgfqpoint{5.045631in}{2.648688in}}%
\pgfpathlineto{\pgfqpoint{5.094020in}{2.648704in}}%
\pgfpathlineto{\pgfqpoint{5.142409in}{2.648706in}}%
\pgfpathlineto{\pgfqpoint{5.190798in}{2.648690in}}%
\pgfpathlineto{\pgfqpoint{5.239187in}{2.648688in}}%
\pgfpathlineto{\pgfqpoint{5.287576in}{2.648706in}}%
\pgfpathlineto{\pgfqpoint{5.335965in}{2.648691in}}%
\pgfpathlineto{\pgfqpoint{5.384354in}{2.648680in}}%
\pgfpathlineto{\pgfqpoint{5.432743in}{2.648724in}}%
\pgfpathlineto{\pgfqpoint{5.481132in}{2.648672in}}%
\pgfpathlineto{\pgfqpoint{5.529522in}{2.648759in}}%
\pgfpathlineto{\pgfqpoint{5.577911in}{2.648656in}}%
\pgfpathlineto{\pgfqpoint{5.626300in}{2.648488in}}%
\pgfpathlineto{\pgfqpoint{5.674689in}{2.649193in}}%
\pgfpathlineto{\pgfqpoint{5.723078in}{2.648243in}}%
\pgfpathlineto{\pgfqpoint{5.771467in}{2.648456in}}%
\pgfpathlineto{\pgfqpoint{5.819856in}{2.650268in}}%
\pgfpathlineto{\pgfqpoint{5.868245in}{2.646219in}}%
\pgfpathlineto{\pgfqpoint{5.916634in}{2.649589in}}%
\pgfpathlineto{\pgfqpoint{5.965023in}{2.652794in}}%
\pgfpathlineto{\pgfqpoint{6.013412in}{2.640173in}}%
\pgfpathlineto{\pgfqpoint{6.061801in}{2.652831in}}%
\pgfpathlineto{\pgfqpoint{6.110190in}{2.659674in}}%
\pgfpathlineto{\pgfqpoint{6.158580in}{2.626295in}}%
\pgfpathlineto{\pgfqpoint{6.206969in}{2.656538in}}%
\pgfpathlineto{\pgfqpoint{6.255358in}{2.679831in}}%
\pgfpathlineto{\pgfqpoint{6.303747in}{2.599931in}}%
\pgfpathlineto{\pgfqpoint{6.352136in}{2.645682in}}%
\pgfpathlineto{\pgfqpoint{6.400525in}{2.732597in}}%
\pgfpathlineto{\pgfqpoint{6.448914in}{2.576501in}}%
\pgfpathlineto{\pgfqpoint{6.497303in}{2.576029in}}%
\pgfpathlineto{\pgfqpoint{6.545692in}{2.814208in}}%
\pgfpathlineto{\pgfqpoint{6.594081in}{2.627929in}}%
\pgfpathlineto{\pgfqpoint{6.642470in}{2.419058in}}%
\pgfpathlineto{\pgfqpoint{6.690859in}{2.819195in}}%
\pgfpathlineto{\pgfqpoint{6.739249in}{2.846497in}}%
\pgfpathlineto{\pgfqpoint{6.787638in}{2.283493in}}%
\pgfpathlineto{\pgfqpoint{6.836027in}{2.568302in}}%
\pgfpathlineto{\pgfqpoint{6.884416in}{3.139414in}}%
\pgfpathlineto{\pgfqpoint{6.932805in}{2.419609in}}%
\pgfpathlineto{\pgfqpoint{6.981194in}{2.130176in}}%
\pgfpathlineto{\pgfqpoint{7.029583in}{3.132143in}}%
\pgfpathlineto{\pgfqpoint{7.077972in}{2.914515in}}%
\pgfpathlineto{\pgfqpoint{7.126361in}{1.818867in}}%
\pgfpathlineto{\pgfqpoint{7.174750in}{2.610394in}}%
\pgfpathlineto{\pgfqpoint{7.223139in}{3.539388in}}%
\pgfpathlineto{\pgfqpoint{7.271528in}{1.941303in}}%
\pgfpathlineto{\pgfqpoint{7.319918in}{1.681140in}}%
\pgfpathlineto{\pgfqpoint{7.368307in}{3.584722in}}%
\pgfpathlineto{\pgfqpoint{7.416696in}{2.925344in}}%
\pgfpathlineto{\pgfqpoint{7.465085in}{0.917557in}}%
\pgfpathlineto{\pgfqpoint{7.513474in}{0.724065in}}%
\pgfpathlineto{\pgfqpoint{7.561863in}{0.723611in}}%
\pgfpathlineto{\pgfqpoint{7.610252in}{0.723611in}}%
\pgfpathlineto{\pgfqpoint{7.658641in}{0.723611in}}%
\pgfusepath{stroke}%
\end{pgfscope}%
\begin{pgfscope}%
\pgfpathrectangle{\pgfqpoint{4.660937in}{0.580556in}}{\pgfqpoint{3.140451in}{3.147222in}}%
\pgfusepath{clip}%
\pgfsetrectcap%
\pgfsetroundjoin%
\pgfsetlinewidth{1.505625pt}%
\definecolor{currentstroke}{rgb}{1.000000,0.498039,0.054902}%
\pgfsetstrokecolor{currentstroke}%
\pgfsetdash{}{0pt}%
\pgfpathmoveto{\pgfqpoint{4.803685in}{2.648695in}}%
\pgfpathlineto{\pgfqpoint{4.852074in}{2.648695in}}%
\pgfpathlineto{\pgfqpoint{4.900463in}{2.648701in}}%
\pgfpathlineto{\pgfqpoint{4.948853in}{2.648701in}}%
\pgfpathlineto{\pgfqpoint{4.997242in}{2.648689in}}%
\pgfpathlineto{\pgfqpoint{5.045631in}{2.648688in}}%
\pgfpathlineto{\pgfqpoint{5.094020in}{2.648704in}}%
\pgfpathlineto{\pgfqpoint{5.142409in}{2.648706in}}%
\pgfpathlineto{\pgfqpoint{5.190798in}{2.648690in}}%
\pgfpathlineto{\pgfqpoint{5.239187in}{2.648689in}}%
\pgfpathlineto{\pgfqpoint{5.287576in}{2.648704in}}%
\pgfpathlineto{\pgfqpoint{5.335965in}{2.648692in}}%
\pgfpathlineto{\pgfqpoint{5.384354in}{2.648686in}}%
\pgfpathlineto{\pgfqpoint{5.432743in}{2.648706in}}%
\pgfpathlineto{\pgfqpoint{5.481132in}{2.648704in}}%
\pgfpathlineto{\pgfqpoint{5.529522in}{2.648733in}}%
\pgfpathlineto{\pgfqpoint{5.577911in}{2.648625in}}%
\pgfpathlineto{\pgfqpoint{5.626300in}{2.648641in}}%
\pgfpathlineto{\pgfqpoint{5.674689in}{2.648925in}}%
\pgfpathlineto{\pgfqpoint{5.723078in}{2.648428in}}%
\pgfpathlineto{\pgfqpoint{5.771467in}{2.648741in}}%
\pgfpathlineto{\pgfqpoint{5.819856in}{2.649264in}}%
\pgfpathlineto{\pgfqpoint{5.868245in}{2.647448in}}%
\pgfpathlineto{\pgfqpoint{5.916634in}{2.649610in}}%
\pgfpathlineto{\pgfqpoint{5.965023in}{2.650182in}}%
\pgfpathlineto{\pgfqpoint{6.013412in}{2.644252in}}%
\pgfpathlineto{\pgfqpoint{6.061801in}{2.651853in}}%
\pgfpathlineto{\pgfqpoint{6.110190in}{2.653345in}}%
\pgfpathlineto{\pgfqpoint{6.158580in}{2.636441in}}%
\pgfpathlineto{\pgfqpoint{6.206969in}{2.655141in}}%
\pgfpathlineto{\pgfqpoint{6.255358in}{2.663955in}}%
\pgfpathlineto{\pgfqpoint{6.303747in}{2.619657in}}%
\pgfpathlineto{\pgfqpoint{6.352136in}{2.651245in}}%
\pgfpathlineto{\pgfqpoint{6.400525in}{2.695670in}}%
\pgfpathlineto{\pgfqpoint{6.448914in}{2.601345in}}%
\pgfpathlineto{\pgfqpoint{6.497303in}{2.610481in}}%
\pgfpathlineto{\pgfqpoint{6.545692in}{2.749104in}}%
\pgfpathlineto{\pgfqpoint{6.594081in}{2.628397in}}%
\pgfpathlineto{\pgfqpoint{6.642470in}{2.506975in}}%
\pgfpathlineto{\pgfqpoint{6.690859in}{2.760992in}}%
\pgfpathlineto{\pgfqpoint{6.739249in}{2.763463in}}%
\pgfpathlineto{\pgfqpoint{6.787638in}{2.406122in}}%
\pgfpathlineto{\pgfqpoint{6.836027in}{2.607807in}}%
\pgfpathlineto{\pgfqpoint{6.884416in}{2.944189in}}%
\pgfpathlineto{\pgfqpoint{6.932805in}{2.495461in}}%
\pgfpathlineto{\pgfqpoint{6.981194in}{2.304785in}}%
\pgfpathlineto{\pgfqpoint{7.029583in}{2.941415in}}%
\pgfpathlineto{\pgfqpoint{7.077972in}{2.810839in}}%
\pgfpathlineto{\pgfqpoint{7.126361in}{2.071036in}}%
\pgfpathlineto{\pgfqpoint{7.174750in}{2.615179in}}%
\pgfpathlineto{\pgfqpoint{7.223139in}{3.157289in}}%
\pgfpathlineto{\pgfqpoint{7.271528in}{2.186400in}}%
\pgfpathlineto{\pgfqpoint{7.319918in}{1.938231in}}%
\pgfpathlineto{\pgfqpoint{7.368307in}{3.135955in}}%
\pgfpathlineto{\pgfqpoint{7.416696in}{2.854085in}}%
\pgfpathlineto{\pgfqpoint{7.465085in}{1.211459in}}%
\pgfpathlineto{\pgfqpoint{7.513474in}{0.732880in}}%
\pgfpathlineto{\pgfqpoint{7.561863in}{0.723612in}}%
\pgfpathlineto{\pgfqpoint{7.610252in}{0.723611in}}%
\pgfpathlineto{\pgfqpoint{7.658641in}{0.723611in}}%
\pgfusepath{stroke}%
\end{pgfscope}%
\begin{pgfscope}%
\pgfpathrectangle{\pgfqpoint{4.660937in}{0.580556in}}{\pgfqpoint{3.140451in}{3.147222in}}%
\pgfusepath{clip}%
\pgfsetrectcap%
\pgfsetroundjoin%
\pgfsetlinewidth{1.505625pt}%
\definecolor{currentstroke}{rgb}{0.172549,0.627451,0.172549}%
\pgfsetstrokecolor{currentstroke}%
\pgfsetdash{}{0pt}%
\pgfpathmoveto{\pgfqpoint{4.803685in}{2.648641in}}%
\pgfpathlineto{\pgfqpoint{4.852074in}{2.648641in}}%
\pgfpathlineto{\pgfqpoint{4.900463in}{2.648641in}}%
\pgfpathlineto{\pgfqpoint{4.948853in}{2.648641in}}%
\pgfpathlineto{\pgfqpoint{4.997242in}{2.648641in}}%
\pgfpathlineto{\pgfqpoint{5.045631in}{2.648641in}}%
\pgfpathlineto{\pgfqpoint{5.094020in}{2.648641in}}%
\pgfpathlineto{\pgfqpoint{5.142409in}{2.648641in}}%
\pgfpathlineto{\pgfqpoint{5.190798in}{2.648641in}}%
\pgfpathlineto{\pgfqpoint{5.239187in}{2.648641in}}%
\pgfpathlineto{\pgfqpoint{5.287576in}{2.648641in}}%
\pgfpathlineto{\pgfqpoint{5.335965in}{2.648641in}}%
\pgfpathlineto{\pgfqpoint{5.384354in}{2.648641in}}%
\pgfpathlineto{\pgfqpoint{5.432743in}{2.648641in}}%
\pgfpathlineto{\pgfqpoint{5.481132in}{2.648641in}}%
\pgfpathlineto{\pgfqpoint{5.529522in}{2.648641in}}%
\pgfpathlineto{\pgfqpoint{5.577911in}{2.648641in}}%
\pgfpathlineto{\pgfqpoint{5.626300in}{2.648641in}}%
\pgfpathlineto{\pgfqpoint{5.674689in}{2.648641in}}%
\pgfpathlineto{\pgfqpoint{5.723078in}{2.648641in}}%
\pgfpathlineto{\pgfqpoint{5.771467in}{2.648641in}}%
\pgfpathlineto{\pgfqpoint{5.819856in}{2.648641in}}%
\pgfpathlineto{\pgfqpoint{5.868245in}{2.648641in}}%
\pgfpathlineto{\pgfqpoint{5.916634in}{2.648641in}}%
\pgfpathlineto{\pgfqpoint{5.965023in}{2.648641in}}%
\pgfpathlineto{\pgfqpoint{6.013412in}{2.648641in}}%
\pgfpathlineto{\pgfqpoint{6.061801in}{2.648641in}}%
\pgfpathlineto{\pgfqpoint{6.110190in}{2.648641in}}%
\pgfpathlineto{\pgfqpoint{6.158580in}{2.648641in}}%
\pgfpathlineto{\pgfqpoint{6.206969in}{2.648641in}}%
\pgfpathlineto{\pgfqpoint{6.255358in}{2.648641in}}%
\pgfpathlineto{\pgfqpoint{6.303747in}{2.648641in}}%
\pgfpathlineto{\pgfqpoint{6.352136in}{2.648641in}}%
\pgfpathlineto{\pgfqpoint{6.400525in}{2.648641in}}%
\pgfpathlineto{\pgfqpoint{6.448914in}{2.648641in}}%
\pgfpathlineto{\pgfqpoint{6.497303in}{2.648641in}}%
\pgfpathlineto{\pgfqpoint{6.545692in}{2.648641in}}%
\pgfpathlineto{\pgfqpoint{6.594081in}{2.648641in}}%
\pgfpathlineto{\pgfqpoint{6.642470in}{2.648641in}}%
\pgfpathlineto{\pgfqpoint{6.690859in}{2.648641in}}%
\pgfpathlineto{\pgfqpoint{6.739249in}{2.648640in}}%
\pgfpathlineto{\pgfqpoint{6.787638in}{2.648639in}}%
\pgfpathlineto{\pgfqpoint{6.836027in}{2.648636in}}%
\pgfpathlineto{\pgfqpoint{6.884416in}{2.648631in}}%
\pgfpathlineto{\pgfqpoint{6.932805in}{2.648620in}}%
\pgfpathlineto{\pgfqpoint{6.981194in}{2.648599in}}%
\pgfpathlineto{\pgfqpoint{7.029583in}{2.648558in}}%
\pgfpathlineto{\pgfqpoint{7.077972in}{2.648481in}}%
\pgfpathlineto{\pgfqpoint{7.126361in}{2.648331in}}%
\pgfpathlineto{\pgfqpoint{7.174750in}{2.648009in}}%
\pgfpathlineto{\pgfqpoint{7.223139in}{2.647115in}}%
\pgfpathlineto{\pgfqpoint{7.271528in}{2.643690in}}%
\pgfpathlineto{\pgfqpoint{7.319918in}{2.627291in}}%
\pgfpathlineto{\pgfqpoint{7.368307in}{2.543720in}}%
\pgfpathlineto{\pgfqpoint{7.416696in}{2.173820in}}%
\pgfpathlineto{\pgfqpoint{7.465085in}{1.294564in}}%
\pgfpathlineto{\pgfqpoint{7.513474in}{0.761433in}}%
\pgfpathlineto{\pgfqpoint{7.561863in}{0.723676in}}%
\pgfpathlineto{\pgfqpoint{7.610252in}{0.723611in}}%
\pgfpathlineto{\pgfqpoint{7.658641in}{0.723611in}}%
\pgfusepath{stroke}%
\end{pgfscope}%
\begin{pgfscope}%
\pgfpathrectangle{\pgfqpoint{4.660937in}{0.580556in}}{\pgfqpoint{3.140451in}{3.147222in}}%
\pgfusepath{clip}%
\pgfsetrectcap%
\pgfsetroundjoin%
\pgfsetlinewidth{1.505625pt}%
\definecolor{currentstroke}{rgb}{0.839216,0.152941,0.156863}%
\pgfsetstrokecolor{currentstroke}%
\pgfsetdash{}{0pt}%
\pgfpathmoveto{\pgfqpoint{4.803685in}{2.648641in}}%
\pgfpathlineto{\pgfqpoint{4.852074in}{2.648641in}}%
\pgfpathlineto{\pgfqpoint{4.900463in}{2.648641in}}%
\pgfpathlineto{\pgfqpoint{4.948853in}{2.648641in}}%
\pgfpathlineto{\pgfqpoint{4.997242in}{2.648641in}}%
\pgfpathlineto{\pgfqpoint{5.045631in}{2.648641in}}%
\pgfpathlineto{\pgfqpoint{5.094020in}{2.648641in}}%
\pgfpathlineto{\pgfqpoint{5.142409in}{2.648641in}}%
\pgfpathlineto{\pgfqpoint{5.190798in}{2.648641in}}%
\pgfpathlineto{\pgfqpoint{5.239187in}{2.648641in}}%
\pgfpathlineto{\pgfqpoint{5.287576in}{2.648641in}}%
\pgfpathlineto{\pgfqpoint{5.335965in}{2.648641in}}%
\pgfpathlineto{\pgfqpoint{5.384354in}{2.648641in}}%
\pgfpathlineto{\pgfqpoint{5.432743in}{2.648641in}}%
\pgfpathlineto{\pgfqpoint{5.481132in}{2.648641in}}%
\pgfpathlineto{\pgfqpoint{5.529522in}{2.648641in}}%
\pgfpathlineto{\pgfqpoint{5.577911in}{2.648641in}}%
\pgfpathlineto{\pgfqpoint{5.626300in}{2.648641in}}%
\pgfpathlineto{\pgfqpoint{5.674689in}{2.648641in}}%
\pgfpathlineto{\pgfqpoint{5.723078in}{2.648641in}}%
\pgfpathlineto{\pgfqpoint{5.771467in}{2.648641in}}%
\pgfpathlineto{\pgfqpoint{5.819856in}{2.648641in}}%
\pgfpathlineto{\pgfqpoint{5.868245in}{2.648641in}}%
\pgfpathlineto{\pgfqpoint{5.916634in}{2.648641in}}%
\pgfpathlineto{\pgfqpoint{5.965023in}{2.648641in}}%
\pgfpathlineto{\pgfqpoint{6.013412in}{2.648641in}}%
\pgfpathlineto{\pgfqpoint{6.061801in}{2.648641in}}%
\pgfpathlineto{\pgfqpoint{6.110190in}{2.648641in}}%
\pgfpathlineto{\pgfqpoint{6.158580in}{2.648641in}}%
\pgfpathlineto{\pgfqpoint{6.206969in}{2.648641in}}%
\pgfpathlineto{\pgfqpoint{6.255358in}{2.648641in}}%
\pgfpathlineto{\pgfqpoint{6.303747in}{2.648641in}}%
\pgfpathlineto{\pgfqpoint{6.352136in}{2.648641in}}%
\pgfpathlineto{\pgfqpoint{6.400525in}{2.648641in}}%
\pgfpathlineto{\pgfqpoint{6.448914in}{2.648641in}}%
\pgfpathlineto{\pgfqpoint{6.497303in}{2.648641in}}%
\pgfpathlineto{\pgfqpoint{6.545692in}{2.648641in}}%
\pgfpathlineto{\pgfqpoint{6.594081in}{2.648641in}}%
\pgfpathlineto{\pgfqpoint{6.642470in}{2.648641in}}%
\pgfpathlineto{\pgfqpoint{6.690859in}{2.648641in}}%
\pgfpathlineto{\pgfqpoint{6.739249in}{2.648640in}}%
\pgfpathlineto{\pgfqpoint{6.787638in}{2.648639in}}%
\pgfpathlineto{\pgfqpoint{6.836027in}{2.648636in}}%
\pgfpathlineto{\pgfqpoint{6.884416in}{2.648631in}}%
\pgfpathlineto{\pgfqpoint{6.932805in}{2.648613in}}%
\pgfpathlineto{\pgfqpoint{6.981194in}{2.547137in}}%
\pgfpathlineto{\pgfqpoint{7.029583in}{0.723611in}}%
\pgfpathlineto{\pgfqpoint{7.077972in}{0.723611in}}%
\pgfpathlineto{\pgfqpoint{7.126361in}{0.723611in}}%
\pgfpathlineto{\pgfqpoint{7.174750in}{0.723611in}}%
\pgfpathlineto{\pgfqpoint{7.223139in}{0.723611in}}%
\pgfpathlineto{\pgfqpoint{7.271528in}{0.723611in}}%
\pgfpathlineto{\pgfqpoint{7.319918in}{0.723611in}}%
\pgfpathlineto{\pgfqpoint{7.368307in}{0.723611in}}%
\pgfpathlineto{\pgfqpoint{7.416696in}{0.723611in}}%
\pgfpathlineto{\pgfqpoint{7.465085in}{0.723611in}}%
\pgfpathlineto{\pgfqpoint{7.513474in}{0.723611in}}%
\pgfpathlineto{\pgfqpoint{7.561863in}{0.723611in}}%
\pgfpathlineto{\pgfqpoint{7.610252in}{0.723611in}}%
\pgfpathlineto{\pgfqpoint{7.658641in}{0.723611in}}%
\pgfusepath{stroke}%
\end{pgfscope}%
\begin{pgfscope}%
\pgfpathrectangle{\pgfqpoint{4.660937in}{0.580556in}}{\pgfqpoint{3.140451in}{3.147222in}}%
\pgfusepath{clip}%
\pgfsetrectcap%
\pgfsetroundjoin%
\pgfsetlinewidth{0.501875pt}%
\definecolor{currentstroke}{rgb}{0.000000,0.000000,0.000000}%
\pgfsetstrokecolor{currentstroke}%
\pgfsetdash{}{0pt}%
\pgfpathmoveto{\pgfqpoint{4.803685in}{2.648641in}}%
\pgfpathlineto{\pgfqpoint{4.852074in}{2.648641in}}%
\pgfpathlineto{\pgfqpoint{4.900463in}{2.648641in}}%
\pgfpathlineto{\pgfqpoint{4.948853in}{2.648641in}}%
\pgfpathlineto{\pgfqpoint{4.997242in}{2.648641in}}%
\pgfpathlineto{\pgfqpoint{5.045631in}{2.648641in}}%
\pgfpathlineto{\pgfqpoint{5.094020in}{2.648641in}}%
\pgfpathlineto{\pgfqpoint{5.142409in}{2.648641in}}%
\pgfpathlineto{\pgfqpoint{5.190798in}{2.648641in}}%
\pgfpathlineto{\pgfqpoint{5.239187in}{2.648641in}}%
\pgfpathlineto{\pgfqpoint{5.287576in}{2.648641in}}%
\pgfpathlineto{\pgfqpoint{5.335965in}{2.648641in}}%
\pgfpathlineto{\pgfqpoint{5.384354in}{2.648641in}}%
\pgfpathlineto{\pgfqpoint{5.432743in}{2.648641in}}%
\pgfpathlineto{\pgfqpoint{5.481132in}{2.648641in}}%
\pgfpathlineto{\pgfqpoint{5.529522in}{2.648641in}}%
\pgfpathlineto{\pgfqpoint{5.577911in}{2.648641in}}%
\pgfpathlineto{\pgfqpoint{5.626300in}{2.648641in}}%
\pgfpathlineto{\pgfqpoint{5.674689in}{2.648641in}}%
\pgfpathlineto{\pgfqpoint{5.723078in}{2.648641in}}%
\pgfpathlineto{\pgfqpoint{5.771467in}{2.648641in}}%
\pgfpathlineto{\pgfqpoint{5.819856in}{2.648641in}}%
\pgfpathlineto{\pgfqpoint{5.868245in}{2.648641in}}%
\pgfpathlineto{\pgfqpoint{5.916634in}{2.648641in}}%
\pgfpathlineto{\pgfqpoint{5.965023in}{2.648641in}}%
\pgfpathlineto{\pgfqpoint{6.013412in}{2.648641in}}%
\pgfpathlineto{\pgfqpoint{6.061801in}{2.648641in}}%
\pgfpathlineto{\pgfqpoint{6.110190in}{2.648641in}}%
\pgfpathlineto{\pgfqpoint{6.158580in}{2.648641in}}%
\pgfpathlineto{\pgfqpoint{6.206969in}{2.648641in}}%
\pgfpathlineto{\pgfqpoint{6.255358in}{2.648641in}}%
\pgfpathlineto{\pgfqpoint{6.303747in}{2.648641in}}%
\pgfpathlineto{\pgfqpoint{6.352136in}{2.648641in}}%
\pgfpathlineto{\pgfqpoint{6.400525in}{2.648641in}}%
\pgfpathlineto{\pgfqpoint{6.448914in}{2.648641in}}%
\pgfpathlineto{\pgfqpoint{6.497303in}{2.648641in}}%
\pgfpathlineto{\pgfqpoint{6.545692in}{2.648641in}}%
\pgfpathlineto{\pgfqpoint{6.594081in}{2.648641in}}%
\pgfpathlineto{\pgfqpoint{6.642470in}{2.648641in}}%
\pgfpathlineto{\pgfqpoint{6.690859in}{2.648641in}}%
\pgfpathlineto{\pgfqpoint{6.739249in}{2.648641in}}%
\pgfpathlineto{\pgfqpoint{6.787638in}{2.648641in}}%
\pgfpathlineto{\pgfqpoint{6.836027in}{2.648641in}}%
\pgfpathlineto{\pgfqpoint{6.884416in}{2.648641in}}%
\pgfpathlineto{\pgfqpoint{6.932805in}{2.648641in}}%
\pgfpathlineto{\pgfqpoint{6.981194in}{2.648641in}}%
\pgfpathlineto{\pgfqpoint{7.029583in}{2.648641in}}%
\pgfpathlineto{\pgfqpoint{7.077972in}{2.648641in}}%
\pgfpathlineto{\pgfqpoint{7.126361in}{2.648641in}}%
\pgfpathlineto{\pgfqpoint{7.174750in}{2.648641in}}%
\pgfpathlineto{\pgfqpoint{7.223139in}{2.648641in}}%
\pgfpathlineto{\pgfqpoint{7.271528in}{2.648641in}}%
\pgfpathlineto{\pgfqpoint{7.319918in}{2.648641in}}%
\pgfpathlineto{\pgfqpoint{7.368307in}{2.648641in}}%
\pgfpathlineto{\pgfqpoint{7.416696in}{2.648641in}}%
\pgfpathlineto{\pgfqpoint{7.465085in}{0.723611in}}%
\pgfpathlineto{\pgfqpoint{7.513474in}{0.723611in}}%
\pgfpathlineto{\pgfqpoint{7.561863in}{0.723611in}}%
\pgfpathlineto{\pgfqpoint{7.610252in}{0.723611in}}%
\pgfpathlineto{\pgfqpoint{7.658641in}{0.723611in}}%
\pgfusepath{stroke}%
\end{pgfscope}%
\begin{pgfscope}%
\pgfsetrectcap%
\pgfsetmiterjoin%
\pgfsetlinewidth{0.803000pt}%
\definecolor{currentstroke}{rgb}{0.000000,0.000000,0.000000}%
\pgfsetstrokecolor{currentstroke}%
\pgfsetdash{}{0pt}%
\pgfpathmoveto{\pgfqpoint{4.660937in}{0.580556in}}%
\pgfpathlineto{\pgfqpoint{4.660937in}{3.727778in}}%
\pgfusepath{stroke}%
\end{pgfscope}%
\begin{pgfscope}%
\pgfsetrectcap%
\pgfsetmiterjoin%
\pgfsetlinewidth{0.803000pt}%
\definecolor{currentstroke}{rgb}{0.000000,0.000000,0.000000}%
\pgfsetstrokecolor{currentstroke}%
\pgfsetdash{}{0pt}%
\pgfpathmoveto{\pgfqpoint{7.801389in}{0.580556in}}%
\pgfpathlineto{\pgfqpoint{7.801389in}{3.727778in}}%
\pgfusepath{stroke}%
\end{pgfscope}%
\begin{pgfscope}%
\pgfsetrectcap%
\pgfsetmiterjoin%
\pgfsetlinewidth{0.803000pt}%
\definecolor{currentstroke}{rgb}{0.000000,0.000000,0.000000}%
\pgfsetstrokecolor{currentstroke}%
\pgfsetdash{}{0pt}%
\pgfpathmoveto{\pgfqpoint{4.660937in}{0.580556in}}%
\pgfpathlineto{\pgfqpoint{7.801389in}{0.580556in}}%
\pgfusepath{stroke}%
\end{pgfscope}%
\begin{pgfscope}%
\pgfsetrectcap%
\pgfsetmiterjoin%
\pgfsetlinewidth{0.803000pt}%
\definecolor{currentstroke}{rgb}{0.000000,0.000000,0.000000}%
\pgfsetstrokecolor{currentstroke}%
\pgfsetdash{}{0pt}%
\pgfpathmoveto{\pgfqpoint{4.660937in}{3.727778in}}%
\pgfpathlineto{\pgfqpoint{7.801389in}{3.727778in}}%
\pgfusepath{stroke}%
\end{pgfscope}%
\begin{pgfscope}%
\definecolor{textcolor}{rgb}{0.000000,0.000000,0.000000}%
\pgfsetstrokecolor{textcolor}%
\pgfsetfillcolor{textcolor}%
\pgftext[x=6.231163in,y=3.811111in,,base]{\color{textcolor}\sffamily\fontsize{12.000000}{14.400000}\selectfont \(\displaystyle  t = 5.0 \)}%
\end{pgfscope}%
\begin{pgfscope}%
\pgfsetbuttcap%
\pgfsetmiterjoin%
\definecolor{currentfill}{rgb}{1.000000,1.000000,1.000000}%
\pgfsetfillcolor{currentfill}%
\pgfsetlinewidth{0.000000pt}%
\definecolor{currentstroke}{rgb}{0.000000,0.000000,0.000000}%
\pgfsetstrokecolor{currentstroke}%
\pgfsetstrokeopacity{0.000000}%
\pgfsetdash{}{0pt}%
\pgfpathmoveto{\pgfqpoint{0.691667in}{8.480556in}}%
\pgfpathlineto{\pgfqpoint{3.832118in}{8.480556in}}%
\pgfpathlineto{\pgfqpoint{3.832118in}{11.627778in}}%
\pgfpathlineto{\pgfqpoint{0.691667in}{11.627778in}}%
\pgfpathclose%
\pgfusepath{fill}%
\end{pgfscope}%
\begin{pgfscope}%
\pgfsetbuttcap%
\pgfsetroundjoin%
\definecolor{currentfill}{rgb}{0.000000,0.000000,0.000000}%
\pgfsetfillcolor{currentfill}%
\pgfsetlinewidth{0.803000pt}%
\definecolor{currentstroke}{rgb}{0.000000,0.000000,0.000000}%
\pgfsetstrokecolor{currentstroke}%
\pgfsetdash{}{0pt}%
\pgfsys@defobject{currentmarker}{\pgfqpoint{0.000000in}{-0.048611in}}{\pgfqpoint{0.000000in}{0.000000in}}{%
\pgfpathmoveto{\pgfqpoint{0.000000in}{0.000000in}}%
\pgfpathlineto{\pgfqpoint{0.000000in}{-0.048611in}}%
\pgfusepath{stroke,fill}%
}%
\begin{pgfscope}%
\pgfsys@transformshift{0.810220in}{8.480556in}%
\pgfsys@useobject{currentmarker}{}%
\end{pgfscope}%
\end{pgfscope}%
\begin{pgfscope}%
\definecolor{textcolor}{rgb}{0.000000,0.000000,0.000000}%
\pgfsetstrokecolor{textcolor}%
\pgfsetfillcolor{textcolor}%
\pgftext[x=0.810220in,y=8.383333in,,top]{\color{textcolor}\sffamily\fontsize{10.000000}{12.000000}\selectfont −3}%
\end{pgfscope}%
\begin{pgfscope}%
\pgfsetbuttcap%
\pgfsetroundjoin%
\definecolor{currentfill}{rgb}{0.000000,0.000000,0.000000}%
\pgfsetfillcolor{currentfill}%
\pgfsetlinewidth{0.803000pt}%
\definecolor{currentstroke}{rgb}{0.000000,0.000000,0.000000}%
\pgfsetstrokecolor{currentstroke}%
\pgfsetdash{}{0pt}%
\pgfsys@defobject{currentmarker}{\pgfqpoint{0.000000in}{-0.048611in}}{\pgfqpoint{0.000000in}{0.000000in}}{%
\pgfpathmoveto{\pgfqpoint{0.000000in}{0.000000in}}%
\pgfpathlineto{\pgfqpoint{0.000000in}{-0.048611in}}%
\pgfusepath{stroke,fill}%
}%
\begin{pgfscope}%
\pgfsys@transformshift{1.294111in}{8.480556in}%
\pgfsys@useobject{currentmarker}{}%
\end{pgfscope}%
\end{pgfscope}%
\begin{pgfscope}%
\definecolor{textcolor}{rgb}{0.000000,0.000000,0.000000}%
\pgfsetstrokecolor{textcolor}%
\pgfsetfillcolor{textcolor}%
\pgftext[x=1.294111in,y=8.383333in,,top]{\color{textcolor}\sffamily\fontsize{10.000000}{12.000000}\selectfont −2}%
\end{pgfscope}%
\begin{pgfscope}%
\pgfsetbuttcap%
\pgfsetroundjoin%
\definecolor{currentfill}{rgb}{0.000000,0.000000,0.000000}%
\pgfsetfillcolor{currentfill}%
\pgfsetlinewidth{0.803000pt}%
\definecolor{currentstroke}{rgb}{0.000000,0.000000,0.000000}%
\pgfsetstrokecolor{currentstroke}%
\pgfsetdash{}{0pt}%
\pgfsys@defobject{currentmarker}{\pgfqpoint{0.000000in}{-0.048611in}}{\pgfqpoint{0.000000in}{0.000000in}}{%
\pgfpathmoveto{\pgfqpoint{0.000000in}{0.000000in}}%
\pgfpathlineto{\pgfqpoint{0.000000in}{-0.048611in}}%
\pgfusepath{stroke,fill}%
}%
\begin{pgfscope}%
\pgfsys@transformshift{1.778002in}{8.480556in}%
\pgfsys@useobject{currentmarker}{}%
\end{pgfscope}%
\end{pgfscope}%
\begin{pgfscope}%
\definecolor{textcolor}{rgb}{0.000000,0.000000,0.000000}%
\pgfsetstrokecolor{textcolor}%
\pgfsetfillcolor{textcolor}%
\pgftext[x=1.778002in,y=8.383333in,,top]{\color{textcolor}\sffamily\fontsize{10.000000}{12.000000}\selectfont −1}%
\end{pgfscope}%
\begin{pgfscope}%
\pgfsetbuttcap%
\pgfsetroundjoin%
\definecolor{currentfill}{rgb}{0.000000,0.000000,0.000000}%
\pgfsetfillcolor{currentfill}%
\pgfsetlinewidth{0.803000pt}%
\definecolor{currentstroke}{rgb}{0.000000,0.000000,0.000000}%
\pgfsetstrokecolor{currentstroke}%
\pgfsetdash{}{0pt}%
\pgfsys@defobject{currentmarker}{\pgfqpoint{0.000000in}{-0.048611in}}{\pgfqpoint{0.000000in}{0.000000in}}{%
\pgfpathmoveto{\pgfqpoint{0.000000in}{0.000000in}}%
\pgfpathlineto{\pgfqpoint{0.000000in}{-0.048611in}}%
\pgfusepath{stroke,fill}%
}%
\begin{pgfscope}%
\pgfsys@transformshift{2.261892in}{8.480556in}%
\pgfsys@useobject{currentmarker}{}%
\end{pgfscope}%
\end{pgfscope}%
\begin{pgfscope}%
\definecolor{textcolor}{rgb}{0.000000,0.000000,0.000000}%
\pgfsetstrokecolor{textcolor}%
\pgfsetfillcolor{textcolor}%
\pgftext[x=2.261892in,y=8.383333in,,top]{\color{textcolor}\sffamily\fontsize{10.000000}{12.000000}\selectfont 0}%
\end{pgfscope}%
\begin{pgfscope}%
\pgfsetbuttcap%
\pgfsetroundjoin%
\definecolor{currentfill}{rgb}{0.000000,0.000000,0.000000}%
\pgfsetfillcolor{currentfill}%
\pgfsetlinewidth{0.803000pt}%
\definecolor{currentstroke}{rgb}{0.000000,0.000000,0.000000}%
\pgfsetstrokecolor{currentstroke}%
\pgfsetdash{}{0pt}%
\pgfsys@defobject{currentmarker}{\pgfqpoint{0.000000in}{-0.048611in}}{\pgfqpoint{0.000000in}{0.000000in}}{%
\pgfpathmoveto{\pgfqpoint{0.000000in}{0.000000in}}%
\pgfpathlineto{\pgfqpoint{0.000000in}{-0.048611in}}%
\pgfusepath{stroke,fill}%
}%
\begin{pgfscope}%
\pgfsys@transformshift{2.745783in}{8.480556in}%
\pgfsys@useobject{currentmarker}{}%
\end{pgfscope}%
\end{pgfscope}%
\begin{pgfscope}%
\definecolor{textcolor}{rgb}{0.000000,0.000000,0.000000}%
\pgfsetstrokecolor{textcolor}%
\pgfsetfillcolor{textcolor}%
\pgftext[x=2.745783in,y=8.383333in,,top]{\color{textcolor}\sffamily\fontsize{10.000000}{12.000000}\selectfont 1}%
\end{pgfscope}%
\begin{pgfscope}%
\pgfsetbuttcap%
\pgfsetroundjoin%
\definecolor{currentfill}{rgb}{0.000000,0.000000,0.000000}%
\pgfsetfillcolor{currentfill}%
\pgfsetlinewidth{0.803000pt}%
\definecolor{currentstroke}{rgb}{0.000000,0.000000,0.000000}%
\pgfsetstrokecolor{currentstroke}%
\pgfsetdash{}{0pt}%
\pgfsys@defobject{currentmarker}{\pgfqpoint{0.000000in}{-0.048611in}}{\pgfqpoint{0.000000in}{0.000000in}}{%
\pgfpathmoveto{\pgfqpoint{0.000000in}{0.000000in}}%
\pgfpathlineto{\pgfqpoint{0.000000in}{-0.048611in}}%
\pgfusepath{stroke,fill}%
}%
\begin{pgfscope}%
\pgfsys@transformshift{3.229674in}{8.480556in}%
\pgfsys@useobject{currentmarker}{}%
\end{pgfscope}%
\end{pgfscope}%
\begin{pgfscope}%
\definecolor{textcolor}{rgb}{0.000000,0.000000,0.000000}%
\pgfsetstrokecolor{textcolor}%
\pgfsetfillcolor{textcolor}%
\pgftext[x=3.229674in,y=8.383333in,,top]{\color{textcolor}\sffamily\fontsize{10.000000}{12.000000}\selectfont 2}%
\end{pgfscope}%
\begin{pgfscope}%
\pgfsetbuttcap%
\pgfsetroundjoin%
\definecolor{currentfill}{rgb}{0.000000,0.000000,0.000000}%
\pgfsetfillcolor{currentfill}%
\pgfsetlinewidth{0.803000pt}%
\definecolor{currentstroke}{rgb}{0.000000,0.000000,0.000000}%
\pgfsetstrokecolor{currentstroke}%
\pgfsetdash{}{0pt}%
\pgfsys@defobject{currentmarker}{\pgfqpoint{0.000000in}{-0.048611in}}{\pgfqpoint{0.000000in}{0.000000in}}{%
\pgfpathmoveto{\pgfqpoint{0.000000in}{0.000000in}}%
\pgfpathlineto{\pgfqpoint{0.000000in}{-0.048611in}}%
\pgfusepath{stroke,fill}%
}%
\begin{pgfscope}%
\pgfsys@transformshift{3.713565in}{8.480556in}%
\pgfsys@useobject{currentmarker}{}%
\end{pgfscope}%
\end{pgfscope}%
\begin{pgfscope}%
\definecolor{textcolor}{rgb}{0.000000,0.000000,0.000000}%
\pgfsetstrokecolor{textcolor}%
\pgfsetfillcolor{textcolor}%
\pgftext[x=3.713565in,y=8.383333in,,top]{\color{textcolor}\sffamily\fontsize{10.000000}{12.000000}\selectfont 3}%
\end{pgfscope}%
\begin{pgfscope}%
\definecolor{textcolor}{rgb}{0.000000,0.000000,0.000000}%
\pgfsetstrokecolor{textcolor}%
\pgfsetfillcolor{textcolor}%
\pgftext[x=2.261892in,y=8.193365in,,top]{\color{textcolor}\sffamily\fontsize{10.000000}{12.000000}\selectfont \(\displaystyle x\)}%
\end{pgfscope}%
\begin{pgfscope}%
\pgfsetbuttcap%
\pgfsetroundjoin%
\definecolor{currentfill}{rgb}{0.000000,0.000000,0.000000}%
\pgfsetfillcolor{currentfill}%
\pgfsetlinewidth{0.803000pt}%
\definecolor{currentstroke}{rgb}{0.000000,0.000000,0.000000}%
\pgfsetstrokecolor{currentstroke}%
\pgfsetdash{}{0pt}%
\pgfsys@defobject{currentmarker}{\pgfqpoint{-0.048611in}{0.000000in}}{\pgfqpoint{0.000000in}{0.000000in}}{%
\pgfpathmoveto{\pgfqpoint{0.000000in}{0.000000in}}%
\pgfpathlineto{\pgfqpoint{-0.048611in}{0.000000in}}%
\pgfusepath{stroke,fill}%
}%
\begin{pgfscope}%
\pgfsys@transformshift{0.691667in}{8.623611in}%
\pgfsys@useobject{currentmarker}{}%
\end{pgfscope}%
\end{pgfscope}%
\begin{pgfscope}%
\definecolor{textcolor}{rgb}{0.000000,0.000000,0.000000}%
\pgfsetstrokecolor{textcolor}%
\pgfsetfillcolor{textcolor}%
\pgftext[x=0.373565in,y=8.570850in,left,base]{\color{textcolor}\sffamily\fontsize{10.000000}{12.000000}\selectfont 0.0}%
\end{pgfscope}%
\begin{pgfscope}%
\pgfsetbuttcap%
\pgfsetroundjoin%
\definecolor{currentfill}{rgb}{0.000000,0.000000,0.000000}%
\pgfsetfillcolor{currentfill}%
\pgfsetlinewidth{0.803000pt}%
\definecolor{currentstroke}{rgb}{0.000000,0.000000,0.000000}%
\pgfsetstrokecolor{currentstroke}%
\pgfsetdash{}{0pt}%
\pgfsys@defobject{currentmarker}{\pgfqpoint{-0.048611in}{0.000000in}}{\pgfqpoint{0.000000in}{0.000000in}}{%
\pgfpathmoveto{\pgfqpoint{0.000000in}{0.000000in}}%
\pgfpathlineto{\pgfqpoint{-0.048611in}{0.000000in}}%
\pgfusepath{stroke,fill}%
}%
\begin{pgfscope}%
\pgfsys@transformshift{0.691667in}{9.195833in}%
\pgfsys@useobject{currentmarker}{}%
\end{pgfscope}%
\end{pgfscope}%
\begin{pgfscope}%
\definecolor{textcolor}{rgb}{0.000000,0.000000,0.000000}%
\pgfsetstrokecolor{textcolor}%
\pgfsetfillcolor{textcolor}%
\pgftext[x=0.373565in,y=9.143072in,left,base]{\color{textcolor}\sffamily\fontsize{10.000000}{12.000000}\selectfont 0.2}%
\end{pgfscope}%
\begin{pgfscope}%
\pgfsetbuttcap%
\pgfsetroundjoin%
\definecolor{currentfill}{rgb}{0.000000,0.000000,0.000000}%
\pgfsetfillcolor{currentfill}%
\pgfsetlinewidth{0.803000pt}%
\definecolor{currentstroke}{rgb}{0.000000,0.000000,0.000000}%
\pgfsetstrokecolor{currentstroke}%
\pgfsetdash{}{0pt}%
\pgfsys@defobject{currentmarker}{\pgfqpoint{-0.048611in}{0.000000in}}{\pgfqpoint{0.000000in}{0.000000in}}{%
\pgfpathmoveto{\pgfqpoint{0.000000in}{0.000000in}}%
\pgfpathlineto{\pgfqpoint{-0.048611in}{0.000000in}}%
\pgfusepath{stroke,fill}%
}%
\begin{pgfscope}%
\pgfsys@transformshift{0.691667in}{9.768056in}%
\pgfsys@useobject{currentmarker}{}%
\end{pgfscope}%
\end{pgfscope}%
\begin{pgfscope}%
\definecolor{textcolor}{rgb}{0.000000,0.000000,0.000000}%
\pgfsetstrokecolor{textcolor}%
\pgfsetfillcolor{textcolor}%
\pgftext[x=0.373565in,y=9.715294in,left,base]{\color{textcolor}\sffamily\fontsize{10.000000}{12.000000}\selectfont 0.4}%
\end{pgfscope}%
\begin{pgfscope}%
\pgfsetbuttcap%
\pgfsetroundjoin%
\definecolor{currentfill}{rgb}{0.000000,0.000000,0.000000}%
\pgfsetfillcolor{currentfill}%
\pgfsetlinewidth{0.803000pt}%
\definecolor{currentstroke}{rgb}{0.000000,0.000000,0.000000}%
\pgfsetstrokecolor{currentstroke}%
\pgfsetdash{}{0pt}%
\pgfsys@defobject{currentmarker}{\pgfqpoint{-0.048611in}{0.000000in}}{\pgfqpoint{0.000000in}{0.000000in}}{%
\pgfpathmoveto{\pgfqpoint{0.000000in}{0.000000in}}%
\pgfpathlineto{\pgfqpoint{-0.048611in}{0.000000in}}%
\pgfusepath{stroke,fill}%
}%
\begin{pgfscope}%
\pgfsys@transformshift{0.691667in}{10.340278in}%
\pgfsys@useobject{currentmarker}{}%
\end{pgfscope}%
\end{pgfscope}%
\begin{pgfscope}%
\definecolor{textcolor}{rgb}{0.000000,0.000000,0.000000}%
\pgfsetstrokecolor{textcolor}%
\pgfsetfillcolor{textcolor}%
\pgftext[x=0.373565in,y=10.287516in,left,base]{\color{textcolor}\sffamily\fontsize{10.000000}{12.000000}\selectfont 0.6}%
\end{pgfscope}%
\begin{pgfscope}%
\pgfsetbuttcap%
\pgfsetroundjoin%
\definecolor{currentfill}{rgb}{0.000000,0.000000,0.000000}%
\pgfsetfillcolor{currentfill}%
\pgfsetlinewidth{0.803000pt}%
\definecolor{currentstroke}{rgb}{0.000000,0.000000,0.000000}%
\pgfsetstrokecolor{currentstroke}%
\pgfsetdash{}{0pt}%
\pgfsys@defobject{currentmarker}{\pgfqpoint{-0.048611in}{0.000000in}}{\pgfqpoint{0.000000in}{0.000000in}}{%
\pgfpathmoveto{\pgfqpoint{0.000000in}{0.000000in}}%
\pgfpathlineto{\pgfqpoint{-0.048611in}{0.000000in}}%
\pgfusepath{stroke,fill}%
}%
\begin{pgfscope}%
\pgfsys@transformshift{0.691667in}{10.912500in}%
\pgfsys@useobject{currentmarker}{}%
\end{pgfscope}%
\end{pgfscope}%
\begin{pgfscope}%
\definecolor{textcolor}{rgb}{0.000000,0.000000,0.000000}%
\pgfsetstrokecolor{textcolor}%
\pgfsetfillcolor{textcolor}%
\pgftext[x=0.373565in,y=10.859738in,left,base]{\color{textcolor}\sffamily\fontsize{10.000000}{12.000000}\selectfont 0.8}%
\end{pgfscope}%
\begin{pgfscope}%
\pgfsetbuttcap%
\pgfsetroundjoin%
\definecolor{currentfill}{rgb}{0.000000,0.000000,0.000000}%
\pgfsetfillcolor{currentfill}%
\pgfsetlinewidth{0.803000pt}%
\definecolor{currentstroke}{rgb}{0.000000,0.000000,0.000000}%
\pgfsetstrokecolor{currentstroke}%
\pgfsetdash{}{0pt}%
\pgfsys@defobject{currentmarker}{\pgfqpoint{-0.048611in}{0.000000in}}{\pgfqpoint{0.000000in}{0.000000in}}{%
\pgfpathmoveto{\pgfqpoint{0.000000in}{0.000000in}}%
\pgfpathlineto{\pgfqpoint{-0.048611in}{0.000000in}}%
\pgfusepath{stroke,fill}%
}%
\begin{pgfscope}%
\pgfsys@transformshift{0.691667in}{11.484722in}%
\pgfsys@useobject{currentmarker}{}%
\end{pgfscope}%
\end{pgfscope}%
\begin{pgfscope}%
\definecolor{textcolor}{rgb}{0.000000,0.000000,0.000000}%
\pgfsetstrokecolor{textcolor}%
\pgfsetfillcolor{textcolor}%
\pgftext[x=0.373565in,y=11.431961in,left,base]{\color{textcolor}\sffamily\fontsize{10.000000}{12.000000}\selectfont 1.0}%
\end{pgfscope}%
\begin{pgfscope}%
\definecolor{textcolor}{rgb}{0.000000,0.000000,0.000000}%
\pgfsetstrokecolor{textcolor}%
\pgfsetfillcolor{textcolor}%
\pgftext[x=0.318009in,y=10.054167in,,bottom,rotate=90.000000]{\color{textcolor}\sffamily\fontsize{10.000000}{12.000000}\selectfont \(\displaystyle U\)}%
\end{pgfscope}%
\begin{pgfscope}%
\pgfpathrectangle{\pgfqpoint{0.691667in}{8.480556in}}{\pgfqpoint{3.140451in}{3.147222in}}%
\pgfusepath{clip}%
\pgfsetrectcap%
\pgfsetroundjoin%
\pgfsetlinewidth{0.501875pt}%
\definecolor{currentstroke}{rgb}{0.000000,0.000000,0.000000}%
\pgfsetstrokecolor{currentstroke}%
\pgfsetdash{}{0pt}%
\pgfpathmoveto{\pgfqpoint{0.834414in}{11.484722in}}%
\pgfpathlineto{\pgfqpoint{0.882804in}{11.484722in}}%
\pgfpathlineto{\pgfqpoint{0.931193in}{11.484722in}}%
\pgfpathlineto{\pgfqpoint{0.979582in}{11.484722in}}%
\pgfpathlineto{\pgfqpoint{1.027971in}{11.484722in}}%
\pgfpathlineto{\pgfqpoint{1.076360in}{11.484722in}}%
\pgfpathlineto{\pgfqpoint{1.124749in}{11.484722in}}%
\pgfpathlineto{\pgfqpoint{1.173138in}{11.484722in}}%
\pgfpathlineto{\pgfqpoint{1.221527in}{11.484722in}}%
\pgfpathlineto{\pgfqpoint{1.269916in}{11.484722in}}%
\pgfpathlineto{\pgfqpoint{1.318305in}{11.484722in}}%
\pgfpathlineto{\pgfqpoint{1.366694in}{11.484722in}}%
\pgfpathlineto{\pgfqpoint{1.415083in}{11.484722in}}%
\pgfpathlineto{\pgfqpoint{1.463473in}{11.484722in}}%
\pgfpathlineto{\pgfqpoint{1.511862in}{11.484413in}}%
\pgfpathlineto{\pgfqpoint{1.560251in}{11.474298in}}%
\pgfpathlineto{\pgfqpoint{1.608640in}{11.449993in}}%
\pgfpathlineto{\pgfqpoint{1.657029in}{11.411742in}}%
\pgfpathlineto{\pgfqpoint{1.705418in}{11.359926in}}%
\pgfpathlineto{\pgfqpoint{1.753807in}{11.295064in}}%
\pgfpathlineto{\pgfqpoint{1.802196in}{11.217803in}}%
\pgfpathlineto{\pgfqpoint{1.850585in}{11.128915in}}%
\pgfpathlineto{\pgfqpoint{1.898974in}{11.029289in}}%
\pgfpathlineto{\pgfqpoint{1.947363in}{10.919919in}}%
\pgfpathlineto{\pgfqpoint{1.995752in}{10.801900in}}%
\pgfpathlineto{\pgfqpoint{2.044141in}{10.676409in}}%
\pgfpathlineto{\pgfqpoint{2.092531in}{10.544701in}}%
\pgfpathlineto{\pgfqpoint{2.140920in}{10.408092in}}%
\pgfpathlineto{\pgfqpoint{2.189309in}{10.267946in}}%
\pgfpathlineto{\pgfqpoint{2.237698in}{10.125665in}}%
\pgfpathlineto{\pgfqpoint{2.286087in}{9.982669in}}%
\pgfpathlineto{\pgfqpoint{2.334476in}{9.840387in}}%
\pgfpathlineto{\pgfqpoint{2.382865in}{9.700242in}}%
\pgfpathlineto{\pgfqpoint{2.431254in}{9.563632in}}%
\pgfpathlineto{\pgfqpoint{2.479643in}{9.431924in}}%
\pgfpathlineto{\pgfqpoint{2.528032in}{9.306434in}}%
\pgfpathlineto{\pgfqpoint{2.576421in}{9.188414in}}%
\pgfpathlineto{\pgfqpoint{2.624810in}{9.079045in}}%
\pgfpathlineto{\pgfqpoint{2.673200in}{8.979418in}}%
\pgfpathlineto{\pgfqpoint{2.721589in}{8.890531in}}%
\pgfpathlineto{\pgfqpoint{2.769978in}{8.813270in}}%
\pgfpathlineto{\pgfqpoint{2.818367in}{8.748407in}}%
\pgfpathlineto{\pgfqpoint{2.866756in}{8.696591in}}%
\pgfpathlineto{\pgfqpoint{2.915145in}{8.658340in}}%
\pgfpathlineto{\pgfqpoint{2.963534in}{8.634036in}}%
\pgfpathlineto{\pgfqpoint{3.011923in}{8.623920in}}%
\pgfpathlineto{\pgfqpoint{3.060312in}{8.623611in}}%
\pgfpathlineto{\pgfqpoint{3.108701in}{8.623611in}}%
\pgfpathlineto{\pgfqpoint{3.157090in}{8.623611in}}%
\pgfpathlineto{\pgfqpoint{3.205479in}{8.623611in}}%
\pgfpathlineto{\pgfqpoint{3.253869in}{8.623611in}}%
\pgfpathlineto{\pgfqpoint{3.302258in}{8.623611in}}%
\pgfpathlineto{\pgfqpoint{3.350647in}{8.623611in}}%
\pgfpathlineto{\pgfqpoint{3.399036in}{8.623611in}}%
\pgfpathlineto{\pgfqpoint{3.447425in}{8.623611in}}%
\pgfpathlineto{\pgfqpoint{3.495814in}{8.623611in}}%
\pgfpathlineto{\pgfqpoint{3.544203in}{8.623611in}}%
\pgfpathlineto{\pgfqpoint{3.592592in}{8.623611in}}%
\pgfpathlineto{\pgfqpoint{3.640981in}{8.623611in}}%
\pgfpathlineto{\pgfqpoint{3.689370in}{8.623611in}}%
\pgfusepath{stroke}%
\end{pgfscope}%
\begin{pgfscope}%
\pgfsetrectcap%
\pgfsetmiterjoin%
\pgfsetlinewidth{0.803000pt}%
\definecolor{currentstroke}{rgb}{0.000000,0.000000,0.000000}%
\pgfsetstrokecolor{currentstroke}%
\pgfsetdash{}{0pt}%
\pgfpathmoveto{\pgfqpoint{0.691667in}{8.480556in}}%
\pgfpathlineto{\pgfqpoint{0.691667in}{11.627778in}}%
\pgfusepath{stroke}%
\end{pgfscope}%
\begin{pgfscope}%
\pgfsetrectcap%
\pgfsetmiterjoin%
\pgfsetlinewidth{0.803000pt}%
\definecolor{currentstroke}{rgb}{0.000000,0.000000,0.000000}%
\pgfsetstrokecolor{currentstroke}%
\pgfsetdash{}{0pt}%
\pgfpathmoveto{\pgfqpoint{3.832118in}{8.480556in}}%
\pgfpathlineto{\pgfqpoint{3.832118in}{11.627778in}}%
\pgfusepath{stroke}%
\end{pgfscope}%
\begin{pgfscope}%
\pgfsetrectcap%
\pgfsetmiterjoin%
\pgfsetlinewidth{0.803000pt}%
\definecolor{currentstroke}{rgb}{0.000000,0.000000,0.000000}%
\pgfsetstrokecolor{currentstroke}%
\pgfsetdash{}{0pt}%
\pgfpathmoveto{\pgfqpoint{0.691667in}{8.480556in}}%
\pgfpathlineto{\pgfqpoint{3.832118in}{8.480556in}}%
\pgfusepath{stroke}%
\end{pgfscope}%
\begin{pgfscope}%
\pgfsetrectcap%
\pgfsetmiterjoin%
\pgfsetlinewidth{0.803000pt}%
\definecolor{currentstroke}{rgb}{0.000000,0.000000,0.000000}%
\pgfsetstrokecolor{currentstroke}%
\pgfsetdash{}{0pt}%
\pgfpathmoveto{\pgfqpoint{0.691667in}{11.627778in}}%
\pgfpathlineto{\pgfqpoint{3.832118in}{11.627778in}}%
\pgfusepath{stroke}%
\end{pgfscope}%
\begin{pgfscope}%
\definecolor{textcolor}{rgb}{0.000000,0.000000,0.000000}%
\pgfsetstrokecolor{textcolor}%
\pgfsetfillcolor{textcolor}%
\pgftext[x=2.261892in,y=11.711111in,,base]{\color{textcolor}\sffamily\fontsize{12.000000}{14.400000}\selectfont \(\displaystyle  t = 0.0 \)}%
\end{pgfscope}%
\begin{pgfscope}%
\pgfsetbuttcap%
\pgfsetmiterjoin%
\definecolor{currentfill}{rgb}{1.000000,1.000000,1.000000}%
\pgfsetfillcolor{currentfill}%
\pgfsetfillopacity{0.800000}%
\pgfsetlinewidth{1.003750pt}%
\definecolor{currentstroke}{rgb}{0.800000,0.800000,0.800000}%
\pgfsetstrokecolor{currentstroke}%
\pgfsetstrokeopacity{0.800000}%
\pgfsetdash{}{0pt}%
\pgfpathmoveto{\pgfqpoint{1.522466in}{9.523690in}}%
\pgfpathlineto{\pgfqpoint{3.734896in}{9.523690in}}%
\pgfpathquadraticcurveto{\pgfqpoint{3.762674in}{9.523690in}}{\pgfqpoint{3.762674in}{9.551468in}}%
\pgfpathlineto{\pgfqpoint{3.762674in}{10.556865in}}%
\pgfpathquadraticcurveto{\pgfqpoint{3.762674in}{10.584643in}}{\pgfqpoint{3.734896in}{10.584643in}}%
\pgfpathlineto{\pgfqpoint{1.522466in}{10.584643in}}%
\pgfpathquadraticcurveto{\pgfqpoint{1.494689in}{10.584643in}}{\pgfqpoint{1.494689in}{10.556865in}}%
\pgfpathlineto{\pgfqpoint{1.494689in}{9.551468in}}%
\pgfpathquadraticcurveto{\pgfqpoint{1.494689in}{9.523690in}}{\pgfqpoint{1.522466in}{9.523690in}}%
\pgfpathclose%
\pgfusepath{stroke,fill}%
\end{pgfscope}%
\begin{pgfscope}%
\pgfsetrectcap%
\pgfsetroundjoin%
\pgfsetlinewidth{1.505625pt}%
\definecolor{currentstroke}{rgb}{0.121569,0.466667,0.705882}%
\pgfsetstrokecolor{currentstroke}%
\pgfsetdash{}{0pt}%
\pgfpathmoveto{\pgfqpoint{1.550244in}{10.472176in}}%
\pgfpathlineto{\pgfqpoint{1.828022in}{10.472176in}}%
\pgfusepath{stroke}%
\end{pgfscope}%
\begin{pgfscope}%
\definecolor{textcolor}{rgb}{0.000000,0.000000,0.000000}%
\pgfsetstrokecolor{textcolor}%
\pgfsetfillcolor{textcolor}%
\pgftext[x=1.939133in,y=10.423564in,left,base]{\color{textcolor}\sffamily\fontsize{10.000000}{12.000000}\selectfont Richtmyer}%
\end{pgfscope}%
\begin{pgfscope}%
\pgfsetrectcap%
\pgfsetroundjoin%
\pgfsetlinewidth{1.505625pt}%
\definecolor{currentstroke}{rgb}{1.000000,0.498039,0.054902}%
\pgfsetstrokecolor{currentstroke}%
\pgfsetdash{}{0pt}%
\pgfpathmoveto{\pgfqpoint{1.550244in}{10.268318in}}%
\pgfpathlineto{\pgfqpoint{1.828022in}{10.268318in}}%
\pgfusepath{stroke}%
\end{pgfscope}%
\begin{pgfscope}%
\definecolor{textcolor}{rgb}{0.000000,0.000000,0.000000}%
\pgfsetstrokecolor{textcolor}%
\pgfsetfillcolor{textcolor}%
\pgftext[x=1.939133in,y=10.219707in,left,base]{\color{textcolor}\sffamily\fontsize{10.000000}{12.000000}\selectfont Lax--Wendroff}%
\end{pgfscope}%
\begin{pgfscope}%
\pgfsetrectcap%
\pgfsetroundjoin%
\pgfsetlinewidth{1.505625pt}%
\definecolor{currentstroke}{rgb}{0.172549,0.627451,0.172549}%
\pgfsetstrokecolor{currentstroke}%
\pgfsetdash{}{0pt}%
\pgfpathmoveto{\pgfqpoint{1.550244in}{10.064461in}}%
\pgfpathlineto{\pgfqpoint{1.828022in}{10.064461in}}%
\pgfusepath{stroke}%
\end{pgfscope}%
\begin{pgfscope}%
\definecolor{textcolor}{rgb}{0.000000,0.000000,0.000000}%
\pgfsetstrokecolor{textcolor}%
\pgfsetfillcolor{textcolor}%
\pgftext[x=1.939133in,y=10.015850in,left,base]{\color{textcolor}\sffamily\fontsize{10.000000}{12.000000}\selectfont Conservative upwind}%
\end{pgfscope}%
\begin{pgfscope}%
\pgfsetrectcap%
\pgfsetroundjoin%
\pgfsetlinewidth{1.505625pt}%
\definecolor{currentstroke}{rgb}{0.839216,0.152941,0.156863}%
\pgfsetstrokecolor{currentstroke}%
\pgfsetdash{}{0pt}%
\pgfpathmoveto{\pgfqpoint{1.550244in}{9.860604in}}%
\pgfpathlineto{\pgfqpoint{1.828022in}{9.860604in}}%
\pgfusepath{stroke}%
\end{pgfscope}%
\begin{pgfscope}%
\definecolor{textcolor}{rgb}{0.000000,0.000000,0.000000}%
\pgfsetstrokecolor{textcolor}%
\pgfsetfillcolor{textcolor}%
\pgftext[x=1.939133in,y=9.811993in,left,base]{\color{textcolor}\sffamily\fontsize{10.000000}{12.000000}\selectfont Non-conservative upwind}%
\end{pgfscope}%
\begin{pgfscope}%
\pgfsetrectcap%
\pgfsetroundjoin%
\pgfsetlinewidth{0.501875pt}%
\definecolor{currentstroke}{rgb}{0.000000,0.000000,0.000000}%
\pgfsetstrokecolor{currentstroke}%
\pgfsetdash{}{0pt}%
\pgfpathmoveto{\pgfqpoint{1.550244in}{9.656747in}}%
\pgfpathlineto{\pgfqpoint{1.828022in}{9.656747in}}%
\pgfusepath{stroke}%
\end{pgfscope}%
\begin{pgfscope}%
\definecolor{textcolor}{rgb}{0.000000,0.000000,0.000000}%
\pgfsetstrokecolor{textcolor}%
\pgfsetfillcolor{textcolor}%
\pgftext[x=1.939133in,y=9.608136in,left,base]{\color{textcolor}\sffamily\fontsize{10.000000}{12.000000}\selectfont Analytical}%
\end{pgfscope}%
\end{pgfpicture}%
\makeatother%
\endgroup%
}
\caption{Heat map of solutions using triangle element for the second equation}
\label{Fig:Rect2}
}
{
\footnotesize Top left: $u_h$ with $ N = 256 $; top right: $ u_h - u $ at nodes with $ N = 256 $; bottom left: $ u_h - u $ at $\Omega$ with $ N = 256 $; bottom right: $ u_h - u $ at $\Omega$ with $ N = 16 $.
}
\end{figure}

We also graph the error curve with respect to different $N$ in Figure \ref{Fig:Err2}. Here $L^2$ error and $H^1$ (semi-norm) error are calculated by Simpson's quadrature. Numerical results are summarized in Table \ref{Tbl:SumTri2} and \ref{Tbl:SumRect2}.

\begin{figure}[htbp]
\centering
%% Creator: Matplotlib, PGF backend
%%
%% To include the figure in your LaTeX document, write
%%   \input{<filename>.pgf}
%%
%% Make sure the required packages are loaded in your preamble
%%   \usepackage{pgf}
%%
%% Figures using additional raster images can only be included by \input if
%% they are in the same directory as the main LaTeX file. For loading figures
%% from other directories you can use the `import` package
%%   \usepackage{import}
%% and then include the figures with
%%   \import{<path to file>}{<filename>.pgf}
%%
%% Matplotlib used the following preamble
%%   \usepackage{fontspec}
%%   \setmainfont{DejaVuSerif.ttf}[Path=/home/lzh/anaconda3/envs/numpde/lib/python3.7/site-packages/matplotlib/mpl-data/fonts/ttf/]
%%   \setsansfont{DejaVuSans.ttf}[Path=/home/lzh/anaconda3/envs/numpde/lib/python3.7/site-packages/matplotlib/mpl-data/fonts/ttf/]
%%   \setmonofont{DejaVuSansMono.ttf}[Path=/home/lzh/anaconda3/envs/numpde/lib/python3.7/site-packages/matplotlib/mpl-data/fonts/ttf/]
%%
\begingroup%
\makeatletter%
\begin{pgfpicture}%
\pgfpathrectangle{\pgfpointorigin}{\pgfqpoint{6.000000in}{4.000000in}}%
\pgfusepath{use as bounding box, clip}%
\begin{pgfscope}%
\pgfsetbuttcap%
\pgfsetmiterjoin%
\definecolor{currentfill}{rgb}{1.000000,1.000000,1.000000}%
\pgfsetfillcolor{currentfill}%
\pgfsetlinewidth{0.000000pt}%
\definecolor{currentstroke}{rgb}{1.000000,1.000000,1.000000}%
\pgfsetstrokecolor{currentstroke}%
\pgfsetdash{}{0pt}%
\pgfpathmoveto{\pgfqpoint{0.000000in}{0.000000in}}%
\pgfpathlineto{\pgfqpoint{6.000000in}{0.000000in}}%
\pgfpathlineto{\pgfqpoint{6.000000in}{4.000000in}}%
\pgfpathlineto{\pgfqpoint{0.000000in}{4.000000in}}%
\pgfpathclose%
\pgfusepath{fill}%
\end{pgfscope}%
\begin{pgfscope}%
\pgfsetbuttcap%
\pgfsetmiterjoin%
\definecolor{currentfill}{rgb}{1.000000,1.000000,1.000000}%
\pgfsetfillcolor{currentfill}%
\pgfsetlinewidth{0.000000pt}%
\definecolor{currentstroke}{rgb}{0.000000,0.000000,0.000000}%
\pgfsetstrokecolor{currentstroke}%
\pgfsetstrokeopacity{0.000000}%
\pgfsetdash{}{0pt}%
\pgfpathmoveto{\pgfqpoint{0.750000in}{0.440000in}}%
\pgfpathlineto{\pgfqpoint{5.400000in}{0.440000in}}%
\pgfpathlineto{\pgfqpoint{5.400000in}{3.520000in}}%
\pgfpathlineto{\pgfqpoint{0.750000in}{3.520000in}}%
\pgfpathclose%
\pgfusepath{fill}%
\end{pgfscope}%
\begin{pgfscope}%
\pgfpathrectangle{\pgfqpoint{0.750000in}{0.440000in}}{\pgfqpoint{4.650000in}{3.080000in}}%
\pgfusepath{clip}%
\pgfsetbuttcap%
\pgfsetroundjoin%
\definecolor{currentfill}{rgb}{0.121569,0.466667,0.705882}%
\pgfsetfillcolor{currentfill}%
\pgfsetlinewidth{1.003750pt}%
\definecolor{currentstroke}{rgb}{0.121569,0.466667,0.705882}%
\pgfsetstrokecolor{currentstroke}%
\pgfsetdash{}{0pt}%
\pgfsys@defobject{currentmarker}{\pgfqpoint{-0.009821in}{-0.009821in}}{\pgfqpoint{0.009821in}{0.009821in}}{%
\pgfpathmoveto{\pgfqpoint{0.000000in}{-0.009821in}}%
\pgfpathcurveto{\pgfqpoint{0.002605in}{-0.009821in}}{\pgfqpoint{0.005103in}{-0.008786in}}{\pgfqpoint{0.006944in}{-0.006944in}}%
\pgfpathcurveto{\pgfqpoint{0.008786in}{-0.005103in}}{\pgfqpoint{0.009821in}{-0.002605in}}{\pgfqpoint{0.009821in}{0.000000in}}%
\pgfpathcurveto{\pgfqpoint{0.009821in}{0.002605in}}{\pgfqpoint{0.008786in}{0.005103in}}{\pgfqpoint{0.006944in}{0.006944in}}%
\pgfpathcurveto{\pgfqpoint{0.005103in}{0.008786in}}{\pgfqpoint{0.002605in}{0.009821in}}{\pgfqpoint{0.000000in}{0.009821in}}%
\pgfpathcurveto{\pgfqpoint{-0.002605in}{0.009821in}}{\pgfqpoint{-0.005103in}{0.008786in}}{\pgfqpoint{-0.006944in}{0.006944in}}%
\pgfpathcurveto{\pgfqpoint{-0.008786in}{0.005103in}}{\pgfqpoint{-0.009821in}{0.002605in}}{\pgfqpoint{-0.009821in}{0.000000in}}%
\pgfpathcurveto{\pgfqpoint{-0.009821in}{-0.002605in}}{\pgfqpoint{-0.008786in}{-0.005103in}}{\pgfqpoint{-0.006944in}{-0.006944in}}%
\pgfpathcurveto{\pgfqpoint{-0.005103in}{-0.008786in}}{\pgfqpoint{-0.002605in}{-0.009821in}}{\pgfqpoint{0.000000in}{-0.009821in}}%
\pgfpathclose%
\pgfusepath{stroke,fill}%
}%
\begin{pgfscope}%
\pgfsys@transformshift{1.154259in}{3.151308in}%
\pgfsys@useobject{currentmarker}{}%
\end{pgfscope}%
\begin{pgfscope}%
\pgfsys@transformshift{1.390735in}{3.047997in}%
\pgfsys@useobject{currentmarker}{}%
\end{pgfscope}%
\begin{pgfscope}%
\pgfsys@transformshift{1.558517in}{2.973932in}%
\pgfsys@useobject{currentmarker}{}%
\end{pgfscope}%
\begin{pgfscope}%
\pgfsys@transformshift{1.794993in}{2.864542in}%
\pgfsys@useobject{currentmarker}{}%
\end{pgfscope}%
\begin{pgfscope}%
\pgfsys@transformshift{1.962776in}{2.781394in}%
\pgfsys@useobject{currentmarker}{}%
\end{pgfscope}%
\begin{pgfscope}%
\pgfsys@transformshift{2.148505in}{2.688434in}%
\pgfsys@useobject{currentmarker}{}%
\end{pgfscope}%
\begin{pgfscope}%
\pgfsys@transformshift{2.367034in}{2.577449in}%
\pgfsys@useobject{currentmarker}{}%
\end{pgfscope}%
\begin{pgfscope}%
\pgfsys@transformshift{2.578689in}{2.468167in}%
\pgfsys@useobject{currentmarker}{}%
\end{pgfscope}%
\begin{pgfscope}%
\pgfsys@transformshift{2.771293in}{2.368066in}%
\pgfsys@useobject{currentmarker}{}%
\end{pgfscope}%
\begin{pgfscope}%
\pgfsys@transformshift{2.970129in}{2.263995in}%
\pgfsys@useobject{currentmarker}{}%
\end{pgfscope}%
\begin{pgfscope}%
\pgfsys@transformshift{3.175552in}{2.160595in}%
\pgfsys@useobject{currentmarker}{}%
\end{pgfscope}%
\begin{pgfscope}%
\pgfsys@transformshift{3.380832in}{2.061361in}%
\pgfsys@useobject{currentmarker}{}%
\end{pgfscope}%
\begin{pgfscope}%
\pgfsys@transformshift{3.579810in}{1.964666in}%
\pgfsys@useobject{currentmarker}{}%
\end{pgfscope}%
\begin{pgfscope}%
\pgfsys@transformshift{3.781877in}{1.866005in}%
\pgfsys@useobject{currentmarker}{}%
\end{pgfscope}%
\begin{pgfscope}%
\pgfsys@transformshift{3.984069in}{1.766862in}%
\pgfsys@useobject{currentmarker}{}%
\end{pgfscope}%
\begin{pgfscope}%
\pgfsys@transformshift{4.186136in}{1.667395in}%
\pgfsys@useobject{currentmarker}{}%
\end{pgfscope}%
\begin{pgfscope}%
\pgfsys@transformshift{4.388328in}{1.567566in}%
\pgfsys@useobject{currentmarker}{}%
\end{pgfscope}%
\begin{pgfscope}%
\pgfsys@transformshift{4.590395in}{1.467536in}%
\pgfsys@useobject{currentmarker}{}%
\end{pgfscope}%
\begin{pgfscope}%
\pgfsys@transformshift{4.792586in}{1.367215in}%
\pgfsys@useobject{currentmarker}{}%
\end{pgfscope}%
\begin{pgfscope}%
\pgfsys@transformshift{4.994653in}{1.266788in}%
\pgfsys@useobject{currentmarker}{}%
\end{pgfscope}%
\begin{pgfscope}%
\pgfsys@transformshift{5.196845in}{1.163649in}%
\pgfsys@useobject{currentmarker}{}%
\end{pgfscope}%
\end{pgfscope}%
\begin{pgfscope}%
\pgfpathrectangle{\pgfqpoint{0.750000in}{0.440000in}}{\pgfqpoint{4.650000in}{3.080000in}}%
\pgfusepath{clip}%
\pgfsetbuttcap%
\pgfsetroundjoin%
\definecolor{currentfill}{rgb}{1.000000,0.498039,0.054902}%
\pgfsetfillcolor{currentfill}%
\pgfsetlinewidth{1.003750pt}%
\definecolor{currentstroke}{rgb}{1.000000,0.498039,0.054902}%
\pgfsetstrokecolor{currentstroke}%
\pgfsetdash{}{0pt}%
\pgfsys@defobject{currentmarker}{\pgfqpoint{-0.009821in}{-0.009821in}}{\pgfqpoint{0.009821in}{0.009821in}}{%
\pgfpathmoveto{\pgfqpoint{0.000000in}{-0.009821in}}%
\pgfpathcurveto{\pgfqpoint{0.002605in}{-0.009821in}}{\pgfqpoint{0.005103in}{-0.008786in}}{\pgfqpoint{0.006944in}{-0.006944in}}%
\pgfpathcurveto{\pgfqpoint{0.008786in}{-0.005103in}}{\pgfqpoint{0.009821in}{-0.002605in}}{\pgfqpoint{0.009821in}{0.000000in}}%
\pgfpathcurveto{\pgfqpoint{0.009821in}{0.002605in}}{\pgfqpoint{0.008786in}{0.005103in}}{\pgfqpoint{0.006944in}{0.006944in}}%
\pgfpathcurveto{\pgfqpoint{0.005103in}{0.008786in}}{\pgfqpoint{0.002605in}{0.009821in}}{\pgfqpoint{0.000000in}{0.009821in}}%
\pgfpathcurveto{\pgfqpoint{-0.002605in}{0.009821in}}{\pgfqpoint{-0.005103in}{0.008786in}}{\pgfqpoint{-0.006944in}{0.006944in}}%
\pgfpathcurveto{\pgfqpoint{-0.008786in}{0.005103in}}{\pgfqpoint{-0.009821in}{0.002605in}}{\pgfqpoint{-0.009821in}{0.000000in}}%
\pgfpathcurveto{\pgfqpoint{-0.009821in}{-0.002605in}}{\pgfqpoint{-0.008786in}{-0.005103in}}{\pgfqpoint{-0.006944in}{-0.006944in}}%
\pgfpathcurveto{\pgfqpoint{-0.005103in}{-0.008786in}}{\pgfqpoint{-0.002605in}{-0.009821in}}{\pgfqpoint{0.000000in}{-0.009821in}}%
\pgfpathclose%
\pgfusepath{stroke,fill}%
}%
\begin{pgfscope}%
\pgfsys@transformshift{1.154259in}{3.023575in}%
\pgfsys@useobject{currentmarker}{}%
\end{pgfscope}%
\begin{pgfscope}%
\pgfsys@transformshift{1.390735in}{2.904730in}%
\pgfsys@useobject{currentmarker}{}%
\end{pgfscope}%
\begin{pgfscope}%
\pgfsys@transformshift{1.558517in}{2.818082in}%
\pgfsys@useobject{currentmarker}{}%
\end{pgfscope}%
\begin{pgfscope}%
\pgfsys@transformshift{1.794993in}{2.694449in}%
\pgfsys@useobject{currentmarker}{}%
\end{pgfscope}%
\begin{pgfscope}%
\pgfsys@transformshift{1.962776in}{2.606164in}%
\pgfsys@useobject{currentmarker}{}%
\end{pgfscope}%
\begin{pgfscope}%
\pgfsys@transformshift{2.148505in}{2.508146in}%
\pgfsys@useobject{currentmarker}{}%
\end{pgfscope}%
\begin{pgfscope}%
\pgfsys@transformshift{2.367034in}{2.392596in}%
\pgfsys@useobject{currentmarker}{}%
\end{pgfscope}%
\begin{pgfscope}%
\pgfsys@transformshift{2.578689in}{2.280555in}%
\pgfsys@useobject{currentmarker}{}%
\end{pgfscope}%
\begin{pgfscope}%
\pgfsys@transformshift{2.771293in}{2.178538in}%
\pgfsys@useobject{currentmarker}{}%
\end{pgfscope}%
\begin{pgfscope}%
\pgfsys@transformshift{2.970129in}{2.073184in}%
\pgfsys@useobject{currentmarker}{}%
\end{pgfscope}%
\begin{pgfscope}%
\pgfsys@transformshift{3.175552in}{1.964318in}%
\pgfsys@useobject{currentmarker}{}%
\end{pgfscope}%
\begin{pgfscope}%
\pgfsys@transformshift{3.380832in}{1.855512in}%
\pgfsys@useobject{currentmarker}{}%
\end{pgfscope}%
\begin{pgfscope}%
\pgfsys@transformshift{3.579810in}{1.750039in}%
\pgfsys@useobject{currentmarker}{}%
\end{pgfscope}%
\begin{pgfscope}%
\pgfsys@transformshift{3.781877in}{1.642922in}%
\pgfsys@useobject{currentmarker}{}%
\end{pgfscope}%
\begin{pgfscope}%
\pgfsys@transformshift{3.984069in}{1.535736in}%
\pgfsys@useobject{currentmarker}{}%
\end{pgfscope}%
\begin{pgfscope}%
\pgfsys@transformshift{4.186136in}{1.428614in}%
\pgfsys@useobject{currentmarker}{}%
\end{pgfscope}%
\begin{pgfscope}%
\pgfsys@transformshift{4.388328in}{1.321424in}%
\pgfsys@useobject{currentmarker}{}%
\end{pgfscope}%
\begin{pgfscope}%
\pgfsys@transformshift{4.590395in}{1.214299in}%
\pgfsys@useobject{currentmarker}{}%
\end{pgfscope}%
\begin{pgfscope}%
\pgfsys@transformshift{4.792586in}{1.107110in}%
\pgfsys@useobject{currentmarker}{}%
\end{pgfscope}%
\begin{pgfscope}%
\pgfsys@transformshift{4.994653in}{0.999996in}%
\pgfsys@useobject{currentmarker}{}%
\end{pgfscope}%
\begin{pgfscope}%
\pgfsys@transformshift{5.196845in}{0.897979in}%
\pgfsys@useobject{currentmarker}{}%
\end{pgfscope}%
\end{pgfscope}%
\begin{pgfscope}%
\pgfpathrectangle{\pgfqpoint{0.750000in}{0.440000in}}{\pgfqpoint{4.650000in}{3.080000in}}%
\pgfusepath{clip}%
\pgfsetbuttcap%
\pgfsetroundjoin%
\definecolor{currentfill}{rgb}{0.172549,0.627451,0.172549}%
\pgfsetfillcolor{currentfill}%
\pgfsetlinewidth{1.003750pt}%
\definecolor{currentstroke}{rgb}{0.172549,0.627451,0.172549}%
\pgfsetstrokecolor{currentstroke}%
\pgfsetdash{}{0pt}%
\pgfsys@defobject{currentmarker}{\pgfqpoint{-0.009821in}{-0.009821in}}{\pgfqpoint{0.009821in}{0.009821in}}{%
\pgfpathmoveto{\pgfqpoint{0.000000in}{-0.009821in}}%
\pgfpathcurveto{\pgfqpoint{0.002605in}{-0.009821in}}{\pgfqpoint{0.005103in}{-0.008786in}}{\pgfqpoint{0.006944in}{-0.006944in}}%
\pgfpathcurveto{\pgfqpoint{0.008786in}{-0.005103in}}{\pgfqpoint{0.009821in}{-0.002605in}}{\pgfqpoint{0.009821in}{0.000000in}}%
\pgfpathcurveto{\pgfqpoint{0.009821in}{0.002605in}}{\pgfqpoint{0.008786in}{0.005103in}}{\pgfqpoint{0.006944in}{0.006944in}}%
\pgfpathcurveto{\pgfqpoint{0.005103in}{0.008786in}}{\pgfqpoint{0.002605in}{0.009821in}}{\pgfqpoint{0.000000in}{0.009821in}}%
\pgfpathcurveto{\pgfqpoint{-0.002605in}{0.009821in}}{\pgfqpoint{-0.005103in}{0.008786in}}{\pgfqpoint{-0.006944in}{0.006944in}}%
\pgfpathcurveto{\pgfqpoint{-0.008786in}{0.005103in}}{\pgfqpoint{-0.009821in}{0.002605in}}{\pgfqpoint{-0.009821in}{0.000000in}}%
\pgfpathcurveto{\pgfqpoint{-0.009821in}{-0.002605in}}{\pgfqpoint{-0.008786in}{-0.005103in}}{\pgfqpoint{-0.006944in}{-0.006944in}}%
\pgfpathcurveto{\pgfqpoint{-0.005103in}{-0.008786in}}{\pgfqpoint{-0.002605in}{-0.009821in}}{\pgfqpoint{0.000000in}{-0.009821in}}%
\pgfpathclose%
\pgfusepath{stroke,fill}%
}%
\begin{pgfscope}%
\pgfsys@transformshift{1.154259in}{3.379610in}%
\pgfsys@useobject{currentmarker}{}%
\end{pgfscope}%
\begin{pgfscope}%
\pgfsys@transformshift{1.390735in}{3.320849in}%
\pgfsys@useobject{currentmarker}{}%
\end{pgfscope}%
\begin{pgfscope}%
\pgfsys@transformshift{1.558517in}{3.277668in}%
\pgfsys@useobject{currentmarker}{}%
\end{pgfscope}%
\begin{pgfscope}%
\pgfsys@transformshift{1.794993in}{3.215799in}%
\pgfsys@useobject{currentmarker}{}%
\end{pgfscope}%
\begin{pgfscope}%
\pgfsys@transformshift{1.962776in}{3.171538in}%
\pgfsys@useobject{currentmarker}{}%
\end{pgfscope}%
\begin{pgfscope}%
\pgfsys@transformshift{2.148505in}{3.122378in}%
\pgfsys@useobject{currentmarker}{}%
\end{pgfscope}%
\begin{pgfscope}%
\pgfsys@transformshift{2.367034in}{3.064437in}%
\pgfsys@useobject{currentmarker}{}%
\end{pgfscope}%
\begin{pgfscope}%
\pgfsys@transformshift{2.578689in}{3.008283in}%
\pgfsys@useobject{currentmarker}{}%
\end{pgfscope}%
\begin{pgfscope}%
\pgfsys@transformshift{2.771293in}{2.957178in}%
\pgfsys@useobject{currentmarker}{}%
\end{pgfscope}%
\begin{pgfscope}%
\pgfsys@transformshift{2.970129in}{2.904425in}%
\pgfsys@useobject{currentmarker}{}%
\end{pgfscope}%
\begin{pgfscope}%
\pgfsys@transformshift{3.175552in}{2.849934in}%
\pgfsys@useobject{currentmarker}{}%
\end{pgfscope}%
\begin{pgfscope}%
\pgfsys@transformshift{3.380832in}{2.795488in}%
\pgfsys@useobject{currentmarker}{}%
\end{pgfscope}%
\begin{pgfscope}%
\pgfsys@transformshift{3.579810in}{2.742721in}%
\pgfsys@useobject{currentmarker}{}%
\end{pgfscope}%
\begin{pgfscope}%
\pgfsys@transformshift{3.781877in}{2.689141in}%
\pgfsys@useobject{currentmarker}{}%
\end{pgfscope}%
\begin{pgfscope}%
\pgfsys@transformshift{3.984069in}{2.635532in}%
\pgfsys@useobject{currentmarker}{}%
\end{pgfscope}%
\begin{pgfscope}%
\pgfsys@transformshift{4.186136in}{2.581959in}%
\pgfsys@useobject{currentmarker}{}%
\end{pgfscope}%
\begin{pgfscope}%
\pgfsys@transformshift{4.388328in}{2.528356in}%
\pgfsys@useobject{currentmarker}{}%
\end{pgfscope}%
\begin{pgfscope}%
\pgfsys@transformshift{4.590395in}{2.474787in}%
\pgfsys@useobject{currentmarker}{}%
\end{pgfscope}%
\begin{pgfscope}%
\pgfsys@transformshift{4.792586in}{2.421187in}%
\pgfsys@useobject{currentmarker}{}%
\end{pgfscope}%
\begin{pgfscope}%
\pgfsys@transformshift{4.994653in}{2.367621in}%
\pgfsys@useobject{currentmarker}{}%
\end{pgfscope}%
\begin{pgfscope}%
\pgfsys@transformshift{5.196845in}{2.314022in}%
\pgfsys@useobject{currentmarker}{}%
\end{pgfscope}%
\end{pgfscope}%
\begin{pgfscope}%
\pgfpathrectangle{\pgfqpoint{0.750000in}{0.440000in}}{\pgfqpoint{4.650000in}{3.080000in}}%
\pgfusepath{clip}%
\pgfsetbuttcap%
\pgfsetroundjoin%
\definecolor{currentfill}{rgb}{0.839216,0.152941,0.156863}%
\pgfsetfillcolor{currentfill}%
\pgfsetlinewidth{1.003750pt}%
\definecolor{currentstroke}{rgb}{0.839216,0.152941,0.156863}%
\pgfsetstrokecolor{currentstroke}%
\pgfsetdash{}{0pt}%
\pgfsys@defobject{currentmarker}{\pgfqpoint{-0.009821in}{-0.009821in}}{\pgfqpoint{0.009821in}{0.009821in}}{%
\pgfpathmoveto{\pgfqpoint{0.000000in}{-0.009821in}}%
\pgfpathcurveto{\pgfqpoint{0.002605in}{-0.009821in}}{\pgfqpoint{0.005103in}{-0.008786in}}{\pgfqpoint{0.006944in}{-0.006944in}}%
\pgfpathcurveto{\pgfqpoint{0.008786in}{-0.005103in}}{\pgfqpoint{0.009821in}{-0.002605in}}{\pgfqpoint{0.009821in}{0.000000in}}%
\pgfpathcurveto{\pgfqpoint{0.009821in}{0.002605in}}{\pgfqpoint{0.008786in}{0.005103in}}{\pgfqpoint{0.006944in}{0.006944in}}%
\pgfpathcurveto{\pgfqpoint{0.005103in}{0.008786in}}{\pgfqpoint{0.002605in}{0.009821in}}{\pgfqpoint{0.000000in}{0.009821in}}%
\pgfpathcurveto{\pgfqpoint{-0.002605in}{0.009821in}}{\pgfqpoint{-0.005103in}{0.008786in}}{\pgfqpoint{-0.006944in}{0.006944in}}%
\pgfpathcurveto{\pgfqpoint{-0.008786in}{0.005103in}}{\pgfqpoint{-0.009821in}{0.002605in}}{\pgfqpoint{-0.009821in}{0.000000in}}%
\pgfpathcurveto{\pgfqpoint{-0.009821in}{-0.002605in}}{\pgfqpoint{-0.008786in}{-0.005103in}}{\pgfqpoint{-0.006944in}{-0.006944in}}%
\pgfpathcurveto{\pgfqpoint{-0.005103in}{-0.008786in}}{\pgfqpoint{-0.002605in}{-0.009821in}}{\pgfqpoint{0.000000in}{-0.009821in}}%
\pgfpathclose%
\pgfusepath{stroke,fill}%
}%
\begin{pgfscope}%
\pgfsys@transformshift{1.154259in}{2.977816in}%
\pgfsys@useobject{currentmarker}{}%
\end{pgfscope}%
\begin{pgfscope}%
\pgfsys@transformshift{1.390735in}{2.881797in}%
\pgfsys@useobject{currentmarker}{}%
\end{pgfscope}%
\begin{pgfscope}%
\pgfsys@transformshift{1.558517in}{2.818005in}%
\pgfsys@useobject{currentmarker}{}%
\end{pgfscope}%
\begin{pgfscope}%
\pgfsys@transformshift{1.794993in}{2.717306in}%
\pgfsys@useobject{currentmarker}{}%
\end{pgfscope}%
\begin{pgfscope}%
\pgfsys@transformshift{1.962776in}{2.640730in}%
\pgfsys@useobject{currentmarker}{}%
\end{pgfscope}%
\begin{pgfscope}%
\pgfsys@transformshift{2.148505in}{2.551330in}%
\pgfsys@useobject{currentmarker}{}%
\end{pgfscope}%
\begin{pgfscope}%
\pgfsys@transformshift{2.367034in}{2.441848in}%
\pgfsys@useobject{currentmarker}{}%
\end{pgfscope}%
\begin{pgfscope}%
\pgfsys@transformshift{2.578689in}{2.333418in}%
\pgfsys@useobject{currentmarker}{}%
\end{pgfscope}%
\begin{pgfscope}%
\pgfsys@transformshift{2.771293in}{2.234386in}%
\pgfsys@useobject{currentmarker}{}%
\end{pgfscope}%
\begin{pgfscope}%
\pgfsys@transformshift{2.970129in}{2.131062in}%
\pgfsys@useobject{currentmarker}{}%
\end{pgfscope}%
\begin{pgfscope}%
\pgfsys@transformshift{3.175552in}{2.023689in}%
\pgfsys@useobject{currentmarker}{}%
\end{pgfscope}%
\begin{pgfscope}%
\pgfsys@transformshift{3.380832in}{1.915965in}%
\pgfsys@useobject{currentmarker}{}%
\end{pgfscope}%
\begin{pgfscope}%
\pgfsys@transformshift{3.579810in}{1.811249in}%
\pgfsys@useobject{currentmarker}{}%
\end{pgfscope}%
\begin{pgfscope}%
\pgfsys@transformshift{3.781877in}{1.704678in}%
\pgfsys@useobject{currentmarker}{}%
\end{pgfscope}%
\begin{pgfscope}%
\pgfsys@transformshift{3.984069in}{1.597893in}%
\pgfsys@useobject{currentmarker}{}%
\end{pgfscope}%
\begin{pgfscope}%
\pgfsys@transformshift{4.186136in}{1.491059in}%
\pgfsys@useobject{currentmarker}{}%
\end{pgfscope}%
\begin{pgfscope}%
\pgfsys@transformshift{4.388328in}{1.384110in}%
\pgfsys@useobject{currentmarker}{}%
\end{pgfscope}%
\begin{pgfscope}%
\pgfsys@transformshift{4.590395in}{1.277244in}%
\pgfsys@useobject{currentmarker}{}%
\end{pgfscope}%
\begin{pgfscope}%
\pgfsys@transformshift{4.792586in}{1.170455in}%
\pgfsys@useobject{currentmarker}{}%
\end{pgfscope}%
\begin{pgfscope}%
\pgfsys@transformshift{4.994653in}{1.063753in}%
\pgfsys@useobject{currentmarker}{}%
\end{pgfscope}%
\begin{pgfscope}%
\pgfsys@transformshift{5.196845in}{0.956704in}%
\pgfsys@useobject{currentmarker}{}%
\end{pgfscope}%
\end{pgfscope}%
\begin{pgfscope}%
\pgfpathrectangle{\pgfqpoint{0.750000in}{0.440000in}}{\pgfqpoint{4.650000in}{3.080000in}}%
\pgfusepath{clip}%
\pgfsetbuttcap%
\pgfsetroundjoin%
\definecolor{currentfill}{rgb}{0.580392,0.403922,0.741176}%
\pgfsetfillcolor{currentfill}%
\pgfsetlinewidth{1.003750pt}%
\definecolor{currentstroke}{rgb}{0.580392,0.403922,0.741176}%
\pgfsetstrokecolor{currentstroke}%
\pgfsetdash{}{0pt}%
\pgfsys@defobject{currentmarker}{\pgfqpoint{-0.009821in}{-0.009821in}}{\pgfqpoint{0.009821in}{0.009821in}}{%
\pgfpathmoveto{\pgfqpoint{0.000000in}{-0.009821in}}%
\pgfpathcurveto{\pgfqpoint{0.002605in}{-0.009821in}}{\pgfqpoint{0.005103in}{-0.008786in}}{\pgfqpoint{0.006944in}{-0.006944in}}%
\pgfpathcurveto{\pgfqpoint{0.008786in}{-0.005103in}}{\pgfqpoint{0.009821in}{-0.002605in}}{\pgfqpoint{0.009821in}{0.000000in}}%
\pgfpathcurveto{\pgfqpoint{0.009821in}{0.002605in}}{\pgfqpoint{0.008786in}{0.005103in}}{\pgfqpoint{0.006944in}{0.006944in}}%
\pgfpathcurveto{\pgfqpoint{0.005103in}{0.008786in}}{\pgfqpoint{0.002605in}{0.009821in}}{\pgfqpoint{0.000000in}{0.009821in}}%
\pgfpathcurveto{\pgfqpoint{-0.002605in}{0.009821in}}{\pgfqpoint{-0.005103in}{0.008786in}}{\pgfqpoint{-0.006944in}{0.006944in}}%
\pgfpathcurveto{\pgfqpoint{-0.008786in}{0.005103in}}{\pgfqpoint{-0.009821in}{0.002605in}}{\pgfqpoint{-0.009821in}{0.000000in}}%
\pgfpathcurveto{\pgfqpoint{-0.009821in}{-0.002605in}}{\pgfqpoint{-0.008786in}{-0.005103in}}{\pgfqpoint{-0.006944in}{-0.006944in}}%
\pgfpathcurveto{\pgfqpoint{-0.005103in}{-0.008786in}}{\pgfqpoint{-0.002605in}{-0.009821in}}{\pgfqpoint{0.000000in}{-0.009821in}}%
\pgfpathclose%
\pgfusepath{stroke,fill}%
}%
\begin{pgfscope}%
\pgfsys@transformshift{1.154259in}{2.834542in}%
\pgfsys@useobject{currentmarker}{}%
\end{pgfscope}%
\begin{pgfscope}%
\pgfsys@transformshift{1.390735in}{2.715838in}%
\pgfsys@useobject{currentmarker}{}%
\end{pgfscope}%
\begin{pgfscope}%
\pgfsys@transformshift{1.558517in}{2.629050in}%
\pgfsys@useobject{currentmarker}{}%
\end{pgfscope}%
\begin{pgfscope}%
\pgfsys@transformshift{1.794993in}{2.505222in}%
\pgfsys@useobject{currentmarker}{}%
\end{pgfscope}%
\begin{pgfscope}%
\pgfsys@transformshift{1.962776in}{2.416814in}%
\pgfsys@useobject{currentmarker}{}%
\end{pgfscope}%
\begin{pgfscope}%
\pgfsys@transformshift{2.148505in}{2.318679in}%
\pgfsys@useobject{currentmarker}{}%
\end{pgfscope}%
\begin{pgfscope}%
\pgfsys@transformshift{2.367034in}{2.203020in}%
\pgfsys@useobject{currentmarker}{}%
\end{pgfscope}%
\begin{pgfscope}%
\pgfsys@transformshift{2.578689in}{2.090901in}%
\pgfsys@useobject{currentmarker}{}%
\end{pgfscope}%
\begin{pgfscope}%
\pgfsys@transformshift{2.771293in}{1.988831in}%
\pgfsys@useobject{currentmarker}{}%
\end{pgfscope}%
\begin{pgfscope}%
\pgfsys@transformshift{2.970129in}{1.883438in}%
\pgfsys@useobject{currentmarker}{}%
\end{pgfscope}%
\begin{pgfscope}%
\pgfsys@transformshift{3.175552in}{1.774542in}%
\pgfsys@useobject{currentmarker}{}%
\end{pgfscope}%
\begin{pgfscope}%
\pgfsys@transformshift{3.380832in}{1.665716in}%
\pgfsys@useobject{currentmarker}{}%
\end{pgfscope}%
\begin{pgfscope}%
\pgfsys@transformshift{3.579810in}{1.560229in}%
\pgfsys@useobject{currentmarker}{}%
\end{pgfscope}%
\begin{pgfscope}%
\pgfsys@transformshift{3.781877in}{1.453102in}%
\pgfsys@useobject{currentmarker}{}%
\end{pgfscope}%
\begin{pgfscope}%
\pgfsys@transformshift{3.984069in}{1.345910in}%
\pgfsys@useobject{currentmarker}{}%
\end{pgfscope}%
\begin{pgfscope}%
\pgfsys@transformshift{4.186136in}{1.238784in}%
\pgfsys@useobject{currentmarker}{}%
\end{pgfscope}%
\begin{pgfscope}%
\pgfsys@transformshift{4.388328in}{1.131594in}%
\pgfsys@useobject{currentmarker}{}%
\end{pgfscope}%
\begin{pgfscope}%
\pgfsys@transformshift{4.590395in}{1.024479in}%
\pgfsys@useobject{currentmarker}{}%
\end{pgfscope}%
\begin{pgfscope}%
\pgfsys@transformshift{4.792586in}{0.917330in}%
\pgfsys@useobject{currentmarker}{}%
\end{pgfscope}%
\begin{pgfscope}%
\pgfsys@transformshift{4.994653in}{0.810333in}%
\pgfsys@useobject{currentmarker}{}%
\end{pgfscope}%
\begin{pgfscope}%
\pgfsys@transformshift{5.196845in}{0.703511in}%
\pgfsys@useobject{currentmarker}{}%
\end{pgfscope}%
\end{pgfscope}%
\begin{pgfscope}%
\pgfpathrectangle{\pgfqpoint{0.750000in}{0.440000in}}{\pgfqpoint{4.650000in}{3.080000in}}%
\pgfusepath{clip}%
\pgfsetbuttcap%
\pgfsetroundjoin%
\definecolor{currentfill}{rgb}{0.549020,0.337255,0.294118}%
\pgfsetfillcolor{currentfill}%
\pgfsetlinewidth{1.003750pt}%
\definecolor{currentstroke}{rgb}{0.549020,0.337255,0.294118}%
\pgfsetstrokecolor{currentstroke}%
\pgfsetdash{}{0pt}%
\pgfsys@defobject{currentmarker}{\pgfqpoint{-0.009821in}{-0.009821in}}{\pgfqpoint{0.009821in}{0.009821in}}{%
\pgfpathmoveto{\pgfqpoint{0.000000in}{-0.009821in}}%
\pgfpathcurveto{\pgfqpoint{0.002605in}{-0.009821in}}{\pgfqpoint{0.005103in}{-0.008786in}}{\pgfqpoint{0.006944in}{-0.006944in}}%
\pgfpathcurveto{\pgfqpoint{0.008786in}{-0.005103in}}{\pgfqpoint{0.009821in}{-0.002605in}}{\pgfqpoint{0.009821in}{0.000000in}}%
\pgfpathcurveto{\pgfqpoint{0.009821in}{0.002605in}}{\pgfqpoint{0.008786in}{0.005103in}}{\pgfqpoint{0.006944in}{0.006944in}}%
\pgfpathcurveto{\pgfqpoint{0.005103in}{0.008786in}}{\pgfqpoint{0.002605in}{0.009821in}}{\pgfqpoint{0.000000in}{0.009821in}}%
\pgfpathcurveto{\pgfqpoint{-0.002605in}{0.009821in}}{\pgfqpoint{-0.005103in}{0.008786in}}{\pgfqpoint{-0.006944in}{0.006944in}}%
\pgfpathcurveto{\pgfqpoint{-0.008786in}{0.005103in}}{\pgfqpoint{-0.009821in}{0.002605in}}{\pgfqpoint{-0.009821in}{0.000000in}}%
\pgfpathcurveto{\pgfqpoint{-0.009821in}{-0.002605in}}{\pgfqpoint{-0.008786in}{-0.005103in}}{\pgfqpoint{-0.006944in}{-0.006944in}}%
\pgfpathcurveto{\pgfqpoint{-0.005103in}{-0.008786in}}{\pgfqpoint{-0.002605in}{-0.009821in}}{\pgfqpoint{0.000000in}{-0.009821in}}%
\pgfpathclose%
\pgfusepath{stroke,fill}%
}%
\begin{pgfscope}%
\pgfsys@transformshift{1.154259in}{3.031032in}%
\pgfsys@useobject{currentmarker}{}%
\end{pgfscope}%
\begin{pgfscope}%
\pgfsys@transformshift{1.390735in}{2.967186in}%
\pgfsys@useobject{currentmarker}{}%
\end{pgfscope}%
\begin{pgfscope}%
\pgfsys@transformshift{1.558517in}{2.922026in}%
\pgfsys@useobject{currentmarker}{}%
\end{pgfscope}%
\begin{pgfscope}%
\pgfsys@transformshift{1.794993in}{2.858775in}%
\pgfsys@useobject{currentmarker}{}%
\end{pgfscope}%
\begin{pgfscope}%
\pgfsys@transformshift{1.962776in}{2.814087in}%
\pgfsys@useobject{currentmarker}{}%
\end{pgfscope}%
\begin{pgfscope}%
\pgfsys@transformshift{2.148505in}{2.764722in}%
\pgfsys@useobject{currentmarker}{}%
\end{pgfscope}%
\begin{pgfscope}%
\pgfsys@transformshift{2.367034in}{2.706715in}%
\pgfsys@useobject{currentmarker}{}%
\end{pgfscope}%
\begin{pgfscope}%
\pgfsys@transformshift{2.578689in}{2.650572in}%
\pgfsys@useobject{currentmarker}{}%
\end{pgfscope}%
\begin{pgfscope}%
\pgfsys@transformshift{2.771293in}{2.599500in}%
\pgfsys@useobject{currentmarker}{}%
\end{pgfscope}%
\begin{pgfscope}%
\pgfsys@transformshift{2.970129in}{2.546784in}%
\pgfsys@useobject{currentmarker}{}%
\end{pgfscope}%
\begin{pgfscope}%
\pgfsys@transformshift{3.175552in}{2.492325in}%
\pgfsys@useobject{currentmarker}{}%
\end{pgfscope}%
\begin{pgfscope}%
\pgfsys@transformshift{3.380832in}{2.437907in}%
\pgfsys@useobject{currentmarker}{}%
\end{pgfscope}%
\begin{pgfscope}%
\pgfsys@transformshift{3.579810in}{2.385161in}%
\pgfsys@useobject{currentmarker}{}%
\end{pgfscope}%
\begin{pgfscope}%
\pgfsys@transformshift{3.781877in}{2.331596in}%
\pgfsys@useobject{currentmarker}{}%
\end{pgfscope}%
\begin{pgfscope}%
\pgfsys@transformshift{3.984069in}{2.277999in}%
\pgfsys@useobject{currentmarker}{}%
\end{pgfscope}%
\begin{pgfscope}%
\pgfsys@transformshift{4.186136in}{2.224435in}%
\pgfsys@useobject{currentmarker}{}%
\end{pgfscope}%
\begin{pgfscope}%
\pgfsys@transformshift{4.388328in}{2.170838in}%
\pgfsys@useobject{currentmarker}{}%
\end{pgfscope}%
\begin{pgfscope}%
\pgfsys@transformshift{4.590395in}{2.117274in}%
\pgfsys@useobject{currentmarker}{}%
\end{pgfscope}%
\begin{pgfscope}%
\pgfsys@transformshift{4.792586in}{2.063677in}%
\pgfsys@useobject{currentmarker}{}%
\end{pgfscope}%
\begin{pgfscope}%
\pgfsys@transformshift{4.994653in}{2.010113in}%
\pgfsys@useobject{currentmarker}{}%
\end{pgfscope}%
\begin{pgfscope}%
\pgfsys@transformshift{5.196845in}{1.956516in}%
\pgfsys@useobject{currentmarker}{}%
\end{pgfscope}%
\end{pgfscope}%
\begin{pgfscope}%
\pgfsetbuttcap%
\pgfsetroundjoin%
\definecolor{currentfill}{rgb}{0.000000,0.000000,0.000000}%
\pgfsetfillcolor{currentfill}%
\pgfsetlinewidth{0.803000pt}%
\definecolor{currentstroke}{rgb}{0.000000,0.000000,0.000000}%
\pgfsetstrokecolor{currentstroke}%
\pgfsetdash{}{0pt}%
\pgfsys@defobject{currentmarker}{\pgfqpoint{0.000000in}{-0.048611in}}{\pgfqpoint{0.000000in}{0.000000in}}{%
\pgfpathmoveto{\pgfqpoint{0.000000in}{0.000000in}}%
\pgfpathlineto{\pgfqpoint{0.000000in}{-0.048611in}}%
\pgfusepath{stroke,fill}%
}%
\begin{pgfscope}%
\pgfsys@transformshift{0.750000in}{0.440000in}%
\pgfsys@useobject{currentmarker}{}%
\end{pgfscope}%
\end{pgfscope}%
\begin{pgfscope}%
\definecolor{textcolor}{rgb}{0.000000,0.000000,0.000000}%
\pgfsetstrokecolor{textcolor}%
\pgfsetfillcolor{textcolor}%
\pgftext[x=0.750000in,y=0.342778in,,top]{\color{textcolor}\sffamily\fontsize{10.000000}{12.000000}\selectfont \(\displaystyle {10^{0}}\)}%
\end{pgfscope}%
\begin{pgfscope}%
\pgfsetbuttcap%
\pgfsetroundjoin%
\definecolor{currentfill}{rgb}{0.000000,0.000000,0.000000}%
\pgfsetfillcolor{currentfill}%
\pgfsetlinewidth{0.803000pt}%
\definecolor{currentstroke}{rgb}{0.000000,0.000000,0.000000}%
\pgfsetstrokecolor{currentstroke}%
\pgfsetdash{}{0pt}%
\pgfsys@defobject{currentmarker}{\pgfqpoint{0.000000in}{-0.048611in}}{\pgfqpoint{0.000000in}{0.000000in}}{%
\pgfpathmoveto{\pgfqpoint{0.000000in}{0.000000in}}%
\pgfpathlineto{\pgfqpoint{0.000000in}{-0.048611in}}%
\pgfusepath{stroke,fill}%
}%
\begin{pgfscope}%
\pgfsys@transformshift{2.092918in}{0.440000in}%
\pgfsys@useobject{currentmarker}{}%
\end{pgfscope}%
\end{pgfscope}%
\begin{pgfscope}%
\definecolor{textcolor}{rgb}{0.000000,0.000000,0.000000}%
\pgfsetstrokecolor{textcolor}%
\pgfsetfillcolor{textcolor}%
\pgftext[x=2.092918in,y=0.342778in,,top]{\color{textcolor}\sffamily\fontsize{10.000000}{12.000000}\selectfont \(\displaystyle {10^{1}}\)}%
\end{pgfscope}%
\begin{pgfscope}%
\pgfsetbuttcap%
\pgfsetroundjoin%
\definecolor{currentfill}{rgb}{0.000000,0.000000,0.000000}%
\pgfsetfillcolor{currentfill}%
\pgfsetlinewidth{0.803000pt}%
\definecolor{currentstroke}{rgb}{0.000000,0.000000,0.000000}%
\pgfsetstrokecolor{currentstroke}%
\pgfsetdash{}{0pt}%
\pgfsys@defobject{currentmarker}{\pgfqpoint{0.000000in}{-0.048611in}}{\pgfqpoint{0.000000in}{0.000000in}}{%
\pgfpathmoveto{\pgfqpoint{0.000000in}{0.000000in}}%
\pgfpathlineto{\pgfqpoint{0.000000in}{-0.048611in}}%
\pgfusepath{stroke,fill}%
}%
\begin{pgfscope}%
\pgfsys@transformshift{3.435836in}{0.440000in}%
\pgfsys@useobject{currentmarker}{}%
\end{pgfscope}%
\end{pgfscope}%
\begin{pgfscope}%
\definecolor{textcolor}{rgb}{0.000000,0.000000,0.000000}%
\pgfsetstrokecolor{textcolor}%
\pgfsetfillcolor{textcolor}%
\pgftext[x=3.435836in,y=0.342778in,,top]{\color{textcolor}\sffamily\fontsize{10.000000}{12.000000}\selectfont \(\displaystyle {10^{2}}\)}%
\end{pgfscope}%
\begin{pgfscope}%
\pgfsetbuttcap%
\pgfsetroundjoin%
\definecolor{currentfill}{rgb}{0.000000,0.000000,0.000000}%
\pgfsetfillcolor{currentfill}%
\pgfsetlinewidth{0.803000pt}%
\definecolor{currentstroke}{rgb}{0.000000,0.000000,0.000000}%
\pgfsetstrokecolor{currentstroke}%
\pgfsetdash{}{0pt}%
\pgfsys@defobject{currentmarker}{\pgfqpoint{0.000000in}{-0.048611in}}{\pgfqpoint{0.000000in}{0.000000in}}{%
\pgfpathmoveto{\pgfqpoint{0.000000in}{0.000000in}}%
\pgfpathlineto{\pgfqpoint{0.000000in}{-0.048611in}}%
\pgfusepath{stroke,fill}%
}%
\begin{pgfscope}%
\pgfsys@transformshift{4.778754in}{0.440000in}%
\pgfsys@useobject{currentmarker}{}%
\end{pgfscope}%
\end{pgfscope}%
\begin{pgfscope}%
\definecolor{textcolor}{rgb}{0.000000,0.000000,0.000000}%
\pgfsetstrokecolor{textcolor}%
\pgfsetfillcolor{textcolor}%
\pgftext[x=4.778754in,y=0.342778in,,top]{\color{textcolor}\sffamily\fontsize{10.000000}{12.000000}\selectfont \(\displaystyle {10^{3}}\)}%
\end{pgfscope}%
\begin{pgfscope}%
\pgfsetbuttcap%
\pgfsetroundjoin%
\definecolor{currentfill}{rgb}{0.000000,0.000000,0.000000}%
\pgfsetfillcolor{currentfill}%
\pgfsetlinewidth{0.602250pt}%
\definecolor{currentstroke}{rgb}{0.000000,0.000000,0.000000}%
\pgfsetstrokecolor{currentstroke}%
\pgfsetdash{}{0pt}%
\pgfsys@defobject{currentmarker}{\pgfqpoint{0.000000in}{-0.027778in}}{\pgfqpoint{0.000000in}{0.000000in}}{%
\pgfpathmoveto{\pgfqpoint{0.000000in}{0.000000in}}%
\pgfpathlineto{\pgfqpoint{0.000000in}{-0.027778in}}%
\pgfusepath{stroke,fill}%
}%
\begin{pgfscope}%
\pgfsys@transformshift{1.154259in}{0.440000in}%
\pgfsys@useobject{currentmarker}{}%
\end{pgfscope}%
\end{pgfscope}%
\begin{pgfscope}%
\pgfsetbuttcap%
\pgfsetroundjoin%
\definecolor{currentfill}{rgb}{0.000000,0.000000,0.000000}%
\pgfsetfillcolor{currentfill}%
\pgfsetlinewidth{0.602250pt}%
\definecolor{currentstroke}{rgb}{0.000000,0.000000,0.000000}%
\pgfsetstrokecolor{currentstroke}%
\pgfsetdash{}{0pt}%
\pgfsys@defobject{currentmarker}{\pgfqpoint{0.000000in}{-0.027778in}}{\pgfqpoint{0.000000in}{0.000000in}}{%
\pgfpathmoveto{\pgfqpoint{0.000000in}{0.000000in}}%
\pgfpathlineto{\pgfqpoint{0.000000in}{-0.027778in}}%
\pgfusepath{stroke,fill}%
}%
\begin{pgfscope}%
\pgfsys@transformshift{1.390735in}{0.440000in}%
\pgfsys@useobject{currentmarker}{}%
\end{pgfscope}%
\end{pgfscope}%
\begin{pgfscope}%
\pgfsetbuttcap%
\pgfsetroundjoin%
\definecolor{currentfill}{rgb}{0.000000,0.000000,0.000000}%
\pgfsetfillcolor{currentfill}%
\pgfsetlinewidth{0.602250pt}%
\definecolor{currentstroke}{rgb}{0.000000,0.000000,0.000000}%
\pgfsetstrokecolor{currentstroke}%
\pgfsetdash{}{0pt}%
\pgfsys@defobject{currentmarker}{\pgfqpoint{0.000000in}{-0.027778in}}{\pgfqpoint{0.000000in}{0.000000in}}{%
\pgfpathmoveto{\pgfqpoint{0.000000in}{0.000000in}}%
\pgfpathlineto{\pgfqpoint{0.000000in}{-0.027778in}}%
\pgfusepath{stroke,fill}%
}%
\begin{pgfscope}%
\pgfsys@transformshift{1.558517in}{0.440000in}%
\pgfsys@useobject{currentmarker}{}%
\end{pgfscope}%
\end{pgfscope}%
\begin{pgfscope}%
\pgfsetbuttcap%
\pgfsetroundjoin%
\definecolor{currentfill}{rgb}{0.000000,0.000000,0.000000}%
\pgfsetfillcolor{currentfill}%
\pgfsetlinewidth{0.602250pt}%
\definecolor{currentstroke}{rgb}{0.000000,0.000000,0.000000}%
\pgfsetstrokecolor{currentstroke}%
\pgfsetdash{}{0pt}%
\pgfsys@defobject{currentmarker}{\pgfqpoint{0.000000in}{-0.027778in}}{\pgfqpoint{0.000000in}{0.000000in}}{%
\pgfpathmoveto{\pgfqpoint{0.000000in}{0.000000in}}%
\pgfpathlineto{\pgfqpoint{0.000000in}{-0.027778in}}%
\pgfusepath{stroke,fill}%
}%
\begin{pgfscope}%
\pgfsys@transformshift{1.688659in}{0.440000in}%
\pgfsys@useobject{currentmarker}{}%
\end{pgfscope}%
\end{pgfscope}%
\begin{pgfscope}%
\pgfsetbuttcap%
\pgfsetroundjoin%
\definecolor{currentfill}{rgb}{0.000000,0.000000,0.000000}%
\pgfsetfillcolor{currentfill}%
\pgfsetlinewidth{0.602250pt}%
\definecolor{currentstroke}{rgb}{0.000000,0.000000,0.000000}%
\pgfsetstrokecolor{currentstroke}%
\pgfsetdash{}{0pt}%
\pgfsys@defobject{currentmarker}{\pgfqpoint{0.000000in}{-0.027778in}}{\pgfqpoint{0.000000in}{0.000000in}}{%
\pgfpathmoveto{\pgfqpoint{0.000000in}{0.000000in}}%
\pgfpathlineto{\pgfqpoint{0.000000in}{-0.027778in}}%
\pgfusepath{stroke,fill}%
}%
\begin{pgfscope}%
\pgfsys@transformshift{1.794993in}{0.440000in}%
\pgfsys@useobject{currentmarker}{}%
\end{pgfscope}%
\end{pgfscope}%
\begin{pgfscope}%
\pgfsetbuttcap%
\pgfsetroundjoin%
\definecolor{currentfill}{rgb}{0.000000,0.000000,0.000000}%
\pgfsetfillcolor{currentfill}%
\pgfsetlinewidth{0.602250pt}%
\definecolor{currentstroke}{rgb}{0.000000,0.000000,0.000000}%
\pgfsetstrokecolor{currentstroke}%
\pgfsetdash{}{0pt}%
\pgfsys@defobject{currentmarker}{\pgfqpoint{0.000000in}{-0.027778in}}{\pgfqpoint{0.000000in}{0.000000in}}{%
\pgfpathmoveto{\pgfqpoint{0.000000in}{0.000000in}}%
\pgfpathlineto{\pgfqpoint{0.000000in}{-0.027778in}}%
\pgfusepath{stroke,fill}%
}%
\begin{pgfscope}%
\pgfsys@transformshift{1.884897in}{0.440000in}%
\pgfsys@useobject{currentmarker}{}%
\end{pgfscope}%
\end{pgfscope}%
\begin{pgfscope}%
\pgfsetbuttcap%
\pgfsetroundjoin%
\definecolor{currentfill}{rgb}{0.000000,0.000000,0.000000}%
\pgfsetfillcolor{currentfill}%
\pgfsetlinewidth{0.602250pt}%
\definecolor{currentstroke}{rgb}{0.000000,0.000000,0.000000}%
\pgfsetstrokecolor{currentstroke}%
\pgfsetdash{}{0pt}%
\pgfsys@defobject{currentmarker}{\pgfqpoint{0.000000in}{-0.027778in}}{\pgfqpoint{0.000000in}{0.000000in}}{%
\pgfpathmoveto{\pgfqpoint{0.000000in}{0.000000in}}%
\pgfpathlineto{\pgfqpoint{0.000000in}{-0.027778in}}%
\pgfusepath{stroke,fill}%
}%
\begin{pgfscope}%
\pgfsys@transformshift{1.962776in}{0.440000in}%
\pgfsys@useobject{currentmarker}{}%
\end{pgfscope}%
\end{pgfscope}%
\begin{pgfscope}%
\pgfsetbuttcap%
\pgfsetroundjoin%
\definecolor{currentfill}{rgb}{0.000000,0.000000,0.000000}%
\pgfsetfillcolor{currentfill}%
\pgfsetlinewidth{0.602250pt}%
\definecolor{currentstroke}{rgb}{0.000000,0.000000,0.000000}%
\pgfsetstrokecolor{currentstroke}%
\pgfsetdash{}{0pt}%
\pgfsys@defobject{currentmarker}{\pgfqpoint{0.000000in}{-0.027778in}}{\pgfqpoint{0.000000in}{0.000000in}}{%
\pgfpathmoveto{\pgfqpoint{0.000000in}{0.000000in}}%
\pgfpathlineto{\pgfqpoint{0.000000in}{-0.027778in}}%
\pgfusepath{stroke,fill}%
}%
\begin{pgfscope}%
\pgfsys@transformshift{2.031469in}{0.440000in}%
\pgfsys@useobject{currentmarker}{}%
\end{pgfscope}%
\end{pgfscope}%
\begin{pgfscope}%
\pgfsetbuttcap%
\pgfsetroundjoin%
\definecolor{currentfill}{rgb}{0.000000,0.000000,0.000000}%
\pgfsetfillcolor{currentfill}%
\pgfsetlinewidth{0.602250pt}%
\definecolor{currentstroke}{rgb}{0.000000,0.000000,0.000000}%
\pgfsetstrokecolor{currentstroke}%
\pgfsetdash{}{0pt}%
\pgfsys@defobject{currentmarker}{\pgfqpoint{0.000000in}{-0.027778in}}{\pgfqpoint{0.000000in}{0.000000in}}{%
\pgfpathmoveto{\pgfqpoint{0.000000in}{0.000000in}}%
\pgfpathlineto{\pgfqpoint{0.000000in}{-0.027778in}}%
\pgfusepath{stroke,fill}%
}%
\begin{pgfscope}%
\pgfsys@transformshift{2.497177in}{0.440000in}%
\pgfsys@useobject{currentmarker}{}%
\end{pgfscope}%
\end{pgfscope}%
\begin{pgfscope}%
\pgfsetbuttcap%
\pgfsetroundjoin%
\definecolor{currentfill}{rgb}{0.000000,0.000000,0.000000}%
\pgfsetfillcolor{currentfill}%
\pgfsetlinewidth{0.602250pt}%
\definecolor{currentstroke}{rgb}{0.000000,0.000000,0.000000}%
\pgfsetstrokecolor{currentstroke}%
\pgfsetdash{}{0pt}%
\pgfsys@defobject{currentmarker}{\pgfqpoint{0.000000in}{-0.027778in}}{\pgfqpoint{0.000000in}{0.000000in}}{%
\pgfpathmoveto{\pgfqpoint{0.000000in}{0.000000in}}%
\pgfpathlineto{\pgfqpoint{0.000000in}{-0.027778in}}%
\pgfusepath{stroke,fill}%
}%
\begin{pgfscope}%
\pgfsys@transformshift{2.733653in}{0.440000in}%
\pgfsys@useobject{currentmarker}{}%
\end{pgfscope}%
\end{pgfscope}%
\begin{pgfscope}%
\pgfsetbuttcap%
\pgfsetroundjoin%
\definecolor{currentfill}{rgb}{0.000000,0.000000,0.000000}%
\pgfsetfillcolor{currentfill}%
\pgfsetlinewidth{0.602250pt}%
\definecolor{currentstroke}{rgb}{0.000000,0.000000,0.000000}%
\pgfsetstrokecolor{currentstroke}%
\pgfsetdash{}{0pt}%
\pgfsys@defobject{currentmarker}{\pgfqpoint{0.000000in}{-0.027778in}}{\pgfqpoint{0.000000in}{0.000000in}}{%
\pgfpathmoveto{\pgfqpoint{0.000000in}{0.000000in}}%
\pgfpathlineto{\pgfqpoint{0.000000in}{-0.027778in}}%
\pgfusepath{stroke,fill}%
}%
\begin{pgfscope}%
\pgfsys@transformshift{2.901435in}{0.440000in}%
\pgfsys@useobject{currentmarker}{}%
\end{pgfscope}%
\end{pgfscope}%
\begin{pgfscope}%
\pgfsetbuttcap%
\pgfsetroundjoin%
\definecolor{currentfill}{rgb}{0.000000,0.000000,0.000000}%
\pgfsetfillcolor{currentfill}%
\pgfsetlinewidth{0.602250pt}%
\definecolor{currentstroke}{rgb}{0.000000,0.000000,0.000000}%
\pgfsetstrokecolor{currentstroke}%
\pgfsetdash{}{0pt}%
\pgfsys@defobject{currentmarker}{\pgfqpoint{0.000000in}{-0.027778in}}{\pgfqpoint{0.000000in}{0.000000in}}{%
\pgfpathmoveto{\pgfqpoint{0.000000in}{0.000000in}}%
\pgfpathlineto{\pgfqpoint{0.000000in}{-0.027778in}}%
\pgfusepath{stroke,fill}%
}%
\begin{pgfscope}%
\pgfsys@transformshift{3.031577in}{0.440000in}%
\pgfsys@useobject{currentmarker}{}%
\end{pgfscope}%
\end{pgfscope}%
\begin{pgfscope}%
\pgfsetbuttcap%
\pgfsetroundjoin%
\definecolor{currentfill}{rgb}{0.000000,0.000000,0.000000}%
\pgfsetfillcolor{currentfill}%
\pgfsetlinewidth{0.602250pt}%
\definecolor{currentstroke}{rgb}{0.000000,0.000000,0.000000}%
\pgfsetstrokecolor{currentstroke}%
\pgfsetdash{}{0pt}%
\pgfsys@defobject{currentmarker}{\pgfqpoint{0.000000in}{-0.027778in}}{\pgfqpoint{0.000000in}{0.000000in}}{%
\pgfpathmoveto{\pgfqpoint{0.000000in}{0.000000in}}%
\pgfpathlineto{\pgfqpoint{0.000000in}{-0.027778in}}%
\pgfusepath{stroke,fill}%
}%
\begin{pgfscope}%
\pgfsys@transformshift{3.137911in}{0.440000in}%
\pgfsys@useobject{currentmarker}{}%
\end{pgfscope}%
\end{pgfscope}%
\begin{pgfscope}%
\pgfsetbuttcap%
\pgfsetroundjoin%
\definecolor{currentfill}{rgb}{0.000000,0.000000,0.000000}%
\pgfsetfillcolor{currentfill}%
\pgfsetlinewidth{0.602250pt}%
\definecolor{currentstroke}{rgb}{0.000000,0.000000,0.000000}%
\pgfsetstrokecolor{currentstroke}%
\pgfsetdash{}{0pt}%
\pgfsys@defobject{currentmarker}{\pgfqpoint{0.000000in}{-0.027778in}}{\pgfqpoint{0.000000in}{0.000000in}}{%
\pgfpathmoveto{\pgfqpoint{0.000000in}{0.000000in}}%
\pgfpathlineto{\pgfqpoint{0.000000in}{-0.027778in}}%
\pgfusepath{stroke,fill}%
}%
\begin{pgfscope}%
\pgfsys@transformshift{3.227815in}{0.440000in}%
\pgfsys@useobject{currentmarker}{}%
\end{pgfscope}%
\end{pgfscope}%
\begin{pgfscope}%
\pgfsetbuttcap%
\pgfsetroundjoin%
\definecolor{currentfill}{rgb}{0.000000,0.000000,0.000000}%
\pgfsetfillcolor{currentfill}%
\pgfsetlinewidth{0.602250pt}%
\definecolor{currentstroke}{rgb}{0.000000,0.000000,0.000000}%
\pgfsetstrokecolor{currentstroke}%
\pgfsetdash{}{0pt}%
\pgfsys@defobject{currentmarker}{\pgfqpoint{0.000000in}{-0.027778in}}{\pgfqpoint{0.000000in}{0.000000in}}{%
\pgfpathmoveto{\pgfqpoint{0.000000in}{0.000000in}}%
\pgfpathlineto{\pgfqpoint{0.000000in}{-0.027778in}}%
\pgfusepath{stroke,fill}%
}%
\begin{pgfscope}%
\pgfsys@transformshift{3.305694in}{0.440000in}%
\pgfsys@useobject{currentmarker}{}%
\end{pgfscope}%
\end{pgfscope}%
\begin{pgfscope}%
\pgfsetbuttcap%
\pgfsetroundjoin%
\definecolor{currentfill}{rgb}{0.000000,0.000000,0.000000}%
\pgfsetfillcolor{currentfill}%
\pgfsetlinewidth{0.602250pt}%
\definecolor{currentstroke}{rgb}{0.000000,0.000000,0.000000}%
\pgfsetstrokecolor{currentstroke}%
\pgfsetdash{}{0pt}%
\pgfsys@defobject{currentmarker}{\pgfqpoint{0.000000in}{-0.027778in}}{\pgfqpoint{0.000000in}{0.000000in}}{%
\pgfpathmoveto{\pgfqpoint{0.000000in}{0.000000in}}%
\pgfpathlineto{\pgfqpoint{0.000000in}{-0.027778in}}%
\pgfusepath{stroke,fill}%
}%
\begin{pgfscope}%
\pgfsys@transformshift{3.374388in}{0.440000in}%
\pgfsys@useobject{currentmarker}{}%
\end{pgfscope}%
\end{pgfscope}%
\begin{pgfscope}%
\pgfsetbuttcap%
\pgfsetroundjoin%
\definecolor{currentfill}{rgb}{0.000000,0.000000,0.000000}%
\pgfsetfillcolor{currentfill}%
\pgfsetlinewidth{0.602250pt}%
\definecolor{currentstroke}{rgb}{0.000000,0.000000,0.000000}%
\pgfsetstrokecolor{currentstroke}%
\pgfsetdash{}{0pt}%
\pgfsys@defobject{currentmarker}{\pgfqpoint{0.000000in}{-0.027778in}}{\pgfqpoint{0.000000in}{0.000000in}}{%
\pgfpathmoveto{\pgfqpoint{0.000000in}{0.000000in}}%
\pgfpathlineto{\pgfqpoint{0.000000in}{-0.027778in}}%
\pgfusepath{stroke,fill}%
}%
\begin{pgfscope}%
\pgfsys@transformshift{3.840095in}{0.440000in}%
\pgfsys@useobject{currentmarker}{}%
\end{pgfscope}%
\end{pgfscope}%
\begin{pgfscope}%
\pgfsetbuttcap%
\pgfsetroundjoin%
\definecolor{currentfill}{rgb}{0.000000,0.000000,0.000000}%
\pgfsetfillcolor{currentfill}%
\pgfsetlinewidth{0.602250pt}%
\definecolor{currentstroke}{rgb}{0.000000,0.000000,0.000000}%
\pgfsetstrokecolor{currentstroke}%
\pgfsetdash{}{0pt}%
\pgfsys@defobject{currentmarker}{\pgfqpoint{0.000000in}{-0.027778in}}{\pgfqpoint{0.000000in}{0.000000in}}{%
\pgfpathmoveto{\pgfqpoint{0.000000in}{0.000000in}}%
\pgfpathlineto{\pgfqpoint{0.000000in}{-0.027778in}}%
\pgfusepath{stroke,fill}%
}%
\begin{pgfscope}%
\pgfsys@transformshift{4.076571in}{0.440000in}%
\pgfsys@useobject{currentmarker}{}%
\end{pgfscope}%
\end{pgfscope}%
\begin{pgfscope}%
\pgfsetbuttcap%
\pgfsetroundjoin%
\definecolor{currentfill}{rgb}{0.000000,0.000000,0.000000}%
\pgfsetfillcolor{currentfill}%
\pgfsetlinewidth{0.602250pt}%
\definecolor{currentstroke}{rgb}{0.000000,0.000000,0.000000}%
\pgfsetstrokecolor{currentstroke}%
\pgfsetdash{}{0pt}%
\pgfsys@defobject{currentmarker}{\pgfqpoint{0.000000in}{-0.027778in}}{\pgfqpoint{0.000000in}{0.000000in}}{%
\pgfpathmoveto{\pgfqpoint{0.000000in}{0.000000in}}%
\pgfpathlineto{\pgfqpoint{0.000000in}{-0.027778in}}%
\pgfusepath{stroke,fill}%
}%
\begin{pgfscope}%
\pgfsys@transformshift{4.244353in}{0.440000in}%
\pgfsys@useobject{currentmarker}{}%
\end{pgfscope}%
\end{pgfscope}%
\begin{pgfscope}%
\pgfsetbuttcap%
\pgfsetroundjoin%
\definecolor{currentfill}{rgb}{0.000000,0.000000,0.000000}%
\pgfsetfillcolor{currentfill}%
\pgfsetlinewidth{0.602250pt}%
\definecolor{currentstroke}{rgb}{0.000000,0.000000,0.000000}%
\pgfsetstrokecolor{currentstroke}%
\pgfsetdash{}{0pt}%
\pgfsys@defobject{currentmarker}{\pgfqpoint{0.000000in}{-0.027778in}}{\pgfqpoint{0.000000in}{0.000000in}}{%
\pgfpathmoveto{\pgfqpoint{0.000000in}{0.000000in}}%
\pgfpathlineto{\pgfqpoint{0.000000in}{-0.027778in}}%
\pgfusepath{stroke,fill}%
}%
\begin{pgfscope}%
\pgfsys@transformshift{4.374496in}{0.440000in}%
\pgfsys@useobject{currentmarker}{}%
\end{pgfscope}%
\end{pgfscope}%
\begin{pgfscope}%
\pgfsetbuttcap%
\pgfsetroundjoin%
\definecolor{currentfill}{rgb}{0.000000,0.000000,0.000000}%
\pgfsetfillcolor{currentfill}%
\pgfsetlinewidth{0.602250pt}%
\definecolor{currentstroke}{rgb}{0.000000,0.000000,0.000000}%
\pgfsetstrokecolor{currentstroke}%
\pgfsetdash{}{0pt}%
\pgfsys@defobject{currentmarker}{\pgfqpoint{0.000000in}{-0.027778in}}{\pgfqpoint{0.000000in}{0.000000in}}{%
\pgfpathmoveto{\pgfqpoint{0.000000in}{0.000000in}}%
\pgfpathlineto{\pgfqpoint{0.000000in}{-0.027778in}}%
\pgfusepath{stroke,fill}%
}%
\begin{pgfscope}%
\pgfsys@transformshift{4.480829in}{0.440000in}%
\pgfsys@useobject{currentmarker}{}%
\end{pgfscope}%
\end{pgfscope}%
\begin{pgfscope}%
\pgfsetbuttcap%
\pgfsetroundjoin%
\definecolor{currentfill}{rgb}{0.000000,0.000000,0.000000}%
\pgfsetfillcolor{currentfill}%
\pgfsetlinewidth{0.602250pt}%
\definecolor{currentstroke}{rgb}{0.000000,0.000000,0.000000}%
\pgfsetstrokecolor{currentstroke}%
\pgfsetdash{}{0pt}%
\pgfsys@defobject{currentmarker}{\pgfqpoint{0.000000in}{-0.027778in}}{\pgfqpoint{0.000000in}{0.000000in}}{%
\pgfpathmoveto{\pgfqpoint{0.000000in}{0.000000in}}%
\pgfpathlineto{\pgfqpoint{0.000000in}{-0.027778in}}%
\pgfusepath{stroke,fill}%
}%
\begin{pgfscope}%
\pgfsys@transformshift{4.570734in}{0.440000in}%
\pgfsys@useobject{currentmarker}{}%
\end{pgfscope}%
\end{pgfscope}%
\begin{pgfscope}%
\pgfsetbuttcap%
\pgfsetroundjoin%
\definecolor{currentfill}{rgb}{0.000000,0.000000,0.000000}%
\pgfsetfillcolor{currentfill}%
\pgfsetlinewidth{0.602250pt}%
\definecolor{currentstroke}{rgb}{0.000000,0.000000,0.000000}%
\pgfsetstrokecolor{currentstroke}%
\pgfsetdash{}{0pt}%
\pgfsys@defobject{currentmarker}{\pgfqpoint{0.000000in}{-0.027778in}}{\pgfqpoint{0.000000in}{0.000000in}}{%
\pgfpathmoveto{\pgfqpoint{0.000000in}{0.000000in}}%
\pgfpathlineto{\pgfqpoint{0.000000in}{-0.027778in}}%
\pgfusepath{stroke,fill}%
}%
\begin{pgfscope}%
\pgfsys@transformshift{4.648612in}{0.440000in}%
\pgfsys@useobject{currentmarker}{}%
\end{pgfscope}%
\end{pgfscope}%
\begin{pgfscope}%
\pgfsetbuttcap%
\pgfsetroundjoin%
\definecolor{currentfill}{rgb}{0.000000,0.000000,0.000000}%
\pgfsetfillcolor{currentfill}%
\pgfsetlinewidth{0.602250pt}%
\definecolor{currentstroke}{rgb}{0.000000,0.000000,0.000000}%
\pgfsetstrokecolor{currentstroke}%
\pgfsetdash{}{0pt}%
\pgfsys@defobject{currentmarker}{\pgfqpoint{0.000000in}{-0.027778in}}{\pgfqpoint{0.000000in}{0.000000in}}{%
\pgfpathmoveto{\pgfqpoint{0.000000in}{0.000000in}}%
\pgfpathlineto{\pgfqpoint{0.000000in}{-0.027778in}}%
\pgfusepath{stroke,fill}%
}%
\begin{pgfscope}%
\pgfsys@transformshift{4.717306in}{0.440000in}%
\pgfsys@useobject{currentmarker}{}%
\end{pgfscope}%
\end{pgfscope}%
\begin{pgfscope}%
\pgfsetbuttcap%
\pgfsetroundjoin%
\definecolor{currentfill}{rgb}{0.000000,0.000000,0.000000}%
\pgfsetfillcolor{currentfill}%
\pgfsetlinewidth{0.602250pt}%
\definecolor{currentstroke}{rgb}{0.000000,0.000000,0.000000}%
\pgfsetstrokecolor{currentstroke}%
\pgfsetdash{}{0pt}%
\pgfsys@defobject{currentmarker}{\pgfqpoint{0.000000in}{-0.027778in}}{\pgfqpoint{0.000000in}{0.000000in}}{%
\pgfpathmoveto{\pgfqpoint{0.000000in}{0.000000in}}%
\pgfpathlineto{\pgfqpoint{0.000000in}{-0.027778in}}%
\pgfusepath{stroke,fill}%
}%
\begin{pgfscope}%
\pgfsys@transformshift{5.183013in}{0.440000in}%
\pgfsys@useobject{currentmarker}{}%
\end{pgfscope}%
\end{pgfscope}%
\begin{pgfscope}%
\definecolor{textcolor}{rgb}{0.000000,0.000000,0.000000}%
\pgfsetstrokecolor{textcolor}%
\pgfsetfillcolor{textcolor}%
\pgftext[x=3.075000in,y=0.152809in,,top]{\color{textcolor}\sffamily\fontsize{10.000000}{12.000000}\selectfont \(\displaystyle N\)}%
\end{pgfscope}%
\begin{pgfscope}%
\pgfsetbuttcap%
\pgfsetroundjoin%
\definecolor{currentfill}{rgb}{0.000000,0.000000,0.000000}%
\pgfsetfillcolor{currentfill}%
\pgfsetlinewidth{0.803000pt}%
\definecolor{currentstroke}{rgb}{0.000000,0.000000,0.000000}%
\pgfsetstrokecolor{currentstroke}%
\pgfsetdash{}{0pt}%
\pgfsys@defobject{currentmarker}{\pgfqpoint{-0.048611in}{0.000000in}}{\pgfqpoint{0.000000in}{0.000000in}}{%
\pgfpathmoveto{\pgfqpoint{0.000000in}{0.000000in}}%
\pgfpathlineto{\pgfqpoint{-0.048611in}{0.000000in}}%
\pgfusepath{stroke,fill}%
}%
\begin{pgfscope}%
\pgfsys@transformshift{0.750000in}{0.773468in}%
\pgfsys@useobject{currentmarker}{}%
\end{pgfscope}%
\end{pgfscope}%
\begin{pgfscope}%
\definecolor{textcolor}{rgb}{0.000000,0.000000,0.000000}%
\pgfsetstrokecolor{textcolor}%
\pgfsetfillcolor{textcolor}%
\pgftext[x=0.364775in,y=0.720707in,left,base]{\color{textcolor}\sffamily\fontsize{10.000000}{12.000000}\selectfont \(\displaystyle {10^{-8}}\)}%
\end{pgfscope}%
\begin{pgfscope}%
\pgfsetbuttcap%
\pgfsetroundjoin%
\definecolor{currentfill}{rgb}{0.000000,0.000000,0.000000}%
\pgfsetfillcolor{currentfill}%
\pgfsetlinewidth{0.803000pt}%
\definecolor{currentstroke}{rgb}{0.000000,0.000000,0.000000}%
\pgfsetstrokecolor{currentstroke}%
\pgfsetdash{}{0pt}%
\pgfsys@defobject{currentmarker}{\pgfqpoint{-0.048611in}{0.000000in}}{\pgfqpoint{0.000000in}{0.000000in}}{%
\pgfpathmoveto{\pgfqpoint{0.000000in}{0.000000in}}%
\pgfpathlineto{\pgfqpoint{-0.048611in}{0.000000in}}%
\pgfusepath{stroke,fill}%
}%
\begin{pgfscope}%
\pgfsys@transformshift{0.750000in}{1.129449in}%
\pgfsys@useobject{currentmarker}{}%
\end{pgfscope}%
\end{pgfscope}%
\begin{pgfscope}%
\definecolor{textcolor}{rgb}{0.000000,0.000000,0.000000}%
\pgfsetstrokecolor{textcolor}%
\pgfsetfillcolor{textcolor}%
\pgftext[x=0.364775in,y=1.076688in,left,base]{\color{textcolor}\sffamily\fontsize{10.000000}{12.000000}\selectfont \(\displaystyle {10^{-7}}\)}%
\end{pgfscope}%
\begin{pgfscope}%
\pgfsetbuttcap%
\pgfsetroundjoin%
\definecolor{currentfill}{rgb}{0.000000,0.000000,0.000000}%
\pgfsetfillcolor{currentfill}%
\pgfsetlinewidth{0.803000pt}%
\definecolor{currentstroke}{rgb}{0.000000,0.000000,0.000000}%
\pgfsetstrokecolor{currentstroke}%
\pgfsetdash{}{0pt}%
\pgfsys@defobject{currentmarker}{\pgfqpoint{-0.048611in}{0.000000in}}{\pgfqpoint{0.000000in}{0.000000in}}{%
\pgfpathmoveto{\pgfqpoint{0.000000in}{0.000000in}}%
\pgfpathlineto{\pgfqpoint{-0.048611in}{0.000000in}}%
\pgfusepath{stroke,fill}%
}%
\begin{pgfscope}%
\pgfsys@transformshift{0.750000in}{1.485431in}%
\pgfsys@useobject{currentmarker}{}%
\end{pgfscope}%
\end{pgfscope}%
\begin{pgfscope}%
\definecolor{textcolor}{rgb}{0.000000,0.000000,0.000000}%
\pgfsetstrokecolor{textcolor}%
\pgfsetfillcolor{textcolor}%
\pgftext[x=0.364775in,y=1.432669in,left,base]{\color{textcolor}\sffamily\fontsize{10.000000}{12.000000}\selectfont \(\displaystyle {10^{-6}}\)}%
\end{pgfscope}%
\begin{pgfscope}%
\pgfsetbuttcap%
\pgfsetroundjoin%
\definecolor{currentfill}{rgb}{0.000000,0.000000,0.000000}%
\pgfsetfillcolor{currentfill}%
\pgfsetlinewidth{0.803000pt}%
\definecolor{currentstroke}{rgb}{0.000000,0.000000,0.000000}%
\pgfsetstrokecolor{currentstroke}%
\pgfsetdash{}{0pt}%
\pgfsys@defobject{currentmarker}{\pgfqpoint{-0.048611in}{0.000000in}}{\pgfqpoint{0.000000in}{0.000000in}}{%
\pgfpathmoveto{\pgfqpoint{0.000000in}{0.000000in}}%
\pgfpathlineto{\pgfqpoint{-0.048611in}{0.000000in}}%
\pgfusepath{stroke,fill}%
}%
\begin{pgfscope}%
\pgfsys@transformshift{0.750000in}{1.841412in}%
\pgfsys@useobject{currentmarker}{}%
\end{pgfscope}%
\end{pgfscope}%
\begin{pgfscope}%
\definecolor{textcolor}{rgb}{0.000000,0.000000,0.000000}%
\pgfsetstrokecolor{textcolor}%
\pgfsetfillcolor{textcolor}%
\pgftext[x=0.364775in,y=1.788651in,left,base]{\color{textcolor}\sffamily\fontsize{10.000000}{12.000000}\selectfont \(\displaystyle {10^{-5}}\)}%
\end{pgfscope}%
\begin{pgfscope}%
\pgfsetbuttcap%
\pgfsetroundjoin%
\definecolor{currentfill}{rgb}{0.000000,0.000000,0.000000}%
\pgfsetfillcolor{currentfill}%
\pgfsetlinewidth{0.803000pt}%
\definecolor{currentstroke}{rgb}{0.000000,0.000000,0.000000}%
\pgfsetstrokecolor{currentstroke}%
\pgfsetdash{}{0pt}%
\pgfsys@defobject{currentmarker}{\pgfqpoint{-0.048611in}{0.000000in}}{\pgfqpoint{0.000000in}{0.000000in}}{%
\pgfpathmoveto{\pgfqpoint{0.000000in}{0.000000in}}%
\pgfpathlineto{\pgfqpoint{-0.048611in}{0.000000in}}%
\pgfusepath{stroke,fill}%
}%
\begin{pgfscope}%
\pgfsys@transformshift{0.750000in}{2.197393in}%
\pgfsys@useobject{currentmarker}{}%
\end{pgfscope}%
\end{pgfscope}%
\begin{pgfscope}%
\definecolor{textcolor}{rgb}{0.000000,0.000000,0.000000}%
\pgfsetstrokecolor{textcolor}%
\pgfsetfillcolor{textcolor}%
\pgftext[x=0.364775in,y=2.144632in,left,base]{\color{textcolor}\sffamily\fontsize{10.000000}{12.000000}\selectfont \(\displaystyle {10^{-4}}\)}%
\end{pgfscope}%
\begin{pgfscope}%
\pgfsetbuttcap%
\pgfsetroundjoin%
\definecolor{currentfill}{rgb}{0.000000,0.000000,0.000000}%
\pgfsetfillcolor{currentfill}%
\pgfsetlinewidth{0.803000pt}%
\definecolor{currentstroke}{rgb}{0.000000,0.000000,0.000000}%
\pgfsetstrokecolor{currentstroke}%
\pgfsetdash{}{0pt}%
\pgfsys@defobject{currentmarker}{\pgfqpoint{-0.048611in}{0.000000in}}{\pgfqpoint{0.000000in}{0.000000in}}{%
\pgfpathmoveto{\pgfqpoint{0.000000in}{0.000000in}}%
\pgfpathlineto{\pgfqpoint{-0.048611in}{0.000000in}}%
\pgfusepath{stroke,fill}%
}%
\begin{pgfscope}%
\pgfsys@transformshift{0.750000in}{2.553375in}%
\pgfsys@useobject{currentmarker}{}%
\end{pgfscope}%
\end{pgfscope}%
\begin{pgfscope}%
\definecolor{textcolor}{rgb}{0.000000,0.000000,0.000000}%
\pgfsetstrokecolor{textcolor}%
\pgfsetfillcolor{textcolor}%
\pgftext[x=0.364775in,y=2.500613in,left,base]{\color{textcolor}\sffamily\fontsize{10.000000}{12.000000}\selectfont \(\displaystyle {10^{-3}}\)}%
\end{pgfscope}%
\begin{pgfscope}%
\pgfsetbuttcap%
\pgfsetroundjoin%
\definecolor{currentfill}{rgb}{0.000000,0.000000,0.000000}%
\pgfsetfillcolor{currentfill}%
\pgfsetlinewidth{0.803000pt}%
\definecolor{currentstroke}{rgb}{0.000000,0.000000,0.000000}%
\pgfsetstrokecolor{currentstroke}%
\pgfsetdash{}{0pt}%
\pgfsys@defobject{currentmarker}{\pgfqpoint{-0.048611in}{0.000000in}}{\pgfqpoint{0.000000in}{0.000000in}}{%
\pgfpathmoveto{\pgfqpoint{0.000000in}{0.000000in}}%
\pgfpathlineto{\pgfqpoint{-0.048611in}{0.000000in}}%
\pgfusepath{stroke,fill}%
}%
\begin{pgfscope}%
\pgfsys@transformshift{0.750000in}{2.909356in}%
\pgfsys@useobject{currentmarker}{}%
\end{pgfscope}%
\end{pgfscope}%
\begin{pgfscope}%
\definecolor{textcolor}{rgb}{0.000000,0.000000,0.000000}%
\pgfsetstrokecolor{textcolor}%
\pgfsetfillcolor{textcolor}%
\pgftext[x=0.364775in,y=2.856594in,left,base]{\color{textcolor}\sffamily\fontsize{10.000000}{12.000000}\selectfont \(\displaystyle {10^{-2}}\)}%
\end{pgfscope}%
\begin{pgfscope}%
\pgfsetbuttcap%
\pgfsetroundjoin%
\definecolor{currentfill}{rgb}{0.000000,0.000000,0.000000}%
\pgfsetfillcolor{currentfill}%
\pgfsetlinewidth{0.803000pt}%
\definecolor{currentstroke}{rgb}{0.000000,0.000000,0.000000}%
\pgfsetstrokecolor{currentstroke}%
\pgfsetdash{}{0pt}%
\pgfsys@defobject{currentmarker}{\pgfqpoint{-0.048611in}{0.000000in}}{\pgfqpoint{0.000000in}{0.000000in}}{%
\pgfpathmoveto{\pgfqpoint{0.000000in}{0.000000in}}%
\pgfpathlineto{\pgfqpoint{-0.048611in}{0.000000in}}%
\pgfusepath{stroke,fill}%
}%
\begin{pgfscope}%
\pgfsys@transformshift{0.750000in}{3.265337in}%
\pgfsys@useobject{currentmarker}{}%
\end{pgfscope}%
\end{pgfscope}%
\begin{pgfscope}%
\definecolor{textcolor}{rgb}{0.000000,0.000000,0.000000}%
\pgfsetstrokecolor{textcolor}%
\pgfsetfillcolor{textcolor}%
\pgftext[x=0.364775in,y=3.212576in,left,base]{\color{textcolor}\sffamily\fontsize{10.000000}{12.000000}\selectfont \(\displaystyle {10^{-1}}\)}%
\end{pgfscope}%
\begin{pgfscope}%
\pgfsetbuttcap%
\pgfsetroundjoin%
\definecolor{currentfill}{rgb}{0.000000,0.000000,0.000000}%
\pgfsetfillcolor{currentfill}%
\pgfsetlinewidth{0.602250pt}%
\definecolor{currentstroke}{rgb}{0.000000,0.000000,0.000000}%
\pgfsetstrokecolor{currentstroke}%
\pgfsetdash{}{0pt}%
\pgfsys@defobject{currentmarker}{\pgfqpoint{-0.027778in}{0.000000in}}{\pgfqpoint{0.000000in}{0.000000in}}{%
\pgfpathmoveto{\pgfqpoint{0.000000in}{0.000000in}}%
\pgfpathlineto{\pgfqpoint{-0.027778in}{0.000000in}}%
\pgfusepath{stroke,fill}%
}%
\begin{pgfscope}%
\pgfsys@transformshift{0.750000in}{0.524648in}%
\pgfsys@useobject{currentmarker}{}%
\end{pgfscope}%
\end{pgfscope}%
\begin{pgfscope}%
\pgfsetbuttcap%
\pgfsetroundjoin%
\definecolor{currentfill}{rgb}{0.000000,0.000000,0.000000}%
\pgfsetfillcolor{currentfill}%
\pgfsetlinewidth{0.602250pt}%
\definecolor{currentstroke}{rgb}{0.000000,0.000000,0.000000}%
\pgfsetstrokecolor{currentstroke}%
\pgfsetdash{}{0pt}%
\pgfsys@defobject{currentmarker}{\pgfqpoint{-0.027778in}{0.000000in}}{\pgfqpoint{0.000000in}{0.000000in}}{%
\pgfpathmoveto{\pgfqpoint{0.000000in}{0.000000in}}%
\pgfpathlineto{\pgfqpoint{-0.027778in}{0.000000in}}%
\pgfusepath{stroke,fill}%
}%
\begin{pgfscope}%
\pgfsys@transformshift{0.750000in}{0.587333in}%
\pgfsys@useobject{currentmarker}{}%
\end{pgfscope}%
\end{pgfscope}%
\begin{pgfscope}%
\pgfsetbuttcap%
\pgfsetroundjoin%
\definecolor{currentfill}{rgb}{0.000000,0.000000,0.000000}%
\pgfsetfillcolor{currentfill}%
\pgfsetlinewidth{0.602250pt}%
\definecolor{currentstroke}{rgb}{0.000000,0.000000,0.000000}%
\pgfsetstrokecolor{currentstroke}%
\pgfsetdash{}{0pt}%
\pgfsys@defobject{currentmarker}{\pgfqpoint{-0.027778in}{0.000000in}}{\pgfqpoint{0.000000in}{0.000000in}}{%
\pgfpathmoveto{\pgfqpoint{0.000000in}{0.000000in}}%
\pgfpathlineto{\pgfqpoint{-0.027778in}{0.000000in}}%
\pgfusepath{stroke,fill}%
}%
\begin{pgfscope}%
\pgfsys@transformshift{0.750000in}{0.631809in}%
\pgfsys@useobject{currentmarker}{}%
\end{pgfscope}%
\end{pgfscope}%
\begin{pgfscope}%
\pgfsetbuttcap%
\pgfsetroundjoin%
\definecolor{currentfill}{rgb}{0.000000,0.000000,0.000000}%
\pgfsetfillcolor{currentfill}%
\pgfsetlinewidth{0.602250pt}%
\definecolor{currentstroke}{rgb}{0.000000,0.000000,0.000000}%
\pgfsetstrokecolor{currentstroke}%
\pgfsetdash{}{0pt}%
\pgfsys@defobject{currentmarker}{\pgfqpoint{-0.027778in}{0.000000in}}{\pgfqpoint{0.000000in}{0.000000in}}{%
\pgfpathmoveto{\pgfqpoint{0.000000in}{0.000000in}}%
\pgfpathlineto{\pgfqpoint{-0.027778in}{0.000000in}}%
\pgfusepath{stroke,fill}%
}%
\begin{pgfscope}%
\pgfsys@transformshift{0.750000in}{0.666307in}%
\pgfsys@useobject{currentmarker}{}%
\end{pgfscope}%
\end{pgfscope}%
\begin{pgfscope}%
\pgfsetbuttcap%
\pgfsetroundjoin%
\definecolor{currentfill}{rgb}{0.000000,0.000000,0.000000}%
\pgfsetfillcolor{currentfill}%
\pgfsetlinewidth{0.602250pt}%
\definecolor{currentstroke}{rgb}{0.000000,0.000000,0.000000}%
\pgfsetstrokecolor{currentstroke}%
\pgfsetdash{}{0pt}%
\pgfsys@defobject{currentmarker}{\pgfqpoint{-0.027778in}{0.000000in}}{\pgfqpoint{0.000000in}{0.000000in}}{%
\pgfpathmoveto{\pgfqpoint{0.000000in}{0.000000in}}%
\pgfpathlineto{\pgfqpoint{-0.027778in}{0.000000in}}%
\pgfusepath{stroke,fill}%
}%
\begin{pgfscope}%
\pgfsys@transformshift{0.750000in}{0.694494in}%
\pgfsys@useobject{currentmarker}{}%
\end{pgfscope}%
\end{pgfscope}%
\begin{pgfscope}%
\pgfsetbuttcap%
\pgfsetroundjoin%
\definecolor{currentfill}{rgb}{0.000000,0.000000,0.000000}%
\pgfsetfillcolor{currentfill}%
\pgfsetlinewidth{0.602250pt}%
\definecolor{currentstroke}{rgb}{0.000000,0.000000,0.000000}%
\pgfsetstrokecolor{currentstroke}%
\pgfsetdash{}{0pt}%
\pgfsys@defobject{currentmarker}{\pgfqpoint{-0.027778in}{0.000000in}}{\pgfqpoint{0.000000in}{0.000000in}}{%
\pgfpathmoveto{\pgfqpoint{0.000000in}{0.000000in}}%
\pgfpathlineto{\pgfqpoint{-0.027778in}{0.000000in}}%
\pgfusepath{stroke,fill}%
}%
\begin{pgfscope}%
\pgfsys@transformshift{0.750000in}{0.718326in}%
\pgfsys@useobject{currentmarker}{}%
\end{pgfscope}%
\end{pgfscope}%
\begin{pgfscope}%
\pgfsetbuttcap%
\pgfsetroundjoin%
\definecolor{currentfill}{rgb}{0.000000,0.000000,0.000000}%
\pgfsetfillcolor{currentfill}%
\pgfsetlinewidth{0.602250pt}%
\definecolor{currentstroke}{rgb}{0.000000,0.000000,0.000000}%
\pgfsetstrokecolor{currentstroke}%
\pgfsetdash{}{0pt}%
\pgfsys@defobject{currentmarker}{\pgfqpoint{-0.027778in}{0.000000in}}{\pgfqpoint{0.000000in}{0.000000in}}{%
\pgfpathmoveto{\pgfqpoint{0.000000in}{0.000000in}}%
\pgfpathlineto{\pgfqpoint{-0.027778in}{0.000000in}}%
\pgfusepath{stroke,fill}%
}%
\begin{pgfscope}%
\pgfsys@transformshift{0.750000in}{0.738970in}%
\pgfsys@useobject{currentmarker}{}%
\end{pgfscope}%
\end{pgfscope}%
\begin{pgfscope}%
\pgfsetbuttcap%
\pgfsetroundjoin%
\definecolor{currentfill}{rgb}{0.000000,0.000000,0.000000}%
\pgfsetfillcolor{currentfill}%
\pgfsetlinewidth{0.602250pt}%
\definecolor{currentstroke}{rgb}{0.000000,0.000000,0.000000}%
\pgfsetstrokecolor{currentstroke}%
\pgfsetdash{}{0pt}%
\pgfsys@defobject{currentmarker}{\pgfqpoint{-0.027778in}{0.000000in}}{\pgfqpoint{0.000000in}{0.000000in}}{%
\pgfpathmoveto{\pgfqpoint{0.000000in}{0.000000in}}%
\pgfpathlineto{\pgfqpoint{-0.027778in}{0.000000in}}%
\pgfusepath{stroke,fill}%
}%
\begin{pgfscope}%
\pgfsys@transformshift{0.750000in}{0.757179in}%
\pgfsys@useobject{currentmarker}{}%
\end{pgfscope}%
\end{pgfscope}%
\begin{pgfscope}%
\pgfsetbuttcap%
\pgfsetroundjoin%
\definecolor{currentfill}{rgb}{0.000000,0.000000,0.000000}%
\pgfsetfillcolor{currentfill}%
\pgfsetlinewidth{0.602250pt}%
\definecolor{currentstroke}{rgb}{0.000000,0.000000,0.000000}%
\pgfsetstrokecolor{currentstroke}%
\pgfsetdash{}{0pt}%
\pgfsys@defobject{currentmarker}{\pgfqpoint{-0.027778in}{0.000000in}}{\pgfqpoint{0.000000in}{0.000000in}}{%
\pgfpathmoveto{\pgfqpoint{0.000000in}{0.000000in}}%
\pgfpathlineto{\pgfqpoint{-0.027778in}{0.000000in}}%
\pgfusepath{stroke,fill}%
}%
\begin{pgfscope}%
\pgfsys@transformshift{0.750000in}{0.880629in}%
\pgfsys@useobject{currentmarker}{}%
\end{pgfscope}%
\end{pgfscope}%
\begin{pgfscope}%
\pgfsetbuttcap%
\pgfsetroundjoin%
\definecolor{currentfill}{rgb}{0.000000,0.000000,0.000000}%
\pgfsetfillcolor{currentfill}%
\pgfsetlinewidth{0.602250pt}%
\definecolor{currentstroke}{rgb}{0.000000,0.000000,0.000000}%
\pgfsetstrokecolor{currentstroke}%
\pgfsetdash{}{0pt}%
\pgfsys@defobject{currentmarker}{\pgfqpoint{-0.027778in}{0.000000in}}{\pgfqpoint{0.000000in}{0.000000in}}{%
\pgfpathmoveto{\pgfqpoint{0.000000in}{0.000000in}}%
\pgfpathlineto{\pgfqpoint{-0.027778in}{0.000000in}}%
\pgfusepath{stroke,fill}%
}%
\begin{pgfscope}%
\pgfsys@transformshift{0.750000in}{0.943314in}%
\pgfsys@useobject{currentmarker}{}%
\end{pgfscope}%
\end{pgfscope}%
\begin{pgfscope}%
\pgfsetbuttcap%
\pgfsetroundjoin%
\definecolor{currentfill}{rgb}{0.000000,0.000000,0.000000}%
\pgfsetfillcolor{currentfill}%
\pgfsetlinewidth{0.602250pt}%
\definecolor{currentstroke}{rgb}{0.000000,0.000000,0.000000}%
\pgfsetstrokecolor{currentstroke}%
\pgfsetdash{}{0pt}%
\pgfsys@defobject{currentmarker}{\pgfqpoint{-0.027778in}{0.000000in}}{\pgfqpoint{0.000000in}{0.000000in}}{%
\pgfpathmoveto{\pgfqpoint{0.000000in}{0.000000in}}%
\pgfpathlineto{\pgfqpoint{-0.027778in}{0.000000in}}%
\pgfusepath{stroke,fill}%
}%
\begin{pgfscope}%
\pgfsys@transformshift{0.750000in}{0.987790in}%
\pgfsys@useobject{currentmarker}{}%
\end{pgfscope}%
\end{pgfscope}%
\begin{pgfscope}%
\pgfsetbuttcap%
\pgfsetroundjoin%
\definecolor{currentfill}{rgb}{0.000000,0.000000,0.000000}%
\pgfsetfillcolor{currentfill}%
\pgfsetlinewidth{0.602250pt}%
\definecolor{currentstroke}{rgb}{0.000000,0.000000,0.000000}%
\pgfsetstrokecolor{currentstroke}%
\pgfsetdash{}{0pt}%
\pgfsys@defobject{currentmarker}{\pgfqpoint{-0.027778in}{0.000000in}}{\pgfqpoint{0.000000in}{0.000000in}}{%
\pgfpathmoveto{\pgfqpoint{0.000000in}{0.000000in}}%
\pgfpathlineto{\pgfqpoint{-0.027778in}{0.000000in}}%
\pgfusepath{stroke,fill}%
}%
\begin{pgfscope}%
\pgfsys@transformshift{0.750000in}{1.022288in}%
\pgfsys@useobject{currentmarker}{}%
\end{pgfscope}%
\end{pgfscope}%
\begin{pgfscope}%
\pgfsetbuttcap%
\pgfsetroundjoin%
\definecolor{currentfill}{rgb}{0.000000,0.000000,0.000000}%
\pgfsetfillcolor{currentfill}%
\pgfsetlinewidth{0.602250pt}%
\definecolor{currentstroke}{rgb}{0.000000,0.000000,0.000000}%
\pgfsetstrokecolor{currentstroke}%
\pgfsetdash{}{0pt}%
\pgfsys@defobject{currentmarker}{\pgfqpoint{-0.027778in}{0.000000in}}{\pgfqpoint{0.000000in}{0.000000in}}{%
\pgfpathmoveto{\pgfqpoint{0.000000in}{0.000000in}}%
\pgfpathlineto{\pgfqpoint{-0.027778in}{0.000000in}}%
\pgfusepath{stroke,fill}%
}%
\begin{pgfscope}%
\pgfsys@transformshift{0.750000in}{1.050475in}%
\pgfsys@useobject{currentmarker}{}%
\end{pgfscope}%
\end{pgfscope}%
\begin{pgfscope}%
\pgfsetbuttcap%
\pgfsetroundjoin%
\definecolor{currentfill}{rgb}{0.000000,0.000000,0.000000}%
\pgfsetfillcolor{currentfill}%
\pgfsetlinewidth{0.602250pt}%
\definecolor{currentstroke}{rgb}{0.000000,0.000000,0.000000}%
\pgfsetstrokecolor{currentstroke}%
\pgfsetdash{}{0pt}%
\pgfsys@defobject{currentmarker}{\pgfqpoint{-0.027778in}{0.000000in}}{\pgfqpoint{0.000000in}{0.000000in}}{%
\pgfpathmoveto{\pgfqpoint{0.000000in}{0.000000in}}%
\pgfpathlineto{\pgfqpoint{-0.027778in}{0.000000in}}%
\pgfusepath{stroke,fill}%
}%
\begin{pgfscope}%
\pgfsys@transformshift{0.750000in}{1.074307in}%
\pgfsys@useobject{currentmarker}{}%
\end{pgfscope}%
\end{pgfscope}%
\begin{pgfscope}%
\pgfsetbuttcap%
\pgfsetroundjoin%
\definecolor{currentfill}{rgb}{0.000000,0.000000,0.000000}%
\pgfsetfillcolor{currentfill}%
\pgfsetlinewidth{0.602250pt}%
\definecolor{currentstroke}{rgb}{0.000000,0.000000,0.000000}%
\pgfsetstrokecolor{currentstroke}%
\pgfsetdash{}{0pt}%
\pgfsys@defobject{currentmarker}{\pgfqpoint{-0.027778in}{0.000000in}}{\pgfqpoint{0.000000in}{0.000000in}}{%
\pgfpathmoveto{\pgfqpoint{0.000000in}{0.000000in}}%
\pgfpathlineto{\pgfqpoint{-0.027778in}{0.000000in}}%
\pgfusepath{stroke,fill}%
}%
\begin{pgfscope}%
\pgfsys@transformshift{0.750000in}{1.094951in}%
\pgfsys@useobject{currentmarker}{}%
\end{pgfscope}%
\end{pgfscope}%
\begin{pgfscope}%
\pgfsetbuttcap%
\pgfsetroundjoin%
\definecolor{currentfill}{rgb}{0.000000,0.000000,0.000000}%
\pgfsetfillcolor{currentfill}%
\pgfsetlinewidth{0.602250pt}%
\definecolor{currentstroke}{rgb}{0.000000,0.000000,0.000000}%
\pgfsetstrokecolor{currentstroke}%
\pgfsetdash{}{0pt}%
\pgfsys@defobject{currentmarker}{\pgfqpoint{-0.027778in}{0.000000in}}{\pgfqpoint{0.000000in}{0.000000in}}{%
\pgfpathmoveto{\pgfqpoint{0.000000in}{0.000000in}}%
\pgfpathlineto{\pgfqpoint{-0.027778in}{0.000000in}}%
\pgfusepath{stroke,fill}%
}%
\begin{pgfscope}%
\pgfsys@transformshift{0.750000in}{1.113161in}%
\pgfsys@useobject{currentmarker}{}%
\end{pgfscope}%
\end{pgfscope}%
\begin{pgfscope}%
\pgfsetbuttcap%
\pgfsetroundjoin%
\definecolor{currentfill}{rgb}{0.000000,0.000000,0.000000}%
\pgfsetfillcolor{currentfill}%
\pgfsetlinewidth{0.602250pt}%
\definecolor{currentstroke}{rgb}{0.000000,0.000000,0.000000}%
\pgfsetstrokecolor{currentstroke}%
\pgfsetdash{}{0pt}%
\pgfsys@defobject{currentmarker}{\pgfqpoint{-0.027778in}{0.000000in}}{\pgfqpoint{0.000000in}{0.000000in}}{%
\pgfpathmoveto{\pgfqpoint{0.000000in}{0.000000in}}%
\pgfpathlineto{\pgfqpoint{-0.027778in}{0.000000in}}%
\pgfusepath{stroke,fill}%
}%
\begin{pgfscope}%
\pgfsys@transformshift{0.750000in}{1.236611in}%
\pgfsys@useobject{currentmarker}{}%
\end{pgfscope}%
\end{pgfscope}%
\begin{pgfscope}%
\pgfsetbuttcap%
\pgfsetroundjoin%
\definecolor{currentfill}{rgb}{0.000000,0.000000,0.000000}%
\pgfsetfillcolor{currentfill}%
\pgfsetlinewidth{0.602250pt}%
\definecolor{currentstroke}{rgb}{0.000000,0.000000,0.000000}%
\pgfsetstrokecolor{currentstroke}%
\pgfsetdash{}{0pt}%
\pgfsys@defobject{currentmarker}{\pgfqpoint{-0.027778in}{0.000000in}}{\pgfqpoint{0.000000in}{0.000000in}}{%
\pgfpathmoveto{\pgfqpoint{0.000000in}{0.000000in}}%
\pgfpathlineto{\pgfqpoint{-0.027778in}{0.000000in}}%
\pgfusepath{stroke,fill}%
}%
\begin{pgfscope}%
\pgfsys@transformshift{0.750000in}{1.299296in}%
\pgfsys@useobject{currentmarker}{}%
\end{pgfscope}%
\end{pgfscope}%
\begin{pgfscope}%
\pgfsetbuttcap%
\pgfsetroundjoin%
\definecolor{currentfill}{rgb}{0.000000,0.000000,0.000000}%
\pgfsetfillcolor{currentfill}%
\pgfsetlinewidth{0.602250pt}%
\definecolor{currentstroke}{rgb}{0.000000,0.000000,0.000000}%
\pgfsetstrokecolor{currentstroke}%
\pgfsetdash{}{0pt}%
\pgfsys@defobject{currentmarker}{\pgfqpoint{-0.027778in}{0.000000in}}{\pgfqpoint{0.000000in}{0.000000in}}{%
\pgfpathmoveto{\pgfqpoint{0.000000in}{0.000000in}}%
\pgfpathlineto{\pgfqpoint{-0.027778in}{0.000000in}}%
\pgfusepath{stroke,fill}%
}%
\begin{pgfscope}%
\pgfsys@transformshift{0.750000in}{1.343772in}%
\pgfsys@useobject{currentmarker}{}%
\end{pgfscope}%
\end{pgfscope}%
\begin{pgfscope}%
\pgfsetbuttcap%
\pgfsetroundjoin%
\definecolor{currentfill}{rgb}{0.000000,0.000000,0.000000}%
\pgfsetfillcolor{currentfill}%
\pgfsetlinewidth{0.602250pt}%
\definecolor{currentstroke}{rgb}{0.000000,0.000000,0.000000}%
\pgfsetstrokecolor{currentstroke}%
\pgfsetdash{}{0pt}%
\pgfsys@defobject{currentmarker}{\pgfqpoint{-0.027778in}{0.000000in}}{\pgfqpoint{0.000000in}{0.000000in}}{%
\pgfpathmoveto{\pgfqpoint{0.000000in}{0.000000in}}%
\pgfpathlineto{\pgfqpoint{-0.027778in}{0.000000in}}%
\pgfusepath{stroke,fill}%
}%
\begin{pgfscope}%
\pgfsys@transformshift{0.750000in}{1.378270in}%
\pgfsys@useobject{currentmarker}{}%
\end{pgfscope}%
\end{pgfscope}%
\begin{pgfscope}%
\pgfsetbuttcap%
\pgfsetroundjoin%
\definecolor{currentfill}{rgb}{0.000000,0.000000,0.000000}%
\pgfsetfillcolor{currentfill}%
\pgfsetlinewidth{0.602250pt}%
\definecolor{currentstroke}{rgb}{0.000000,0.000000,0.000000}%
\pgfsetstrokecolor{currentstroke}%
\pgfsetdash{}{0pt}%
\pgfsys@defobject{currentmarker}{\pgfqpoint{-0.027778in}{0.000000in}}{\pgfqpoint{0.000000in}{0.000000in}}{%
\pgfpathmoveto{\pgfqpoint{0.000000in}{0.000000in}}%
\pgfpathlineto{\pgfqpoint{-0.027778in}{0.000000in}}%
\pgfusepath{stroke,fill}%
}%
\begin{pgfscope}%
\pgfsys@transformshift{0.750000in}{1.406457in}%
\pgfsys@useobject{currentmarker}{}%
\end{pgfscope}%
\end{pgfscope}%
\begin{pgfscope}%
\pgfsetbuttcap%
\pgfsetroundjoin%
\definecolor{currentfill}{rgb}{0.000000,0.000000,0.000000}%
\pgfsetfillcolor{currentfill}%
\pgfsetlinewidth{0.602250pt}%
\definecolor{currentstroke}{rgb}{0.000000,0.000000,0.000000}%
\pgfsetstrokecolor{currentstroke}%
\pgfsetdash{}{0pt}%
\pgfsys@defobject{currentmarker}{\pgfqpoint{-0.027778in}{0.000000in}}{\pgfqpoint{0.000000in}{0.000000in}}{%
\pgfpathmoveto{\pgfqpoint{0.000000in}{0.000000in}}%
\pgfpathlineto{\pgfqpoint{-0.027778in}{0.000000in}}%
\pgfusepath{stroke,fill}%
}%
\begin{pgfscope}%
\pgfsys@transformshift{0.750000in}{1.430289in}%
\pgfsys@useobject{currentmarker}{}%
\end{pgfscope}%
\end{pgfscope}%
\begin{pgfscope}%
\pgfsetbuttcap%
\pgfsetroundjoin%
\definecolor{currentfill}{rgb}{0.000000,0.000000,0.000000}%
\pgfsetfillcolor{currentfill}%
\pgfsetlinewidth{0.602250pt}%
\definecolor{currentstroke}{rgb}{0.000000,0.000000,0.000000}%
\pgfsetstrokecolor{currentstroke}%
\pgfsetdash{}{0pt}%
\pgfsys@defobject{currentmarker}{\pgfqpoint{-0.027778in}{0.000000in}}{\pgfqpoint{0.000000in}{0.000000in}}{%
\pgfpathmoveto{\pgfqpoint{0.000000in}{0.000000in}}%
\pgfpathlineto{\pgfqpoint{-0.027778in}{0.000000in}}%
\pgfusepath{stroke,fill}%
}%
\begin{pgfscope}%
\pgfsys@transformshift{0.750000in}{1.450933in}%
\pgfsys@useobject{currentmarker}{}%
\end{pgfscope}%
\end{pgfscope}%
\begin{pgfscope}%
\pgfsetbuttcap%
\pgfsetroundjoin%
\definecolor{currentfill}{rgb}{0.000000,0.000000,0.000000}%
\pgfsetfillcolor{currentfill}%
\pgfsetlinewidth{0.602250pt}%
\definecolor{currentstroke}{rgb}{0.000000,0.000000,0.000000}%
\pgfsetstrokecolor{currentstroke}%
\pgfsetdash{}{0pt}%
\pgfsys@defobject{currentmarker}{\pgfqpoint{-0.027778in}{0.000000in}}{\pgfqpoint{0.000000in}{0.000000in}}{%
\pgfpathmoveto{\pgfqpoint{0.000000in}{0.000000in}}%
\pgfpathlineto{\pgfqpoint{-0.027778in}{0.000000in}}%
\pgfusepath{stroke,fill}%
}%
\begin{pgfscope}%
\pgfsys@transformshift{0.750000in}{1.469142in}%
\pgfsys@useobject{currentmarker}{}%
\end{pgfscope}%
\end{pgfscope}%
\begin{pgfscope}%
\pgfsetbuttcap%
\pgfsetroundjoin%
\definecolor{currentfill}{rgb}{0.000000,0.000000,0.000000}%
\pgfsetfillcolor{currentfill}%
\pgfsetlinewidth{0.602250pt}%
\definecolor{currentstroke}{rgb}{0.000000,0.000000,0.000000}%
\pgfsetstrokecolor{currentstroke}%
\pgfsetdash{}{0pt}%
\pgfsys@defobject{currentmarker}{\pgfqpoint{-0.027778in}{0.000000in}}{\pgfqpoint{0.000000in}{0.000000in}}{%
\pgfpathmoveto{\pgfqpoint{0.000000in}{0.000000in}}%
\pgfpathlineto{\pgfqpoint{-0.027778in}{0.000000in}}%
\pgfusepath{stroke,fill}%
}%
\begin{pgfscope}%
\pgfsys@transformshift{0.750000in}{1.592592in}%
\pgfsys@useobject{currentmarker}{}%
\end{pgfscope}%
\end{pgfscope}%
\begin{pgfscope}%
\pgfsetbuttcap%
\pgfsetroundjoin%
\definecolor{currentfill}{rgb}{0.000000,0.000000,0.000000}%
\pgfsetfillcolor{currentfill}%
\pgfsetlinewidth{0.602250pt}%
\definecolor{currentstroke}{rgb}{0.000000,0.000000,0.000000}%
\pgfsetstrokecolor{currentstroke}%
\pgfsetdash{}{0pt}%
\pgfsys@defobject{currentmarker}{\pgfqpoint{-0.027778in}{0.000000in}}{\pgfqpoint{0.000000in}{0.000000in}}{%
\pgfpathmoveto{\pgfqpoint{0.000000in}{0.000000in}}%
\pgfpathlineto{\pgfqpoint{-0.027778in}{0.000000in}}%
\pgfusepath{stroke,fill}%
}%
\begin{pgfscope}%
\pgfsys@transformshift{0.750000in}{1.655277in}%
\pgfsys@useobject{currentmarker}{}%
\end{pgfscope}%
\end{pgfscope}%
\begin{pgfscope}%
\pgfsetbuttcap%
\pgfsetroundjoin%
\definecolor{currentfill}{rgb}{0.000000,0.000000,0.000000}%
\pgfsetfillcolor{currentfill}%
\pgfsetlinewidth{0.602250pt}%
\definecolor{currentstroke}{rgb}{0.000000,0.000000,0.000000}%
\pgfsetstrokecolor{currentstroke}%
\pgfsetdash{}{0pt}%
\pgfsys@defobject{currentmarker}{\pgfqpoint{-0.027778in}{0.000000in}}{\pgfqpoint{0.000000in}{0.000000in}}{%
\pgfpathmoveto{\pgfqpoint{0.000000in}{0.000000in}}%
\pgfpathlineto{\pgfqpoint{-0.027778in}{0.000000in}}%
\pgfusepath{stroke,fill}%
}%
\begin{pgfscope}%
\pgfsys@transformshift{0.750000in}{1.699753in}%
\pgfsys@useobject{currentmarker}{}%
\end{pgfscope}%
\end{pgfscope}%
\begin{pgfscope}%
\pgfsetbuttcap%
\pgfsetroundjoin%
\definecolor{currentfill}{rgb}{0.000000,0.000000,0.000000}%
\pgfsetfillcolor{currentfill}%
\pgfsetlinewidth{0.602250pt}%
\definecolor{currentstroke}{rgb}{0.000000,0.000000,0.000000}%
\pgfsetstrokecolor{currentstroke}%
\pgfsetdash{}{0pt}%
\pgfsys@defobject{currentmarker}{\pgfqpoint{-0.027778in}{0.000000in}}{\pgfqpoint{0.000000in}{0.000000in}}{%
\pgfpathmoveto{\pgfqpoint{0.000000in}{0.000000in}}%
\pgfpathlineto{\pgfqpoint{-0.027778in}{0.000000in}}%
\pgfusepath{stroke,fill}%
}%
\begin{pgfscope}%
\pgfsys@transformshift{0.750000in}{1.734251in}%
\pgfsys@useobject{currentmarker}{}%
\end{pgfscope}%
\end{pgfscope}%
\begin{pgfscope}%
\pgfsetbuttcap%
\pgfsetroundjoin%
\definecolor{currentfill}{rgb}{0.000000,0.000000,0.000000}%
\pgfsetfillcolor{currentfill}%
\pgfsetlinewidth{0.602250pt}%
\definecolor{currentstroke}{rgb}{0.000000,0.000000,0.000000}%
\pgfsetstrokecolor{currentstroke}%
\pgfsetdash{}{0pt}%
\pgfsys@defobject{currentmarker}{\pgfqpoint{-0.027778in}{0.000000in}}{\pgfqpoint{0.000000in}{0.000000in}}{%
\pgfpathmoveto{\pgfqpoint{0.000000in}{0.000000in}}%
\pgfpathlineto{\pgfqpoint{-0.027778in}{0.000000in}}%
\pgfusepath{stroke,fill}%
}%
\begin{pgfscope}%
\pgfsys@transformshift{0.750000in}{1.762438in}%
\pgfsys@useobject{currentmarker}{}%
\end{pgfscope}%
\end{pgfscope}%
\begin{pgfscope}%
\pgfsetbuttcap%
\pgfsetroundjoin%
\definecolor{currentfill}{rgb}{0.000000,0.000000,0.000000}%
\pgfsetfillcolor{currentfill}%
\pgfsetlinewidth{0.602250pt}%
\definecolor{currentstroke}{rgb}{0.000000,0.000000,0.000000}%
\pgfsetstrokecolor{currentstroke}%
\pgfsetdash{}{0pt}%
\pgfsys@defobject{currentmarker}{\pgfqpoint{-0.027778in}{0.000000in}}{\pgfqpoint{0.000000in}{0.000000in}}{%
\pgfpathmoveto{\pgfqpoint{0.000000in}{0.000000in}}%
\pgfpathlineto{\pgfqpoint{-0.027778in}{0.000000in}}%
\pgfusepath{stroke,fill}%
}%
\begin{pgfscope}%
\pgfsys@transformshift{0.750000in}{1.786270in}%
\pgfsys@useobject{currentmarker}{}%
\end{pgfscope}%
\end{pgfscope}%
\begin{pgfscope}%
\pgfsetbuttcap%
\pgfsetroundjoin%
\definecolor{currentfill}{rgb}{0.000000,0.000000,0.000000}%
\pgfsetfillcolor{currentfill}%
\pgfsetlinewidth{0.602250pt}%
\definecolor{currentstroke}{rgb}{0.000000,0.000000,0.000000}%
\pgfsetstrokecolor{currentstroke}%
\pgfsetdash{}{0pt}%
\pgfsys@defobject{currentmarker}{\pgfqpoint{-0.027778in}{0.000000in}}{\pgfqpoint{0.000000in}{0.000000in}}{%
\pgfpathmoveto{\pgfqpoint{0.000000in}{0.000000in}}%
\pgfpathlineto{\pgfqpoint{-0.027778in}{0.000000in}}%
\pgfusepath{stroke,fill}%
}%
\begin{pgfscope}%
\pgfsys@transformshift{0.750000in}{1.806914in}%
\pgfsys@useobject{currentmarker}{}%
\end{pgfscope}%
\end{pgfscope}%
\begin{pgfscope}%
\pgfsetbuttcap%
\pgfsetroundjoin%
\definecolor{currentfill}{rgb}{0.000000,0.000000,0.000000}%
\pgfsetfillcolor{currentfill}%
\pgfsetlinewidth{0.602250pt}%
\definecolor{currentstroke}{rgb}{0.000000,0.000000,0.000000}%
\pgfsetstrokecolor{currentstroke}%
\pgfsetdash{}{0pt}%
\pgfsys@defobject{currentmarker}{\pgfqpoint{-0.027778in}{0.000000in}}{\pgfqpoint{0.000000in}{0.000000in}}{%
\pgfpathmoveto{\pgfqpoint{0.000000in}{0.000000in}}%
\pgfpathlineto{\pgfqpoint{-0.027778in}{0.000000in}}%
\pgfusepath{stroke,fill}%
}%
\begin{pgfscope}%
\pgfsys@transformshift{0.750000in}{1.825123in}%
\pgfsys@useobject{currentmarker}{}%
\end{pgfscope}%
\end{pgfscope}%
\begin{pgfscope}%
\pgfsetbuttcap%
\pgfsetroundjoin%
\definecolor{currentfill}{rgb}{0.000000,0.000000,0.000000}%
\pgfsetfillcolor{currentfill}%
\pgfsetlinewidth{0.602250pt}%
\definecolor{currentstroke}{rgb}{0.000000,0.000000,0.000000}%
\pgfsetstrokecolor{currentstroke}%
\pgfsetdash{}{0pt}%
\pgfsys@defobject{currentmarker}{\pgfqpoint{-0.027778in}{0.000000in}}{\pgfqpoint{0.000000in}{0.000000in}}{%
\pgfpathmoveto{\pgfqpoint{0.000000in}{0.000000in}}%
\pgfpathlineto{\pgfqpoint{-0.027778in}{0.000000in}}%
\pgfusepath{stroke,fill}%
}%
\begin{pgfscope}%
\pgfsys@transformshift{0.750000in}{1.948573in}%
\pgfsys@useobject{currentmarker}{}%
\end{pgfscope}%
\end{pgfscope}%
\begin{pgfscope}%
\pgfsetbuttcap%
\pgfsetroundjoin%
\definecolor{currentfill}{rgb}{0.000000,0.000000,0.000000}%
\pgfsetfillcolor{currentfill}%
\pgfsetlinewidth{0.602250pt}%
\definecolor{currentstroke}{rgb}{0.000000,0.000000,0.000000}%
\pgfsetstrokecolor{currentstroke}%
\pgfsetdash{}{0pt}%
\pgfsys@defobject{currentmarker}{\pgfqpoint{-0.027778in}{0.000000in}}{\pgfqpoint{0.000000in}{0.000000in}}{%
\pgfpathmoveto{\pgfqpoint{0.000000in}{0.000000in}}%
\pgfpathlineto{\pgfqpoint{-0.027778in}{0.000000in}}%
\pgfusepath{stroke,fill}%
}%
\begin{pgfscope}%
\pgfsys@transformshift{0.750000in}{2.011258in}%
\pgfsys@useobject{currentmarker}{}%
\end{pgfscope}%
\end{pgfscope}%
\begin{pgfscope}%
\pgfsetbuttcap%
\pgfsetroundjoin%
\definecolor{currentfill}{rgb}{0.000000,0.000000,0.000000}%
\pgfsetfillcolor{currentfill}%
\pgfsetlinewidth{0.602250pt}%
\definecolor{currentstroke}{rgb}{0.000000,0.000000,0.000000}%
\pgfsetstrokecolor{currentstroke}%
\pgfsetdash{}{0pt}%
\pgfsys@defobject{currentmarker}{\pgfqpoint{-0.027778in}{0.000000in}}{\pgfqpoint{0.000000in}{0.000000in}}{%
\pgfpathmoveto{\pgfqpoint{0.000000in}{0.000000in}}%
\pgfpathlineto{\pgfqpoint{-0.027778in}{0.000000in}}%
\pgfusepath{stroke,fill}%
}%
\begin{pgfscope}%
\pgfsys@transformshift{0.750000in}{2.055734in}%
\pgfsys@useobject{currentmarker}{}%
\end{pgfscope}%
\end{pgfscope}%
\begin{pgfscope}%
\pgfsetbuttcap%
\pgfsetroundjoin%
\definecolor{currentfill}{rgb}{0.000000,0.000000,0.000000}%
\pgfsetfillcolor{currentfill}%
\pgfsetlinewidth{0.602250pt}%
\definecolor{currentstroke}{rgb}{0.000000,0.000000,0.000000}%
\pgfsetstrokecolor{currentstroke}%
\pgfsetdash{}{0pt}%
\pgfsys@defobject{currentmarker}{\pgfqpoint{-0.027778in}{0.000000in}}{\pgfqpoint{0.000000in}{0.000000in}}{%
\pgfpathmoveto{\pgfqpoint{0.000000in}{0.000000in}}%
\pgfpathlineto{\pgfqpoint{-0.027778in}{0.000000in}}%
\pgfusepath{stroke,fill}%
}%
\begin{pgfscope}%
\pgfsys@transformshift{0.750000in}{2.090232in}%
\pgfsys@useobject{currentmarker}{}%
\end{pgfscope}%
\end{pgfscope}%
\begin{pgfscope}%
\pgfsetbuttcap%
\pgfsetroundjoin%
\definecolor{currentfill}{rgb}{0.000000,0.000000,0.000000}%
\pgfsetfillcolor{currentfill}%
\pgfsetlinewidth{0.602250pt}%
\definecolor{currentstroke}{rgb}{0.000000,0.000000,0.000000}%
\pgfsetstrokecolor{currentstroke}%
\pgfsetdash{}{0pt}%
\pgfsys@defobject{currentmarker}{\pgfqpoint{-0.027778in}{0.000000in}}{\pgfqpoint{0.000000in}{0.000000in}}{%
\pgfpathmoveto{\pgfqpoint{0.000000in}{0.000000in}}%
\pgfpathlineto{\pgfqpoint{-0.027778in}{0.000000in}}%
\pgfusepath{stroke,fill}%
}%
\begin{pgfscope}%
\pgfsys@transformshift{0.750000in}{2.118419in}%
\pgfsys@useobject{currentmarker}{}%
\end{pgfscope}%
\end{pgfscope}%
\begin{pgfscope}%
\pgfsetbuttcap%
\pgfsetroundjoin%
\definecolor{currentfill}{rgb}{0.000000,0.000000,0.000000}%
\pgfsetfillcolor{currentfill}%
\pgfsetlinewidth{0.602250pt}%
\definecolor{currentstroke}{rgb}{0.000000,0.000000,0.000000}%
\pgfsetstrokecolor{currentstroke}%
\pgfsetdash{}{0pt}%
\pgfsys@defobject{currentmarker}{\pgfqpoint{-0.027778in}{0.000000in}}{\pgfqpoint{0.000000in}{0.000000in}}{%
\pgfpathmoveto{\pgfqpoint{0.000000in}{0.000000in}}%
\pgfpathlineto{\pgfqpoint{-0.027778in}{0.000000in}}%
\pgfusepath{stroke,fill}%
}%
\begin{pgfscope}%
\pgfsys@transformshift{0.750000in}{2.142251in}%
\pgfsys@useobject{currentmarker}{}%
\end{pgfscope}%
\end{pgfscope}%
\begin{pgfscope}%
\pgfsetbuttcap%
\pgfsetroundjoin%
\definecolor{currentfill}{rgb}{0.000000,0.000000,0.000000}%
\pgfsetfillcolor{currentfill}%
\pgfsetlinewidth{0.602250pt}%
\definecolor{currentstroke}{rgb}{0.000000,0.000000,0.000000}%
\pgfsetstrokecolor{currentstroke}%
\pgfsetdash{}{0pt}%
\pgfsys@defobject{currentmarker}{\pgfqpoint{-0.027778in}{0.000000in}}{\pgfqpoint{0.000000in}{0.000000in}}{%
\pgfpathmoveto{\pgfqpoint{0.000000in}{0.000000in}}%
\pgfpathlineto{\pgfqpoint{-0.027778in}{0.000000in}}%
\pgfusepath{stroke,fill}%
}%
\begin{pgfscope}%
\pgfsys@transformshift{0.750000in}{2.162895in}%
\pgfsys@useobject{currentmarker}{}%
\end{pgfscope}%
\end{pgfscope}%
\begin{pgfscope}%
\pgfsetbuttcap%
\pgfsetroundjoin%
\definecolor{currentfill}{rgb}{0.000000,0.000000,0.000000}%
\pgfsetfillcolor{currentfill}%
\pgfsetlinewidth{0.602250pt}%
\definecolor{currentstroke}{rgb}{0.000000,0.000000,0.000000}%
\pgfsetstrokecolor{currentstroke}%
\pgfsetdash{}{0pt}%
\pgfsys@defobject{currentmarker}{\pgfqpoint{-0.027778in}{0.000000in}}{\pgfqpoint{0.000000in}{0.000000in}}{%
\pgfpathmoveto{\pgfqpoint{0.000000in}{0.000000in}}%
\pgfpathlineto{\pgfqpoint{-0.027778in}{0.000000in}}%
\pgfusepath{stroke,fill}%
}%
\begin{pgfscope}%
\pgfsys@transformshift{0.750000in}{2.181104in}%
\pgfsys@useobject{currentmarker}{}%
\end{pgfscope}%
\end{pgfscope}%
\begin{pgfscope}%
\pgfsetbuttcap%
\pgfsetroundjoin%
\definecolor{currentfill}{rgb}{0.000000,0.000000,0.000000}%
\pgfsetfillcolor{currentfill}%
\pgfsetlinewidth{0.602250pt}%
\definecolor{currentstroke}{rgb}{0.000000,0.000000,0.000000}%
\pgfsetstrokecolor{currentstroke}%
\pgfsetdash{}{0pt}%
\pgfsys@defobject{currentmarker}{\pgfqpoint{-0.027778in}{0.000000in}}{\pgfqpoint{0.000000in}{0.000000in}}{%
\pgfpathmoveto{\pgfqpoint{0.000000in}{0.000000in}}%
\pgfpathlineto{\pgfqpoint{-0.027778in}{0.000000in}}%
\pgfusepath{stroke,fill}%
}%
\begin{pgfscope}%
\pgfsys@transformshift{0.750000in}{2.304554in}%
\pgfsys@useobject{currentmarker}{}%
\end{pgfscope}%
\end{pgfscope}%
\begin{pgfscope}%
\pgfsetbuttcap%
\pgfsetroundjoin%
\definecolor{currentfill}{rgb}{0.000000,0.000000,0.000000}%
\pgfsetfillcolor{currentfill}%
\pgfsetlinewidth{0.602250pt}%
\definecolor{currentstroke}{rgb}{0.000000,0.000000,0.000000}%
\pgfsetstrokecolor{currentstroke}%
\pgfsetdash{}{0pt}%
\pgfsys@defobject{currentmarker}{\pgfqpoint{-0.027778in}{0.000000in}}{\pgfqpoint{0.000000in}{0.000000in}}{%
\pgfpathmoveto{\pgfqpoint{0.000000in}{0.000000in}}%
\pgfpathlineto{\pgfqpoint{-0.027778in}{0.000000in}}%
\pgfusepath{stroke,fill}%
}%
\begin{pgfscope}%
\pgfsys@transformshift{0.750000in}{2.367240in}%
\pgfsys@useobject{currentmarker}{}%
\end{pgfscope}%
\end{pgfscope}%
\begin{pgfscope}%
\pgfsetbuttcap%
\pgfsetroundjoin%
\definecolor{currentfill}{rgb}{0.000000,0.000000,0.000000}%
\pgfsetfillcolor{currentfill}%
\pgfsetlinewidth{0.602250pt}%
\definecolor{currentstroke}{rgb}{0.000000,0.000000,0.000000}%
\pgfsetstrokecolor{currentstroke}%
\pgfsetdash{}{0pt}%
\pgfsys@defobject{currentmarker}{\pgfqpoint{-0.027778in}{0.000000in}}{\pgfqpoint{0.000000in}{0.000000in}}{%
\pgfpathmoveto{\pgfqpoint{0.000000in}{0.000000in}}%
\pgfpathlineto{\pgfqpoint{-0.027778in}{0.000000in}}%
\pgfusepath{stroke,fill}%
}%
\begin{pgfscope}%
\pgfsys@transformshift{0.750000in}{2.411715in}%
\pgfsys@useobject{currentmarker}{}%
\end{pgfscope}%
\end{pgfscope}%
\begin{pgfscope}%
\pgfsetbuttcap%
\pgfsetroundjoin%
\definecolor{currentfill}{rgb}{0.000000,0.000000,0.000000}%
\pgfsetfillcolor{currentfill}%
\pgfsetlinewidth{0.602250pt}%
\definecolor{currentstroke}{rgb}{0.000000,0.000000,0.000000}%
\pgfsetstrokecolor{currentstroke}%
\pgfsetdash{}{0pt}%
\pgfsys@defobject{currentmarker}{\pgfqpoint{-0.027778in}{0.000000in}}{\pgfqpoint{0.000000in}{0.000000in}}{%
\pgfpathmoveto{\pgfqpoint{0.000000in}{0.000000in}}%
\pgfpathlineto{\pgfqpoint{-0.027778in}{0.000000in}}%
\pgfusepath{stroke,fill}%
}%
\begin{pgfscope}%
\pgfsys@transformshift{0.750000in}{2.446214in}%
\pgfsys@useobject{currentmarker}{}%
\end{pgfscope}%
\end{pgfscope}%
\begin{pgfscope}%
\pgfsetbuttcap%
\pgfsetroundjoin%
\definecolor{currentfill}{rgb}{0.000000,0.000000,0.000000}%
\pgfsetfillcolor{currentfill}%
\pgfsetlinewidth{0.602250pt}%
\definecolor{currentstroke}{rgb}{0.000000,0.000000,0.000000}%
\pgfsetstrokecolor{currentstroke}%
\pgfsetdash{}{0pt}%
\pgfsys@defobject{currentmarker}{\pgfqpoint{-0.027778in}{0.000000in}}{\pgfqpoint{0.000000in}{0.000000in}}{%
\pgfpathmoveto{\pgfqpoint{0.000000in}{0.000000in}}%
\pgfpathlineto{\pgfqpoint{-0.027778in}{0.000000in}}%
\pgfusepath{stroke,fill}%
}%
\begin{pgfscope}%
\pgfsys@transformshift{0.750000in}{2.474401in}%
\pgfsys@useobject{currentmarker}{}%
\end{pgfscope}%
\end{pgfscope}%
\begin{pgfscope}%
\pgfsetbuttcap%
\pgfsetroundjoin%
\definecolor{currentfill}{rgb}{0.000000,0.000000,0.000000}%
\pgfsetfillcolor{currentfill}%
\pgfsetlinewidth{0.602250pt}%
\definecolor{currentstroke}{rgb}{0.000000,0.000000,0.000000}%
\pgfsetstrokecolor{currentstroke}%
\pgfsetdash{}{0pt}%
\pgfsys@defobject{currentmarker}{\pgfqpoint{-0.027778in}{0.000000in}}{\pgfqpoint{0.000000in}{0.000000in}}{%
\pgfpathmoveto{\pgfqpoint{0.000000in}{0.000000in}}%
\pgfpathlineto{\pgfqpoint{-0.027778in}{0.000000in}}%
\pgfusepath{stroke,fill}%
}%
\begin{pgfscope}%
\pgfsys@transformshift{0.750000in}{2.498232in}%
\pgfsys@useobject{currentmarker}{}%
\end{pgfscope}%
\end{pgfscope}%
\begin{pgfscope}%
\pgfsetbuttcap%
\pgfsetroundjoin%
\definecolor{currentfill}{rgb}{0.000000,0.000000,0.000000}%
\pgfsetfillcolor{currentfill}%
\pgfsetlinewidth{0.602250pt}%
\definecolor{currentstroke}{rgb}{0.000000,0.000000,0.000000}%
\pgfsetstrokecolor{currentstroke}%
\pgfsetdash{}{0pt}%
\pgfsys@defobject{currentmarker}{\pgfqpoint{-0.027778in}{0.000000in}}{\pgfqpoint{0.000000in}{0.000000in}}{%
\pgfpathmoveto{\pgfqpoint{0.000000in}{0.000000in}}%
\pgfpathlineto{\pgfqpoint{-0.027778in}{0.000000in}}%
\pgfusepath{stroke,fill}%
}%
\begin{pgfscope}%
\pgfsys@transformshift{0.750000in}{2.518876in}%
\pgfsys@useobject{currentmarker}{}%
\end{pgfscope}%
\end{pgfscope}%
\begin{pgfscope}%
\pgfsetbuttcap%
\pgfsetroundjoin%
\definecolor{currentfill}{rgb}{0.000000,0.000000,0.000000}%
\pgfsetfillcolor{currentfill}%
\pgfsetlinewidth{0.602250pt}%
\definecolor{currentstroke}{rgb}{0.000000,0.000000,0.000000}%
\pgfsetstrokecolor{currentstroke}%
\pgfsetdash{}{0pt}%
\pgfsys@defobject{currentmarker}{\pgfqpoint{-0.027778in}{0.000000in}}{\pgfqpoint{0.000000in}{0.000000in}}{%
\pgfpathmoveto{\pgfqpoint{0.000000in}{0.000000in}}%
\pgfpathlineto{\pgfqpoint{-0.027778in}{0.000000in}}%
\pgfusepath{stroke,fill}%
}%
\begin{pgfscope}%
\pgfsys@transformshift{0.750000in}{2.537086in}%
\pgfsys@useobject{currentmarker}{}%
\end{pgfscope}%
\end{pgfscope}%
\begin{pgfscope}%
\pgfsetbuttcap%
\pgfsetroundjoin%
\definecolor{currentfill}{rgb}{0.000000,0.000000,0.000000}%
\pgfsetfillcolor{currentfill}%
\pgfsetlinewidth{0.602250pt}%
\definecolor{currentstroke}{rgb}{0.000000,0.000000,0.000000}%
\pgfsetstrokecolor{currentstroke}%
\pgfsetdash{}{0pt}%
\pgfsys@defobject{currentmarker}{\pgfqpoint{-0.027778in}{0.000000in}}{\pgfqpoint{0.000000in}{0.000000in}}{%
\pgfpathmoveto{\pgfqpoint{0.000000in}{0.000000in}}%
\pgfpathlineto{\pgfqpoint{-0.027778in}{0.000000in}}%
\pgfusepath{stroke,fill}%
}%
\begin{pgfscope}%
\pgfsys@transformshift{0.750000in}{2.660536in}%
\pgfsys@useobject{currentmarker}{}%
\end{pgfscope}%
\end{pgfscope}%
\begin{pgfscope}%
\pgfsetbuttcap%
\pgfsetroundjoin%
\definecolor{currentfill}{rgb}{0.000000,0.000000,0.000000}%
\pgfsetfillcolor{currentfill}%
\pgfsetlinewidth{0.602250pt}%
\definecolor{currentstroke}{rgb}{0.000000,0.000000,0.000000}%
\pgfsetstrokecolor{currentstroke}%
\pgfsetdash{}{0pt}%
\pgfsys@defobject{currentmarker}{\pgfqpoint{-0.027778in}{0.000000in}}{\pgfqpoint{0.000000in}{0.000000in}}{%
\pgfpathmoveto{\pgfqpoint{0.000000in}{0.000000in}}%
\pgfpathlineto{\pgfqpoint{-0.027778in}{0.000000in}}%
\pgfusepath{stroke,fill}%
}%
\begin{pgfscope}%
\pgfsys@transformshift{0.750000in}{2.723221in}%
\pgfsys@useobject{currentmarker}{}%
\end{pgfscope}%
\end{pgfscope}%
\begin{pgfscope}%
\pgfsetbuttcap%
\pgfsetroundjoin%
\definecolor{currentfill}{rgb}{0.000000,0.000000,0.000000}%
\pgfsetfillcolor{currentfill}%
\pgfsetlinewidth{0.602250pt}%
\definecolor{currentstroke}{rgb}{0.000000,0.000000,0.000000}%
\pgfsetstrokecolor{currentstroke}%
\pgfsetdash{}{0pt}%
\pgfsys@defobject{currentmarker}{\pgfqpoint{-0.027778in}{0.000000in}}{\pgfqpoint{0.000000in}{0.000000in}}{%
\pgfpathmoveto{\pgfqpoint{0.000000in}{0.000000in}}%
\pgfpathlineto{\pgfqpoint{-0.027778in}{0.000000in}}%
\pgfusepath{stroke,fill}%
}%
\begin{pgfscope}%
\pgfsys@transformshift{0.750000in}{2.767697in}%
\pgfsys@useobject{currentmarker}{}%
\end{pgfscope}%
\end{pgfscope}%
\begin{pgfscope}%
\pgfsetbuttcap%
\pgfsetroundjoin%
\definecolor{currentfill}{rgb}{0.000000,0.000000,0.000000}%
\pgfsetfillcolor{currentfill}%
\pgfsetlinewidth{0.602250pt}%
\definecolor{currentstroke}{rgb}{0.000000,0.000000,0.000000}%
\pgfsetstrokecolor{currentstroke}%
\pgfsetdash{}{0pt}%
\pgfsys@defobject{currentmarker}{\pgfqpoint{-0.027778in}{0.000000in}}{\pgfqpoint{0.000000in}{0.000000in}}{%
\pgfpathmoveto{\pgfqpoint{0.000000in}{0.000000in}}%
\pgfpathlineto{\pgfqpoint{-0.027778in}{0.000000in}}%
\pgfusepath{stroke,fill}%
}%
\begin{pgfscope}%
\pgfsys@transformshift{0.750000in}{2.802195in}%
\pgfsys@useobject{currentmarker}{}%
\end{pgfscope}%
\end{pgfscope}%
\begin{pgfscope}%
\pgfsetbuttcap%
\pgfsetroundjoin%
\definecolor{currentfill}{rgb}{0.000000,0.000000,0.000000}%
\pgfsetfillcolor{currentfill}%
\pgfsetlinewidth{0.602250pt}%
\definecolor{currentstroke}{rgb}{0.000000,0.000000,0.000000}%
\pgfsetstrokecolor{currentstroke}%
\pgfsetdash{}{0pt}%
\pgfsys@defobject{currentmarker}{\pgfqpoint{-0.027778in}{0.000000in}}{\pgfqpoint{0.000000in}{0.000000in}}{%
\pgfpathmoveto{\pgfqpoint{0.000000in}{0.000000in}}%
\pgfpathlineto{\pgfqpoint{-0.027778in}{0.000000in}}%
\pgfusepath{stroke,fill}%
}%
\begin{pgfscope}%
\pgfsys@transformshift{0.750000in}{2.830382in}%
\pgfsys@useobject{currentmarker}{}%
\end{pgfscope}%
\end{pgfscope}%
\begin{pgfscope}%
\pgfsetbuttcap%
\pgfsetroundjoin%
\definecolor{currentfill}{rgb}{0.000000,0.000000,0.000000}%
\pgfsetfillcolor{currentfill}%
\pgfsetlinewidth{0.602250pt}%
\definecolor{currentstroke}{rgb}{0.000000,0.000000,0.000000}%
\pgfsetstrokecolor{currentstroke}%
\pgfsetdash{}{0pt}%
\pgfsys@defobject{currentmarker}{\pgfqpoint{-0.027778in}{0.000000in}}{\pgfqpoint{0.000000in}{0.000000in}}{%
\pgfpathmoveto{\pgfqpoint{0.000000in}{0.000000in}}%
\pgfpathlineto{\pgfqpoint{-0.027778in}{0.000000in}}%
\pgfusepath{stroke,fill}%
}%
\begin{pgfscope}%
\pgfsys@transformshift{0.750000in}{2.854214in}%
\pgfsys@useobject{currentmarker}{}%
\end{pgfscope}%
\end{pgfscope}%
\begin{pgfscope}%
\pgfsetbuttcap%
\pgfsetroundjoin%
\definecolor{currentfill}{rgb}{0.000000,0.000000,0.000000}%
\pgfsetfillcolor{currentfill}%
\pgfsetlinewidth{0.602250pt}%
\definecolor{currentstroke}{rgb}{0.000000,0.000000,0.000000}%
\pgfsetstrokecolor{currentstroke}%
\pgfsetdash{}{0pt}%
\pgfsys@defobject{currentmarker}{\pgfqpoint{-0.027778in}{0.000000in}}{\pgfqpoint{0.000000in}{0.000000in}}{%
\pgfpathmoveto{\pgfqpoint{0.000000in}{0.000000in}}%
\pgfpathlineto{\pgfqpoint{-0.027778in}{0.000000in}}%
\pgfusepath{stroke,fill}%
}%
\begin{pgfscope}%
\pgfsys@transformshift{0.750000in}{2.874858in}%
\pgfsys@useobject{currentmarker}{}%
\end{pgfscope}%
\end{pgfscope}%
\begin{pgfscope}%
\pgfsetbuttcap%
\pgfsetroundjoin%
\definecolor{currentfill}{rgb}{0.000000,0.000000,0.000000}%
\pgfsetfillcolor{currentfill}%
\pgfsetlinewidth{0.602250pt}%
\definecolor{currentstroke}{rgb}{0.000000,0.000000,0.000000}%
\pgfsetstrokecolor{currentstroke}%
\pgfsetdash{}{0pt}%
\pgfsys@defobject{currentmarker}{\pgfqpoint{-0.027778in}{0.000000in}}{\pgfqpoint{0.000000in}{0.000000in}}{%
\pgfpathmoveto{\pgfqpoint{0.000000in}{0.000000in}}%
\pgfpathlineto{\pgfqpoint{-0.027778in}{0.000000in}}%
\pgfusepath{stroke,fill}%
}%
\begin{pgfscope}%
\pgfsys@transformshift{0.750000in}{2.893067in}%
\pgfsys@useobject{currentmarker}{}%
\end{pgfscope}%
\end{pgfscope}%
\begin{pgfscope}%
\pgfsetbuttcap%
\pgfsetroundjoin%
\definecolor{currentfill}{rgb}{0.000000,0.000000,0.000000}%
\pgfsetfillcolor{currentfill}%
\pgfsetlinewidth{0.602250pt}%
\definecolor{currentstroke}{rgb}{0.000000,0.000000,0.000000}%
\pgfsetstrokecolor{currentstroke}%
\pgfsetdash{}{0pt}%
\pgfsys@defobject{currentmarker}{\pgfqpoint{-0.027778in}{0.000000in}}{\pgfqpoint{0.000000in}{0.000000in}}{%
\pgfpathmoveto{\pgfqpoint{0.000000in}{0.000000in}}%
\pgfpathlineto{\pgfqpoint{-0.027778in}{0.000000in}}%
\pgfusepath{stroke,fill}%
}%
\begin{pgfscope}%
\pgfsys@transformshift{0.750000in}{3.016517in}%
\pgfsys@useobject{currentmarker}{}%
\end{pgfscope}%
\end{pgfscope}%
\begin{pgfscope}%
\pgfsetbuttcap%
\pgfsetroundjoin%
\definecolor{currentfill}{rgb}{0.000000,0.000000,0.000000}%
\pgfsetfillcolor{currentfill}%
\pgfsetlinewidth{0.602250pt}%
\definecolor{currentstroke}{rgb}{0.000000,0.000000,0.000000}%
\pgfsetstrokecolor{currentstroke}%
\pgfsetdash{}{0pt}%
\pgfsys@defobject{currentmarker}{\pgfqpoint{-0.027778in}{0.000000in}}{\pgfqpoint{0.000000in}{0.000000in}}{%
\pgfpathmoveto{\pgfqpoint{0.000000in}{0.000000in}}%
\pgfpathlineto{\pgfqpoint{-0.027778in}{0.000000in}}%
\pgfusepath{stroke,fill}%
}%
\begin{pgfscope}%
\pgfsys@transformshift{0.750000in}{3.079202in}%
\pgfsys@useobject{currentmarker}{}%
\end{pgfscope}%
\end{pgfscope}%
\begin{pgfscope}%
\pgfsetbuttcap%
\pgfsetroundjoin%
\definecolor{currentfill}{rgb}{0.000000,0.000000,0.000000}%
\pgfsetfillcolor{currentfill}%
\pgfsetlinewidth{0.602250pt}%
\definecolor{currentstroke}{rgb}{0.000000,0.000000,0.000000}%
\pgfsetstrokecolor{currentstroke}%
\pgfsetdash{}{0pt}%
\pgfsys@defobject{currentmarker}{\pgfqpoint{-0.027778in}{0.000000in}}{\pgfqpoint{0.000000in}{0.000000in}}{%
\pgfpathmoveto{\pgfqpoint{0.000000in}{0.000000in}}%
\pgfpathlineto{\pgfqpoint{-0.027778in}{0.000000in}}%
\pgfusepath{stroke,fill}%
}%
\begin{pgfscope}%
\pgfsys@transformshift{0.750000in}{3.123678in}%
\pgfsys@useobject{currentmarker}{}%
\end{pgfscope}%
\end{pgfscope}%
\begin{pgfscope}%
\pgfsetbuttcap%
\pgfsetroundjoin%
\definecolor{currentfill}{rgb}{0.000000,0.000000,0.000000}%
\pgfsetfillcolor{currentfill}%
\pgfsetlinewidth{0.602250pt}%
\definecolor{currentstroke}{rgb}{0.000000,0.000000,0.000000}%
\pgfsetstrokecolor{currentstroke}%
\pgfsetdash{}{0pt}%
\pgfsys@defobject{currentmarker}{\pgfqpoint{-0.027778in}{0.000000in}}{\pgfqpoint{0.000000in}{0.000000in}}{%
\pgfpathmoveto{\pgfqpoint{0.000000in}{0.000000in}}%
\pgfpathlineto{\pgfqpoint{-0.027778in}{0.000000in}}%
\pgfusepath{stroke,fill}%
}%
\begin{pgfscope}%
\pgfsys@transformshift{0.750000in}{3.158176in}%
\pgfsys@useobject{currentmarker}{}%
\end{pgfscope}%
\end{pgfscope}%
\begin{pgfscope}%
\pgfsetbuttcap%
\pgfsetroundjoin%
\definecolor{currentfill}{rgb}{0.000000,0.000000,0.000000}%
\pgfsetfillcolor{currentfill}%
\pgfsetlinewidth{0.602250pt}%
\definecolor{currentstroke}{rgb}{0.000000,0.000000,0.000000}%
\pgfsetstrokecolor{currentstroke}%
\pgfsetdash{}{0pt}%
\pgfsys@defobject{currentmarker}{\pgfqpoint{-0.027778in}{0.000000in}}{\pgfqpoint{0.000000in}{0.000000in}}{%
\pgfpathmoveto{\pgfqpoint{0.000000in}{0.000000in}}%
\pgfpathlineto{\pgfqpoint{-0.027778in}{0.000000in}}%
\pgfusepath{stroke,fill}%
}%
\begin{pgfscope}%
\pgfsys@transformshift{0.750000in}{3.186363in}%
\pgfsys@useobject{currentmarker}{}%
\end{pgfscope}%
\end{pgfscope}%
\begin{pgfscope}%
\pgfsetbuttcap%
\pgfsetroundjoin%
\definecolor{currentfill}{rgb}{0.000000,0.000000,0.000000}%
\pgfsetfillcolor{currentfill}%
\pgfsetlinewidth{0.602250pt}%
\definecolor{currentstroke}{rgb}{0.000000,0.000000,0.000000}%
\pgfsetstrokecolor{currentstroke}%
\pgfsetdash{}{0pt}%
\pgfsys@defobject{currentmarker}{\pgfqpoint{-0.027778in}{0.000000in}}{\pgfqpoint{0.000000in}{0.000000in}}{%
\pgfpathmoveto{\pgfqpoint{0.000000in}{0.000000in}}%
\pgfpathlineto{\pgfqpoint{-0.027778in}{0.000000in}}%
\pgfusepath{stroke,fill}%
}%
\begin{pgfscope}%
\pgfsys@transformshift{0.750000in}{3.210195in}%
\pgfsys@useobject{currentmarker}{}%
\end{pgfscope}%
\end{pgfscope}%
\begin{pgfscope}%
\pgfsetbuttcap%
\pgfsetroundjoin%
\definecolor{currentfill}{rgb}{0.000000,0.000000,0.000000}%
\pgfsetfillcolor{currentfill}%
\pgfsetlinewidth{0.602250pt}%
\definecolor{currentstroke}{rgb}{0.000000,0.000000,0.000000}%
\pgfsetstrokecolor{currentstroke}%
\pgfsetdash{}{0pt}%
\pgfsys@defobject{currentmarker}{\pgfqpoint{-0.027778in}{0.000000in}}{\pgfqpoint{0.000000in}{0.000000in}}{%
\pgfpathmoveto{\pgfqpoint{0.000000in}{0.000000in}}%
\pgfpathlineto{\pgfqpoint{-0.027778in}{0.000000in}}%
\pgfusepath{stroke,fill}%
}%
\begin{pgfscope}%
\pgfsys@transformshift{0.750000in}{3.230839in}%
\pgfsys@useobject{currentmarker}{}%
\end{pgfscope}%
\end{pgfscope}%
\begin{pgfscope}%
\pgfsetbuttcap%
\pgfsetroundjoin%
\definecolor{currentfill}{rgb}{0.000000,0.000000,0.000000}%
\pgfsetfillcolor{currentfill}%
\pgfsetlinewidth{0.602250pt}%
\definecolor{currentstroke}{rgb}{0.000000,0.000000,0.000000}%
\pgfsetstrokecolor{currentstroke}%
\pgfsetdash{}{0pt}%
\pgfsys@defobject{currentmarker}{\pgfqpoint{-0.027778in}{0.000000in}}{\pgfqpoint{0.000000in}{0.000000in}}{%
\pgfpathmoveto{\pgfqpoint{0.000000in}{0.000000in}}%
\pgfpathlineto{\pgfqpoint{-0.027778in}{0.000000in}}%
\pgfusepath{stroke,fill}%
}%
\begin{pgfscope}%
\pgfsys@transformshift{0.750000in}{3.249048in}%
\pgfsys@useobject{currentmarker}{}%
\end{pgfscope}%
\end{pgfscope}%
\begin{pgfscope}%
\pgfsetbuttcap%
\pgfsetroundjoin%
\definecolor{currentfill}{rgb}{0.000000,0.000000,0.000000}%
\pgfsetfillcolor{currentfill}%
\pgfsetlinewidth{0.602250pt}%
\definecolor{currentstroke}{rgb}{0.000000,0.000000,0.000000}%
\pgfsetstrokecolor{currentstroke}%
\pgfsetdash{}{0pt}%
\pgfsys@defobject{currentmarker}{\pgfqpoint{-0.027778in}{0.000000in}}{\pgfqpoint{0.000000in}{0.000000in}}{%
\pgfpathmoveto{\pgfqpoint{0.000000in}{0.000000in}}%
\pgfpathlineto{\pgfqpoint{-0.027778in}{0.000000in}}%
\pgfusepath{stroke,fill}%
}%
\begin{pgfscope}%
\pgfsys@transformshift{0.750000in}{3.372498in}%
\pgfsys@useobject{currentmarker}{}%
\end{pgfscope}%
\end{pgfscope}%
\begin{pgfscope}%
\pgfsetbuttcap%
\pgfsetroundjoin%
\definecolor{currentfill}{rgb}{0.000000,0.000000,0.000000}%
\pgfsetfillcolor{currentfill}%
\pgfsetlinewidth{0.602250pt}%
\definecolor{currentstroke}{rgb}{0.000000,0.000000,0.000000}%
\pgfsetstrokecolor{currentstroke}%
\pgfsetdash{}{0pt}%
\pgfsys@defobject{currentmarker}{\pgfqpoint{-0.027778in}{0.000000in}}{\pgfqpoint{0.000000in}{0.000000in}}{%
\pgfpathmoveto{\pgfqpoint{0.000000in}{0.000000in}}%
\pgfpathlineto{\pgfqpoint{-0.027778in}{0.000000in}}%
\pgfusepath{stroke,fill}%
}%
\begin{pgfscope}%
\pgfsys@transformshift{0.750000in}{3.435183in}%
\pgfsys@useobject{currentmarker}{}%
\end{pgfscope}%
\end{pgfscope}%
\begin{pgfscope}%
\pgfsetbuttcap%
\pgfsetroundjoin%
\definecolor{currentfill}{rgb}{0.000000,0.000000,0.000000}%
\pgfsetfillcolor{currentfill}%
\pgfsetlinewidth{0.602250pt}%
\definecolor{currentstroke}{rgb}{0.000000,0.000000,0.000000}%
\pgfsetstrokecolor{currentstroke}%
\pgfsetdash{}{0pt}%
\pgfsys@defobject{currentmarker}{\pgfqpoint{-0.027778in}{0.000000in}}{\pgfqpoint{0.000000in}{0.000000in}}{%
\pgfpathmoveto{\pgfqpoint{0.000000in}{0.000000in}}%
\pgfpathlineto{\pgfqpoint{-0.027778in}{0.000000in}}%
\pgfusepath{stroke,fill}%
}%
\begin{pgfscope}%
\pgfsys@transformshift{0.750000in}{3.479659in}%
\pgfsys@useobject{currentmarker}{}%
\end{pgfscope}%
\end{pgfscope}%
\begin{pgfscope}%
\pgfsetbuttcap%
\pgfsetroundjoin%
\definecolor{currentfill}{rgb}{0.000000,0.000000,0.000000}%
\pgfsetfillcolor{currentfill}%
\pgfsetlinewidth{0.602250pt}%
\definecolor{currentstroke}{rgb}{0.000000,0.000000,0.000000}%
\pgfsetstrokecolor{currentstroke}%
\pgfsetdash{}{0pt}%
\pgfsys@defobject{currentmarker}{\pgfqpoint{-0.027778in}{0.000000in}}{\pgfqpoint{0.000000in}{0.000000in}}{%
\pgfpathmoveto{\pgfqpoint{0.000000in}{0.000000in}}%
\pgfpathlineto{\pgfqpoint{-0.027778in}{0.000000in}}%
\pgfusepath{stroke,fill}%
}%
\begin{pgfscope}%
\pgfsys@transformshift{0.750000in}{3.514157in}%
\pgfsys@useobject{currentmarker}{}%
\end{pgfscope}%
\end{pgfscope}%
\begin{pgfscope}%
\definecolor{textcolor}{rgb}{0.000000,0.000000,0.000000}%
\pgfsetstrokecolor{textcolor}%
\pgfsetfillcolor{textcolor}%
\pgftext[x=0.309220in,y=1.980000in,,bottom,rotate=90.000000]{\color{textcolor}\sffamily\fontsize{10.000000}{12.000000}\selectfont Error}%
\end{pgfscope}%
\begin{pgfscope}%
\pgfpathrectangle{\pgfqpoint{0.750000in}{0.440000in}}{\pgfqpoint{4.650000in}{3.080000in}}%
\pgfusepath{clip}%
\pgfsetrectcap%
\pgfsetroundjoin%
\pgfsetlinewidth{1.505625pt}%
\definecolor{currentstroke}{rgb}{0.121569,0.466667,0.705882}%
\pgfsetstrokecolor{currentstroke}%
\pgfsetdash{}{0pt}%
\pgfpathmoveto{\pgfqpoint{1.154259in}{3.151308in}}%
\pgfpathlineto{\pgfqpoint{1.390735in}{3.047997in}}%
\pgfpathlineto{\pgfqpoint{1.558517in}{2.973932in}}%
\pgfpathlineto{\pgfqpoint{1.794993in}{2.864542in}}%
\pgfpathlineto{\pgfqpoint{1.962776in}{2.781394in}}%
\pgfpathlineto{\pgfqpoint{2.148505in}{2.688434in}}%
\pgfpathlineto{\pgfqpoint{2.367034in}{2.577449in}}%
\pgfpathlineto{\pgfqpoint{2.578689in}{2.468167in}}%
\pgfpathlineto{\pgfqpoint{2.771293in}{2.368066in}}%
\pgfpathlineto{\pgfqpoint{2.970129in}{2.263995in}}%
\pgfpathlineto{\pgfqpoint{3.175552in}{2.160595in}}%
\pgfpathlineto{\pgfqpoint{3.380832in}{2.061361in}}%
\pgfpathlineto{\pgfqpoint{3.579810in}{1.964666in}}%
\pgfpathlineto{\pgfqpoint{3.781877in}{1.866005in}}%
\pgfpathlineto{\pgfqpoint{3.984069in}{1.766862in}}%
\pgfpathlineto{\pgfqpoint{4.186136in}{1.667395in}}%
\pgfpathlineto{\pgfqpoint{4.388328in}{1.567566in}}%
\pgfpathlineto{\pgfqpoint{4.590395in}{1.467536in}}%
\pgfpathlineto{\pgfqpoint{4.792586in}{1.367215in}}%
\pgfpathlineto{\pgfqpoint{4.994653in}{1.266788in}}%
\pgfpathlineto{\pgfqpoint{5.196845in}{1.163649in}}%
\pgfusepath{stroke}%
\end{pgfscope}%
\begin{pgfscope}%
\pgfpathrectangle{\pgfqpoint{0.750000in}{0.440000in}}{\pgfqpoint{4.650000in}{3.080000in}}%
\pgfusepath{clip}%
\pgfsetrectcap%
\pgfsetroundjoin%
\pgfsetlinewidth{1.505625pt}%
\definecolor{currentstroke}{rgb}{1.000000,0.498039,0.054902}%
\pgfsetstrokecolor{currentstroke}%
\pgfsetdash{}{0pt}%
\pgfpathmoveto{\pgfqpoint{1.154259in}{3.023575in}}%
\pgfpathlineto{\pgfqpoint{1.390735in}{2.904730in}}%
\pgfpathlineto{\pgfqpoint{1.558517in}{2.818082in}}%
\pgfpathlineto{\pgfqpoint{1.794993in}{2.694449in}}%
\pgfpathlineto{\pgfqpoint{1.962776in}{2.606164in}}%
\pgfpathlineto{\pgfqpoint{2.148505in}{2.508146in}}%
\pgfpathlineto{\pgfqpoint{2.367034in}{2.392596in}}%
\pgfpathlineto{\pgfqpoint{2.578689in}{2.280555in}}%
\pgfpathlineto{\pgfqpoint{2.771293in}{2.178538in}}%
\pgfpathlineto{\pgfqpoint{2.970129in}{2.073184in}}%
\pgfpathlineto{\pgfqpoint{3.175552in}{1.964318in}}%
\pgfpathlineto{\pgfqpoint{3.380832in}{1.855512in}}%
\pgfpathlineto{\pgfqpoint{3.579810in}{1.750039in}}%
\pgfpathlineto{\pgfqpoint{3.781877in}{1.642922in}}%
\pgfpathlineto{\pgfqpoint{3.984069in}{1.535736in}}%
\pgfpathlineto{\pgfqpoint{4.186136in}{1.428614in}}%
\pgfpathlineto{\pgfqpoint{4.388328in}{1.321424in}}%
\pgfpathlineto{\pgfqpoint{4.590395in}{1.214299in}}%
\pgfpathlineto{\pgfqpoint{4.792586in}{1.107110in}}%
\pgfpathlineto{\pgfqpoint{4.994653in}{0.999996in}}%
\pgfpathlineto{\pgfqpoint{5.196845in}{0.897979in}}%
\pgfusepath{stroke}%
\end{pgfscope}%
\begin{pgfscope}%
\pgfpathrectangle{\pgfqpoint{0.750000in}{0.440000in}}{\pgfqpoint{4.650000in}{3.080000in}}%
\pgfusepath{clip}%
\pgfsetrectcap%
\pgfsetroundjoin%
\pgfsetlinewidth{1.505625pt}%
\definecolor{currentstroke}{rgb}{0.172549,0.627451,0.172549}%
\pgfsetstrokecolor{currentstroke}%
\pgfsetdash{}{0pt}%
\pgfpathmoveto{\pgfqpoint{1.154259in}{3.379610in}}%
\pgfpathlineto{\pgfqpoint{1.390735in}{3.320849in}}%
\pgfpathlineto{\pgfqpoint{1.558517in}{3.277668in}}%
\pgfpathlineto{\pgfqpoint{1.794993in}{3.215799in}}%
\pgfpathlineto{\pgfqpoint{1.962776in}{3.171538in}}%
\pgfpathlineto{\pgfqpoint{2.148505in}{3.122378in}}%
\pgfpathlineto{\pgfqpoint{2.367034in}{3.064437in}}%
\pgfpathlineto{\pgfqpoint{2.578689in}{3.008283in}}%
\pgfpathlineto{\pgfqpoint{2.771293in}{2.957178in}}%
\pgfpathlineto{\pgfqpoint{2.970129in}{2.904425in}}%
\pgfpathlineto{\pgfqpoint{3.175552in}{2.849934in}}%
\pgfpathlineto{\pgfqpoint{3.380832in}{2.795488in}}%
\pgfpathlineto{\pgfqpoint{3.579810in}{2.742721in}}%
\pgfpathlineto{\pgfqpoint{3.781877in}{2.689141in}}%
\pgfpathlineto{\pgfqpoint{3.984069in}{2.635532in}}%
\pgfpathlineto{\pgfqpoint{4.186136in}{2.581959in}}%
\pgfpathlineto{\pgfqpoint{4.388328in}{2.528356in}}%
\pgfpathlineto{\pgfqpoint{4.590395in}{2.474787in}}%
\pgfpathlineto{\pgfqpoint{4.792586in}{2.421187in}}%
\pgfpathlineto{\pgfqpoint{4.994653in}{2.367621in}}%
\pgfpathlineto{\pgfqpoint{5.196845in}{2.314022in}}%
\pgfusepath{stroke}%
\end{pgfscope}%
\begin{pgfscope}%
\pgfpathrectangle{\pgfqpoint{0.750000in}{0.440000in}}{\pgfqpoint{4.650000in}{3.080000in}}%
\pgfusepath{clip}%
\pgfsetrectcap%
\pgfsetroundjoin%
\pgfsetlinewidth{1.505625pt}%
\definecolor{currentstroke}{rgb}{0.839216,0.152941,0.156863}%
\pgfsetstrokecolor{currentstroke}%
\pgfsetdash{}{0pt}%
\pgfpathmoveto{\pgfqpoint{1.154259in}{2.977816in}}%
\pgfpathlineto{\pgfqpoint{1.390735in}{2.881797in}}%
\pgfpathlineto{\pgfqpoint{1.558517in}{2.818005in}}%
\pgfpathlineto{\pgfqpoint{1.794993in}{2.717306in}}%
\pgfpathlineto{\pgfqpoint{1.962776in}{2.640730in}}%
\pgfpathlineto{\pgfqpoint{2.148505in}{2.551330in}}%
\pgfpathlineto{\pgfqpoint{2.367034in}{2.441848in}}%
\pgfpathlineto{\pgfqpoint{2.578689in}{2.333418in}}%
\pgfpathlineto{\pgfqpoint{2.771293in}{2.234386in}}%
\pgfpathlineto{\pgfqpoint{2.970129in}{2.131062in}}%
\pgfpathlineto{\pgfqpoint{3.175552in}{2.023689in}}%
\pgfpathlineto{\pgfqpoint{3.380832in}{1.915965in}}%
\pgfpathlineto{\pgfqpoint{3.579810in}{1.811249in}}%
\pgfpathlineto{\pgfqpoint{3.781877in}{1.704678in}}%
\pgfpathlineto{\pgfqpoint{3.984069in}{1.597893in}}%
\pgfpathlineto{\pgfqpoint{4.186136in}{1.491059in}}%
\pgfpathlineto{\pgfqpoint{4.388328in}{1.384110in}}%
\pgfpathlineto{\pgfqpoint{4.590395in}{1.277244in}}%
\pgfpathlineto{\pgfqpoint{4.792586in}{1.170455in}}%
\pgfpathlineto{\pgfqpoint{4.994653in}{1.063753in}}%
\pgfpathlineto{\pgfqpoint{5.196845in}{0.956704in}}%
\pgfusepath{stroke}%
\end{pgfscope}%
\begin{pgfscope}%
\pgfpathrectangle{\pgfqpoint{0.750000in}{0.440000in}}{\pgfqpoint{4.650000in}{3.080000in}}%
\pgfusepath{clip}%
\pgfsetrectcap%
\pgfsetroundjoin%
\pgfsetlinewidth{1.505625pt}%
\definecolor{currentstroke}{rgb}{0.580392,0.403922,0.741176}%
\pgfsetstrokecolor{currentstroke}%
\pgfsetdash{}{0pt}%
\pgfpathmoveto{\pgfqpoint{1.154259in}{2.834542in}}%
\pgfpathlineto{\pgfqpoint{1.390735in}{2.715838in}}%
\pgfpathlineto{\pgfqpoint{1.558517in}{2.629050in}}%
\pgfpathlineto{\pgfqpoint{1.794993in}{2.505222in}}%
\pgfpathlineto{\pgfqpoint{1.962776in}{2.416814in}}%
\pgfpathlineto{\pgfqpoint{2.148505in}{2.318679in}}%
\pgfpathlineto{\pgfqpoint{2.367034in}{2.203020in}}%
\pgfpathlineto{\pgfqpoint{2.578689in}{2.090901in}}%
\pgfpathlineto{\pgfqpoint{2.771293in}{1.988831in}}%
\pgfpathlineto{\pgfqpoint{2.970129in}{1.883438in}}%
\pgfpathlineto{\pgfqpoint{3.175552in}{1.774542in}}%
\pgfpathlineto{\pgfqpoint{3.380832in}{1.665716in}}%
\pgfpathlineto{\pgfqpoint{3.579810in}{1.560229in}}%
\pgfpathlineto{\pgfqpoint{3.781877in}{1.453102in}}%
\pgfpathlineto{\pgfqpoint{3.984069in}{1.345910in}}%
\pgfpathlineto{\pgfqpoint{4.186136in}{1.238784in}}%
\pgfpathlineto{\pgfqpoint{4.388328in}{1.131594in}}%
\pgfpathlineto{\pgfqpoint{4.590395in}{1.024479in}}%
\pgfpathlineto{\pgfqpoint{4.792586in}{0.917330in}}%
\pgfpathlineto{\pgfqpoint{4.994653in}{0.810333in}}%
\pgfpathlineto{\pgfqpoint{5.196845in}{0.703511in}}%
\pgfusepath{stroke}%
\end{pgfscope}%
\begin{pgfscope}%
\pgfpathrectangle{\pgfqpoint{0.750000in}{0.440000in}}{\pgfqpoint{4.650000in}{3.080000in}}%
\pgfusepath{clip}%
\pgfsetrectcap%
\pgfsetroundjoin%
\pgfsetlinewidth{1.505625pt}%
\definecolor{currentstroke}{rgb}{0.549020,0.337255,0.294118}%
\pgfsetstrokecolor{currentstroke}%
\pgfsetdash{}{0pt}%
\pgfpathmoveto{\pgfqpoint{1.154259in}{3.031032in}}%
\pgfpathlineto{\pgfqpoint{1.390735in}{2.967186in}}%
\pgfpathlineto{\pgfqpoint{1.558517in}{2.922026in}}%
\pgfpathlineto{\pgfqpoint{1.794993in}{2.858775in}}%
\pgfpathlineto{\pgfqpoint{1.962776in}{2.814087in}}%
\pgfpathlineto{\pgfqpoint{2.148505in}{2.764722in}}%
\pgfpathlineto{\pgfqpoint{2.367034in}{2.706715in}}%
\pgfpathlineto{\pgfqpoint{2.578689in}{2.650572in}}%
\pgfpathlineto{\pgfqpoint{2.771293in}{2.599500in}}%
\pgfpathlineto{\pgfqpoint{2.970129in}{2.546784in}}%
\pgfpathlineto{\pgfqpoint{3.175552in}{2.492325in}}%
\pgfpathlineto{\pgfqpoint{3.380832in}{2.437907in}}%
\pgfpathlineto{\pgfqpoint{3.579810in}{2.385161in}}%
\pgfpathlineto{\pgfqpoint{3.781877in}{2.331596in}}%
\pgfpathlineto{\pgfqpoint{3.984069in}{2.277999in}}%
\pgfpathlineto{\pgfqpoint{4.186136in}{2.224435in}}%
\pgfpathlineto{\pgfqpoint{4.388328in}{2.170838in}}%
\pgfpathlineto{\pgfqpoint{4.590395in}{2.117274in}}%
\pgfpathlineto{\pgfqpoint{4.792586in}{2.063677in}}%
\pgfpathlineto{\pgfqpoint{4.994653in}{2.010113in}}%
\pgfpathlineto{\pgfqpoint{5.196845in}{1.956516in}}%
\pgfusepath{stroke}%
\end{pgfscope}%
\begin{pgfscope}%
\pgfpathrectangle{\pgfqpoint{0.750000in}{0.440000in}}{\pgfqpoint{4.650000in}{3.080000in}}%
\pgfusepath{clip}%
\pgfsetrectcap%
\pgfsetroundjoin%
\pgfsetlinewidth{0.501875pt}%
\definecolor{currentstroke}{rgb}{0.000000,0.000000,0.000000}%
\pgfsetstrokecolor{currentstroke}%
\pgfsetdash{}{0pt}%
\pgfpathmoveto{\pgfqpoint{1.154259in}{3.210195in}}%
\pgfpathlineto{\pgfqpoint{5.196845in}{2.138585in}}%
\pgfusepath{stroke}%
\end{pgfscope}%
\begin{pgfscope}%
\pgfpathrectangle{\pgfqpoint{0.750000in}{0.440000in}}{\pgfqpoint{4.650000in}{3.080000in}}%
\pgfusepath{clip}%
\pgfsetbuttcap%
\pgfsetroundjoin%
\pgfsetlinewidth{0.501875pt}%
\definecolor{currentstroke}{rgb}{0.000000,0.000000,0.000000}%
\pgfsetstrokecolor{currentstroke}%
\pgfsetdash{{1.850000pt}{0.800000pt}}{0.000000pt}%
\pgfpathmoveto{\pgfqpoint{1.154259in}{2.723221in}}%
\pgfpathlineto{\pgfqpoint{5.196845in}{0.580000in}}%
\pgfusepath{stroke}%
\end{pgfscope}%
\begin{pgfscope}%
\pgfsetrectcap%
\pgfsetmiterjoin%
\pgfsetlinewidth{0.803000pt}%
\definecolor{currentstroke}{rgb}{0.000000,0.000000,0.000000}%
\pgfsetstrokecolor{currentstroke}%
\pgfsetdash{}{0pt}%
\pgfpathmoveto{\pgfqpoint{0.750000in}{0.440000in}}%
\pgfpathlineto{\pgfqpoint{0.750000in}{3.520000in}}%
\pgfusepath{stroke}%
\end{pgfscope}%
\begin{pgfscope}%
\pgfsetrectcap%
\pgfsetmiterjoin%
\pgfsetlinewidth{0.803000pt}%
\definecolor{currentstroke}{rgb}{0.000000,0.000000,0.000000}%
\pgfsetstrokecolor{currentstroke}%
\pgfsetdash{}{0pt}%
\pgfpathmoveto{\pgfqpoint{5.400000in}{0.440000in}}%
\pgfpathlineto{\pgfqpoint{5.400000in}{3.520000in}}%
\pgfusepath{stroke}%
\end{pgfscope}%
\begin{pgfscope}%
\pgfsetrectcap%
\pgfsetmiterjoin%
\pgfsetlinewidth{0.803000pt}%
\definecolor{currentstroke}{rgb}{0.000000,0.000000,0.000000}%
\pgfsetstrokecolor{currentstroke}%
\pgfsetdash{}{0pt}%
\pgfpathmoveto{\pgfqpoint{0.750000in}{0.440000in}}%
\pgfpathlineto{\pgfqpoint{5.400000in}{0.440000in}}%
\pgfusepath{stroke}%
\end{pgfscope}%
\begin{pgfscope}%
\pgfsetrectcap%
\pgfsetmiterjoin%
\pgfsetlinewidth{0.803000pt}%
\definecolor{currentstroke}{rgb}{0.000000,0.000000,0.000000}%
\pgfsetstrokecolor{currentstroke}%
\pgfsetdash{}{0pt}%
\pgfpathmoveto{\pgfqpoint{0.750000in}{3.520000in}}%
\pgfpathlineto{\pgfqpoint{5.400000in}{3.520000in}}%
\pgfusepath{stroke}%
\end{pgfscope}%
\begin{pgfscope}%
\pgfsetbuttcap%
\pgfsetmiterjoin%
\definecolor{currentfill}{rgb}{1.000000,1.000000,1.000000}%
\pgfsetfillcolor{currentfill}%
\pgfsetfillopacity{0.800000}%
\pgfsetlinewidth{1.003750pt}%
\definecolor{currentstroke}{rgb}{0.800000,0.800000,0.800000}%
\pgfsetstrokecolor{currentstroke}%
\pgfsetstrokeopacity{0.800000}%
\pgfsetdash{}{0pt}%
\pgfpathmoveto{\pgfqpoint{0.847222in}{0.509444in}}%
\pgfpathlineto{\pgfqpoint{2.239875in}{0.509444in}}%
\pgfpathquadraticcurveto{\pgfqpoint{2.267653in}{0.509444in}}{\pgfqpoint{2.267653in}{0.537222in}}%
\pgfpathlineto{\pgfqpoint{2.267653in}{2.212101in}}%
\pgfpathquadraticcurveto{\pgfqpoint{2.267653in}{2.239878in}}{\pgfqpoint{2.239875in}{2.239878in}}%
\pgfpathlineto{\pgfqpoint{0.847222in}{2.239878in}}%
\pgfpathquadraticcurveto{\pgfqpoint{0.819444in}{2.239878in}}{\pgfqpoint{0.819444in}{2.212101in}}%
\pgfpathlineto{\pgfqpoint{0.819444in}{0.537222in}}%
\pgfpathquadraticcurveto{\pgfqpoint{0.819444in}{0.509444in}}{\pgfqpoint{0.847222in}{0.509444in}}%
\pgfpathclose%
\pgfusepath{stroke,fill}%
\end{pgfscope}%
\begin{pgfscope}%
\pgfsetrectcap%
\pgfsetroundjoin%
\pgfsetlinewidth{1.505625pt}%
\definecolor{currentstroke}{rgb}{0.121569,0.466667,0.705882}%
\pgfsetstrokecolor{currentstroke}%
\pgfsetdash{}{0pt}%
\pgfpathmoveto{\pgfqpoint{0.875000in}{2.127411in}}%
\pgfpathlineto{\pgfqpoint{1.152778in}{2.127411in}}%
\pgfusepath{stroke}%
\end{pgfscope}%
\begin{pgfscope}%
\definecolor{textcolor}{rgb}{0.000000,0.000000,0.000000}%
\pgfsetstrokecolor{textcolor}%
\pgfsetfillcolor{textcolor}%
\pgftext[x=1.263889in,y=2.078800in,left,base]{\color{textcolor}\sffamily\fontsize{10.000000}{12.000000}\selectfont Triangle \(\displaystyle L^{\infty}\)}%
\end{pgfscope}%
\begin{pgfscope}%
\pgfsetrectcap%
\pgfsetroundjoin%
\pgfsetlinewidth{1.505625pt}%
\definecolor{currentstroke}{rgb}{1.000000,0.498039,0.054902}%
\pgfsetstrokecolor{currentstroke}%
\pgfsetdash{}{0pt}%
\pgfpathmoveto{\pgfqpoint{0.875000in}{1.909076in}}%
\pgfpathlineto{\pgfqpoint{1.152778in}{1.909076in}}%
\pgfusepath{stroke}%
\end{pgfscope}%
\begin{pgfscope}%
\definecolor{textcolor}{rgb}{0.000000,0.000000,0.000000}%
\pgfsetstrokecolor{textcolor}%
\pgfsetfillcolor{textcolor}%
\pgftext[x=1.263889in,y=1.860465in,left,base]{\color{textcolor}\sffamily\fontsize{10.000000}{12.000000}\selectfont Triangle \(\displaystyle L^2\)}%
\end{pgfscope}%
\begin{pgfscope}%
\pgfsetrectcap%
\pgfsetroundjoin%
\pgfsetlinewidth{1.505625pt}%
\definecolor{currentstroke}{rgb}{0.172549,0.627451,0.172549}%
\pgfsetstrokecolor{currentstroke}%
\pgfsetdash{}{0pt}%
\pgfpathmoveto{\pgfqpoint{0.875000in}{1.690742in}}%
\pgfpathlineto{\pgfqpoint{1.152778in}{1.690742in}}%
\pgfusepath{stroke}%
\end{pgfscope}%
\begin{pgfscope}%
\definecolor{textcolor}{rgb}{0.000000,0.000000,0.000000}%
\pgfsetstrokecolor{textcolor}%
\pgfsetfillcolor{textcolor}%
\pgftext[x=1.263889in,y=1.642131in,left,base]{\color{textcolor}\sffamily\fontsize{10.000000}{12.000000}\selectfont Triangle \(\displaystyle H^1\)}%
\end{pgfscope}%
\begin{pgfscope}%
\pgfsetrectcap%
\pgfsetroundjoin%
\pgfsetlinewidth{1.505625pt}%
\definecolor{currentstroke}{rgb}{0.839216,0.152941,0.156863}%
\pgfsetstrokecolor{currentstroke}%
\pgfsetdash{}{0pt}%
\pgfpathmoveto{\pgfqpoint{0.875000in}{1.486884in}}%
\pgfpathlineto{\pgfqpoint{1.152778in}{1.486884in}}%
\pgfusepath{stroke}%
\end{pgfscope}%
\begin{pgfscope}%
\definecolor{textcolor}{rgb}{0.000000,0.000000,0.000000}%
\pgfsetstrokecolor{textcolor}%
\pgfsetfillcolor{textcolor}%
\pgftext[x=1.263889in,y=1.438273in,left,base]{\color{textcolor}\sffamily\fontsize{10.000000}{12.000000}\selectfont Rectangle \(\displaystyle L^{\infty}\)}%
\end{pgfscope}%
\begin{pgfscope}%
\pgfsetrectcap%
\pgfsetroundjoin%
\pgfsetlinewidth{1.505625pt}%
\definecolor{currentstroke}{rgb}{0.580392,0.403922,0.741176}%
\pgfsetstrokecolor{currentstroke}%
\pgfsetdash{}{0pt}%
\pgfpathmoveto{\pgfqpoint{0.875000in}{1.268550in}}%
\pgfpathlineto{\pgfqpoint{1.152778in}{1.268550in}}%
\pgfusepath{stroke}%
\end{pgfscope}%
\begin{pgfscope}%
\definecolor{textcolor}{rgb}{0.000000,0.000000,0.000000}%
\pgfsetstrokecolor{textcolor}%
\pgfsetfillcolor{textcolor}%
\pgftext[x=1.263889in,y=1.219939in,left,base]{\color{textcolor}\sffamily\fontsize{10.000000}{12.000000}\selectfont Rectangle \(\displaystyle L^2\)}%
\end{pgfscope}%
\begin{pgfscope}%
\pgfsetrectcap%
\pgfsetroundjoin%
\pgfsetlinewidth{1.505625pt}%
\definecolor{currentstroke}{rgb}{0.549020,0.337255,0.294118}%
\pgfsetstrokecolor{currentstroke}%
\pgfsetdash{}{0pt}%
\pgfpathmoveto{\pgfqpoint{0.875000in}{1.050215in}}%
\pgfpathlineto{\pgfqpoint{1.152778in}{1.050215in}}%
\pgfusepath{stroke}%
\end{pgfscope}%
\begin{pgfscope}%
\definecolor{textcolor}{rgb}{0.000000,0.000000,0.000000}%
\pgfsetstrokecolor{textcolor}%
\pgfsetfillcolor{textcolor}%
\pgftext[x=1.263889in,y=1.001604in,left,base]{\color{textcolor}\sffamily\fontsize{10.000000}{12.000000}\selectfont Rectangle \(\displaystyle H^1\)}%
\end{pgfscope}%
\begin{pgfscope}%
\pgfsetrectcap%
\pgfsetroundjoin%
\pgfsetlinewidth{0.501875pt}%
\definecolor{currentstroke}{rgb}{0.000000,0.000000,0.000000}%
\pgfsetstrokecolor{currentstroke}%
\pgfsetdash{}{0pt}%
\pgfpathmoveto{\pgfqpoint{0.875000in}{0.846358in}}%
\pgfpathlineto{\pgfqpoint{1.152778in}{0.846358in}}%
\pgfusepath{stroke}%
\end{pgfscope}%
\begin{pgfscope}%
\definecolor{textcolor}{rgb}{0.000000,0.000000,0.000000}%
\pgfsetstrokecolor{textcolor}%
\pgfsetfillcolor{textcolor}%
\pgftext[x=1.263889in,y=0.797747in,left,base]{\color{textcolor}\sffamily\fontsize{10.000000}{12.000000}\selectfont Slope \(\displaystyle -1\)}%
\end{pgfscope}%
\begin{pgfscope}%
\pgfsetbuttcap%
\pgfsetroundjoin%
\pgfsetlinewidth{0.501875pt}%
\definecolor{currentstroke}{rgb}{0.000000,0.000000,0.000000}%
\pgfsetstrokecolor{currentstroke}%
\pgfsetdash{{1.850000pt}{0.800000pt}}{0.000000pt}%
\pgfpathmoveto{\pgfqpoint{0.875000in}{0.642501in}}%
\pgfpathlineto{\pgfqpoint{1.152778in}{0.642501in}}%
\pgfusepath{stroke}%
\end{pgfscope}%
\begin{pgfscope}%
\definecolor{textcolor}{rgb}{0.000000,0.000000,0.000000}%
\pgfsetstrokecolor{textcolor}%
\pgfsetfillcolor{textcolor}%
\pgftext[x=1.263889in,y=0.593890in,left,base]{\color{textcolor}\sffamily\fontsize{10.000000}{12.000000}\selectfont Slope \(\displaystyle -2\)}%
\end{pgfscope}%
\end{pgfpicture}%
\makeatother%
\endgroup%

\caption{Error for the second equation with respect to $N$}
\label{Fig:Err2}
\end{figure}

\begin{table}[htbp]
\centering
\begin{tabular}{|c|c|c|c|c|c|}
\hline
$N$ & $L^{\infty}$ error & $L^2$ error & $H^1$ error & Time (\Si{s}) & \#Iterations \\
\hline
\input{Table3.tbl}
\end{tabular}
\caption{Numerical results using triangle elements for the second equation}
\label{Tbl:SumTri2}
\end{table}

\begin{table}[htbp]
\centering
\begin{tabular}{|c|c|c|c|c|c|}
\hline
$N$ & $L^{\infty}$ error & $L^2$ error & $H^1$ error & Time (\Si{s}) & \#Iterations \\
\hline
\input{Table4.tbl}
\end{tabular}
\caption{Numerical results using rectangle elements for the second equation}
\label{Tbl:SumRect2}
\end{table}

It can be directly verify the second-order convergence of $L^{\infty}$ and $L^2$ norm, as well as the first-order convergence under $H^1$ semi-norm. We discover that the rectangle elements are again more accurate and faster.

\end{document}
