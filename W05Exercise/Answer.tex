%! TeX encoding = UTF-8
%! TeX program = LuaLaTeX

\documentclass[english, nochinese]{pnote}
\usepackage[paper, cgu]{pdef}

\title{Answers to Exercises (Week 05)}
\author{Zhihan Li, 1600010653}
\date{October 16, 2018}

\begin{document}

\maketitle

\textbf{Problem 1. (Page 85 Exercise 18)} \textit{Proof.} We apply comparison theorem here. Choose the comparison function $\varPhi$ to be
\begin{equation}
\varPhi_j^m =
\begin{cases}
0, & j = 0, N, \\
\frac{1}{2} \rbr{ x_j - \frac{1}{2} }^2, & 1 \le j \le N - 1,
\end{cases}
\end{equation}
which satisfies $ 0 \le \varPhi \le 1 / 8 $ on $J$. (We denote $J_D$ to be nodes on $ \sbr{ 0, 1 } \times \cbr{0} \cup  $ and $ J_{ \Omega, 1 } $ to be nodes in $ \rbr{ 0, 1 } \times \rsbr{ 0, t_{\text{max}} } $ with $ J_{ \Omega, 2 } $ on $ \cbr{1} \times \rsbr{ 0, t_{\text{max}} } $.) Since
\begin{equation}
\abs{ L_h U_j^m } = \abs{ \frac{ U_{ j - 1 }^{ m - 1 } + U_{ j + 1 }^{ m - 1 } - 2 U_j^{ m - 1 } }{h^2} - \frac{ U_j^m - U_j^{ m - 1 } }{ 2 \tau } } \le K_1 \rbr{ h^2 + \tau }
\end{equation}
with some uniform $K_1$ on $ J_{ \Omega, 1 } $ and
\begin{equation}
\abs{ L_h U_0^m } = \abs{ \frac{ U_1^m - U_0^m }{h} - \alpha^m U_0^m } \le K_2 h
\end{equation}
for some uniform $K_2$ on $ J_{ \Omega, 2 } $, and for $ \mu = \tau / h^2 \le 1 / 2 $, $L_h$ satisfies the condition of maximal principle (on $ J_{ \Omega, 1 }  $ note that $ \alpha_m \ge 0 $), and
\begin{equation}
L_h U_j^m = 1
\end{equation}
on $ J_{ \Omega, 1 } $ and
\begin{equation}
L_h U_0^m \ge \frac{1}{ 8 h }
\end{equation}
on $ J_{ \Omega, 2 } $, we conclude that
\begin{equation}
\norm{e}_{\infty} \le \norm{e}_{ \infty, J_D } + \frac{1}{8} \max \cbr{ 8 K_2 h^2, K_1 \rbr{ h^2 + \tau } } = O \rbr{ h^2 + \tau }
\end{equation}
when  $ \norm{e}_{ \infty, J_D } = 0 $ as desired.
\hfill$\Box$

\textbf{Problem 2. (Page 85 Exercise 19)} \textit{Answer.} The $\theta$-scheme is
\begin{equation}
\rbr{ 1 - \theta } \frac{ U_{ j + 1 }^m - 2 U_j^m + U_{ j - 1 }^m }{h^2} + \theta \frac{ U_{ j + 1 }^{ m + 1 } - 2 U_j^{ m + 1 } + U_{ j - 1 }^{ m + 1 } }{h^2} - \frac{ U_j^{ m + 1 } - U_j^m }{\tau} = 0,
\end{equation}
and the truncation error at $ \rbr{ j, m } $ is (after tedious expansion and noting that $ u_t = u_{ x x } $)
\begin{equation}
u_{ t t } \rbr{ \tau \rbr{ \theta - \frac{1}{2} } + \frac{h^2}{12} } + \rbr{ -\frac{1}{6} u_{ t t } + \frac{\theta}{2} u_{ t t t } } \tau^2 + \frac{\theta}{12} u_{ t t t } h^2 \tau + \frac{1}{360} u_{ t t t } h^4 + O
\end{equation}
where the last $O$ stands for $ O \rbr{ \tau^3 + h^2 \tau^2 + h^4 \tau + h^6 } $.

For the first discretized boundary condition, the equation at boundary is
\begin{equation}
\frac{ U_1^m - U_0^m }{h} - \alpha^m U_0^m = g_0^m
\end{equation}
with truncation error at $ \rbr{ 0, m } $
\begin{equation}
\frac{h}{2} u_{ x x } + O \rbr{h^2}.
\end{equation}
The combined scheme is
\begin{equation}
\begin{split}
&\ptrel{=} \rbr{ 1 - \theta } \frac{ U_2^m - \rbr{ 2 - \beta^m } U_1^m }{h^2} + \theta \frac{ U_2^{ m + 1 } - \rbr{ 2 - \beta^{ m + 1 } } U_1^{ m + 1 } }{h^2} - \frac{ U_1^{ m + 1 } - U_1^m }{\tau} \\
&= \rbr{ 1 - \theta } \beta^m \frac{g_0^m}{h} + \theta \beta^{ m + 1 } \frac{g_0^{ m + 1 }}{h},
\end{split}
\end{equation}
where
\begin{equation}
\beta^m = \frac{1}{ 1 + \alpha^m h }
\end{equation}
and the error equation is
\begin{equation}
\rbr{ 1 - \theta } \frac{ e_2^m - \rbr{ 2 - \beta^m } e_1^m }{h^2} + \theta \frac{ e_2^{ m + 1 } - \rbr{ 2 - \beta^{ m + 1 } } e_1^{ m + 1 } }{h^2} - \frac{ e_1^{ m + 1 } - e_1^m }{\tau} = T_1^m.
\end{equation}
The local truncation error expanded at $ \rbr{ 0, m } $ is
\begin{equation}
\frac{1}{2} u_{ x x } \rbr{ \rbr{ 1 - \theta } \beta^m + \theta \beta^{ m + 1 } } + O \rbr{ \tau + h^2 } = O \rbr{1}.
\end{equation}
The condition of maximal principle at the boundary is $ \mu \le 1 / \rbr{ 2 - \beta^m } $. Considering the interior nodes, the final condition is still $ \mu \le 1 / 2 $.

For the second discretized boundary condition, the equation at boundary is
\begin{equation}
\frac{ U_1^m - U_0^m }{h} - \frac{1}{2} \alpha^m \rbr{ U_1^m + U_0^m } = g_{ 1 / 2 }^m
\end{equation}
with truncation error at $ \rbr{ 1 / 2, m } $
\begin{equation}
\frac{h^2}{24} u_{ x x x } - \frac{h^2}{8} \alpha^m u_{ x x } + O \rbr{h^4}.
\end{equation}
The combined scheme is
\begin{equation}
\begin{split}
&\ptrel{=} \rbr{ 1 - \theta } \frac{ U_2^m - \rbr{ 2 - \xi^m } U_1^m }{h^2} + \theta \frac{ U_2^{ m + 1 } - \rbr{ 2 - \xi^{ m + 1 } } U_1^{ m + 1 } }{h^2} - \frac{ U_1^{ m + 1 } - U_1^m }{\tau} \\
&= \rbr{ 1 - \theta } \eta^m \frac{g_0^m}{h} + \theta \eta^{ m + 1 } \frac{g_0^{ m + 1 }}{h},
\end{split}
\end{equation}
where
\begin{gather}
\xi^m = \frac{ 2 - \alpha^m h }{ 2 + \alpha^m h }, \\
\eta^m = \frac{2}{ 2 + \alpha^m h }
\end{gather}
and the error equation is
\begin{equation}
\rbr{ 1 - \theta } \frac{ e_2^m - \rbr{ 2 - \xi^m } e_1^m }{h^2} + \theta \frac{ e_2^{ m + 1 } - \rbr{ 2 - \xi^{ m + 1 } } e_1^{ m + 1 } }{h^2} - \frac{ e_1^{ m + 1 } - e_1^m }{\tau} = T_1^m.
\end{equation}
The local truncation error is
\begin{equation}
\begin{split}
&\ptrel{=} \rbr{ \frac{h}{24} u_{ x x x } \rbr{ \rbr{ 1 - \theta } \eta^m + \theta \eta^{ m + 1 } } - \frac{h}{8} u_{ x x } \rbr{ \rbr{ 1 - \theta } \eta^m \alpha^m + \theta \eta^{ m + 1 } \alpha^{ m + 1 } } }_{ 1 / 2 }^m \\
&+ \rbr{ u_{ t t } \rbr{ \theta - \frac{1}{2} } \tau + \rbr{ -\frac{1}{6} u_{ t t } + \frac{\theta}{2} u_{ t t t } } \tau^2 }_1^m + O \rbr{ \tau^3 + h^2 } \\
&=
\begin{cases}
O \rbr{ \tau + h }, & \theta \neq 1 / 2, \\
O \rbr{ \tau^2 + h }, & \theta = 1 / 2.
\end{cases}
\end{split}
\end{equation}
The condition of maximal principle at the boundary is $ \mu \le 1 / \rbr{ 2 - \xi^m } $ and the global condition is $ \mu \le 1 / 2 $ also.

For the third discretized boundary condition, the equation at boundary is
\begin{equation}
\frac{ U_1^m - U_{-1}^m }{ 2 h } - \alpha^m U_0^m = g_0^m
\end{equation}
with truncation error at $ \rbr{ 0, m } $
\begin{equation}
\frac{1}{6} u_{ x x x } h^2 + O \rbr{h^4}.
\end{equation}
The combined scheme is
\begin{equation}
\begin{split}
&\ptrel{=} \rbr{ 1 - \theta } \frac{ U_1^m - 2 \rbr{ 1 + \alpha^m h } U_0^m }{h^2} + \theta \frac{ U_1^{ m + 1 } - 2 \rbr{ 1 + \alpha^{ m + 1 } h } U_0^{ m + 1 } }{h^2} - \frac{ U_0^{ m + 1 } - U_0^m }{\tau} \\
&= 2 \rbr{ 1 - \theta } \frac{g_0^m}{h} + 2 \theta \frac{g_0^{ m + 1 }}{h},
\end{split}
\end{equation}
and the error equation is
\begin{equation}
\rbr{ 1 - \theta } \frac{ e_1^m - 2 \rbr{ 1 + \alpha^m h } e_0^m }{h^2} + \theta \frac{ e_1^{ m + 1 } - 2 \rbr{ 1 + \alpha^{ m + 1 } h } e_0^{ m + 1 } }{h^2} - \frac{ e_0^{ m + 1 } - e_0^m }{\tau} = T_0^m.
\end{equation}
The local truncation error expanded at $ \rbr{ 0, m } $ is
\begin{equation}
\begin{split}
&\ptrel{=} \frac{h}{3} u_{ x x x } + u_{ t t } \rbr{ \theta - \frac{1}{2} } \tau + \rbr{ -\frac{1}{6} u_{ t t } + \frac{\theta}{2} u_{ t t t } } \tau^2 + O \rbr{ h^2 + \tau^3 } \\
&=
\begin{cases}
O \rbr{ \tau + h }, & \theta \neq 1 / 2, \\
O \rbr{ \tau^2 + h }, & \theta = 1 / 2.
\end{cases}
\end{split}
\end{equation}
The condition of maximal principle at the boundary is $ \mu \le \beta^m / 2 \rbr{ 1 - \theta } $.


\end{document}
